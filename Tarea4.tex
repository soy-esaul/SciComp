\documentclass{article}
\usepackage[utf8]{inputenc}

\usepackage[utf8]{inputenc}
\usepackage[spanish,es-tabla,es-nodecimaldot]{babel}
\usepackage{amsmath,amsthm,amsfonts,amssymb,mathtools,dsfont,mathrsfs}
\usepackage{enumerate,graphicx,xcolor}
\usepackage{lmodern}
\usepackage[T1]{fontenc}
\usepackage[left=2cm,top=2.5cm,right=2cm,bottom=2.5cm]{geometry}
\usepackage[activate={true,nocompatibility},final,tracking=true,kerning=true,spacing=true,factor=1100,stretch=10,shrink=10]{microtype}
\usepackage{hyperref}


%\DeclarePairedDelimiter{\norm}{\lVert}{\rVert}




\newcommand{\N}{\mathbb{N}}
\newcommand{\R}{\mathbb R}
\newcommand{\Z}{\mathbb Z}
\newcommand{\Rbar}{\overline{\mathbb R}}
\newcommand{\F}{\mathscr F}
\newcommand{\A}{\mathscr A}
\newcommand{\To}{\Rightarrow}
\newcommand{\C}{\mathscr C}
\newcommand{\La}{\mathscr L_A}
\newcommand{\B}{\mathcal B}
\newcommand{\Q}{\mathbb Q}
\renewcommand{\epsilon}{\varepsilon}
\renewcommand{\L}{\mathcal L}
\renewcommand{\d}{\mathrm d}
\newcommand{\abs}[1]{\left| #1 \right|}
\newcommand{\pts}[1]{\left( #1 \right)}
\newcommand{\norm}[1]{\left\lVert#1\right\rVert}
\renewcommand{\P}[1]{\mathbb P\left( #1 \right)}
\newcommand{\E}[1]{\mathbb E \left( #1 \right)}


\newcommand{\ols}[1]{\mskip.5\thinmuskip\overline{\mskip-.5\thinmuskip {#1} \mskip-.5\thinmuskip}\mskip.5\thinmuskip} % overline short
\newcommand{\olsi}[1]{\,\overline{\!{#1}}} % overline short italic
\makeatletter
\newcommand\closure[1]{
  \tctestifnum{\count@stringtoks{#1}>1} %checks if number of chars in arg > 1 (including '\')
  {\ols{#1}} %if arg is longer than just one char, e.g. \mathbb{Q}, \mathbb{F},...
  {\olsi{#1}} %if arg is just one char, e.g. K, L,...
}
% FROM TOKCYCLE:
\long\def\count@stringtoks#1{\tc@earg\count@toks{\string#1}}
\long\def\count@toks#1{\the\numexpr-1\count@@toks#1.\tc@endcnt}
\long\def\count@@toks#1#2\tc@endcnt{+1\tc@ifempty{#2}{\relax}{\count@@toks#2\tc@endcnt}}
\def\tc@ifempty#1{\tc@testxifx{\expandafter\relax\detokenize{#1}\relax}}
\long\def\tc@earg#1#2{\expandafter#1\expandafter{#2}}
\long\def\tctestifnum#1{\tctestifcon{\ifnum#1\relax}}
\long\def\tctestifcon#1{#1\expandafter\tc@exfirst\else\expandafter\tc@exsecond\fi}
\long\def\tc@testxifx{\tc@earg\tctestifx}
\long\def\tctestifx#1{\tctestifcon{\ifx#1}}
\long\def\tc@exfirst#1#2{#1}
\long\def\tc@exsecond#1#2{#2}
\makeatother

\newtheorem{lemma}{Lema}
\newtheorem{theorem}{Teorema}

\setlength\parindent{0pt}
\setlength\parskip{4pt}

\renewcommand{\C}{\mathbb{C}}

\title{Cómputo científico para probabilidad y estadística. Tarea 4.\\
Cálculo de eigenvalores}
\author{Juan Esaul González Rangel}
\date{Septiembre 2023}



\begin{document}

\maketitle


\begin{enumerate}

    \item Dado el siguiente

    \begin{theorem}[Gershgorin]
    Dada una matriz $A = a_{ij}$ de $m \times m$, cada eigenvalor de $A$ está en al 
    menos uno de los discos en el plano complejo con centro en $a_{ii}$ y radio 
    $\sum_{j\neq i} |a_{ij}|$. Además, si $n$ de estos discos forman un dominio conexo, 
    disjunto de los otros $m - n$ discos, entonces hay exactamente $n$ eigenvalores 
    en ese dominio.
    \end{theorem}

    Deduce estimaciones de los eigenvalores de

    \[ A = \begin{pmatrix}
        8 & 1 & 0 \\
        1 & 4 & \epsilon \\
        0 & \epsilon & 1
    \end{pmatrix} \]



con $|\epsilon| < 1$.

    Por el Teorema de Gershgorin, los tres eigenvalores de $A$ se encuentran en el dominio,

    \begin{equation*}
        \{ x \in \C : |8 + x| \le 1 \} \cup \{ x \in \C : |4 + x| \le 1 + |\epsilon| \} \cup \{ x \in \C : |1+x| \le |\epsilon|\}.
    \end{equation*}

    Como $|\epsilon|<1$, cada disco es disjunto de los otros dos, y además es conexo. Por lo
    tanto, 

    \begin{align*}
        \lambda_1 &\in \{ x \in \C : |8 + x| \le 1 \},\\
        \lambda_2 &\in \{ x \in \C : |4 + x| \le 1 + |\epsilon| \},\\
        \lambda_3 &\in \{ x \in \C : |1+x| \le |\epsilon|\}.
    \end{align*}

    Notemos que si $\epsilon \in \R$, entonces $A$ es hermitiana y por lo tanto todos sus
    valores propios son reales, de donde obtenemos una mejor estimación de los eigenvalores,

    \begin{align*}
        \lambda_1 &\in \{ x \in \C : |8 + x| \le 1 \}\cap \R \qquad \Rightarrow 7 \le \lambda_1 \le 9,\\
        \lambda_2 &\in \{ x \in \C : |4 + x| \le 1 + |\epsilon| \}\cap\R \qquad
        \Rightarrow 4 - (1 + |\epsilon|) \le \lambda_2 \le 4 + (1 + |\epsilon|),\\
        \lambda_3 &\in \{ x \in \C : |1+x| \le |\epsilon|\}\cap\R \qquad 
        \Rightarrow 1 - \epsilon \le \lambda_1 \le 1 + \epsilon.
    \end{align*}


    \item Implementa la iteración $QR$ con shift. Aplícala a la matriz $A$ del 
    Ejercicio 1 con $\epsilon = 10^{-N}$ para $N = 1, 3, 4, 5$.

    En el archivo \texttt{Tarea4.py}, la función se encuentra implementada con el nombre de
    \texttt{QR\_shift}. La función toma tres argumentos; \texttt{Matrix}, la matriz a la cuál se le aplicará
    el algoritmo; \texttt{shift}, el desplazamiento que se aplicará a la matriz, y \texttt{iterations},
    la cantidad de veces que se repetirá el algoritmo antes de parar.

    El siguiente es un ejemplo de uso del algoritmo,

    \begin{lstlisting}[language=Python]
B = np.array( [[1,2,3],[4,5,6],[7,8,9]] )
vals = QR_shift(B,2,200)\end{lstlisting}

    El tiempo de ejecución del código anterior es menor a un segundo y la salida es \texttt{array([ 1.61168440e+01, -1.11684397e+00,  1.55431223e-15])}.
        
    Para la matriz $A$, se iteró un total de 10,000 veces por cada valor de $\epsilon$ y se obtuvieron
    las siguientes estimaciones de los eigenvalores de la matriz,

    \begin{lstlisting}[language=Python]
[array([8.23614095, 3.76735469, 0.99650436]),
array([8.23606798, 3.76393237, 0.99999965]),
array([8.23606798, 3.76393203, 1.        ]),
array([8.23606798, 3.76393202, 1.        ])]\end{lstlisting}

        Es decir, para $N=1$ los eigenvalores estimados son $(8.23614095, 3.76735469, 0.99650436)$,
        para $N=3$, son $(8.23606798, 3.76393237, 0.99999965)$, para $N=4$ son $(8.23606798, 3.76393203, 1.)$,
        y para $N=5$ son $(8.23606798, 3.76393202, 1.)$.
    
    Cuando ejecutamos el código con 10 iteraciones, obtenemos los siguientes valores

\begin{lstlisting}[language=Python]
[array([8.23614095, 3.76735469, 0.99650436]),
array([8.23606798, 3.76393237, 0.99999965]),
array([8.23606798, 3.76393203, 1.        ]),
array([8.23606798, 3.76393202, 1.        ])]\end{lstlisting}

    Los valores coinciden exactamente con los que obtenemos para 10,000 iteraciones. Concluimos
    que en este caso la convergencia del algoritmo bastante rápida.

    \item Determina todos los eigenvalores y eigenvectores de una matriz de Householder.
    
    Sea $H$ una matriz de Householder de tamaño $n\times n$, entonces $H$ es de la forma

    \[ H = I - 2vv^*, \]

    con $v \in \C^{n}$ un eigenvector de norma 1. Sea $u$ cualquier vector linealmente independiente a $v$, entonces

    \[ Hu = (I - 2vv^*)u = u - 2v(v^*u) = u - 2v(0) = u. \]

    Por lo tanto, $u$ es un eigenvector de $H$ con eigenvalor 1. Como $\C^n$ tiene dimensión $n$,
    existe un total de $n-1$ vectores linealmente independientes entre sí y linealmente independientes
    a $v$ (pues de otra forma se podría formar una base de $\C^n$ de tamaño distinto a $n$). El valor
    $1$ entonces es un eigenvalor con multiplicidad $n-1$.

    Ahora, para $Hv$ se cumple,

    \[ Hv = (I - 2vv^*)v = v - 2v\|v\| = -v. \]

    Por lo tanto $v$ es un eigenvector con eigenvalor $-1$. Los eigenvalores de $H$ son $1$ con
    multiplicidad $n-1$ y $-1$ con multiplicidad 1.

    \item Demuestra que no es posible construir la transformación de similaridad del 
    teorema de Schur con un número finito de transformaciones de similaridad de 
    Householder.

    \begin{proof}
        Supongamos que existe un algoritmo que nos permita encontrar de manera exacta los eigenvalores
        de cualquier matriz $A$ con un número finito de transformaciones de similaridad de Householder.

        Sea entonces $p(\lambda)$ cualquier polinomio de orden mayor a 4. Sabemos que podemos encontrar una matriz $A$
        acompañante de $p(\lambda)$ tal que los eigenvalores de $A$ son las raíces de $p(\lambda)$.
        Aplicando nuestro algoritmo, podemos encontrar los eigenvalores de $A$ en una cantidad finita de
        transformaciones de Householder. Pero las transformaciones de Householder son tranformaciones
        lineales, por lo que en particular son operaciones algebraicas. 
        
        Lo anterior significa que podemos encontrar las raíces de cualquier polinomio de grado 
        mayor que 4 usando una cantidad finita de operaciones algebraicas, lo que es una 
        contradicción del Teorema de Abel-Rufini.
    
    \end{proof}

    \item ¿Qué pasa si aplicas la iteración $QR$ sin shift a una matriz ortogonal? 
    o \textbf{hagan el que quieran}. Sea $A$ una matriz de Hessenberg superior y sea $QR = A$ 
    la factorización QR de $A$. Muestra que $RQ$ es una matriz superior de Hessenberg.
    
    Contestando a la primera pregunta. Sea $A$ una matriz ortogonal, sabemos que la descomposición 
    $QR$ en una matriz ortogonal y una triangular superior es única (salvo un signo). Además se 
    satisface que $A = AI$, y como $A$ es ortogonal y $I$ es triangular superior, esta es la única
    descomposición $QR$ de $A$. 
    
    Por lo anterior, $Q_1 R_1 = A_0 = A$, entonces $Q_1 = A, R_1 = I$, por un argumento recursivo
    llegamos a que para cada paso del algoritmo tenemos

    \begin{align*}
        Q_{k}R_{k} &= A_{k-1} = A, \qquad \Rightarrow Q_{k} = A, R_{k} = I,\\
        \Rightarrow A_k &= R_k Q_k = I A = A.
    \end{align*}

    Es decir, la matriz no cambia después de todas las iteraciones, por lo que el algoritmo no 
    representa ninguna utilidad para encontrar los eigenvalores.

\end{enumerate}




 \end{document}