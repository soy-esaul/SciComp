\documentclass{article}
\usepackage[utf8]{inputenc}

\usepackage[utf8]{inputenc}
\usepackage[spanish,es-tabla,es-nodecimaldot]{babel}
\usepackage{amsmath,amsthm,amsfonts,amssymb,mathtools,dsfont,mathrsfs}
\usepackage{enumerate,graphicx,xcolor}
\usepackage{lmodern}
\usepackage[T1]{fontenc}
\usepackage[left=2cm,top=2.5cm,right=2cm,bottom=2.5cm]{geometry}
\usepackage[activate={true,nocompatibility},final,tracking=true,kerning=true,spacing=true,factor=1100,stretch=10,shrink=10]{microtype}
\usepackage{hyperref}
\usepackage{listings}
\usepackage{booktabs}


%\DeclarePairedDelimiter{\norm}{\lVert}{\rVert}




\newcommand{\N}{\mathbb{N}}
\newcommand{\R}{\mathbb R}
\newcommand{\Z}{\mathbb Z}
\newcommand{\Rbar}{\overline{\mathbb R}}
\newcommand{\F}{\mathscr F}
\newcommand{\A}{\mathscr A}
\newcommand{\To}{\Rightarrow}
\newcommand{\C}{\mathscr C}
\newcommand{\La}{\mathscr L_A}
\newcommand{\B}{\mathcal B}
\newcommand{\Q}{\mathbb Q}
\renewcommand{\epsilon}{\varepsilon}
\renewcommand{\L}{\mathcal L}
\renewcommand{\d}{\mathrm d}
\newcommand{\abs}[1]{\left| #1 \right|}
\newcommand{\pts}[1]{\left( #1 \right)}
\newcommand{\norm}[1]{\left\lVert#1\right\rVert}
\renewcommand{\P}[1]{\mathbb P\left( #1 \right)}
\newcommand{\E}[1]{\mathbb E \left( #1 \right)}


\newcommand{\ols}[1]{\mskip.5\thinmuskip\overline{\mskip-.5\thinmuskip {#1} \mskip-.5\thinmuskip}\mskip.5\thinmuskip} % overline short
\newcommand{\olsi}[1]{\,\overline{\!{#1}}} % overline short italic
\makeatletter
\newcommand\closure[1]{
  \tctestifnum{\count@stringtoks{#1}>1} %checks if number of chars in arg > 1 (including '\')
  {\ols{#1}} %if arg is longer than just one char, e.g. \mathbb{Q}, \mathbb{F},...
  {\olsi{#1}} %if arg is just one char, e.g. K, L,...
}
% FROM TOKCYCLE:
\long\def\count@stringtoks#1{\tc@earg\count@toks{\string#1}}
\long\def\count@toks#1{\the\numexpr-1\count@@toks#1.\tc@endcnt}
\long\def\count@@toks#1#2\tc@endcnt{+1\tc@ifempty{#2}{\relax}{\count@@toks#2\tc@endcnt}}
\def\tc@ifempty#1{\tc@testxifx{\expandafter\relax\detokenize{#1}\relax}}
\long\def\tc@earg#1#2{\expandafter#1\expandafter{#2}}
\long\def\tctestifnum#1{\tctestifcon{\ifnum#1\relax}}
\long\def\tctestifcon#1{#1\expandafter\tc@exfirst\else\expandafter\tc@exsecond\fi}
\long\def\tc@testxifx{\tc@earg\tctestifx}
\long\def\tctestifx#1{\tctestifcon{\ifx#1}}
\long\def\tc@exfirst#1#2{#1}
\long\def\tc@exsecond#1#2{#2}
\makeatother

\newtheorem{lemma}{Lema}
\newtheorem{theorem}{Teorema}

\setlength\parindent{0pt}
\setlength\parskip{4pt}



%% Listings

\definecolor{codegreen}{rgb}{0,0.6,0}
\definecolor{codegray}{rgb}{0.5,0.5,0.5}
\definecolor{codepurple}{rgb}{0.58,0,0.82}
\definecolor{backcolour}{rgb}{0.95,0.95,0.92}

\lstdefinestyle{mystyle}{
    backgroundcolor=\color{backcolour},   
    commentstyle=\color{codegreen},
    keywordstyle=\color{magenta},
    numberstyle=\tiny\color{codegray},
    stringstyle=\color{codepurple},
    basicstyle=\ttfamily\footnotesize,
    breakatwhitespace=false,         
    breaklines=true,                 
    captionpos=b,                    
    keepspaces=true,                 
    numbers=left,                    
    numbersep=5pt,                  
    showspaces=false,                
    showstringspaces=false,
    showtabs=false,                  
    tabsize=2
}

\lstset{style=mystyle}


\title{Cómputo científico para probabilidad y estadística. Tarea 7.\\
MCMC: Metropolis-Hastings II}
\author{Juan Esaul González Rangel}
\date{Octubre 2023}



\begin{document}

\maketitle

\textbf{ \large Con el algoritmo Metropolis-Hastings (MH), simular lo siguiente: }

\begin{enumerate}

    \item Sean $x_i \sim Ga(\alpha, \beta); i = 1, 2, . . . , n$. Simular datos $x_i$ con $\alpha = 3$ y $\beta = 100$ considerando los casos $n = 4$ y 30.
    
    Con $\alpha \sim U(1,4), \beta \sim exp(1)$ distribuciones a priori, se tiene la posterior

    \[ f(\alpha, \beta | \bar x) \propto \frac{\beta^{n\alpha}}{\Gamma(\alpha)^n} r_1^{\alpha - 1} e^{-\beta(r_2 + 1)} \mathds 1_{1 \le \alpha \le 4} \mathds 1_{\beta > 0}, \]

    con $r_2 = \sum_{i=1}^n x_i$ y $r_1 = \prod_{i=1}^n x_i$.

    En ambos casos, grafica los contornos para visualizar dónde está concentrada la posterior.
    
    Utilizar la propuesta

    \[ q \left( \binom{\alpha_p}{\beta_p} \left| \binom{\alpha}\beta \right. \right) = \binom{\alpha}{\beta} + \binom{\epsilon_1}{\epsilon_2}, \]

    donde

    \[ \binom{\epsilon_1}{\epsilon_2} \sim \mathcal N_2\left( \binom 00, \begin{pmatrix}
        \sigma_1^2 & 0 \\
        0 & \sigma_2^2
    \end{pmatrix} \right). \]


    \item Simular de la distribución Gamma$(\alpha,1)$ con la propuesta Gamma$([\alpha],1)$, donde $[\alpha]$ denota la parte entera de $\alpha$. 
    
    Además, realizar el siguiente experimento: poner como punto inicial $x_0 = 900$ y graficar la evolución de la cadena, es decir, $f (X_t)$ vs $t$.

    \item Implementar Random Walk Metropolis Hasting (RWMH) donde la distribución objetivo es $\mathcal N_2(\mu, \Sigma)$, con

    \[ \mu = \binom{3}{5} \quad \Sigma = \begin{pmatrix}
        1 & 0.9 \\
        0.9 & 1
    \end{pmatrix}. \]


Utilizar como propuesta $\epsilon_t \sim \mathcal N_2 (\mathbf0, \sigma I)$. ¿Cómo elegir $\sigma$ para que la cadena sea eficiente? ¿Qué consecuencias tiene la elección de $\sigma$?

Como experimento, elige como punto inicial $x_o = \binom{1000}{1}$ y comenta los resultados.

\textbf{Para todos los incisos del ejercicio anterior:}

\begin{itemize}
    \item Establece cual es tu distribución inicial.
    \item Grafica la evolución de la cadena.
    \item Indica cuál es el Burn-in.
    \item Comenta qué tan eficiente es la cadena.
    \item Implementa el algoritmo MH considerando una propuesta diferente.
\end{itemize}
   
\end{enumerate}





 \end{document}