\documentclass{article}
\usepackage[utf8]{inputenc}

\usepackage[utf8]{inputenc}
\usepackage[spanish,es-tabla,es-nodecimaldot]{babel}
\usepackage{amsmath,amsthm,amsfonts,amssymb,mathtools,dsfont,mathrsfs}
\usepackage{enumerate,graphicx,xcolor}
\usepackage{lmodern}
\usepackage[T1]{fontenc}
\usepackage[left=2cm,top=2.5cm,right=2cm,bottom=2.5cm]{geometry}
\usepackage[activate={true,nocompatibility},final,tracking=true,kerning=true,spacing=true,factor=1100,stretch=10,shrink=10]{microtype}
\usepackage{hyperref}


%\DeclarePairedDelimiter{\norm}{\lVert}{\rVert}




\newcommand{\N}{\mathbb{N}}
\newcommand{\R}{\mathbb R}
\newcommand{\Z}{\mathbb Z}
\newcommand{\Rbar}{\overline{\mathbb R}}
\newcommand{\F}{\mathscr F}
\newcommand{\A}{\mathscr A}
\newcommand{\To}{\Rightarrow}
\newcommand{\C}{\mathscr C}
\newcommand{\La}{\mathscr L_A}
\newcommand{\B}{\mathcal B}
\newcommand{\Q}{\mathbb Q}
\renewcommand{\epsilon}{\varepsilon}
\renewcommand{\L}{\mathcal L}
\renewcommand{\d}{\mathrm d}
\newcommand{\abs}[1]{\left| #1 \right|}
\newcommand{\pts}[1]{\left( #1 \right)}
\newcommand{\norm}[1]{\left\lVert#1\right\rVert}
\renewcommand{\P}[1]{\mathbb P\left( #1 \right)}
\newcommand{\E}[1]{\mathbb E \left( #1 \right)}


\newcommand{\ols}[1]{\mskip.5\thinmuskip\overline{\mskip-.5\thinmuskip {#1} \mskip-.5\thinmuskip}\mskip.5\thinmuskip} % overline short
\newcommand{\olsi}[1]{\,\overline{\!{#1}}} % overline short italic
\makeatletter
\newcommand\closure[1]{
  \tctestifnum{\count@stringtoks{#1}>1} %checks if number of chars in arg > 1 (including '\')
  {\ols{#1}} %if arg is longer than just one char, e.g. \mathbb{Q}, \mathbb{F},...
  {\olsi{#1}} %if arg is just one char, e.g. K, L,...
}
% FROM TOKCYCLE:
\long\def\count@stringtoks#1{\tc@earg\count@toks{\string#1}}
\long\def\count@toks#1{\the\numexpr-1\count@@toks#1.\tc@endcnt}
\long\def\count@@toks#1#2\tc@endcnt{+1\tc@ifempty{#2}{\relax}{\count@@toks#2\tc@endcnt}}
\def\tc@ifempty#1{\tc@testxifx{\expandafter\relax\detokenize{#1}\relax}}
\long\def\tc@earg#1#2{\expandafter#1\expandafter{#2}}
\long\def\tctestifnum#1{\tctestifcon{\ifnum#1\relax}}
\long\def\tctestifcon#1{#1\expandafter\tc@exfirst\else\expandafter\tc@exsecond\fi}
\long\def\tc@testxifx{\tc@earg\tctestifx}
\long\def\tctestifx#1{\tctestifcon{\ifx#1}}
\long\def\tc@exfirst#1#2{#1}
\long\def\tc@exsecond#1#2{#2}
\makeatother

\newtheorem{lemma}{Lema}
\newtheorem{theorem}{Teorema}

\setlength\parindent{0pt}
\setlength\parskip{4pt}


\title{Cómputo científico para probabilidad y estadística. Tarea 9.\\
MCMC: Tarea Final}
\author{Juan Esaul González Rangel}
\date{Diciembre 2023}



\begin{document}

\maketitle

En ambos problemas hay que diseñar e implementar el MCMC, investigar sobre su 
convergencia y tener algún grado de certeza sobre si sí se está simulando de la 
posterior correspondiente. Más aún, recuerde que se trata de un problema de 
inferencia: Hay que hablar del problema en sí, comentar sobre las posteriores 
simuladas y posibles estimadores (a partir de la muestra de posterior) que se 
pueden proporcionar de cada parámetro.

\begin{enumerate}
    \item (\textbf{Problema en ecología}) Sean $X_1, \dots, X_m$ variables aleatorias donde $X_i$ denota el número de individuos de una especie en cierta región. Suponga que $X_i|N, p \sim $Binomial$(N, p)$, entonces
    
    \[f (\bar x|N, p) = \prod_{i=1}^m \frac{N!}{x_i!(N - x_i)!}p^{x_i} (1 - p)^{N - x_i}.\]

    Asumiendo la distribución a priori $p \sim $Beta$(\alpha, \beta)$ y $N \sim h(\cdot)$, donde $h$ es una dist. discreta en $\{0, 1, 2, \dots , N_{\max} \}$, se tiene definida la distribución posterior $f (N, p |\bar x)$.
    
    A partir del algoritmo MH, simule valores de la distribución posterior usando un kernel híbrido. Para ello considere como sugerencia la siguiente distribución inicial para el MCMC 
    
    \[p \sim U(0, 1) \text{ y } N \sim U_d \left\{\max_{i\in\{1,\dots,m\}}(x_i), \max_{i\in\{1,\dots,m\}}(x_i) + 1, \dots , N_{\max}\right\}\]
    
    y las propuestas

    \begin{itemize}
        \item Propuesta 1: De la condicional total de p (kernel Gibbs).
        \item Propuesta 2: De la a priori.
        \item Propuesta 3: Propuesta hipergeométrica (¿?).
        \item Propuesta 4: Poisson: $N_p \sim \max_{i\in\{1,dots,m\}}(x_i) + $Poisson(?).
        \item Propuesta 5: Caminata aleatoria
    
    \[N_p = N + \epsilon, \qquad P(\epsilon = 1) = \frac12 = P(\epsilon = -1).\]
    
    Los datos son estos: 7, 7, 8, 8, 9, 4, 7, 5, 5, 6, 9, 8, 11, 7, 5, 5, 7, 3, 10, 3.
    
    A priori, esperamos que sea difícil observar a los individuos entonces $\alpha = 1, \beta = 20$. La especie no es muy abundante y entonces $N_{\max} = 1000$ y $h(N ) = 1/(N_{\max} + 1); N \in \{0, 1, 2, . . . , N_{\max}\}$.

    Las propuestas y distribución inicial para el MCMC de arriba son \textbf{solamente sugerencia}, propongan otras propuestas, experimenten y comenten.
    \end{itemize}


    \item (\textbf{Estudio de mercado}) Se tiene un producto y se realiza una encuesta con el fin de estudiar cuánto se consume dependiendo de la edad. Sea $Y_i$ el monto de compra y $X_i$ la covariable la cual representa la edad.
    
    Suponga que $Y_i \sim Po(\lambda_i)$ (distribución Poisson con intensidad $\lambda_i$)
    
    \[\lambda_i = cg_b(x_i - a)\]

    para $g_b$ la siguiente función de liga 
    
    \[ g_b(x) = \exp\left(- \frac{x^2}{2b^2}\right) .\]

    O sea, se trata de regresión Poisson con una función liga no usual. Si $\lambda_i = 0$ entonces $P(Y_i = 0) = 1$. $a = $años medio del segmento (años), $c = $gasto promedio (pesos), $b = $``amplitud'' del segmento (años).

    Considere las distribuciones a priori
    
    \[a \sim N (35, 5), \qquad c \sim Gama(3, 3/950), \qquad b \sim Gama(2, 2/5).\]
    
    El segundo parámetro de la normal es desviación estandard y el segundo parámetro de las gammas es taza (rate).
    
    Usando MH simule de la distribución posterior de a, c y b.
    
    Los datos son estos, n = 100:

\begin{verbatim}
X = array([ 25, 18, 19, 51, 16, 59, 16, 54, 52, 16, 31, 31, 54, 26, 19, 13, 59, 48, 54, 23, 50, 59, 
55, 37, 61, 53, 56, 31, 34, 15, 41, 14, 13, 13, 32, 46, 17, 52, 54, 25, 61, 15, 53, 39, 33, 52, 65, 
35, 65, 26, 54, 16, 47, 14, 42, 47, 48, 25, 15, 46, 31, 50, 42, 23, 17, 47, 32, 65, 45, 28, 12, 22, 
30, 36, 33, 16, 39, 50, 13, 23, 50, 34, 19, 46, 43, 56, 52,42, 48, 55, 37, 21, 45, 64, 53, 16, 62, 
16, 25, 62])

Y = array([1275, 325, 517, 0, 86, 0, 101, 0, 0, 89, 78, 83, 0, 1074, 508, 5, 0, 0, 0, 1447, 0, 0, 
0, 0, 0, 0, 0, 87, 7, 37, 0, 15, 5, 6, 35, 0, 158, 0, 0, 1349, 0, 35, 0, 0, 12, 0, 0, 2, 0, 1117, 0, 
79, 0, 13, 0, 0, 0, 1334, 56, 0, 81, 0, 0, 1480, 177, 0, 29, 0, 0, 551, 0, 1338, 196, 0, 9, 104, 0, 
0, 3, 1430, 0, 2, 492, 0, 0, 0, 0, 0, 0, 0, 0, 1057, 0, 0, 0, 68, 0, 87, 1362, 0]) \end{verbatim}


    \begin{proof}[Solución]
        
    \end{proof}

    \item Investiga y describe muy brevemente los softwares OpenBugs, Nimble, 
    JAGS, DRAM, Rtwalk, Mcee Hammer, PyMCMC.

    En la página oficial de OpenBUGS se menciona que es un programa de código 
    abierto para la modelación Bayesiana desarrollado sobre WinBUGS. Se menciona
    tabién que el desarrollo de OpenBUGS ya no está activo y actualmente BUGS se
    enfoca en mantener el software MultiBUGS. Más concretamente, en 
    \textit{The BUGS project: Evolution, critique and future directions} se indica 
    que el proyecto BUGS (Bayesian inference Using Gibbs Sampling) surge en la 
    unidad de bioestadística del consejo de investigación médica de la Universidad
    de Cambridge en 1989.  
    
    Fundamentalmente, OpenBUGS (o WinBUGS o MultiBUGS) se basa en la modelación 
    gráfica de las dependencias entre parámetros. Mediante una gráfica dirigida
    acícilca (DAP) se puede expresar cuáles son los componentes del modelo y cómo
    se relacionan entre ellos, mediante dependencias estocásticas y funcionales.
    Las ventajas de utilizar esta representación de los modelos en OpenBUGS es que
    se permite expresar relaciones de manera sencilla mediante el lenguaje propio que
    incorpora el software, y que el tener la gráfica de relaciones permite que
    WinBUGS sepa exactamente qué cálculos son los mínimos que se necesitan realizar
    para encontrar un condicionamiento total, logrando de esta manera ahorrar recursos
    como tiempo o almacenamiento.

    Los softwares de BUGS son bastante populares, y esta popularidad se debe a que
    son muy flexibles y por lo tanto resultan útiles en una gran diversidad de 
    contextos. La desventaja de la flexibilidad que presenta OpenBUGS es que al
    ser usado en tantas situaciones distintas, pueden aparecer nuevos problemas
    derivados de casos específicos de aplicación quen no se habían considerado
    previamente.

    Según su página oficial, Nimble es una adopción de OpenBUGS a R que permite
    manipular modelos y objetos para usarlos dentro de R, a comparación de otras
    interfaces de OpenBUGS para R que únicamente permiten usar simulaciones MCMC
    dentro de R. Entre las ventajas que Nimble provee están el permitir distantas 
    parametrizaciones de las distribuciones y usar distribuciones definidas por
    el usuario.




   
\end{enumerate}




 \end{document}