\documentclass{article}
\usepackage[utf8]{inputenc}

\usepackage[utf8]{inputenc}
\usepackage[spanish,es-tabla,es-nodecimaldot]{babel}
\usepackage{amsmath,amsthm,amsfonts,amssymb,mathtools,dsfont,mathrsfs}
\usepackage{enumerate,graphicx,xcolor}
\usepackage{lmodern}
\usepackage[T1]{fontenc}
\usepackage[left=2cm,top=2.5cm,right=2cm,bottom=2.5cm]{geometry}
\usepackage[activate={true,nocompatibility},final,tracking=true,kerning=true,spacing=true,factor=1100,stretch=10,shrink=10]{microtype}
\usepackage{hyperref}


%\DeclarePairedDelimiter{\norm}{\lVert}{\rVert}




\newcommand{\N}{\mathbb{N}}
\newcommand{\R}{\mathbb R}
\newcommand{\Z}{\mathbb Z}
\newcommand{\Rbar}{\overline{\mathbb R}}
\newcommand{\F}{\mathscr F}
\newcommand{\A}{\mathscr A}
\newcommand{\To}{\Rightarrow}
\newcommand{\C}{\mathscr C}
\newcommand{\La}{\mathscr L_A}
\newcommand{\B}{\mathcal B}
\newcommand{\Q}{\mathbb Q}
\renewcommand{\epsilon}{\varepsilon}
\renewcommand{\L}{\mathcal L}
\renewcommand{\d}{\mathrm d}
\newcommand{\abs}[1]{\left| #1 \right|}
\newcommand{\pts}[1]{\left( #1 \right)}
\newcommand{\norm}[1]{\left\lVert#1\right\rVert}
\renewcommand{\P}[1]{\mathbb P\left( #1 \right)}
\newcommand{\E}[1]{\mathbb E \left( #1 \right)}


\newcommand{\ols}[1]{\mskip.5\thinmuskip\overline{\mskip-.5\thinmuskip {#1} \mskip-.5\thinmuskip}\mskip.5\thinmuskip} % overline short
\newcommand{\olsi}[1]{\,\overline{\!{#1}}} % overline short italic
\makeatletter
\newcommand\closure[1]{
  \tctestifnum{\count@stringtoks{#1}>1} %checks if number of chars in arg > 1 (including '\')
  {\ols{#1}} %if arg is longer than just one char, e.g. \mathbb{Q}, \mathbb{F},...
  {\olsi{#1}} %if arg is just one char, e.g. K, L,...
}
% FROM TOKCYCLE:
\long\def\count@stringtoks#1{\tc@earg\count@toks{\string#1}}
\long\def\count@toks#1{\the\numexpr-1\count@@toks#1.\tc@endcnt}
\long\def\count@@toks#1#2\tc@endcnt{+1\tc@ifempty{#2}{\relax}{\count@@toks#2\tc@endcnt}}
\def\tc@ifempty#1{\tc@testxifx{\expandafter\relax\detokenize{#1}\relax}}
\long\def\tc@earg#1#2{\expandafter#1\expandafter{#2}}
\long\def\tctestifnum#1{\tctestifcon{\ifnum#1\relax}}
\long\def\tctestifcon#1{#1\expandafter\tc@exfirst\else\expandafter\tc@exsecond\fi}
\long\def\tc@testxifx{\tc@earg\tctestifx}
\long\def\tctestifx#1{\tctestifcon{\ifx#1}}
\long\def\tc@exfirst#1#2{#1}
\long\def\tc@exsecond#1#2{#2}
\makeatother

\newtheorem{lemma}{Lema}
\newtheorem{theorem}{Teorema}

\setlength\parindent{0pt}
\setlength\parskip{4pt}


\title{Cómputo científico para probabilidad y estadística. Tarea 11.\\
Deep neural networks}
\author{Juan Esaul González Rangel}
\date{Noviembre 2023}



\begin{document}

\maketitle


\begin{enumerate}

    \item Usando la base de datos MNIST realice lo siguiente y explique cada decisión tomada:

    \begin{itemize}
        \item Diseñe una red neuronal de una sola capa oculta para la clasificación de las imágenes. Use una función de pérdida predefinida.
        \item Entrene la red neuronal
        \item Presente la matriz de confusión (Confusion matrix),
        \item Describa que es la precisión y la exhaustividad (precision and recall) y calculelos a partir de su matriz de confusión
    \end{itemize}


    \item Usando la base de datos Fashion MNIST realice lo siguiente y explique
cada decisión tomada:

    \begin{itemize}
        \item Diseñe una red neuronal de dos capas ocultas para la clasificación de las imágenes, cada una con diferente función de activación y use una función de pérdida en la que el error de clasificar mal un calzado sea el doble que el resto de prendas.
        \item Entrene la red neuronal.
        \item Presente la matriz de confusión (Confusion matrix).
    \end{itemize}


    \item Con la base de datos CIFAR-10 realice lo siguiente:

    \begin{itemize}
        \item Diseñe una red neuronal de una capa oculta completamente conectada de al menos 10,000 neuronas para la clasificación de las imágenes
        \item Entrene la red neuronal
        \item Presente la matriz de confusión (Confusion matrix)
        \item Entrene una segunda red neuronal usando la estructura de LeNet 5.
        \item Presente la matriz de confusión (Confusion matrix)
        \item Compare ambas redes. Para esta comparación será necesario que al menos presente las curvas de error de entrenamiento y predicción para ambas redes, los tiempos de entrenamiento y que tome en cuenta las matrices de confusión.
    \end{itemize}


    \item Usando la base de datos MNIST y la estructura LeNet 5 realice lo siguiente:

    \begin{itemize}
        \item Entrene la red neuronal
        \item Implemente el algoritmo \textit{Deepfool} y construya un ejemplo adversario para cada una de las categorías, presente la imagen original, el ruido que añadió y la nueva imagen. 
    \end{itemize}
        
   
\end{enumerate}




 \end{document}