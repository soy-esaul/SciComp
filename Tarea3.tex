\documentclass{article}
\usepackage[utf8]{inputenc}

\usepackage[utf8]{inputenc}
\usepackage[spanish,es-tabla,es-nodecimaldot]{babel}
\usepackage{amsmath,amsthm,amsfonts,amssymb,mathtools,dsfont,mathrsfs}
\usepackage{enumerate,graphicx,xcolor}
\usepackage{lmodern}
\usepackage[T1]{fontenc}
\usepackage[left=2cm,top=2.5cm,right=2cm,bottom=2.5cm]{geometry}
\usepackage[activate={true,nocompatibility},final,tracking=true,kerning=true,spacing=true,factor=1100,stretch=10,shrink=10]{microtype}
\usepackage{hyperref}


%\DeclarePairedDelimiter{\norm}{\lVert}{\rVert}




\newcommand{\N}{\mathbb{N}}
\newcommand{\R}{\mathbb R}
\newcommand{\Z}{\mathbb Z}
\newcommand{\Rbar}{\overline{\mathbb R}}
\newcommand{\F}{\mathscr F}
\newcommand{\A}{\mathscr A}
\newcommand{\To}{\Rightarrow}
\newcommand{\C}{\mathscr C}
\newcommand{\La}{\mathscr L_A}
\newcommand{\B}{\mathcal B}
\newcommand{\Q}{\mathbb Q}
\renewcommand{\epsilon}{\varepsilon}
\renewcommand{\L}{\mathcal L}
\renewcommand{\d}{\mathrm d}
\newcommand{\abs}[1]{\left| #1 \right|}
\newcommand{\pts}[1]{\left( #1 \right)}
\newcommand{\norm}[1]{\left\lVert#1\right\rVert}
\renewcommand{\P}[1]{\mathbb P\left( #1 \right)}
\newcommand{\E}[1]{\mathbb E \left( #1 \right)}


\newcommand{\ols}[1]{\mskip.5\thinmuskip\overline{\mskip-.5\thinmuskip {#1} \mskip-.5\thinmuskip}\mskip.5\thinmuskip} % overline short
\newcommand{\olsi}[1]{\,\overline{\!{#1}}} % overline short italic
\makeatletter
\newcommand\closure[1]{
  \tctestifnum{\count@stringtoks{#1}>1} %checks if number of chars in arg > 1 (including '\')
  {\ols{#1}} %if arg is longer than just one char, e.g. \mathbb{Q}, \mathbb{F},...
  {\olsi{#1}} %if arg is just one char, e.g. K, L,...
}
% FROM TOKCYCLE:
\long\def\count@stringtoks#1{\tc@earg\count@toks{\string#1}}
\long\def\count@toks#1{\the\numexpr-1\count@@toks#1.\tc@endcnt}
\long\def\count@@toks#1#2\tc@endcnt{+1\tc@ifempty{#2}{\relax}{\count@@toks#2\tc@endcnt}}
\def\tc@ifempty#1{\tc@testxifx{\expandafter\relax\detokenize{#1}\relax}}
\long\def\tc@earg#1#2{\expandafter#1\expandafter{#2}}
\long\def\tctestifnum#1{\tctestifcon{\ifnum#1\relax}}
\long\def\tctestifcon#1{#1\expandafter\tc@exfirst\else\expandafter\tc@exsecond\fi}
\long\def\tc@testxifx{\tc@earg\tctestifx}
\long\def\tctestifx#1{\tctestifcon{\ifx#1}}
\long\def\tc@exfirst#1#2{#1}
\long\def\tc@exsecond#1#2{#2}
\makeatother

\newtheorem{lemma}{Lema}
\newtheorem{theorem}{Teorema}

\setlength\parindent{0pt}
\setlength\parskip{4pt}


\title{Cómputo científico para probabilidad y estadística. Tarea 3.\\
Estabilidad.}
\author{Juan Esaul González Rangel}
\date{Septiembre 2023}



\begin{document}

\maketitle


\begin{enumerate}

    \item Sea $Q$ una matriz unitaria aleatoria de $20 \times 20$ (eg. con $A$ una 
    matriz de tamaño $20 \times 20$ aleatoria calculen su descomposición $QR$). 
    Sean $\lambda_1 > \lambda_2 > ... \ge \lambda_{20} = 1 > 0$ y 
    
    \[B = Q^*diag(\lambda_1, \lambda_2, ..., \lambda_{20})Q, \text{ y } B_\epsilon 
    = Q^*diag(\lambda_1+\epsilon_1, \lambda_2+\epsilon_2, ..., \lambda_{20}+\epsilon_{20})Q,\]

con $\epsilon_i \sim N (0, \sigma)$, con $\sigma = 0.02\lambda_{20} = 0.01$.

\begin{enumerate}
    \item Comparar la descomposición de Cholesky de $B$ y de $B_\epsilon$ usando el 
    algoritmo de la tarea 1. Considerar los casos cuando $B$ tiene un buen número de 
    condición y un mal número de condición.

    Para comparar la descomposición Cholesky se usó el siguiente procedimiento:

    \begin{itemize}
        \item Se construyeron las matrices $B$ y $B_\epsilon$ y se obtuvo su descomposición 
        Cholesky, llamadas $B$-chol y $B_\epsilon$-chol.
        \item Se obtuvo la diferencia $B' = B$-chol$-B_\epsilon$-chol, que es una matriz triangular 
        superior y se evaluaron las siguientes métricas:
        \begin{itemize}
            \item Valor máximo de $|B'|$.
            \item Promedio de los valores de $B'$.
            \item Suma de los valores absolutos de las entradas de $B'$.
        \end{itemize}
    \end{itemize}

    Con base en estas comparaciones, podemos apreciar cómo es que la descomposición Cholesky de
    $B$ y de $B_\epsilon$ difieren entre sí. Los resultados de la comparación se resumen en 
    la siguiente tabla

    \begin{center}
        \begin{tabular}{@{}lrr@{}}
            \toprule
            Métrica                & \multicolumn{1}{l}{Caso bien condicionado} & \multicolumn{1}{l}{Caso mal condicionado} \\ \midrule
            Entrada máxima         & 0.011919704                                & 0.125                                     \\
            Media                  & -9.25E-07                                  & 0.0001284255518                           \\
            Suma de valor absoluto & 62.69409725                                & 42.63486008                               \\ \bottomrule
            \end{tabular}
    \end{center}
    
    Notemos que los errores de ambas matrices están centrados en un número cercano a cero, lo que significa que 
    modificar ligeramente a $B$ de manera centrada tuvo un efecto ``centrado'' también en el error que se ocasiona
    en su descomposición de Cholesky, pero el caso mal condionado tiene una media mayor, por lo que podemos decir
    que la perturbación tendió a ``sesgar'' más la aproximación.

    También podemos notar que, aunque la entrada máxima de la diferencia entre $B$-chol y $B_\epsilon$-chol
    es mayor para la matriz mal condicionada, la suma de los valores absolutos es mayor en el caso bien condicionado.
    La interpretación es que, a pesar de que en la matriz mal condicionada hay entradas de $B_\epsilon$-chol que cambian
    mucho respecto a las de $B$-chol, en general el error de aproximación ocasiona que las diferencias entre $B$ y $B_\epsilon$ no
    se vean reflejadas en las factorizaciones de Cholesky de cada una, por lo que podemos decir que en 
    el caso mal condicionado nuestra factorización de Cholesky es menos precisa (los pequeños cambios en la entrada 
    no se reflejan en la salida).

    \item Con el caso mal condicionado, comparar el resultado de su algoritmo con 
    el del algoritmo de Cholesky de scipy.
    
    Al utilizar los mismo pasos que en la comparación anterior, para la matriz mal condicionada
    y el algoritmo de Scipy, obtenemos la siguiente tabla. Recordemos que lo que se evalua son las entradas
    de la matriz $B\text{-chol} - B_\epsilon\text{-chol}$, donde $B\text{-chol}$ es la descomposición Cholesky de
    $B$ y $B_\epsilon\text{-chol}$, la correspondiente para $B_\epsilon$.

    \begin{center}
        \begin{tabular}{@{}lrr@{}}
            \toprule
            Métrica                & \multicolumn{1}{l}{Algoritmo propio} & \multicolumn{1}{l}{Algoritmo de Scipy} \\ \midrule
            Entrada máxima         & 0.125                                & 0.212919468                            \\
            Media                  & 0.0001284255518                      & -0.001174055979                        \\
            Suma de valor absoluto & 42.63486008                          & 66.79376308                            \\ \bottomrule
            \end{tabular}
    \end{center}

    Nuevamente tenemos que las diferencias están relativamente centradas, aunque el algoritmo de Scipy
    presenta una media un poco más lejana de cero. En esta ocasión, tanto la entrada máxima como la suma en valor 
    absoluto de la matriz obtenida con Scipy son mayores, lo que puede ser un indicio de que el algoritmo de
    Scipy implemente alguna función de robustez para evitar que el error se propague en problemas mal condicionados
    puesto que es esperable que los resultados de las descomposiciones para $B$ y $B_\epsilon$ varíen.

    \item Medir el tiempo de ejecución de su algoritmo de Cholesky con el de scipy.
    
    El tiempo de ejecución de ambos algoritmos era muy corto, lo que resultaba en mediciones nulas
    varias veces, por lo que la solución fue medir el tiempo total que se tarda cada algoritmo en
    obtener la factorización Cholesky un total de 1,000 veces para cada matriz ($B$ o $B_\epsilon$ 
    en los casos bien y mal condicionados). En la siguiente tabla, donde las cantidades se expresan 
    en segundos, se resume dicha información.

    \begin{center}
        \begin{tabular}{@{}lrrr@{}}
            \toprule
             & \multicolumn{1}{l}{Algoritmo propio} & \multicolumn{1}{l}{Algoritmo de Scipy} & \multicolumn{1}{l}{Proporción} \\ \midrule
            $B$ bien condicionada & 1.27788901 & 0.00851154 & 154.042549 \\
            $B_\epsilon$ bien condicionada & 1.36065578 & 0.00852394 & 142.8957261 \\
            $B$ mal condicionada & 1.34203696 & 0.01675606 & 72.9913062 \\
            $B_\epsilon$ mal condicionada & 1.22572351 & 0.00680399 & 180.0227416 \\ \bottomrule
            \end{tabular}
    \end{center}
    
    En la columna proporción se indica qué tanto más grande es el tiempo que llevó el cálculo
    con el algoritmo propio comparado con el cálculo con el algoritmo de Scipy. Notemos que Scipy
    es en general entre 70 y 180 veces más rápido para resolver estos problemas, y esto puede deberse al
    lenguaje en que está implementado Scipy.

    Una observación es que el algoritmo de Scipy tiene tiempos de ejecución similares para tres de los cuatro
    casos, excepto el caso $B$ mal condicionado en el que tarda alrededor del doble que en los otros, mientras 
    nuestro algoritmo propio presenta muy poca diferencia en cualquier caso. Una posible explicación es que 
    hay funciones extra del algoritmo de Scipy para lidiar con casos mal condicionados que causan un mayor
    tiempo de ejecución.

\end{enumerate}


    \item Resolver el problema de mínimos cuadrados,
    
        \[y = X\beta + \epsilon, \quad \epsilon_i \sim N (0, \sigma) \]

    usando su implementación de la descomposición $QR$; $\beta$ es de tamaño 
    $n \times 1$ y $X$ de tamaño $n \times d$.
    
    Sean $d = 5, n = 20, \beta = (5, 4, 3, 2, 1)'$ y $\sigma = 0.13$

    \begin{enumerate}
        \item Hacer $X$ con entradas aleatorias $U (0, 1)$ y simular $y$. Encontrar $\hat\beta$
         y compararlo con el obtenido $\hat\beta_p$ haciendo $X + \Delta X$, donde las 
         entradas de $\Delta X$ son $N (0, \sigma = 0.01)$. Comparar a su vez con 
         $\hat\beta_c = ((X + \Delta X)'(X + \Delta X))^{-1}(X + \Delta X)'y$ usando el 
         algoritmo genérico para invertir matrices \texttt{scipy.linalg.inv}.

        En la siguiente tabla se resumen los estimadores $\hat\beta$, $\hat\beta_p$ y $\hat\beta_c$
        para una matriz $X$ bien condicionada. 
        \begin{center}
            \begin{tabular}{@{}lrrrrr@{}}
                \toprule
                Valor real & 5 & 4 & 3 & 2 & 1 \\ \midrule
                $\hat\beta$ & 5.08438964 & 3.90654472 & 3.07956781 & 2.01345467 & 0.93787969 \\
                $\hat\beta_p$ & 5.04854568 & 3.93883121 & 3.17391487 & 1.94807777 & 0.8944391 \\
                $\hat\beta_c$ & 5.04854568 & 3.93883121 & 3.17391487 & 1.94807777 & 0.8944391 \\ \bottomrule
                \end{tabular}
        \end{center}

        Hay varias cosas que podemos observar en ella. Lo primero es que el error absoluto en todos los casos
        es del orden de 0.1 unidades, lo que significa que el error relativo es de entre 0.02 (para $\beta_1 = 5$)
        y 0.1 (para $\beta_5 = 1$). Considerando que la matriz aleatoria que se creó tiene entradas uniformes en $(0,1)$
        y que el ruido que se le añadió es Normal(0,0.13), podemos decir que la perturbación promedio de cada observación
        es de 13\%, por lo que es aceptable que los estimadores tengan una variación de entre 2\% y 10\%.

        También podemos notar que al añadir ruido a $X$ (cuando se considera $X + \Delta X$), los estimadores cambian, 
        pero este cambio no afecta a todas las entradas de la misma manera, pues las dos primeras entradas de 
        $\hat\beta_p$ son más cercanas a las correspondientes de $\beta$ que las de $\hat\beta$, aunque en las tres restantes
        la tendencia se revierte. Aunque tomando en cuenta el error total de todas las entradas, $\hat\beta$ es más
        cercano a $\beta$ que $\hat\beta_p$.

        Así mismo, notamos que $\hat\beta_c$ coincide a la perfección con $\hat\beta_p$. El estimador $\hat\beta_c$
        es el estimador teórico de mínimos cuadrados, y el hecho de que este coincida con $\beta_p$ nos indica
        que el algoritmo que estamos utilizando es efectivo para encontrar el estimador que buscamos. La
        diferencia es que el cálculo explícito de $\hat\beta_c$ requiere la inversión de una matriz, lo cual
        supone un costo computacional muy alto y por lo tanto es menos recomendable, mientras que $\hat\beta_p$ fue
        obtenido mediante un método más eficiente.

        En conclusión, con esta matriz notamos que los estimadores aproximan a los valores reales con un nivel de error
        aceptable (teniendo en cuenta la perturbación), que los cambios en la estimación cuando se realiza un cambio
        pequeño en la matriz $X$ también son pequeños, y que el método que estamos usando para encontrar el estimador
        de mínimos cuadrados es equivalente al método analítico clásico, aunque más eficiente numéricamente.

        \item Lo mismo que el anterior pero con $X$ mal condicionada (ie. con casi colinealidad).
        
        En la siguiente tabla se resumen los reusltados de este segundo inciso.

        \begin{center}
            \begin{tabular}{@{}lrrrrr@{}}
                \toprule
                Valor real & 5 & 4 & 3 & 2 & 1 \\ \midrule
                $\hat\beta$ & -73.13937309 & -2332.410907 & -2411.89615 & 158.6812259 & 4673.707698 \\
                $\hat\beta_p$ & 4.4716246 & 4.61787602 & 2.73137268 & 3.50345316 & -0.34596404 \\
                $\hat\beta_c$ & 4.4716246 & 4.61787602 & 2.73137268 & 3.50345316 & -0.34596404 \\ \bottomrule
                \end{tabular}
        \end{center}

        Lo primero que llama la atención es que ahora el estimador $\hat\beta$ es muy alejado tanto en magnitudes absolutas como
        relativas de lo que se esperaría, por lo que en la práctica un estimador como este es inutilizable. Además, aún en 
        este caso se conserva que $\hat\beta_p = \hat\beta_c$, lo que ssignifica que aún en el caso mal condicionado, el método
        de solución de mínimos cuadrados mediante $QR$ es equivalente al método clásico de solución de mínimos cuadrados.

        Es algo sorprendente que ahora $\hat\beta_p$ sea más cercano a $\beta$ que $\hat\beta$, y la explicación es que
        al agregar ruido aleatorio a cada columna de $X$, la matriz $X + \Delta X$ tiene columnas que están menos
        próximas a ser linealmente dependientes, y por lo tanto su número de condición es menor. La estimación $\hat\beta_p$
        es considerablemente mejor que $\hat\beta$ para cualquier entrada, pero aún así no llega a ser tan buena como cualquiera
        de las estimaciones del caso bien condicionado.

        Podemos concluir de todo esto que el método de resolución de mínimos cuadrados funciona incluso en el caso mal condicionado,
        pero que trabajar con una matriz con un mal número de condición puede alterar los resultados numéricos mucho más que trbajar 
        con datos en los que existe un error de aproximación. De hecho, como lo vimos en el caso mal condicionado, una matriz con datos
        aproximados puede ofrecernos una mejor solución cuando esto disminuye el número de condición.

    \end{enumerate}



   
\end{enumerate}




 \end{document}