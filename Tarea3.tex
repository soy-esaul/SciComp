\documentclass{article}
\usepackage[utf8]{inputenc}

\usepackage[utf8]{inputenc}
\usepackage[spanish,es-tabla,es-nodecimaldot]{babel}
\usepackage{amsmath,amsthm,amsfonts,amssymb,mathtools,dsfont,mathrsfs}
\usepackage{enumerate,graphicx,xcolor}
\usepackage{lmodern}
\usepackage[T1]{fontenc}
\usepackage[left=2cm,top=2.5cm,right=2cm,bottom=2.5cm]{geometry}
\usepackage[activate={true,nocompatibility},final,tracking=true,kerning=true,spacing=true,factor=1100,stretch=10,shrink=10]{microtype}
\usepackage{hyperref}


%\DeclarePairedDelimiter{\norm}{\lVert}{\rVert}




\newcommand{\N}{\mathbb{N}}
\newcommand{\R}{\mathbb R}
\newcommand{\Z}{\mathbb Z}
\newcommand{\Rbar}{\overline{\mathbb R}}
\newcommand{\F}{\mathscr F}
\newcommand{\A}{\mathscr A}
\newcommand{\To}{\Rightarrow}
\newcommand{\C}{\mathscr C}
\newcommand{\La}{\mathscr L_A}
\newcommand{\B}{\mathcal B}
\newcommand{\Q}{\mathbb Q}
\renewcommand{\epsilon}{\varepsilon}
\renewcommand{\L}{\mathcal L}
\renewcommand{\d}{\mathrm d}
\newcommand{\abs}[1]{\left| #1 \right|}
\newcommand{\pts}[1]{\left( #1 \right)}
\newcommand{\norm}[1]{\left\lVert#1\right\rVert}
\renewcommand{\P}[1]{\mathbb P\left( #1 \right)}
\newcommand{\E}[1]{\mathbb E \left( #1 \right)}


\newcommand{\ols}[1]{\mskip.5\thinmuskip\overline{\mskip-.5\thinmuskip {#1} \mskip-.5\thinmuskip}\mskip.5\thinmuskip} % overline short
\newcommand{\olsi}[1]{\,\overline{\!{#1}}} % overline short italic
\makeatletter
\newcommand\closure[1]{
  \tctestifnum{\count@stringtoks{#1}>1} %checks if number of chars in arg > 1 (including '\')
  {\ols{#1}} %if arg is longer than just one char, e.g. \mathbb{Q}, \mathbb{F},...
  {\olsi{#1}} %if arg is just one char, e.g. K, L,...
}
% FROM TOKCYCLE:
\long\def\count@stringtoks#1{\tc@earg\count@toks{\string#1}}
\long\def\count@toks#1{\the\numexpr-1\count@@toks#1.\tc@endcnt}
\long\def\count@@toks#1#2\tc@endcnt{+1\tc@ifempty{#2}{\relax}{\count@@toks#2\tc@endcnt}}
\def\tc@ifempty#1{\tc@testxifx{\expandafter\relax\detokenize{#1}\relax}}
\long\def\tc@earg#1#2{\expandafter#1\expandafter{#2}}
\long\def\tctestifnum#1{\tctestifcon{\ifnum#1\relax}}
\long\def\tctestifcon#1{#1\expandafter\tc@exfirst\else\expandafter\tc@exsecond\fi}
\long\def\tc@testxifx{\tc@earg\tctestifx}
\long\def\tctestifx#1{\tctestifcon{\ifx#1}}
\long\def\tc@exfirst#1#2{#1}
\long\def\tc@exsecond#1#2{#2}
\makeatother

\newtheorem{lemma}{Lema}
\newtheorem{theorem}{Teorema}

\setlength\parindent{0pt}
\setlength\parskip{4pt}


\title{Cómputo científico para probabilidad y estadística. Tarea 3.\\
Estabilidad.}
\author{Juan Esaul González Rangel}
\date{Septiembre 2023}



\begin{document}

\maketitle


\begin{enumerate}

    \item Sea $Q$ una matriz unitaria aleatoria de $20 \times 20$ (eg. con $A$ una 
    matriz de tamaño $20 \times 20$ aleatoria calculen su descomposición $QR$). 
    Sean $\lambda_1 > \lambda_2 > ... \ge \lambda_{20} = 1 > 0$ y 
    
    \[B = Q^*diag(\lambda_1, \lambda_2, ..., \lambda_{20})Q, \text{ y } B_\epsilon 
    = Q^*diag(\lambda_1+\epsilon_1, \lambda_2+\epsilon_2, ..., \lambda_{20}+\epsilon_{20})Q,\]

con $\epsilon_i \sim N (0, \sigma)$, con $\sigma = 0.02\lambda_{20} = 0.01$.

\begin{enumerate}
    \item Comparar la descomposición de Cholesky de $B$ y de $B_\epsilon$ usando el 
    algoritmo de la tarea 1. Considerar los casos cuando $B$ tiene un buen número de 
    condición y un mal número de condición.
    
    \item Con el caso mal condicionado, comparar el resultado de su algoritmo con 
    el del algoritmo de Cholesky de scipy.
    
    \item Medir el tiempo de ejecución de su algoritmo de Cholesky con el de scipy.
    
\end{enumerate}


    \item Resolver el problema de mínimos cuadrados,
    
        \[y = X\beta + \epsilon, \quad \epsilon_i \sim N (0, \sigma) \]

    usando su implementación de la descomposición $QR$; $\beta$ es de tamaño 
    $n \times 1$ y $X$ de tamaño $n \times d$.
    
    Sean $d = 5, n = 20, \beta = (5, 4, 3, 2, 1)'$ y $\sigma = 0.13$

    \begin{enumerate}
        \item Hacer $X$ con entradas aleatorias $U (0, 1)$ y simular $y$. Encontrar $\hat\beta$
         y compararlo con el obtenido $\hat\beta_p$ haciendo $X + \Delta X$, donde las 
         entradas de $\Delta X$ son $N (0, \sigma = 0.01)$. Comparar a su vez con 
         $\hat\beta_c = ((X + \Delta X)'(X + \Delta X))^{-1}(X + \Delta X)'y$ usando el 
         algoritmo genérico para invertir matrices \texttt{scipy.linalg.inv}.
        \item Lo mismo que el anterior pero con $X$ mal condicionada (ie. con casi colinealidad).
    \end{enumerate}



   
\end{enumerate}




 \end{document}