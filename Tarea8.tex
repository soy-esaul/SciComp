\documentclass{article}
\usepackage[utf8]{inputenc}

\usepackage[utf8]{inputenc}
\usepackage[spanish,es-tabla,es-nodecimaldot]{babel}
\usepackage{amsmath,amsthm,amsfonts,amssymb,mathtools,dsfont,mathrsfs}
\usepackage{enumerate,graphicx,xcolor}
\usepackage{lmodern}
\usepackage[T1]{fontenc}
\usepackage[left=2cm,top=2.5cm,right=2cm,bottom=2.5cm]{geometry}
\usepackage[activate={true,nocompatibility},final,tracking=true,kerning=true,spacing=true,factor=1100,stretch=10,shrink=10]{microtype}
\usepackage{hyperref}


%\DeclarePairedDelimiter{\norm}{\lVert}{\rVert}




\newcommand{\N}{\mathbb{N}}
\newcommand{\R}{\mathbb R}
\newcommand{\Z}{\mathbb Z}
\newcommand{\Rbar}{\overline{\mathbb R}}
\newcommand{\F}{\mathscr F}
\newcommand{\A}{\mathscr A}
\newcommand{\To}{\Rightarrow}
\newcommand{\C}{\mathscr C}
\newcommand{\La}{\mathscr L_A}
\newcommand{\B}{\mathcal B}
\newcommand{\Q}{\mathbb Q}
\renewcommand{\epsilon}{\varepsilon}
\renewcommand{\L}{\mathcal L}
\renewcommand{\d}{\mathrm d}
\newcommand{\abs}[1]{\left| #1 \right|}
\newcommand{\pts}[1]{\left( #1 \right)}
\newcommand{\norm}[1]{\left\lVert#1\right\rVert}
\renewcommand{\P}[1]{\mathbb P\left( #1 \right)}
\newcommand{\E}[1]{\mathbb E \left( #1 \right)}


\newcommand{\ols}[1]{\mskip.5\thinmuskip\overline{\mskip-.5\thinmuskip {#1} \mskip-.5\thinmuskip}\mskip.5\thinmuskip} % overline short
\newcommand{\olsi}[1]{\,\overline{\!{#1}}} % overline short italic
\makeatletter
\newcommand\closure[1]{
  \tctestifnum{\count@stringtoks{#1}>1} %checks if number of chars in arg > 1 (including '\')
  {\ols{#1}} %if arg is longer than just one char, e.g. \mathbb{Q}, \mathbb{F},...
  {\olsi{#1}} %if arg is just one char, e.g. K, L,...
}
% FROM TOKCYCLE:
\long\def\count@stringtoks#1{\tc@earg\count@toks{\string#1}}
\long\def\count@toks#1{\the\numexpr-1\count@@toks#1.\tc@endcnt}
\long\def\count@@toks#1#2\tc@endcnt{+1\tc@ifempty{#2}{\relax}{\count@@toks#2\tc@endcnt}}
\def\tc@ifempty#1{\tc@testxifx{\expandafter\relax\detokenize{#1}\relax}}
\long\def\tc@earg#1#2{\expandafter#1\expandafter{#2}}
\long\def\tctestifnum#1{\tctestifcon{\ifnum#1\relax}}
\long\def\tctestifcon#1{#1\expandafter\tc@exfirst\else\expandafter\tc@exsecond\fi}
\long\def\tc@testxifx{\tc@earg\tctestifx}
\long\def\tctestifx#1{\tctestifcon{\ifx#1}}
\long\def\tc@exfirst#1#2{#1}
\long\def\tc@exsecond#1#2{#2}
\makeatother

\newtheorem{lemma}{Lema}
\newtheorem{theorem}{Teorema}

\setlength\parindent{0pt}
\setlength\parskip{4pt}


\title{Cómputo científico para probabilidad y estadística. Tarea 8.\\
MCMC: MH con Kerneles Híbridos y Gibbs
Sampler}
\author{Juan Esaul González Rangel}
\date{Noviembre 2023}



\begin{document}

\maketitle


\begin{enumerate}

    \item Aplique el algoritmo de Metropolis-Hastings considerando como función 
    objetivo la distribución normal bivariada

    \[ f_{X_1,X_2}(\bar x) = \frac1{2\pi} |\Sigma|^{-1/2} \exp\left\{ -\frac12 
    (\bar x - \mu)'\Sigma^{-1}(\bar x - \mu) \right\} \]

    donde, 

    \[ \mu = \binom{\mu_1}{\mu_2} \quad \Sigma = \begin{pmatrix}
        \sigma_1^2 & \rho \sigma_1\sigma_2 \\
        \rho\sigma_1\sigma_2 & \sigma_2^2
    \end{pmatrix} \]
    
    Así, se tienen las siguientes distribuciones condicionales:
    
    \[ X_1 | X_2 = x_2 \sim N\left( \mu_1 + \rho \frac{\sigma_1}{\sigma_2}(x_2 - 
    \mu_2), \sigma_1^2(1 - \rho^2) \right) \]

    \[ X_2 | X_1 = x_1 \sim N\left( \mu_2 + \rho \frac{\sigma_2}{\sigma_1}(x_1 - 
    \mu_1), \sigma_2^2(1-\rho^2) \right) \]
    
    Considere las siguientes propuestas:

    \[ q_1 ((x_1',x_2') | (x_1,x_2)) = f_{X_1|X_2}(x_1'|x_2)\mathds 1_{(x_2' = x_2)}  \]

    \[ q_2 ((x_1',x_2') | (x_1,x_2)) = f_{X_2|X_1}(x_2'|x_1)\mathds 1_{(x_1' = x_1)}  \]

    
    A partir del algoritmo MH usando Kerneles híbridos simule valores de la distribución 
    normal bivariada, fijando $\sigma_1 = \sigma_2 = 1$, considere los casos 
    $\rho = 0.8$ y $\rho = 0.95$\footnote{Ver la tesis de Cricelio Montesinos para 
    una explicación más extensa del Gibbs, Montesinos, C (2016) ``Distribución de 
    Direcciones en el Gibbs Sampler Generalizad'', MSc Dissertation, CIMAT. 
    \url{https://www.cimat.mx/es/Tesis_digitales/}. También vean la Enciclopedia 
    de Estadística de Wiley, la entrada de \textit{Gibbs Sampler}: 
    \url{https://www.cimat.mx/~jac/2016WileytStatsRef_GibbsSampling.pdf.}}.



    \item Considere los tiempos de falla $t_1, \dots, t_n$ con distribución 
    \textit{Weibull}$(\alpha, \lambda)$:
    
    \[ f (t_i|\alpha, \lambda) = \alpha\lambda t^{\alpha-1}_i e^{-t^\alpha_i 
    \lambda} \]
    
    Se asumen como a priori $\alpha \sim \exp(c)$ y $\lambda|\alpha \sim 
    $Gama$(\alpha, b)$, por lo tanto, $f (\alpha, \lambda) = f (\lambda|\alpha) 
    f (\alpha)$\footnote{Este ejemplo aparece en Kundu, D. (2008), ``Bayesian 
    Inference and Life Testing Plan for the Weibull Distribution in Presence of 
    Progressive Censoring'', Technometrics, 50(2), 144–154.}. Así, para la 
    distribución posterior se tiene:
    
    \[f (\alpha, \lambda|\bar t) \propto f (\bar t|\alpha, \lambda)f (\alpha, 
    \lambda)\]
    
    A partir del algoritmo MH usando Kerneles híbridos simule valores de la distribución 
    posterior $f(\alpha, \lambda|\bar t)$, considerando las siguientes propuestas:


    \underline{Propuesta 1}:

    \[ \lambda_p|\alpha, \bar t \sim Gama \left(\alpha + n , b +\sum_{i=1}^n 
    t^\alpha_i \right) \quad \text{y dejando $\alpha$ fijo.} \]

    \underline{Propuesta 2:}

    \[\alpha_p|\lambda, \bar t \sim Gama (n + 1 , -\log(b) - \log(r_1) + c), 
    \text{ con } r_1 = \prod_{i=1}^n t_i \text{ y dejando $\lambda$ fijo. }\]
    
    \underline{Propuesta 3:}
    
    $\alpha_p \sim \exp(c)$ y $\lambda_p|\alpha_p \sim Gama(\alpha_p, b)$.

    \underline{Propuesta 4 (RWMH):} 
    
    $\alpha_p = \alpha + \epsilon$, con $\epsilon \sim N (0, \sigma)$ y dejando 
    $\lambda$ fijo. 
    
    Simular datos usando $\alpha = 1$ y $\lambda = 1$ con $n = 20$. 
    Para la a priori usar $c = 1$ y $b = 1$.



    \item Considere el ejemplo referente al número de fallas de bombas de agua en 
    una central nuclear\footnote{Este ejemplo fue usado en el artículo original 
    del Gibbs sampler del Gelfand y Smith (1990). Vea también Norton, R.A., 
    Christen, J.A. y Fox, C. (2017), ``Sampling hyperparameters in hierarchical 
    models: improving on Gibbs for high-dimensional latent fields and large data 
    set'' Communications in Statistics - Simulation and Computation, 
    \url{http://dx.doi.org/10.1080/03610918.2017.1353618} }, donde $p_i$ 
    representa el número de fallas en el tiempo de operación $t_i$, con $i = 1, 
    \dots, n$.

    Se considera el modelo $p_i \sim $Poisson$(\lambda_i t_i)$, (las $\lambda_i$ 
    son independientes entre si), con distribuciones a priori $\lambda_i|\beta 
    \sim Gama(\alpha, \beta)$ y $\beta \sim Gama(\gamma, \delta)$, por lo tanto:
    
    \[f (\lambda_1, \dots , \lambda_n, \beta) = f (\lambda_1|\beta)f (\lambda_2|
    \beta) \dots f(\lambda_n|\beta)f(\beta)\]
    
    Para la distribución posterior se tiene:
    
    \[f (\lambda_1, \dots , \lambda_n, \beta|\bar p) \propto L(\bar p, \bar \lambda,
     \beta)f (\lambda_1, \dots , \lambda_n, \beta)\]
    
    Simule valores de la distribución posterior $f (\lambda_1, \dots , \lambda_n, 
    \beta|\bar p)$, usando un kernel híbrido, considerando las propuestas: 
    
    \[\lambda_i|\bar \lambda_{-i}, \beta, \bar t \sim Gama(p_i + \alpha , \beta + 
    t_i)\]

    \begin{table}[!h] \centering
        \begin{tabular}{|l|l|l|l|l|l|l|l|l|l|l|}
            \hline
            Bomba ($i$)         & 1     & 2     & 3     & 4      & 5    & 6     & 7    & 8    & 9   & 10    \\ \hline
            T. de uso ($t_i$)    & 94.32 & 15.72 & 62.88 & 125.76 & 5.24 & 31.44 & 1.05 & 1.05 & 2.1 & 10.48 \\ \hline
            \# de fallas ($p_i$) & 5     & 1     & 5     & 14     & 3    & 17    & 1    & 1    & 4   & 22    \\ \hline
        \end{tabular} 
        \label{tab1}
        \caption{Datos de bombas de agua en centrales nucleares (Robert y Casella,
        p. 385) para el ejemplo 8.3.}
    \end{table}



    \[\beta|\bar \lambda, \bar t \sim Gama \left( n\alpha + \gamma , \delta + 
    \sum_{i=1}^n\lambda_i \right).\]
    
    Verifique que estas son propuestas Gibbs.

    Use los datos del Cuadro 1 con los parámetros a priori $\alpha = 1.8, 
    \gamma = 0.01$ y $\delta = 1$.


   
\end{enumerate}




 \end{document}