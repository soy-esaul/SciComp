\documentclass{article}
\usepackage[utf8]{inputenc}
\usepackage{tikz}
\usepackage{pgfplots}
\usepackage[utf8]{inputenc}
\usepackage[spanish,es-tabla,es-nodecimaldot]{babel}
\usepackage{amsmath,amsthm,amsfonts,amssymb,mathtools,dsfont,mathrsfs}
\usepackage{enumerate,graphicx,xcolor}
\usepackage{lmodern}
\usepackage[T1]{fontenc}
\usepackage[left=2cm,top=2.5cm,right=2cm,bottom=2.5cm]{geometry}
\usepackage[activate={true,nocompatibility},final,tracking=true,kerning=true,spacing=true,factor=1100,stretch=10,shrink=10]{microtype}
\usepackage{hyperref}
\usepackage{listings}
\usepackage{booktabs}


%\DeclarePairedDelimiter{\norm}{\lVert}{\rVert}




\newcommand{\N}{\mathbb{N}}
\newcommand{\R}{\mathbb R}
\newcommand{\Z}{\mathbb Z}
\newcommand{\Rbar}{\overline{\mathbb R}}
\newcommand{\F}{\mathscr F}
\newcommand{\A}{\mathscr A}
\newcommand{\To}{\Rightarrow}
\newcommand{\C}{\mathscr C}
\newcommand{\La}{\mathscr L_A}
\newcommand{\B}{\mathcal B}
\newcommand{\Q}{\mathbb Q}
\renewcommand{\epsilon}{\varepsilon}
\renewcommand{\L}{\mathcal L}
\renewcommand{\d}{\mathrm d}
\newcommand{\abs}[1]{\left| #1 \right|}
\newcommand{\pts}[1]{\left( #1 \right)}
\newcommand{\norm}[1]{\left\lVert#1\right\rVert}
\renewcommand{\P}[1]{\mathbb P\left( #1 \right)}
\newcommand{\E}[1]{\mathbb E \left( #1 \right)}


\newcommand{\ols}[1]{\mskip.5\thinmuskip\overline{\mskip-.5\thinmuskip {#1} \mskip-.5\thinmuskip}\mskip.5\thinmuskip} % overline short
\newcommand{\olsi}[1]{\,\overline{\!{#1}}} % overline short italic
\makeatletter
\newcommand\closure[1]{
  \tctestifnum{\count@stringtoks{#1}>1} %checks if number of chars in arg > 1 (including '\')
  {\ols{#1}} %if arg is longer than just one char, e.g. \mathbb{Q}, \mathbb{F},...
  {\olsi{#1}} %if arg is just one char, e.g. K, L,...
}
% FROM TOKCYCLE:
\long\def\count@stringtoks#1{\tc@earg\count@toks{\string#1}}
\long\def\count@toks#1{\the\numexpr-1\count@@toks#1.\tc@endcnt}
\long\def\count@@toks#1#2\tc@endcnt{+1\tc@ifempty{#2}{\relax}{\count@@toks#2\tc@endcnt}}
\def\tc@ifempty#1{\tc@testxifx{\expandafter\relax\detokenize{#1}\relax}}
\long\def\tc@earg#1#2{\expandafter#1\expandafter{#2}}
\long\def\tctestifnum#1{\tctestifcon{\ifnum#1\relax}}
\long\def\tctestifcon#1{#1\expandafter\tc@exfirst\else\expandafter\tc@exsecond\fi}
\long\def\tc@testxifx{\tc@earg\tctestifx}
\long\def\tctestifx#1{\tctestifcon{\ifx#1}}
\long\def\tc@exfirst#1#2{#1}
\long\def\tc@exsecond#1#2{#2}
\makeatother

\newtheorem{lemma}{Lema}
\newtheorem{theorem}{Teorema}

\setlength\parindent{0pt}
\setlength\parskip{4pt}



%% Listings

\definecolor{codegreen}{rgb}{0,0.6,0}
\definecolor{codegray}{rgb}{0.5,0.5,0.5}
\definecolor{codepurple}{rgb}{0.58,0,0.82}
\definecolor{backcolour}{rgb}{0.95,0.95,0.92}

\lstdefinestyle{mystyle}{
    backgroundcolor=\color{backcolour},   
    commentstyle=\color{codegreen},
    keywordstyle=\color{magenta},
    numberstyle=\tiny\color{codegray},
    stringstyle=\color{codepurple},
    basicstyle=\ttfamily\footnotesize,
    breakatwhitespace=false,         
    breaklines=true,                 
    captionpos=b,                    
    keepspaces=true,                 
    numbers=left,                    
    numbersep=5pt,                  
    showspaces=false,                
    showstringspaces=false,
    showtabs=false,                  
    tabsize=2
}

\lstset{style=mystyle}


\title{Cómputo científico para probabilidad y estadística. Tarea 6.\\
MCMC: Metropolis-Hastings.}
\author{Juan Esaul González Rangel}
\date{Octubre 2023}



\begin{document}

\maketitle


\begin{enumerate}

    \item  Simular $n = 5$ y $n = 40$ v.a Bernoulli $Be(1/3)$; sea $r$ el número de
éxitos en cada caso.

    La simulación de las variables aleaotorias se logra con el siguiente código,

    \begin{lstlisting}[language=Python]
# Samples of Bernoulli
sample5 = np.random.binomial(5,1/3,1)
sample40 = np.random.binomial(40,1/3,1)
r5 = np.sum(sample5)
r40 = np.sum(sample40)\end{lstlisting}

    Usando la semilla 57 encontramos los siguiente valores para $r_5$ y $r_{40}$,

    \begin{align*}
        r_5 &= 3,\\
        r_{40} &= 15.
    \end{align*}

    \item Implementar el algoritmo Metropolis-Hastings para simular de la posterior
    
    \[f (p|\bar{x}) \propto p^r(1 - p)^{n-r} \cos(\pi p)I_{[0, 1/2 ]}(p),\]

    con los dos casos de $n$ y $r$ de arriba. Para ello poner la propuesta $(p'|p) = p' \sim Beta(r + 1, n - r + 1)$ 
    y la distribución inicial de la cadena $\mu \sim U (0, 1/2 )$.

    El algoritmo se encuentra implementado en la función \texttt{MH\_beta()} del archivo
    \texttt{Tarea6.py}. La función toma tres parámetros con nombre \texttt{iterations}, \texttt{trials}
    y \texttt{successes}. El segundo y tercer argumentos se refieren a la cantidad de ensayos
    Bernoulli independientes que se realizaron y cuáles de ellos resultaron en éxito. 
    
    El primer argumento representa la cantidad de iteraciones que realizará el algoritmo. 
    Este número coincide con la cantidad de valores de la muestra que
    se obtienen, pero algunos serán repetidos. Una manera de evaluar qué tan buena es la
    propuesta de transición dada para el problema es cuestión es medir qué tanto 
    tiempo permanece la cadena en un mismo estado. Cuando una cadena tarda más en aceptar
    la transición, es natural esperar que tarde más en converger a la distribución estacionaria.

    El siguiente es un ejemplo sencillo de uso de la función,

\begin{lstlisting}[language=Python]
simple_sample = MH_beta(1000,50,28)
plt.hist(simple_sample,density=True)\end{lstlisting}

    En el código anterior, se considera que se obtuvo una muestra de 50 ensayos Bernoulli
    de los cuáles 28 fueron éxitos.

    El histograma de la muestra anterior es el siguiente,

    \begin{center}
        %% Creator: Matplotlib, PGF backend
%%
%% To include the figure in your LaTeX document, write
%%   \input{<filename>.pgf}
%%
%% Make sure the required packages are loaded in your preamble
%%   \usepackage{pgf}
%%
%% Also ensure that all the required font packages are loaded; for instance,
%% the lmodern package is sometimes necessary when using math font.
%%   \usepackage{lmodern}
%%
%% Figures using additional raster images can only be included by \input if
%% they are in the same directory as the main LaTeX file. For loading figures
%% from other directories you can use the `import` package
%%   \usepackage{import}
%%
%% and then include the figures with
%%   \import{<path to file>}{<filename>.pgf}
%%
%% Matplotlib used the following preamble
%%   
%%   \makeatletter\@ifpackageloaded{underscore}{}{\usepackage[strings]{underscore}}\makeatother
%%
\begingroup%
\makeatletter%
\begin{pgfpicture}%
\pgfpathrectangle{\pgfpointorigin}{\pgfqpoint{8.000000in}{5.500000in}}%
\pgfusepath{use as bounding box, clip}%
\begin{pgfscope}%
\pgfsetbuttcap%
\pgfsetmiterjoin%
\definecolor{currentfill}{rgb}{1.000000,1.000000,1.000000}%
\pgfsetfillcolor{currentfill}%
\pgfsetlinewidth{0.000000pt}%
\definecolor{currentstroke}{rgb}{1.000000,1.000000,1.000000}%
\pgfsetstrokecolor{currentstroke}%
\pgfsetdash{}{0pt}%
\pgfpathmoveto{\pgfqpoint{0.000000in}{0.000000in}}%
\pgfpathlineto{\pgfqpoint{8.000000in}{0.000000in}}%
\pgfpathlineto{\pgfqpoint{8.000000in}{5.500000in}}%
\pgfpathlineto{\pgfqpoint{0.000000in}{5.500000in}}%
\pgfpathlineto{\pgfqpoint{0.000000in}{0.000000in}}%
\pgfpathclose%
\pgfusepath{fill}%
\end{pgfscope}%
\begin{pgfscope}%
\pgfsetbuttcap%
\pgfsetmiterjoin%
\definecolor{currentfill}{rgb}{0.917647,0.917647,0.949020}%
\pgfsetfillcolor{currentfill}%
\pgfsetlinewidth{0.000000pt}%
\definecolor{currentstroke}{rgb}{0.000000,0.000000,0.000000}%
\pgfsetstrokecolor{currentstroke}%
\pgfsetstrokeopacity{0.000000}%
\pgfsetdash{}{0pt}%
\pgfpathmoveto{\pgfqpoint{1.000000in}{0.605000in}}%
\pgfpathlineto{\pgfqpoint{7.200000in}{0.605000in}}%
\pgfpathlineto{\pgfqpoint{7.200000in}{4.840000in}}%
\pgfpathlineto{\pgfqpoint{1.000000in}{4.840000in}}%
\pgfpathlineto{\pgfqpoint{1.000000in}{0.605000in}}%
\pgfpathclose%
\pgfusepath{fill}%
\end{pgfscope}%
\begin{pgfscope}%
\pgfpathrectangle{\pgfqpoint{1.000000in}{0.605000in}}{\pgfqpoint{6.200000in}{4.235000in}}%
\pgfusepath{clip}%
\pgfsetroundcap%
\pgfsetroundjoin%
\pgfsetlinewidth{1.003750pt}%
\definecolor{currentstroke}{rgb}{1.000000,1.000000,1.000000}%
\pgfsetstrokecolor{currentstroke}%
\pgfsetdash{}{0pt}%
\pgfpathmoveto{\pgfqpoint{1.986109in}{0.605000in}}%
\pgfpathlineto{\pgfqpoint{1.986109in}{4.840000in}}%
\pgfusepath{stroke}%
\end{pgfscope}%
\begin{pgfscope}%
\definecolor{textcolor}{rgb}{0.150000,0.150000,0.150000}%
\pgfsetstrokecolor{textcolor}%
\pgfsetfillcolor{textcolor}%
\pgftext[x=1.986109in,y=0.507778in,,top]{\color{textcolor}\rmfamily\fontsize{10.000000}{12.000000}\selectfont \(\displaystyle {0.1}\)}%
\end{pgfscope}%
\begin{pgfscope}%
\pgfpathrectangle{\pgfqpoint{1.000000in}{0.605000in}}{\pgfqpoint{6.200000in}{4.235000in}}%
\pgfusepath{clip}%
\pgfsetroundcap%
\pgfsetroundjoin%
\pgfsetlinewidth{1.003750pt}%
\definecolor{currentstroke}{rgb}{1.000000,1.000000,1.000000}%
\pgfsetstrokecolor{currentstroke}%
\pgfsetdash{}{0pt}%
\pgfpathmoveto{\pgfqpoint{3.236511in}{0.605000in}}%
\pgfpathlineto{\pgfqpoint{3.236511in}{4.840000in}}%
\pgfusepath{stroke}%
\end{pgfscope}%
\begin{pgfscope}%
\definecolor{textcolor}{rgb}{0.150000,0.150000,0.150000}%
\pgfsetstrokecolor{textcolor}%
\pgfsetfillcolor{textcolor}%
\pgftext[x=3.236511in,y=0.507778in,,top]{\color{textcolor}\rmfamily\fontsize{10.000000}{12.000000}\selectfont \(\displaystyle {0.2}\)}%
\end{pgfscope}%
\begin{pgfscope}%
\pgfpathrectangle{\pgfqpoint{1.000000in}{0.605000in}}{\pgfqpoint{6.200000in}{4.235000in}}%
\pgfusepath{clip}%
\pgfsetroundcap%
\pgfsetroundjoin%
\pgfsetlinewidth{1.003750pt}%
\definecolor{currentstroke}{rgb}{1.000000,1.000000,1.000000}%
\pgfsetstrokecolor{currentstroke}%
\pgfsetdash{}{0pt}%
\pgfpathmoveto{\pgfqpoint{4.486913in}{0.605000in}}%
\pgfpathlineto{\pgfqpoint{4.486913in}{4.840000in}}%
\pgfusepath{stroke}%
\end{pgfscope}%
\begin{pgfscope}%
\definecolor{textcolor}{rgb}{0.150000,0.150000,0.150000}%
\pgfsetstrokecolor{textcolor}%
\pgfsetfillcolor{textcolor}%
\pgftext[x=4.486913in,y=0.507778in,,top]{\color{textcolor}\rmfamily\fontsize{10.000000}{12.000000}\selectfont \(\displaystyle {0.3}\)}%
\end{pgfscope}%
\begin{pgfscope}%
\pgfpathrectangle{\pgfqpoint{1.000000in}{0.605000in}}{\pgfqpoint{6.200000in}{4.235000in}}%
\pgfusepath{clip}%
\pgfsetroundcap%
\pgfsetroundjoin%
\pgfsetlinewidth{1.003750pt}%
\definecolor{currentstroke}{rgb}{1.000000,1.000000,1.000000}%
\pgfsetstrokecolor{currentstroke}%
\pgfsetdash{}{0pt}%
\pgfpathmoveto{\pgfqpoint{5.737315in}{0.605000in}}%
\pgfpathlineto{\pgfqpoint{5.737315in}{4.840000in}}%
\pgfusepath{stroke}%
\end{pgfscope}%
\begin{pgfscope}%
\definecolor{textcolor}{rgb}{0.150000,0.150000,0.150000}%
\pgfsetstrokecolor{textcolor}%
\pgfsetfillcolor{textcolor}%
\pgftext[x=5.737315in,y=0.507778in,,top]{\color{textcolor}\rmfamily\fontsize{10.000000}{12.000000}\selectfont \(\displaystyle {0.4}\)}%
\end{pgfscope}%
\begin{pgfscope}%
\pgfpathrectangle{\pgfqpoint{1.000000in}{0.605000in}}{\pgfqpoint{6.200000in}{4.235000in}}%
\pgfusepath{clip}%
\pgfsetroundcap%
\pgfsetroundjoin%
\pgfsetlinewidth{1.003750pt}%
\definecolor{currentstroke}{rgb}{1.000000,1.000000,1.000000}%
\pgfsetstrokecolor{currentstroke}%
\pgfsetdash{}{0pt}%
\pgfpathmoveto{\pgfqpoint{6.987717in}{0.605000in}}%
\pgfpathlineto{\pgfqpoint{6.987717in}{4.840000in}}%
\pgfusepath{stroke}%
\end{pgfscope}%
\begin{pgfscope}%
\definecolor{textcolor}{rgb}{0.150000,0.150000,0.150000}%
\pgfsetstrokecolor{textcolor}%
\pgfsetfillcolor{textcolor}%
\pgftext[x=6.987717in,y=0.507778in,,top]{\color{textcolor}\rmfamily\fontsize{10.000000}{12.000000}\selectfont \(\displaystyle {0.5}\)}%
\end{pgfscope}%
\begin{pgfscope}%
\definecolor{textcolor}{rgb}{0.150000,0.150000,0.150000}%
\pgfsetstrokecolor{textcolor}%
\pgfsetfillcolor{textcolor}%
\pgftext[x=4.100000in,y=0.328766in,,top]{\color{textcolor}\rmfamily\fontsize{11.000000}{13.200000}\selectfont Valor}%
\end{pgfscope}%
\begin{pgfscope}%
\pgfpathrectangle{\pgfqpoint{1.000000in}{0.605000in}}{\pgfqpoint{6.200000in}{4.235000in}}%
\pgfusepath{clip}%
\pgfsetroundcap%
\pgfsetroundjoin%
\pgfsetlinewidth{1.003750pt}%
\definecolor{currentstroke}{rgb}{1.000000,1.000000,1.000000}%
\pgfsetstrokecolor{currentstroke}%
\pgfsetdash{}{0pt}%
\pgfpathmoveto{\pgfqpoint{1.000000in}{0.605000in}}%
\pgfpathlineto{\pgfqpoint{7.200000in}{0.605000in}}%
\pgfusepath{stroke}%
\end{pgfscope}%
\begin{pgfscope}%
\definecolor{textcolor}{rgb}{0.150000,0.150000,0.150000}%
\pgfsetstrokecolor{textcolor}%
\pgfsetfillcolor{textcolor}%
\pgftext[x=0.833333in, y=0.556775in, left, base]{\color{textcolor}\rmfamily\fontsize{10.000000}{12.000000}\selectfont \(\displaystyle {0}\)}%
\end{pgfscope}%
\begin{pgfscope}%
\pgfpathrectangle{\pgfqpoint{1.000000in}{0.605000in}}{\pgfqpoint{6.200000in}{4.235000in}}%
\pgfusepath{clip}%
\pgfsetroundcap%
\pgfsetroundjoin%
\pgfsetlinewidth{1.003750pt}%
\definecolor{currentstroke}{rgb}{1.000000,1.000000,1.000000}%
\pgfsetstrokecolor{currentstroke}%
\pgfsetdash{}{0pt}%
\pgfpathmoveto{\pgfqpoint{1.000000in}{1.214682in}}%
\pgfpathlineto{\pgfqpoint{7.200000in}{1.214682in}}%
\pgfusepath{stroke}%
\end{pgfscope}%
\begin{pgfscope}%
\definecolor{textcolor}{rgb}{0.150000,0.150000,0.150000}%
\pgfsetstrokecolor{textcolor}%
\pgfsetfillcolor{textcolor}%
\pgftext[x=0.833333in, y=1.166457in, left, base]{\color{textcolor}\rmfamily\fontsize{10.000000}{12.000000}\selectfont \(\displaystyle {2}\)}%
\end{pgfscope}%
\begin{pgfscope}%
\pgfpathrectangle{\pgfqpoint{1.000000in}{0.605000in}}{\pgfqpoint{6.200000in}{4.235000in}}%
\pgfusepath{clip}%
\pgfsetroundcap%
\pgfsetroundjoin%
\pgfsetlinewidth{1.003750pt}%
\definecolor{currentstroke}{rgb}{1.000000,1.000000,1.000000}%
\pgfsetstrokecolor{currentstroke}%
\pgfsetdash{}{0pt}%
\pgfpathmoveto{\pgfqpoint{1.000000in}{1.824364in}}%
\pgfpathlineto{\pgfqpoint{7.200000in}{1.824364in}}%
\pgfusepath{stroke}%
\end{pgfscope}%
\begin{pgfscope}%
\definecolor{textcolor}{rgb}{0.150000,0.150000,0.150000}%
\pgfsetstrokecolor{textcolor}%
\pgfsetfillcolor{textcolor}%
\pgftext[x=0.833333in, y=1.776138in, left, base]{\color{textcolor}\rmfamily\fontsize{10.000000}{12.000000}\selectfont \(\displaystyle {4}\)}%
\end{pgfscope}%
\begin{pgfscope}%
\pgfpathrectangle{\pgfqpoint{1.000000in}{0.605000in}}{\pgfqpoint{6.200000in}{4.235000in}}%
\pgfusepath{clip}%
\pgfsetroundcap%
\pgfsetroundjoin%
\pgfsetlinewidth{1.003750pt}%
\definecolor{currentstroke}{rgb}{1.000000,1.000000,1.000000}%
\pgfsetstrokecolor{currentstroke}%
\pgfsetdash{}{0pt}%
\pgfpathmoveto{\pgfqpoint{1.000000in}{2.434045in}}%
\pgfpathlineto{\pgfqpoint{7.200000in}{2.434045in}}%
\pgfusepath{stroke}%
\end{pgfscope}%
\begin{pgfscope}%
\definecolor{textcolor}{rgb}{0.150000,0.150000,0.150000}%
\pgfsetstrokecolor{textcolor}%
\pgfsetfillcolor{textcolor}%
\pgftext[x=0.833333in, y=2.385820in, left, base]{\color{textcolor}\rmfamily\fontsize{10.000000}{12.000000}\selectfont \(\displaystyle {6}\)}%
\end{pgfscope}%
\begin{pgfscope}%
\pgfpathrectangle{\pgfqpoint{1.000000in}{0.605000in}}{\pgfqpoint{6.200000in}{4.235000in}}%
\pgfusepath{clip}%
\pgfsetroundcap%
\pgfsetroundjoin%
\pgfsetlinewidth{1.003750pt}%
\definecolor{currentstroke}{rgb}{1.000000,1.000000,1.000000}%
\pgfsetstrokecolor{currentstroke}%
\pgfsetdash{}{0pt}%
\pgfpathmoveto{\pgfqpoint{1.000000in}{3.043727in}}%
\pgfpathlineto{\pgfqpoint{7.200000in}{3.043727in}}%
\pgfusepath{stroke}%
\end{pgfscope}%
\begin{pgfscope}%
\definecolor{textcolor}{rgb}{0.150000,0.150000,0.150000}%
\pgfsetstrokecolor{textcolor}%
\pgfsetfillcolor{textcolor}%
\pgftext[x=0.833333in, y=2.995502in, left, base]{\color{textcolor}\rmfamily\fontsize{10.000000}{12.000000}\selectfont \(\displaystyle {8}\)}%
\end{pgfscope}%
\begin{pgfscope}%
\pgfpathrectangle{\pgfqpoint{1.000000in}{0.605000in}}{\pgfqpoint{6.200000in}{4.235000in}}%
\pgfusepath{clip}%
\pgfsetroundcap%
\pgfsetroundjoin%
\pgfsetlinewidth{1.003750pt}%
\definecolor{currentstroke}{rgb}{1.000000,1.000000,1.000000}%
\pgfsetstrokecolor{currentstroke}%
\pgfsetdash{}{0pt}%
\pgfpathmoveto{\pgfqpoint{1.000000in}{3.653409in}}%
\pgfpathlineto{\pgfqpoint{7.200000in}{3.653409in}}%
\pgfusepath{stroke}%
\end{pgfscope}%
\begin{pgfscope}%
\definecolor{textcolor}{rgb}{0.150000,0.150000,0.150000}%
\pgfsetstrokecolor{textcolor}%
\pgfsetfillcolor{textcolor}%
\pgftext[x=0.763888in, y=3.605184in, left, base]{\color{textcolor}\rmfamily\fontsize{10.000000}{12.000000}\selectfont \(\displaystyle {10}\)}%
\end{pgfscope}%
\begin{pgfscope}%
\pgfpathrectangle{\pgfqpoint{1.000000in}{0.605000in}}{\pgfqpoint{6.200000in}{4.235000in}}%
\pgfusepath{clip}%
\pgfsetroundcap%
\pgfsetroundjoin%
\pgfsetlinewidth{1.003750pt}%
\definecolor{currentstroke}{rgb}{1.000000,1.000000,1.000000}%
\pgfsetstrokecolor{currentstroke}%
\pgfsetdash{}{0pt}%
\pgfpathmoveto{\pgfqpoint{1.000000in}{4.263091in}}%
\pgfpathlineto{\pgfqpoint{7.200000in}{4.263091in}}%
\pgfusepath{stroke}%
\end{pgfscope}%
\begin{pgfscope}%
\definecolor{textcolor}{rgb}{0.150000,0.150000,0.150000}%
\pgfsetstrokecolor{textcolor}%
\pgfsetfillcolor{textcolor}%
\pgftext[x=0.763888in, y=4.214866in, left, base]{\color{textcolor}\rmfamily\fontsize{10.000000}{12.000000}\selectfont \(\displaystyle {12}\)}%
\end{pgfscope}%
\begin{pgfscope}%
\definecolor{textcolor}{rgb}{0.150000,0.150000,0.150000}%
\pgfsetstrokecolor{textcolor}%
\pgfsetfillcolor{textcolor}%
\pgftext[x=0.708333in,y=2.722500in,,bottom,rotate=90.000000]{\color{textcolor}\rmfamily\fontsize{11.000000}{13.200000}\selectfont Frecuencia}%
\end{pgfscope}%
\begin{pgfscope}%
\pgfpathrectangle{\pgfqpoint{1.000000in}{0.605000in}}{\pgfqpoint{6.200000in}{4.235000in}}%
\pgfusepath{clip}%
\pgfsetbuttcap%
\pgfsetmiterjoin%
\definecolor{currentfill}{rgb}{0.298039,0.447059,0.690196}%
\pgfsetfillcolor{currentfill}%
\pgfsetlinewidth{0.000000pt}%
\definecolor{currentstroke}{rgb}{0.000000,0.000000,0.000000}%
\pgfsetstrokecolor{currentstroke}%
\pgfsetstrokeopacity{0.000000}%
\pgfsetdash{}{0pt}%
\pgfpathmoveto{\pgfqpoint{1.281818in}{0.605000in}}%
\pgfpathlineto{\pgfqpoint{1.845455in}{0.605000in}}%
\pgfpathlineto{\pgfqpoint{1.845455in}{0.733364in}}%
\pgfpathlineto{\pgfqpoint{1.281818in}{0.733364in}}%
\pgfpathlineto{\pgfqpoint{1.281818in}{0.605000in}}%
\pgfpathclose%
\pgfusepath{fill}%
\end{pgfscope}%
\begin{pgfscope}%
\pgfpathrectangle{\pgfqpoint{1.000000in}{0.605000in}}{\pgfqpoint{6.200000in}{4.235000in}}%
\pgfusepath{clip}%
\pgfsetbuttcap%
\pgfsetmiterjoin%
\definecolor{currentfill}{rgb}{0.298039,0.447059,0.690196}%
\pgfsetfillcolor{currentfill}%
\pgfsetlinewidth{0.000000pt}%
\definecolor{currentstroke}{rgb}{0.000000,0.000000,0.000000}%
\pgfsetstrokecolor{currentstroke}%
\pgfsetstrokeopacity{0.000000}%
\pgfsetdash{}{0pt}%
\pgfpathmoveto{\pgfqpoint{1.845455in}{0.605000in}}%
\pgfpathlineto{\pgfqpoint{2.409091in}{0.605000in}}%
\pgfpathlineto{\pgfqpoint{2.409091in}{0.605000in}}%
\pgfpathlineto{\pgfqpoint{1.845455in}{0.605000in}}%
\pgfpathlineto{\pgfqpoint{1.845455in}{0.605000in}}%
\pgfpathclose%
\pgfusepath{fill}%
\end{pgfscope}%
\begin{pgfscope}%
\pgfpathrectangle{\pgfqpoint{1.000000in}{0.605000in}}{\pgfqpoint{6.200000in}{4.235000in}}%
\pgfusepath{clip}%
\pgfsetbuttcap%
\pgfsetmiterjoin%
\definecolor{currentfill}{rgb}{0.298039,0.447059,0.690196}%
\pgfsetfillcolor{currentfill}%
\pgfsetlinewidth{0.000000pt}%
\definecolor{currentstroke}{rgb}{0.000000,0.000000,0.000000}%
\pgfsetstrokecolor{currentstroke}%
\pgfsetstrokeopacity{0.000000}%
\pgfsetdash{}{0pt}%
\pgfpathmoveto{\pgfqpoint{2.409091in}{0.605000in}}%
\pgfpathlineto{\pgfqpoint{2.972727in}{0.605000in}}%
\pgfpathlineto{\pgfqpoint{2.972727in}{0.605000in}}%
\pgfpathlineto{\pgfqpoint{2.409091in}{0.605000in}}%
\pgfpathlineto{\pgfqpoint{2.409091in}{0.605000in}}%
\pgfpathclose%
\pgfusepath{fill}%
\end{pgfscope}%
\begin{pgfscope}%
\pgfpathrectangle{\pgfqpoint{1.000000in}{0.605000in}}{\pgfqpoint{6.200000in}{4.235000in}}%
\pgfusepath{clip}%
\pgfsetbuttcap%
\pgfsetmiterjoin%
\definecolor{currentfill}{rgb}{0.298039,0.447059,0.690196}%
\pgfsetfillcolor{currentfill}%
\pgfsetlinewidth{0.000000pt}%
\definecolor{currentstroke}{rgb}{0.000000,0.000000,0.000000}%
\pgfsetstrokecolor{currentstroke}%
\pgfsetstrokeopacity{0.000000}%
\pgfsetdash{}{0pt}%
\pgfpathmoveto{\pgfqpoint{2.972727in}{0.605000in}}%
\pgfpathlineto{\pgfqpoint{3.536364in}{0.605000in}}%
\pgfpathlineto{\pgfqpoint{3.536364in}{0.605000in}}%
\pgfpathlineto{\pgfqpoint{2.972727in}{0.605000in}}%
\pgfpathlineto{\pgfqpoint{2.972727in}{0.605000in}}%
\pgfpathclose%
\pgfusepath{fill}%
\end{pgfscope}%
\begin{pgfscope}%
\pgfpathrectangle{\pgfqpoint{1.000000in}{0.605000in}}{\pgfqpoint{6.200000in}{4.235000in}}%
\pgfusepath{clip}%
\pgfsetbuttcap%
\pgfsetmiterjoin%
\definecolor{currentfill}{rgb}{0.298039,0.447059,0.690196}%
\pgfsetfillcolor{currentfill}%
\pgfsetlinewidth{0.000000pt}%
\definecolor{currentstroke}{rgb}{0.000000,0.000000,0.000000}%
\pgfsetstrokecolor{currentstroke}%
\pgfsetstrokeopacity{0.000000}%
\pgfsetdash{}{0pt}%
\pgfpathmoveto{\pgfqpoint{3.536364in}{0.605000in}}%
\pgfpathlineto{\pgfqpoint{4.100000in}{0.605000in}}%
\pgfpathlineto{\pgfqpoint{4.100000in}{0.605000in}}%
\pgfpathlineto{\pgfqpoint{3.536364in}{0.605000in}}%
\pgfpathlineto{\pgfqpoint{3.536364in}{0.605000in}}%
\pgfpathclose%
\pgfusepath{fill}%
\end{pgfscope}%
\begin{pgfscope}%
\pgfpathrectangle{\pgfqpoint{1.000000in}{0.605000in}}{\pgfqpoint{6.200000in}{4.235000in}}%
\pgfusepath{clip}%
\pgfsetbuttcap%
\pgfsetmiterjoin%
\definecolor{currentfill}{rgb}{0.298039,0.447059,0.690196}%
\pgfsetfillcolor{currentfill}%
\pgfsetlinewidth{0.000000pt}%
\definecolor{currentstroke}{rgb}{0.000000,0.000000,0.000000}%
\pgfsetstrokecolor{currentstroke}%
\pgfsetstrokeopacity{0.000000}%
\pgfsetdash{}{0pt}%
\pgfpathmoveto{\pgfqpoint{4.100000in}{0.605000in}}%
\pgfpathlineto{\pgfqpoint{4.663636in}{0.605000in}}%
\pgfpathlineto{\pgfqpoint{4.663636in}{0.605000in}}%
\pgfpathlineto{\pgfqpoint{4.100000in}{0.605000in}}%
\pgfpathlineto{\pgfqpoint{4.100000in}{0.605000in}}%
\pgfpathclose%
\pgfusepath{fill}%
\end{pgfscope}%
\begin{pgfscope}%
\pgfpathrectangle{\pgfqpoint{1.000000in}{0.605000in}}{\pgfqpoint{6.200000in}{4.235000in}}%
\pgfusepath{clip}%
\pgfsetbuttcap%
\pgfsetmiterjoin%
\definecolor{currentfill}{rgb}{0.298039,0.447059,0.690196}%
\pgfsetfillcolor{currentfill}%
\pgfsetlinewidth{0.000000pt}%
\definecolor{currentstroke}{rgb}{0.000000,0.000000,0.000000}%
\pgfsetstrokecolor{currentstroke}%
\pgfsetstrokeopacity{0.000000}%
\pgfsetdash{}{0pt}%
\pgfpathmoveto{\pgfqpoint{4.663636in}{0.605000in}}%
\pgfpathlineto{\pgfqpoint{5.227273in}{0.605000in}}%
\pgfpathlineto{\pgfqpoint{5.227273in}{0.841460in}}%
\pgfpathlineto{\pgfqpoint{4.663636in}{0.841460in}}%
\pgfpathlineto{\pgfqpoint{4.663636in}{0.605000in}}%
\pgfpathclose%
\pgfusepath{fill}%
\end{pgfscope}%
\begin{pgfscope}%
\pgfpathrectangle{\pgfqpoint{1.000000in}{0.605000in}}{\pgfqpoint{6.200000in}{4.235000in}}%
\pgfusepath{clip}%
\pgfsetbuttcap%
\pgfsetmiterjoin%
\definecolor{currentfill}{rgb}{0.298039,0.447059,0.690196}%
\pgfsetfillcolor{currentfill}%
\pgfsetlinewidth{0.000000pt}%
\definecolor{currentstroke}{rgb}{0.000000,0.000000,0.000000}%
\pgfsetstrokecolor{currentstroke}%
\pgfsetstrokeopacity{0.000000}%
\pgfsetdash{}{0pt}%
\pgfpathmoveto{\pgfqpoint{5.227273in}{0.605000in}}%
\pgfpathlineto{\pgfqpoint{5.790909in}{0.605000in}}%
\pgfpathlineto{\pgfqpoint{5.790909in}{1.017116in}}%
\pgfpathlineto{\pgfqpoint{5.227273in}{1.017116in}}%
\pgfpathlineto{\pgfqpoint{5.227273in}{0.605000in}}%
\pgfpathclose%
\pgfusepath{fill}%
\end{pgfscope}%
\begin{pgfscope}%
\pgfpathrectangle{\pgfqpoint{1.000000in}{0.605000in}}{\pgfqpoint{6.200000in}{4.235000in}}%
\pgfusepath{clip}%
\pgfsetbuttcap%
\pgfsetmiterjoin%
\definecolor{currentfill}{rgb}{0.298039,0.447059,0.690196}%
\pgfsetfillcolor{currentfill}%
\pgfsetlinewidth{0.000000pt}%
\definecolor{currentstroke}{rgb}{0.000000,0.000000,0.000000}%
\pgfsetstrokecolor{currentstroke}%
\pgfsetstrokeopacity{0.000000}%
\pgfsetdash{}{0pt}%
\pgfpathmoveto{\pgfqpoint{5.790909in}{0.605000in}}%
\pgfpathlineto{\pgfqpoint{6.354545in}{0.605000in}}%
\pgfpathlineto{\pgfqpoint{6.354545in}{4.638333in}}%
\pgfpathlineto{\pgfqpoint{5.790909in}{4.638333in}}%
\pgfpathlineto{\pgfqpoint{5.790909in}{0.605000in}}%
\pgfpathclose%
\pgfusepath{fill}%
\end{pgfscope}%
\begin{pgfscope}%
\pgfpathrectangle{\pgfqpoint{1.000000in}{0.605000in}}{\pgfqpoint{6.200000in}{4.235000in}}%
\pgfusepath{clip}%
\pgfsetbuttcap%
\pgfsetmiterjoin%
\definecolor{currentfill}{rgb}{0.298039,0.447059,0.690196}%
\pgfsetfillcolor{currentfill}%
\pgfsetlinewidth{0.000000pt}%
\definecolor{currentstroke}{rgb}{0.000000,0.000000,0.000000}%
\pgfsetstrokecolor{currentstroke}%
\pgfsetstrokeopacity{0.000000}%
\pgfsetdash{}{0pt}%
\pgfpathmoveto{\pgfqpoint{6.354545in}{0.605000in}}%
\pgfpathlineto{\pgfqpoint{6.918182in}{0.605000in}}%
\pgfpathlineto{\pgfqpoint{6.918182in}{2.557485in}}%
\pgfpathlineto{\pgfqpoint{6.354545in}{2.557485in}}%
\pgfpathlineto{\pgfqpoint{6.354545in}{0.605000in}}%
\pgfpathclose%
\pgfusepath{fill}%
\end{pgfscope}%
\begin{pgfscope}%
\pgfsetrectcap%
\pgfsetmiterjoin%
\pgfsetlinewidth{0.000000pt}%
\definecolor{currentstroke}{rgb}{1.000000,1.000000,1.000000}%
\pgfsetstrokecolor{currentstroke}%
\pgfsetdash{}{0pt}%
\pgfpathmoveto{\pgfqpoint{1.000000in}{0.605000in}}%
\pgfpathlineto{\pgfqpoint{1.000000in}{4.840000in}}%
\pgfusepath{}%
\end{pgfscope}%
\begin{pgfscope}%
\pgfsetrectcap%
\pgfsetmiterjoin%
\pgfsetlinewidth{0.000000pt}%
\definecolor{currentstroke}{rgb}{1.000000,1.000000,1.000000}%
\pgfsetstrokecolor{currentstroke}%
\pgfsetdash{}{0pt}%
\pgfpathmoveto{\pgfqpoint{7.200000in}{0.605000in}}%
\pgfpathlineto{\pgfqpoint{7.200000in}{4.840000in}}%
\pgfusepath{}%
\end{pgfscope}%
\begin{pgfscope}%
\pgfsetrectcap%
\pgfsetmiterjoin%
\pgfsetlinewidth{0.000000pt}%
\definecolor{currentstroke}{rgb}{1.000000,1.000000,1.000000}%
\pgfsetstrokecolor{currentstroke}%
\pgfsetdash{}{0pt}%
\pgfpathmoveto{\pgfqpoint{1.000000in}{0.605000in}}%
\pgfpathlineto{\pgfqpoint{7.200000in}{0.605000in}}%
\pgfusepath{}%
\end{pgfscope}%
\begin{pgfscope}%
\pgfsetrectcap%
\pgfsetmiterjoin%
\pgfsetlinewidth{0.000000pt}%
\definecolor{currentstroke}{rgb}{1.000000,1.000000,1.000000}%
\pgfsetstrokecolor{currentstroke}%
\pgfsetdash{}{0pt}%
\pgfpathmoveto{\pgfqpoint{1.000000in}{4.840000in}}%
\pgfpathlineto{\pgfqpoint{7.200000in}{4.840000in}}%
\pgfusepath{}%
\end{pgfscope}%
\begin{pgfscope}%
\definecolor{textcolor}{rgb}{0.150000,0.150000,0.150000}%
\pgfsetstrokecolor{textcolor}%
\pgfsetfillcolor{textcolor}%
\pgftext[x=4.100000in,y=4.923333in,,base]{\color{textcolor}\rmfamily\fontsize{12.000000}{14.400000}\selectfont Histograma con propuesta beta y \(\displaystyle n=50, r=28\)}%
\end{pgfscope}%
\end{pgfpicture}%
\makeatother%
\endgroup%

    \end{center}

    \item Argumentar porque la cadena es $f$-irreducible y porque es ergódica. 
    Implementar el algoritmo con los datos descritos y discutir los resultados.

    Por diseño del kernel de transición de nuestra cadena, este satisface las ecuaciones
    de balance detallado con $f(p\mid \bar x)$, de donde se sigue que $f$ es estacionaria
    para $K$. Por lo tanto, basta mostrar que $K$ es $f(p\mid \bar x)$-irreducible para
    gerantizar que la cadena es Harris-estacionaria, y ergódica.

    Mostramos a continuación que el kernel construido efectivamente es $f(p\mid \bar x)$-irreducuble.

    \begin{proof}
        Denotemos por $\mu_f$ a la medida de probabilidad asociada a $f(p\mid \bar x)$, y sea
        $A$ un conjunto medible tal que $\mu_f(A)>0$. 
        Notemos que el soporte de $f$ es un subconjunto del soporte de $\cos(\pi p)\mathds 1_{[0,1/2]}(p)$ 
        y el soporte de $p^r(1-p)^{n-r}$. Entonces el soporte de $f$ es un subconjunto del intervalo 
        $(0,1/2)$, y se sigue que $A \subset (0,1/2)$.

        Sea $a \in A$, entonces $K(x,a) = q(a\mid x) \frac{ f(y) q(a \mid y) }{f(a) q(y\mid a)} + (1 - r(a))\delta_{a}(y)$.
        Como la densidad beta es positiva en $(0,1)$, en particular es positiva en $(0,1/2)$. Además, si el
        punto de transición propuesto está en el soporte de $f$, entonces existe posibilidad positiva de aceptar
        la transición, luego, $K(x,a) > 0$ para $a\in A$. Es decir, con la densidad escogida, el kernel
        de transición en un pasoes positivo sobre el soporte de $A$.

        % Para $K(x,A)$ tenemos

        % \begin{align*}
        %     K(x,A) &= q(A\mid x) \frac{ f(y) q(A \mid y) }{\mu_f(A) q(y\mid A)} + (1 - r(A))\mathds 1_{A}(y),\\
        %     %&= \int_A B(r+1,n-r+1) p^r(1-p)^{n-r} dp \frac{p^r(1-p)^{n-r} \cos (\pi p) \mathds 1_{(0,1/2)} (p) }
        %     %{\mu_f(A) p^r(1-p)^{n-r}}.
        % \end{align*}

        % En la última igualdad el término $(1-r(A))\mathds1_A (y)$ no aparece, pues
    \end{proof}

    Antes de implementar el algoritmo de Metropolis-Hastings para simular, graficamos las densidades
    objetivo con los valores que tenemos de $n$ y $r$.

    \begin{center}
        %% Creator: Matplotlib, PGF backend
%%
%% To include the figure in your LaTeX document, write
%%   \input{<filename>.pgf}
%%
%% Make sure the required packages are loaded in your preamble
%%   \usepackage{pgf}
%%
%% Also ensure that all the required font packages are loaded; for instance,
%% the lmodern package is sometimes necessary when using math font.
%%   \usepackage{lmodern}
%%
%% Figures using additional raster images can only be included by \input if
%% they are in the same directory as the main LaTeX file. For loading figures
%% from other directories you can use the `import` package
%%   \usepackage{import}
%%
%% and then include the figures with
%%   \import{<path to file>}{<filename>.pgf}
%%
%% Matplotlib used the following preamble
%%   
%%   \makeatletter\@ifpackageloaded{underscore}{}{\usepackage[strings]{underscore}}\makeatother
%%
\begingroup%
\makeatletter%
\begin{pgfpicture}%
\pgfpathrectangle{\pgfpointorigin}{\pgfqpoint{7.000000in}{4.000000in}}%
\pgfusepath{use as bounding box, clip}%
\begin{pgfscope}%
\pgfsetbuttcap%
\pgfsetmiterjoin%
\definecolor{currentfill}{rgb}{1.000000,1.000000,1.000000}%
\pgfsetfillcolor{currentfill}%
\pgfsetlinewidth{0.000000pt}%
\definecolor{currentstroke}{rgb}{1.000000,1.000000,1.000000}%
\pgfsetstrokecolor{currentstroke}%
\pgfsetdash{}{0pt}%
\pgfpathmoveto{\pgfqpoint{0.000000in}{0.000000in}}%
\pgfpathlineto{\pgfqpoint{7.000000in}{0.000000in}}%
\pgfpathlineto{\pgfqpoint{7.000000in}{4.000000in}}%
\pgfpathlineto{\pgfqpoint{0.000000in}{4.000000in}}%
\pgfpathlineto{\pgfqpoint{0.000000in}{0.000000in}}%
\pgfpathclose%
\pgfusepath{fill}%
\end{pgfscope}%
\begin{pgfscope}%
\pgfsetbuttcap%
\pgfsetmiterjoin%
\definecolor{currentfill}{rgb}{0.917647,0.917647,0.949020}%
\pgfsetfillcolor{currentfill}%
\pgfsetlinewidth{0.000000pt}%
\definecolor{currentstroke}{rgb}{0.000000,0.000000,0.000000}%
\pgfsetstrokecolor{currentstroke}%
\pgfsetstrokeopacity{0.000000}%
\pgfsetdash{}{0pt}%
\pgfpathmoveto{\pgfqpoint{0.875000in}{0.440000in}}%
\pgfpathlineto{\pgfqpoint{6.300000in}{0.440000in}}%
\pgfpathlineto{\pgfqpoint{6.300000in}{3.520000in}}%
\pgfpathlineto{\pgfqpoint{0.875000in}{3.520000in}}%
\pgfpathlineto{\pgfqpoint{0.875000in}{0.440000in}}%
\pgfpathclose%
\pgfusepath{fill}%
\end{pgfscope}%
\begin{pgfscope}%
\pgfpathrectangle{\pgfqpoint{0.875000in}{0.440000in}}{\pgfqpoint{5.425000in}{3.080000in}}%
\pgfusepath{clip}%
\pgfsetroundcap%
\pgfsetroundjoin%
\pgfsetlinewidth{1.003750pt}%
\definecolor{currentstroke}{rgb}{1.000000,1.000000,1.000000}%
\pgfsetstrokecolor{currentstroke}%
\pgfsetdash{}{0pt}%
\pgfpathmoveto{\pgfqpoint{1.121591in}{0.440000in}}%
\pgfpathlineto{\pgfqpoint{1.121591in}{3.520000in}}%
\pgfusepath{stroke}%
\end{pgfscope}%
\begin{pgfscope}%
\definecolor{textcolor}{rgb}{0.150000,0.150000,0.150000}%
\pgfsetstrokecolor{textcolor}%
\pgfsetfillcolor{textcolor}%
\pgftext[x=1.121591in,y=0.342778in,,top]{\color{textcolor}\rmfamily\fontsize{10.000000}{12.000000}\selectfont \(\displaystyle {0.0}\)}%
\end{pgfscope}%
\begin{pgfscope}%
\pgfpathrectangle{\pgfqpoint{0.875000in}{0.440000in}}{\pgfqpoint{5.425000in}{3.080000in}}%
\pgfusepath{clip}%
\pgfsetroundcap%
\pgfsetroundjoin%
\pgfsetlinewidth{1.003750pt}%
\definecolor{currentstroke}{rgb}{1.000000,1.000000,1.000000}%
\pgfsetstrokecolor{currentstroke}%
\pgfsetdash{}{0pt}%
\pgfpathmoveto{\pgfqpoint{2.107955in}{0.440000in}}%
\pgfpathlineto{\pgfqpoint{2.107955in}{3.520000in}}%
\pgfusepath{stroke}%
\end{pgfscope}%
\begin{pgfscope}%
\definecolor{textcolor}{rgb}{0.150000,0.150000,0.150000}%
\pgfsetstrokecolor{textcolor}%
\pgfsetfillcolor{textcolor}%
\pgftext[x=2.107955in,y=0.342778in,,top]{\color{textcolor}\rmfamily\fontsize{10.000000}{12.000000}\selectfont \(\displaystyle {0.1}\)}%
\end{pgfscope}%
\begin{pgfscope}%
\pgfpathrectangle{\pgfqpoint{0.875000in}{0.440000in}}{\pgfqpoint{5.425000in}{3.080000in}}%
\pgfusepath{clip}%
\pgfsetroundcap%
\pgfsetroundjoin%
\pgfsetlinewidth{1.003750pt}%
\definecolor{currentstroke}{rgb}{1.000000,1.000000,1.000000}%
\pgfsetstrokecolor{currentstroke}%
\pgfsetdash{}{0pt}%
\pgfpathmoveto{\pgfqpoint{3.094318in}{0.440000in}}%
\pgfpathlineto{\pgfqpoint{3.094318in}{3.520000in}}%
\pgfusepath{stroke}%
\end{pgfscope}%
\begin{pgfscope}%
\definecolor{textcolor}{rgb}{0.150000,0.150000,0.150000}%
\pgfsetstrokecolor{textcolor}%
\pgfsetfillcolor{textcolor}%
\pgftext[x=3.094318in,y=0.342778in,,top]{\color{textcolor}\rmfamily\fontsize{10.000000}{12.000000}\selectfont \(\displaystyle {0.2}\)}%
\end{pgfscope}%
\begin{pgfscope}%
\pgfpathrectangle{\pgfqpoint{0.875000in}{0.440000in}}{\pgfqpoint{5.425000in}{3.080000in}}%
\pgfusepath{clip}%
\pgfsetroundcap%
\pgfsetroundjoin%
\pgfsetlinewidth{1.003750pt}%
\definecolor{currentstroke}{rgb}{1.000000,1.000000,1.000000}%
\pgfsetstrokecolor{currentstroke}%
\pgfsetdash{}{0pt}%
\pgfpathmoveto{\pgfqpoint{4.080682in}{0.440000in}}%
\pgfpathlineto{\pgfqpoint{4.080682in}{3.520000in}}%
\pgfusepath{stroke}%
\end{pgfscope}%
\begin{pgfscope}%
\definecolor{textcolor}{rgb}{0.150000,0.150000,0.150000}%
\pgfsetstrokecolor{textcolor}%
\pgfsetfillcolor{textcolor}%
\pgftext[x=4.080682in,y=0.342778in,,top]{\color{textcolor}\rmfamily\fontsize{10.000000}{12.000000}\selectfont \(\displaystyle {0.3}\)}%
\end{pgfscope}%
\begin{pgfscope}%
\pgfpathrectangle{\pgfqpoint{0.875000in}{0.440000in}}{\pgfqpoint{5.425000in}{3.080000in}}%
\pgfusepath{clip}%
\pgfsetroundcap%
\pgfsetroundjoin%
\pgfsetlinewidth{1.003750pt}%
\definecolor{currentstroke}{rgb}{1.000000,1.000000,1.000000}%
\pgfsetstrokecolor{currentstroke}%
\pgfsetdash{}{0pt}%
\pgfpathmoveto{\pgfqpoint{5.067045in}{0.440000in}}%
\pgfpathlineto{\pgfqpoint{5.067045in}{3.520000in}}%
\pgfusepath{stroke}%
\end{pgfscope}%
\begin{pgfscope}%
\definecolor{textcolor}{rgb}{0.150000,0.150000,0.150000}%
\pgfsetstrokecolor{textcolor}%
\pgfsetfillcolor{textcolor}%
\pgftext[x=5.067045in,y=0.342778in,,top]{\color{textcolor}\rmfamily\fontsize{10.000000}{12.000000}\selectfont \(\displaystyle {0.4}\)}%
\end{pgfscope}%
\begin{pgfscope}%
\pgfpathrectangle{\pgfqpoint{0.875000in}{0.440000in}}{\pgfqpoint{5.425000in}{3.080000in}}%
\pgfusepath{clip}%
\pgfsetroundcap%
\pgfsetroundjoin%
\pgfsetlinewidth{1.003750pt}%
\definecolor{currentstroke}{rgb}{1.000000,1.000000,1.000000}%
\pgfsetstrokecolor{currentstroke}%
\pgfsetdash{}{0pt}%
\pgfpathmoveto{\pgfqpoint{6.053409in}{0.440000in}}%
\pgfpathlineto{\pgfqpoint{6.053409in}{3.520000in}}%
\pgfusepath{stroke}%
\end{pgfscope}%
\begin{pgfscope}%
\definecolor{textcolor}{rgb}{0.150000,0.150000,0.150000}%
\pgfsetstrokecolor{textcolor}%
\pgfsetfillcolor{textcolor}%
\pgftext[x=6.053409in,y=0.342778in,,top]{\color{textcolor}\rmfamily\fontsize{10.000000}{12.000000}\selectfont \(\displaystyle {0.5}\)}%
\end{pgfscope}%
\begin{pgfscope}%
\pgfpathrectangle{\pgfqpoint{0.875000in}{0.440000in}}{\pgfqpoint{5.425000in}{3.080000in}}%
\pgfusepath{clip}%
\pgfsetroundcap%
\pgfsetroundjoin%
\pgfsetlinewidth{1.003750pt}%
\definecolor{currentstroke}{rgb}{1.000000,1.000000,1.000000}%
\pgfsetstrokecolor{currentstroke}%
\pgfsetdash{}{0pt}%
\pgfpathmoveto{\pgfqpoint{0.875000in}{0.580000in}}%
\pgfpathlineto{\pgfqpoint{6.300000in}{0.580000in}}%
\pgfusepath{stroke}%
\end{pgfscope}%
\begin{pgfscope}%
\definecolor{textcolor}{rgb}{0.150000,0.150000,0.150000}%
\pgfsetstrokecolor{textcolor}%
\pgfsetfillcolor{textcolor}%
\pgftext[x=0.461419in, y=0.531775in, left, base]{\color{textcolor}\rmfamily\fontsize{10.000000}{12.000000}\selectfont \(\displaystyle {0.000}\)}%
\end{pgfscope}%
\begin{pgfscope}%
\pgfpathrectangle{\pgfqpoint{0.875000in}{0.440000in}}{\pgfqpoint{5.425000in}{3.080000in}}%
\pgfusepath{clip}%
\pgfsetroundcap%
\pgfsetroundjoin%
\pgfsetlinewidth{1.003750pt}%
\definecolor{currentstroke}{rgb}{1.000000,1.000000,1.000000}%
\pgfsetstrokecolor{currentstroke}%
\pgfsetdash{}{0pt}%
\pgfpathmoveto{\pgfqpoint{0.875000in}{1.259062in}}%
\pgfpathlineto{\pgfqpoint{6.300000in}{1.259062in}}%
\pgfusepath{stroke}%
\end{pgfscope}%
\begin{pgfscope}%
\definecolor{textcolor}{rgb}{0.150000,0.150000,0.150000}%
\pgfsetstrokecolor{textcolor}%
\pgfsetfillcolor{textcolor}%
\pgftext[x=0.461419in, y=1.210836in, left, base]{\color{textcolor}\rmfamily\fontsize{10.000000}{12.000000}\selectfont \(\displaystyle {0.002}\)}%
\end{pgfscope}%
\begin{pgfscope}%
\pgfpathrectangle{\pgfqpoint{0.875000in}{0.440000in}}{\pgfqpoint{5.425000in}{3.080000in}}%
\pgfusepath{clip}%
\pgfsetroundcap%
\pgfsetroundjoin%
\pgfsetlinewidth{1.003750pt}%
\definecolor{currentstroke}{rgb}{1.000000,1.000000,1.000000}%
\pgfsetstrokecolor{currentstroke}%
\pgfsetdash{}{0pt}%
\pgfpathmoveto{\pgfqpoint{0.875000in}{1.938123in}}%
\pgfpathlineto{\pgfqpoint{6.300000in}{1.938123in}}%
\pgfusepath{stroke}%
\end{pgfscope}%
\begin{pgfscope}%
\definecolor{textcolor}{rgb}{0.150000,0.150000,0.150000}%
\pgfsetstrokecolor{textcolor}%
\pgfsetfillcolor{textcolor}%
\pgftext[x=0.461419in, y=1.889898in, left, base]{\color{textcolor}\rmfamily\fontsize{10.000000}{12.000000}\selectfont \(\displaystyle {0.004}\)}%
\end{pgfscope}%
\begin{pgfscope}%
\pgfpathrectangle{\pgfqpoint{0.875000in}{0.440000in}}{\pgfqpoint{5.425000in}{3.080000in}}%
\pgfusepath{clip}%
\pgfsetroundcap%
\pgfsetroundjoin%
\pgfsetlinewidth{1.003750pt}%
\definecolor{currentstroke}{rgb}{1.000000,1.000000,1.000000}%
\pgfsetstrokecolor{currentstroke}%
\pgfsetdash{}{0pt}%
\pgfpathmoveto{\pgfqpoint{0.875000in}{2.617185in}}%
\pgfpathlineto{\pgfqpoint{6.300000in}{2.617185in}}%
\pgfusepath{stroke}%
\end{pgfscope}%
\begin{pgfscope}%
\definecolor{textcolor}{rgb}{0.150000,0.150000,0.150000}%
\pgfsetstrokecolor{textcolor}%
\pgfsetfillcolor{textcolor}%
\pgftext[x=0.461419in, y=2.568960in, left, base]{\color{textcolor}\rmfamily\fontsize{10.000000}{12.000000}\selectfont \(\displaystyle {0.006}\)}%
\end{pgfscope}%
\begin{pgfscope}%
\pgfpathrectangle{\pgfqpoint{0.875000in}{0.440000in}}{\pgfqpoint{5.425000in}{3.080000in}}%
\pgfusepath{clip}%
\pgfsetroundcap%
\pgfsetroundjoin%
\pgfsetlinewidth{1.003750pt}%
\definecolor{currentstroke}{rgb}{1.000000,1.000000,1.000000}%
\pgfsetstrokecolor{currentstroke}%
\pgfsetdash{}{0pt}%
\pgfpathmoveto{\pgfqpoint{0.875000in}{3.296247in}}%
\pgfpathlineto{\pgfqpoint{6.300000in}{3.296247in}}%
\pgfusepath{stroke}%
\end{pgfscope}%
\begin{pgfscope}%
\definecolor{textcolor}{rgb}{0.150000,0.150000,0.150000}%
\pgfsetstrokecolor{textcolor}%
\pgfsetfillcolor{textcolor}%
\pgftext[x=0.461419in, y=3.248021in, left, base]{\color{textcolor}\rmfamily\fontsize{10.000000}{12.000000}\selectfont \(\displaystyle {0.008}\)}%
\end{pgfscope}%
\begin{pgfscope}%
\pgfpathrectangle{\pgfqpoint{0.875000in}{0.440000in}}{\pgfqpoint{5.425000in}{3.080000in}}%
\pgfusepath{clip}%
\pgfsetroundcap%
\pgfsetroundjoin%
\pgfsetlinewidth{1.756562pt}%
\definecolor{currentstroke}{rgb}{0.298039,0.447059,0.690196}%
\pgfsetstrokecolor{currentstroke}%
\pgfsetdash{}{0pt}%
\pgfpathmoveto{\pgfqpoint{1.121591in}{0.580000in}}%
\pgfpathlineto{\pgfqpoint{1.171407in}{0.580043in}}%
\pgfpathlineto{\pgfqpoint{1.221224in}{0.580343in}}%
\pgfpathlineto{\pgfqpoint{1.271040in}{0.581144in}}%
\pgfpathlineto{\pgfqpoint{1.320856in}{0.582682in}}%
\pgfpathlineto{\pgfqpoint{1.370673in}{0.585179in}}%
\pgfpathlineto{\pgfqpoint{1.420489in}{0.588844in}}%
\pgfpathlineto{\pgfqpoint{1.470305in}{0.593875in}}%
\pgfpathlineto{\pgfqpoint{1.520122in}{0.600456in}}%
\pgfpathlineto{\pgfqpoint{1.569938in}{0.608758in}}%
\pgfpathlineto{\pgfqpoint{1.619754in}{0.618938in}}%
\pgfpathlineto{\pgfqpoint{1.669571in}{0.631141in}}%
\pgfpathlineto{\pgfqpoint{1.719387in}{0.645494in}}%
\pgfpathlineto{\pgfqpoint{1.769203in}{0.662115in}}%
\pgfpathlineto{\pgfqpoint{1.819020in}{0.681104in}}%
\pgfpathlineto{\pgfqpoint{1.868836in}{0.702549in}}%
\pgfpathlineto{\pgfqpoint{1.918652in}{0.726524in}}%
\pgfpathlineto{\pgfqpoint{1.968469in}{0.753086in}}%
\pgfpathlineto{\pgfqpoint{2.018285in}{0.782282in}}%
\pgfpathlineto{\pgfqpoint{2.068101in}{0.814142in}}%
\pgfpathlineto{\pgfqpoint{2.117918in}{0.848682in}}%
\pgfpathlineto{\pgfqpoint{2.167734in}{0.885907in}}%
\pgfpathlineto{\pgfqpoint{2.217551in}{0.925806in}}%
\pgfpathlineto{\pgfqpoint{2.267367in}{0.968355in}}%
\pgfpathlineto{\pgfqpoint{2.317183in}{1.013516in}}%
\pgfpathlineto{\pgfqpoint{2.367000in}{1.061242in}}%
\pgfpathlineto{\pgfqpoint{2.416816in}{1.111468in}}%
\pgfpathlineto{\pgfqpoint{2.466632in}{1.164121in}}%
\pgfpathlineto{\pgfqpoint{2.516449in}{1.219113in}}%
\pgfpathlineto{\pgfqpoint{2.566265in}{1.276347in}}%
\pgfpathlineto{\pgfqpoint{2.616081in}{1.335713in}}%
\pgfpathlineto{\pgfqpoint{2.665898in}{1.397091in}}%
\pgfpathlineto{\pgfqpoint{2.715714in}{1.460351in}}%
\pgfpathlineto{\pgfqpoint{2.765530in}{1.525352in}}%
\pgfpathlineto{\pgfqpoint{2.815347in}{1.591944in}}%
\pgfpathlineto{\pgfqpoint{2.865163in}{1.659969in}}%
\pgfpathlineto{\pgfqpoint{2.914979in}{1.729258in}}%
\pgfpathlineto{\pgfqpoint{2.964796in}{1.799638in}}%
\pgfpathlineto{\pgfqpoint{3.014612in}{1.870924in}}%
\pgfpathlineto{\pgfqpoint{3.064428in}{1.942928in}}%
\pgfpathlineto{\pgfqpoint{3.114245in}{2.015453in}}%
\pgfpathlineto{\pgfqpoint{3.164061in}{2.088298in}}%
\pgfpathlineto{\pgfqpoint{3.213877in}{2.161256in}}%
\pgfpathlineto{\pgfqpoint{3.263694in}{2.234115in}}%
\pgfpathlineto{\pgfqpoint{3.313510in}{2.306659in}}%
\pgfpathlineto{\pgfqpoint{3.363326in}{2.378670in}}%
\pgfpathlineto{\pgfqpoint{3.413143in}{2.449926in}}%
\pgfpathlineto{\pgfqpoint{3.462959in}{2.520201in}}%
\pgfpathlineto{\pgfqpoint{3.512775in}{2.589271in}}%
\pgfpathlineto{\pgfqpoint{3.562592in}{2.656908in}}%
\pgfpathlineto{\pgfqpoint{3.612408in}{2.722885in}}%
\pgfpathlineto{\pgfqpoint{3.662225in}{2.786974in}}%
\pgfpathlineto{\pgfqpoint{3.712041in}{2.848950in}}%
\pgfpathlineto{\pgfqpoint{3.761857in}{2.908586in}}%
\pgfpathlineto{\pgfqpoint{3.811674in}{2.965659in}}%
\pgfpathlineto{\pgfqpoint{3.861490in}{3.019950in}}%
\pgfpathlineto{\pgfqpoint{3.911306in}{3.071239in}}%
\pgfpathlineto{\pgfqpoint{3.961123in}{3.119313in}}%
\pgfpathlineto{\pgfqpoint{4.010939in}{3.163962in}}%
\pgfpathlineto{\pgfqpoint{4.060755in}{3.204981in}}%
\pgfpathlineto{\pgfqpoint{4.110572in}{3.242170in}}%
\pgfpathlineto{\pgfqpoint{4.160388in}{3.275335in}}%
\pgfpathlineto{\pgfqpoint{4.210204in}{3.304289in}}%
\pgfpathlineto{\pgfqpoint{4.260021in}{3.328851in}}%
\pgfpathlineto{\pgfqpoint{4.309837in}{3.348848in}}%
\pgfpathlineto{\pgfqpoint{4.359653in}{3.364114in}}%
\pgfpathlineto{\pgfqpoint{4.409470in}{3.374492in}}%
\pgfpathlineto{\pgfqpoint{4.459286in}{3.379834in}}%
\pgfpathlineto{\pgfqpoint{4.509102in}{3.380000in}}%
\pgfpathlineto{\pgfqpoint{4.558919in}{3.374861in}}%
\pgfpathlineto{\pgfqpoint{4.608735in}{3.364296in}}%
\pgfpathlineto{\pgfqpoint{4.658551in}{3.348197in}}%
\pgfpathlineto{\pgfqpoint{4.708368in}{3.326465in}}%
\pgfpathlineto{\pgfqpoint{4.758184in}{3.299011in}}%
\pgfpathlineto{\pgfqpoint{4.808000in}{3.265760in}}%
\pgfpathlineto{\pgfqpoint{4.857817in}{3.226647in}}%
\pgfpathlineto{\pgfqpoint{4.907633in}{3.181618in}}%
\pgfpathlineto{\pgfqpoint{4.957449in}{3.130632in}}%
\pgfpathlineto{\pgfqpoint{5.007266in}{3.073662in}}%
\pgfpathlineto{\pgfqpoint{5.057082in}{3.010690in}}%
\pgfpathlineto{\pgfqpoint{5.106899in}{2.941714in}}%
\pgfpathlineto{\pgfqpoint{5.156715in}{2.866741in}}%
\pgfpathlineto{\pgfqpoint{5.206531in}{2.785796in}}%
\pgfpathlineto{\pgfqpoint{5.256348in}{2.698912in}}%
\pgfpathlineto{\pgfqpoint{5.306164in}{2.606138in}}%
\pgfpathlineto{\pgfqpoint{5.355980in}{2.507535in}}%
\pgfpathlineto{\pgfqpoint{5.405797in}{2.403177in}}%
\pgfpathlineto{\pgfqpoint{5.455613in}{2.293151in}}%
\pgfpathlineto{\pgfqpoint{5.505429in}{2.177559in}}%
\pgfpathlineto{\pgfqpoint{5.555246in}{2.056513in}}%
\pgfpathlineto{\pgfqpoint{5.605062in}{1.930139in}}%
\pgfpathlineto{\pgfqpoint{5.654878in}{1.798576in}}%
\pgfpathlineto{\pgfqpoint{5.704695in}{1.661977in}}%
\pgfpathlineto{\pgfqpoint{5.754511in}{1.520504in}}%
\pgfpathlineto{\pgfqpoint{5.804327in}{1.374333in}}%
\pgfpathlineto{\pgfqpoint{5.854144in}{1.223653in}}%
\pgfpathlineto{\pgfqpoint{5.903960in}{1.068663in}}%
\pgfpathlineto{\pgfqpoint{5.953776in}{0.909574in}}%
\pgfpathlineto{\pgfqpoint{6.003593in}{0.746609in}}%
\pgfpathlineto{\pgfqpoint{6.053409in}{0.580000in}}%
\pgfusepath{stroke}%
\end{pgfscope}%
\begin{pgfscope}%
\pgfsetrectcap%
\pgfsetmiterjoin%
\pgfsetlinewidth{0.000000pt}%
\definecolor{currentstroke}{rgb}{1.000000,1.000000,1.000000}%
\pgfsetstrokecolor{currentstroke}%
\pgfsetdash{}{0pt}%
\pgfpathmoveto{\pgfqpoint{0.875000in}{0.440000in}}%
\pgfpathlineto{\pgfqpoint{0.875000in}{3.520000in}}%
\pgfusepath{}%
\end{pgfscope}%
\begin{pgfscope}%
\pgfsetrectcap%
\pgfsetmiterjoin%
\pgfsetlinewidth{0.000000pt}%
\definecolor{currentstroke}{rgb}{1.000000,1.000000,1.000000}%
\pgfsetstrokecolor{currentstroke}%
\pgfsetdash{}{0pt}%
\pgfpathmoveto{\pgfqpoint{6.300000in}{0.440000in}}%
\pgfpathlineto{\pgfqpoint{6.300000in}{3.520000in}}%
\pgfusepath{}%
\end{pgfscope}%
\begin{pgfscope}%
\pgfsetrectcap%
\pgfsetmiterjoin%
\pgfsetlinewidth{0.000000pt}%
\definecolor{currentstroke}{rgb}{1.000000,1.000000,1.000000}%
\pgfsetstrokecolor{currentstroke}%
\pgfsetdash{}{0pt}%
\pgfpathmoveto{\pgfqpoint{0.875000in}{0.440000in}}%
\pgfpathlineto{\pgfqpoint{6.300000in}{0.440000in}}%
\pgfusepath{}%
\end{pgfscope}%
\begin{pgfscope}%
\pgfsetrectcap%
\pgfsetmiterjoin%
\pgfsetlinewidth{0.000000pt}%
\definecolor{currentstroke}{rgb}{1.000000,1.000000,1.000000}%
\pgfsetstrokecolor{currentstroke}%
\pgfsetdash{}{0pt}%
\pgfpathmoveto{\pgfqpoint{0.875000in}{3.520000in}}%
\pgfpathlineto{\pgfqpoint{6.300000in}{3.520000in}}%
\pgfusepath{}%
\end{pgfscope}%
\begin{pgfscope}%
\definecolor{textcolor}{rgb}{0.150000,0.150000,0.150000}%
\pgfsetstrokecolor{textcolor}%
\pgfsetfillcolor{textcolor}%
\pgftext[x=3.500000in,y=3.920000in,,top]{\color{textcolor}\rmfamily\fontsize{12.000000}{14.400000}\selectfont Múltiplo de la densidad objetivo con \(\displaystyle n=5, r=\)3}%
\end{pgfscope}%
\end{pgfpicture}%
\makeatother%
\endgroup%

        %% Creator: Matplotlib, PGF backend
%%
%% To include the figure in your LaTeX document, write
%%   \input{<filename>.pgf}
%%
%% Make sure the required packages are loaded in your preamble
%%   \usepackage{pgf}
%%
%% Also ensure that all the required font packages are loaded; for instance,
%% the lmodern package is sometimes necessary when using math font.
%%   \usepackage{lmodern}
%%
%% Figures using additional raster images can only be included by \input if
%% they are in the same directory as the main LaTeX file. For loading figures
%% from other directories you can use the `import` package
%%   \usepackage{import}
%%
%% and then include the figures with
%%   \import{<path to file>}{<filename>.pgf}
%%
%% Matplotlib used the following preamble
%%   
%%   \makeatletter\@ifpackageloaded{underscore}{}{\usepackage[strings]{underscore}}\makeatother
%%
\begingroup%
\makeatletter%
\begin{pgfpicture}%
\pgfpathrectangle{\pgfpointorigin}{\pgfqpoint{7.000000in}{4.000000in}}%
\pgfusepath{use as bounding box, clip}%
\begin{pgfscope}%
\pgfsetbuttcap%
\pgfsetmiterjoin%
\definecolor{currentfill}{rgb}{1.000000,1.000000,1.000000}%
\pgfsetfillcolor{currentfill}%
\pgfsetlinewidth{0.000000pt}%
\definecolor{currentstroke}{rgb}{1.000000,1.000000,1.000000}%
\pgfsetstrokecolor{currentstroke}%
\pgfsetdash{}{0pt}%
\pgfpathmoveto{\pgfqpoint{0.000000in}{0.000000in}}%
\pgfpathlineto{\pgfqpoint{7.000000in}{0.000000in}}%
\pgfpathlineto{\pgfqpoint{7.000000in}{4.000000in}}%
\pgfpathlineto{\pgfqpoint{0.000000in}{4.000000in}}%
\pgfpathlineto{\pgfqpoint{0.000000in}{0.000000in}}%
\pgfpathclose%
\pgfusepath{fill}%
\end{pgfscope}%
\begin{pgfscope}%
\pgfsetbuttcap%
\pgfsetmiterjoin%
\definecolor{currentfill}{rgb}{0.917647,0.917647,0.949020}%
\pgfsetfillcolor{currentfill}%
\pgfsetlinewidth{0.000000pt}%
\definecolor{currentstroke}{rgb}{0.000000,0.000000,0.000000}%
\pgfsetstrokecolor{currentstroke}%
\pgfsetstrokeopacity{0.000000}%
\pgfsetdash{}{0pt}%
\pgfpathmoveto{\pgfqpoint{0.875000in}{0.440000in}}%
\pgfpathlineto{\pgfqpoint{6.300000in}{0.440000in}}%
\pgfpathlineto{\pgfqpoint{6.300000in}{3.520000in}}%
\pgfpathlineto{\pgfqpoint{0.875000in}{3.520000in}}%
\pgfpathlineto{\pgfqpoint{0.875000in}{0.440000in}}%
\pgfpathclose%
\pgfusepath{fill}%
\end{pgfscope}%
\begin{pgfscope}%
\pgfpathrectangle{\pgfqpoint{0.875000in}{0.440000in}}{\pgfqpoint{5.425000in}{3.080000in}}%
\pgfusepath{clip}%
\pgfsetroundcap%
\pgfsetroundjoin%
\pgfsetlinewidth{1.003750pt}%
\definecolor{currentstroke}{rgb}{1.000000,1.000000,1.000000}%
\pgfsetstrokecolor{currentstroke}%
\pgfsetdash{}{0pt}%
\pgfpathmoveto{\pgfqpoint{1.121591in}{0.440000in}}%
\pgfpathlineto{\pgfqpoint{1.121591in}{3.520000in}}%
\pgfusepath{stroke}%
\end{pgfscope}%
\begin{pgfscope}%
\definecolor{textcolor}{rgb}{0.150000,0.150000,0.150000}%
\pgfsetstrokecolor{textcolor}%
\pgfsetfillcolor{textcolor}%
\pgftext[x=1.121591in,y=0.342778in,,top]{\color{textcolor}\rmfamily\fontsize{10.000000}{12.000000}\selectfont \(\displaystyle {0.0}\)}%
\end{pgfscope}%
\begin{pgfscope}%
\pgfpathrectangle{\pgfqpoint{0.875000in}{0.440000in}}{\pgfqpoint{5.425000in}{3.080000in}}%
\pgfusepath{clip}%
\pgfsetroundcap%
\pgfsetroundjoin%
\pgfsetlinewidth{1.003750pt}%
\definecolor{currentstroke}{rgb}{1.000000,1.000000,1.000000}%
\pgfsetstrokecolor{currentstroke}%
\pgfsetdash{}{0pt}%
\pgfpathmoveto{\pgfqpoint{2.107955in}{0.440000in}}%
\pgfpathlineto{\pgfqpoint{2.107955in}{3.520000in}}%
\pgfusepath{stroke}%
\end{pgfscope}%
\begin{pgfscope}%
\definecolor{textcolor}{rgb}{0.150000,0.150000,0.150000}%
\pgfsetstrokecolor{textcolor}%
\pgfsetfillcolor{textcolor}%
\pgftext[x=2.107955in,y=0.342778in,,top]{\color{textcolor}\rmfamily\fontsize{10.000000}{12.000000}\selectfont \(\displaystyle {0.1}\)}%
\end{pgfscope}%
\begin{pgfscope}%
\pgfpathrectangle{\pgfqpoint{0.875000in}{0.440000in}}{\pgfqpoint{5.425000in}{3.080000in}}%
\pgfusepath{clip}%
\pgfsetroundcap%
\pgfsetroundjoin%
\pgfsetlinewidth{1.003750pt}%
\definecolor{currentstroke}{rgb}{1.000000,1.000000,1.000000}%
\pgfsetstrokecolor{currentstroke}%
\pgfsetdash{}{0pt}%
\pgfpathmoveto{\pgfqpoint{3.094318in}{0.440000in}}%
\pgfpathlineto{\pgfqpoint{3.094318in}{3.520000in}}%
\pgfusepath{stroke}%
\end{pgfscope}%
\begin{pgfscope}%
\definecolor{textcolor}{rgb}{0.150000,0.150000,0.150000}%
\pgfsetstrokecolor{textcolor}%
\pgfsetfillcolor{textcolor}%
\pgftext[x=3.094318in,y=0.342778in,,top]{\color{textcolor}\rmfamily\fontsize{10.000000}{12.000000}\selectfont \(\displaystyle {0.2}\)}%
\end{pgfscope}%
\begin{pgfscope}%
\pgfpathrectangle{\pgfqpoint{0.875000in}{0.440000in}}{\pgfqpoint{5.425000in}{3.080000in}}%
\pgfusepath{clip}%
\pgfsetroundcap%
\pgfsetroundjoin%
\pgfsetlinewidth{1.003750pt}%
\definecolor{currentstroke}{rgb}{1.000000,1.000000,1.000000}%
\pgfsetstrokecolor{currentstroke}%
\pgfsetdash{}{0pt}%
\pgfpathmoveto{\pgfqpoint{4.080682in}{0.440000in}}%
\pgfpathlineto{\pgfqpoint{4.080682in}{3.520000in}}%
\pgfusepath{stroke}%
\end{pgfscope}%
\begin{pgfscope}%
\definecolor{textcolor}{rgb}{0.150000,0.150000,0.150000}%
\pgfsetstrokecolor{textcolor}%
\pgfsetfillcolor{textcolor}%
\pgftext[x=4.080682in,y=0.342778in,,top]{\color{textcolor}\rmfamily\fontsize{10.000000}{12.000000}\selectfont \(\displaystyle {0.3}\)}%
\end{pgfscope}%
\begin{pgfscope}%
\pgfpathrectangle{\pgfqpoint{0.875000in}{0.440000in}}{\pgfqpoint{5.425000in}{3.080000in}}%
\pgfusepath{clip}%
\pgfsetroundcap%
\pgfsetroundjoin%
\pgfsetlinewidth{1.003750pt}%
\definecolor{currentstroke}{rgb}{1.000000,1.000000,1.000000}%
\pgfsetstrokecolor{currentstroke}%
\pgfsetdash{}{0pt}%
\pgfpathmoveto{\pgfqpoint{5.067045in}{0.440000in}}%
\pgfpathlineto{\pgfqpoint{5.067045in}{3.520000in}}%
\pgfusepath{stroke}%
\end{pgfscope}%
\begin{pgfscope}%
\definecolor{textcolor}{rgb}{0.150000,0.150000,0.150000}%
\pgfsetstrokecolor{textcolor}%
\pgfsetfillcolor{textcolor}%
\pgftext[x=5.067045in,y=0.342778in,,top]{\color{textcolor}\rmfamily\fontsize{10.000000}{12.000000}\selectfont \(\displaystyle {0.4}\)}%
\end{pgfscope}%
\begin{pgfscope}%
\pgfpathrectangle{\pgfqpoint{0.875000in}{0.440000in}}{\pgfqpoint{5.425000in}{3.080000in}}%
\pgfusepath{clip}%
\pgfsetroundcap%
\pgfsetroundjoin%
\pgfsetlinewidth{1.003750pt}%
\definecolor{currentstroke}{rgb}{1.000000,1.000000,1.000000}%
\pgfsetstrokecolor{currentstroke}%
\pgfsetdash{}{0pt}%
\pgfpathmoveto{\pgfqpoint{6.053409in}{0.440000in}}%
\pgfpathlineto{\pgfqpoint{6.053409in}{3.520000in}}%
\pgfusepath{stroke}%
\end{pgfscope}%
\begin{pgfscope}%
\definecolor{textcolor}{rgb}{0.150000,0.150000,0.150000}%
\pgfsetstrokecolor{textcolor}%
\pgfsetfillcolor{textcolor}%
\pgftext[x=6.053409in,y=0.342778in,,top]{\color{textcolor}\rmfamily\fontsize{10.000000}{12.000000}\selectfont \(\displaystyle {0.5}\)}%
\end{pgfscope}%
\begin{pgfscope}%
\pgfpathrectangle{\pgfqpoint{0.875000in}{0.440000in}}{\pgfqpoint{5.425000in}{3.080000in}}%
\pgfusepath{clip}%
\pgfsetroundcap%
\pgfsetroundjoin%
\pgfsetlinewidth{1.003750pt}%
\definecolor{currentstroke}{rgb}{1.000000,1.000000,1.000000}%
\pgfsetstrokecolor{currentstroke}%
\pgfsetdash{}{0pt}%
\pgfpathmoveto{\pgfqpoint{0.875000in}{0.580000in}}%
\pgfpathlineto{\pgfqpoint{6.300000in}{0.580000in}}%
\pgfusepath{stroke}%
\end{pgfscope}%
\begin{pgfscope}%
\definecolor{textcolor}{rgb}{0.150000,0.150000,0.150000}%
\pgfsetstrokecolor{textcolor}%
\pgfsetfillcolor{textcolor}%
\pgftext[x=0.600308in, y=0.531775in, left, base]{\color{textcolor}\rmfamily\fontsize{10.000000}{12.000000}\selectfont \(\displaystyle {0.0}\)}%
\end{pgfscope}%
\begin{pgfscope}%
\pgfpathrectangle{\pgfqpoint{0.875000in}{0.440000in}}{\pgfqpoint{5.425000in}{3.080000in}}%
\pgfusepath{clip}%
\pgfsetroundcap%
\pgfsetroundjoin%
\pgfsetlinewidth{1.003750pt}%
\definecolor{currentstroke}{rgb}{1.000000,1.000000,1.000000}%
\pgfsetstrokecolor{currentstroke}%
\pgfsetdash{}{0pt}%
\pgfpathmoveto{\pgfqpoint{0.875000in}{0.982149in}}%
\pgfpathlineto{\pgfqpoint{6.300000in}{0.982149in}}%
\pgfusepath{stroke}%
\end{pgfscope}%
\begin{pgfscope}%
\definecolor{textcolor}{rgb}{0.150000,0.150000,0.150000}%
\pgfsetstrokecolor{textcolor}%
\pgfsetfillcolor{textcolor}%
\pgftext[x=0.600308in, y=0.933924in, left, base]{\color{textcolor}\rmfamily\fontsize{10.000000}{12.000000}\selectfont \(\displaystyle {0.2}\)}%
\end{pgfscope}%
\begin{pgfscope}%
\pgfpathrectangle{\pgfqpoint{0.875000in}{0.440000in}}{\pgfqpoint{5.425000in}{3.080000in}}%
\pgfusepath{clip}%
\pgfsetroundcap%
\pgfsetroundjoin%
\pgfsetlinewidth{1.003750pt}%
\definecolor{currentstroke}{rgb}{1.000000,1.000000,1.000000}%
\pgfsetstrokecolor{currentstroke}%
\pgfsetdash{}{0pt}%
\pgfpathmoveto{\pgfqpoint{0.875000in}{1.384298in}}%
\pgfpathlineto{\pgfqpoint{6.300000in}{1.384298in}}%
\pgfusepath{stroke}%
\end{pgfscope}%
\begin{pgfscope}%
\definecolor{textcolor}{rgb}{0.150000,0.150000,0.150000}%
\pgfsetstrokecolor{textcolor}%
\pgfsetfillcolor{textcolor}%
\pgftext[x=0.600308in, y=1.336072in, left, base]{\color{textcolor}\rmfamily\fontsize{10.000000}{12.000000}\selectfont \(\displaystyle {0.4}\)}%
\end{pgfscope}%
\begin{pgfscope}%
\pgfpathrectangle{\pgfqpoint{0.875000in}{0.440000in}}{\pgfqpoint{5.425000in}{3.080000in}}%
\pgfusepath{clip}%
\pgfsetroundcap%
\pgfsetroundjoin%
\pgfsetlinewidth{1.003750pt}%
\definecolor{currentstroke}{rgb}{1.000000,1.000000,1.000000}%
\pgfsetstrokecolor{currentstroke}%
\pgfsetdash{}{0pt}%
\pgfpathmoveto{\pgfqpoint{0.875000in}{1.786446in}}%
\pgfpathlineto{\pgfqpoint{6.300000in}{1.786446in}}%
\pgfusepath{stroke}%
\end{pgfscope}%
\begin{pgfscope}%
\definecolor{textcolor}{rgb}{0.150000,0.150000,0.150000}%
\pgfsetstrokecolor{textcolor}%
\pgfsetfillcolor{textcolor}%
\pgftext[x=0.600308in, y=1.738221in, left, base]{\color{textcolor}\rmfamily\fontsize{10.000000}{12.000000}\selectfont \(\displaystyle {0.6}\)}%
\end{pgfscope}%
\begin{pgfscope}%
\pgfpathrectangle{\pgfqpoint{0.875000in}{0.440000in}}{\pgfqpoint{5.425000in}{3.080000in}}%
\pgfusepath{clip}%
\pgfsetroundcap%
\pgfsetroundjoin%
\pgfsetlinewidth{1.003750pt}%
\definecolor{currentstroke}{rgb}{1.000000,1.000000,1.000000}%
\pgfsetstrokecolor{currentstroke}%
\pgfsetdash{}{0pt}%
\pgfpathmoveto{\pgfqpoint{0.875000in}{2.188595in}}%
\pgfpathlineto{\pgfqpoint{6.300000in}{2.188595in}}%
\pgfusepath{stroke}%
\end{pgfscope}%
\begin{pgfscope}%
\definecolor{textcolor}{rgb}{0.150000,0.150000,0.150000}%
\pgfsetstrokecolor{textcolor}%
\pgfsetfillcolor{textcolor}%
\pgftext[x=0.600308in, y=2.140370in, left, base]{\color{textcolor}\rmfamily\fontsize{10.000000}{12.000000}\selectfont \(\displaystyle {0.8}\)}%
\end{pgfscope}%
\begin{pgfscope}%
\pgfpathrectangle{\pgfqpoint{0.875000in}{0.440000in}}{\pgfqpoint{5.425000in}{3.080000in}}%
\pgfusepath{clip}%
\pgfsetroundcap%
\pgfsetroundjoin%
\pgfsetlinewidth{1.003750pt}%
\definecolor{currentstroke}{rgb}{1.000000,1.000000,1.000000}%
\pgfsetstrokecolor{currentstroke}%
\pgfsetdash{}{0pt}%
\pgfpathmoveto{\pgfqpoint{0.875000in}{2.590744in}}%
\pgfpathlineto{\pgfqpoint{6.300000in}{2.590744in}}%
\pgfusepath{stroke}%
\end{pgfscope}%
\begin{pgfscope}%
\definecolor{textcolor}{rgb}{0.150000,0.150000,0.150000}%
\pgfsetstrokecolor{textcolor}%
\pgfsetfillcolor{textcolor}%
\pgftext[x=0.600308in, y=2.542519in, left, base]{\color{textcolor}\rmfamily\fontsize{10.000000}{12.000000}\selectfont \(\displaystyle {1.0}\)}%
\end{pgfscope}%
\begin{pgfscope}%
\pgfpathrectangle{\pgfqpoint{0.875000in}{0.440000in}}{\pgfqpoint{5.425000in}{3.080000in}}%
\pgfusepath{clip}%
\pgfsetroundcap%
\pgfsetroundjoin%
\pgfsetlinewidth{1.003750pt}%
\definecolor{currentstroke}{rgb}{1.000000,1.000000,1.000000}%
\pgfsetstrokecolor{currentstroke}%
\pgfsetdash{}{0pt}%
\pgfpathmoveto{\pgfqpoint{0.875000in}{2.992893in}}%
\pgfpathlineto{\pgfqpoint{6.300000in}{2.992893in}}%
\pgfusepath{stroke}%
\end{pgfscope}%
\begin{pgfscope}%
\definecolor{textcolor}{rgb}{0.150000,0.150000,0.150000}%
\pgfsetstrokecolor{textcolor}%
\pgfsetfillcolor{textcolor}%
\pgftext[x=0.600308in, y=2.944668in, left, base]{\color{textcolor}\rmfamily\fontsize{10.000000}{12.000000}\selectfont \(\displaystyle {1.2}\)}%
\end{pgfscope}%
\begin{pgfscope}%
\pgfpathrectangle{\pgfqpoint{0.875000in}{0.440000in}}{\pgfqpoint{5.425000in}{3.080000in}}%
\pgfusepath{clip}%
\pgfsetroundcap%
\pgfsetroundjoin%
\pgfsetlinewidth{1.003750pt}%
\definecolor{currentstroke}{rgb}{1.000000,1.000000,1.000000}%
\pgfsetstrokecolor{currentstroke}%
\pgfsetdash{}{0pt}%
\pgfpathmoveto{\pgfqpoint{0.875000in}{3.395042in}}%
\pgfpathlineto{\pgfqpoint{6.300000in}{3.395042in}}%
\pgfusepath{stroke}%
\end{pgfscope}%
\begin{pgfscope}%
\definecolor{textcolor}{rgb}{0.150000,0.150000,0.150000}%
\pgfsetstrokecolor{textcolor}%
\pgfsetfillcolor{textcolor}%
\pgftext[x=0.600308in, y=3.346816in, left, base]{\color{textcolor}\rmfamily\fontsize{10.000000}{12.000000}\selectfont \(\displaystyle {1.4}\)}%
\end{pgfscope}%
\begin{pgfscope}%
\definecolor{textcolor}{rgb}{0.150000,0.150000,0.150000}%
\pgfsetstrokecolor{textcolor}%
\pgfsetfillcolor{textcolor}%
\pgftext[x=0.875000in,y=3.561667in,left,base]{\color{textcolor}\rmfamily\fontsize{10.000000}{12.000000}\selectfont \(\displaystyle \times{10^{\ensuremath{-}12}}{}\)}%
\end{pgfscope}%
\begin{pgfscope}%
\pgfpathrectangle{\pgfqpoint{0.875000in}{0.440000in}}{\pgfqpoint{5.425000in}{3.080000in}}%
\pgfusepath{clip}%
\pgfsetroundcap%
\pgfsetroundjoin%
\pgfsetlinewidth{1.756562pt}%
\definecolor{currentstroke}{rgb}{0.298039,0.447059,0.690196}%
\pgfsetstrokecolor{currentstroke}%
\pgfsetdash{}{0pt}%
\pgfpathmoveto{\pgfqpoint{1.121591in}{0.580000in}}%
\pgfpathlineto{\pgfqpoint{1.171407in}{0.580000in}}%
\pgfpathlineto{\pgfqpoint{1.221224in}{0.580000in}}%
\pgfpathlineto{\pgfqpoint{1.271040in}{0.580000in}}%
\pgfpathlineto{\pgfqpoint{1.320856in}{0.580000in}}%
\pgfpathlineto{\pgfqpoint{1.370673in}{0.580000in}}%
\pgfpathlineto{\pgfqpoint{1.420489in}{0.580000in}}%
\pgfpathlineto{\pgfqpoint{1.470305in}{0.580000in}}%
\pgfpathlineto{\pgfqpoint{1.520122in}{0.580000in}}%
\pgfpathlineto{\pgfqpoint{1.569938in}{0.580000in}}%
\pgfpathlineto{\pgfqpoint{1.619754in}{0.580000in}}%
\pgfpathlineto{\pgfqpoint{1.669571in}{0.580000in}}%
\pgfpathlineto{\pgfqpoint{1.719387in}{0.580000in}}%
\pgfpathlineto{\pgfqpoint{1.769203in}{0.580001in}}%
\pgfpathlineto{\pgfqpoint{1.819020in}{0.580002in}}%
\pgfpathlineto{\pgfqpoint{1.868836in}{0.580004in}}%
\pgfpathlineto{\pgfqpoint{1.918652in}{0.580010in}}%
\pgfpathlineto{\pgfqpoint{1.968469in}{0.580021in}}%
\pgfpathlineto{\pgfqpoint{2.018285in}{0.580043in}}%
\pgfpathlineto{\pgfqpoint{2.068101in}{0.580083in}}%
\pgfpathlineto{\pgfqpoint{2.117918in}{0.580155in}}%
\pgfpathlineto{\pgfqpoint{2.167734in}{0.580278in}}%
\pgfpathlineto{\pgfqpoint{2.217551in}{0.580483in}}%
\pgfpathlineto{\pgfqpoint{2.267367in}{0.580811in}}%
\pgfpathlineto{\pgfqpoint{2.317183in}{0.581322in}}%
\pgfpathlineto{\pgfqpoint{2.367000in}{0.582098in}}%
\pgfpathlineto{\pgfqpoint{2.416816in}{0.583247in}}%
\pgfpathlineto{\pgfqpoint{2.466632in}{0.584909in}}%
\pgfpathlineto{\pgfqpoint{2.516449in}{0.587261in}}%
\pgfpathlineto{\pgfqpoint{2.566265in}{0.590524in}}%
\pgfpathlineto{\pgfqpoint{2.616081in}{0.594967in}}%
\pgfpathlineto{\pgfqpoint{2.665898in}{0.600907in}}%
\pgfpathlineto{\pgfqpoint{2.715714in}{0.608717in}}%
\pgfpathlineto{\pgfqpoint{2.765530in}{0.618823in}}%
\pgfpathlineto{\pgfqpoint{2.815347in}{0.631702in}}%
\pgfpathlineto{\pgfqpoint{2.865163in}{0.647879in}}%
\pgfpathlineto{\pgfqpoint{2.914979in}{0.667920in}}%
\pgfpathlineto{\pgfqpoint{2.964796in}{0.692420in}}%
\pgfpathlineto{\pgfqpoint{3.014612in}{0.721991in}}%
\pgfpathlineto{\pgfqpoint{3.064428in}{0.757247in}}%
\pgfpathlineto{\pgfqpoint{3.114245in}{0.798785in}}%
\pgfpathlineto{\pgfqpoint{3.164061in}{0.847166in}}%
\pgfpathlineto{\pgfqpoint{3.213877in}{0.902891in}}%
\pgfpathlineto{\pgfqpoint{3.263694in}{0.966378in}}%
\pgfpathlineto{\pgfqpoint{3.313510in}{1.037943in}}%
\pgfpathlineto{\pgfqpoint{3.363326in}{1.117775in}}%
\pgfpathlineto{\pgfqpoint{3.413143in}{1.205917in}}%
\pgfpathlineto{\pgfqpoint{3.462959in}{1.302249in}}%
\pgfpathlineto{\pgfqpoint{3.512775in}{1.406472in}}%
\pgfpathlineto{\pgfqpoint{3.562592in}{1.518103in}}%
\pgfpathlineto{\pgfqpoint{3.612408in}{1.636463in}}%
\pgfpathlineto{\pgfqpoint{3.662225in}{1.760685in}}%
\pgfpathlineto{\pgfqpoint{3.712041in}{1.889714in}}%
\pgfpathlineto{\pgfqpoint{3.761857in}{2.022324in}}%
\pgfpathlineto{\pgfqpoint{3.811674in}{2.157128in}}%
\pgfpathlineto{\pgfqpoint{3.861490in}{2.292608in}}%
\pgfpathlineto{\pgfqpoint{3.911306in}{2.427139in}}%
\pgfpathlineto{\pgfqpoint{3.961123in}{2.559019in}}%
\pgfpathlineto{\pgfqpoint{4.010939in}{2.686506in}}%
\pgfpathlineto{\pgfqpoint{4.060755in}{2.807853in}}%
\pgfpathlineto{\pgfqpoint{4.110572in}{2.921347in}}%
\pgfpathlineto{\pgfqpoint{4.160388in}{3.025343in}}%
\pgfpathlineto{\pgfqpoint{4.210204in}{3.118308in}}%
\pgfpathlineto{\pgfqpoint{4.260021in}{3.198849in}}%
\pgfpathlineto{\pgfqpoint{4.309837in}{3.265747in}}%
\pgfpathlineto{\pgfqpoint{4.359653in}{3.317985in}}%
\pgfpathlineto{\pgfqpoint{4.409470in}{3.354767in}}%
\pgfpathlineto{\pgfqpoint{4.459286in}{3.375540in}}%
\pgfpathlineto{\pgfqpoint{4.509102in}{3.380000in}}%
\pgfpathlineto{\pgfqpoint{4.558919in}{3.368101in}}%
\pgfpathlineto{\pgfqpoint{4.608735in}{3.340051in}}%
\pgfpathlineto{\pgfqpoint{4.658551in}{3.296306in}}%
\pgfpathlineto{\pgfqpoint{4.708368in}{3.237556in}}%
\pgfpathlineto{\pgfqpoint{4.758184in}{3.164708in}}%
\pgfpathlineto{\pgfqpoint{4.808000in}{3.078864in}}%
\pgfpathlineto{\pgfqpoint{4.857817in}{2.981290in}}%
\pgfpathlineto{\pgfqpoint{4.907633in}{2.873394in}}%
\pgfpathlineto{\pgfqpoint{4.957449in}{2.756687in}}%
\pgfpathlineto{\pgfqpoint{5.007266in}{2.632756in}}%
\pgfpathlineto{\pgfqpoint{5.057082in}{2.503228in}}%
\pgfpathlineto{\pgfqpoint{5.106899in}{2.369737in}}%
\pgfpathlineto{\pgfqpoint{5.156715in}{2.233895in}}%
\pgfpathlineto{\pgfqpoint{5.206531in}{2.097264in}}%
\pgfpathlineto{\pgfqpoint{5.256348in}{1.961326in}}%
\pgfpathlineto{\pgfqpoint{5.306164in}{1.827464in}}%
\pgfpathlineto{\pgfqpoint{5.355980in}{1.696939in}}%
\pgfpathlineto{\pgfqpoint{5.405797in}{1.570879in}}%
\pgfpathlineto{\pgfqpoint{5.455613in}{1.450264in}}%
\pgfpathlineto{\pgfqpoint{5.505429in}{1.335921in}}%
\pgfpathlineto{\pgfqpoint{5.555246in}{1.228519in}}%
\pgfpathlineto{\pgfqpoint{5.605062in}{1.128569in}}%
\pgfpathlineto{\pgfqpoint{5.654878in}{1.036428in}}%
\pgfpathlineto{\pgfqpoint{5.704695in}{0.952308in}}%
\pgfpathlineto{\pgfqpoint{5.754511in}{0.876282in}}%
\pgfpathlineto{\pgfqpoint{5.804327in}{0.808297in}}%
\pgfpathlineto{\pgfqpoint{5.854144in}{0.748185in}}%
\pgfpathlineto{\pgfqpoint{5.903960in}{0.695681in}}%
\pgfpathlineto{\pgfqpoint{5.953776in}{0.650437in}}%
\pgfpathlineto{\pgfqpoint{6.003593in}{0.612034in}}%
\pgfpathlineto{\pgfqpoint{6.053409in}{0.580000in}}%
\pgfusepath{stroke}%
\end{pgfscope}%
\begin{pgfscope}%
\pgfsetrectcap%
\pgfsetmiterjoin%
\pgfsetlinewidth{0.000000pt}%
\definecolor{currentstroke}{rgb}{1.000000,1.000000,1.000000}%
\pgfsetstrokecolor{currentstroke}%
\pgfsetdash{}{0pt}%
\pgfpathmoveto{\pgfqpoint{0.875000in}{0.440000in}}%
\pgfpathlineto{\pgfqpoint{0.875000in}{3.520000in}}%
\pgfusepath{}%
\end{pgfscope}%
\begin{pgfscope}%
\pgfsetrectcap%
\pgfsetmiterjoin%
\pgfsetlinewidth{0.000000pt}%
\definecolor{currentstroke}{rgb}{1.000000,1.000000,1.000000}%
\pgfsetstrokecolor{currentstroke}%
\pgfsetdash{}{0pt}%
\pgfpathmoveto{\pgfqpoint{6.300000in}{0.440000in}}%
\pgfpathlineto{\pgfqpoint{6.300000in}{3.520000in}}%
\pgfusepath{}%
\end{pgfscope}%
\begin{pgfscope}%
\pgfsetrectcap%
\pgfsetmiterjoin%
\pgfsetlinewidth{0.000000pt}%
\definecolor{currentstroke}{rgb}{1.000000,1.000000,1.000000}%
\pgfsetstrokecolor{currentstroke}%
\pgfsetdash{}{0pt}%
\pgfpathmoveto{\pgfqpoint{0.875000in}{0.440000in}}%
\pgfpathlineto{\pgfqpoint{6.300000in}{0.440000in}}%
\pgfusepath{}%
\end{pgfscope}%
\begin{pgfscope}%
\pgfsetrectcap%
\pgfsetmiterjoin%
\pgfsetlinewidth{0.000000pt}%
\definecolor{currentstroke}{rgb}{1.000000,1.000000,1.000000}%
\pgfsetstrokecolor{currentstroke}%
\pgfsetdash{}{0pt}%
\pgfpathmoveto{\pgfqpoint{0.875000in}{3.520000in}}%
\pgfpathlineto{\pgfqpoint{6.300000in}{3.520000in}}%
\pgfusepath{}%
\end{pgfscope}%
\begin{pgfscope}%
\definecolor{textcolor}{rgb}{0.150000,0.150000,0.150000}%
\pgfsetstrokecolor{textcolor}%
\pgfsetfillcolor{textcolor}%
\pgftext[x=3.500000in,y=3.920000in,,top]{\color{textcolor}\rmfamily\fontsize{12.000000}{14.400000}\selectfont Múltiplo de la densidad objetivo con \(\displaystyle n=40, r=\)15}%
\end{pgfscope}%
\end{pgfpicture}%
\makeatother%
\endgroup%

    \end{center}

    Estas gráficas en realidad corresponden a múltiplos de la densidad, ya que no conocemos sus constantes
    de normalización. Precisamente parte de la utilidad del algoritmo de Metropolis-Hastings es que únicamente
    necesitamos conocer un múltiplo de la densidad objetivo para muestrear de ella.

    Al implementar el algoritmo con los datos descritos obtenemos los siguientes histogramas

    \begin{center}
        %% Creator: Matplotlib, PGF backend
%%
%% To include the figure in your LaTeX document, write
%%   \input{<filename>.pgf}
%%
%% Make sure the required packages are loaded in your preamble
%%   \usepackage{pgf}
%%
%% Also ensure that all the required font packages are loaded; for instance,
%% the lmodern package is sometimes necessary when using math font.
%%   \usepackage{lmodern}
%%
%% Figures using additional raster images can only be included by \input if
%% they are in the same directory as the main LaTeX file. For loading figures
%% from other directories you can use the `import` package
%%   \usepackage{import}
%%
%% and then include the figures with
%%   \import{<path to file>}{<filename>.pgf}
%%
%% Matplotlib used the following preamble
%%   
%%   \makeatletter\@ifpackageloaded{underscore}{}{\usepackage[strings]{underscore}}\makeatother
%%
\begingroup%
\makeatletter%
\begin{pgfpicture}%
\pgfpathrectangle{\pgfpointorigin}{\pgfqpoint{7.000000in}{4.000000in}}%
\pgfusepath{use as bounding box, clip}%
\begin{pgfscope}%
\pgfsetbuttcap%
\pgfsetmiterjoin%
\definecolor{currentfill}{rgb}{1.000000,1.000000,1.000000}%
\pgfsetfillcolor{currentfill}%
\pgfsetlinewidth{0.000000pt}%
\definecolor{currentstroke}{rgb}{1.000000,1.000000,1.000000}%
\pgfsetstrokecolor{currentstroke}%
\pgfsetdash{}{0pt}%
\pgfpathmoveto{\pgfqpoint{0.000000in}{0.000000in}}%
\pgfpathlineto{\pgfqpoint{7.000000in}{0.000000in}}%
\pgfpathlineto{\pgfqpoint{7.000000in}{4.000000in}}%
\pgfpathlineto{\pgfqpoint{0.000000in}{4.000000in}}%
\pgfpathlineto{\pgfqpoint{0.000000in}{0.000000in}}%
\pgfpathclose%
\pgfusepath{fill}%
\end{pgfscope}%
\begin{pgfscope}%
\pgfsetbuttcap%
\pgfsetmiterjoin%
\definecolor{currentfill}{rgb}{0.917647,0.917647,0.949020}%
\pgfsetfillcolor{currentfill}%
\pgfsetlinewidth{0.000000pt}%
\definecolor{currentstroke}{rgb}{0.000000,0.000000,0.000000}%
\pgfsetstrokecolor{currentstroke}%
\pgfsetstrokeopacity{0.000000}%
\pgfsetdash{}{0pt}%
\pgfpathmoveto{\pgfqpoint{0.875000in}{0.440000in}}%
\pgfpathlineto{\pgfqpoint{6.300000in}{0.440000in}}%
\pgfpathlineto{\pgfqpoint{6.300000in}{3.520000in}}%
\pgfpathlineto{\pgfqpoint{0.875000in}{3.520000in}}%
\pgfpathlineto{\pgfqpoint{0.875000in}{0.440000in}}%
\pgfpathclose%
\pgfusepath{fill}%
\end{pgfscope}%
\begin{pgfscope}%
\pgfpathrectangle{\pgfqpoint{0.875000in}{0.440000in}}{\pgfqpoint{5.425000in}{3.080000in}}%
\pgfusepath{clip}%
\pgfsetroundcap%
\pgfsetroundjoin%
\pgfsetlinewidth{1.003750pt}%
\definecolor{currentstroke}{rgb}{1.000000,1.000000,1.000000}%
\pgfsetstrokecolor{currentstroke}%
\pgfsetdash{}{0pt}%
\pgfpathmoveto{\pgfqpoint{1.834268in}{0.440000in}}%
\pgfpathlineto{\pgfqpoint{1.834268in}{3.520000in}}%
\pgfusepath{stroke}%
\end{pgfscope}%
\begin{pgfscope}%
\definecolor{textcolor}{rgb}{0.150000,0.150000,0.150000}%
\pgfsetstrokecolor{textcolor}%
\pgfsetfillcolor{textcolor}%
\pgftext[x=1.834268in,y=0.342778in,,top]{\color{textcolor}\rmfamily\fontsize{10.000000}{12.000000}\selectfont \(\displaystyle {0.1}\)}%
\end{pgfscope}%
\begin{pgfscope}%
\pgfpathrectangle{\pgfqpoint{0.875000in}{0.440000in}}{\pgfqpoint{5.425000in}{3.080000in}}%
\pgfusepath{clip}%
\pgfsetroundcap%
\pgfsetroundjoin%
\pgfsetlinewidth{1.003750pt}%
\definecolor{currentstroke}{rgb}{1.000000,1.000000,1.000000}%
\pgfsetstrokecolor{currentstroke}%
\pgfsetdash{}{0pt}%
\pgfpathmoveto{\pgfqpoint{2.891043in}{0.440000in}}%
\pgfpathlineto{\pgfqpoint{2.891043in}{3.520000in}}%
\pgfusepath{stroke}%
\end{pgfscope}%
\begin{pgfscope}%
\definecolor{textcolor}{rgb}{0.150000,0.150000,0.150000}%
\pgfsetstrokecolor{textcolor}%
\pgfsetfillcolor{textcolor}%
\pgftext[x=2.891043in,y=0.342778in,,top]{\color{textcolor}\rmfamily\fontsize{10.000000}{12.000000}\selectfont \(\displaystyle {0.2}\)}%
\end{pgfscope}%
\begin{pgfscope}%
\pgfpathrectangle{\pgfqpoint{0.875000in}{0.440000in}}{\pgfqpoint{5.425000in}{3.080000in}}%
\pgfusepath{clip}%
\pgfsetroundcap%
\pgfsetroundjoin%
\pgfsetlinewidth{1.003750pt}%
\definecolor{currentstroke}{rgb}{1.000000,1.000000,1.000000}%
\pgfsetstrokecolor{currentstroke}%
\pgfsetdash{}{0pt}%
\pgfpathmoveto{\pgfqpoint{3.947818in}{0.440000in}}%
\pgfpathlineto{\pgfqpoint{3.947818in}{3.520000in}}%
\pgfusepath{stroke}%
\end{pgfscope}%
\begin{pgfscope}%
\definecolor{textcolor}{rgb}{0.150000,0.150000,0.150000}%
\pgfsetstrokecolor{textcolor}%
\pgfsetfillcolor{textcolor}%
\pgftext[x=3.947818in,y=0.342778in,,top]{\color{textcolor}\rmfamily\fontsize{10.000000}{12.000000}\selectfont \(\displaystyle {0.3}\)}%
\end{pgfscope}%
\begin{pgfscope}%
\pgfpathrectangle{\pgfqpoint{0.875000in}{0.440000in}}{\pgfqpoint{5.425000in}{3.080000in}}%
\pgfusepath{clip}%
\pgfsetroundcap%
\pgfsetroundjoin%
\pgfsetlinewidth{1.003750pt}%
\definecolor{currentstroke}{rgb}{1.000000,1.000000,1.000000}%
\pgfsetstrokecolor{currentstroke}%
\pgfsetdash{}{0pt}%
\pgfpathmoveto{\pgfqpoint{5.004592in}{0.440000in}}%
\pgfpathlineto{\pgfqpoint{5.004592in}{3.520000in}}%
\pgfusepath{stroke}%
\end{pgfscope}%
\begin{pgfscope}%
\definecolor{textcolor}{rgb}{0.150000,0.150000,0.150000}%
\pgfsetstrokecolor{textcolor}%
\pgfsetfillcolor{textcolor}%
\pgftext[x=5.004592in,y=0.342778in,,top]{\color{textcolor}\rmfamily\fontsize{10.000000}{12.000000}\selectfont \(\displaystyle {0.4}\)}%
\end{pgfscope}%
\begin{pgfscope}%
\pgfpathrectangle{\pgfqpoint{0.875000in}{0.440000in}}{\pgfqpoint{5.425000in}{3.080000in}}%
\pgfusepath{clip}%
\pgfsetroundcap%
\pgfsetroundjoin%
\pgfsetlinewidth{1.003750pt}%
\definecolor{currentstroke}{rgb}{1.000000,1.000000,1.000000}%
\pgfsetstrokecolor{currentstroke}%
\pgfsetdash{}{0pt}%
\pgfpathmoveto{\pgfqpoint{6.061367in}{0.440000in}}%
\pgfpathlineto{\pgfqpoint{6.061367in}{3.520000in}}%
\pgfusepath{stroke}%
\end{pgfscope}%
\begin{pgfscope}%
\definecolor{textcolor}{rgb}{0.150000,0.150000,0.150000}%
\pgfsetstrokecolor{textcolor}%
\pgfsetfillcolor{textcolor}%
\pgftext[x=6.061367in,y=0.342778in,,top]{\color{textcolor}\rmfamily\fontsize{10.000000}{12.000000}\selectfont \(\displaystyle {0.5}\)}%
\end{pgfscope}%
\begin{pgfscope}%
\definecolor{textcolor}{rgb}{0.150000,0.150000,0.150000}%
\pgfsetstrokecolor{textcolor}%
\pgfsetfillcolor{textcolor}%
\pgftext[x=3.587500in,y=0.163766in,,top]{\color{textcolor}\rmfamily\fontsize{11.000000}{13.200000}\selectfont Valor}%
\end{pgfscope}%
\begin{pgfscope}%
\pgfpathrectangle{\pgfqpoint{0.875000in}{0.440000in}}{\pgfqpoint{5.425000in}{3.080000in}}%
\pgfusepath{clip}%
\pgfsetroundcap%
\pgfsetroundjoin%
\pgfsetlinewidth{1.003750pt}%
\definecolor{currentstroke}{rgb}{1.000000,1.000000,1.000000}%
\pgfsetstrokecolor{currentstroke}%
\pgfsetdash{}{0pt}%
\pgfpathmoveto{\pgfqpoint{0.875000in}{0.440000in}}%
\pgfpathlineto{\pgfqpoint{6.300000in}{0.440000in}}%
\pgfusepath{stroke}%
\end{pgfscope}%
\begin{pgfscope}%
\definecolor{textcolor}{rgb}{0.150000,0.150000,0.150000}%
\pgfsetstrokecolor{textcolor}%
\pgfsetfillcolor{textcolor}%
\pgftext[x=0.600308in, y=0.391775in, left, base]{\color{textcolor}\rmfamily\fontsize{10.000000}{12.000000}\selectfont \(\displaystyle {0.0}\)}%
\end{pgfscope}%
\begin{pgfscope}%
\pgfpathrectangle{\pgfqpoint{0.875000in}{0.440000in}}{\pgfqpoint{5.425000in}{3.080000in}}%
\pgfusepath{clip}%
\pgfsetroundcap%
\pgfsetroundjoin%
\pgfsetlinewidth{1.003750pt}%
\definecolor{currentstroke}{rgb}{1.000000,1.000000,1.000000}%
\pgfsetstrokecolor{currentstroke}%
\pgfsetdash{}{0pt}%
\pgfpathmoveto{\pgfqpoint{0.875000in}{0.782998in}}%
\pgfpathlineto{\pgfqpoint{6.300000in}{0.782998in}}%
\pgfusepath{stroke}%
\end{pgfscope}%
\begin{pgfscope}%
\definecolor{textcolor}{rgb}{0.150000,0.150000,0.150000}%
\pgfsetstrokecolor{textcolor}%
\pgfsetfillcolor{textcolor}%
\pgftext[x=0.600308in, y=0.734772in, left, base]{\color{textcolor}\rmfamily\fontsize{10.000000}{12.000000}\selectfont \(\displaystyle {0.5}\)}%
\end{pgfscope}%
\begin{pgfscope}%
\pgfpathrectangle{\pgfqpoint{0.875000in}{0.440000in}}{\pgfqpoint{5.425000in}{3.080000in}}%
\pgfusepath{clip}%
\pgfsetroundcap%
\pgfsetroundjoin%
\pgfsetlinewidth{1.003750pt}%
\definecolor{currentstroke}{rgb}{1.000000,1.000000,1.000000}%
\pgfsetstrokecolor{currentstroke}%
\pgfsetdash{}{0pt}%
\pgfpathmoveto{\pgfqpoint{0.875000in}{1.125995in}}%
\pgfpathlineto{\pgfqpoint{6.300000in}{1.125995in}}%
\pgfusepath{stroke}%
\end{pgfscope}%
\begin{pgfscope}%
\definecolor{textcolor}{rgb}{0.150000,0.150000,0.150000}%
\pgfsetstrokecolor{textcolor}%
\pgfsetfillcolor{textcolor}%
\pgftext[x=0.600308in, y=1.077770in, left, base]{\color{textcolor}\rmfamily\fontsize{10.000000}{12.000000}\selectfont \(\displaystyle {1.0}\)}%
\end{pgfscope}%
\begin{pgfscope}%
\pgfpathrectangle{\pgfqpoint{0.875000in}{0.440000in}}{\pgfqpoint{5.425000in}{3.080000in}}%
\pgfusepath{clip}%
\pgfsetroundcap%
\pgfsetroundjoin%
\pgfsetlinewidth{1.003750pt}%
\definecolor{currentstroke}{rgb}{1.000000,1.000000,1.000000}%
\pgfsetstrokecolor{currentstroke}%
\pgfsetdash{}{0pt}%
\pgfpathmoveto{\pgfqpoint{0.875000in}{1.468993in}}%
\pgfpathlineto{\pgfqpoint{6.300000in}{1.468993in}}%
\pgfusepath{stroke}%
\end{pgfscope}%
\begin{pgfscope}%
\definecolor{textcolor}{rgb}{0.150000,0.150000,0.150000}%
\pgfsetstrokecolor{textcolor}%
\pgfsetfillcolor{textcolor}%
\pgftext[x=0.600308in, y=1.420768in, left, base]{\color{textcolor}\rmfamily\fontsize{10.000000}{12.000000}\selectfont \(\displaystyle {1.5}\)}%
\end{pgfscope}%
\begin{pgfscope}%
\pgfpathrectangle{\pgfqpoint{0.875000in}{0.440000in}}{\pgfqpoint{5.425000in}{3.080000in}}%
\pgfusepath{clip}%
\pgfsetroundcap%
\pgfsetroundjoin%
\pgfsetlinewidth{1.003750pt}%
\definecolor{currentstroke}{rgb}{1.000000,1.000000,1.000000}%
\pgfsetstrokecolor{currentstroke}%
\pgfsetdash{}{0pt}%
\pgfpathmoveto{\pgfqpoint{0.875000in}{1.811991in}}%
\pgfpathlineto{\pgfqpoint{6.300000in}{1.811991in}}%
\pgfusepath{stroke}%
\end{pgfscope}%
\begin{pgfscope}%
\definecolor{textcolor}{rgb}{0.150000,0.150000,0.150000}%
\pgfsetstrokecolor{textcolor}%
\pgfsetfillcolor{textcolor}%
\pgftext[x=0.600308in, y=1.763766in, left, base]{\color{textcolor}\rmfamily\fontsize{10.000000}{12.000000}\selectfont \(\displaystyle {2.0}\)}%
\end{pgfscope}%
\begin{pgfscope}%
\pgfpathrectangle{\pgfqpoint{0.875000in}{0.440000in}}{\pgfqpoint{5.425000in}{3.080000in}}%
\pgfusepath{clip}%
\pgfsetroundcap%
\pgfsetroundjoin%
\pgfsetlinewidth{1.003750pt}%
\definecolor{currentstroke}{rgb}{1.000000,1.000000,1.000000}%
\pgfsetstrokecolor{currentstroke}%
\pgfsetdash{}{0pt}%
\pgfpathmoveto{\pgfqpoint{0.875000in}{2.154989in}}%
\pgfpathlineto{\pgfqpoint{6.300000in}{2.154989in}}%
\pgfusepath{stroke}%
\end{pgfscope}%
\begin{pgfscope}%
\definecolor{textcolor}{rgb}{0.150000,0.150000,0.150000}%
\pgfsetstrokecolor{textcolor}%
\pgfsetfillcolor{textcolor}%
\pgftext[x=0.600308in, y=2.106763in, left, base]{\color{textcolor}\rmfamily\fontsize{10.000000}{12.000000}\selectfont \(\displaystyle {2.5}\)}%
\end{pgfscope}%
\begin{pgfscope}%
\pgfpathrectangle{\pgfqpoint{0.875000in}{0.440000in}}{\pgfqpoint{5.425000in}{3.080000in}}%
\pgfusepath{clip}%
\pgfsetroundcap%
\pgfsetroundjoin%
\pgfsetlinewidth{1.003750pt}%
\definecolor{currentstroke}{rgb}{1.000000,1.000000,1.000000}%
\pgfsetstrokecolor{currentstroke}%
\pgfsetdash{}{0pt}%
\pgfpathmoveto{\pgfqpoint{0.875000in}{2.497986in}}%
\pgfpathlineto{\pgfqpoint{6.300000in}{2.497986in}}%
\pgfusepath{stroke}%
\end{pgfscope}%
\begin{pgfscope}%
\definecolor{textcolor}{rgb}{0.150000,0.150000,0.150000}%
\pgfsetstrokecolor{textcolor}%
\pgfsetfillcolor{textcolor}%
\pgftext[x=0.600308in, y=2.449761in, left, base]{\color{textcolor}\rmfamily\fontsize{10.000000}{12.000000}\selectfont \(\displaystyle {3.0}\)}%
\end{pgfscope}%
\begin{pgfscope}%
\pgfpathrectangle{\pgfqpoint{0.875000in}{0.440000in}}{\pgfqpoint{5.425000in}{3.080000in}}%
\pgfusepath{clip}%
\pgfsetroundcap%
\pgfsetroundjoin%
\pgfsetlinewidth{1.003750pt}%
\definecolor{currentstroke}{rgb}{1.000000,1.000000,1.000000}%
\pgfsetstrokecolor{currentstroke}%
\pgfsetdash{}{0pt}%
\pgfpathmoveto{\pgfqpoint{0.875000in}{2.840984in}}%
\pgfpathlineto{\pgfqpoint{6.300000in}{2.840984in}}%
\pgfusepath{stroke}%
\end{pgfscope}%
\begin{pgfscope}%
\definecolor{textcolor}{rgb}{0.150000,0.150000,0.150000}%
\pgfsetstrokecolor{textcolor}%
\pgfsetfillcolor{textcolor}%
\pgftext[x=0.600308in, y=2.792759in, left, base]{\color{textcolor}\rmfamily\fontsize{10.000000}{12.000000}\selectfont \(\displaystyle {3.5}\)}%
\end{pgfscope}%
\begin{pgfscope}%
\pgfpathrectangle{\pgfqpoint{0.875000in}{0.440000in}}{\pgfqpoint{5.425000in}{3.080000in}}%
\pgfusepath{clip}%
\pgfsetroundcap%
\pgfsetroundjoin%
\pgfsetlinewidth{1.003750pt}%
\definecolor{currentstroke}{rgb}{1.000000,1.000000,1.000000}%
\pgfsetstrokecolor{currentstroke}%
\pgfsetdash{}{0pt}%
\pgfpathmoveto{\pgfqpoint{0.875000in}{3.183982in}}%
\pgfpathlineto{\pgfqpoint{6.300000in}{3.183982in}}%
\pgfusepath{stroke}%
\end{pgfscope}%
\begin{pgfscope}%
\definecolor{textcolor}{rgb}{0.150000,0.150000,0.150000}%
\pgfsetstrokecolor{textcolor}%
\pgfsetfillcolor{textcolor}%
\pgftext[x=0.600308in, y=3.135757in, left, base]{\color{textcolor}\rmfamily\fontsize{10.000000}{12.000000}\selectfont \(\displaystyle {4.0}\)}%
\end{pgfscope}%
\begin{pgfscope}%
\definecolor{textcolor}{rgb}{0.150000,0.150000,0.150000}%
\pgfsetstrokecolor{textcolor}%
\pgfsetfillcolor{textcolor}%
\pgftext[x=0.544752in,y=1.980000in,,bottom,rotate=90.000000]{\color{textcolor}\rmfamily\fontsize{11.000000}{13.200000}\selectfont Frecuencia}%
\end{pgfscope}%
\begin{pgfscope}%
\pgfpathrectangle{\pgfqpoint{0.875000in}{0.440000in}}{\pgfqpoint{5.425000in}{3.080000in}}%
\pgfusepath{clip}%
\pgfsetbuttcap%
\pgfsetmiterjoin%
\definecolor{currentfill}{rgb}{0.298039,0.447059,0.690196}%
\pgfsetfillcolor{currentfill}%
\pgfsetfillopacity{0.800000}%
\pgfsetlinewidth{0.000000pt}%
\definecolor{currentstroke}{rgb}{0.000000,0.000000,0.000000}%
\pgfsetstrokecolor{currentstroke}%
\pgfsetstrokeopacity{0.800000}%
\pgfsetdash{}{0pt}%
\pgfpathmoveto{\pgfqpoint{1.121591in}{0.440000in}}%
\pgfpathlineto{\pgfqpoint{1.285985in}{0.440000in}}%
\pgfpathlineto{\pgfqpoint{1.285985in}{0.489389in}}%
\pgfpathlineto{\pgfqpoint{1.121591in}{0.489389in}}%
\pgfpathlineto{\pgfqpoint{1.121591in}{0.440000in}}%
\pgfpathclose%
\pgfusepath{fill}%
\end{pgfscope}%
\begin{pgfscope}%
\pgfpathrectangle{\pgfqpoint{0.875000in}{0.440000in}}{\pgfqpoint{5.425000in}{3.080000in}}%
\pgfusepath{clip}%
\pgfsetbuttcap%
\pgfsetmiterjoin%
\definecolor{currentfill}{rgb}{0.298039,0.447059,0.690196}%
\pgfsetfillcolor{currentfill}%
\pgfsetfillopacity{0.800000}%
\pgfsetlinewidth{0.000000pt}%
\definecolor{currentstroke}{rgb}{0.000000,0.000000,0.000000}%
\pgfsetstrokecolor{currentstroke}%
\pgfsetstrokeopacity{0.800000}%
\pgfsetdash{}{0pt}%
\pgfpathmoveto{\pgfqpoint{1.285985in}{0.440000in}}%
\pgfpathlineto{\pgfqpoint{1.450379in}{0.440000in}}%
\pgfpathlineto{\pgfqpoint{1.450379in}{0.495562in}}%
\pgfpathlineto{\pgfqpoint{1.285985in}{0.495562in}}%
\pgfpathlineto{\pgfqpoint{1.285985in}{0.440000in}}%
\pgfpathclose%
\pgfusepath{fill}%
\end{pgfscope}%
\begin{pgfscope}%
\pgfpathrectangle{\pgfqpoint{0.875000in}{0.440000in}}{\pgfqpoint{5.425000in}{3.080000in}}%
\pgfusepath{clip}%
\pgfsetbuttcap%
\pgfsetmiterjoin%
\definecolor{currentfill}{rgb}{0.298039,0.447059,0.690196}%
\pgfsetfillcolor{currentfill}%
\pgfsetfillopacity{0.800000}%
\pgfsetlinewidth{0.000000pt}%
\definecolor{currentstroke}{rgb}{0.000000,0.000000,0.000000}%
\pgfsetstrokecolor{currentstroke}%
\pgfsetstrokeopacity{0.800000}%
\pgfsetdash{}{0pt}%
\pgfpathmoveto{\pgfqpoint{1.450379in}{0.440000in}}%
\pgfpathlineto{\pgfqpoint{1.614773in}{0.440000in}}%
\pgfpathlineto{\pgfqpoint{1.614773in}{0.479687in}}%
\pgfpathlineto{\pgfqpoint{1.450379in}{0.479687in}}%
\pgfpathlineto{\pgfqpoint{1.450379in}{0.440000in}}%
\pgfpathclose%
\pgfusepath{fill}%
\end{pgfscope}%
\begin{pgfscope}%
\pgfpathrectangle{\pgfqpoint{0.875000in}{0.440000in}}{\pgfqpoint{5.425000in}{3.080000in}}%
\pgfusepath{clip}%
\pgfsetbuttcap%
\pgfsetmiterjoin%
\definecolor{currentfill}{rgb}{0.298039,0.447059,0.690196}%
\pgfsetfillcolor{currentfill}%
\pgfsetfillopacity{0.800000}%
\pgfsetlinewidth{0.000000pt}%
\definecolor{currentstroke}{rgb}{0.000000,0.000000,0.000000}%
\pgfsetstrokecolor{currentstroke}%
\pgfsetstrokeopacity{0.800000}%
\pgfsetdash{}{0pt}%
\pgfpathmoveto{\pgfqpoint{1.614773in}{0.440000in}}%
\pgfpathlineto{\pgfqpoint{1.779167in}{0.440000in}}%
\pgfpathlineto{\pgfqpoint{1.779167in}{0.605805in}}%
\pgfpathlineto{\pgfqpoint{1.614773in}{0.605805in}}%
\pgfpathlineto{\pgfqpoint{1.614773in}{0.440000in}}%
\pgfpathclose%
\pgfusepath{fill}%
\end{pgfscope}%
\begin{pgfscope}%
\pgfpathrectangle{\pgfqpoint{0.875000in}{0.440000in}}{\pgfqpoint{5.425000in}{3.080000in}}%
\pgfusepath{clip}%
\pgfsetbuttcap%
\pgfsetmiterjoin%
\definecolor{currentfill}{rgb}{0.298039,0.447059,0.690196}%
\pgfsetfillcolor{currentfill}%
\pgfsetfillopacity{0.800000}%
\pgfsetlinewidth{0.000000pt}%
\definecolor{currentstroke}{rgb}{0.000000,0.000000,0.000000}%
\pgfsetstrokecolor{currentstroke}%
\pgfsetstrokeopacity{0.800000}%
\pgfsetdash{}{0pt}%
\pgfpathmoveto{\pgfqpoint{1.779167in}{0.440000in}}%
\pgfpathlineto{\pgfqpoint{1.943561in}{0.440000in}}%
\pgfpathlineto{\pgfqpoint{1.943561in}{0.679888in}}%
\pgfpathlineto{\pgfqpoint{1.779167in}{0.679888in}}%
\pgfpathlineto{\pgfqpoint{1.779167in}{0.440000in}}%
\pgfpathclose%
\pgfusepath{fill}%
\end{pgfscope}%
\begin{pgfscope}%
\pgfpathrectangle{\pgfqpoint{0.875000in}{0.440000in}}{\pgfqpoint{5.425000in}{3.080000in}}%
\pgfusepath{clip}%
\pgfsetbuttcap%
\pgfsetmiterjoin%
\definecolor{currentfill}{rgb}{0.298039,0.447059,0.690196}%
\pgfsetfillcolor{currentfill}%
\pgfsetfillopacity{0.800000}%
\pgfsetlinewidth{0.000000pt}%
\definecolor{currentstroke}{rgb}{0.000000,0.000000,0.000000}%
\pgfsetstrokecolor{currentstroke}%
\pgfsetstrokeopacity{0.800000}%
\pgfsetdash{}{0pt}%
\pgfpathmoveto{\pgfqpoint{1.943561in}{0.440000in}}%
\pgfpathlineto{\pgfqpoint{2.107955in}{0.440000in}}%
\pgfpathlineto{\pgfqpoint{2.107955in}{0.992095in}}%
\pgfpathlineto{\pgfqpoint{1.943561in}{0.992095in}}%
\pgfpathlineto{\pgfqpoint{1.943561in}{0.440000in}}%
\pgfpathclose%
\pgfusepath{fill}%
\end{pgfscope}%
\begin{pgfscope}%
\pgfpathrectangle{\pgfqpoint{0.875000in}{0.440000in}}{\pgfqpoint{5.425000in}{3.080000in}}%
\pgfusepath{clip}%
\pgfsetbuttcap%
\pgfsetmiterjoin%
\definecolor{currentfill}{rgb}{0.298039,0.447059,0.690196}%
\pgfsetfillcolor{currentfill}%
\pgfsetfillopacity{0.800000}%
\pgfsetlinewidth{0.000000pt}%
\definecolor{currentstroke}{rgb}{0.000000,0.000000,0.000000}%
\pgfsetstrokecolor{currentstroke}%
\pgfsetstrokeopacity{0.800000}%
\pgfsetdash{}{0pt}%
\pgfpathmoveto{\pgfqpoint{2.107955in}{0.440000in}}%
\pgfpathlineto{\pgfqpoint{2.272348in}{0.440000in}}%
\pgfpathlineto{\pgfqpoint{2.272348in}{0.956817in}}%
\pgfpathlineto{\pgfqpoint{2.107955in}{0.956817in}}%
\pgfpathlineto{\pgfqpoint{2.107955in}{0.440000in}}%
\pgfpathclose%
\pgfusepath{fill}%
\end{pgfscope}%
\begin{pgfscope}%
\pgfpathrectangle{\pgfqpoint{0.875000in}{0.440000in}}{\pgfqpoint{5.425000in}{3.080000in}}%
\pgfusepath{clip}%
\pgfsetbuttcap%
\pgfsetmiterjoin%
\definecolor{currentfill}{rgb}{0.298039,0.447059,0.690196}%
\pgfsetfillcolor{currentfill}%
\pgfsetfillopacity{0.800000}%
\pgfsetlinewidth{0.000000pt}%
\definecolor{currentstroke}{rgb}{0.000000,0.000000,0.000000}%
\pgfsetstrokecolor{currentstroke}%
\pgfsetstrokeopacity{0.800000}%
\pgfsetdash{}{0pt}%
\pgfpathmoveto{\pgfqpoint{2.272348in}{0.440000in}}%
\pgfpathlineto{\pgfqpoint{2.436742in}{0.440000in}}%
\pgfpathlineto{\pgfqpoint{2.436742in}{1.098809in}}%
\pgfpathlineto{\pgfqpoint{2.272348in}{1.098809in}}%
\pgfpathlineto{\pgfqpoint{2.272348in}{0.440000in}}%
\pgfpathclose%
\pgfusepath{fill}%
\end{pgfscope}%
\begin{pgfscope}%
\pgfpathrectangle{\pgfqpoint{0.875000in}{0.440000in}}{\pgfqpoint{5.425000in}{3.080000in}}%
\pgfusepath{clip}%
\pgfsetbuttcap%
\pgfsetmiterjoin%
\definecolor{currentfill}{rgb}{0.298039,0.447059,0.690196}%
\pgfsetfillcolor{currentfill}%
\pgfsetfillopacity{0.800000}%
\pgfsetlinewidth{0.000000pt}%
\definecolor{currentstroke}{rgb}{0.000000,0.000000,0.000000}%
\pgfsetstrokecolor{currentstroke}%
\pgfsetstrokeopacity{0.800000}%
\pgfsetdash{}{0pt}%
\pgfpathmoveto{\pgfqpoint{2.436742in}{0.440000in}}%
\pgfpathlineto{\pgfqpoint{2.601136in}{0.440000in}}%
\pgfpathlineto{\pgfqpoint{2.601136in}{1.412780in}}%
\pgfpathlineto{\pgfqpoint{2.436742in}{1.412780in}}%
\pgfpathlineto{\pgfqpoint{2.436742in}{0.440000in}}%
\pgfpathclose%
\pgfusepath{fill}%
\end{pgfscope}%
\begin{pgfscope}%
\pgfpathrectangle{\pgfqpoint{0.875000in}{0.440000in}}{\pgfqpoint{5.425000in}{3.080000in}}%
\pgfusepath{clip}%
\pgfsetbuttcap%
\pgfsetmiterjoin%
\definecolor{currentfill}{rgb}{0.298039,0.447059,0.690196}%
\pgfsetfillcolor{currentfill}%
\pgfsetfillopacity{0.800000}%
\pgfsetlinewidth{0.000000pt}%
\definecolor{currentstroke}{rgb}{0.000000,0.000000,0.000000}%
\pgfsetstrokecolor{currentstroke}%
\pgfsetstrokeopacity{0.800000}%
\pgfsetdash{}{0pt}%
\pgfpathmoveto{\pgfqpoint{2.601136in}{0.440000in}}%
\pgfpathlineto{\pgfqpoint{2.765530in}{0.440000in}}%
\pgfpathlineto{\pgfqpoint{2.765530in}{1.612099in}}%
\pgfpathlineto{\pgfqpoint{2.601136in}{1.612099in}}%
\pgfpathlineto{\pgfqpoint{2.601136in}{0.440000in}}%
\pgfpathclose%
\pgfusepath{fill}%
\end{pgfscope}%
\begin{pgfscope}%
\pgfpathrectangle{\pgfqpoint{0.875000in}{0.440000in}}{\pgfqpoint{5.425000in}{3.080000in}}%
\pgfusepath{clip}%
\pgfsetbuttcap%
\pgfsetmiterjoin%
\definecolor{currentfill}{rgb}{0.298039,0.447059,0.690196}%
\pgfsetfillcolor{currentfill}%
\pgfsetfillopacity{0.800000}%
\pgfsetlinewidth{0.000000pt}%
\definecolor{currentstroke}{rgb}{0.000000,0.000000,0.000000}%
\pgfsetstrokecolor{currentstroke}%
\pgfsetstrokeopacity{0.800000}%
\pgfsetdash{}{0pt}%
\pgfpathmoveto{\pgfqpoint{2.765530in}{0.440000in}}%
\pgfpathlineto{\pgfqpoint{2.929924in}{0.440000in}}%
\pgfpathlineto{\pgfqpoint{2.929924in}{1.835230in}}%
\pgfpathlineto{\pgfqpoint{2.765530in}{1.835230in}}%
\pgfpathlineto{\pgfqpoint{2.765530in}{0.440000in}}%
\pgfpathclose%
\pgfusepath{fill}%
\end{pgfscope}%
\begin{pgfscope}%
\pgfpathrectangle{\pgfqpoint{0.875000in}{0.440000in}}{\pgfqpoint{5.425000in}{3.080000in}}%
\pgfusepath{clip}%
\pgfsetbuttcap%
\pgfsetmiterjoin%
\definecolor{currentfill}{rgb}{0.298039,0.447059,0.690196}%
\pgfsetfillcolor{currentfill}%
\pgfsetfillopacity{0.800000}%
\pgfsetlinewidth{0.000000pt}%
\definecolor{currentstroke}{rgb}{0.000000,0.000000,0.000000}%
\pgfsetstrokecolor{currentstroke}%
\pgfsetstrokeopacity{0.800000}%
\pgfsetdash{}{0pt}%
\pgfpathmoveto{\pgfqpoint{2.929924in}{0.440000in}}%
\pgfpathlineto{\pgfqpoint{3.094318in}{0.440000in}}%
\pgfpathlineto{\pgfqpoint{3.094318in}{2.032784in}}%
\pgfpathlineto{\pgfqpoint{2.929924in}{2.032784in}}%
\pgfpathlineto{\pgfqpoint{2.929924in}{0.440000in}}%
\pgfpathclose%
\pgfusepath{fill}%
\end{pgfscope}%
\begin{pgfscope}%
\pgfpathrectangle{\pgfqpoint{0.875000in}{0.440000in}}{\pgfqpoint{5.425000in}{3.080000in}}%
\pgfusepath{clip}%
\pgfsetbuttcap%
\pgfsetmiterjoin%
\definecolor{currentfill}{rgb}{0.298039,0.447059,0.690196}%
\pgfsetfillcolor{currentfill}%
\pgfsetfillopacity{0.800000}%
\pgfsetlinewidth{0.000000pt}%
\definecolor{currentstroke}{rgb}{0.000000,0.000000,0.000000}%
\pgfsetstrokecolor{currentstroke}%
\pgfsetstrokeopacity{0.800000}%
\pgfsetdash{}{0pt}%
\pgfpathmoveto{\pgfqpoint{3.094318in}{0.440000in}}%
\pgfpathlineto{\pgfqpoint{3.258712in}{0.440000in}}%
\pgfpathlineto{\pgfqpoint{3.258712in}{2.045131in}}%
\pgfpathlineto{\pgfqpoint{3.094318in}{2.045131in}}%
\pgfpathlineto{\pgfqpoint{3.094318in}{0.440000in}}%
\pgfpathclose%
\pgfusepath{fill}%
\end{pgfscope}%
\begin{pgfscope}%
\pgfpathrectangle{\pgfqpoint{0.875000in}{0.440000in}}{\pgfqpoint{5.425000in}{3.080000in}}%
\pgfusepath{clip}%
\pgfsetbuttcap%
\pgfsetmiterjoin%
\definecolor{currentfill}{rgb}{0.298039,0.447059,0.690196}%
\pgfsetfillcolor{currentfill}%
\pgfsetfillopacity{0.800000}%
\pgfsetlinewidth{0.000000pt}%
\definecolor{currentstroke}{rgb}{0.000000,0.000000,0.000000}%
\pgfsetstrokecolor{currentstroke}%
\pgfsetstrokeopacity{0.800000}%
\pgfsetdash{}{0pt}%
\pgfpathmoveto{\pgfqpoint{3.258712in}{0.440000in}}%
\pgfpathlineto{\pgfqpoint{3.423106in}{0.440000in}}%
\pgfpathlineto{\pgfqpoint{3.423106in}{2.467581in}}%
\pgfpathlineto{\pgfqpoint{3.258712in}{2.467581in}}%
\pgfpathlineto{\pgfqpoint{3.258712in}{0.440000in}}%
\pgfpathclose%
\pgfusepath{fill}%
\end{pgfscope}%
\begin{pgfscope}%
\pgfpathrectangle{\pgfqpoint{0.875000in}{0.440000in}}{\pgfqpoint{5.425000in}{3.080000in}}%
\pgfusepath{clip}%
\pgfsetbuttcap%
\pgfsetmiterjoin%
\definecolor{currentfill}{rgb}{0.298039,0.447059,0.690196}%
\pgfsetfillcolor{currentfill}%
\pgfsetfillopacity{0.800000}%
\pgfsetlinewidth{0.000000pt}%
\definecolor{currentstroke}{rgb}{0.000000,0.000000,0.000000}%
\pgfsetstrokecolor{currentstroke}%
\pgfsetstrokeopacity{0.800000}%
\pgfsetdash{}{0pt}%
\pgfpathmoveto{\pgfqpoint{3.423106in}{0.440000in}}%
\pgfpathlineto{\pgfqpoint{3.587500in}{0.440000in}}%
\pgfpathlineto{\pgfqpoint{3.587500in}{2.583115in}}%
\pgfpathlineto{\pgfqpoint{3.423106in}{2.583115in}}%
\pgfpathlineto{\pgfqpoint{3.423106in}{0.440000in}}%
\pgfpathclose%
\pgfusepath{fill}%
\end{pgfscope}%
\begin{pgfscope}%
\pgfpathrectangle{\pgfqpoint{0.875000in}{0.440000in}}{\pgfqpoint{5.425000in}{3.080000in}}%
\pgfusepath{clip}%
\pgfsetbuttcap%
\pgfsetmiterjoin%
\definecolor{currentfill}{rgb}{0.298039,0.447059,0.690196}%
\pgfsetfillcolor{currentfill}%
\pgfsetfillopacity{0.800000}%
\pgfsetlinewidth{0.000000pt}%
\definecolor{currentstroke}{rgb}{0.000000,0.000000,0.000000}%
\pgfsetstrokecolor{currentstroke}%
\pgfsetstrokeopacity{0.800000}%
\pgfsetdash{}{0pt}%
\pgfpathmoveto{\pgfqpoint{3.587500in}{0.440000in}}%
\pgfpathlineto{\pgfqpoint{3.751894in}{0.440000in}}%
\pgfpathlineto{\pgfqpoint{3.751894in}{2.755093in}}%
\pgfpathlineto{\pgfqpoint{3.587500in}{2.755093in}}%
\pgfpathlineto{\pgfqpoint{3.587500in}{0.440000in}}%
\pgfpathclose%
\pgfusepath{fill}%
\end{pgfscope}%
\begin{pgfscope}%
\pgfpathrectangle{\pgfqpoint{0.875000in}{0.440000in}}{\pgfqpoint{5.425000in}{3.080000in}}%
\pgfusepath{clip}%
\pgfsetbuttcap%
\pgfsetmiterjoin%
\definecolor{currentfill}{rgb}{0.298039,0.447059,0.690196}%
\pgfsetfillcolor{currentfill}%
\pgfsetfillopacity{0.800000}%
\pgfsetlinewidth{0.000000pt}%
\definecolor{currentstroke}{rgb}{0.000000,0.000000,0.000000}%
\pgfsetstrokecolor{currentstroke}%
\pgfsetstrokeopacity{0.800000}%
\pgfsetdash{}{0pt}%
\pgfpathmoveto{\pgfqpoint{3.751894in}{0.440000in}}%
\pgfpathlineto{\pgfqpoint{3.916288in}{0.440000in}}%
\pgfpathlineto{\pgfqpoint{3.916288in}{2.772732in}}%
\pgfpathlineto{\pgfqpoint{3.751894in}{2.772732in}}%
\pgfpathlineto{\pgfqpoint{3.751894in}{0.440000in}}%
\pgfpathclose%
\pgfusepath{fill}%
\end{pgfscope}%
\begin{pgfscope}%
\pgfpathrectangle{\pgfqpoint{0.875000in}{0.440000in}}{\pgfqpoint{5.425000in}{3.080000in}}%
\pgfusepath{clip}%
\pgfsetbuttcap%
\pgfsetmiterjoin%
\definecolor{currentfill}{rgb}{0.298039,0.447059,0.690196}%
\pgfsetfillcolor{currentfill}%
\pgfsetfillopacity{0.800000}%
\pgfsetlinewidth{0.000000pt}%
\definecolor{currentstroke}{rgb}{0.000000,0.000000,0.000000}%
\pgfsetstrokecolor{currentstroke}%
\pgfsetstrokeopacity{0.800000}%
\pgfsetdash{}{0pt}%
\pgfpathmoveto{\pgfqpoint{3.916288in}{0.440000in}}%
\pgfpathlineto{\pgfqpoint{4.080682in}{0.440000in}}%
\pgfpathlineto{\pgfqpoint{4.080682in}{3.076119in}}%
\pgfpathlineto{\pgfqpoint{3.916288in}{3.076119in}}%
\pgfpathlineto{\pgfqpoint{3.916288in}{0.440000in}}%
\pgfpathclose%
\pgfusepath{fill}%
\end{pgfscope}%
\begin{pgfscope}%
\pgfpathrectangle{\pgfqpoint{0.875000in}{0.440000in}}{\pgfqpoint{5.425000in}{3.080000in}}%
\pgfusepath{clip}%
\pgfsetbuttcap%
\pgfsetmiterjoin%
\definecolor{currentfill}{rgb}{0.298039,0.447059,0.690196}%
\pgfsetfillcolor{currentfill}%
\pgfsetfillopacity{0.800000}%
\pgfsetlinewidth{0.000000pt}%
\definecolor{currentstroke}{rgb}{0.000000,0.000000,0.000000}%
\pgfsetstrokecolor{currentstroke}%
\pgfsetstrokeopacity{0.800000}%
\pgfsetdash{}{0pt}%
\pgfpathmoveto{\pgfqpoint{4.080682in}{0.440000in}}%
\pgfpathlineto{\pgfqpoint{4.245076in}{0.440000in}}%
\pgfpathlineto{\pgfqpoint{4.245076in}{3.160786in}}%
\pgfpathlineto{\pgfqpoint{4.080682in}{3.160786in}}%
\pgfpathlineto{\pgfqpoint{4.080682in}{0.440000in}}%
\pgfpathclose%
\pgfusepath{fill}%
\end{pgfscope}%
\begin{pgfscope}%
\pgfpathrectangle{\pgfqpoint{0.875000in}{0.440000in}}{\pgfqpoint{5.425000in}{3.080000in}}%
\pgfusepath{clip}%
\pgfsetbuttcap%
\pgfsetmiterjoin%
\definecolor{currentfill}{rgb}{0.298039,0.447059,0.690196}%
\pgfsetfillcolor{currentfill}%
\pgfsetfillopacity{0.800000}%
\pgfsetlinewidth{0.000000pt}%
\definecolor{currentstroke}{rgb}{0.000000,0.000000,0.000000}%
\pgfsetstrokecolor{currentstroke}%
\pgfsetstrokeopacity{0.800000}%
\pgfsetdash{}{0pt}%
\pgfpathmoveto{\pgfqpoint{4.245076in}{0.440000in}}%
\pgfpathlineto{\pgfqpoint{4.409470in}{0.440000in}}%
\pgfpathlineto{\pgfqpoint{4.409470in}{3.082293in}}%
\pgfpathlineto{\pgfqpoint{4.245076in}{3.082293in}}%
\pgfpathlineto{\pgfqpoint{4.245076in}{0.440000in}}%
\pgfpathclose%
\pgfusepath{fill}%
\end{pgfscope}%
\begin{pgfscope}%
\pgfpathrectangle{\pgfqpoint{0.875000in}{0.440000in}}{\pgfqpoint{5.425000in}{3.080000in}}%
\pgfusepath{clip}%
\pgfsetbuttcap%
\pgfsetmiterjoin%
\definecolor{currentfill}{rgb}{0.298039,0.447059,0.690196}%
\pgfsetfillcolor{currentfill}%
\pgfsetfillopacity{0.800000}%
\pgfsetlinewidth{0.000000pt}%
\definecolor{currentstroke}{rgb}{0.000000,0.000000,0.000000}%
\pgfsetstrokecolor{currentstroke}%
\pgfsetstrokeopacity{0.800000}%
\pgfsetdash{}{0pt}%
\pgfpathmoveto{\pgfqpoint{4.409470in}{0.440000in}}%
\pgfpathlineto{\pgfqpoint{4.573864in}{0.440000in}}%
\pgfpathlineto{\pgfqpoint{4.573864in}{3.373333in}}%
\pgfpathlineto{\pgfqpoint{4.409470in}{3.373333in}}%
\pgfpathlineto{\pgfqpoint{4.409470in}{0.440000in}}%
\pgfpathclose%
\pgfusepath{fill}%
\end{pgfscope}%
\begin{pgfscope}%
\pgfpathrectangle{\pgfqpoint{0.875000in}{0.440000in}}{\pgfqpoint{5.425000in}{3.080000in}}%
\pgfusepath{clip}%
\pgfsetbuttcap%
\pgfsetmiterjoin%
\definecolor{currentfill}{rgb}{0.298039,0.447059,0.690196}%
\pgfsetfillcolor{currentfill}%
\pgfsetfillopacity{0.800000}%
\pgfsetlinewidth{0.000000pt}%
\definecolor{currentstroke}{rgb}{0.000000,0.000000,0.000000}%
\pgfsetstrokecolor{currentstroke}%
\pgfsetstrokeopacity{0.800000}%
\pgfsetdash{}{0pt}%
\pgfpathmoveto{\pgfqpoint{4.573864in}{0.440000in}}%
\pgfpathlineto{\pgfqpoint{4.738258in}{0.440000in}}%
\pgfpathlineto{\pgfqpoint{4.738258in}{3.203119in}}%
\pgfpathlineto{\pgfqpoint{4.573864in}{3.203119in}}%
\pgfpathlineto{\pgfqpoint{4.573864in}{0.440000in}}%
\pgfpathclose%
\pgfusepath{fill}%
\end{pgfscope}%
\begin{pgfscope}%
\pgfpathrectangle{\pgfqpoint{0.875000in}{0.440000in}}{\pgfqpoint{5.425000in}{3.080000in}}%
\pgfusepath{clip}%
\pgfsetbuttcap%
\pgfsetmiterjoin%
\definecolor{currentfill}{rgb}{0.298039,0.447059,0.690196}%
\pgfsetfillcolor{currentfill}%
\pgfsetfillopacity{0.800000}%
\pgfsetlinewidth{0.000000pt}%
\definecolor{currentstroke}{rgb}{0.000000,0.000000,0.000000}%
\pgfsetstrokecolor{currentstroke}%
\pgfsetstrokeopacity{0.800000}%
\pgfsetdash{}{0pt}%
\pgfpathmoveto{\pgfqpoint{4.738258in}{0.440000in}}%
\pgfpathlineto{\pgfqpoint{4.902652in}{0.440000in}}%
\pgfpathlineto{\pgfqpoint{4.902652in}{3.022321in}}%
\pgfpathlineto{\pgfqpoint{4.738258in}{3.022321in}}%
\pgfpathlineto{\pgfqpoint{4.738258in}{0.440000in}}%
\pgfpathclose%
\pgfusepath{fill}%
\end{pgfscope}%
\begin{pgfscope}%
\pgfpathrectangle{\pgfqpoint{0.875000in}{0.440000in}}{\pgfqpoint{5.425000in}{3.080000in}}%
\pgfusepath{clip}%
\pgfsetbuttcap%
\pgfsetmiterjoin%
\definecolor{currentfill}{rgb}{0.298039,0.447059,0.690196}%
\pgfsetfillcolor{currentfill}%
\pgfsetfillopacity{0.800000}%
\pgfsetlinewidth{0.000000pt}%
\definecolor{currentstroke}{rgb}{0.000000,0.000000,0.000000}%
\pgfsetstrokecolor{currentstroke}%
\pgfsetstrokeopacity{0.800000}%
\pgfsetdash{}{0pt}%
\pgfpathmoveto{\pgfqpoint{4.902652in}{0.440000in}}%
\pgfpathlineto{\pgfqpoint{5.067045in}{0.440000in}}%
\pgfpathlineto{\pgfqpoint{5.067045in}{2.796544in}}%
\pgfpathlineto{\pgfqpoint{4.902652in}{2.796544in}}%
\pgfpathlineto{\pgfqpoint{4.902652in}{0.440000in}}%
\pgfpathclose%
\pgfusepath{fill}%
\end{pgfscope}%
\begin{pgfscope}%
\pgfpathrectangle{\pgfqpoint{0.875000in}{0.440000in}}{\pgfqpoint{5.425000in}{3.080000in}}%
\pgfusepath{clip}%
\pgfsetbuttcap%
\pgfsetmiterjoin%
\definecolor{currentfill}{rgb}{0.298039,0.447059,0.690196}%
\pgfsetfillcolor{currentfill}%
\pgfsetfillopacity{0.800000}%
\pgfsetlinewidth{0.000000pt}%
\definecolor{currentstroke}{rgb}{0.000000,0.000000,0.000000}%
\pgfsetstrokecolor{currentstroke}%
\pgfsetstrokeopacity{0.800000}%
\pgfsetdash{}{0pt}%
\pgfpathmoveto{\pgfqpoint{5.067045in}{0.440000in}}%
\pgfpathlineto{\pgfqpoint{5.231439in}{0.440000in}}%
\pgfpathlineto{\pgfqpoint{5.231439in}{2.559302in}}%
\pgfpathlineto{\pgfqpoint{5.067045in}{2.559302in}}%
\pgfpathlineto{\pgfqpoint{5.067045in}{0.440000in}}%
\pgfpathclose%
\pgfusepath{fill}%
\end{pgfscope}%
\begin{pgfscope}%
\pgfpathrectangle{\pgfqpoint{0.875000in}{0.440000in}}{\pgfqpoint{5.425000in}{3.080000in}}%
\pgfusepath{clip}%
\pgfsetbuttcap%
\pgfsetmiterjoin%
\definecolor{currentfill}{rgb}{0.298039,0.447059,0.690196}%
\pgfsetfillcolor{currentfill}%
\pgfsetfillopacity{0.800000}%
\pgfsetlinewidth{0.000000pt}%
\definecolor{currentstroke}{rgb}{0.000000,0.000000,0.000000}%
\pgfsetstrokecolor{currentstroke}%
\pgfsetstrokeopacity{0.800000}%
\pgfsetdash{}{0pt}%
\pgfpathmoveto{\pgfqpoint{5.231439in}{0.440000in}}%
\pgfpathlineto{\pgfqpoint{5.395833in}{0.440000in}}%
\pgfpathlineto{\pgfqpoint{5.395833in}{2.218874in}}%
\pgfpathlineto{\pgfqpoint{5.231439in}{2.218874in}}%
\pgfpathlineto{\pgfqpoint{5.231439in}{0.440000in}}%
\pgfpathclose%
\pgfusepath{fill}%
\end{pgfscope}%
\begin{pgfscope}%
\pgfpathrectangle{\pgfqpoint{0.875000in}{0.440000in}}{\pgfqpoint{5.425000in}{3.080000in}}%
\pgfusepath{clip}%
\pgfsetbuttcap%
\pgfsetmiterjoin%
\definecolor{currentfill}{rgb}{0.298039,0.447059,0.690196}%
\pgfsetfillcolor{currentfill}%
\pgfsetfillopacity{0.800000}%
\pgfsetlinewidth{0.000000pt}%
\definecolor{currentstroke}{rgb}{0.000000,0.000000,0.000000}%
\pgfsetstrokecolor{currentstroke}%
\pgfsetstrokeopacity{0.800000}%
\pgfsetdash{}{0pt}%
\pgfpathmoveto{\pgfqpoint{5.395833in}{0.440000in}}%
\pgfpathlineto{\pgfqpoint{5.560227in}{0.440000in}}%
\pgfpathlineto{\pgfqpoint{5.560227in}{2.031020in}}%
\pgfpathlineto{\pgfqpoint{5.395833in}{2.031020in}}%
\pgfpathlineto{\pgfqpoint{5.395833in}{0.440000in}}%
\pgfpathclose%
\pgfusepath{fill}%
\end{pgfscope}%
\begin{pgfscope}%
\pgfpathrectangle{\pgfqpoint{0.875000in}{0.440000in}}{\pgfqpoint{5.425000in}{3.080000in}}%
\pgfusepath{clip}%
\pgfsetbuttcap%
\pgfsetmiterjoin%
\definecolor{currentfill}{rgb}{0.298039,0.447059,0.690196}%
\pgfsetfillcolor{currentfill}%
\pgfsetfillopacity{0.800000}%
\pgfsetlinewidth{0.000000pt}%
\definecolor{currentstroke}{rgb}{0.000000,0.000000,0.000000}%
\pgfsetstrokecolor{currentstroke}%
\pgfsetstrokeopacity{0.800000}%
\pgfsetdash{}{0pt}%
\pgfpathmoveto{\pgfqpoint{5.560227in}{0.440000in}}%
\pgfpathlineto{\pgfqpoint{5.724621in}{0.440000in}}%
\pgfpathlineto{\pgfqpoint{5.724621in}{1.652668in}}%
\pgfpathlineto{\pgfqpoint{5.560227in}{1.652668in}}%
\pgfpathlineto{\pgfqpoint{5.560227in}{0.440000in}}%
\pgfpathclose%
\pgfusepath{fill}%
\end{pgfscope}%
\begin{pgfscope}%
\pgfpathrectangle{\pgfqpoint{0.875000in}{0.440000in}}{\pgfqpoint{5.425000in}{3.080000in}}%
\pgfusepath{clip}%
\pgfsetbuttcap%
\pgfsetmiterjoin%
\definecolor{currentfill}{rgb}{0.298039,0.447059,0.690196}%
\pgfsetfillcolor{currentfill}%
\pgfsetfillopacity{0.800000}%
\pgfsetlinewidth{0.000000pt}%
\definecolor{currentstroke}{rgb}{0.000000,0.000000,0.000000}%
\pgfsetstrokecolor{currentstroke}%
\pgfsetstrokeopacity{0.800000}%
\pgfsetdash{}{0pt}%
\pgfpathmoveto{\pgfqpoint{5.724621in}{0.440000in}}%
\pgfpathlineto{\pgfqpoint{5.889015in}{0.440000in}}%
\pgfpathlineto{\pgfqpoint{5.889015in}{1.118212in}}%
\pgfpathlineto{\pgfqpoint{5.724621in}{1.118212in}}%
\pgfpathlineto{\pgfqpoint{5.724621in}{0.440000in}}%
\pgfpathclose%
\pgfusepath{fill}%
\end{pgfscope}%
\begin{pgfscope}%
\pgfpathrectangle{\pgfqpoint{0.875000in}{0.440000in}}{\pgfqpoint{5.425000in}{3.080000in}}%
\pgfusepath{clip}%
\pgfsetbuttcap%
\pgfsetmiterjoin%
\definecolor{currentfill}{rgb}{0.298039,0.447059,0.690196}%
\pgfsetfillcolor{currentfill}%
\pgfsetfillopacity{0.800000}%
\pgfsetlinewidth{0.000000pt}%
\definecolor{currentstroke}{rgb}{0.000000,0.000000,0.000000}%
\pgfsetstrokecolor{currentstroke}%
\pgfsetstrokeopacity{0.800000}%
\pgfsetdash{}{0pt}%
\pgfpathmoveto{\pgfqpoint{5.889015in}{0.440000in}}%
\pgfpathlineto{\pgfqpoint{6.053409in}{0.440000in}}%
\pgfpathlineto{\pgfqpoint{6.053409in}{0.688707in}}%
\pgfpathlineto{\pgfqpoint{5.889015in}{0.688707in}}%
\pgfpathlineto{\pgfqpoint{5.889015in}{0.440000in}}%
\pgfpathclose%
\pgfusepath{fill}%
\end{pgfscope}%
\begin{pgfscope}%
\pgfsetrectcap%
\pgfsetmiterjoin%
\pgfsetlinewidth{0.000000pt}%
\definecolor{currentstroke}{rgb}{1.000000,1.000000,1.000000}%
\pgfsetstrokecolor{currentstroke}%
\pgfsetdash{}{0pt}%
\pgfpathmoveto{\pgfqpoint{0.875000in}{0.440000in}}%
\pgfpathlineto{\pgfqpoint{0.875000in}{3.520000in}}%
\pgfusepath{}%
\end{pgfscope}%
\begin{pgfscope}%
\pgfsetrectcap%
\pgfsetmiterjoin%
\pgfsetlinewidth{0.000000pt}%
\definecolor{currentstroke}{rgb}{1.000000,1.000000,1.000000}%
\pgfsetstrokecolor{currentstroke}%
\pgfsetdash{}{0pt}%
\pgfpathmoveto{\pgfqpoint{6.300000in}{0.440000in}}%
\pgfpathlineto{\pgfqpoint{6.300000in}{3.520000in}}%
\pgfusepath{}%
\end{pgfscope}%
\begin{pgfscope}%
\pgfsetrectcap%
\pgfsetmiterjoin%
\pgfsetlinewidth{0.000000pt}%
\definecolor{currentstroke}{rgb}{1.000000,1.000000,1.000000}%
\pgfsetstrokecolor{currentstroke}%
\pgfsetdash{}{0pt}%
\pgfpathmoveto{\pgfqpoint{0.875000in}{0.440000in}}%
\pgfpathlineto{\pgfqpoint{6.300000in}{0.440000in}}%
\pgfusepath{}%
\end{pgfscope}%
\begin{pgfscope}%
\pgfsetrectcap%
\pgfsetmiterjoin%
\pgfsetlinewidth{0.000000pt}%
\definecolor{currentstroke}{rgb}{1.000000,1.000000,1.000000}%
\pgfsetstrokecolor{currentstroke}%
\pgfsetdash{}{0pt}%
\pgfpathmoveto{\pgfqpoint{0.875000in}{3.520000in}}%
\pgfpathlineto{\pgfqpoint{6.300000in}{3.520000in}}%
\pgfusepath{}%
\end{pgfscope}%
\begin{pgfscope}%
\definecolor{textcolor}{rgb}{0.150000,0.150000,0.150000}%
\pgfsetstrokecolor{textcolor}%
\pgfsetfillcolor{textcolor}%
\pgftext[x=3.500000in,y=3.920000in,,top]{\color{textcolor}\rmfamily\fontsize{12.000000}{14.400000}\selectfont Histograma para la muestra con propuesta beta y \(\displaystyle n=5, r=\)3}%
\end{pgfscope}%
\end{pgfpicture}%
\makeatother%
\endgroup%

        %% Creator: Matplotlib, PGF backend
%%
%% To include the figure in your LaTeX document, write
%%   \input{<filename>.pgf}
%%
%% Make sure the required packages are loaded in your preamble
%%   \usepackage{pgf}
%%
%% Also ensure that all the required font packages are loaded; for instance,
%% the lmodern package is sometimes necessary when using math font.
%%   \usepackage{lmodern}
%%
%% Figures using additional raster images can only be included by \input if
%% they are in the same directory as the main LaTeX file. For loading figures
%% from other directories you can use the `import` package
%%   \usepackage{import}
%%
%% and then include the figures with
%%   \import{<path to file>}{<filename>.pgf}
%%
%% Matplotlib used the following preamble
%%   
%%   \makeatletter\@ifpackageloaded{underscore}{}{\usepackage[strings]{underscore}}\makeatother
%%
\begingroup%
\makeatletter%
\begin{pgfpicture}%
\pgfpathrectangle{\pgfpointorigin}{\pgfqpoint{7.000000in}{4.000000in}}%
\pgfusepath{use as bounding box, clip}%
\begin{pgfscope}%
\pgfsetbuttcap%
\pgfsetmiterjoin%
\definecolor{currentfill}{rgb}{1.000000,1.000000,1.000000}%
\pgfsetfillcolor{currentfill}%
\pgfsetlinewidth{0.000000pt}%
\definecolor{currentstroke}{rgb}{1.000000,1.000000,1.000000}%
\pgfsetstrokecolor{currentstroke}%
\pgfsetdash{}{0pt}%
\pgfpathmoveto{\pgfqpoint{0.000000in}{0.000000in}}%
\pgfpathlineto{\pgfqpoint{7.000000in}{0.000000in}}%
\pgfpathlineto{\pgfqpoint{7.000000in}{4.000000in}}%
\pgfpathlineto{\pgfqpoint{0.000000in}{4.000000in}}%
\pgfpathlineto{\pgfqpoint{0.000000in}{0.000000in}}%
\pgfpathclose%
\pgfusepath{fill}%
\end{pgfscope}%
\begin{pgfscope}%
\pgfsetbuttcap%
\pgfsetmiterjoin%
\definecolor{currentfill}{rgb}{0.917647,0.917647,0.949020}%
\pgfsetfillcolor{currentfill}%
\pgfsetlinewidth{0.000000pt}%
\definecolor{currentstroke}{rgb}{0.000000,0.000000,0.000000}%
\pgfsetstrokecolor{currentstroke}%
\pgfsetstrokeopacity{0.000000}%
\pgfsetdash{}{0pt}%
\pgfpathmoveto{\pgfqpoint{0.875000in}{0.440000in}}%
\pgfpathlineto{\pgfqpoint{6.300000in}{0.440000in}}%
\pgfpathlineto{\pgfqpoint{6.300000in}{3.520000in}}%
\pgfpathlineto{\pgfqpoint{0.875000in}{3.520000in}}%
\pgfpathlineto{\pgfqpoint{0.875000in}{0.440000in}}%
\pgfpathclose%
\pgfusepath{fill}%
\end{pgfscope}%
\begin{pgfscope}%
\pgfpathrectangle{\pgfqpoint{0.875000in}{0.440000in}}{\pgfqpoint{5.425000in}{3.080000in}}%
\pgfusepath{clip}%
\pgfsetroundcap%
\pgfsetroundjoin%
\pgfsetlinewidth{1.003750pt}%
\definecolor{currentstroke}{rgb}{1.000000,1.000000,1.000000}%
\pgfsetstrokecolor{currentstroke}%
\pgfsetdash{}{0pt}%
\pgfpathmoveto{\pgfqpoint{0.924088in}{0.440000in}}%
\pgfpathlineto{\pgfqpoint{0.924088in}{3.520000in}}%
\pgfusepath{stroke}%
\end{pgfscope}%
\begin{pgfscope}%
\definecolor{textcolor}{rgb}{0.150000,0.150000,0.150000}%
\pgfsetstrokecolor{textcolor}%
\pgfsetfillcolor{textcolor}%
\pgftext[x=0.924088in,y=0.342778in,,top]{\color{textcolor}\rmfamily\fontsize{10.000000}{12.000000}\selectfont \(\displaystyle {0.0}\)}%
\end{pgfscope}%
\begin{pgfscope}%
\pgfpathrectangle{\pgfqpoint{0.875000in}{0.440000in}}{\pgfqpoint{5.425000in}{3.080000in}}%
\pgfusepath{clip}%
\pgfsetroundcap%
\pgfsetroundjoin%
\pgfsetlinewidth{1.003750pt}%
\definecolor{currentstroke}{rgb}{1.000000,1.000000,1.000000}%
\pgfsetstrokecolor{currentstroke}%
\pgfsetdash{}{0pt}%
\pgfpathmoveto{\pgfqpoint{1.951188in}{0.440000in}}%
\pgfpathlineto{\pgfqpoint{1.951188in}{3.520000in}}%
\pgfusepath{stroke}%
\end{pgfscope}%
\begin{pgfscope}%
\definecolor{textcolor}{rgb}{0.150000,0.150000,0.150000}%
\pgfsetstrokecolor{textcolor}%
\pgfsetfillcolor{textcolor}%
\pgftext[x=1.951188in,y=0.342778in,,top]{\color{textcolor}\rmfamily\fontsize{10.000000}{12.000000}\selectfont \(\displaystyle {0.1}\)}%
\end{pgfscope}%
\begin{pgfscope}%
\pgfpathrectangle{\pgfqpoint{0.875000in}{0.440000in}}{\pgfqpoint{5.425000in}{3.080000in}}%
\pgfusepath{clip}%
\pgfsetroundcap%
\pgfsetroundjoin%
\pgfsetlinewidth{1.003750pt}%
\definecolor{currentstroke}{rgb}{1.000000,1.000000,1.000000}%
\pgfsetstrokecolor{currentstroke}%
\pgfsetdash{}{0pt}%
\pgfpathmoveto{\pgfqpoint{2.978288in}{0.440000in}}%
\pgfpathlineto{\pgfqpoint{2.978288in}{3.520000in}}%
\pgfusepath{stroke}%
\end{pgfscope}%
\begin{pgfscope}%
\definecolor{textcolor}{rgb}{0.150000,0.150000,0.150000}%
\pgfsetstrokecolor{textcolor}%
\pgfsetfillcolor{textcolor}%
\pgftext[x=2.978288in,y=0.342778in,,top]{\color{textcolor}\rmfamily\fontsize{10.000000}{12.000000}\selectfont \(\displaystyle {0.2}\)}%
\end{pgfscope}%
\begin{pgfscope}%
\pgfpathrectangle{\pgfqpoint{0.875000in}{0.440000in}}{\pgfqpoint{5.425000in}{3.080000in}}%
\pgfusepath{clip}%
\pgfsetroundcap%
\pgfsetroundjoin%
\pgfsetlinewidth{1.003750pt}%
\definecolor{currentstroke}{rgb}{1.000000,1.000000,1.000000}%
\pgfsetstrokecolor{currentstroke}%
\pgfsetdash{}{0pt}%
\pgfpathmoveto{\pgfqpoint{4.005387in}{0.440000in}}%
\pgfpathlineto{\pgfqpoint{4.005387in}{3.520000in}}%
\pgfusepath{stroke}%
\end{pgfscope}%
\begin{pgfscope}%
\definecolor{textcolor}{rgb}{0.150000,0.150000,0.150000}%
\pgfsetstrokecolor{textcolor}%
\pgfsetfillcolor{textcolor}%
\pgftext[x=4.005387in,y=0.342778in,,top]{\color{textcolor}\rmfamily\fontsize{10.000000}{12.000000}\selectfont \(\displaystyle {0.3}\)}%
\end{pgfscope}%
\begin{pgfscope}%
\pgfpathrectangle{\pgfqpoint{0.875000in}{0.440000in}}{\pgfqpoint{5.425000in}{3.080000in}}%
\pgfusepath{clip}%
\pgfsetroundcap%
\pgfsetroundjoin%
\pgfsetlinewidth{1.003750pt}%
\definecolor{currentstroke}{rgb}{1.000000,1.000000,1.000000}%
\pgfsetstrokecolor{currentstroke}%
\pgfsetdash{}{0pt}%
\pgfpathmoveto{\pgfqpoint{5.032487in}{0.440000in}}%
\pgfpathlineto{\pgfqpoint{5.032487in}{3.520000in}}%
\pgfusepath{stroke}%
\end{pgfscope}%
\begin{pgfscope}%
\definecolor{textcolor}{rgb}{0.150000,0.150000,0.150000}%
\pgfsetstrokecolor{textcolor}%
\pgfsetfillcolor{textcolor}%
\pgftext[x=5.032487in,y=0.342778in,,top]{\color{textcolor}\rmfamily\fontsize{10.000000}{12.000000}\selectfont \(\displaystyle {0.4}\)}%
\end{pgfscope}%
\begin{pgfscope}%
\pgfpathrectangle{\pgfqpoint{0.875000in}{0.440000in}}{\pgfqpoint{5.425000in}{3.080000in}}%
\pgfusepath{clip}%
\pgfsetroundcap%
\pgfsetroundjoin%
\pgfsetlinewidth{1.003750pt}%
\definecolor{currentstroke}{rgb}{1.000000,1.000000,1.000000}%
\pgfsetstrokecolor{currentstroke}%
\pgfsetdash{}{0pt}%
\pgfpathmoveto{\pgfqpoint{6.059587in}{0.440000in}}%
\pgfpathlineto{\pgfqpoint{6.059587in}{3.520000in}}%
\pgfusepath{stroke}%
\end{pgfscope}%
\begin{pgfscope}%
\definecolor{textcolor}{rgb}{0.150000,0.150000,0.150000}%
\pgfsetstrokecolor{textcolor}%
\pgfsetfillcolor{textcolor}%
\pgftext[x=6.059587in,y=0.342778in,,top]{\color{textcolor}\rmfamily\fontsize{10.000000}{12.000000}\selectfont \(\displaystyle {0.5}\)}%
\end{pgfscope}%
\begin{pgfscope}%
\definecolor{textcolor}{rgb}{0.150000,0.150000,0.150000}%
\pgfsetstrokecolor{textcolor}%
\pgfsetfillcolor{textcolor}%
\pgftext[x=3.587500in,y=0.163766in,,top]{\color{textcolor}\rmfamily\fontsize{11.000000}{13.200000}\selectfont Valor}%
\end{pgfscope}%
\begin{pgfscope}%
\pgfpathrectangle{\pgfqpoint{0.875000in}{0.440000in}}{\pgfqpoint{5.425000in}{3.080000in}}%
\pgfusepath{clip}%
\pgfsetroundcap%
\pgfsetroundjoin%
\pgfsetlinewidth{1.003750pt}%
\definecolor{currentstroke}{rgb}{1.000000,1.000000,1.000000}%
\pgfsetstrokecolor{currentstroke}%
\pgfsetdash{}{0pt}%
\pgfpathmoveto{\pgfqpoint{0.875000in}{0.440000in}}%
\pgfpathlineto{\pgfqpoint{6.300000in}{0.440000in}}%
\pgfusepath{stroke}%
\end{pgfscope}%
\begin{pgfscope}%
\definecolor{textcolor}{rgb}{0.150000,0.150000,0.150000}%
\pgfsetstrokecolor{textcolor}%
\pgfsetfillcolor{textcolor}%
\pgftext[x=0.708333in, y=0.391775in, left, base]{\color{textcolor}\rmfamily\fontsize{10.000000}{12.000000}\selectfont \(\displaystyle {0}\)}%
\end{pgfscope}%
\begin{pgfscope}%
\pgfpathrectangle{\pgfqpoint{0.875000in}{0.440000in}}{\pgfqpoint{5.425000in}{3.080000in}}%
\pgfusepath{clip}%
\pgfsetroundcap%
\pgfsetroundjoin%
\pgfsetlinewidth{1.003750pt}%
\definecolor{currentstroke}{rgb}{1.000000,1.000000,1.000000}%
\pgfsetstrokecolor{currentstroke}%
\pgfsetdash{}{0pt}%
\pgfpathmoveto{\pgfqpoint{0.875000in}{0.894598in}}%
\pgfpathlineto{\pgfqpoint{6.300000in}{0.894598in}}%
\pgfusepath{stroke}%
\end{pgfscope}%
\begin{pgfscope}%
\definecolor{textcolor}{rgb}{0.150000,0.150000,0.150000}%
\pgfsetstrokecolor{textcolor}%
\pgfsetfillcolor{textcolor}%
\pgftext[x=0.708333in, y=0.846372in, left, base]{\color{textcolor}\rmfamily\fontsize{10.000000}{12.000000}\selectfont \(\displaystyle {1}\)}%
\end{pgfscope}%
\begin{pgfscope}%
\pgfpathrectangle{\pgfqpoint{0.875000in}{0.440000in}}{\pgfqpoint{5.425000in}{3.080000in}}%
\pgfusepath{clip}%
\pgfsetroundcap%
\pgfsetroundjoin%
\pgfsetlinewidth{1.003750pt}%
\definecolor{currentstroke}{rgb}{1.000000,1.000000,1.000000}%
\pgfsetstrokecolor{currentstroke}%
\pgfsetdash{}{0pt}%
\pgfpathmoveto{\pgfqpoint{0.875000in}{1.349195in}}%
\pgfpathlineto{\pgfqpoint{6.300000in}{1.349195in}}%
\pgfusepath{stroke}%
\end{pgfscope}%
\begin{pgfscope}%
\definecolor{textcolor}{rgb}{0.150000,0.150000,0.150000}%
\pgfsetstrokecolor{textcolor}%
\pgfsetfillcolor{textcolor}%
\pgftext[x=0.708333in, y=1.300970in, left, base]{\color{textcolor}\rmfamily\fontsize{10.000000}{12.000000}\selectfont \(\displaystyle {2}\)}%
\end{pgfscope}%
\begin{pgfscope}%
\pgfpathrectangle{\pgfqpoint{0.875000in}{0.440000in}}{\pgfqpoint{5.425000in}{3.080000in}}%
\pgfusepath{clip}%
\pgfsetroundcap%
\pgfsetroundjoin%
\pgfsetlinewidth{1.003750pt}%
\definecolor{currentstroke}{rgb}{1.000000,1.000000,1.000000}%
\pgfsetstrokecolor{currentstroke}%
\pgfsetdash{}{0pt}%
\pgfpathmoveto{\pgfqpoint{0.875000in}{1.803793in}}%
\pgfpathlineto{\pgfqpoint{6.300000in}{1.803793in}}%
\pgfusepath{stroke}%
\end{pgfscope}%
\begin{pgfscope}%
\definecolor{textcolor}{rgb}{0.150000,0.150000,0.150000}%
\pgfsetstrokecolor{textcolor}%
\pgfsetfillcolor{textcolor}%
\pgftext[x=0.708333in, y=1.755567in, left, base]{\color{textcolor}\rmfamily\fontsize{10.000000}{12.000000}\selectfont \(\displaystyle {3}\)}%
\end{pgfscope}%
\begin{pgfscope}%
\pgfpathrectangle{\pgfqpoint{0.875000in}{0.440000in}}{\pgfqpoint{5.425000in}{3.080000in}}%
\pgfusepath{clip}%
\pgfsetroundcap%
\pgfsetroundjoin%
\pgfsetlinewidth{1.003750pt}%
\definecolor{currentstroke}{rgb}{1.000000,1.000000,1.000000}%
\pgfsetstrokecolor{currentstroke}%
\pgfsetdash{}{0pt}%
\pgfpathmoveto{\pgfqpoint{0.875000in}{2.258390in}}%
\pgfpathlineto{\pgfqpoint{6.300000in}{2.258390in}}%
\pgfusepath{stroke}%
\end{pgfscope}%
\begin{pgfscope}%
\definecolor{textcolor}{rgb}{0.150000,0.150000,0.150000}%
\pgfsetstrokecolor{textcolor}%
\pgfsetfillcolor{textcolor}%
\pgftext[x=0.708333in, y=2.210165in, left, base]{\color{textcolor}\rmfamily\fontsize{10.000000}{12.000000}\selectfont \(\displaystyle {4}\)}%
\end{pgfscope}%
\begin{pgfscope}%
\pgfpathrectangle{\pgfqpoint{0.875000in}{0.440000in}}{\pgfqpoint{5.425000in}{3.080000in}}%
\pgfusepath{clip}%
\pgfsetroundcap%
\pgfsetroundjoin%
\pgfsetlinewidth{1.003750pt}%
\definecolor{currentstroke}{rgb}{1.000000,1.000000,1.000000}%
\pgfsetstrokecolor{currentstroke}%
\pgfsetdash{}{0pt}%
\pgfpathmoveto{\pgfqpoint{0.875000in}{2.712988in}}%
\pgfpathlineto{\pgfqpoint{6.300000in}{2.712988in}}%
\pgfusepath{stroke}%
\end{pgfscope}%
\begin{pgfscope}%
\definecolor{textcolor}{rgb}{0.150000,0.150000,0.150000}%
\pgfsetstrokecolor{textcolor}%
\pgfsetfillcolor{textcolor}%
\pgftext[x=0.708333in, y=2.664762in, left, base]{\color{textcolor}\rmfamily\fontsize{10.000000}{12.000000}\selectfont \(\displaystyle {5}\)}%
\end{pgfscope}%
\begin{pgfscope}%
\pgfpathrectangle{\pgfqpoint{0.875000in}{0.440000in}}{\pgfqpoint{5.425000in}{3.080000in}}%
\pgfusepath{clip}%
\pgfsetroundcap%
\pgfsetroundjoin%
\pgfsetlinewidth{1.003750pt}%
\definecolor{currentstroke}{rgb}{1.000000,1.000000,1.000000}%
\pgfsetstrokecolor{currentstroke}%
\pgfsetdash{}{0pt}%
\pgfpathmoveto{\pgfqpoint{0.875000in}{3.167585in}}%
\pgfpathlineto{\pgfqpoint{6.300000in}{3.167585in}}%
\pgfusepath{stroke}%
\end{pgfscope}%
\begin{pgfscope}%
\definecolor{textcolor}{rgb}{0.150000,0.150000,0.150000}%
\pgfsetstrokecolor{textcolor}%
\pgfsetfillcolor{textcolor}%
\pgftext[x=0.708333in, y=3.119360in, left, base]{\color{textcolor}\rmfamily\fontsize{10.000000}{12.000000}\selectfont \(\displaystyle {6}\)}%
\end{pgfscope}%
\begin{pgfscope}%
\definecolor{textcolor}{rgb}{0.150000,0.150000,0.150000}%
\pgfsetstrokecolor{textcolor}%
\pgfsetfillcolor{textcolor}%
\pgftext[x=0.652778in,y=1.980000in,,bottom,rotate=90.000000]{\color{textcolor}\rmfamily\fontsize{11.000000}{13.200000}\selectfont Frecuencia}%
\end{pgfscope}%
\begin{pgfscope}%
\pgfpathrectangle{\pgfqpoint{0.875000in}{0.440000in}}{\pgfqpoint{5.425000in}{3.080000in}}%
\pgfusepath{clip}%
\pgfsetbuttcap%
\pgfsetmiterjoin%
\definecolor{currentfill}{rgb}{0.298039,0.447059,0.690196}%
\pgfsetfillcolor{currentfill}%
\pgfsetfillopacity{0.800000}%
\pgfsetlinewidth{0.000000pt}%
\definecolor{currentstroke}{rgb}{0.000000,0.000000,0.000000}%
\pgfsetstrokecolor{currentstroke}%
\pgfsetstrokeopacity{0.800000}%
\pgfsetdash{}{0pt}%
\pgfpathmoveto{\pgfqpoint{1.121591in}{0.440000in}}%
\pgfpathlineto{\pgfqpoint{1.285985in}{0.440000in}}%
\pgfpathlineto{\pgfqpoint{1.285985in}{0.440568in}}%
\pgfpathlineto{\pgfqpoint{1.121591in}{0.440568in}}%
\pgfpathlineto{\pgfqpoint{1.121591in}{0.440000in}}%
\pgfpathclose%
\pgfusepath{fill}%
\end{pgfscope}%
\begin{pgfscope}%
\pgfpathrectangle{\pgfqpoint{0.875000in}{0.440000in}}{\pgfqpoint{5.425000in}{3.080000in}}%
\pgfusepath{clip}%
\pgfsetbuttcap%
\pgfsetmiterjoin%
\definecolor{currentfill}{rgb}{0.298039,0.447059,0.690196}%
\pgfsetfillcolor{currentfill}%
\pgfsetfillopacity{0.800000}%
\pgfsetlinewidth{0.000000pt}%
\definecolor{currentstroke}{rgb}{0.000000,0.000000,0.000000}%
\pgfsetstrokecolor{currentstroke}%
\pgfsetstrokeopacity{0.800000}%
\pgfsetdash{}{0pt}%
\pgfpathmoveto{\pgfqpoint{1.285985in}{0.440000in}}%
\pgfpathlineto{\pgfqpoint{1.450379in}{0.440000in}}%
\pgfpathlineto{\pgfqpoint{1.450379in}{0.440000in}}%
\pgfpathlineto{\pgfqpoint{1.285985in}{0.440000in}}%
\pgfpathlineto{\pgfqpoint{1.285985in}{0.440000in}}%
\pgfpathclose%
\pgfusepath{fill}%
\end{pgfscope}%
\begin{pgfscope}%
\pgfpathrectangle{\pgfqpoint{0.875000in}{0.440000in}}{\pgfqpoint{5.425000in}{3.080000in}}%
\pgfusepath{clip}%
\pgfsetbuttcap%
\pgfsetmiterjoin%
\definecolor{currentfill}{rgb}{0.298039,0.447059,0.690196}%
\pgfsetfillcolor{currentfill}%
\pgfsetfillopacity{0.800000}%
\pgfsetlinewidth{0.000000pt}%
\definecolor{currentstroke}{rgb}{0.000000,0.000000,0.000000}%
\pgfsetstrokecolor{currentstroke}%
\pgfsetstrokeopacity{0.800000}%
\pgfsetdash{}{0pt}%
\pgfpathmoveto{\pgfqpoint{1.450379in}{0.440000in}}%
\pgfpathlineto{\pgfqpoint{1.614773in}{0.440000in}}%
\pgfpathlineto{\pgfqpoint{1.614773in}{0.440000in}}%
\pgfpathlineto{\pgfqpoint{1.450379in}{0.440000in}}%
\pgfpathlineto{\pgfqpoint{1.450379in}{0.440000in}}%
\pgfpathclose%
\pgfusepath{fill}%
\end{pgfscope}%
\begin{pgfscope}%
\pgfpathrectangle{\pgfqpoint{0.875000in}{0.440000in}}{\pgfqpoint{5.425000in}{3.080000in}}%
\pgfusepath{clip}%
\pgfsetbuttcap%
\pgfsetmiterjoin%
\definecolor{currentfill}{rgb}{0.298039,0.447059,0.690196}%
\pgfsetfillcolor{currentfill}%
\pgfsetfillopacity{0.800000}%
\pgfsetlinewidth{0.000000pt}%
\definecolor{currentstroke}{rgb}{0.000000,0.000000,0.000000}%
\pgfsetstrokecolor{currentstroke}%
\pgfsetstrokeopacity{0.800000}%
\pgfsetdash{}{0pt}%
\pgfpathmoveto{\pgfqpoint{1.614773in}{0.440000in}}%
\pgfpathlineto{\pgfqpoint{1.779167in}{0.440000in}}%
\pgfpathlineto{\pgfqpoint{1.779167in}{0.440000in}}%
\pgfpathlineto{\pgfqpoint{1.614773in}{0.440000in}}%
\pgfpathlineto{\pgfqpoint{1.614773in}{0.440000in}}%
\pgfpathclose%
\pgfusepath{fill}%
\end{pgfscope}%
\begin{pgfscope}%
\pgfpathrectangle{\pgfqpoint{0.875000in}{0.440000in}}{\pgfqpoint{5.425000in}{3.080000in}}%
\pgfusepath{clip}%
\pgfsetbuttcap%
\pgfsetmiterjoin%
\definecolor{currentfill}{rgb}{0.298039,0.447059,0.690196}%
\pgfsetfillcolor{currentfill}%
\pgfsetfillopacity{0.800000}%
\pgfsetlinewidth{0.000000pt}%
\definecolor{currentstroke}{rgb}{0.000000,0.000000,0.000000}%
\pgfsetstrokecolor{currentstroke}%
\pgfsetstrokeopacity{0.800000}%
\pgfsetdash{}{0pt}%
\pgfpathmoveto{\pgfqpoint{1.779167in}{0.440000in}}%
\pgfpathlineto{\pgfqpoint{1.943561in}{0.440000in}}%
\pgfpathlineto{\pgfqpoint{1.943561in}{0.440000in}}%
\pgfpathlineto{\pgfqpoint{1.779167in}{0.440000in}}%
\pgfpathlineto{\pgfqpoint{1.779167in}{0.440000in}}%
\pgfpathclose%
\pgfusepath{fill}%
\end{pgfscope}%
\begin{pgfscope}%
\pgfpathrectangle{\pgfqpoint{0.875000in}{0.440000in}}{\pgfqpoint{5.425000in}{3.080000in}}%
\pgfusepath{clip}%
\pgfsetbuttcap%
\pgfsetmiterjoin%
\definecolor{currentfill}{rgb}{0.298039,0.447059,0.690196}%
\pgfsetfillcolor{currentfill}%
\pgfsetfillopacity{0.800000}%
\pgfsetlinewidth{0.000000pt}%
\definecolor{currentstroke}{rgb}{0.000000,0.000000,0.000000}%
\pgfsetstrokecolor{currentstroke}%
\pgfsetstrokeopacity{0.800000}%
\pgfsetdash{}{0pt}%
\pgfpathmoveto{\pgfqpoint{1.943561in}{0.440000in}}%
\pgfpathlineto{\pgfqpoint{2.107955in}{0.440000in}}%
\pgfpathlineto{\pgfqpoint{2.107955in}{0.440000in}}%
\pgfpathlineto{\pgfqpoint{1.943561in}{0.440000in}}%
\pgfpathlineto{\pgfqpoint{1.943561in}{0.440000in}}%
\pgfpathclose%
\pgfusepath{fill}%
\end{pgfscope}%
\begin{pgfscope}%
\pgfpathrectangle{\pgfqpoint{0.875000in}{0.440000in}}{\pgfqpoint{5.425000in}{3.080000in}}%
\pgfusepath{clip}%
\pgfsetbuttcap%
\pgfsetmiterjoin%
\definecolor{currentfill}{rgb}{0.298039,0.447059,0.690196}%
\pgfsetfillcolor{currentfill}%
\pgfsetfillopacity{0.800000}%
\pgfsetlinewidth{0.000000pt}%
\definecolor{currentstroke}{rgb}{0.000000,0.000000,0.000000}%
\pgfsetstrokecolor{currentstroke}%
\pgfsetstrokeopacity{0.800000}%
\pgfsetdash{}{0pt}%
\pgfpathmoveto{\pgfqpoint{2.107955in}{0.440000in}}%
\pgfpathlineto{\pgfqpoint{2.272348in}{0.440000in}}%
\pgfpathlineto{\pgfqpoint{2.272348in}{0.441136in}}%
\pgfpathlineto{\pgfqpoint{2.107955in}{0.441136in}}%
\pgfpathlineto{\pgfqpoint{2.107955in}{0.440000in}}%
\pgfpathclose%
\pgfusepath{fill}%
\end{pgfscope}%
\begin{pgfscope}%
\pgfpathrectangle{\pgfqpoint{0.875000in}{0.440000in}}{\pgfqpoint{5.425000in}{3.080000in}}%
\pgfusepath{clip}%
\pgfsetbuttcap%
\pgfsetmiterjoin%
\definecolor{currentfill}{rgb}{0.298039,0.447059,0.690196}%
\pgfsetfillcolor{currentfill}%
\pgfsetfillopacity{0.800000}%
\pgfsetlinewidth{0.000000pt}%
\definecolor{currentstroke}{rgb}{0.000000,0.000000,0.000000}%
\pgfsetstrokecolor{currentstroke}%
\pgfsetstrokeopacity{0.800000}%
\pgfsetdash{}{0pt}%
\pgfpathmoveto{\pgfqpoint{2.272348in}{0.440000in}}%
\pgfpathlineto{\pgfqpoint{2.436742in}{0.440000in}}%
\pgfpathlineto{\pgfqpoint{2.436742in}{0.451929in}}%
\pgfpathlineto{\pgfqpoint{2.272348in}{0.451929in}}%
\pgfpathlineto{\pgfqpoint{2.272348in}{0.440000in}}%
\pgfpathclose%
\pgfusepath{fill}%
\end{pgfscope}%
\begin{pgfscope}%
\pgfpathrectangle{\pgfqpoint{0.875000in}{0.440000in}}{\pgfqpoint{5.425000in}{3.080000in}}%
\pgfusepath{clip}%
\pgfsetbuttcap%
\pgfsetmiterjoin%
\definecolor{currentfill}{rgb}{0.298039,0.447059,0.690196}%
\pgfsetfillcolor{currentfill}%
\pgfsetfillopacity{0.800000}%
\pgfsetlinewidth{0.000000pt}%
\definecolor{currentstroke}{rgb}{0.000000,0.000000,0.000000}%
\pgfsetstrokecolor{currentstroke}%
\pgfsetstrokeopacity{0.800000}%
\pgfsetdash{}{0pt}%
\pgfpathmoveto{\pgfqpoint{2.436742in}{0.440000in}}%
\pgfpathlineto{\pgfqpoint{2.601136in}{0.440000in}}%
\pgfpathlineto{\pgfqpoint{2.601136in}{0.460449in}}%
\pgfpathlineto{\pgfqpoint{2.436742in}{0.460449in}}%
\pgfpathlineto{\pgfqpoint{2.436742in}{0.440000in}}%
\pgfpathclose%
\pgfusepath{fill}%
\end{pgfscope}%
\begin{pgfscope}%
\pgfpathrectangle{\pgfqpoint{0.875000in}{0.440000in}}{\pgfqpoint{5.425000in}{3.080000in}}%
\pgfusepath{clip}%
\pgfsetbuttcap%
\pgfsetmiterjoin%
\definecolor{currentfill}{rgb}{0.298039,0.447059,0.690196}%
\pgfsetfillcolor{currentfill}%
\pgfsetfillopacity{0.800000}%
\pgfsetlinewidth{0.000000pt}%
\definecolor{currentstroke}{rgb}{0.000000,0.000000,0.000000}%
\pgfsetstrokecolor{currentstroke}%
\pgfsetstrokeopacity{0.800000}%
\pgfsetdash{}{0pt}%
\pgfpathmoveto{\pgfqpoint{2.601136in}{0.440000in}}%
\pgfpathlineto{\pgfqpoint{2.765530in}{0.440000in}}%
\pgfpathlineto{\pgfqpoint{2.765530in}{0.493395in}}%
\pgfpathlineto{\pgfqpoint{2.601136in}{0.493395in}}%
\pgfpathlineto{\pgfqpoint{2.601136in}{0.440000in}}%
\pgfpathclose%
\pgfusepath{fill}%
\end{pgfscope}%
\begin{pgfscope}%
\pgfpathrectangle{\pgfqpoint{0.875000in}{0.440000in}}{\pgfqpoint{5.425000in}{3.080000in}}%
\pgfusepath{clip}%
\pgfsetbuttcap%
\pgfsetmiterjoin%
\definecolor{currentfill}{rgb}{0.298039,0.447059,0.690196}%
\pgfsetfillcolor{currentfill}%
\pgfsetfillopacity{0.800000}%
\pgfsetlinewidth{0.000000pt}%
\definecolor{currentstroke}{rgb}{0.000000,0.000000,0.000000}%
\pgfsetstrokecolor{currentstroke}%
\pgfsetstrokeopacity{0.800000}%
\pgfsetdash{}{0pt}%
\pgfpathmoveto{\pgfqpoint{2.765530in}{0.440000in}}%
\pgfpathlineto{\pgfqpoint{2.929924in}{0.440000in}}%
\pgfpathlineto{\pgfqpoint{2.929924in}{0.554743in}}%
\pgfpathlineto{\pgfqpoint{2.765530in}{0.554743in}}%
\pgfpathlineto{\pgfqpoint{2.765530in}{0.440000in}}%
\pgfpathclose%
\pgfusepath{fill}%
\end{pgfscope}%
\begin{pgfscope}%
\pgfpathrectangle{\pgfqpoint{0.875000in}{0.440000in}}{\pgfqpoint{5.425000in}{3.080000in}}%
\pgfusepath{clip}%
\pgfsetbuttcap%
\pgfsetmiterjoin%
\definecolor{currentfill}{rgb}{0.298039,0.447059,0.690196}%
\pgfsetfillcolor{currentfill}%
\pgfsetfillopacity{0.800000}%
\pgfsetlinewidth{0.000000pt}%
\definecolor{currentstroke}{rgb}{0.000000,0.000000,0.000000}%
\pgfsetstrokecolor{currentstroke}%
\pgfsetstrokeopacity{0.800000}%
\pgfsetdash{}{0pt}%
\pgfpathmoveto{\pgfqpoint{2.929924in}{0.440000in}}%
\pgfpathlineto{\pgfqpoint{3.094318in}{0.440000in}}%
\pgfpathlineto{\pgfqpoint{3.094318in}{0.726290in}}%
\pgfpathlineto{\pgfqpoint{2.929924in}{0.726290in}}%
\pgfpathlineto{\pgfqpoint{2.929924in}{0.440000in}}%
\pgfpathclose%
\pgfusepath{fill}%
\end{pgfscope}%
\begin{pgfscope}%
\pgfpathrectangle{\pgfqpoint{0.875000in}{0.440000in}}{\pgfqpoint{5.425000in}{3.080000in}}%
\pgfusepath{clip}%
\pgfsetbuttcap%
\pgfsetmiterjoin%
\definecolor{currentfill}{rgb}{0.298039,0.447059,0.690196}%
\pgfsetfillcolor{currentfill}%
\pgfsetfillopacity{0.800000}%
\pgfsetlinewidth{0.000000pt}%
\definecolor{currentstroke}{rgb}{0.000000,0.000000,0.000000}%
\pgfsetstrokecolor{currentstroke}%
\pgfsetstrokeopacity{0.800000}%
\pgfsetdash{}{0pt}%
\pgfpathmoveto{\pgfqpoint{3.094318in}{0.440000in}}%
\pgfpathlineto{\pgfqpoint{3.258712in}{0.440000in}}%
\pgfpathlineto{\pgfqpoint{3.258712in}{0.864322in}}%
\pgfpathlineto{\pgfqpoint{3.094318in}{0.864322in}}%
\pgfpathlineto{\pgfqpoint{3.094318in}{0.440000in}}%
\pgfpathclose%
\pgfusepath{fill}%
\end{pgfscope}%
\begin{pgfscope}%
\pgfpathrectangle{\pgfqpoint{0.875000in}{0.440000in}}{\pgfqpoint{5.425000in}{3.080000in}}%
\pgfusepath{clip}%
\pgfsetbuttcap%
\pgfsetmiterjoin%
\definecolor{currentfill}{rgb}{0.298039,0.447059,0.690196}%
\pgfsetfillcolor{currentfill}%
\pgfsetfillopacity{0.800000}%
\pgfsetlinewidth{0.000000pt}%
\definecolor{currentstroke}{rgb}{0.000000,0.000000,0.000000}%
\pgfsetstrokecolor{currentstroke}%
\pgfsetstrokeopacity{0.800000}%
\pgfsetdash{}{0pt}%
\pgfpathmoveto{\pgfqpoint{3.258712in}{0.440000in}}%
\pgfpathlineto{\pgfqpoint{3.423106in}{0.440000in}}%
\pgfpathlineto{\pgfqpoint{3.423106in}{1.207415in}}%
\pgfpathlineto{\pgfqpoint{3.258712in}{1.207415in}}%
\pgfpathlineto{\pgfqpoint{3.258712in}{0.440000in}}%
\pgfpathclose%
\pgfusepath{fill}%
\end{pgfscope}%
\begin{pgfscope}%
\pgfpathrectangle{\pgfqpoint{0.875000in}{0.440000in}}{\pgfqpoint{5.425000in}{3.080000in}}%
\pgfusepath{clip}%
\pgfsetbuttcap%
\pgfsetmiterjoin%
\definecolor{currentfill}{rgb}{0.298039,0.447059,0.690196}%
\pgfsetfillcolor{currentfill}%
\pgfsetfillopacity{0.800000}%
\pgfsetlinewidth{0.000000pt}%
\definecolor{currentstroke}{rgb}{0.000000,0.000000,0.000000}%
\pgfsetstrokecolor{currentstroke}%
\pgfsetstrokeopacity{0.800000}%
\pgfsetdash{}{0pt}%
\pgfpathmoveto{\pgfqpoint{3.423106in}{0.440000in}}%
\pgfpathlineto{\pgfqpoint{3.587500in}{0.440000in}}%
\pgfpathlineto{\pgfqpoint{3.587500in}{1.519267in}}%
\pgfpathlineto{\pgfqpoint{3.423106in}{1.519267in}}%
\pgfpathlineto{\pgfqpoint{3.423106in}{0.440000in}}%
\pgfpathclose%
\pgfusepath{fill}%
\end{pgfscope}%
\begin{pgfscope}%
\pgfpathrectangle{\pgfqpoint{0.875000in}{0.440000in}}{\pgfqpoint{5.425000in}{3.080000in}}%
\pgfusepath{clip}%
\pgfsetbuttcap%
\pgfsetmiterjoin%
\definecolor{currentfill}{rgb}{0.298039,0.447059,0.690196}%
\pgfsetfillcolor{currentfill}%
\pgfsetfillopacity{0.800000}%
\pgfsetlinewidth{0.000000pt}%
\definecolor{currentstroke}{rgb}{0.000000,0.000000,0.000000}%
\pgfsetstrokecolor{currentstroke}%
\pgfsetstrokeopacity{0.800000}%
\pgfsetdash{}{0pt}%
\pgfpathmoveto{\pgfqpoint{3.587500in}{0.440000in}}%
\pgfpathlineto{\pgfqpoint{3.751894in}{0.440000in}}%
\pgfpathlineto{\pgfqpoint{3.751894in}{1.856112in}}%
\pgfpathlineto{\pgfqpoint{3.587500in}{1.856112in}}%
\pgfpathlineto{\pgfqpoint{3.587500in}{0.440000in}}%
\pgfpathclose%
\pgfusepath{fill}%
\end{pgfscope}%
\begin{pgfscope}%
\pgfpathrectangle{\pgfqpoint{0.875000in}{0.440000in}}{\pgfqpoint{5.425000in}{3.080000in}}%
\pgfusepath{clip}%
\pgfsetbuttcap%
\pgfsetmiterjoin%
\definecolor{currentfill}{rgb}{0.298039,0.447059,0.690196}%
\pgfsetfillcolor{currentfill}%
\pgfsetfillopacity{0.800000}%
\pgfsetlinewidth{0.000000pt}%
\definecolor{currentstroke}{rgb}{0.000000,0.000000,0.000000}%
\pgfsetstrokecolor{currentstroke}%
\pgfsetstrokeopacity{0.800000}%
\pgfsetdash{}{0pt}%
\pgfpathmoveto{\pgfqpoint{3.751894in}{0.440000in}}%
\pgfpathlineto{\pgfqpoint{3.916288in}{0.440000in}}%
\pgfpathlineto{\pgfqpoint{3.916288in}{2.266233in}}%
\pgfpathlineto{\pgfqpoint{3.751894in}{2.266233in}}%
\pgfpathlineto{\pgfqpoint{3.751894in}{0.440000in}}%
\pgfpathclose%
\pgfusepath{fill}%
\end{pgfscope}%
\begin{pgfscope}%
\pgfpathrectangle{\pgfqpoint{0.875000in}{0.440000in}}{\pgfqpoint{5.425000in}{3.080000in}}%
\pgfusepath{clip}%
\pgfsetbuttcap%
\pgfsetmiterjoin%
\definecolor{currentfill}{rgb}{0.298039,0.447059,0.690196}%
\pgfsetfillcolor{currentfill}%
\pgfsetfillopacity{0.800000}%
\pgfsetlinewidth{0.000000pt}%
\definecolor{currentstroke}{rgb}{0.000000,0.000000,0.000000}%
\pgfsetstrokecolor{currentstroke}%
\pgfsetstrokeopacity{0.800000}%
\pgfsetdash{}{0pt}%
\pgfpathmoveto{\pgfqpoint{3.916288in}{0.440000in}}%
\pgfpathlineto{\pgfqpoint{4.080682in}{0.440000in}}%
\pgfpathlineto{\pgfqpoint{4.080682in}{2.714413in}}%
\pgfpathlineto{\pgfqpoint{3.916288in}{2.714413in}}%
\pgfpathlineto{\pgfqpoint{3.916288in}{0.440000in}}%
\pgfpathclose%
\pgfusepath{fill}%
\end{pgfscope}%
\begin{pgfscope}%
\pgfpathrectangle{\pgfqpoint{0.875000in}{0.440000in}}{\pgfqpoint{5.425000in}{3.080000in}}%
\pgfusepath{clip}%
\pgfsetbuttcap%
\pgfsetmiterjoin%
\definecolor{currentfill}{rgb}{0.298039,0.447059,0.690196}%
\pgfsetfillcolor{currentfill}%
\pgfsetfillopacity{0.800000}%
\pgfsetlinewidth{0.000000pt}%
\definecolor{currentstroke}{rgb}{0.000000,0.000000,0.000000}%
\pgfsetstrokecolor{currentstroke}%
\pgfsetstrokeopacity{0.800000}%
\pgfsetdash{}{0pt}%
\pgfpathmoveto{\pgfqpoint{4.080682in}{0.440000in}}%
\pgfpathlineto{\pgfqpoint{4.245076in}{0.440000in}}%
\pgfpathlineto{\pgfqpoint{4.245076in}{3.105221in}}%
\pgfpathlineto{\pgfqpoint{4.080682in}{3.105221in}}%
\pgfpathlineto{\pgfqpoint{4.080682in}{0.440000in}}%
\pgfpathclose%
\pgfusepath{fill}%
\end{pgfscope}%
\begin{pgfscope}%
\pgfpathrectangle{\pgfqpoint{0.875000in}{0.440000in}}{\pgfqpoint{5.425000in}{3.080000in}}%
\pgfusepath{clip}%
\pgfsetbuttcap%
\pgfsetmiterjoin%
\definecolor{currentfill}{rgb}{0.298039,0.447059,0.690196}%
\pgfsetfillcolor{currentfill}%
\pgfsetfillopacity{0.800000}%
\pgfsetlinewidth{0.000000pt}%
\definecolor{currentstroke}{rgb}{0.000000,0.000000,0.000000}%
\pgfsetstrokecolor{currentstroke}%
\pgfsetstrokeopacity{0.800000}%
\pgfsetdash{}{0pt}%
\pgfpathmoveto{\pgfqpoint{4.245076in}{0.440000in}}%
\pgfpathlineto{\pgfqpoint{4.409470in}{0.440000in}}%
\pgfpathlineto{\pgfqpoint{4.409470in}{3.227348in}}%
\pgfpathlineto{\pgfqpoint{4.245076in}{3.227348in}}%
\pgfpathlineto{\pgfqpoint{4.245076in}{0.440000in}}%
\pgfpathclose%
\pgfusepath{fill}%
\end{pgfscope}%
\begin{pgfscope}%
\pgfpathrectangle{\pgfqpoint{0.875000in}{0.440000in}}{\pgfqpoint{5.425000in}{3.080000in}}%
\pgfusepath{clip}%
\pgfsetbuttcap%
\pgfsetmiterjoin%
\definecolor{currentfill}{rgb}{0.298039,0.447059,0.690196}%
\pgfsetfillcolor{currentfill}%
\pgfsetfillopacity{0.800000}%
\pgfsetlinewidth{0.000000pt}%
\definecolor{currentstroke}{rgb}{0.000000,0.000000,0.000000}%
\pgfsetstrokecolor{currentstroke}%
\pgfsetstrokeopacity{0.800000}%
\pgfsetdash{}{0pt}%
\pgfpathmoveto{\pgfqpoint{4.409470in}{0.440000in}}%
\pgfpathlineto{\pgfqpoint{4.573864in}{0.440000in}}%
\pgfpathlineto{\pgfqpoint{4.573864in}{3.373333in}}%
\pgfpathlineto{\pgfqpoint{4.409470in}{3.373333in}}%
\pgfpathlineto{\pgfqpoint{4.409470in}{0.440000in}}%
\pgfpathclose%
\pgfusepath{fill}%
\end{pgfscope}%
\begin{pgfscope}%
\pgfpathrectangle{\pgfqpoint{0.875000in}{0.440000in}}{\pgfqpoint{5.425000in}{3.080000in}}%
\pgfusepath{clip}%
\pgfsetbuttcap%
\pgfsetmiterjoin%
\definecolor{currentfill}{rgb}{0.298039,0.447059,0.690196}%
\pgfsetfillcolor{currentfill}%
\pgfsetfillopacity{0.800000}%
\pgfsetlinewidth{0.000000pt}%
\definecolor{currentstroke}{rgb}{0.000000,0.000000,0.000000}%
\pgfsetstrokecolor{currentstroke}%
\pgfsetstrokeopacity{0.800000}%
\pgfsetdash{}{0pt}%
\pgfpathmoveto{\pgfqpoint{4.573864in}{0.440000in}}%
\pgfpathlineto{\pgfqpoint{4.738258in}{0.440000in}}%
\pgfpathlineto{\pgfqpoint{4.738258in}{3.240413in}}%
\pgfpathlineto{\pgfqpoint{4.573864in}{3.240413in}}%
\pgfpathlineto{\pgfqpoint{4.573864in}{0.440000in}}%
\pgfpathclose%
\pgfusepath{fill}%
\end{pgfscope}%
\begin{pgfscope}%
\pgfpathrectangle{\pgfqpoint{0.875000in}{0.440000in}}{\pgfqpoint{5.425000in}{3.080000in}}%
\pgfusepath{clip}%
\pgfsetbuttcap%
\pgfsetmiterjoin%
\definecolor{currentfill}{rgb}{0.298039,0.447059,0.690196}%
\pgfsetfillcolor{currentfill}%
\pgfsetfillopacity{0.800000}%
\pgfsetlinewidth{0.000000pt}%
\definecolor{currentstroke}{rgb}{0.000000,0.000000,0.000000}%
\pgfsetstrokecolor{currentstroke}%
\pgfsetstrokeopacity{0.800000}%
\pgfsetdash{}{0pt}%
\pgfpathmoveto{\pgfqpoint{4.738258in}{0.440000in}}%
\pgfpathlineto{\pgfqpoint{4.902652in}{0.440000in}}%
\pgfpathlineto{\pgfqpoint{4.902652in}{2.922313in}}%
\pgfpathlineto{\pgfqpoint{4.738258in}{2.922313in}}%
\pgfpathlineto{\pgfqpoint{4.738258in}{0.440000in}}%
\pgfpathclose%
\pgfusepath{fill}%
\end{pgfscope}%
\begin{pgfscope}%
\pgfpathrectangle{\pgfqpoint{0.875000in}{0.440000in}}{\pgfqpoint{5.425000in}{3.080000in}}%
\pgfusepath{clip}%
\pgfsetbuttcap%
\pgfsetmiterjoin%
\definecolor{currentfill}{rgb}{0.298039,0.447059,0.690196}%
\pgfsetfillcolor{currentfill}%
\pgfsetfillopacity{0.800000}%
\pgfsetlinewidth{0.000000pt}%
\definecolor{currentstroke}{rgb}{0.000000,0.000000,0.000000}%
\pgfsetstrokecolor{currentstroke}%
\pgfsetstrokeopacity{0.800000}%
\pgfsetdash{}{0pt}%
\pgfpathmoveto{\pgfqpoint{4.902652in}{0.440000in}}%
\pgfpathlineto{\pgfqpoint{5.067045in}{0.440000in}}%
\pgfpathlineto{\pgfqpoint{5.067045in}{2.450276in}}%
\pgfpathlineto{\pgfqpoint{4.902652in}{2.450276in}}%
\pgfpathlineto{\pgfqpoint{4.902652in}{0.440000in}}%
\pgfpathclose%
\pgfusepath{fill}%
\end{pgfscope}%
\begin{pgfscope}%
\pgfpathrectangle{\pgfqpoint{0.875000in}{0.440000in}}{\pgfqpoint{5.425000in}{3.080000in}}%
\pgfusepath{clip}%
\pgfsetbuttcap%
\pgfsetmiterjoin%
\definecolor{currentfill}{rgb}{0.298039,0.447059,0.690196}%
\pgfsetfillcolor{currentfill}%
\pgfsetfillopacity{0.800000}%
\pgfsetlinewidth{0.000000pt}%
\definecolor{currentstroke}{rgb}{0.000000,0.000000,0.000000}%
\pgfsetstrokecolor{currentstroke}%
\pgfsetstrokeopacity{0.800000}%
\pgfsetdash{}{0pt}%
\pgfpathmoveto{\pgfqpoint{5.067045in}{0.440000in}}%
\pgfpathlineto{\pgfqpoint{5.231439in}{0.440000in}}%
\pgfpathlineto{\pgfqpoint{5.231439in}{2.066853in}}%
\pgfpathlineto{\pgfqpoint{5.067045in}{2.066853in}}%
\pgfpathlineto{\pgfqpoint{5.067045in}{0.440000in}}%
\pgfpathclose%
\pgfusepath{fill}%
\end{pgfscope}%
\begin{pgfscope}%
\pgfpathrectangle{\pgfqpoint{0.875000in}{0.440000in}}{\pgfqpoint{5.425000in}{3.080000in}}%
\pgfusepath{clip}%
\pgfsetbuttcap%
\pgfsetmiterjoin%
\definecolor{currentfill}{rgb}{0.298039,0.447059,0.690196}%
\pgfsetfillcolor{currentfill}%
\pgfsetfillopacity{0.800000}%
\pgfsetlinewidth{0.000000pt}%
\definecolor{currentstroke}{rgb}{0.000000,0.000000,0.000000}%
\pgfsetstrokecolor{currentstroke}%
\pgfsetstrokeopacity{0.800000}%
\pgfsetdash{}{0pt}%
\pgfpathmoveto{\pgfqpoint{5.231439in}{0.440000in}}%
\pgfpathlineto{\pgfqpoint{5.395833in}{0.440000in}}%
\pgfpathlineto{\pgfqpoint{5.395833in}{1.662980in}}%
\pgfpathlineto{\pgfqpoint{5.231439in}{1.662980in}}%
\pgfpathlineto{\pgfqpoint{5.231439in}{0.440000in}}%
\pgfpathclose%
\pgfusepath{fill}%
\end{pgfscope}%
\begin{pgfscope}%
\pgfpathrectangle{\pgfqpoint{0.875000in}{0.440000in}}{\pgfqpoint{5.425000in}{3.080000in}}%
\pgfusepath{clip}%
\pgfsetbuttcap%
\pgfsetmiterjoin%
\definecolor{currentfill}{rgb}{0.298039,0.447059,0.690196}%
\pgfsetfillcolor{currentfill}%
\pgfsetfillopacity{0.800000}%
\pgfsetlinewidth{0.000000pt}%
\definecolor{currentstroke}{rgb}{0.000000,0.000000,0.000000}%
\pgfsetstrokecolor{currentstroke}%
\pgfsetstrokeopacity{0.800000}%
\pgfsetdash{}{0pt}%
\pgfpathmoveto{\pgfqpoint{5.395833in}{0.440000in}}%
\pgfpathlineto{\pgfqpoint{5.560227in}{0.440000in}}%
\pgfpathlineto{\pgfqpoint{5.560227in}{1.249450in}}%
\pgfpathlineto{\pgfqpoint{5.395833in}{1.249450in}}%
\pgfpathlineto{\pgfqpoint{5.395833in}{0.440000in}}%
\pgfpathclose%
\pgfusepath{fill}%
\end{pgfscope}%
\begin{pgfscope}%
\pgfpathrectangle{\pgfqpoint{0.875000in}{0.440000in}}{\pgfqpoint{5.425000in}{3.080000in}}%
\pgfusepath{clip}%
\pgfsetbuttcap%
\pgfsetmiterjoin%
\definecolor{currentfill}{rgb}{0.298039,0.447059,0.690196}%
\pgfsetfillcolor{currentfill}%
\pgfsetfillopacity{0.800000}%
\pgfsetlinewidth{0.000000pt}%
\definecolor{currentstroke}{rgb}{0.000000,0.000000,0.000000}%
\pgfsetstrokecolor{currentstroke}%
\pgfsetstrokeopacity{0.800000}%
\pgfsetdash{}{0pt}%
\pgfpathmoveto{\pgfqpoint{5.560227in}{0.440000in}}%
\pgfpathlineto{\pgfqpoint{5.724621in}{0.440000in}}%
\pgfpathlineto{\pgfqpoint{5.724621in}{0.943847in}}%
\pgfpathlineto{\pgfqpoint{5.560227in}{0.943847in}}%
\pgfpathlineto{\pgfqpoint{5.560227in}{0.440000in}}%
\pgfpathclose%
\pgfusepath{fill}%
\end{pgfscope}%
\begin{pgfscope}%
\pgfpathrectangle{\pgfqpoint{0.875000in}{0.440000in}}{\pgfqpoint{5.425000in}{3.080000in}}%
\pgfusepath{clip}%
\pgfsetbuttcap%
\pgfsetmiterjoin%
\definecolor{currentfill}{rgb}{0.298039,0.447059,0.690196}%
\pgfsetfillcolor{currentfill}%
\pgfsetfillopacity{0.800000}%
\pgfsetlinewidth{0.000000pt}%
\definecolor{currentstroke}{rgb}{0.000000,0.000000,0.000000}%
\pgfsetstrokecolor{currentstroke}%
\pgfsetstrokeopacity{0.800000}%
\pgfsetdash{}{0pt}%
\pgfpathmoveto{\pgfqpoint{5.724621in}{0.440000in}}%
\pgfpathlineto{\pgfqpoint{5.889015in}{0.440000in}}%
\pgfpathlineto{\pgfqpoint{5.889015in}{0.660398in}}%
\pgfpathlineto{\pgfqpoint{5.724621in}{0.660398in}}%
\pgfpathlineto{\pgfqpoint{5.724621in}{0.440000in}}%
\pgfpathclose%
\pgfusepath{fill}%
\end{pgfscope}%
\begin{pgfscope}%
\pgfpathrectangle{\pgfqpoint{0.875000in}{0.440000in}}{\pgfqpoint{5.425000in}{3.080000in}}%
\pgfusepath{clip}%
\pgfsetbuttcap%
\pgfsetmiterjoin%
\definecolor{currentfill}{rgb}{0.298039,0.447059,0.690196}%
\pgfsetfillcolor{currentfill}%
\pgfsetfillopacity{0.800000}%
\pgfsetlinewidth{0.000000pt}%
\definecolor{currentstroke}{rgb}{0.000000,0.000000,0.000000}%
\pgfsetstrokecolor{currentstroke}%
\pgfsetstrokeopacity{0.800000}%
\pgfsetdash{}{0pt}%
\pgfpathmoveto{\pgfqpoint{5.889015in}{0.440000in}}%
\pgfpathlineto{\pgfqpoint{6.053409in}{0.440000in}}%
\pgfpathlineto{\pgfqpoint{6.053409in}{0.503620in}}%
\pgfpathlineto{\pgfqpoint{5.889015in}{0.503620in}}%
\pgfpathlineto{\pgfqpoint{5.889015in}{0.440000in}}%
\pgfpathclose%
\pgfusepath{fill}%
\end{pgfscope}%
\begin{pgfscope}%
\pgfsetrectcap%
\pgfsetmiterjoin%
\pgfsetlinewidth{0.000000pt}%
\definecolor{currentstroke}{rgb}{1.000000,1.000000,1.000000}%
\pgfsetstrokecolor{currentstroke}%
\pgfsetdash{}{0pt}%
\pgfpathmoveto{\pgfqpoint{0.875000in}{0.440000in}}%
\pgfpathlineto{\pgfqpoint{0.875000in}{3.520000in}}%
\pgfusepath{}%
\end{pgfscope}%
\begin{pgfscope}%
\pgfsetrectcap%
\pgfsetmiterjoin%
\pgfsetlinewidth{0.000000pt}%
\definecolor{currentstroke}{rgb}{1.000000,1.000000,1.000000}%
\pgfsetstrokecolor{currentstroke}%
\pgfsetdash{}{0pt}%
\pgfpathmoveto{\pgfqpoint{6.300000in}{0.440000in}}%
\pgfpathlineto{\pgfqpoint{6.300000in}{3.520000in}}%
\pgfusepath{}%
\end{pgfscope}%
\begin{pgfscope}%
\pgfsetrectcap%
\pgfsetmiterjoin%
\pgfsetlinewidth{0.000000pt}%
\definecolor{currentstroke}{rgb}{1.000000,1.000000,1.000000}%
\pgfsetstrokecolor{currentstroke}%
\pgfsetdash{}{0pt}%
\pgfpathmoveto{\pgfqpoint{0.875000in}{0.440000in}}%
\pgfpathlineto{\pgfqpoint{6.300000in}{0.440000in}}%
\pgfusepath{}%
\end{pgfscope}%
\begin{pgfscope}%
\pgfsetrectcap%
\pgfsetmiterjoin%
\pgfsetlinewidth{0.000000pt}%
\definecolor{currentstroke}{rgb}{1.000000,1.000000,1.000000}%
\pgfsetstrokecolor{currentstroke}%
\pgfsetdash{}{0pt}%
\pgfpathmoveto{\pgfqpoint{0.875000in}{3.520000in}}%
\pgfpathlineto{\pgfqpoint{6.300000in}{3.520000in}}%
\pgfusepath{}%
\end{pgfscope}%
\begin{pgfscope}%
\definecolor{textcolor}{rgb}{0.150000,0.150000,0.150000}%
\pgfsetstrokecolor{textcolor}%
\pgfsetfillcolor{textcolor}%
\pgftext[x=3.500000in,y=3.920000in,,top]{\color{textcolor}\rmfamily\fontsize{12.000000}{14.400000}\selectfont Histograma para la muestra con propuesta beta y \(\displaystyle n=40, r=\)15}%
\end{pgfscope}%
\end{pgfpicture}%
\makeatother%
\endgroup%

    \end{center}

    Notamos que en ambos casos los histogramas son parecidos a la densidad objetivo, aunque presentan
    diferencias, como más masa cerca del punto cero. Esto puede deberse a que el punto inicial uniforme comenzó
    cerca del cero, lo que ocasiona muestreo de puntos poco probables para la densidad. No podemos 
    afirmar que desde el primer valor que obtuvimos estamos muestreando de $f$,
    ya que la convergencia a la distribución objetivo es asintótica. Para poder hablar de un muestreo
    real de $f$ sería necesario descartar una cantidad fija de valores que se consideran observaciones
    de la cadena que aún no son ``suficientemente cercanas'' a la distribución de $f$.

    % Podemos evaluar cuál es el tamaño de la muestra que se obtuvo para cada observación de las
    % Bernoulli. Obtenemos que de las 1,000 propuestas en cada caso, 887 fueron aceptadas para
    % $r_5$ y 832 fueron aceptadas para $r_{40}$. El porcentaje de aceptación es de 88.7\% y 83.2\%
    % respectivamente. Basandanos únicamente en estas observaciones, los datos parecen sugerir que 
    % la distribución de transición propuesta funciona mejor para muestrear de $f(p|r_5)$ que de $f(p|r_{40})$.

    Otra observación importante es que, a pesar de que se obtuvieron 1,000 valores de muestreo, 
    estos no constituyen una muestra i. i. d., pues al provenir de observaciones simuladas de una cadena
    de Márkov existe una correlación. Para poder considerar a la muestra independiente, sería necesario
    tomar valores suficientemente espaciados a lo largo de toda la muestra, de manera que la correlación
    entre ellos sea cercana a cero. 

    La manera de lidiar el problemas anterior se basa en heurísticas que se revisarán en
    el curso y se incluirán en tareas posteriores. 

    Para solucionar el problema de muestrear desde los primeros valores, podemos usar el logaritmo
    de la densidad evaluada en el proceso $X_t$ y revisar en qué momento este valor parece estabilizarse.
    Las siguientes figuras muestran como se comporta esta función para los dos muestreos,

    \begin{center}
        %% Creator: Matplotlib, PGF backend
%%
%% To include the figure in your LaTeX document, write
%%   \input{<filename>.pgf}
%%
%% Make sure the required packages are loaded in your preamble
%%   \usepackage{pgf}
%%
%% Also ensure that all the required font packages are loaded; for instance,
%% the lmodern package is sometimes necessary when using math font.
%%   \usepackage{lmodern}
%%
%% Figures using additional raster images can only be included by \input if
%% they are in the same directory as the main LaTeX file. For loading figures
%% from other directories you can use the `import` package
%%   \usepackage{import}
%%
%% and then include the figures with
%%   \import{<path to file>}{<filename>.pgf}
%%
%% Matplotlib used the following preamble
%%   
%%   \makeatletter\@ifpackageloaded{underscore}{}{\usepackage[strings]{underscore}}\makeatother
%%
\begingroup%
\makeatletter%
\begin{pgfpicture}%
\pgfpathrectangle{\pgfpointorigin}{\pgfqpoint{7.000000in}{4.000000in}}%
\pgfusepath{use as bounding box, clip}%
\begin{pgfscope}%
\pgfsetbuttcap%
\pgfsetmiterjoin%
\definecolor{currentfill}{rgb}{1.000000,1.000000,1.000000}%
\pgfsetfillcolor{currentfill}%
\pgfsetlinewidth{0.000000pt}%
\definecolor{currentstroke}{rgb}{1.000000,1.000000,1.000000}%
\pgfsetstrokecolor{currentstroke}%
\pgfsetdash{}{0pt}%
\pgfpathmoveto{\pgfqpoint{0.000000in}{0.000000in}}%
\pgfpathlineto{\pgfqpoint{7.000000in}{0.000000in}}%
\pgfpathlineto{\pgfqpoint{7.000000in}{4.000000in}}%
\pgfpathlineto{\pgfqpoint{0.000000in}{4.000000in}}%
\pgfpathlineto{\pgfqpoint{0.000000in}{0.000000in}}%
\pgfpathclose%
\pgfusepath{fill}%
\end{pgfscope}%
\begin{pgfscope}%
\pgfsetbuttcap%
\pgfsetmiterjoin%
\definecolor{currentfill}{rgb}{0.917647,0.917647,0.949020}%
\pgfsetfillcolor{currentfill}%
\pgfsetlinewidth{0.000000pt}%
\definecolor{currentstroke}{rgb}{0.000000,0.000000,0.000000}%
\pgfsetstrokecolor{currentstroke}%
\pgfsetstrokeopacity{0.000000}%
\pgfsetdash{}{0pt}%
\pgfpathmoveto{\pgfqpoint{0.875000in}{0.440000in}}%
\pgfpathlineto{\pgfqpoint{6.300000in}{0.440000in}}%
\pgfpathlineto{\pgfqpoint{6.300000in}{3.520000in}}%
\pgfpathlineto{\pgfqpoint{0.875000in}{3.520000in}}%
\pgfpathlineto{\pgfqpoint{0.875000in}{0.440000in}}%
\pgfpathclose%
\pgfusepath{fill}%
\end{pgfscope}%
\begin{pgfscope}%
\pgfpathrectangle{\pgfqpoint{0.875000in}{0.440000in}}{\pgfqpoint{5.425000in}{3.080000in}}%
\pgfusepath{clip}%
\pgfsetroundcap%
\pgfsetroundjoin%
\pgfsetlinewidth{1.003750pt}%
\definecolor{currentstroke}{rgb}{1.000000,1.000000,1.000000}%
\pgfsetstrokecolor{currentstroke}%
\pgfsetdash{}{0pt}%
\pgfpathmoveto{\pgfqpoint{1.121591in}{0.440000in}}%
\pgfpathlineto{\pgfqpoint{1.121591in}{3.520000in}}%
\pgfusepath{stroke}%
\end{pgfscope}%
\begin{pgfscope}%
\definecolor{textcolor}{rgb}{0.150000,0.150000,0.150000}%
\pgfsetstrokecolor{textcolor}%
\pgfsetfillcolor{textcolor}%
\pgftext[x=1.121591in,y=0.342778in,,top]{\color{textcolor}\rmfamily\fontsize{10.000000}{12.000000}\selectfont \(\displaystyle {0}\)}%
\end{pgfscope}%
\begin{pgfscope}%
\pgfpathrectangle{\pgfqpoint{0.875000in}{0.440000in}}{\pgfqpoint{5.425000in}{3.080000in}}%
\pgfusepath{clip}%
\pgfsetroundcap%
\pgfsetroundjoin%
\pgfsetlinewidth{1.003750pt}%
\definecolor{currentstroke}{rgb}{1.000000,1.000000,1.000000}%
\pgfsetstrokecolor{currentstroke}%
\pgfsetdash{}{0pt}%
\pgfpathmoveto{\pgfqpoint{2.107955in}{0.440000in}}%
\pgfpathlineto{\pgfqpoint{2.107955in}{3.520000in}}%
\pgfusepath{stroke}%
\end{pgfscope}%
\begin{pgfscope}%
\definecolor{textcolor}{rgb}{0.150000,0.150000,0.150000}%
\pgfsetstrokecolor{textcolor}%
\pgfsetfillcolor{textcolor}%
\pgftext[x=2.107955in,y=0.342778in,,top]{\color{textcolor}\rmfamily\fontsize{10.000000}{12.000000}\selectfont \(\displaystyle {200}\)}%
\end{pgfscope}%
\begin{pgfscope}%
\pgfpathrectangle{\pgfqpoint{0.875000in}{0.440000in}}{\pgfqpoint{5.425000in}{3.080000in}}%
\pgfusepath{clip}%
\pgfsetroundcap%
\pgfsetroundjoin%
\pgfsetlinewidth{1.003750pt}%
\definecolor{currentstroke}{rgb}{1.000000,1.000000,1.000000}%
\pgfsetstrokecolor{currentstroke}%
\pgfsetdash{}{0pt}%
\pgfpathmoveto{\pgfqpoint{3.094318in}{0.440000in}}%
\pgfpathlineto{\pgfqpoint{3.094318in}{3.520000in}}%
\pgfusepath{stroke}%
\end{pgfscope}%
\begin{pgfscope}%
\definecolor{textcolor}{rgb}{0.150000,0.150000,0.150000}%
\pgfsetstrokecolor{textcolor}%
\pgfsetfillcolor{textcolor}%
\pgftext[x=3.094318in,y=0.342778in,,top]{\color{textcolor}\rmfamily\fontsize{10.000000}{12.000000}\selectfont \(\displaystyle {400}\)}%
\end{pgfscope}%
\begin{pgfscope}%
\pgfpathrectangle{\pgfqpoint{0.875000in}{0.440000in}}{\pgfqpoint{5.425000in}{3.080000in}}%
\pgfusepath{clip}%
\pgfsetroundcap%
\pgfsetroundjoin%
\pgfsetlinewidth{1.003750pt}%
\definecolor{currentstroke}{rgb}{1.000000,1.000000,1.000000}%
\pgfsetstrokecolor{currentstroke}%
\pgfsetdash{}{0pt}%
\pgfpathmoveto{\pgfqpoint{4.080682in}{0.440000in}}%
\pgfpathlineto{\pgfqpoint{4.080682in}{3.520000in}}%
\pgfusepath{stroke}%
\end{pgfscope}%
\begin{pgfscope}%
\definecolor{textcolor}{rgb}{0.150000,0.150000,0.150000}%
\pgfsetstrokecolor{textcolor}%
\pgfsetfillcolor{textcolor}%
\pgftext[x=4.080682in,y=0.342778in,,top]{\color{textcolor}\rmfamily\fontsize{10.000000}{12.000000}\selectfont \(\displaystyle {600}\)}%
\end{pgfscope}%
\begin{pgfscope}%
\pgfpathrectangle{\pgfqpoint{0.875000in}{0.440000in}}{\pgfqpoint{5.425000in}{3.080000in}}%
\pgfusepath{clip}%
\pgfsetroundcap%
\pgfsetroundjoin%
\pgfsetlinewidth{1.003750pt}%
\definecolor{currentstroke}{rgb}{1.000000,1.000000,1.000000}%
\pgfsetstrokecolor{currentstroke}%
\pgfsetdash{}{0pt}%
\pgfpathmoveto{\pgfqpoint{5.067045in}{0.440000in}}%
\pgfpathlineto{\pgfqpoint{5.067045in}{3.520000in}}%
\pgfusepath{stroke}%
\end{pgfscope}%
\begin{pgfscope}%
\definecolor{textcolor}{rgb}{0.150000,0.150000,0.150000}%
\pgfsetstrokecolor{textcolor}%
\pgfsetfillcolor{textcolor}%
\pgftext[x=5.067045in,y=0.342778in,,top]{\color{textcolor}\rmfamily\fontsize{10.000000}{12.000000}\selectfont \(\displaystyle {800}\)}%
\end{pgfscope}%
\begin{pgfscope}%
\pgfpathrectangle{\pgfqpoint{0.875000in}{0.440000in}}{\pgfqpoint{5.425000in}{3.080000in}}%
\pgfusepath{clip}%
\pgfsetroundcap%
\pgfsetroundjoin%
\pgfsetlinewidth{1.003750pt}%
\definecolor{currentstroke}{rgb}{1.000000,1.000000,1.000000}%
\pgfsetstrokecolor{currentstroke}%
\pgfsetdash{}{0pt}%
\pgfpathmoveto{\pgfqpoint{6.053409in}{0.440000in}}%
\pgfpathlineto{\pgfqpoint{6.053409in}{3.520000in}}%
\pgfusepath{stroke}%
\end{pgfscope}%
\begin{pgfscope}%
\definecolor{textcolor}{rgb}{0.150000,0.150000,0.150000}%
\pgfsetstrokecolor{textcolor}%
\pgfsetfillcolor{textcolor}%
\pgftext[x=6.053409in,y=0.342778in,,top]{\color{textcolor}\rmfamily\fontsize{10.000000}{12.000000}\selectfont \(\displaystyle {1000}\)}%
\end{pgfscope}%
\begin{pgfscope}%
\pgfpathrectangle{\pgfqpoint{0.875000in}{0.440000in}}{\pgfqpoint{5.425000in}{3.080000in}}%
\pgfusepath{clip}%
\pgfsetroundcap%
\pgfsetroundjoin%
\pgfsetlinewidth{1.003750pt}%
\definecolor{currentstroke}{rgb}{1.000000,1.000000,1.000000}%
\pgfsetstrokecolor{currentstroke}%
\pgfsetdash{}{0pt}%
\pgfpathmoveto{\pgfqpoint{0.875000in}{0.642245in}}%
\pgfpathlineto{\pgfqpoint{6.300000in}{0.642245in}}%
\pgfusepath{stroke}%
\end{pgfscope}%
\begin{pgfscope}%
\definecolor{textcolor}{rgb}{0.150000,0.150000,0.150000}%
\pgfsetstrokecolor{textcolor}%
\pgfsetfillcolor{textcolor}%
\pgftext[x=0.492283in, y=0.594020in, left, base]{\color{textcolor}\rmfamily\fontsize{10.000000}{12.000000}\selectfont \(\displaystyle {\ensuremath{-}7.0}\)}%
\end{pgfscope}%
\begin{pgfscope}%
\pgfpathrectangle{\pgfqpoint{0.875000in}{0.440000in}}{\pgfqpoint{5.425000in}{3.080000in}}%
\pgfusepath{clip}%
\pgfsetroundcap%
\pgfsetroundjoin%
\pgfsetlinewidth{1.003750pt}%
\definecolor{currentstroke}{rgb}{1.000000,1.000000,1.000000}%
\pgfsetstrokecolor{currentstroke}%
\pgfsetdash{}{0pt}%
\pgfpathmoveto{\pgfqpoint{0.875000in}{1.263825in}}%
\pgfpathlineto{\pgfqpoint{6.300000in}{1.263825in}}%
\pgfusepath{stroke}%
\end{pgfscope}%
\begin{pgfscope}%
\definecolor{textcolor}{rgb}{0.150000,0.150000,0.150000}%
\pgfsetstrokecolor{textcolor}%
\pgfsetfillcolor{textcolor}%
\pgftext[x=0.492283in, y=1.215600in, left, base]{\color{textcolor}\rmfamily\fontsize{10.000000}{12.000000}\selectfont \(\displaystyle {\ensuremath{-}6.5}\)}%
\end{pgfscope}%
\begin{pgfscope}%
\pgfpathrectangle{\pgfqpoint{0.875000in}{0.440000in}}{\pgfqpoint{5.425000in}{3.080000in}}%
\pgfusepath{clip}%
\pgfsetroundcap%
\pgfsetroundjoin%
\pgfsetlinewidth{1.003750pt}%
\definecolor{currentstroke}{rgb}{1.000000,1.000000,1.000000}%
\pgfsetstrokecolor{currentstroke}%
\pgfsetdash{}{0pt}%
\pgfpathmoveto{\pgfqpoint{0.875000in}{1.885405in}}%
\pgfpathlineto{\pgfqpoint{6.300000in}{1.885405in}}%
\pgfusepath{stroke}%
\end{pgfscope}%
\begin{pgfscope}%
\definecolor{textcolor}{rgb}{0.150000,0.150000,0.150000}%
\pgfsetstrokecolor{textcolor}%
\pgfsetfillcolor{textcolor}%
\pgftext[x=0.492283in, y=1.837180in, left, base]{\color{textcolor}\rmfamily\fontsize{10.000000}{12.000000}\selectfont \(\displaystyle {\ensuremath{-}6.0}\)}%
\end{pgfscope}%
\begin{pgfscope}%
\pgfpathrectangle{\pgfqpoint{0.875000in}{0.440000in}}{\pgfqpoint{5.425000in}{3.080000in}}%
\pgfusepath{clip}%
\pgfsetroundcap%
\pgfsetroundjoin%
\pgfsetlinewidth{1.003750pt}%
\definecolor{currentstroke}{rgb}{1.000000,1.000000,1.000000}%
\pgfsetstrokecolor{currentstroke}%
\pgfsetdash{}{0pt}%
\pgfpathmoveto{\pgfqpoint{0.875000in}{2.506985in}}%
\pgfpathlineto{\pgfqpoint{6.300000in}{2.506985in}}%
\pgfusepath{stroke}%
\end{pgfscope}%
\begin{pgfscope}%
\definecolor{textcolor}{rgb}{0.150000,0.150000,0.150000}%
\pgfsetstrokecolor{textcolor}%
\pgfsetfillcolor{textcolor}%
\pgftext[x=0.492283in, y=2.458760in, left, base]{\color{textcolor}\rmfamily\fontsize{10.000000}{12.000000}\selectfont \(\displaystyle {\ensuremath{-}5.5}\)}%
\end{pgfscope}%
\begin{pgfscope}%
\pgfpathrectangle{\pgfqpoint{0.875000in}{0.440000in}}{\pgfqpoint{5.425000in}{3.080000in}}%
\pgfusepath{clip}%
\pgfsetroundcap%
\pgfsetroundjoin%
\pgfsetlinewidth{1.003750pt}%
\definecolor{currentstroke}{rgb}{1.000000,1.000000,1.000000}%
\pgfsetstrokecolor{currentstroke}%
\pgfsetdash{}{0pt}%
\pgfpathmoveto{\pgfqpoint{0.875000in}{3.128565in}}%
\pgfpathlineto{\pgfqpoint{6.300000in}{3.128565in}}%
\pgfusepath{stroke}%
\end{pgfscope}%
\begin{pgfscope}%
\definecolor{textcolor}{rgb}{0.150000,0.150000,0.150000}%
\pgfsetstrokecolor{textcolor}%
\pgfsetfillcolor{textcolor}%
\pgftext[x=0.492283in, y=3.080339in, left, base]{\color{textcolor}\rmfamily\fontsize{10.000000}{12.000000}\selectfont \(\displaystyle {\ensuremath{-}5.0}\)}%
\end{pgfscope}%
\begin{pgfscope}%
\pgfpathrectangle{\pgfqpoint{0.875000in}{0.440000in}}{\pgfqpoint{5.425000in}{3.080000in}}%
\pgfusepath{clip}%
\pgfsetroundcap%
\pgfsetroundjoin%
\pgfsetlinewidth{1.756562pt}%
\definecolor{currentstroke}{rgb}{0.298039,0.447059,0.690196}%
\pgfsetstrokecolor{currentstroke}%
\pgfsetdash{}{0pt}%
\pgfpathmoveto{\pgfqpoint{1.121591in}{3.325656in}}%
\pgfpathlineto{\pgfqpoint{1.126523in}{3.365852in}}%
\pgfpathlineto{\pgfqpoint{1.205432in}{3.365852in}}%
\pgfpathlineto{\pgfqpoint{1.210364in}{2.198886in}}%
\pgfpathlineto{\pgfqpoint{1.254750in}{2.198886in}}%
\pgfpathlineto{\pgfqpoint{1.259682in}{2.315732in}}%
\pgfpathlineto{\pgfqpoint{1.264614in}{2.315732in}}%
\pgfpathlineto{\pgfqpoint{1.269545in}{3.325400in}}%
\pgfpathlineto{\pgfqpoint{1.279409in}{3.325400in}}%
\pgfpathlineto{\pgfqpoint{1.284341in}{3.089061in}}%
\pgfpathlineto{\pgfqpoint{1.299136in}{3.089061in}}%
\pgfpathlineto{\pgfqpoint{1.304068in}{3.363201in}}%
\pgfpathlineto{\pgfqpoint{1.309000in}{3.363201in}}%
\pgfpathlineto{\pgfqpoint{1.313932in}{2.888321in}}%
\pgfpathlineto{\pgfqpoint{1.323795in}{2.888321in}}%
\pgfpathlineto{\pgfqpoint{1.328727in}{2.760468in}}%
\pgfpathlineto{\pgfqpoint{1.373114in}{2.760468in}}%
\pgfpathlineto{\pgfqpoint{1.378045in}{2.892015in}}%
\pgfpathlineto{\pgfqpoint{1.407636in}{2.892015in}}%
\pgfpathlineto{\pgfqpoint{1.412568in}{3.358003in}}%
\pgfpathlineto{\pgfqpoint{1.427364in}{3.358003in}}%
\pgfpathlineto{\pgfqpoint{1.432295in}{3.333650in}}%
\pgfpathlineto{\pgfqpoint{1.437227in}{3.333650in}}%
\pgfpathlineto{\pgfqpoint{1.442159in}{2.696582in}}%
\pgfpathlineto{\pgfqpoint{1.447091in}{2.696582in}}%
\pgfpathlineto{\pgfqpoint{1.452023in}{3.379996in}}%
\pgfpathlineto{\pgfqpoint{1.456955in}{3.357388in}}%
\pgfpathlineto{\pgfqpoint{1.461886in}{3.357388in}}%
\pgfpathlineto{\pgfqpoint{1.466818in}{2.318125in}}%
\pgfpathlineto{\pgfqpoint{1.501341in}{2.318125in}}%
\pgfpathlineto{\pgfqpoint{1.506273in}{3.028902in}}%
\pgfpathlineto{\pgfqpoint{1.511205in}{3.028902in}}%
\pgfpathlineto{\pgfqpoint{1.516136in}{3.358700in}}%
\pgfpathlineto{\pgfqpoint{1.535864in}{3.358700in}}%
\pgfpathlineto{\pgfqpoint{1.540795in}{1.813817in}}%
\pgfpathlineto{\pgfqpoint{1.555591in}{1.813817in}}%
\pgfpathlineto{\pgfqpoint{1.560523in}{2.103401in}}%
\pgfpathlineto{\pgfqpoint{1.565455in}{2.985350in}}%
\pgfpathlineto{\pgfqpoint{1.570386in}{2.850269in}}%
\pgfpathlineto{\pgfqpoint{1.575318in}{3.349359in}}%
\pgfpathlineto{\pgfqpoint{1.590114in}{3.349359in}}%
\pgfpathlineto{\pgfqpoint{1.595045in}{3.275930in}}%
\pgfpathlineto{\pgfqpoint{1.609841in}{3.275930in}}%
\pgfpathlineto{\pgfqpoint{1.614773in}{3.376747in}}%
\pgfpathlineto{\pgfqpoint{1.639432in}{3.376747in}}%
\pgfpathlineto{\pgfqpoint{1.644364in}{3.270257in}}%
\pgfpathlineto{\pgfqpoint{1.649295in}{3.378647in}}%
\pgfpathlineto{\pgfqpoint{1.669023in}{3.378647in}}%
\pgfpathlineto{\pgfqpoint{1.673955in}{3.371896in}}%
\pgfpathlineto{\pgfqpoint{1.678886in}{3.371896in}}%
\pgfpathlineto{\pgfqpoint{1.683818in}{2.468910in}}%
\pgfpathlineto{\pgfqpoint{1.688750in}{2.468910in}}%
\pgfpathlineto{\pgfqpoint{1.693682in}{3.238922in}}%
\pgfpathlineto{\pgfqpoint{1.713409in}{3.238922in}}%
\pgfpathlineto{\pgfqpoint{1.718341in}{3.101894in}}%
\pgfpathlineto{\pgfqpoint{1.767659in}{3.101894in}}%
\pgfpathlineto{\pgfqpoint{1.772591in}{3.036540in}}%
\pgfpathlineto{\pgfqpoint{1.777523in}{3.036540in}}%
\pgfpathlineto{\pgfqpoint{1.782455in}{3.073582in}}%
\pgfpathlineto{\pgfqpoint{1.861364in}{3.073582in}}%
\pgfpathlineto{\pgfqpoint{1.866295in}{2.991535in}}%
\pgfpathlineto{\pgfqpoint{1.876159in}{2.991535in}}%
\pgfpathlineto{\pgfqpoint{1.881091in}{2.976996in}}%
\pgfpathlineto{\pgfqpoint{1.886023in}{2.976996in}}%
\pgfpathlineto{\pgfqpoint{1.890955in}{3.252992in}}%
\pgfpathlineto{\pgfqpoint{1.910682in}{3.252992in}}%
\pgfpathlineto{\pgfqpoint{1.915614in}{3.357701in}}%
\pgfpathlineto{\pgfqpoint{1.940273in}{3.357701in}}%
\pgfpathlineto{\pgfqpoint{1.945205in}{3.009864in}}%
\pgfpathlineto{\pgfqpoint{1.974795in}{3.009864in}}%
\pgfpathlineto{\pgfqpoint{1.979727in}{3.357391in}}%
\pgfpathlineto{\pgfqpoint{1.994523in}{3.357391in}}%
\pgfpathlineto{\pgfqpoint{1.999455in}{2.702620in}}%
\pgfpathlineto{\pgfqpoint{2.004386in}{1.713390in}}%
\pgfpathlineto{\pgfqpoint{2.009318in}{1.299579in}}%
\pgfpathlineto{\pgfqpoint{2.043841in}{1.299579in}}%
\pgfpathlineto{\pgfqpoint{2.048773in}{3.190848in}}%
\pgfpathlineto{\pgfqpoint{2.058636in}{3.190848in}}%
\pgfpathlineto{\pgfqpoint{2.063568in}{2.797456in}}%
\pgfpathlineto{\pgfqpoint{2.068500in}{3.160483in}}%
\pgfpathlineto{\pgfqpoint{2.083295in}{3.160483in}}%
\pgfpathlineto{\pgfqpoint{2.088227in}{3.342811in}}%
\pgfpathlineto{\pgfqpoint{2.093159in}{3.374248in}}%
\pgfpathlineto{\pgfqpoint{2.098091in}{3.309997in}}%
\pgfpathlineto{\pgfqpoint{2.107955in}{3.309997in}}%
\pgfpathlineto{\pgfqpoint{2.112886in}{3.379833in}}%
\pgfpathlineto{\pgfqpoint{2.117818in}{3.117912in}}%
\pgfpathlineto{\pgfqpoint{2.152341in}{3.117912in}}%
\pgfpathlineto{\pgfqpoint{2.157273in}{2.198683in}}%
\pgfpathlineto{\pgfqpoint{2.167136in}{2.198683in}}%
\pgfpathlineto{\pgfqpoint{2.172068in}{2.688021in}}%
\pgfpathlineto{\pgfqpoint{2.191795in}{2.688021in}}%
\pgfpathlineto{\pgfqpoint{2.196727in}{3.345608in}}%
\pgfpathlineto{\pgfqpoint{2.201659in}{3.345608in}}%
\pgfpathlineto{\pgfqpoint{2.206591in}{3.380000in}}%
\pgfpathlineto{\pgfqpoint{2.221386in}{3.380000in}}%
\pgfpathlineto{\pgfqpoint{2.226318in}{3.333062in}}%
\pgfpathlineto{\pgfqpoint{2.236182in}{3.333062in}}%
\pgfpathlineto{\pgfqpoint{2.241114in}{3.345795in}}%
\pgfpathlineto{\pgfqpoint{2.246045in}{2.673600in}}%
\pgfpathlineto{\pgfqpoint{2.255909in}{2.673600in}}%
\pgfpathlineto{\pgfqpoint{2.260841in}{2.774796in}}%
\pgfpathlineto{\pgfqpoint{2.265773in}{3.379515in}}%
\pgfpathlineto{\pgfqpoint{2.275636in}{3.379515in}}%
\pgfpathlineto{\pgfqpoint{2.280568in}{3.231824in}}%
\pgfpathlineto{\pgfqpoint{2.300295in}{3.231824in}}%
\pgfpathlineto{\pgfqpoint{2.305227in}{3.124501in}}%
\pgfpathlineto{\pgfqpoint{2.320023in}{3.124501in}}%
\pgfpathlineto{\pgfqpoint{2.324955in}{2.293060in}}%
\pgfpathlineto{\pgfqpoint{2.329886in}{3.369710in}}%
\pgfpathlineto{\pgfqpoint{2.349614in}{3.369710in}}%
\pgfpathlineto{\pgfqpoint{2.354545in}{3.358782in}}%
\pgfpathlineto{\pgfqpoint{2.359477in}{3.358782in}}%
\pgfpathlineto{\pgfqpoint{2.364409in}{3.251343in}}%
\pgfpathlineto{\pgfqpoint{2.389068in}{3.251343in}}%
\pgfpathlineto{\pgfqpoint{2.394000in}{3.374876in}}%
\pgfpathlineto{\pgfqpoint{2.398932in}{3.374876in}}%
\pgfpathlineto{\pgfqpoint{2.403864in}{3.376953in}}%
\pgfpathlineto{\pgfqpoint{2.418659in}{3.376953in}}%
\pgfpathlineto{\pgfqpoint{2.423591in}{3.351147in}}%
\pgfpathlineto{\pgfqpoint{2.443318in}{3.351147in}}%
\pgfpathlineto{\pgfqpoint{2.448250in}{3.164263in}}%
\pgfpathlineto{\pgfqpoint{2.517295in}{3.164263in}}%
\pgfpathlineto{\pgfqpoint{2.522227in}{3.235826in}}%
\pgfpathlineto{\pgfqpoint{2.537023in}{3.235826in}}%
\pgfpathlineto{\pgfqpoint{2.541955in}{3.051357in}}%
\pgfpathlineto{\pgfqpoint{2.576477in}{3.051357in}}%
\pgfpathlineto{\pgfqpoint{2.581409in}{3.359199in}}%
\pgfpathlineto{\pgfqpoint{2.611000in}{3.359199in}}%
\pgfpathlineto{\pgfqpoint{2.615932in}{3.354745in}}%
\pgfpathlineto{\pgfqpoint{2.620864in}{3.354745in}}%
\pgfpathlineto{\pgfqpoint{2.625795in}{2.768719in}}%
\pgfpathlineto{\pgfqpoint{2.680045in}{2.768719in}}%
\pgfpathlineto{\pgfqpoint{2.684977in}{3.353257in}}%
\pgfpathlineto{\pgfqpoint{2.729364in}{3.353257in}}%
\pgfpathlineto{\pgfqpoint{2.734295in}{3.096508in}}%
\pgfpathlineto{\pgfqpoint{2.788545in}{3.096508in}}%
\pgfpathlineto{\pgfqpoint{2.793477in}{2.217612in}}%
\pgfpathlineto{\pgfqpoint{2.803341in}{2.217612in}}%
\pgfpathlineto{\pgfqpoint{2.808273in}{3.325379in}}%
\pgfpathlineto{\pgfqpoint{2.897045in}{3.325379in}}%
\pgfpathlineto{\pgfqpoint{2.901977in}{3.163602in}}%
\pgfpathlineto{\pgfqpoint{2.906909in}{3.231379in}}%
\pgfpathlineto{\pgfqpoint{2.911841in}{3.231379in}}%
\pgfpathlineto{\pgfqpoint{2.916773in}{3.219593in}}%
\pgfpathlineto{\pgfqpoint{2.921705in}{1.561091in}}%
\pgfpathlineto{\pgfqpoint{2.926636in}{1.561091in}}%
\pgfpathlineto{\pgfqpoint{2.931568in}{2.854572in}}%
\pgfpathlineto{\pgfqpoint{2.946364in}{2.854572in}}%
\pgfpathlineto{\pgfqpoint{2.956227in}{3.358084in}}%
\pgfpathlineto{\pgfqpoint{2.961159in}{3.358084in}}%
\pgfpathlineto{\pgfqpoint{2.966091in}{3.035274in}}%
\pgfpathlineto{\pgfqpoint{3.000614in}{3.035274in}}%
\pgfpathlineto{\pgfqpoint{3.005545in}{3.354485in}}%
\pgfpathlineto{\pgfqpoint{3.020341in}{3.354485in}}%
\pgfpathlineto{\pgfqpoint{3.025273in}{1.846065in}}%
\pgfpathlineto{\pgfqpoint{3.040068in}{1.846065in}}%
\pgfpathlineto{\pgfqpoint{3.045000in}{2.891345in}}%
\pgfpathlineto{\pgfqpoint{3.059795in}{2.891345in}}%
\pgfpathlineto{\pgfqpoint{3.064727in}{2.269406in}}%
\pgfpathlineto{\pgfqpoint{3.099250in}{2.269406in}}%
\pgfpathlineto{\pgfqpoint{3.104182in}{2.947398in}}%
\pgfpathlineto{\pgfqpoint{3.138705in}{2.947398in}}%
\pgfpathlineto{\pgfqpoint{3.143636in}{3.268003in}}%
\pgfpathlineto{\pgfqpoint{3.192955in}{3.268003in}}%
\pgfpathlineto{\pgfqpoint{3.197886in}{3.365693in}}%
\pgfpathlineto{\pgfqpoint{3.252136in}{3.365693in}}%
\pgfpathlineto{\pgfqpoint{3.257068in}{3.131550in}}%
\pgfpathlineto{\pgfqpoint{3.262000in}{3.131550in}}%
\pgfpathlineto{\pgfqpoint{3.266932in}{2.939990in}}%
\pgfpathlineto{\pgfqpoint{3.276795in}{2.939990in}}%
\pgfpathlineto{\pgfqpoint{3.281727in}{3.134098in}}%
\pgfpathlineto{\pgfqpoint{3.360636in}{3.134098in}}%
\pgfpathlineto{\pgfqpoint{3.365568in}{3.070335in}}%
\pgfpathlineto{\pgfqpoint{3.390227in}{3.070335in}}%
\pgfpathlineto{\pgfqpoint{3.395159in}{2.778919in}}%
\pgfpathlineto{\pgfqpoint{3.400091in}{2.778919in}}%
\pgfpathlineto{\pgfqpoint{3.405023in}{3.267410in}}%
\pgfpathlineto{\pgfqpoint{3.419818in}{3.267410in}}%
\pgfpathlineto{\pgfqpoint{3.424750in}{3.379940in}}%
\pgfpathlineto{\pgfqpoint{3.429682in}{3.379940in}}%
\pgfpathlineto{\pgfqpoint{3.434614in}{3.372499in}}%
\pgfpathlineto{\pgfqpoint{3.439545in}{3.175475in}}%
\pgfpathlineto{\pgfqpoint{3.464205in}{3.175475in}}%
\pgfpathlineto{\pgfqpoint{3.469136in}{3.353455in}}%
\pgfpathlineto{\pgfqpoint{3.474068in}{3.353455in}}%
\pgfpathlineto{\pgfqpoint{3.479000in}{2.818756in}}%
\pgfpathlineto{\pgfqpoint{3.488864in}{2.818756in}}%
\pgfpathlineto{\pgfqpoint{3.493795in}{2.702313in}}%
\pgfpathlineto{\pgfqpoint{3.518455in}{2.702313in}}%
\pgfpathlineto{\pgfqpoint{3.523386in}{3.379772in}}%
\pgfpathlineto{\pgfqpoint{3.528318in}{3.379772in}}%
\pgfpathlineto{\pgfqpoint{3.533250in}{3.249846in}}%
\pgfpathlineto{\pgfqpoint{3.538182in}{2.655029in}}%
\pgfpathlineto{\pgfqpoint{3.552977in}{2.655029in}}%
\pgfpathlineto{\pgfqpoint{3.557909in}{2.256009in}}%
\pgfpathlineto{\pgfqpoint{3.587500in}{2.256009in}}%
\pgfpathlineto{\pgfqpoint{3.592432in}{3.249087in}}%
\pgfpathlineto{\pgfqpoint{3.597364in}{3.092414in}}%
\pgfpathlineto{\pgfqpoint{3.602295in}{3.092414in}}%
\pgfpathlineto{\pgfqpoint{3.607227in}{3.336142in}}%
\pgfpathlineto{\pgfqpoint{3.686136in}{3.336142in}}%
\pgfpathlineto{\pgfqpoint{3.691068in}{3.082496in}}%
\pgfpathlineto{\pgfqpoint{3.696000in}{3.072003in}}%
\pgfpathlineto{\pgfqpoint{3.700932in}{3.041761in}}%
\pgfpathlineto{\pgfqpoint{3.710795in}{3.041761in}}%
\pgfpathlineto{\pgfqpoint{3.715727in}{2.902910in}}%
\pgfpathlineto{\pgfqpoint{3.730523in}{2.902910in}}%
\pgfpathlineto{\pgfqpoint{3.735455in}{2.471726in}}%
\pgfpathlineto{\pgfqpoint{3.740386in}{2.471726in}}%
\pgfpathlineto{\pgfqpoint{3.745318in}{3.376241in}}%
\pgfpathlineto{\pgfqpoint{3.765045in}{3.376241in}}%
\pgfpathlineto{\pgfqpoint{3.769977in}{3.029027in}}%
\pgfpathlineto{\pgfqpoint{3.774909in}{3.314745in}}%
\pgfpathlineto{\pgfqpoint{3.794636in}{3.314745in}}%
\pgfpathlineto{\pgfqpoint{3.799568in}{3.225420in}}%
\pgfpathlineto{\pgfqpoint{3.814364in}{3.225420in}}%
\pgfpathlineto{\pgfqpoint{3.819295in}{3.175291in}}%
\pgfpathlineto{\pgfqpoint{3.858750in}{3.175291in}}%
\pgfpathlineto{\pgfqpoint{3.863682in}{3.378239in}}%
\pgfpathlineto{\pgfqpoint{3.873545in}{3.378239in}}%
\pgfpathlineto{\pgfqpoint{3.878477in}{3.255595in}}%
\pgfpathlineto{\pgfqpoint{3.962318in}{3.255595in}}%
\pgfpathlineto{\pgfqpoint{3.967250in}{3.369485in}}%
\pgfpathlineto{\pgfqpoint{3.972182in}{3.369485in}}%
\pgfpathlineto{\pgfqpoint{3.977114in}{3.021165in}}%
\pgfpathlineto{\pgfqpoint{3.982045in}{3.021165in}}%
\pgfpathlineto{\pgfqpoint{3.986977in}{3.128904in}}%
\pgfpathlineto{\pgfqpoint{4.016568in}{3.128904in}}%
\pgfpathlineto{\pgfqpoint{4.021500in}{3.305453in}}%
\pgfpathlineto{\pgfqpoint{4.065886in}{3.305453in}}%
\pgfpathlineto{\pgfqpoint{4.070818in}{3.336214in}}%
\pgfpathlineto{\pgfqpoint{4.075750in}{3.336214in}}%
\pgfpathlineto{\pgfqpoint{4.080682in}{3.225502in}}%
\pgfpathlineto{\pgfqpoint{4.085614in}{3.272797in}}%
\pgfpathlineto{\pgfqpoint{4.090545in}{3.272797in}}%
\pgfpathlineto{\pgfqpoint{4.095477in}{3.375606in}}%
\pgfpathlineto{\pgfqpoint{4.100409in}{2.708501in}}%
\pgfpathlineto{\pgfqpoint{4.115205in}{2.708501in}}%
\pgfpathlineto{\pgfqpoint{4.120136in}{3.379296in}}%
\pgfpathlineto{\pgfqpoint{4.149727in}{3.379296in}}%
\pgfpathlineto{\pgfqpoint{4.154659in}{3.348898in}}%
\pgfpathlineto{\pgfqpoint{4.169455in}{3.348898in}}%
\pgfpathlineto{\pgfqpoint{4.174386in}{3.370042in}}%
\pgfpathlineto{\pgfqpoint{4.189182in}{3.370042in}}%
\pgfpathlineto{\pgfqpoint{4.194114in}{2.335855in}}%
\pgfpathlineto{\pgfqpoint{4.199045in}{2.327250in}}%
\pgfpathlineto{\pgfqpoint{4.208909in}{2.327250in}}%
\pgfpathlineto{\pgfqpoint{4.213841in}{2.408628in}}%
\pgfpathlineto{\pgfqpoint{4.218773in}{3.309407in}}%
\pgfpathlineto{\pgfqpoint{4.233568in}{3.309407in}}%
\pgfpathlineto{\pgfqpoint{4.238500in}{3.024311in}}%
\pgfpathlineto{\pgfqpoint{4.243432in}{3.024311in}}%
\pgfpathlineto{\pgfqpoint{4.248364in}{2.848496in}}%
\pgfpathlineto{\pgfqpoint{4.258227in}{2.848496in}}%
\pgfpathlineto{\pgfqpoint{4.263159in}{3.267952in}}%
\pgfpathlineto{\pgfqpoint{4.273023in}{3.267952in}}%
\pgfpathlineto{\pgfqpoint{4.277955in}{3.326103in}}%
\pgfpathlineto{\pgfqpoint{4.282886in}{2.273658in}}%
\pgfpathlineto{\pgfqpoint{4.381523in}{2.273658in}}%
\pgfpathlineto{\pgfqpoint{4.386455in}{3.226184in}}%
\pgfpathlineto{\pgfqpoint{4.406182in}{3.226184in}}%
\pgfpathlineto{\pgfqpoint{4.411114in}{3.033208in}}%
\pgfpathlineto{\pgfqpoint{4.416045in}{3.292313in}}%
\pgfpathlineto{\pgfqpoint{4.420977in}{0.580000in}}%
\pgfpathlineto{\pgfqpoint{4.475227in}{0.580000in}}%
\pgfpathlineto{\pgfqpoint{4.480159in}{3.375776in}}%
\pgfpathlineto{\pgfqpoint{4.490023in}{3.375776in}}%
\pgfpathlineto{\pgfqpoint{4.494955in}{2.568142in}}%
\pgfpathlineto{\pgfqpoint{4.598523in}{2.568142in}}%
\pgfpathlineto{\pgfqpoint{4.603455in}{3.207958in}}%
\pgfpathlineto{\pgfqpoint{4.633045in}{3.207958in}}%
\pgfpathlineto{\pgfqpoint{4.637977in}{3.049475in}}%
\pgfpathlineto{\pgfqpoint{4.687295in}{3.049475in}}%
\pgfpathlineto{\pgfqpoint{4.692227in}{3.042352in}}%
\pgfpathlineto{\pgfqpoint{4.697159in}{3.366955in}}%
\pgfpathlineto{\pgfqpoint{4.702091in}{3.366955in}}%
\pgfpathlineto{\pgfqpoint{4.707023in}{3.167811in}}%
\pgfpathlineto{\pgfqpoint{4.711955in}{3.167811in}}%
\pgfpathlineto{\pgfqpoint{4.716886in}{2.703894in}}%
\pgfpathlineto{\pgfqpoint{4.781000in}{2.703894in}}%
\pgfpathlineto{\pgfqpoint{4.785932in}{1.400410in}}%
\pgfpathlineto{\pgfqpoint{4.879636in}{1.400410in}}%
\pgfpathlineto{\pgfqpoint{4.884568in}{3.086724in}}%
\pgfpathlineto{\pgfqpoint{4.914159in}{3.086724in}}%
\pgfpathlineto{\pgfqpoint{4.919091in}{3.212250in}}%
\pgfpathlineto{\pgfqpoint{4.924023in}{2.957861in}}%
\pgfpathlineto{\pgfqpoint{4.933886in}{2.957861in}}%
\pgfpathlineto{\pgfqpoint{4.938818in}{3.366962in}}%
\pgfpathlineto{\pgfqpoint{4.948682in}{3.366962in}}%
\pgfpathlineto{\pgfqpoint{4.953614in}{3.301005in}}%
\pgfpathlineto{\pgfqpoint{4.958545in}{3.301005in}}%
\pgfpathlineto{\pgfqpoint{4.963477in}{3.296173in}}%
\pgfpathlineto{\pgfqpoint{4.968409in}{3.296173in}}%
\pgfpathlineto{\pgfqpoint{4.973341in}{3.096353in}}%
\pgfpathlineto{\pgfqpoint{5.062114in}{3.096353in}}%
\pgfpathlineto{\pgfqpoint{5.067045in}{3.310703in}}%
\pgfpathlineto{\pgfqpoint{5.076909in}{3.310703in}}%
\pgfpathlineto{\pgfqpoint{5.081841in}{3.249988in}}%
\pgfpathlineto{\pgfqpoint{5.086773in}{3.369042in}}%
\pgfpathlineto{\pgfqpoint{5.096636in}{3.369042in}}%
\pgfpathlineto{\pgfqpoint{5.101568in}{3.342679in}}%
\pgfpathlineto{\pgfqpoint{5.106500in}{3.342679in}}%
\pgfpathlineto{\pgfqpoint{5.111432in}{3.060078in}}%
\pgfpathlineto{\pgfqpoint{5.160750in}{3.060078in}}%
\pgfpathlineto{\pgfqpoint{5.165682in}{3.357733in}}%
\pgfpathlineto{\pgfqpoint{5.219932in}{3.357733in}}%
\pgfpathlineto{\pgfqpoint{5.224864in}{3.008538in}}%
\pgfpathlineto{\pgfqpoint{5.229795in}{3.341372in}}%
\pgfpathlineto{\pgfqpoint{5.234727in}{3.353177in}}%
\pgfpathlineto{\pgfqpoint{5.249523in}{3.353177in}}%
\pgfpathlineto{\pgfqpoint{5.254455in}{2.342755in}}%
\pgfpathlineto{\pgfqpoint{5.259386in}{3.349068in}}%
\pgfpathlineto{\pgfqpoint{5.264318in}{3.349068in}}%
\pgfpathlineto{\pgfqpoint{5.269250in}{2.627604in}}%
\pgfpathlineto{\pgfqpoint{5.274182in}{3.375470in}}%
\pgfpathlineto{\pgfqpoint{5.279114in}{2.956408in}}%
\pgfpathlineto{\pgfqpoint{5.284045in}{3.379945in}}%
\pgfpathlineto{\pgfqpoint{5.288977in}{3.379945in}}%
\pgfpathlineto{\pgfqpoint{5.293909in}{3.376692in}}%
\pgfpathlineto{\pgfqpoint{5.303773in}{3.376692in}}%
\pgfpathlineto{\pgfqpoint{5.308705in}{3.199522in}}%
\pgfpathlineto{\pgfqpoint{5.333364in}{3.199522in}}%
\pgfpathlineto{\pgfqpoint{5.338295in}{3.265494in}}%
\pgfpathlineto{\pgfqpoint{5.377750in}{3.265494in}}%
\pgfpathlineto{\pgfqpoint{5.382682in}{2.556052in}}%
\pgfpathlineto{\pgfqpoint{5.387614in}{2.556052in}}%
\pgfpathlineto{\pgfqpoint{5.392545in}{3.377780in}}%
\pgfpathlineto{\pgfqpoint{5.407341in}{3.377780in}}%
\pgfpathlineto{\pgfqpoint{5.412273in}{2.597656in}}%
\pgfpathlineto{\pgfqpoint{5.417205in}{3.376768in}}%
\pgfpathlineto{\pgfqpoint{5.446795in}{3.376768in}}%
\pgfpathlineto{\pgfqpoint{5.451727in}{3.022696in}}%
\pgfpathlineto{\pgfqpoint{5.456659in}{3.300445in}}%
\pgfpathlineto{\pgfqpoint{5.466523in}{3.300445in}}%
\pgfpathlineto{\pgfqpoint{5.471455in}{3.371759in}}%
\pgfpathlineto{\pgfqpoint{5.491182in}{3.371759in}}%
\pgfpathlineto{\pgfqpoint{5.496114in}{3.234138in}}%
\pgfpathlineto{\pgfqpoint{5.575023in}{3.234138in}}%
\pgfpathlineto{\pgfqpoint{5.579955in}{2.842560in}}%
\pgfpathlineto{\pgfqpoint{5.589818in}{2.842560in}}%
\pgfpathlineto{\pgfqpoint{5.594750in}{2.994899in}}%
\pgfpathlineto{\pgfqpoint{5.599682in}{3.312337in}}%
\pgfpathlineto{\pgfqpoint{5.634205in}{3.312337in}}%
\pgfpathlineto{\pgfqpoint{5.639136in}{3.217657in}}%
\pgfpathlineto{\pgfqpoint{5.713114in}{3.217657in}}%
\pgfpathlineto{\pgfqpoint{5.718045in}{3.265762in}}%
\pgfpathlineto{\pgfqpoint{5.722977in}{3.374394in}}%
\pgfpathlineto{\pgfqpoint{5.727909in}{3.374394in}}%
\pgfpathlineto{\pgfqpoint{5.732841in}{2.594338in}}%
\pgfpathlineto{\pgfqpoint{5.762432in}{2.594338in}}%
\pgfpathlineto{\pgfqpoint{5.767364in}{3.269655in}}%
\pgfpathlineto{\pgfqpoint{5.792023in}{3.269655in}}%
\pgfpathlineto{\pgfqpoint{5.796955in}{3.379365in}}%
\pgfpathlineto{\pgfqpoint{5.841341in}{3.379365in}}%
\pgfpathlineto{\pgfqpoint{5.846273in}{3.343809in}}%
\pgfpathlineto{\pgfqpoint{5.870932in}{3.343809in}}%
\pgfpathlineto{\pgfqpoint{5.875864in}{3.314055in}}%
\pgfpathlineto{\pgfqpoint{5.915318in}{3.314055in}}%
\pgfpathlineto{\pgfqpoint{5.920250in}{2.181245in}}%
\pgfpathlineto{\pgfqpoint{5.925182in}{3.218264in}}%
\pgfpathlineto{\pgfqpoint{5.984364in}{3.218264in}}%
\pgfpathlineto{\pgfqpoint{5.989295in}{3.303369in}}%
\pgfpathlineto{\pgfqpoint{6.023818in}{3.303369in}}%
\pgfpathlineto{\pgfqpoint{6.028750in}{2.694851in}}%
\pgfpathlineto{\pgfqpoint{6.033682in}{2.694851in}}%
\pgfpathlineto{\pgfqpoint{6.038614in}{3.316152in}}%
\pgfpathlineto{\pgfqpoint{6.053409in}{3.316152in}}%
\pgfpathlineto{\pgfqpoint{6.053409in}{3.316152in}}%
\pgfusepath{stroke}%
\end{pgfscope}%
\begin{pgfscope}%
\pgfsetrectcap%
\pgfsetmiterjoin%
\pgfsetlinewidth{0.000000pt}%
\definecolor{currentstroke}{rgb}{1.000000,1.000000,1.000000}%
\pgfsetstrokecolor{currentstroke}%
\pgfsetdash{}{0pt}%
\pgfpathmoveto{\pgfqpoint{0.875000in}{0.440000in}}%
\pgfpathlineto{\pgfqpoint{0.875000in}{3.520000in}}%
\pgfusepath{}%
\end{pgfscope}%
\begin{pgfscope}%
\pgfsetrectcap%
\pgfsetmiterjoin%
\pgfsetlinewidth{0.000000pt}%
\definecolor{currentstroke}{rgb}{1.000000,1.000000,1.000000}%
\pgfsetstrokecolor{currentstroke}%
\pgfsetdash{}{0pt}%
\pgfpathmoveto{\pgfqpoint{6.300000in}{0.440000in}}%
\pgfpathlineto{\pgfqpoint{6.300000in}{3.520000in}}%
\pgfusepath{}%
\end{pgfscope}%
\begin{pgfscope}%
\pgfsetrectcap%
\pgfsetmiterjoin%
\pgfsetlinewidth{0.000000pt}%
\definecolor{currentstroke}{rgb}{1.000000,1.000000,1.000000}%
\pgfsetstrokecolor{currentstroke}%
\pgfsetdash{}{0pt}%
\pgfpathmoveto{\pgfqpoint{0.875000in}{0.440000in}}%
\pgfpathlineto{\pgfqpoint{6.300000in}{0.440000in}}%
\pgfusepath{}%
\end{pgfscope}%
\begin{pgfscope}%
\pgfsetrectcap%
\pgfsetmiterjoin%
\pgfsetlinewidth{0.000000pt}%
\definecolor{currentstroke}{rgb}{1.000000,1.000000,1.000000}%
\pgfsetstrokecolor{currentstroke}%
\pgfsetdash{}{0pt}%
\pgfpathmoveto{\pgfqpoint{0.875000in}{3.520000in}}%
\pgfpathlineto{\pgfqpoint{6.300000in}{3.520000in}}%
\pgfusepath{}%
\end{pgfscope}%
\begin{pgfscope}%
\definecolor{textcolor}{rgb}{0.150000,0.150000,0.150000}%
\pgfsetstrokecolor{textcolor}%
\pgfsetfillcolor{textcolor}%
\pgftext[x=3.500000in,y=3.920000in,,top]{\color{textcolor}\rmfamily\fontsize{12.000000}{14.400000}\selectfont Gráfica de \(\displaystyle \log f(X_t)\) para propuesta beta, \(\displaystyle n=5, r=\)3}%
\end{pgfscope}%
\end{pgfpicture}%
\makeatother%
\endgroup%

        %% Creator: Matplotlib, PGF backend
%%
%% To include the figure in your LaTeX document, write
%%   \input{<filename>.pgf}
%%
%% Make sure the required packages are loaded in your preamble
%%   \usepackage{pgf}
%%
%% Also ensure that all the required font packages are loaded; for instance,
%% the lmodern package is sometimes necessary when using math font.
%%   \usepackage{lmodern}
%%
%% Figures using additional raster images can only be included by \input if
%% they are in the same directory as the main LaTeX file. For loading figures
%% from other directories you can use the `import` package
%%   \usepackage{import}
%%
%% and then include the figures with
%%   \import{<path to file>}{<filename>.pgf}
%%
%% Matplotlib used the following preamble
%%   
%%   \makeatletter\@ifpackageloaded{underscore}{}{\usepackage[strings]{underscore}}\makeatother
%%
\begingroup%
\makeatletter%
\begin{pgfpicture}%
\pgfpathrectangle{\pgfpointorigin}{\pgfqpoint{7.000000in}{4.000000in}}%
\pgfusepath{use as bounding box, clip}%
\begin{pgfscope}%
\pgfsetbuttcap%
\pgfsetmiterjoin%
\definecolor{currentfill}{rgb}{1.000000,1.000000,1.000000}%
\pgfsetfillcolor{currentfill}%
\pgfsetlinewidth{0.000000pt}%
\definecolor{currentstroke}{rgb}{1.000000,1.000000,1.000000}%
\pgfsetstrokecolor{currentstroke}%
\pgfsetdash{}{0pt}%
\pgfpathmoveto{\pgfqpoint{0.000000in}{0.000000in}}%
\pgfpathlineto{\pgfqpoint{7.000000in}{0.000000in}}%
\pgfpathlineto{\pgfqpoint{7.000000in}{4.000000in}}%
\pgfpathlineto{\pgfqpoint{0.000000in}{4.000000in}}%
\pgfpathlineto{\pgfqpoint{0.000000in}{0.000000in}}%
\pgfpathclose%
\pgfusepath{fill}%
\end{pgfscope}%
\begin{pgfscope}%
\pgfsetbuttcap%
\pgfsetmiterjoin%
\definecolor{currentfill}{rgb}{0.917647,0.917647,0.949020}%
\pgfsetfillcolor{currentfill}%
\pgfsetlinewidth{0.000000pt}%
\definecolor{currentstroke}{rgb}{0.000000,0.000000,0.000000}%
\pgfsetstrokecolor{currentstroke}%
\pgfsetstrokeopacity{0.000000}%
\pgfsetdash{}{0pt}%
\pgfpathmoveto{\pgfqpoint{0.875000in}{0.440000in}}%
\pgfpathlineto{\pgfqpoint{6.300000in}{0.440000in}}%
\pgfpathlineto{\pgfqpoint{6.300000in}{3.520000in}}%
\pgfpathlineto{\pgfqpoint{0.875000in}{3.520000in}}%
\pgfpathlineto{\pgfqpoint{0.875000in}{0.440000in}}%
\pgfpathclose%
\pgfusepath{fill}%
\end{pgfscope}%
\begin{pgfscope}%
\pgfpathrectangle{\pgfqpoint{0.875000in}{0.440000in}}{\pgfqpoint{5.425000in}{3.080000in}}%
\pgfusepath{clip}%
\pgfsetroundcap%
\pgfsetroundjoin%
\pgfsetlinewidth{1.003750pt}%
\definecolor{currentstroke}{rgb}{1.000000,1.000000,1.000000}%
\pgfsetstrokecolor{currentstroke}%
\pgfsetdash{}{0pt}%
\pgfpathmoveto{\pgfqpoint{1.121591in}{0.440000in}}%
\pgfpathlineto{\pgfqpoint{1.121591in}{3.520000in}}%
\pgfusepath{stroke}%
\end{pgfscope}%
\begin{pgfscope}%
\definecolor{textcolor}{rgb}{0.150000,0.150000,0.150000}%
\pgfsetstrokecolor{textcolor}%
\pgfsetfillcolor{textcolor}%
\pgftext[x=1.121591in,y=0.342778in,,top]{\color{textcolor}\rmfamily\fontsize{10.000000}{12.000000}\selectfont \(\displaystyle {0}\)}%
\end{pgfscope}%
\begin{pgfscope}%
\pgfpathrectangle{\pgfqpoint{0.875000in}{0.440000in}}{\pgfqpoint{5.425000in}{3.080000in}}%
\pgfusepath{clip}%
\pgfsetroundcap%
\pgfsetroundjoin%
\pgfsetlinewidth{1.003750pt}%
\definecolor{currentstroke}{rgb}{1.000000,1.000000,1.000000}%
\pgfsetstrokecolor{currentstroke}%
\pgfsetdash{}{0pt}%
\pgfpathmoveto{\pgfqpoint{2.107955in}{0.440000in}}%
\pgfpathlineto{\pgfqpoint{2.107955in}{3.520000in}}%
\pgfusepath{stroke}%
\end{pgfscope}%
\begin{pgfscope}%
\definecolor{textcolor}{rgb}{0.150000,0.150000,0.150000}%
\pgfsetstrokecolor{textcolor}%
\pgfsetfillcolor{textcolor}%
\pgftext[x=2.107955in,y=0.342778in,,top]{\color{textcolor}\rmfamily\fontsize{10.000000}{12.000000}\selectfont \(\displaystyle {200}\)}%
\end{pgfscope}%
\begin{pgfscope}%
\pgfpathrectangle{\pgfqpoint{0.875000in}{0.440000in}}{\pgfqpoint{5.425000in}{3.080000in}}%
\pgfusepath{clip}%
\pgfsetroundcap%
\pgfsetroundjoin%
\pgfsetlinewidth{1.003750pt}%
\definecolor{currentstroke}{rgb}{1.000000,1.000000,1.000000}%
\pgfsetstrokecolor{currentstroke}%
\pgfsetdash{}{0pt}%
\pgfpathmoveto{\pgfqpoint{3.094318in}{0.440000in}}%
\pgfpathlineto{\pgfqpoint{3.094318in}{3.520000in}}%
\pgfusepath{stroke}%
\end{pgfscope}%
\begin{pgfscope}%
\definecolor{textcolor}{rgb}{0.150000,0.150000,0.150000}%
\pgfsetstrokecolor{textcolor}%
\pgfsetfillcolor{textcolor}%
\pgftext[x=3.094318in,y=0.342778in,,top]{\color{textcolor}\rmfamily\fontsize{10.000000}{12.000000}\selectfont \(\displaystyle {400}\)}%
\end{pgfscope}%
\begin{pgfscope}%
\pgfpathrectangle{\pgfqpoint{0.875000in}{0.440000in}}{\pgfqpoint{5.425000in}{3.080000in}}%
\pgfusepath{clip}%
\pgfsetroundcap%
\pgfsetroundjoin%
\pgfsetlinewidth{1.003750pt}%
\definecolor{currentstroke}{rgb}{1.000000,1.000000,1.000000}%
\pgfsetstrokecolor{currentstroke}%
\pgfsetdash{}{0pt}%
\pgfpathmoveto{\pgfqpoint{4.080682in}{0.440000in}}%
\pgfpathlineto{\pgfqpoint{4.080682in}{3.520000in}}%
\pgfusepath{stroke}%
\end{pgfscope}%
\begin{pgfscope}%
\definecolor{textcolor}{rgb}{0.150000,0.150000,0.150000}%
\pgfsetstrokecolor{textcolor}%
\pgfsetfillcolor{textcolor}%
\pgftext[x=4.080682in,y=0.342778in,,top]{\color{textcolor}\rmfamily\fontsize{10.000000}{12.000000}\selectfont \(\displaystyle {600}\)}%
\end{pgfscope}%
\begin{pgfscope}%
\pgfpathrectangle{\pgfqpoint{0.875000in}{0.440000in}}{\pgfqpoint{5.425000in}{3.080000in}}%
\pgfusepath{clip}%
\pgfsetroundcap%
\pgfsetroundjoin%
\pgfsetlinewidth{1.003750pt}%
\definecolor{currentstroke}{rgb}{1.000000,1.000000,1.000000}%
\pgfsetstrokecolor{currentstroke}%
\pgfsetdash{}{0pt}%
\pgfpathmoveto{\pgfqpoint{5.067045in}{0.440000in}}%
\pgfpathlineto{\pgfqpoint{5.067045in}{3.520000in}}%
\pgfusepath{stroke}%
\end{pgfscope}%
\begin{pgfscope}%
\definecolor{textcolor}{rgb}{0.150000,0.150000,0.150000}%
\pgfsetstrokecolor{textcolor}%
\pgfsetfillcolor{textcolor}%
\pgftext[x=5.067045in,y=0.342778in,,top]{\color{textcolor}\rmfamily\fontsize{10.000000}{12.000000}\selectfont \(\displaystyle {800}\)}%
\end{pgfscope}%
\begin{pgfscope}%
\pgfpathrectangle{\pgfqpoint{0.875000in}{0.440000in}}{\pgfqpoint{5.425000in}{3.080000in}}%
\pgfusepath{clip}%
\pgfsetroundcap%
\pgfsetroundjoin%
\pgfsetlinewidth{1.003750pt}%
\definecolor{currentstroke}{rgb}{1.000000,1.000000,1.000000}%
\pgfsetstrokecolor{currentstroke}%
\pgfsetdash{}{0pt}%
\pgfpathmoveto{\pgfqpoint{6.053409in}{0.440000in}}%
\pgfpathlineto{\pgfqpoint{6.053409in}{3.520000in}}%
\pgfusepath{stroke}%
\end{pgfscope}%
\begin{pgfscope}%
\definecolor{textcolor}{rgb}{0.150000,0.150000,0.150000}%
\pgfsetstrokecolor{textcolor}%
\pgfsetfillcolor{textcolor}%
\pgftext[x=6.053409in,y=0.342778in,,top]{\color{textcolor}\rmfamily\fontsize{10.000000}{12.000000}\selectfont \(\displaystyle {1000}\)}%
\end{pgfscope}%
\begin{pgfscope}%
\pgfpathrectangle{\pgfqpoint{0.875000in}{0.440000in}}{\pgfqpoint{5.425000in}{3.080000in}}%
\pgfusepath{clip}%
\pgfsetroundcap%
\pgfsetroundjoin%
\pgfsetlinewidth{1.003750pt}%
\definecolor{currentstroke}{rgb}{1.000000,1.000000,1.000000}%
\pgfsetstrokecolor{currentstroke}%
\pgfsetdash{}{0pt}%
\pgfpathmoveto{\pgfqpoint{0.875000in}{0.771938in}}%
\pgfpathlineto{\pgfqpoint{6.300000in}{0.771938in}}%
\pgfusepath{stroke}%
\end{pgfscope}%
\begin{pgfscope}%
\definecolor{textcolor}{rgb}{0.150000,0.150000,0.150000}%
\pgfsetstrokecolor{textcolor}%
\pgfsetfillcolor{textcolor}%
\pgftext[x=0.530863in, y=0.723713in, left, base]{\color{textcolor}\rmfamily\fontsize{10.000000}{12.000000}\selectfont \(\displaystyle {\ensuremath{-}55}\)}%
\end{pgfscope}%
\begin{pgfscope}%
\pgfpathrectangle{\pgfqpoint{0.875000in}{0.440000in}}{\pgfqpoint{5.425000in}{3.080000in}}%
\pgfusepath{clip}%
\pgfsetroundcap%
\pgfsetroundjoin%
\pgfsetlinewidth{1.003750pt}%
\definecolor{currentstroke}{rgb}{1.000000,1.000000,1.000000}%
\pgfsetstrokecolor{currentstroke}%
\pgfsetdash{}{0pt}%
\pgfpathmoveto{\pgfqpoint{0.875000in}{1.242703in}}%
\pgfpathlineto{\pgfqpoint{6.300000in}{1.242703in}}%
\pgfusepath{stroke}%
\end{pgfscope}%
\begin{pgfscope}%
\definecolor{textcolor}{rgb}{0.150000,0.150000,0.150000}%
\pgfsetstrokecolor{textcolor}%
\pgfsetfillcolor{textcolor}%
\pgftext[x=0.530863in, y=1.194478in, left, base]{\color{textcolor}\rmfamily\fontsize{10.000000}{12.000000}\selectfont \(\displaystyle {\ensuremath{-}50}\)}%
\end{pgfscope}%
\begin{pgfscope}%
\pgfpathrectangle{\pgfqpoint{0.875000in}{0.440000in}}{\pgfqpoint{5.425000in}{3.080000in}}%
\pgfusepath{clip}%
\pgfsetroundcap%
\pgfsetroundjoin%
\pgfsetlinewidth{1.003750pt}%
\definecolor{currentstroke}{rgb}{1.000000,1.000000,1.000000}%
\pgfsetstrokecolor{currentstroke}%
\pgfsetdash{}{0pt}%
\pgfpathmoveto{\pgfqpoint{0.875000in}{1.713468in}}%
\pgfpathlineto{\pgfqpoint{6.300000in}{1.713468in}}%
\pgfusepath{stroke}%
\end{pgfscope}%
\begin{pgfscope}%
\definecolor{textcolor}{rgb}{0.150000,0.150000,0.150000}%
\pgfsetstrokecolor{textcolor}%
\pgfsetfillcolor{textcolor}%
\pgftext[x=0.530863in, y=1.665243in, left, base]{\color{textcolor}\rmfamily\fontsize{10.000000}{12.000000}\selectfont \(\displaystyle {\ensuremath{-}45}\)}%
\end{pgfscope}%
\begin{pgfscope}%
\pgfpathrectangle{\pgfqpoint{0.875000in}{0.440000in}}{\pgfqpoint{5.425000in}{3.080000in}}%
\pgfusepath{clip}%
\pgfsetroundcap%
\pgfsetroundjoin%
\pgfsetlinewidth{1.003750pt}%
\definecolor{currentstroke}{rgb}{1.000000,1.000000,1.000000}%
\pgfsetstrokecolor{currentstroke}%
\pgfsetdash{}{0pt}%
\pgfpathmoveto{\pgfqpoint{0.875000in}{2.184233in}}%
\pgfpathlineto{\pgfqpoint{6.300000in}{2.184233in}}%
\pgfusepath{stroke}%
\end{pgfscope}%
\begin{pgfscope}%
\definecolor{textcolor}{rgb}{0.150000,0.150000,0.150000}%
\pgfsetstrokecolor{textcolor}%
\pgfsetfillcolor{textcolor}%
\pgftext[x=0.530863in, y=2.136008in, left, base]{\color{textcolor}\rmfamily\fontsize{10.000000}{12.000000}\selectfont \(\displaystyle {\ensuremath{-}40}\)}%
\end{pgfscope}%
\begin{pgfscope}%
\pgfpathrectangle{\pgfqpoint{0.875000in}{0.440000in}}{\pgfqpoint{5.425000in}{3.080000in}}%
\pgfusepath{clip}%
\pgfsetroundcap%
\pgfsetroundjoin%
\pgfsetlinewidth{1.003750pt}%
\definecolor{currentstroke}{rgb}{1.000000,1.000000,1.000000}%
\pgfsetstrokecolor{currentstroke}%
\pgfsetdash{}{0pt}%
\pgfpathmoveto{\pgfqpoint{0.875000in}{2.654999in}}%
\pgfpathlineto{\pgfqpoint{6.300000in}{2.654999in}}%
\pgfusepath{stroke}%
\end{pgfscope}%
\begin{pgfscope}%
\definecolor{textcolor}{rgb}{0.150000,0.150000,0.150000}%
\pgfsetstrokecolor{textcolor}%
\pgfsetfillcolor{textcolor}%
\pgftext[x=0.530863in, y=2.606773in, left, base]{\color{textcolor}\rmfamily\fontsize{10.000000}{12.000000}\selectfont \(\displaystyle {\ensuremath{-}35}\)}%
\end{pgfscope}%
\begin{pgfscope}%
\pgfpathrectangle{\pgfqpoint{0.875000in}{0.440000in}}{\pgfqpoint{5.425000in}{3.080000in}}%
\pgfusepath{clip}%
\pgfsetroundcap%
\pgfsetroundjoin%
\pgfsetlinewidth{1.003750pt}%
\definecolor{currentstroke}{rgb}{1.000000,1.000000,1.000000}%
\pgfsetstrokecolor{currentstroke}%
\pgfsetdash{}{0pt}%
\pgfpathmoveto{\pgfqpoint{0.875000in}{3.125764in}}%
\pgfpathlineto{\pgfqpoint{6.300000in}{3.125764in}}%
\pgfusepath{stroke}%
\end{pgfscope}%
\begin{pgfscope}%
\definecolor{textcolor}{rgb}{0.150000,0.150000,0.150000}%
\pgfsetstrokecolor{textcolor}%
\pgfsetfillcolor{textcolor}%
\pgftext[x=0.530863in, y=3.077538in, left, base]{\color{textcolor}\rmfamily\fontsize{10.000000}{12.000000}\selectfont \(\displaystyle {\ensuremath{-}30}\)}%
\end{pgfscope}%
\begin{pgfscope}%
\pgfpathrectangle{\pgfqpoint{0.875000in}{0.440000in}}{\pgfqpoint{5.425000in}{3.080000in}}%
\pgfusepath{clip}%
\pgfsetroundcap%
\pgfsetroundjoin%
\pgfsetlinewidth{1.756562pt}%
\definecolor{currentstroke}{rgb}{0.298039,0.447059,0.690196}%
\pgfsetstrokecolor{currentstroke}%
\pgfsetdash{}{0pt}%
\pgfpathmoveto{\pgfqpoint{1.121591in}{0.580000in}}%
\pgfpathlineto{\pgfqpoint{1.151182in}{0.580000in}}%
\pgfpathlineto{\pgfqpoint{1.156114in}{3.379366in}}%
\pgfpathlineto{\pgfqpoint{1.161045in}{3.356449in}}%
\pgfpathlineto{\pgfqpoint{1.165977in}{3.377705in}}%
\pgfpathlineto{\pgfqpoint{1.170909in}{3.376406in}}%
\pgfpathlineto{\pgfqpoint{1.175841in}{3.172951in}}%
\pgfpathlineto{\pgfqpoint{1.180773in}{3.172951in}}%
\pgfpathlineto{\pgfqpoint{1.185705in}{3.379990in}}%
\pgfpathlineto{\pgfqpoint{1.190636in}{3.355069in}}%
\pgfpathlineto{\pgfqpoint{1.200500in}{3.355069in}}%
\pgfpathlineto{\pgfqpoint{1.205432in}{3.379932in}}%
\pgfpathlineto{\pgfqpoint{1.210364in}{3.349150in}}%
\pgfpathlineto{\pgfqpoint{1.215295in}{3.369729in}}%
\pgfpathlineto{\pgfqpoint{1.220227in}{3.369729in}}%
\pgfpathlineto{\pgfqpoint{1.225159in}{3.256396in}}%
\pgfpathlineto{\pgfqpoint{1.230091in}{3.377739in}}%
\pgfpathlineto{\pgfqpoint{1.235023in}{3.318194in}}%
\pgfpathlineto{\pgfqpoint{1.239955in}{3.318194in}}%
\pgfpathlineto{\pgfqpoint{1.244886in}{3.325151in}}%
\pgfpathlineto{\pgfqpoint{1.249818in}{3.379971in}}%
\pgfpathlineto{\pgfqpoint{1.254750in}{3.366113in}}%
\pgfpathlineto{\pgfqpoint{1.259682in}{3.366113in}}%
\pgfpathlineto{\pgfqpoint{1.264614in}{3.378268in}}%
\pgfpathlineto{\pgfqpoint{1.269545in}{3.378268in}}%
\pgfpathlineto{\pgfqpoint{1.274477in}{3.376722in}}%
\pgfpathlineto{\pgfqpoint{1.279409in}{3.342751in}}%
\pgfpathlineto{\pgfqpoint{1.284341in}{3.376931in}}%
\pgfpathlineto{\pgfqpoint{1.294205in}{3.376931in}}%
\pgfpathlineto{\pgfqpoint{1.299136in}{3.374710in}}%
\pgfpathlineto{\pgfqpoint{1.304068in}{3.377383in}}%
\pgfpathlineto{\pgfqpoint{1.309000in}{3.377383in}}%
\pgfpathlineto{\pgfqpoint{1.313932in}{3.362860in}}%
\pgfpathlineto{\pgfqpoint{1.318864in}{3.379435in}}%
\pgfpathlineto{\pgfqpoint{1.323795in}{3.379435in}}%
\pgfpathlineto{\pgfqpoint{1.328727in}{3.200257in}}%
\pgfpathlineto{\pgfqpoint{1.333659in}{3.375514in}}%
\pgfpathlineto{\pgfqpoint{1.338591in}{3.376408in}}%
\pgfpathlineto{\pgfqpoint{1.343523in}{3.370216in}}%
\pgfpathlineto{\pgfqpoint{1.348455in}{3.370216in}}%
\pgfpathlineto{\pgfqpoint{1.353386in}{3.364339in}}%
\pgfpathlineto{\pgfqpoint{1.358318in}{3.364339in}}%
\pgfpathlineto{\pgfqpoint{1.363250in}{3.357609in}}%
\pgfpathlineto{\pgfqpoint{1.368182in}{3.361712in}}%
\pgfpathlineto{\pgfqpoint{1.373114in}{3.372785in}}%
\pgfpathlineto{\pgfqpoint{1.382977in}{3.335053in}}%
\pgfpathlineto{\pgfqpoint{1.392841in}{3.339480in}}%
\pgfpathlineto{\pgfqpoint{1.397773in}{3.264794in}}%
\pgfpathlineto{\pgfqpoint{1.402705in}{3.366962in}}%
\pgfpathlineto{\pgfqpoint{1.407636in}{3.366962in}}%
\pgfpathlineto{\pgfqpoint{1.412568in}{3.280376in}}%
\pgfpathlineto{\pgfqpoint{1.427364in}{3.280376in}}%
\pgfpathlineto{\pgfqpoint{1.432295in}{3.379931in}}%
\pgfpathlineto{\pgfqpoint{1.437227in}{3.377651in}}%
\pgfpathlineto{\pgfqpoint{1.442159in}{3.374111in}}%
\pgfpathlineto{\pgfqpoint{1.447091in}{3.351977in}}%
\pgfpathlineto{\pgfqpoint{1.452023in}{3.359018in}}%
\pgfpathlineto{\pgfqpoint{1.456955in}{3.371899in}}%
\pgfpathlineto{\pgfqpoint{1.461886in}{3.371899in}}%
\pgfpathlineto{\pgfqpoint{1.466818in}{3.379373in}}%
\pgfpathlineto{\pgfqpoint{1.471750in}{3.253829in}}%
\pgfpathlineto{\pgfqpoint{1.501341in}{3.253829in}}%
\pgfpathlineto{\pgfqpoint{1.506273in}{3.379635in}}%
\pgfpathlineto{\pgfqpoint{1.511205in}{3.379635in}}%
\pgfpathlineto{\pgfqpoint{1.516136in}{3.372341in}}%
\pgfpathlineto{\pgfqpoint{1.521068in}{3.180984in}}%
\pgfpathlineto{\pgfqpoint{1.526000in}{3.358487in}}%
\pgfpathlineto{\pgfqpoint{1.530932in}{3.332830in}}%
\pgfpathlineto{\pgfqpoint{1.535864in}{3.374007in}}%
\pgfpathlineto{\pgfqpoint{1.540795in}{3.379954in}}%
\pgfpathlineto{\pgfqpoint{1.545727in}{3.317737in}}%
\pgfpathlineto{\pgfqpoint{1.550659in}{3.379488in}}%
\pgfpathlineto{\pgfqpoint{1.555591in}{3.375152in}}%
\pgfpathlineto{\pgfqpoint{1.560523in}{3.279855in}}%
\pgfpathlineto{\pgfqpoint{1.565455in}{3.221489in}}%
\pgfpathlineto{\pgfqpoint{1.570386in}{3.376349in}}%
\pgfpathlineto{\pgfqpoint{1.575318in}{3.347520in}}%
\pgfpathlineto{\pgfqpoint{1.595045in}{3.347520in}}%
\pgfpathlineto{\pgfqpoint{1.599977in}{3.366560in}}%
\pgfpathlineto{\pgfqpoint{1.604909in}{3.366560in}}%
\pgfpathlineto{\pgfqpoint{1.609841in}{3.376840in}}%
\pgfpathlineto{\pgfqpoint{1.614773in}{3.376986in}}%
\pgfpathlineto{\pgfqpoint{1.619705in}{3.379709in}}%
\pgfpathlineto{\pgfqpoint{1.624636in}{3.379709in}}%
\pgfpathlineto{\pgfqpoint{1.629568in}{3.376478in}}%
\pgfpathlineto{\pgfqpoint{1.634500in}{3.226093in}}%
\pgfpathlineto{\pgfqpoint{1.639432in}{3.226093in}}%
\pgfpathlineto{\pgfqpoint{1.644364in}{3.346390in}}%
\pgfpathlineto{\pgfqpoint{1.649295in}{3.346390in}}%
\pgfpathlineto{\pgfqpoint{1.654227in}{3.379244in}}%
\pgfpathlineto{\pgfqpoint{1.659159in}{3.246480in}}%
\pgfpathlineto{\pgfqpoint{1.664091in}{3.352948in}}%
\pgfpathlineto{\pgfqpoint{1.669023in}{3.322933in}}%
\pgfpathlineto{\pgfqpoint{1.673955in}{3.363810in}}%
\pgfpathlineto{\pgfqpoint{1.678886in}{3.366012in}}%
\pgfpathlineto{\pgfqpoint{1.683818in}{3.322648in}}%
\pgfpathlineto{\pgfqpoint{1.693682in}{3.322648in}}%
\pgfpathlineto{\pgfqpoint{1.698614in}{3.355540in}}%
\pgfpathlineto{\pgfqpoint{1.703545in}{3.355436in}}%
\pgfpathlineto{\pgfqpoint{1.708477in}{3.268962in}}%
\pgfpathlineto{\pgfqpoint{1.713409in}{3.345746in}}%
\pgfpathlineto{\pgfqpoint{1.718341in}{3.379613in}}%
\pgfpathlineto{\pgfqpoint{1.723273in}{3.379613in}}%
\pgfpathlineto{\pgfqpoint{1.728205in}{3.253172in}}%
\pgfpathlineto{\pgfqpoint{1.733136in}{3.253172in}}%
\pgfpathlineto{\pgfqpoint{1.738068in}{3.378974in}}%
\pgfpathlineto{\pgfqpoint{1.743000in}{3.330808in}}%
\pgfpathlineto{\pgfqpoint{1.747932in}{3.322086in}}%
\pgfpathlineto{\pgfqpoint{1.752864in}{3.257337in}}%
\pgfpathlineto{\pgfqpoint{1.757795in}{3.369883in}}%
\pgfpathlineto{\pgfqpoint{1.762727in}{3.378888in}}%
\pgfpathlineto{\pgfqpoint{1.767659in}{3.351078in}}%
\pgfpathlineto{\pgfqpoint{1.772591in}{3.378210in}}%
\pgfpathlineto{\pgfqpoint{1.777523in}{3.135317in}}%
\pgfpathlineto{\pgfqpoint{1.782455in}{3.135317in}}%
\pgfpathlineto{\pgfqpoint{1.787386in}{3.343272in}}%
\pgfpathlineto{\pgfqpoint{1.797250in}{3.343272in}}%
\pgfpathlineto{\pgfqpoint{1.802182in}{3.267901in}}%
\pgfpathlineto{\pgfqpoint{1.807114in}{3.342373in}}%
\pgfpathlineto{\pgfqpoint{1.812045in}{3.349409in}}%
\pgfpathlineto{\pgfqpoint{1.821909in}{3.377745in}}%
\pgfpathlineto{\pgfqpoint{1.826841in}{3.378709in}}%
\pgfpathlineto{\pgfqpoint{1.831773in}{3.357987in}}%
\pgfpathlineto{\pgfqpoint{1.836705in}{3.357987in}}%
\pgfpathlineto{\pgfqpoint{1.841636in}{3.364488in}}%
\pgfpathlineto{\pgfqpoint{1.846568in}{3.214851in}}%
\pgfpathlineto{\pgfqpoint{1.851500in}{3.346615in}}%
\pgfpathlineto{\pgfqpoint{1.856432in}{3.237845in}}%
\pgfpathlineto{\pgfqpoint{1.861364in}{3.305079in}}%
\pgfpathlineto{\pgfqpoint{1.866295in}{3.305079in}}%
\pgfpathlineto{\pgfqpoint{1.871227in}{3.336256in}}%
\pgfpathlineto{\pgfqpoint{1.876159in}{3.379693in}}%
\pgfpathlineto{\pgfqpoint{1.881091in}{3.354119in}}%
\pgfpathlineto{\pgfqpoint{1.895886in}{3.354119in}}%
\pgfpathlineto{\pgfqpoint{1.900818in}{3.379689in}}%
\pgfpathlineto{\pgfqpoint{1.905750in}{3.363528in}}%
\pgfpathlineto{\pgfqpoint{1.910682in}{3.371491in}}%
\pgfpathlineto{\pgfqpoint{1.915614in}{3.372866in}}%
\pgfpathlineto{\pgfqpoint{1.920545in}{3.293483in}}%
\pgfpathlineto{\pgfqpoint{1.925477in}{3.357954in}}%
\pgfpathlineto{\pgfqpoint{1.930409in}{3.344236in}}%
\pgfpathlineto{\pgfqpoint{1.935341in}{3.322949in}}%
\pgfpathlineto{\pgfqpoint{1.940273in}{3.258783in}}%
\pgfpathlineto{\pgfqpoint{1.945205in}{3.371884in}}%
\pgfpathlineto{\pgfqpoint{1.950136in}{3.341629in}}%
\pgfpathlineto{\pgfqpoint{1.955068in}{3.341629in}}%
\pgfpathlineto{\pgfqpoint{1.960000in}{3.347198in}}%
\pgfpathlineto{\pgfqpoint{1.964932in}{3.378087in}}%
\pgfpathlineto{\pgfqpoint{1.969864in}{3.378087in}}%
\pgfpathlineto{\pgfqpoint{1.974795in}{3.286051in}}%
\pgfpathlineto{\pgfqpoint{1.979727in}{3.286051in}}%
\pgfpathlineto{\pgfqpoint{1.984659in}{3.377008in}}%
\pgfpathlineto{\pgfqpoint{1.989591in}{3.375904in}}%
\pgfpathlineto{\pgfqpoint{1.994523in}{3.371291in}}%
\pgfpathlineto{\pgfqpoint{1.999455in}{3.320193in}}%
\pgfpathlineto{\pgfqpoint{2.004386in}{3.343509in}}%
\pgfpathlineto{\pgfqpoint{2.009318in}{3.325655in}}%
\pgfpathlineto{\pgfqpoint{2.014250in}{3.352877in}}%
\pgfpathlineto{\pgfqpoint{2.019182in}{3.339469in}}%
\pgfpathlineto{\pgfqpoint{2.024114in}{3.339469in}}%
\pgfpathlineto{\pgfqpoint{2.029045in}{3.379224in}}%
\pgfpathlineto{\pgfqpoint{2.033977in}{3.379224in}}%
\pgfpathlineto{\pgfqpoint{2.038909in}{3.316405in}}%
\pgfpathlineto{\pgfqpoint{2.043841in}{3.316405in}}%
\pgfpathlineto{\pgfqpoint{2.048773in}{3.377170in}}%
\pgfpathlineto{\pgfqpoint{2.053705in}{3.379947in}}%
\pgfpathlineto{\pgfqpoint{2.058636in}{3.379947in}}%
\pgfpathlineto{\pgfqpoint{2.063568in}{3.364298in}}%
\pgfpathlineto{\pgfqpoint{2.068500in}{3.364298in}}%
\pgfpathlineto{\pgfqpoint{2.073432in}{3.377512in}}%
\pgfpathlineto{\pgfqpoint{2.078364in}{3.365656in}}%
\pgfpathlineto{\pgfqpoint{2.083295in}{3.300317in}}%
\pgfpathlineto{\pgfqpoint{2.088227in}{3.375052in}}%
\pgfpathlineto{\pgfqpoint{2.093159in}{3.179041in}}%
\pgfpathlineto{\pgfqpoint{2.098091in}{3.367540in}}%
\pgfpathlineto{\pgfqpoint{2.103023in}{3.363453in}}%
\pgfpathlineto{\pgfqpoint{2.107955in}{3.363453in}}%
\pgfpathlineto{\pgfqpoint{2.112886in}{3.367168in}}%
\pgfpathlineto{\pgfqpoint{2.117818in}{3.234298in}}%
\pgfpathlineto{\pgfqpoint{2.122750in}{3.316766in}}%
\pgfpathlineto{\pgfqpoint{2.127682in}{3.316766in}}%
\pgfpathlineto{\pgfqpoint{2.132614in}{3.324334in}}%
\pgfpathlineto{\pgfqpoint{2.137545in}{3.272852in}}%
\pgfpathlineto{\pgfqpoint{2.152341in}{3.272852in}}%
\pgfpathlineto{\pgfqpoint{2.157273in}{3.377980in}}%
\pgfpathlineto{\pgfqpoint{2.162205in}{3.377980in}}%
\pgfpathlineto{\pgfqpoint{2.167136in}{3.366046in}}%
\pgfpathlineto{\pgfqpoint{2.172068in}{3.379879in}}%
\pgfpathlineto{\pgfqpoint{2.177000in}{3.259328in}}%
\pgfpathlineto{\pgfqpoint{2.186864in}{3.259328in}}%
\pgfpathlineto{\pgfqpoint{2.191795in}{3.379980in}}%
\pgfpathlineto{\pgfqpoint{2.196727in}{3.336561in}}%
\pgfpathlineto{\pgfqpoint{2.201659in}{3.376776in}}%
\pgfpathlineto{\pgfqpoint{2.206591in}{3.366386in}}%
\pgfpathlineto{\pgfqpoint{2.211523in}{3.366386in}}%
\pgfpathlineto{\pgfqpoint{2.216455in}{3.370277in}}%
\pgfpathlineto{\pgfqpoint{2.221386in}{3.323574in}}%
\pgfpathlineto{\pgfqpoint{2.226318in}{3.208496in}}%
\pgfpathlineto{\pgfqpoint{2.236182in}{3.208496in}}%
\pgfpathlineto{\pgfqpoint{2.241114in}{3.350292in}}%
\pgfpathlineto{\pgfqpoint{2.246045in}{3.298447in}}%
\pgfpathlineto{\pgfqpoint{2.250977in}{3.375420in}}%
\pgfpathlineto{\pgfqpoint{2.255909in}{3.375420in}}%
\pgfpathlineto{\pgfqpoint{2.260841in}{3.363647in}}%
\pgfpathlineto{\pgfqpoint{2.265773in}{3.363647in}}%
\pgfpathlineto{\pgfqpoint{2.270705in}{3.379676in}}%
\pgfpathlineto{\pgfqpoint{2.280568in}{3.379255in}}%
\pgfpathlineto{\pgfqpoint{2.285500in}{3.302095in}}%
\pgfpathlineto{\pgfqpoint{2.290432in}{3.251360in}}%
\pgfpathlineto{\pgfqpoint{2.295364in}{3.251360in}}%
\pgfpathlineto{\pgfqpoint{2.300295in}{3.371751in}}%
\pgfpathlineto{\pgfqpoint{2.305227in}{3.323881in}}%
\pgfpathlineto{\pgfqpoint{2.310159in}{3.353759in}}%
\pgfpathlineto{\pgfqpoint{2.315091in}{3.293772in}}%
\pgfpathlineto{\pgfqpoint{2.320023in}{3.293772in}}%
\pgfpathlineto{\pgfqpoint{2.324955in}{3.295168in}}%
\pgfpathlineto{\pgfqpoint{2.329886in}{3.377476in}}%
\pgfpathlineto{\pgfqpoint{2.334818in}{3.375573in}}%
\pgfpathlineto{\pgfqpoint{2.339750in}{3.371296in}}%
\pgfpathlineto{\pgfqpoint{2.344682in}{3.378063in}}%
\pgfpathlineto{\pgfqpoint{2.349614in}{3.353615in}}%
\pgfpathlineto{\pgfqpoint{2.354545in}{2.868545in}}%
\pgfpathlineto{\pgfqpoint{2.359477in}{3.330182in}}%
\pgfpathlineto{\pgfqpoint{2.374273in}{3.330182in}}%
\pgfpathlineto{\pgfqpoint{2.379205in}{3.376802in}}%
\pgfpathlineto{\pgfqpoint{2.384136in}{3.328721in}}%
\pgfpathlineto{\pgfqpoint{2.389068in}{3.378644in}}%
\pgfpathlineto{\pgfqpoint{2.394000in}{3.306646in}}%
\pgfpathlineto{\pgfqpoint{2.403864in}{3.307178in}}%
\pgfpathlineto{\pgfqpoint{2.408795in}{3.301659in}}%
\pgfpathlineto{\pgfqpoint{2.413727in}{3.377757in}}%
\pgfpathlineto{\pgfqpoint{2.418659in}{3.285161in}}%
\pgfpathlineto{\pgfqpoint{2.423591in}{3.285161in}}%
\pgfpathlineto{\pgfqpoint{2.428523in}{3.378441in}}%
\pgfpathlineto{\pgfqpoint{2.433455in}{3.331633in}}%
\pgfpathlineto{\pgfqpoint{2.438386in}{3.331633in}}%
\pgfpathlineto{\pgfqpoint{2.443318in}{3.379927in}}%
\pgfpathlineto{\pgfqpoint{2.453182in}{3.379927in}}%
\pgfpathlineto{\pgfqpoint{2.458114in}{3.367644in}}%
\pgfpathlineto{\pgfqpoint{2.463045in}{3.379594in}}%
\pgfpathlineto{\pgfqpoint{2.472909in}{3.256244in}}%
\pgfpathlineto{\pgfqpoint{2.477841in}{3.374646in}}%
\pgfpathlineto{\pgfqpoint{2.482773in}{3.355711in}}%
\pgfpathlineto{\pgfqpoint{2.487705in}{3.187548in}}%
\pgfpathlineto{\pgfqpoint{2.492636in}{3.187548in}}%
\pgfpathlineto{\pgfqpoint{2.497568in}{3.317660in}}%
\pgfpathlineto{\pgfqpoint{2.502500in}{3.185976in}}%
\pgfpathlineto{\pgfqpoint{2.507432in}{3.377974in}}%
\pgfpathlineto{\pgfqpoint{2.512364in}{3.375758in}}%
\pgfpathlineto{\pgfqpoint{2.517295in}{3.379660in}}%
\pgfpathlineto{\pgfqpoint{2.522227in}{3.379660in}}%
\pgfpathlineto{\pgfqpoint{2.527159in}{3.373730in}}%
\pgfpathlineto{\pgfqpoint{2.532091in}{3.373730in}}%
\pgfpathlineto{\pgfqpoint{2.537023in}{3.361834in}}%
\pgfpathlineto{\pgfqpoint{2.541955in}{3.369511in}}%
\pgfpathlineto{\pgfqpoint{2.546886in}{3.369511in}}%
\pgfpathlineto{\pgfqpoint{2.551818in}{3.362743in}}%
\pgfpathlineto{\pgfqpoint{2.556750in}{3.362743in}}%
\pgfpathlineto{\pgfqpoint{2.561682in}{3.359658in}}%
\pgfpathlineto{\pgfqpoint{2.566614in}{3.345176in}}%
\pgfpathlineto{\pgfqpoint{2.571545in}{3.345176in}}%
\pgfpathlineto{\pgfqpoint{2.576477in}{3.290169in}}%
\pgfpathlineto{\pgfqpoint{2.581409in}{3.290169in}}%
\pgfpathlineto{\pgfqpoint{2.586341in}{3.379943in}}%
\pgfpathlineto{\pgfqpoint{2.591273in}{3.315308in}}%
\pgfpathlineto{\pgfqpoint{2.596205in}{3.374768in}}%
\pgfpathlineto{\pgfqpoint{2.601136in}{3.374768in}}%
\pgfpathlineto{\pgfqpoint{2.606068in}{3.378480in}}%
\pgfpathlineto{\pgfqpoint{2.611000in}{3.327510in}}%
\pgfpathlineto{\pgfqpoint{2.615932in}{3.323583in}}%
\pgfpathlineto{\pgfqpoint{2.620864in}{3.337322in}}%
\pgfpathlineto{\pgfqpoint{2.625795in}{3.337322in}}%
\pgfpathlineto{\pgfqpoint{2.630727in}{3.374265in}}%
\pgfpathlineto{\pgfqpoint{2.635659in}{3.329953in}}%
\pgfpathlineto{\pgfqpoint{2.645523in}{3.329953in}}%
\pgfpathlineto{\pgfqpoint{2.650455in}{3.321206in}}%
\pgfpathlineto{\pgfqpoint{2.655386in}{3.375689in}}%
\pgfpathlineto{\pgfqpoint{2.660318in}{3.330484in}}%
\pgfpathlineto{\pgfqpoint{2.665250in}{3.330484in}}%
\pgfpathlineto{\pgfqpoint{2.675114in}{3.371331in}}%
\pgfpathlineto{\pgfqpoint{2.680045in}{3.371331in}}%
\pgfpathlineto{\pgfqpoint{2.684977in}{3.365107in}}%
\pgfpathlineto{\pgfqpoint{2.689909in}{3.345192in}}%
\pgfpathlineto{\pgfqpoint{2.694841in}{3.269780in}}%
\pgfpathlineto{\pgfqpoint{2.699773in}{3.360673in}}%
\pgfpathlineto{\pgfqpoint{2.714568in}{3.359894in}}%
\pgfpathlineto{\pgfqpoint{2.719500in}{3.359894in}}%
\pgfpathlineto{\pgfqpoint{2.729364in}{3.370890in}}%
\pgfpathlineto{\pgfqpoint{2.734295in}{3.371002in}}%
\pgfpathlineto{\pgfqpoint{2.739227in}{3.373580in}}%
\pgfpathlineto{\pgfqpoint{2.744159in}{3.356626in}}%
\pgfpathlineto{\pgfqpoint{2.749091in}{3.355201in}}%
\pgfpathlineto{\pgfqpoint{2.754023in}{3.379997in}}%
\pgfpathlineto{\pgfqpoint{2.763886in}{3.379990in}}%
\pgfpathlineto{\pgfqpoint{2.768818in}{3.369465in}}%
\pgfpathlineto{\pgfqpoint{2.773750in}{3.369465in}}%
\pgfpathlineto{\pgfqpoint{2.778682in}{3.376428in}}%
\pgfpathlineto{\pgfqpoint{2.783614in}{3.339527in}}%
\pgfpathlineto{\pgfqpoint{2.788545in}{3.349176in}}%
\pgfpathlineto{\pgfqpoint{2.793477in}{3.353765in}}%
\pgfpathlineto{\pgfqpoint{2.798409in}{3.375454in}}%
\pgfpathlineto{\pgfqpoint{2.803341in}{3.264795in}}%
\pgfpathlineto{\pgfqpoint{2.808273in}{3.020037in}}%
\pgfpathlineto{\pgfqpoint{2.818136in}{3.020037in}}%
\pgfpathlineto{\pgfqpoint{2.823068in}{3.333856in}}%
\pgfpathlineto{\pgfqpoint{2.828000in}{3.378891in}}%
\pgfpathlineto{\pgfqpoint{2.832932in}{3.353188in}}%
\pgfpathlineto{\pgfqpoint{2.837864in}{3.377129in}}%
\pgfpathlineto{\pgfqpoint{2.842795in}{3.210760in}}%
\pgfpathlineto{\pgfqpoint{2.847727in}{3.374377in}}%
\pgfpathlineto{\pgfqpoint{2.852659in}{3.328723in}}%
\pgfpathlineto{\pgfqpoint{2.857591in}{3.379319in}}%
\pgfpathlineto{\pgfqpoint{2.862523in}{3.374442in}}%
\pgfpathlineto{\pgfqpoint{2.867455in}{3.374442in}}%
\pgfpathlineto{\pgfqpoint{2.872386in}{3.364370in}}%
\pgfpathlineto{\pgfqpoint{2.877318in}{3.364370in}}%
\pgfpathlineto{\pgfqpoint{2.882250in}{3.379999in}}%
\pgfpathlineto{\pgfqpoint{2.892114in}{3.379245in}}%
\pgfpathlineto{\pgfqpoint{2.897045in}{3.247170in}}%
\pgfpathlineto{\pgfqpoint{2.906909in}{3.374668in}}%
\pgfpathlineto{\pgfqpoint{2.911841in}{3.278109in}}%
\pgfpathlineto{\pgfqpoint{2.916773in}{3.366712in}}%
\pgfpathlineto{\pgfqpoint{2.921705in}{3.366712in}}%
\pgfpathlineto{\pgfqpoint{2.926636in}{3.368686in}}%
\pgfpathlineto{\pgfqpoint{2.931568in}{3.361435in}}%
\pgfpathlineto{\pgfqpoint{2.936500in}{3.085481in}}%
\pgfpathlineto{\pgfqpoint{2.941432in}{3.085481in}}%
\pgfpathlineto{\pgfqpoint{2.946364in}{3.357719in}}%
\pgfpathlineto{\pgfqpoint{2.951295in}{3.369102in}}%
\pgfpathlineto{\pgfqpoint{2.956227in}{3.357543in}}%
\pgfpathlineto{\pgfqpoint{2.961159in}{3.334598in}}%
\pgfpathlineto{\pgfqpoint{2.966091in}{3.379945in}}%
\pgfpathlineto{\pgfqpoint{2.975955in}{3.379945in}}%
\pgfpathlineto{\pgfqpoint{2.980886in}{3.378695in}}%
\pgfpathlineto{\pgfqpoint{2.985818in}{3.379996in}}%
\pgfpathlineto{\pgfqpoint{2.990750in}{3.379996in}}%
\pgfpathlineto{\pgfqpoint{2.995682in}{3.372436in}}%
\pgfpathlineto{\pgfqpoint{3.000614in}{3.378457in}}%
\pgfpathlineto{\pgfqpoint{3.005545in}{3.354665in}}%
\pgfpathlineto{\pgfqpoint{3.010477in}{3.211118in}}%
\pgfpathlineto{\pgfqpoint{3.015409in}{3.214725in}}%
\pgfpathlineto{\pgfqpoint{3.020341in}{3.354951in}}%
\pgfpathlineto{\pgfqpoint{3.025273in}{3.354951in}}%
\pgfpathlineto{\pgfqpoint{3.030205in}{3.299588in}}%
\pgfpathlineto{\pgfqpoint{3.035136in}{3.354781in}}%
\pgfpathlineto{\pgfqpoint{3.040068in}{3.376074in}}%
\pgfpathlineto{\pgfqpoint{3.045000in}{3.355050in}}%
\pgfpathlineto{\pgfqpoint{3.049932in}{3.371234in}}%
\pgfpathlineto{\pgfqpoint{3.054864in}{3.371234in}}%
\pgfpathlineto{\pgfqpoint{3.059795in}{3.249929in}}%
\pgfpathlineto{\pgfqpoint{3.064727in}{3.379986in}}%
\pgfpathlineto{\pgfqpoint{3.069659in}{3.372289in}}%
\pgfpathlineto{\pgfqpoint{3.074591in}{3.343710in}}%
\pgfpathlineto{\pgfqpoint{3.079523in}{3.343710in}}%
\pgfpathlineto{\pgfqpoint{3.084455in}{3.352909in}}%
\pgfpathlineto{\pgfqpoint{3.089386in}{3.295064in}}%
\pgfpathlineto{\pgfqpoint{3.094318in}{3.377052in}}%
\pgfpathlineto{\pgfqpoint{3.099250in}{3.316013in}}%
\pgfpathlineto{\pgfqpoint{3.104182in}{3.092637in}}%
\pgfpathlineto{\pgfqpoint{3.109114in}{3.307953in}}%
\pgfpathlineto{\pgfqpoint{3.114045in}{3.307953in}}%
\pgfpathlineto{\pgfqpoint{3.118977in}{3.379818in}}%
\pgfpathlineto{\pgfqpoint{3.128841in}{3.379818in}}%
\pgfpathlineto{\pgfqpoint{3.133773in}{3.357218in}}%
\pgfpathlineto{\pgfqpoint{3.148568in}{3.357218in}}%
\pgfpathlineto{\pgfqpoint{3.153500in}{3.371972in}}%
\pgfpathlineto{\pgfqpoint{3.158432in}{3.292995in}}%
\pgfpathlineto{\pgfqpoint{3.163364in}{3.379930in}}%
\pgfpathlineto{\pgfqpoint{3.168295in}{3.374603in}}%
\pgfpathlineto{\pgfqpoint{3.178159in}{3.375709in}}%
\pgfpathlineto{\pgfqpoint{3.183091in}{3.379571in}}%
\pgfpathlineto{\pgfqpoint{3.192955in}{3.306423in}}%
\pgfpathlineto{\pgfqpoint{3.197886in}{3.357113in}}%
\pgfpathlineto{\pgfqpoint{3.202818in}{3.282314in}}%
\pgfpathlineto{\pgfqpoint{3.212682in}{3.282314in}}%
\pgfpathlineto{\pgfqpoint{3.222545in}{3.343681in}}%
\pgfpathlineto{\pgfqpoint{3.227477in}{3.368715in}}%
\pgfpathlineto{\pgfqpoint{3.232409in}{3.368715in}}%
\pgfpathlineto{\pgfqpoint{3.237341in}{3.339442in}}%
\pgfpathlineto{\pgfqpoint{3.242273in}{3.347066in}}%
\pgfpathlineto{\pgfqpoint{3.247205in}{3.377826in}}%
\pgfpathlineto{\pgfqpoint{3.257068in}{3.377538in}}%
\pgfpathlineto{\pgfqpoint{3.262000in}{3.371317in}}%
\pgfpathlineto{\pgfqpoint{3.266932in}{3.371317in}}%
\pgfpathlineto{\pgfqpoint{3.271864in}{3.364894in}}%
\pgfpathlineto{\pgfqpoint{3.276795in}{3.378879in}}%
\pgfpathlineto{\pgfqpoint{3.286659in}{3.379680in}}%
\pgfpathlineto{\pgfqpoint{3.296523in}{3.379939in}}%
\pgfpathlineto{\pgfqpoint{3.301455in}{3.377622in}}%
\pgfpathlineto{\pgfqpoint{3.306386in}{3.161062in}}%
\pgfpathlineto{\pgfqpoint{3.311318in}{3.294660in}}%
\pgfpathlineto{\pgfqpoint{3.321182in}{3.294660in}}%
\pgfpathlineto{\pgfqpoint{3.326114in}{3.191596in}}%
\pgfpathlineto{\pgfqpoint{3.331045in}{3.354858in}}%
\pgfpathlineto{\pgfqpoint{3.335977in}{3.368313in}}%
\pgfpathlineto{\pgfqpoint{3.340909in}{3.368313in}}%
\pgfpathlineto{\pgfqpoint{3.345841in}{3.320197in}}%
\pgfpathlineto{\pgfqpoint{3.350773in}{3.320197in}}%
\pgfpathlineto{\pgfqpoint{3.355705in}{3.260446in}}%
\pgfpathlineto{\pgfqpoint{3.360636in}{3.260446in}}%
\pgfpathlineto{\pgfqpoint{3.365568in}{3.192420in}}%
\pgfpathlineto{\pgfqpoint{3.370500in}{3.351205in}}%
\pgfpathlineto{\pgfqpoint{3.375432in}{3.055372in}}%
\pgfpathlineto{\pgfqpoint{3.380364in}{3.379789in}}%
\pgfpathlineto{\pgfqpoint{3.385295in}{3.191647in}}%
\pgfpathlineto{\pgfqpoint{3.395159in}{3.191647in}}%
\pgfpathlineto{\pgfqpoint{3.400091in}{3.312566in}}%
\pgfpathlineto{\pgfqpoint{3.405023in}{3.312566in}}%
\pgfpathlineto{\pgfqpoint{3.409955in}{3.378789in}}%
\pgfpathlineto{\pgfqpoint{3.414886in}{3.347622in}}%
\pgfpathlineto{\pgfqpoint{3.419818in}{3.348847in}}%
\pgfpathlineto{\pgfqpoint{3.424750in}{3.374423in}}%
\pgfpathlineto{\pgfqpoint{3.429682in}{3.374423in}}%
\pgfpathlineto{\pgfqpoint{3.434614in}{3.345309in}}%
\pgfpathlineto{\pgfqpoint{3.439545in}{3.379054in}}%
\pgfpathlineto{\pgfqpoint{3.454341in}{3.379054in}}%
\pgfpathlineto{\pgfqpoint{3.459273in}{3.368926in}}%
\pgfpathlineto{\pgfqpoint{3.464205in}{3.274537in}}%
\pgfpathlineto{\pgfqpoint{3.469136in}{3.274537in}}%
\pgfpathlineto{\pgfqpoint{3.474068in}{3.375606in}}%
\pgfpathlineto{\pgfqpoint{3.479000in}{3.379494in}}%
\pgfpathlineto{\pgfqpoint{3.483932in}{3.379494in}}%
\pgfpathlineto{\pgfqpoint{3.488864in}{3.373092in}}%
\pgfpathlineto{\pgfqpoint{3.498727in}{3.373092in}}%
\pgfpathlineto{\pgfqpoint{3.503659in}{3.305807in}}%
\pgfpathlineto{\pgfqpoint{3.508591in}{3.309593in}}%
\pgfpathlineto{\pgfqpoint{3.513523in}{3.281501in}}%
\pgfpathlineto{\pgfqpoint{3.518455in}{3.378069in}}%
\pgfpathlineto{\pgfqpoint{3.523386in}{3.345380in}}%
\pgfpathlineto{\pgfqpoint{3.528318in}{3.377902in}}%
\pgfpathlineto{\pgfqpoint{3.533250in}{3.377902in}}%
\pgfpathlineto{\pgfqpoint{3.538182in}{3.336709in}}%
\pgfpathlineto{\pgfqpoint{3.543114in}{3.366803in}}%
\pgfpathlineto{\pgfqpoint{3.548045in}{3.231995in}}%
\pgfpathlineto{\pgfqpoint{3.552977in}{3.231995in}}%
\pgfpathlineto{\pgfqpoint{3.557909in}{3.353914in}}%
\pgfpathlineto{\pgfqpoint{3.562841in}{3.271168in}}%
\pgfpathlineto{\pgfqpoint{3.567773in}{3.271168in}}%
\pgfpathlineto{\pgfqpoint{3.572705in}{3.206116in}}%
\pgfpathlineto{\pgfqpoint{3.577636in}{3.379930in}}%
\pgfpathlineto{\pgfqpoint{3.582568in}{3.372075in}}%
\pgfpathlineto{\pgfqpoint{3.592432in}{3.372075in}}%
\pgfpathlineto{\pgfqpoint{3.597364in}{3.378202in}}%
\pgfpathlineto{\pgfqpoint{3.607227in}{3.378824in}}%
\pgfpathlineto{\pgfqpoint{3.612159in}{3.173928in}}%
\pgfpathlineto{\pgfqpoint{3.617091in}{3.345838in}}%
\pgfpathlineto{\pgfqpoint{3.622023in}{3.354128in}}%
\pgfpathlineto{\pgfqpoint{3.626955in}{3.373098in}}%
\pgfpathlineto{\pgfqpoint{3.631886in}{3.379977in}}%
\pgfpathlineto{\pgfqpoint{3.636818in}{3.361473in}}%
\pgfpathlineto{\pgfqpoint{3.641750in}{3.281727in}}%
\pgfpathlineto{\pgfqpoint{3.646682in}{3.275984in}}%
\pgfpathlineto{\pgfqpoint{3.651614in}{3.275984in}}%
\pgfpathlineto{\pgfqpoint{3.656545in}{3.349538in}}%
\pgfpathlineto{\pgfqpoint{3.661477in}{3.359792in}}%
\pgfpathlineto{\pgfqpoint{3.666409in}{3.359792in}}%
\pgfpathlineto{\pgfqpoint{3.671341in}{3.311617in}}%
\pgfpathlineto{\pgfqpoint{3.676273in}{3.376906in}}%
\pgfpathlineto{\pgfqpoint{3.681205in}{3.376906in}}%
\pgfpathlineto{\pgfqpoint{3.686136in}{3.323280in}}%
\pgfpathlineto{\pgfqpoint{3.696000in}{3.375262in}}%
\pgfpathlineto{\pgfqpoint{3.700932in}{3.375262in}}%
\pgfpathlineto{\pgfqpoint{3.705864in}{3.353225in}}%
\pgfpathlineto{\pgfqpoint{3.710795in}{3.353225in}}%
\pgfpathlineto{\pgfqpoint{3.715727in}{3.376858in}}%
\pgfpathlineto{\pgfqpoint{3.720659in}{3.329088in}}%
\pgfpathlineto{\pgfqpoint{3.725591in}{3.379970in}}%
\pgfpathlineto{\pgfqpoint{3.730523in}{3.371417in}}%
\pgfpathlineto{\pgfqpoint{3.735455in}{3.310848in}}%
\pgfpathlineto{\pgfqpoint{3.740386in}{3.310848in}}%
\pgfpathlineto{\pgfqpoint{3.745318in}{3.374876in}}%
\pgfpathlineto{\pgfqpoint{3.750250in}{3.282670in}}%
\pgfpathlineto{\pgfqpoint{3.755182in}{3.282670in}}%
\pgfpathlineto{\pgfqpoint{3.760114in}{3.156471in}}%
\pgfpathlineto{\pgfqpoint{3.765045in}{3.317448in}}%
\pgfpathlineto{\pgfqpoint{3.769977in}{3.133878in}}%
\pgfpathlineto{\pgfqpoint{3.779841in}{3.133878in}}%
\pgfpathlineto{\pgfqpoint{3.784773in}{3.377802in}}%
\pgfpathlineto{\pgfqpoint{3.789705in}{3.377791in}}%
\pgfpathlineto{\pgfqpoint{3.794636in}{3.367632in}}%
\pgfpathlineto{\pgfqpoint{3.799568in}{3.375737in}}%
\pgfpathlineto{\pgfqpoint{3.804500in}{3.375737in}}%
\pgfpathlineto{\pgfqpoint{3.809432in}{3.379560in}}%
\pgfpathlineto{\pgfqpoint{3.814364in}{3.379560in}}%
\pgfpathlineto{\pgfqpoint{3.819295in}{3.364159in}}%
\pgfpathlineto{\pgfqpoint{3.824227in}{3.373880in}}%
\pgfpathlineto{\pgfqpoint{3.829159in}{3.373880in}}%
\pgfpathlineto{\pgfqpoint{3.834091in}{3.360828in}}%
\pgfpathlineto{\pgfqpoint{3.839023in}{3.215629in}}%
\pgfpathlineto{\pgfqpoint{3.843955in}{3.313641in}}%
\pgfpathlineto{\pgfqpoint{3.848886in}{3.379578in}}%
\pgfpathlineto{\pgfqpoint{3.853818in}{3.370239in}}%
\pgfpathlineto{\pgfqpoint{3.858750in}{3.375760in}}%
\pgfpathlineto{\pgfqpoint{3.863682in}{3.377643in}}%
\pgfpathlineto{\pgfqpoint{3.868614in}{3.239887in}}%
\pgfpathlineto{\pgfqpoint{3.873545in}{3.341601in}}%
\pgfpathlineto{\pgfqpoint{3.878477in}{3.138596in}}%
\pgfpathlineto{\pgfqpoint{3.883409in}{3.138596in}}%
\pgfpathlineto{\pgfqpoint{3.888341in}{3.369670in}}%
\pgfpathlineto{\pgfqpoint{3.893273in}{3.363030in}}%
\pgfpathlineto{\pgfqpoint{3.898205in}{3.363030in}}%
\pgfpathlineto{\pgfqpoint{3.903136in}{3.311241in}}%
\pgfpathlineto{\pgfqpoint{3.908068in}{3.365316in}}%
\pgfpathlineto{\pgfqpoint{3.913000in}{3.376468in}}%
\pgfpathlineto{\pgfqpoint{3.917932in}{3.376468in}}%
\pgfpathlineto{\pgfqpoint{3.922864in}{3.379998in}}%
\pgfpathlineto{\pgfqpoint{3.927795in}{3.235832in}}%
\pgfpathlineto{\pgfqpoint{3.932727in}{3.340741in}}%
\pgfpathlineto{\pgfqpoint{3.937659in}{3.286047in}}%
\pgfpathlineto{\pgfqpoint{3.942591in}{3.358778in}}%
\pgfpathlineto{\pgfqpoint{3.967250in}{3.358778in}}%
\pgfpathlineto{\pgfqpoint{3.972182in}{3.272381in}}%
\pgfpathlineto{\pgfqpoint{3.982045in}{3.272381in}}%
\pgfpathlineto{\pgfqpoint{3.986977in}{3.370316in}}%
\pgfpathlineto{\pgfqpoint{3.991909in}{3.363061in}}%
\pgfpathlineto{\pgfqpoint{3.996841in}{3.363695in}}%
\pgfpathlineto{\pgfqpoint{4.001773in}{3.379976in}}%
\pgfpathlineto{\pgfqpoint{4.011636in}{3.379784in}}%
\pgfpathlineto{\pgfqpoint{4.016568in}{3.378597in}}%
\pgfpathlineto{\pgfqpoint{4.021500in}{3.338315in}}%
\pgfpathlineto{\pgfqpoint{4.026432in}{3.379278in}}%
\pgfpathlineto{\pgfqpoint{4.031364in}{3.379611in}}%
\pgfpathlineto{\pgfqpoint{4.036295in}{3.329918in}}%
\pgfpathlineto{\pgfqpoint{4.041227in}{3.372953in}}%
\pgfpathlineto{\pgfqpoint{4.046159in}{3.372953in}}%
\pgfpathlineto{\pgfqpoint{4.051091in}{3.365863in}}%
\pgfpathlineto{\pgfqpoint{4.056023in}{3.365863in}}%
\pgfpathlineto{\pgfqpoint{4.060955in}{3.379618in}}%
\pgfpathlineto{\pgfqpoint{4.070818in}{3.379687in}}%
\pgfpathlineto{\pgfqpoint{4.075750in}{3.362923in}}%
\pgfpathlineto{\pgfqpoint{4.080682in}{3.376018in}}%
\pgfpathlineto{\pgfqpoint{4.095477in}{3.376018in}}%
\pgfpathlineto{\pgfqpoint{4.100409in}{3.368277in}}%
\pgfpathlineto{\pgfqpoint{4.105341in}{3.368277in}}%
\pgfpathlineto{\pgfqpoint{4.110273in}{3.378148in}}%
\pgfpathlineto{\pgfqpoint{4.115205in}{3.331385in}}%
\pgfpathlineto{\pgfqpoint{4.120136in}{3.296048in}}%
\pgfpathlineto{\pgfqpoint{4.125068in}{3.372340in}}%
\pgfpathlineto{\pgfqpoint{4.130000in}{3.380000in}}%
\pgfpathlineto{\pgfqpoint{4.134932in}{3.335284in}}%
\pgfpathlineto{\pgfqpoint{4.139864in}{3.376417in}}%
\pgfpathlineto{\pgfqpoint{4.144795in}{3.293177in}}%
\pgfpathlineto{\pgfqpoint{4.149727in}{3.347165in}}%
\pgfpathlineto{\pgfqpoint{4.159591in}{3.243550in}}%
\pgfpathlineto{\pgfqpoint{4.164523in}{3.243550in}}%
\pgfpathlineto{\pgfqpoint{4.169455in}{3.331490in}}%
\pgfpathlineto{\pgfqpoint{4.174386in}{3.331490in}}%
\pgfpathlineto{\pgfqpoint{4.179318in}{3.319446in}}%
\pgfpathlineto{\pgfqpoint{4.184250in}{3.319446in}}%
\pgfpathlineto{\pgfqpoint{4.189182in}{3.345190in}}%
\pgfpathlineto{\pgfqpoint{4.194114in}{3.378799in}}%
\pgfpathlineto{\pgfqpoint{4.199045in}{3.351828in}}%
\pgfpathlineto{\pgfqpoint{4.203977in}{3.366238in}}%
\pgfpathlineto{\pgfqpoint{4.208909in}{3.374957in}}%
\pgfpathlineto{\pgfqpoint{4.213841in}{3.200970in}}%
\pgfpathlineto{\pgfqpoint{4.218773in}{3.379867in}}%
\pgfpathlineto{\pgfqpoint{4.223705in}{3.336567in}}%
\pgfpathlineto{\pgfqpoint{4.228636in}{3.359749in}}%
\pgfpathlineto{\pgfqpoint{4.233568in}{3.359749in}}%
\pgfpathlineto{\pgfqpoint{4.238500in}{3.373242in}}%
\pgfpathlineto{\pgfqpoint{4.243432in}{3.361495in}}%
\pgfpathlineto{\pgfqpoint{4.248364in}{3.361495in}}%
\pgfpathlineto{\pgfqpoint{4.253295in}{3.365065in}}%
\pgfpathlineto{\pgfqpoint{4.258227in}{3.360437in}}%
\pgfpathlineto{\pgfqpoint{4.263159in}{3.366881in}}%
\pgfpathlineto{\pgfqpoint{4.268091in}{3.342756in}}%
\pgfpathlineto{\pgfqpoint{4.273023in}{3.342756in}}%
\pgfpathlineto{\pgfqpoint{4.277955in}{3.286274in}}%
\pgfpathlineto{\pgfqpoint{4.282886in}{3.355722in}}%
\pgfpathlineto{\pgfqpoint{4.287818in}{3.326735in}}%
\pgfpathlineto{\pgfqpoint{4.292750in}{3.289097in}}%
\pgfpathlineto{\pgfqpoint{4.302614in}{3.289097in}}%
\pgfpathlineto{\pgfqpoint{4.307545in}{3.359212in}}%
\pgfpathlineto{\pgfqpoint{4.312477in}{3.359212in}}%
\pgfpathlineto{\pgfqpoint{4.317409in}{3.377058in}}%
\pgfpathlineto{\pgfqpoint{4.322341in}{3.377058in}}%
\pgfpathlineto{\pgfqpoint{4.327273in}{3.379077in}}%
\pgfpathlineto{\pgfqpoint{4.332205in}{3.365294in}}%
\pgfpathlineto{\pgfqpoint{4.337136in}{3.365294in}}%
\pgfpathlineto{\pgfqpoint{4.342068in}{3.358707in}}%
\pgfpathlineto{\pgfqpoint{4.347000in}{3.358707in}}%
\pgfpathlineto{\pgfqpoint{4.351932in}{3.354554in}}%
\pgfpathlineto{\pgfqpoint{4.356864in}{3.373709in}}%
\pgfpathlineto{\pgfqpoint{4.361795in}{3.373709in}}%
\pgfpathlineto{\pgfqpoint{4.366727in}{3.375270in}}%
\pgfpathlineto{\pgfqpoint{4.371659in}{3.375270in}}%
\pgfpathlineto{\pgfqpoint{4.376591in}{3.062794in}}%
\pgfpathlineto{\pgfqpoint{4.381523in}{3.379703in}}%
\pgfpathlineto{\pgfqpoint{4.386455in}{3.378244in}}%
\pgfpathlineto{\pgfqpoint{4.391386in}{3.279559in}}%
\pgfpathlineto{\pgfqpoint{4.396318in}{3.279559in}}%
\pgfpathlineto{\pgfqpoint{4.401250in}{3.318140in}}%
\pgfpathlineto{\pgfqpoint{4.406182in}{3.379749in}}%
\pgfpathlineto{\pgfqpoint{4.416045in}{3.378649in}}%
\pgfpathlineto{\pgfqpoint{4.420977in}{3.379396in}}%
\pgfpathlineto{\pgfqpoint{4.425909in}{3.266267in}}%
\pgfpathlineto{\pgfqpoint{4.430841in}{3.259710in}}%
\pgfpathlineto{\pgfqpoint{4.440705in}{3.259710in}}%
\pgfpathlineto{\pgfqpoint{4.445636in}{3.127991in}}%
\pgfpathlineto{\pgfqpoint{4.450568in}{3.364260in}}%
\pgfpathlineto{\pgfqpoint{4.455500in}{3.377840in}}%
\pgfpathlineto{\pgfqpoint{4.460432in}{3.302289in}}%
\pgfpathlineto{\pgfqpoint{4.465364in}{3.302289in}}%
\pgfpathlineto{\pgfqpoint{4.470295in}{3.363890in}}%
\pgfpathlineto{\pgfqpoint{4.475227in}{3.341540in}}%
\pgfpathlineto{\pgfqpoint{4.480159in}{3.352016in}}%
\pgfpathlineto{\pgfqpoint{4.485091in}{3.378993in}}%
\pgfpathlineto{\pgfqpoint{4.490023in}{3.378993in}}%
\pgfpathlineto{\pgfqpoint{4.494955in}{3.366184in}}%
\pgfpathlineto{\pgfqpoint{4.499886in}{3.375823in}}%
\pgfpathlineto{\pgfqpoint{4.504818in}{3.376332in}}%
\pgfpathlineto{\pgfqpoint{4.509750in}{3.131598in}}%
\pgfpathlineto{\pgfqpoint{4.514682in}{3.375842in}}%
\pgfpathlineto{\pgfqpoint{4.524545in}{3.375842in}}%
\pgfpathlineto{\pgfqpoint{4.529477in}{3.374272in}}%
\pgfpathlineto{\pgfqpoint{4.534409in}{3.374272in}}%
\pgfpathlineto{\pgfqpoint{4.539341in}{3.066427in}}%
\pgfpathlineto{\pgfqpoint{4.544273in}{3.379197in}}%
\pgfpathlineto{\pgfqpoint{4.554136in}{3.379197in}}%
\pgfpathlineto{\pgfqpoint{4.559068in}{3.367543in}}%
\pgfpathlineto{\pgfqpoint{4.564000in}{3.289177in}}%
\pgfpathlineto{\pgfqpoint{4.568932in}{3.364998in}}%
\pgfpathlineto{\pgfqpoint{4.573864in}{3.291218in}}%
\pgfpathlineto{\pgfqpoint{4.578795in}{3.374187in}}%
\pgfpathlineto{\pgfqpoint{4.583727in}{3.374187in}}%
\pgfpathlineto{\pgfqpoint{4.588659in}{3.378850in}}%
\pgfpathlineto{\pgfqpoint{4.593591in}{3.311360in}}%
\pgfpathlineto{\pgfqpoint{4.598523in}{3.313634in}}%
\pgfpathlineto{\pgfqpoint{4.603455in}{3.374422in}}%
\pgfpathlineto{\pgfqpoint{4.608386in}{3.378839in}}%
\pgfpathlineto{\pgfqpoint{4.613318in}{3.361212in}}%
\pgfpathlineto{\pgfqpoint{4.618250in}{3.361212in}}%
\pgfpathlineto{\pgfqpoint{4.623182in}{3.292211in}}%
\pgfpathlineto{\pgfqpoint{4.628114in}{3.292211in}}%
\pgfpathlineto{\pgfqpoint{4.633045in}{3.191930in}}%
\pgfpathlineto{\pgfqpoint{4.637977in}{3.191930in}}%
\pgfpathlineto{\pgfqpoint{4.642909in}{3.379858in}}%
\pgfpathlineto{\pgfqpoint{4.652773in}{3.379858in}}%
\pgfpathlineto{\pgfqpoint{4.657705in}{3.371006in}}%
\pgfpathlineto{\pgfqpoint{4.662636in}{3.371006in}}%
\pgfpathlineto{\pgfqpoint{4.667568in}{3.377195in}}%
\pgfpathlineto{\pgfqpoint{4.672500in}{3.369507in}}%
\pgfpathlineto{\pgfqpoint{4.682364in}{3.369507in}}%
\pgfpathlineto{\pgfqpoint{4.687295in}{3.371713in}}%
\pgfpathlineto{\pgfqpoint{4.692227in}{3.379486in}}%
\pgfpathlineto{\pgfqpoint{4.702091in}{3.379486in}}%
\pgfpathlineto{\pgfqpoint{4.707023in}{3.362914in}}%
\pgfpathlineto{\pgfqpoint{4.711955in}{3.362914in}}%
\pgfpathlineto{\pgfqpoint{4.716886in}{3.379926in}}%
\pgfpathlineto{\pgfqpoint{4.721818in}{3.379926in}}%
\pgfpathlineto{\pgfqpoint{4.726750in}{3.293316in}}%
\pgfpathlineto{\pgfqpoint{4.731682in}{3.293316in}}%
\pgfpathlineto{\pgfqpoint{4.736614in}{3.202492in}}%
\pgfpathlineto{\pgfqpoint{4.741545in}{3.370577in}}%
\pgfpathlineto{\pgfqpoint{4.746477in}{3.372240in}}%
\pgfpathlineto{\pgfqpoint{4.751409in}{3.378833in}}%
\pgfpathlineto{\pgfqpoint{4.756341in}{3.235456in}}%
\pgfpathlineto{\pgfqpoint{4.766205in}{3.353871in}}%
\pgfpathlineto{\pgfqpoint{4.771136in}{3.353871in}}%
\pgfpathlineto{\pgfqpoint{4.776068in}{3.282905in}}%
\pgfpathlineto{\pgfqpoint{4.781000in}{3.186478in}}%
\pgfpathlineto{\pgfqpoint{4.830318in}{3.186478in}}%
\pgfpathlineto{\pgfqpoint{4.835250in}{3.379931in}}%
\pgfpathlineto{\pgfqpoint{4.840182in}{3.379931in}}%
\pgfpathlineto{\pgfqpoint{4.845114in}{3.376233in}}%
\pgfpathlineto{\pgfqpoint{4.850045in}{3.355187in}}%
\pgfpathlineto{\pgfqpoint{4.854977in}{3.202733in}}%
\pgfpathlineto{\pgfqpoint{4.859909in}{3.202733in}}%
\pgfpathlineto{\pgfqpoint{4.864841in}{3.369678in}}%
\pgfpathlineto{\pgfqpoint{4.869773in}{3.333804in}}%
\pgfpathlineto{\pgfqpoint{4.874705in}{3.183611in}}%
\pgfpathlineto{\pgfqpoint{4.879636in}{3.379934in}}%
\pgfpathlineto{\pgfqpoint{4.884568in}{3.318942in}}%
\pgfpathlineto{\pgfqpoint{4.889500in}{3.379857in}}%
\pgfpathlineto{\pgfqpoint{4.894432in}{3.272852in}}%
\pgfpathlineto{\pgfqpoint{4.899364in}{3.338146in}}%
\pgfpathlineto{\pgfqpoint{4.904295in}{3.365490in}}%
\pgfpathlineto{\pgfqpoint{4.914159in}{3.365490in}}%
\pgfpathlineto{\pgfqpoint{4.919091in}{3.347142in}}%
\pgfpathlineto{\pgfqpoint{4.924023in}{3.364031in}}%
\pgfpathlineto{\pgfqpoint{4.938818in}{3.364031in}}%
\pgfpathlineto{\pgfqpoint{4.943750in}{3.376056in}}%
\pgfpathlineto{\pgfqpoint{4.948682in}{3.353953in}}%
\pgfpathlineto{\pgfqpoint{4.968409in}{3.353953in}}%
\pgfpathlineto{\pgfqpoint{4.973341in}{3.379890in}}%
\pgfpathlineto{\pgfqpoint{4.983205in}{3.373705in}}%
\pgfpathlineto{\pgfqpoint{4.988136in}{3.373705in}}%
\pgfpathlineto{\pgfqpoint{4.993068in}{3.367211in}}%
\pgfpathlineto{\pgfqpoint{4.998000in}{3.376686in}}%
\pgfpathlineto{\pgfqpoint{5.002932in}{3.355038in}}%
\pgfpathlineto{\pgfqpoint{5.007864in}{3.370258in}}%
\pgfpathlineto{\pgfqpoint{5.012795in}{3.342547in}}%
\pgfpathlineto{\pgfqpoint{5.017727in}{3.355999in}}%
\pgfpathlineto{\pgfqpoint{5.027591in}{3.355999in}}%
\pgfpathlineto{\pgfqpoint{5.032523in}{3.352870in}}%
\pgfpathlineto{\pgfqpoint{5.037455in}{3.244112in}}%
\pgfpathlineto{\pgfqpoint{5.042386in}{3.309134in}}%
\pgfpathlineto{\pgfqpoint{5.057182in}{3.309134in}}%
\pgfpathlineto{\pgfqpoint{5.062114in}{3.354860in}}%
\pgfpathlineto{\pgfqpoint{5.067045in}{3.293372in}}%
\pgfpathlineto{\pgfqpoint{5.071977in}{3.374656in}}%
\pgfpathlineto{\pgfqpoint{5.076909in}{3.309192in}}%
\pgfpathlineto{\pgfqpoint{5.081841in}{3.309192in}}%
\pgfpathlineto{\pgfqpoint{5.086773in}{3.337845in}}%
\pgfpathlineto{\pgfqpoint{5.091705in}{3.342801in}}%
\pgfpathlineto{\pgfqpoint{5.096636in}{3.306640in}}%
\pgfpathlineto{\pgfqpoint{5.101568in}{3.368212in}}%
\pgfpathlineto{\pgfqpoint{5.106500in}{3.379452in}}%
\pgfpathlineto{\pgfqpoint{5.111432in}{3.379452in}}%
\pgfpathlineto{\pgfqpoint{5.116364in}{3.358125in}}%
\pgfpathlineto{\pgfqpoint{5.121295in}{3.379478in}}%
\pgfpathlineto{\pgfqpoint{5.131159in}{3.339366in}}%
\pgfpathlineto{\pgfqpoint{5.136091in}{3.306516in}}%
\pgfpathlineto{\pgfqpoint{5.141023in}{3.355848in}}%
\pgfpathlineto{\pgfqpoint{5.145955in}{3.334064in}}%
\pgfpathlineto{\pgfqpoint{5.150886in}{3.334064in}}%
\pgfpathlineto{\pgfqpoint{5.155818in}{3.378369in}}%
\pgfpathlineto{\pgfqpoint{5.165682in}{3.378369in}}%
\pgfpathlineto{\pgfqpoint{5.170614in}{3.357065in}}%
\pgfpathlineto{\pgfqpoint{5.180477in}{3.357065in}}%
\pgfpathlineto{\pgfqpoint{5.185409in}{3.379631in}}%
\pgfpathlineto{\pgfqpoint{5.190341in}{3.379631in}}%
\pgfpathlineto{\pgfqpoint{5.195273in}{3.378005in}}%
\pgfpathlineto{\pgfqpoint{5.200205in}{3.279381in}}%
\pgfpathlineto{\pgfqpoint{5.205136in}{3.279381in}}%
\pgfpathlineto{\pgfqpoint{5.210068in}{3.242206in}}%
\pgfpathlineto{\pgfqpoint{5.224864in}{3.242206in}}%
\pgfpathlineto{\pgfqpoint{5.229795in}{3.305863in}}%
\pgfpathlineto{\pgfqpoint{5.234727in}{3.236630in}}%
\pgfpathlineto{\pgfqpoint{5.239659in}{3.236630in}}%
\pgfpathlineto{\pgfqpoint{5.244591in}{3.355958in}}%
\pgfpathlineto{\pgfqpoint{5.249523in}{3.365057in}}%
\pgfpathlineto{\pgfqpoint{5.264318in}{3.365057in}}%
\pgfpathlineto{\pgfqpoint{5.269250in}{3.251673in}}%
\pgfpathlineto{\pgfqpoint{5.274182in}{3.361912in}}%
\pgfpathlineto{\pgfqpoint{5.279114in}{3.326402in}}%
\pgfpathlineto{\pgfqpoint{5.284045in}{3.379992in}}%
\pgfpathlineto{\pgfqpoint{5.293909in}{3.378958in}}%
\pgfpathlineto{\pgfqpoint{5.298841in}{3.365004in}}%
\pgfpathlineto{\pgfqpoint{5.303773in}{3.365004in}}%
\pgfpathlineto{\pgfqpoint{5.308705in}{3.377805in}}%
\pgfpathlineto{\pgfqpoint{5.313636in}{3.351522in}}%
\pgfpathlineto{\pgfqpoint{5.318568in}{3.368707in}}%
\pgfpathlineto{\pgfqpoint{5.323500in}{3.332776in}}%
\pgfpathlineto{\pgfqpoint{5.328432in}{3.370243in}}%
\pgfpathlineto{\pgfqpoint{5.333364in}{3.370243in}}%
\pgfpathlineto{\pgfqpoint{5.338295in}{3.375491in}}%
\pgfpathlineto{\pgfqpoint{5.343227in}{3.310982in}}%
\pgfpathlineto{\pgfqpoint{5.348159in}{3.369355in}}%
\pgfpathlineto{\pgfqpoint{5.353091in}{3.369355in}}%
\pgfpathlineto{\pgfqpoint{5.358023in}{3.365234in}}%
\pgfpathlineto{\pgfqpoint{5.362955in}{3.366146in}}%
\pgfpathlineto{\pgfqpoint{5.367886in}{3.375163in}}%
\pgfpathlineto{\pgfqpoint{5.372818in}{3.370317in}}%
\pgfpathlineto{\pgfqpoint{5.377750in}{3.362288in}}%
\pgfpathlineto{\pgfqpoint{5.382682in}{3.375881in}}%
\pgfpathlineto{\pgfqpoint{5.387614in}{3.375881in}}%
\pgfpathlineto{\pgfqpoint{5.392545in}{3.367250in}}%
\pgfpathlineto{\pgfqpoint{5.407341in}{3.367250in}}%
\pgfpathlineto{\pgfqpoint{5.412273in}{3.362194in}}%
\pgfpathlineto{\pgfqpoint{5.417205in}{3.362194in}}%
\pgfpathlineto{\pgfqpoint{5.422136in}{3.373737in}}%
\pgfpathlineto{\pgfqpoint{5.427068in}{3.312988in}}%
\pgfpathlineto{\pgfqpoint{5.432000in}{3.282655in}}%
\pgfpathlineto{\pgfqpoint{5.436932in}{3.321952in}}%
\pgfpathlineto{\pgfqpoint{5.441864in}{3.321952in}}%
\pgfpathlineto{\pgfqpoint{5.446795in}{3.379582in}}%
\pgfpathlineto{\pgfqpoint{5.451727in}{3.372583in}}%
\pgfpathlineto{\pgfqpoint{5.456659in}{3.332612in}}%
\pgfpathlineto{\pgfqpoint{5.461591in}{3.373876in}}%
\pgfpathlineto{\pgfqpoint{5.466523in}{3.373876in}}%
\pgfpathlineto{\pgfqpoint{5.471455in}{3.368613in}}%
\pgfpathlineto{\pgfqpoint{5.476386in}{3.367366in}}%
\pgfpathlineto{\pgfqpoint{5.481318in}{3.275964in}}%
\pgfpathlineto{\pgfqpoint{5.491182in}{3.376738in}}%
\pgfpathlineto{\pgfqpoint{5.496114in}{3.379344in}}%
\pgfpathlineto{\pgfqpoint{5.501045in}{3.351025in}}%
\pgfpathlineto{\pgfqpoint{5.505977in}{3.353241in}}%
\pgfpathlineto{\pgfqpoint{5.510909in}{3.353241in}}%
\pgfpathlineto{\pgfqpoint{5.515841in}{3.376267in}}%
\pgfpathlineto{\pgfqpoint{5.520773in}{3.379600in}}%
\pgfpathlineto{\pgfqpoint{5.525705in}{3.186433in}}%
\pgfpathlineto{\pgfqpoint{5.530636in}{3.296142in}}%
\pgfpathlineto{\pgfqpoint{5.540500in}{3.296142in}}%
\pgfpathlineto{\pgfqpoint{5.545432in}{3.312840in}}%
\pgfpathlineto{\pgfqpoint{5.550364in}{3.369337in}}%
\pgfpathlineto{\pgfqpoint{5.560227in}{3.369337in}}%
\pgfpathlineto{\pgfqpoint{5.565159in}{3.347857in}}%
\pgfpathlineto{\pgfqpoint{5.579955in}{3.347857in}}%
\pgfpathlineto{\pgfqpoint{5.584886in}{3.282220in}}%
\pgfpathlineto{\pgfqpoint{5.589818in}{3.355075in}}%
\pgfpathlineto{\pgfqpoint{5.594750in}{3.371722in}}%
\pgfpathlineto{\pgfqpoint{5.599682in}{3.315094in}}%
\pgfpathlineto{\pgfqpoint{5.604614in}{3.378414in}}%
\pgfpathlineto{\pgfqpoint{5.609545in}{3.369870in}}%
\pgfpathlineto{\pgfqpoint{5.614477in}{3.368266in}}%
\pgfpathlineto{\pgfqpoint{5.619409in}{3.235899in}}%
\pgfpathlineto{\pgfqpoint{5.624341in}{3.235899in}}%
\pgfpathlineto{\pgfqpoint{5.629273in}{3.322611in}}%
\pgfpathlineto{\pgfqpoint{5.634205in}{3.292728in}}%
\pgfpathlineto{\pgfqpoint{5.639136in}{3.307915in}}%
\pgfpathlineto{\pgfqpoint{5.649000in}{3.367335in}}%
\pgfpathlineto{\pgfqpoint{5.653932in}{3.292077in}}%
\pgfpathlineto{\pgfqpoint{5.658864in}{3.379863in}}%
\pgfpathlineto{\pgfqpoint{5.668727in}{3.379798in}}%
\pgfpathlineto{\pgfqpoint{5.673659in}{3.320859in}}%
\pgfpathlineto{\pgfqpoint{5.678591in}{3.379561in}}%
\pgfpathlineto{\pgfqpoint{5.688455in}{3.379082in}}%
\pgfpathlineto{\pgfqpoint{5.693386in}{3.375254in}}%
\pgfpathlineto{\pgfqpoint{5.698318in}{3.288380in}}%
\pgfpathlineto{\pgfqpoint{5.703250in}{3.375094in}}%
\pgfpathlineto{\pgfqpoint{5.708182in}{3.354441in}}%
\pgfpathlineto{\pgfqpoint{5.713114in}{3.377650in}}%
\pgfpathlineto{\pgfqpoint{5.718045in}{3.367832in}}%
\pgfpathlineto{\pgfqpoint{5.732841in}{3.367832in}}%
\pgfpathlineto{\pgfqpoint{5.737773in}{3.358825in}}%
\pgfpathlineto{\pgfqpoint{5.742705in}{3.372540in}}%
\pgfpathlineto{\pgfqpoint{5.747636in}{3.348673in}}%
\pgfpathlineto{\pgfqpoint{5.752568in}{3.373479in}}%
\pgfpathlineto{\pgfqpoint{5.757500in}{3.330118in}}%
\pgfpathlineto{\pgfqpoint{5.762432in}{3.369791in}}%
\pgfpathlineto{\pgfqpoint{5.767364in}{3.357459in}}%
\pgfpathlineto{\pgfqpoint{5.772295in}{3.377023in}}%
\pgfpathlineto{\pgfqpoint{5.782159in}{3.376267in}}%
\pgfpathlineto{\pgfqpoint{5.787091in}{3.368738in}}%
\pgfpathlineto{\pgfqpoint{5.792023in}{3.368738in}}%
\pgfpathlineto{\pgfqpoint{5.796955in}{3.373117in}}%
\pgfpathlineto{\pgfqpoint{5.801886in}{3.373117in}}%
\pgfpathlineto{\pgfqpoint{5.806818in}{3.366430in}}%
\pgfpathlineto{\pgfqpoint{5.811750in}{3.342047in}}%
\pgfpathlineto{\pgfqpoint{5.816682in}{3.342047in}}%
\pgfpathlineto{\pgfqpoint{5.821614in}{3.356364in}}%
\pgfpathlineto{\pgfqpoint{5.826545in}{3.250834in}}%
\pgfpathlineto{\pgfqpoint{5.831477in}{3.339756in}}%
\pgfpathlineto{\pgfqpoint{5.836409in}{3.364841in}}%
\pgfpathlineto{\pgfqpoint{5.841341in}{3.364841in}}%
\pgfpathlineto{\pgfqpoint{5.846273in}{3.360890in}}%
\pgfpathlineto{\pgfqpoint{5.851205in}{3.377756in}}%
\pgfpathlineto{\pgfqpoint{5.856136in}{3.374660in}}%
\pgfpathlineto{\pgfqpoint{5.861068in}{3.374660in}}%
\pgfpathlineto{\pgfqpoint{5.866000in}{3.349726in}}%
\pgfpathlineto{\pgfqpoint{5.870932in}{3.367845in}}%
\pgfpathlineto{\pgfqpoint{5.875864in}{3.367845in}}%
\pgfpathlineto{\pgfqpoint{5.880795in}{3.355136in}}%
\pgfpathlineto{\pgfqpoint{5.885727in}{3.370929in}}%
\pgfpathlineto{\pgfqpoint{5.890659in}{3.360432in}}%
\pgfpathlineto{\pgfqpoint{5.895591in}{3.280563in}}%
\pgfpathlineto{\pgfqpoint{5.900523in}{3.317742in}}%
\pgfpathlineto{\pgfqpoint{5.905455in}{3.317742in}}%
\pgfpathlineto{\pgfqpoint{5.910386in}{3.326632in}}%
\pgfpathlineto{\pgfqpoint{5.915318in}{3.332385in}}%
\pgfpathlineto{\pgfqpoint{5.920250in}{3.379677in}}%
\pgfpathlineto{\pgfqpoint{5.925182in}{3.379677in}}%
\pgfpathlineto{\pgfqpoint{5.935045in}{3.367628in}}%
\pgfpathlineto{\pgfqpoint{5.939977in}{3.266647in}}%
\pgfpathlineto{\pgfqpoint{5.944909in}{3.321807in}}%
\pgfpathlineto{\pgfqpoint{5.959705in}{3.321807in}}%
\pgfpathlineto{\pgfqpoint{5.964636in}{3.363630in}}%
\pgfpathlineto{\pgfqpoint{5.969568in}{3.363630in}}%
\pgfpathlineto{\pgfqpoint{5.974500in}{3.379834in}}%
\pgfpathlineto{\pgfqpoint{5.979432in}{3.343265in}}%
\pgfpathlineto{\pgfqpoint{5.984364in}{3.358083in}}%
\pgfpathlineto{\pgfqpoint{5.989295in}{3.276940in}}%
\pgfpathlineto{\pgfqpoint{5.994227in}{3.360640in}}%
\pgfpathlineto{\pgfqpoint{5.999159in}{3.360640in}}%
\pgfpathlineto{\pgfqpoint{6.004091in}{3.364623in}}%
\pgfpathlineto{\pgfqpoint{6.009023in}{3.364623in}}%
\pgfpathlineto{\pgfqpoint{6.013955in}{3.379976in}}%
\pgfpathlineto{\pgfqpoint{6.018886in}{3.350285in}}%
\pgfpathlineto{\pgfqpoint{6.023818in}{3.379993in}}%
\pgfpathlineto{\pgfqpoint{6.038614in}{3.379786in}}%
\pgfpathlineto{\pgfqpoint{6.043545in}{3.106385in}}%
\pgfpathlineto{\pgfqpoint{6.048477in}{3.370738in}}%
\pgfpathlineto{\pgfqpoint{6.053409in}{3.376068in}}%
\pgfpathlineto{\pgfqpoint{6.053409in}{3.376068in}}%
\pgfusepath{stroke}%
\end{pgfscope}%
\begin{pgfscope}%
\pgfsetrectcap%
\pgfsetmiterjoin%
\pgfsetlinewidth{0.000000pt}%
\definecolor{currentstroke}{rgb}{1.000000,1.000000,1.000000}%
\pgfsetstrokecolor{currentstroke}%
\pgfsetdash{}{0pt}%
\pgfpathmoveto{\pgfqpoint{0.875000in}{0.440000in}}%
\pgfpathlineto{\pgfqpoint{0.875000in}{3.520000in}}%
\pgfusepath{}%
\end{pgfscope}%
\begin{pgfscope}%
\pgfsetrectcap%
\pgfsetmiterjoin%
\pgfsetlinewidth{0.000000pt}%
\definecolor{currentstroke}{rgb}{1.000000,1.000000,1.000000}%
\pgfsetstrokecolor{currentstroke}%
\pgfsetdash{}{0pt}%
\pgfpathmoveto{\pgfqpoint{6.300000in}{0.440000in}}%
\pgfpathlineto{\pgfqpoint{6.300000in}{3.520000in}}%
\pgfusepath{}%
\end{pgfscope}%
\begin{pgfscope}%
\pgfsetrectcap%
\pgfsetmiterjoin%
\pgfsetlinewidth{0.000000pt}%
\definecolor{currentstroke}{rgb}{1.000000,1.000000,1.000000}%
\pgfsetstrokecolor{currentstroke}%
\pgfsetdash{}{0pt}%
\pgfpathmoveto{\pgfqpoint{0.875000in}{0.440000in}}%
\pgfpathlineto{\pgfqpoint{6.300000in}{0.440000in}}%
\pgfusepath{}%
\end{pgfscope}%
\begin{pgfscope}%
\pgfsetrectcap%
\pgfsetmiterjoin%
\pgfsetlinewidth{0.000000pt}%
\definecolor{currentstroke}{rgb}{1.000000,1.000000,1.000000}%
\pgfsetstrokecolor{currentstroke}%
\pgfsetdash{}{0pt}%
\pgfpathmoveto{\pgfqpoint{0.875000in}{3.520000in}}%
\pgfpathlineto{\pgfqpoint{6.300000in}{3.520000in}}%
\pgfusepath{}%
\end{pgfscope}%
\begin{pgfscope}%
\definecolor{textcolor}{rgb}{0.150000,0.150000,0.150000}%
\pgfsetstrokecolor{textcolor}%
\pgfsetfillcolor{textcolor}%
\pgftext[x=3.500000in,y=3.920000in,,top]{\color{textcolor}\rmfamily\fontsize{12.000000}{14.400000}\selectfont Gráfica de \(\displaystyle \log f(X_t)\) para propuesta beta, \(\displaystyle n=40, r=\)15}%
\end{pgfscope}%
\end{pgfpicture}%
\makeatother%
\endgroup%

    \end{center}

    Como podemos notar, en la segunda figura, el algoritmo comienza con un valor de log-densidad
    muy alejado del resto de los puntos, pero eventualmente se estabiliza. Podemos considerar
    que una vez que este se ha estabilizado estamos muestreando desde la distribución objetivo.
    El primer punto en que se estabiliza la densidad parece ser alrededor de $t=50$

    En la segunda figura (para $n=5, r=3$) el logaritmo de la densidad parece variar de manera
    uniforme a lo largo de toda la trayectoria del proceso. Podemos concluir que según esta heurística,
    es posible considerar desde el primer valor como parte del muestreo de la distribución
    objetivo.

    Con base en lo anterior, descartamos las primeras 50 iteraciones del muestreo con $n=40$ y $r=15$,
    miestras que conservamos todas las observaciones cuando $n=5, r=3$.
    
    

    \item  Implementar el algoritmo Metropolis-Hastings con la posterior de arriba
    tomando una propuesta diferente.

    Como buscamos distribuciones que tengan soporte en $(0,1/2)$, una propuesta puede ser la uniforme
    continua en $(0,1/2)$. Una variante del algoritmo de Metropolis-Hastings que utiliza una densidad
    uniforme como función de transición de la cadena original se encuentra implementada en el archivo
    \texttt{Tarea6.py} con el nombre de \texttt{MH\_unif()}. La función toma como entrada los mismos 
    argumentos que la función \texttt{MH\_beta()}.

    Al hacer 1,000 iteraciones de \texttt{MH\_unif()} para $r_5$ y 1,000 para $r_{40}$ encontramos
    los histogramas siguientes,

    \begin{center}
        %% Creator: Matplotlib, PGF backend
%%
%% To include the figure in your LaTeX document, write
%%   \input{<filename>.pgf}
%%
%% Make sure the required packages are loaded in your preamble
%%   \usepackage{pgf}
%%
%% Also ensure that all the required font packages are loaded; for instance,
%% the lmodern package is sometimes necessary when using math font.
%%   \usepackage{lmodern}
%%
%% Figures using additional raster images can only be included by \input if
%% they are in the same directory as the main LaTeX file. For loading figures
%% from other directories you can use the `import` package
%%   \usepackage{import}
%%
%% and then include the figures with
%%   \import{<path to file>}{<filename>.pgf}
%%
%% Matplotlib used the following preamble
%%   
%%   \makeatletter\@ifpackageloaded{underscore}{}{\usepackage[strings]{underscore}}\makeatother
%%
\begingroup%
\makeatletter%
\begin{pgfpicture}%
\pgfpathrectangle{\pgfpointorigin}{\pgfqpoint{7.000000in}{4.000000in}}%
\pgfusepath{use as bounding box, clip}%
\begin{pgfscope}%
\pgfsetbuttcap%
\pgfsetmiterjoin%
\definecolor{currentfill}{rgb}{1.000000,1.000000,1.000000}%
\pgfsetfillcolor{currentfill}%
\pgfsetlinewidth{0.000000pt}%
\definecolor{currentstroke}{rgb}{1.000000,1.000000,1.000000}%
\pgfsetstrokecolor{currentstroke}%
\pgfsetdash{}{0pt}%
\pgfpathmoveto{\pgfqpoint{0.000000in}{0.000000in}}%
\pgfpathlineto{\pgfqpoint{7.000000in}{0.000000in}}%
\pgfpathlineto{\pgfqpoint{7.000000in}{4.000000in}}%
\pgfpathlineto{\pgfqpoint{0.000000in}{4.000000in}}%
\pgfpathlineto{\pgfqpoint{0.000000in}{0.000000in}}%
\pgfpathclose%
\pgfusepath{fill}%
\end{pgfscope}%
\begin{pgfscope}%
\pgfsetbuttcap%
\pgfsetmiterjoin%
\definecolor{currentfill}{rgb}{0.917647,0.917647,0.949020}%
\pgfsetfillcolor{currentfill}%
\pgfsetlinewidth{0.000000pt}%
\definecolor{currentstroke}{rgb}{0.000000,0.000000,0.000000}%
\pgfsetstrokecolor{currentstroke}%
\pgfsetstrokeopacity{0.000000}%
\pgfsetdash{}{0pt}%
\pgfpathmoveto{\pgfqpoint{0.875000in}{0.440000in}}%
\pgfpathlineto{\pgfqpoint{6.300000in}{0.440000in}}%
\pgfpathlineto{\pgfqpoint{6.300000in}{3.520000in}}%
\pgfpathlineto{\pgfqpoint{0.875000in}{3.520000in}}%
\pgfpathlineto{\pgfqpoint{0.875000in}{0.440000in}}%
\pgfpathclose%
\pgfusepath{fill}%
\end{pgfscope}%
\begin{pgfscope}%
\pgfpathrectangle{\pgfqpoint{0.875000in}{0.440000in}}{\pgfqpoint{5.425000in}{3.080000in}}%
\pgfusepath{clip}%
\pgfsetroundcap%
\pgfsetroundjoin%
\pgfsetlinewidth{1.003750pt}%
\definecolor{currentstroke}{rgb}{1.000000,1.000000,1.000000}%
\pgfsetstrokecolor{currentstroke}%
\pgfsetdash{}{0pt}%
\pgfpathmoveto{\pgfqpoint{1.883906in}{0.440000in}}%
\pgfpathlineto{\pgfqpoint{1.883906in}{3.520000in}}%
\pgfusepath{stroke}%
\end{pgfscope}%
\begin{pgfscope}%
\definecolor{textcolor}{rgb}{0.150000,0.150000,0.150000}%
\pgfsetstrokecolor{textcolor}%
\pgfsetfillcolor{textcolor}%
\pgftext[x=1.883906in,y=0.342778in,,top]{\color{textcolor}\rmfamily\fontsize{10.000000}{12.000000}\selectfont \(\displaystyle {0.1}\)}%
\end{pgfscope}%
\begin{pgfscope}%
\pgfpathrectangle{\pgfqpoint{0.875000in}{0.440000in}}{\pgfqpoint{5.425000in}{3.080000in}}%
\pgfusepath{clip}%
\pgfsetroundcap%
\pgfsetroundjoin%
\pgfsetlinewidth{1.003750pt}%
\definecolor{currentstroke}{rgb}{1.000000,1.000000,1.000000}%
\pgfsetstrokecolor{currentstroke}%
\pgfsetdash{}{0pt}%
\pgfpathmoveto{\pgfqpoint{2.928708in}{0.440000in}}%
\pgfpathlineto{\pgfqpoint{2.928708in}{3.520000in}}%
\pgfusepath{stroke}%
\end{pgfscope}%
\begin{pgfscope}%
\definecolor{textcolor}{rgb}{0.150000,0.150000,0.150000}%
\pgfsetstrokecolor{textcolor}%
\pgfsetfillcolor{textcolor}%
\pgftext[x=2.928708in,y=0.342778in,,top]{\color{textcolor}\rmfamily\fontsize{10.000000}{12.000000}\selectfont \(\displaystyle {0.2}\)}%
\end{pgfscope}%
\begin{pgfscope}%
\pgfpathrectangle{\pgfqpoint{0.875000in}{0.440000in}}{\pgfqpoint{5.425000in}{3.080000in}}%
\pgfusepath{clip}%
\pgfsetroundcap%
\pgfsetroundjoin%
\pgfsetlinewidth{1.003750pt}%
\definecolor{currentstroke}{rgb}{1.000000,1.000000,1.000000}%
\pgfsetstrokecolor{currentstroke}%
\pgfsetdash{}{0pt}%
\pgfpathmoveto{\pgfqpoint{3.973511in}{0.440000in}}%
\pgfpathlineto{\pgfqpoint{3.973511in}{3.520000in}}%
\pgfusepath{stroke}%
\end{pgfscope}%
\begin{pgfscope}%
\definecolor{textcolor}{rgb}{0.150000,0.150000,0.150000}%
\pgfsetstrokecolor{textcolor}%
\pgfsetfillcolor{textcolor}%
\pgftext[x=3.973511in,y=0.342778in,,top]{\color{textcolor}\rmfamily\fontsize{10.000000}{12.000000}\selectfont \(\displaystyle {0.3}\)}%
\end{pgfscope}%
\begin{pgfscope}%
\pgfpathrectangle{\pgfqpoint{0.875000in}{0.440000in}}{\pgfqpoint{5.425000in}{3.080000in}}%
\pgfusepath{clip}%
\pgfsetroundcap%
\pgfsetroundjoin%
\pgfsetlinewidth{1.003750pt}%
\definecolor{currentstroke}{rgb}{1.000000,1.000000,1.000000}%
\pgfsetstrokecolor{currentstroke}%
\pgfsetdash{}{0pt}%
\pgfpathmoveto{\pgfqpoint{5.018313in}{0.440000in}}%
\pgfpathlineto{\pgfqpoint{5.018313in}{3.520000in}}%
\pgfusepath{stroke}%
\end{pgfscope}%
\begin{pgfscope}%
\definecolor{textcolor}{rgb}{0.150000,0.150000,0.150000}%
\pgfsetstrokecolor{textcolor}%
\pgfsetfillcolor{textcolor}%
\pgftext[x=5.018313in,y=0.342778in,,top]{\color{textcolor}\rmfamily\fontsize{10.000000}{12.000000}\selectfont \(\displaystyle {0.4}\)}%
\end{pgfscope}%
\begin{pgfscope}%
\pgfpathrectangle{\pgfqpoint{0.875000in}{0.440000in}}{\pgfqpoint{5.425000in}{3.080000in}}%
\pgfusepath{clip}%
\pgfsetroundcap%
\pgfsetroundjoin%
\pgfsetlinewidth{1.003750pt}%
\definecolor{currentstroke}{rgb}{1.000000,1.000000,1.000000}%
\pgfsetstrokecolor{currentstroke}%
\pgfsetdash{}{0pt}%
\pgfpathmoveto{\pgfqpoint{6.063116in}{0.440000in}}%
\pgfpathlineto{\pgfqpoint{6.063116in}{3.520000in}}%
\pgfusepath{stroke}%
\end{pgfscope}%
\begin{pgfscope}%
\definecolor{textcolor}{rgb}{0.150000,0.150000,0.150000}%
\pgfsetstrokecolor{textcolor}%
\pgfsetfillcolor{textcolor}%
\pgftext[x=6.063116in,y=0.342778in,,top]{\color{textcolor}\rmfamily\fontsize{10.000000}{12.000000}\selectfont \(\displaystyle {0.5}\)}%
\end{pgfscope}%
\begin{pgfscope}%
\definecolor{textcolor}{rgb}{0.150000,0.150000,0.150000}%
\pgfsetstrokecolor{textcolor}%
\pgfsetfillcolor{textcolor}%
\pgftext[x=3.587500in,y=0.163766in,,top]{\color{textcolor}\rmfamily\fontsize{11.000000}{13.200000}\selectfont Valor}%
\end{pgfscope}%
\begin{pgfscope}%
\pgfpathrectangle{\pgfqpoint{0.875000in}{0.440000in}}{\pgfqpoint{5.425000in}{3.080000in}}%
\pgfusepath{clip}%
\pgfsetroundcap%
\pgfsetroundjoin%
\pgfsetlinewidth{1.003750pt}%
\definecolor{currentstroke}{rgb}{1.000000,1.000000,1.000000}%
\pgfsetstrokecolor{currentstroke}%
\pgfsetdash{}{0pt}%
\pgfpathmoveto{\pgfqpoint{0.875000in}{0.440000in}}%
\pgfpathlineto{\pgfqpoint{6.300000in}{0.440000in}}%
\pgfusepath{stroke}%
\end{pgfscope}%
\begin{pgfscope}%
\definecolor{textcolor}{rgb}{0.150000,0.150000,0.150000}%
\pgfsetstrokecolor{textcolor}%
\pgfsetfillcolor{textcolor}%
\pgftext[x=0.600308in, y=0.391775in, left, base]{\color{textcolor}\rmfamily\fontsize{10.000000}{12.000000}\selectfont \(\displaystyle {0.0}\)}%
\end{pgfscope}%
\begin{pgfscope}%
\pgfpathrectangle{\pgfqpoint{0.875000in}{0.440000in}}{\pgfqpoint{5.425000in}{3.080000in}}%
\pgfusepath{clip}%
\pgfsetroundcap%
\pgfsetroundjoin%
\pgfsetlinewidth{1.003750pt}%
\definecolor{currentstroke}{rgb}{1.000000,1.000000,1.000000}%
\pgfsetstrokecolor{currentstroke}%
\pgfsetdash{}{0pt}%
\pgfpathmoveto{\pgfqpoint{0.875000in}{0.796467in}}%
\pgfpathlineto{\pgfqpoint{6.300000in}{0.796467in}}%
\pgfusepath{stroke}%
\end{pgfscope}%
\begin{pgfscope}%
\definecolor{textcolor}{rgb}{0.150000,0.150000,0.150000}%
\pgfsetstrokecolor{textcolor}%
\pgfsetfillcolor{textcolor}%
\pgftext[x=0.600308in, y=0.748241in, left, base]{\color{textcolor}\rmfamily\fontsize{10.000000}{12.000000}\selectfont \(\displaystyle {0.5}\)}%
\end{pgfscope}%
\begin{pgfscope}%
\pgfpathrectangle{\pgfqpoint{0.875000in}{0.440000in}}{\pgfqpoint{5.425000in}{3.080000in}}%
\pgfusepath{clip}%
\pgfsetroundcap%
\pgfsetroundjoin%
\pgfsetlinewidth{1.003750pt}%
\definecolor{currentstroke}{rgb}{1.000000,1.000000,1.000000}%
\pgfsetstrokecolor{currentstroke}%
\pgfsetdash{}{0pt}%
\pgfpathmoveto{\pgfqpoint{0.875000in}{1.152934in}}%
\pgfpathlineto{\pgfqpoint{6.300000in}{1.152934in}}%
\pgfusepath{stroke}%
\end{pgfscope}%
\begin{pgfscope}%
\definecolor{textcolor}{rgb}{0.150000,0.150000,0.150000}%
\pgfsetstrokecolor{textcolor}%
\pgfsetfillcolor{textcolor}%
\pgftext[x=0.600308in, y=1.104708in, left, base]{\color{textcolor}\rmfamily\fontsize{10.000000}{12.000000}\selectfont \(\displaystyle {1.0}\)}%
\end{pgfscope}%
\begin{pgfscope}%
\pgfpathrectangle{\pgfqpoint{0.875000in}{0.440000in}}{\pgfqpoint{5.425000in}{3.080000in}}%
\pgfusepath{clip}%
\pgfsetroundcap%
\pgfsetroundjoin%
\pgfsetlinewidth{1.003750pt}%
\definecolor{currentstroke}{rgb}{1.000000,1.000000,1.000000}%
\pgfsetstrokecolor{currentstroke}%
\pgfsetdash{}{0pt}%
\pgfpathmoveto{\pgfqpoint{0.875000in}{1.509400in}}%
\pgfpathlineto{\pgfqpoint{6.300000in}{1.509400in}}%
\pgfusepath{stroke}%
\end{pgfscope}%
\begin{pgfscope}%
\definecolor{textcolor}{rgb}{0.150000,0.150000,0.150000}%
\pgfsetstrokecolor{textcolor}%
\pgfsetfillcolor{textcolor}%
\pgftext[x=0.600308in, y=1.461175in, left, base]{\color{textcolor}\rmfamily\fontsize{10.000000}{12.000000}\selectfont \(\displaystyle {1.5}\)}%
\end{pgfscope}%
\begin{pgfscope}%
\pgfpathrectangle{\pgfqpoint{0.875000in}{0.440000in}}{\pgfqpoint{5.425000in}{3.080000in}}%
\pgfusepath{clip}%
\pgfsetroundcap%
\pgfsetroundjoin%
\pgfsetlinewidth{1.003750pt}%
\definecolor{currentstroke}{rgb}{1.000000,1.000000,1.000000}%
\pgfsetstrokecolor{currentstroke}%
\pgfsetdash{}{0pt}%
\pgfpathmoveto{\pgfqpoint{0.875000in}{1.865867in}}%
\pgfpathlineto{\pgfqpoint{6.300000in}{1.865867in}}%
\pgfusepath{stroke}%
\end{pgfscope}%
\begin{pgfscope}%
\definecolor{textcolor}{rgb}{0.150000,0.150000,0.150000}%
\pgfsetstrokecolor{textcolor}%
\pgfsetfillcolor{textcolor}%
\pgftext[x=0.600308in, y=1.817642in, left, base]{\color{textcolor}\rmfamily\fontsize{10.000000}{12.000000}\selectfont \(\displaystyle {2.0}\)}%
\end{pgfscope}%
\begin{pgfscope}%
\pgfpathrectangle{\pgfqpoint{0.875000in}{0.440000in}}{\pgfqpoint{5.425000in}{3.080000in}}%
\pgfusepath{clip}%
\pgfsetroundcap%
\pgfsetroundjoin%
\pgfsetlinewidth{1.003750pt}%
\definecolor{currentstroke}{rgb}{1.000000,1.000000,1.000000}%
\pgfsetstrokecolor{currentstroke}%
\pgfsetdash{}{0pt}%
\pgfpathmoveto{\pgfqpoint{0.875000in}{2.222334in}}%
\pgfpathlineto{\pgfqpoint{6.300000in}{2.222334in}}%
\pgfusepath{stroke}%
\end{pgfscope}%
\begin{pgfscope}%
\definecolor{textcolor}{rgb}{0.150000,0.150000,0.150000}%
\pgfsetstrokecolor{textcolor}%
\pgfsetfillcolor{textcolor}%
\pgftext[x=0.600308in, y=2.174109in, left, base]{\color{textcolor}\rmfamily\fontsize{10.000000}{12.000000}\selectfont \(\displaystyle {2.5}\)}%
\end{pgfscope}%
\begin{pgfscope}%
\pgfpathrectangle{\pgfqpoint{0.875000in}{0.440000in}}{\pgfqpoint{5.425000in}{3.080000in}}%
\pgfusepath{clip}%
\pgfsetroundcap%
\pgfsetroundjoin%
\pgfsetlinewidth{1.003750pt}%
\definecolor{currentstroke}{rgb}{1.000000,1.000000,1.000000}%
\pgfsetstrokecolor{currentstroke}%
\pgfsetdash{}{0pt}%
\pgfpathmoveto{\pgfqpoint{0.875000in}{2.578801in}}%
\pgfpathlineto{\pgfqpoint{6.300000in}{2.578801in}}%
\pgfusepath{stroke}%
\end{pgfscope}%
\begin{pgfscope}%
\definecolor{textcolor}{rgb}{0.150000,0.150000,0.150000}%
\pgfsetstrokecolor{textcolor}%
\pgfsetfillcolor{textcolor}%
\pgftext[x=0.600308in, y=2.530575in, left, base]{\color{textcolor}\rmfamily\fontsize{10.000000}{12.000000}\selectfont \(\displaystyle {3.0}\)}%
\end{pgfscope}%
\begin{pgfscope}%
\pgfpathrectangle{\pgfqpoint{0.875000in}{0.440000in}}{\pgfqpoint{5.425000in}{3.080000in}}%
\pgfusepath{clip}%
\pgfsetroundcap%
\pgfsetroundjoin%
\pgfsetlinewidth{1.003750pt}%
\definecolor{currentstroke}{rgb}{1.000000,1.000000,1.000000}%
\pgfsetstrokecolor{currentstroke}%
\pgfsetdash{}{0pt}%
\pgfpathmoveto{\pgfqpoint{0.875000in}{2.935267in}}%
\pgfpathlineto{\pgfqpoint{6.300000in}{2.935267in}}%
\pgfusepath{stroke}%
\end{pgfscope}%
\begin{pgfscope}%
\definecolor{textcolor}{rgb}{0.150000,0.150000,0.150000}%
\pgfsetstrokecolor{textcolor}%
\pgfsetfillcolor{textcolor}%
\pgftext[x=0.600308in, y=2.887042in, left, base]{\color{textcolor}\rmfamily\fontsize{10.000000}{12.000000}\selectfont \(\displaystyle {3.5}\)}%
\end{pgfscope}%
\begin{pgfscope}%
\pgfpathrectangle{\pgfqpoint{0.875000in}{0.440000in}}{\pgfqpoint{5.425000in}{3.080000in}}%
\pgfusepath{clip}%
\pgfsetroundcap%
\pgfsetroundjoin%
\pgfsetlinewidth{1.003750pt}%
\definecolor{currentstroke}{rgb}{1.000000,1.000000,1.000000}%
\pgfsetstrokecolor{currentstroke}%
\pgfsetdash{}{0pt}%
\pgfpathmoveto{\pgfqpoint{0.875000in}{3.291734in}}%
\pgfpathlineto{\pgfqpoint{6.300000in}{3.291734in}}%
\pgfusepath{stroke}%
\end{pgfscope}%
\begin{pgfscope}%
\definecolor{textcolor}{rgb}{0.150000,0.150000,0.150000}%
\pgfsetstrokecolor{textcolor}%
\pgfsetfillcolor{textcolor}%
\pgftext[x=0.600308in, y=3.243509in, left, base]{\color{textcolor}\rmfamily\fontsize{10.000000}{12.000000}\selectfont \(\displaystyle {4.0}\)}%
\end{pgfscope}%
\begin{pgfscope}%
\definecolor{textcolor}{rgb}{0.150000,0.150000,0.150000}%
\pgfsetstrokecolor{textcolor}%
\pgfsetfillcolor{textcolor}%
\pgftext[x=0.544752in,y=1.980000in,,bottom,rotate=90.000000]{\color{textcolor}\rmfamily\fontsize{11.000000}{13.200000}\selectfont Frecuencia}%
\end{pgfscope}%
\begin{pgfscope}%
\pgfpathrectangle{\pgfqpoint{0.875000in}{0.440000in}}{\pgfqpoint{5.425000in}{3.080000in}}%
\pgfusepath{clip}%
\pgfsetbuttcap%
\pgfsetmiterjoin%
\definecolor{currentfill}{rgb}{0.172549,0.627451,0.172549}%
\pgfsetfillcolor{currentfill}%
\pgfsetfillopacity{0.500000}%
\pgfsetlinewidth{0.000000pt}%
\definecolor{currentstroke}{rgb}{0.000000,0.000000,0.000000}%
\pgfsetstrokecolor{currentstroke}%
\pgfsetstrokeopacity{0.500000}%
\pgfsetdash{}{0pt}%
\pgfpathmoveto{\pgfqpoint{1.121591in}{0.440000in}}%
\pgfpathlineto{\pgfqpoint{1.285985in}{0.440000in}}%
\pgfpathlineto{\pgfqpoint{1.285985in}{0.459936in}}%
\pgfpathlineto{\pgfqpoint{1.121591in}{0.459936in}}%
\pgfpathlineto{\pgfqpoint{1.121591in}{0.440000in}}%
\pgfpathclose%
\pgfusepath{fill}%
\end{pgfscope}%
\begin{pgfscope}%
\pgfpathrectangle{\pgfqpoint{0.875000in}{0.440000in}}{\pgfqpoint{5.425000in}{3.080000in}}%
\pgfusepath{clip}%
\pgfsetbuttcap%
\pgfsetmiterjoin%
\definecolor{currentfill}{rgb}{0.172549,0.627451,0.172549}%
\pgfsetfillcolor{currentfill}%
\pgfsetfillopacity{0.500000}%
\pgfsetlinewidth{0.000000pt}%
\definecolor{currentstroke}{rgb}{0.000000,0.000000,0.000000}%
\pgfsetstrokecolor{currentstroke}%
\pgfsetstrokeopacity{0.500000}%
\pgfsetdash{}{0pt}%
\pgfpathmoveto{\pgfqpoint{1.285985in}{0.440000in}}%
\pgfpathlineto{\pgfqpoint{1.450379in}{0.440000in}}%
\pgfpathlineto{\pgfqpoint{1.450379in}{0.476248in}}%
\pgfpathlineto{\pgfqpoint{1.285985in}{0.476248in}}%
\pgfpathlineto{\pgfqpoint{1.285985in}{0.440000in}}%
\pgfpathclose%
\pgfusepath{fill}%
\end{pgfscope}%
\begin{pgfscope}%
\pgfpathrectangle{\pgfqpoint{0.875000in}{0.440000in}}{\pgfqpoint{5.425000in}{3.080000in}}%
\pgfusepath{clip}%
\pgfsetbuttcap%
\pgfsetmiterjoin%
\definecolor{currentfill}{rgb}{0.172549,0.627451,0.172549}%
\pgfsetfillcolor{currentfill}%
\pgfsetfillopacity{0.500000}%
\pgfsetlinewidth{0.000000pt}%
\definecolor{currentstroke}{rgb}{0.000000,0.000000,0.000000}%
\pgfsetstrokecolor{currentstroke}%
\pgfsetstrokeopacity{0.500000}%
\pgfsetdash{}{0pt}%
\pgfpathmoveto{\pgfqpoint{1.450379in}{0.440000in}}%
\pgfpathlineto{\pgfqpoint{1.614773in}{0.440000in}}%
\pgfpathlineto{\pgfqpoint{1.614773in}{0.540587in}}%
\pgfpathlineto{\pgfqpoint{1.450379in}{0.540587in}}%
\pgfpathlineto{\pgfqpoint{1.450379in}{0.440000in}}%
\pgfpathclose%
\pgfusepath{fill}%
\end{pgfscope}%
\begin{pgfscope}%
\pgfpathrectangle{\pgfqpoint{0.875000in}{0.440000in}}{\pgfqpoint{5.425000in}{3.080000in}}%
\pgfusepath{clip}%
\pgfsetbuttcap%
\pgfsetmiterjoin%
\definecolor{currentfill}{rgb}{0.172549,0.627451,0.172549}%
\pgfsetfillcolor{currentfill}%
\pgfsetfillopacity{0.500000}%
\pgfsetlinewidth{0.000000pt}%
\definecolor{currentstroke}{rgb}{0.000000,0.000000,0.000000}%
\pgfsetstrokecolor{currentstroke}%
\pgfsetstrokeopacity{0.500000}%
\pgfsetdash{}{0pt}%
\pgfpathmoveto{\pgfqpoint{1.614773in}{0.440000in}}%
\pgfpathlineto{\pgfqpoint{1.779167in}{0.440000in}}%
\pgfpathlineto{\pgfqpoint{1.779167in}{0.605833in}}%
\pgfpathlineto{\pgfqpoint{1.614773in}{0.605833in}}%
\pgfpathlineto{\pgfqpoint{1.614773in}{0.440000in}}%
\pgfpathclose%
\pgfusepath{fill}%
\end{pgfscope}%
\begin{pgfscope}%
\pgfpathrectangle{\pgfqpoint{0.875000in}{0.440000in}}{\pgfqpoint{5.425000in}{3.080000in}}%
\pgfusepath{clip}%
\pgfsetbuttcap%
\pgfsetmiterjoin%
\definecolor{currentfill}{rgb}{0.172549,0.627451,0.172549}%
\pgfsetfillcolor{currentfill}%
\pgfsetfillopacity{0.500000}%
\pgfsetlinewidth{0.000000pt}%
\definecolor{currentstroke}{rgb}{0.000000,0.000000,0.000000}%
\pgfsetstrokecolor{currentstroke}%
\pgfsetstrokeopacity{0.500000}%
\pgfsetdash{}{0pt}%
\pgfpathmoveto{\pgfqpoint{1.779167in}{0.440000in}}%
\pgfpathlineto{\pgfqpoint{1.943561in}{0.440000in}}%
\pgfpathlineto{\pgfqpoint{1.943561in}{0.683765in}}%
\pgfpathlineto{\pgfqpoint{1.779167in}{0.683765in}}%
\pgfpathlineto{\pgfqpoint{1.779167in}{0.440000in}}%
\pgfpathclose%
\pgfusepath{fill}%
\end{pgfscope}%
\begin{pgfscope}%
\pgfpathrectangle{\pgfqpoint{0.875000in}{0.440000in}}{\pgfqpoint{5.425000in}{3.080000in}}%
\pgfusepath{clip}%
\pgfsetbuttcap%
\pgfsetmiterjoin%
\definecolor{currentfill}{rgb}{0.172549,0.627451,0.172549}%
\pgfsetfillcolor{currentfill}%
\pgfsetfillopacity{0.500000}%
\pgfsetlinewidth{0.000000pt}%
\definecolor{currentstroke}{rgb}{0.000000,0.000000,0.000000}%
\pgfsetstrokecolor{currentstroke}%
\pgfsetstrokeopacity{0.500000}%
\pgfsetdash{}{0pt}%
\pgfpathmoveto{\pgfqpoint{1.943561in}{0.440000in}}%
\pgfpathlineto{\pgfqpoint{2.107955in}{0.440000in}}%
\pgfpathlineto{\pgfqpoint{2.107955in}{0.843254in}}%
\pgfpathlineto{\pgfqpoint{1.943561in}{0.843254in}}%
\pgfpathlineto{\pgfqpoint{1.943561in}{0.440000in}}%
\pgfpathclose%
\pgfusepath{fill}%
\end{pgfscope}%
\begin{pgfscope}%
\pgfpathrectangle{\pgfqpoint{0.875000in}{0.440000in}}{\pgfqpoint{5.425000in}{3.080000in}}%
\pgfusepath{clip}%
\pgfsetbuttcap%
\pgfsetmiterjoin%
\definecolor{currentfill}{rgb}{0.172549,0.627451,0.172549}%
\pgfsetfillcolor{currentfill}%
\pgfsetfillopacity{0.500000}%
\pgfsetlinewidth{0.000000pt}%
\definecolor{currentstroke}{rgb}{0.000000,0.000000,0.000000}%
\pgfsetstrokecolor{currentstroke}%
\pgfsetstrokeopacity{0.500000}%
\pgfsetdash{}{0pt}%
\pgfpathmoveto{\pgfqpoint{2.107955in}{0.440000in}}%
\pgfpathlineto{\pgfqpoint{2.272348in}{0.440000in}}%
\pgfpathlineto{\pgfqpoint{2.272348in}{0.941122in}}%
\pgfpathlineto{\pgfqpoint{2.107955in}{0.941122in}}%
\pgfpathlineto{\pgfqpoint{2.107955in}{0.440000in}}%
\pgfpathclose%
\pgfusepath{fill}%
\end{pgfscope}%
\begin{pgfscope}%
\pgfpathrectangle{\pgfqpoint{0.875000in}{0.440000in}}{\pgfqpoint{5.425000in}{3.080000in}}%
\pgfusepath{clip}%
\pgfsetbuttcap%
\pgfsetmiterjoin%
\definecolor{currentfill}{rgb}{0.172549,0.627451,0.172549}%
\pgfsetfillcolor{currentfill}%
\pgfsetfillopacity{0.500000}%
\pgfsetlinewidth{0.000000pt}%
\definecolor{currentstroke}{rgb}{0.000000,0.000000,0.000000}%
\pgfsetstrokecolor{currentstroke}%
\pgfsetstrokeopacity{0.500000}%
\pgfsetdash{}{0pt}%
\pgfpathmoveto{\pgfqpoint{2.272348in}{0.440000in}}%
\pgfpathlineto{\pgfqpoint{2.436742in}{0.440000in}}%
\pgfpathlineto{\pgfqpoint{2.436742in}{1.154983in}}%
\pgfpathlineto{\pgfqpoint{2.272348in}{1.154983in}}%
\pgfpathlineto{\pgfqpoint{2.272348in}{0.440000in}}%
\pgfpathclose%
\pgfusepath{fill}%
\end{pgfscope}%
\begin{pgfscope}%
\pgfpathrectangle{\pgfqpoint{0.875000in}{0.440000in}}{\pgfqpoint{5.425000in}{3.080000in}}%
\pgfusepath{clip}%
\pgfsetbuttcap%
\pgfsetmiterjoin%
\definecolor{currentfill}{rgb}{0.172549,0.627451,0.172549}%
\pgfsetfillcolor{currentfill}%
\pgfsetfillopacity{0.500000}%
\pgfsetlinewidth{0.000000pt}%
\definecolor{currentstroke}{rgb}{0.000000,0.000000,0.000000}%
\pgfsetstrokecolor{currentstroke}%
\pgfsetstrokeopacity{0.500000}%
\pgfsetdash{}{0pt}%
\pgfpathmoveto{\pgfqpoint{2.436742in}{0.440000in}}%
\pgfpathlineto{\pgfqpoint{2.601136in}{0.440000in}}%
\pgfpathlineto{\pgfqpoint{2.601136in}{1.223853in}}%
\pgfpathlineto{\pgfqpoint{2.436742in}{1.223853in}}%
\pgfpathlineto{\pgfqpoint{2.436742in}{0.440000in}}%
\pgfpathclose%
\pgfusepath{fill}%
\end{pgfscope}%
\begin{pgfscope}%
\pgfpathrectangle{\pgfqpoint{0.875000in}{0.440000in}}{\pgfqpoint{5.425000in}{3.080000in}}%
\pgfusepath{clip}%
\pgfsetbuttcap%
\pgfsetmiterjoin%
\definecolor{currentfill}{rgb}{0.172549,0.627451,0.172549}%
\pgfsetfillcolor{currentfill}%
\pgfsetfillopacity{0.500000}%
\pgfsetlinewidth{0.000000pt}%
\definecolor{currentstroke}{rgb}{0.000000,0.000000,0.000000}%
\pgfsetstrokecolor{currentstroke}%
\pgfsetstrokeopacity{0.500000}%
\pgfsetdash{}{0pt}%
\pgfpathmoveto{\pgfqpoint{2.601136in}{0.440000in}}%
\pgfpathlineto{\pgfqpoint{2.765530in}{0.440000in}}%
\pgfpathlineto{\pgfqpoint{2.765530in}{1.603546in}}%
\pgfpathlineto{\pgfqpoint{2.601136in}{1.603546in}}%
\pgfpathlineto{\pgfqpoint{2.601136in}{0.440000in}}%
\pgfpathclose%
\pgfusepath{fill}%
\end{pgfscope}%
\begin{pgfscope}%
\pgfpathrectangle{\pgfqpoint{0.875000in}{0.440000in}}{\pgfqpoint{5.425000in}{3.080000in}}%
\pgfusepath{clip}%
\pgfsetbuttcap%
\pgfsetmiterjoin%
\definecolor{currentfill}{rgb}{0.172549,0.627451,0.172549}%
\pgfsetfillcolor{currentfill}%
\pgfsetfillopacity{0.500000}%
\pgfsetlinewidth{0.000000pt}%
\definecolor{currentstroke}{rgb}{0.000000,0.000000,0.000000}%
\pgfsetstrokecolor{currentstroke}%
\pgfsetstrokeopacity{0.500000}%
\pgfsetdash{}{0pt}%
\pgfpathmoveto{\pgfqpoint{2.765530in}{0.440000in}}%
\pgfpathlineto{\pgfqpoint{2.929924in}{0.440000in}}%
\pgfpathlineto{\pgfqpoint{2.929924in}{1.719539in}}%
\pgfpathlineto{\pgfqpoint{2.765530in}{1.719539in}}%
\pgfpathlineto{\pgfqpoint{2.765530in}{0.440000in}}%
\pgfpathclose%
\pgfusepath{fill}%
\end{pgfscope}%
\begin{pgfscope}%
\pgfpathrectangle{\pgfqpoint{0.875000in}{0.440000in}}{\pgfqpoint{5.425000in}{3.080000in}}%
\pgfusepath{clip}%
\pgfsetbuttcap%
\pgfsetmiterjoin%
\definecolor{currentfill}{rgb}{0.172549,0.627451,0.172549}%
\pgfsetfillcolor{currentfill}%
\pgfsetfillopacity{0.500000}%
\pgfsetlinewidth{0.000000pt}%
\definecolor{currentstroke}{rgb}{0.000000,0.000000,0.000000}%
\pgfsetstrokecolor{currentstroke}%
\pgfsetstrokeopacity{0.500000}%
\pgfsetdash{}{0pt}%
\pgfpathmoveto{\pgfqpoint{2.929924in}{0.440000in}}%
\pgfpathlineto{\pgfqpoint{3.094318in}{0.440000in}}%
\pgfpathlineto{\pgfqpoint{3.094318in}{1.985052in}}%
\pgfpathlineto{\pgfqpoint{2.929924in}{1.985052in}}%
\pgfpathlineto{\pgfqpoint{2.929924in}{0.440000in}}%
\pgfpathclose%
\pgfusepath{fill}%
\end{pgfscope}%
\begin{pgfscope}%
\pgfpathrectangle{\pgfqpoint{0.875000in}{0.440000in}}{\pgfqpoint{5.425000in}{3.080000in}}%
\pgfusepath{clip}%
\pgfsetbuttcap%
\pgfsetmiterjoin%
\definecolor{currentfill}{rgb}{0.172549,0.627451,0.172549}%
\pgfsetfillcolor{currentfill}%
\pgfsetfillopacity{0.500000}%
\pgfsetlinewidth{0.000000pt}%
\definecolor{currentstroke}{rgb}{0.000000,0.000000,0.000000}%
\pgfsetstrokecolor{currentstroke}%
\pgfsetstrokeopacity{0.500000}%
\pgfsetdash{}{0pt}%
\pgfpathmoveto{\pgfqpoint{3.094318in}{0.440000in}}%
\pgfpathlineto{\pgfqpoint{3.258712in}{0.440000in}}%
\pgfpathlineto{\pgfqpoint{3.258712in}{2.213412in}}%
\pgfpathlineto{\pgfqpoint{3.094318in}{2.213412in}}%
\pgfpathlineto{\pgfqpoint{3.094318in}{0.440000in}}%
\pgfpathclose%
\pgfusepath{fill}%
\end{pgfscope}%
\begin{pgfscope}%
\pgfpathrectangle{\pgfqpoint{0.875000in}{0.440000in}}{\pgfqpoint{5.425000in}{3.080000in}}%
\pgfusepath{clip}%
\pgfsetbuttcap%
\pgfsetmiterjoin%
\definecolor{currentfill}{rgb}{0.172549,0.627451,0.172549}%
\pgfsetfillcolor{currentfill}%
\pgfsetfillopacity{0.500000}%
\pgfsetlinewidth{0.000000pt}%
\definecolor{currentstroke}{rgb}{0.000000,0.000000,0.000000}%
\pgfsetstrokecolor{currentstroke}%
\pgfsetstrokeopacity{0.500000}%
\pgfsetdash{}{0pt}%
\pgfpathmoveto{\pgfqpoint{3.258712in}{0.440000in}}%
\pgfpathlineto{\pgfqpoint{3.423106in}{0.440000in}}%
\pgfpathlineto{\pgfqpoint{3.423106in}{2.537827in}}%
\pgfpathlineto{\pgfqpoint{3.258712in}{2.537827in}}%
\pgfpathlineto{\pgfqpoint{3.258712in}{0.440000in}}%
\pgfpathclose%
\pgfusepath{fill}%
\end{pgfscope}%
\begin{pgfscope}%
\pgfpathrectangle{\pgfqpoint{0.875000in}{0.440000in}}{\pgfqpoint{5.425000in}{3.080000in}}%
\pgfusepath{clip}%
\pgfsetbuttcap%
\pgfsetmiterjoin%
\definecolor{currentfill}{rgb}{0.172549,0.627451,0.172549}%
\pgfsetfillcolor{currentfill}%
\pgfsetfillopacity{0.500000}%
\pgfsetlinewidth{0.000000pt}%
\definecolor{currentstroke}{rgb}{0.000000,0.000000,0.000000}%
\pgfsetstrokecolor{currentstroke}%
\pgfsetstrokeopacity{0.500000}%
\pgfsetdash{}{0pt}%
\pgfpathmoveto{\pgfqpoint{3.423106in}{0.440000in}}%
\pgfpathlineto{\pgfqpoint{3.587500in}{0.440000in}}%
\pgfpathlineto{\pgfqpoint{3.587500in}{2.570450in}}%
\pgfpathlineto{\pgfqpoint{3.423106in}{2.570450in}}%
\pgfpathlineto{\pgfqpoint{3.423106in}{0.440000in}}%
\pgfpathclose%
\pgfusepath{fill}%
\end{pgfscope}%
\begin{pgfscope}%
\pgfpathrectangle{\pgfqpoint{0.875000in}{0.440000in}}{\pgfqpoint{5.425000in}{3.080000in}}%
\pgfusepath{clip}%
\pgfsetbuttcap%
\pgfsetmiterjoin%
\definecolor{currentfill}{rgb}{0.172549,0.627451,0.172549}%
\pgfsetfillcolor{currentfill}%
\pgfsetfillopacity{0.500000}%
\pgfsetlinewidth{0.000000pt}%
\definecolor{currentstroke}{rgb}{0.000000,0.000000,0.000000}%
\pgfsetstrokecolor{currentstroke}%
\pgfsetstrokeopacity{0.500000}%
\pgfsetdash{}{0pt}%
\pgfpathmoveto{\pgfqpoint{3.587500in}{0.440000in}}%
\pgfpathlineto{\pgfqpoint{3.751894in}{0.440000in}}%
\pgfpathlineto{\pgfqpoint{3.751894in}{2.937456in}}%
\pgfpathlineto{\pgfqpoint{3.587500in}{2.937456in}}%
\pgfpathlineto{\pgfqpoint{3.587500in}{0.440000in}}%
\pgfpathclose%
\pgfusepath{fill}%
\end{pgfscope}%
\begin{pgfscope}%
\pgfpathrectangle{\pgfqpoint{0.875000in}{0.440000in}}{\pgfqpoint{5.425000in}{3.080000in}}%
\pgfusepath{clip}%
\pgfsetbuttcap%
\pgfsetmiterjoin%
\definecolor{currentfill}{rgb}{0.172549,0.627451,0.172549}%
\pgfsetfillcolor{currentfill}%
\pgfsetfillopacity{0.500000}%
\pgfsetlinewidth{0.000000pt}%
\definecolor{currentstroke}{rgb}{0.000000,0.000000,0.000000}%
\pgfsetstrokecolor{currentstroke}%
\pgfsetstrokeopacity{0.500000}%
\pgfsetdash{}{0pt}%
\pgfpathmoveto{\pgfqpoint{3.751894in}{0.440000in}}%
\pgfpathlineto{\pgfqpoint{3.916288in}{0.440000in}}%
\pgfpathlineto{\pgfqpoint{3.916288in}{3.022638in}}%
\pgfpathlineto{\pgfqpoint{3.751894in}{3.022638in}}%
\pgfpathlineto{\pgfqpoint{3.751894in}{0.440000in}}%
\pgfpathclose%
\pgfusepath{fill}%
\end{pgfscope}%
\begin{pgfscope}%
\pgfpathrectangle{\pgfqpoint{0.875000in}{0.440000in}}{\pgfqpoint{5.425000in}{3.080000in}}%
\pgfusepath{clip}%
\pgfsetbuttcap%
\pgfsetmiterjoin%
\definecolor{currentfill}{rgb}{0.172549,0.627451,0.172549}%
\pgfsetfillcolor{currentfill}%
\pgfsetfillopacity{0.500000}%
\pgfsetlinewidth{0.000000pt}%
\definecolor{currentstroke}{rgb}{0.000000,0.000000,0.000000}%
\pgfsetstrokecolor{currentstroke}%
\pgfsetstrokeopacity{0.500000}%
\pgfsetdash{}{0pt}%
\pgfpathmoveto{\pgfqpoint{3.916288in}{0.440000in}}%
\pgfpathlineto{\pgfqpoint{4.080682in}{0.440000in}}%
\pgfpathlineto{\pgfqpoint{4.080682in}{3.169441in}}%
\pgfpathlineto{\pgfqpoint{3.916288in}{3.169441in}}%
\pgfpathlineto{\pgfqpoint{3.916288in}{0.440000in}}%
\pgfpathclose%
\pgfusepath{fill}%
\end{pgfscope}%
\begin{pgfscope}%
\pgfpathrectangle{\pgfqpoint{0.875000in}{0.440000in}}{\pgfqpoint{5.425000in}{3.080000in}}%
\pgfusepath{clip}%
\pgfsetbuttcap%
\pgfsetmiterjoin%
\definecolor{currentfill}{rgb}{0.172549,0.627451,0.172549}%
\pgfsetfillcolor{currentfill}%
\pgfsetfillopacity{0.500000}%
\pgfsetlinewidth{0.000000pt}%
\definecolor{currentstroke}{rgb}{0.000000,0.000000,0.000000}%
\pgfsetstrokecolor{currentstroke}%
\pgfsetstrokeopacity{0.500000}%
\pgfsetdash{}{0pt}%
\pgfpathmoveto{\pgfqpoint{4.080682in}{0.440000in}}%
\pgfpathlineto{\pgfqpoint{4.245076in}{0.440000in}}%
\pgfpathlineto{\pgfqpoint{4.245076in}{3.193002in}}%
\pgfpathlineto{\pgfqpoint{4.080682in}{3.193002in}}%
\pgfpathlineto{\pgfqpoint{4.080682in}{0.440000in}}%
\pgfpathclose%
\pgfusepath{fill}%
\end{pgfscope}%
\begin{pgfscope}%
\pgfpathrectangle{\pgfqpoint{0.875000in}{0.440000in}}{\pgfqpoint{5.425000in}{3.080000in}}%
\pgfusepath{clip}%
\pgfsetbuttcap%
\pgfsetmiterjoin%
\definecolor{currentfill}{rgb}{0.172549,0.627451,0.172549}%
\pgfsetfillcolor{currentfill}%
\pgfsetfillopacity{0.500000}%
\pgfsetlinewidth{0.000000pt}%
\definecolor{currentstroke}{rgb}{0.000000,0.000000,0.000000}%
\pgfsetstrokecolor{currentstroke}%
\pgfsetstrokeopacity{0.500000}%
\pgfsetdash{}{0pt}%
\pgfpathmoveto{\pgfqpoint{4.245076in}{0.440000in}}%
\pgfpathlineto{\pgfqpoint{4.409470in}{0.440000in}}%
\pgfpathlineto{\pgfqpoint{4.409470in}{3.278184in}}%
\pgfpathlineto{\pgfqpoint{4.245076in}{3.278184in}}%
\pgfpathlineto{\pgfqpoint{4.245076in}{0.440000in}}%
\pgfpathclose%
\pgfusepath{fill}%
\end{pgfscope}%
\begin{pgfscope}%
\pgfpathrectangle{\pgfqpoint{0.875000in}{0.440000in}}{\pgfqpoint{5.425000in}{3.080000in}}%
\pgfusepath{clip}%
\pgfsetbuttcap%
\pgfsetmiterjoin%
\definecolor{currentfill}{rgb}{0.172549,0.627451,0.172549}%
\pgfsetfillcolor{currentfill}%
\pgfsetfillopacity{0.500000}%
\pgfsetlinewidth{0.000000pt}%
\definecolor{currentstroke}{rgb}{0.000000,0.000000,0.000000}%
\pgfsetstrokecolor{currentstroke}%
\pgfsetstrokeopacity{0.500000}%
\pgfsetdash{}{0pt}%
\pgfpathmoveto{\pgfqpoint{4.409470in}{0.440000in}}%
\pgfpathlineto{\pgfqpoint{4.573864in}{0.440000in}}%
\pgfpathlineto{\pgfqpoint{4.573864in}{3.373333in}}%
\pgfpathlineto{\pgfqpoint{4.409470in}{3.373333in}}%
\pgfpathlineto{\pgfqpoint{4.409470in}{0.440000in}}%
\pgfpathclose%
\pgfusepath{fill}%
\end{pgfscope}%
\begin{pgfscope}%
\pgfpathrectangle{\pgfqpoint{0.875000in}{0.440000in}}{\pgfqpoint{5.425000in}{3.080000in}}%
\pgfusepath{clip}%
\pgfsetbuttcap%
\pgfsetmiterjoin%
\definecolor{currentfill}{rgb}{0.172549,0.627451,0.172549}%
\pgfsetfillcolor{currentfill}%
\pgfsetfillopacity{0.500000}%
\pgfsetlinewidth{0.000000pt}%
\definecolor{currentstroke}{rgb}{0.000000,0.000000,0.000000}%
\pgfsetstrokecolor{currentstroke}%
\pgfsetstrokeopacity{0.500000}%
\pgfsetdash{}{0pt}%
\pgfpathmoveto{\pgfqpoint{4.573864in}{0.440000in}}%
\pgfpathlineto{\pgfqpoint{4.738258in}{0.440000in}}%
\pgfpathlineto{\pgfqpoint{4.738258in}{3.212938in}}%
\pgfpathlineto{\pgfqpoint{4.573864in}{3.212938in}}%
\pgfpathlineto{\pgfqpoint{4.573864in}{0.440000in}}%
\pgfpathclose%
\pgfusepath{fill}%
\end{pgfscope}%
\begin{pgfscope}%
\pgfpathrectangle{\pgfqpoint{0.875000in}{0.440000in}}{\pgfqpoint{5.425000in}{3.080000in}}%
\pgfusepath{clip}%
\pgfsetbuttcap%
\pgfsetmiterjoin%
\definecolor{currentfill}{rgb}{0.172549,0.627451,0.172549}%
\pgfsetfillcolor{currentfill}%
\pgfsetfillopacity{0.500000}%
\pgfsetlinewidth{0.000000pt}%
\definecolor{currentstroke}{rgb}{0.000000,0.000000,0.000000}%
\pgfsetstrokecolor{currentstroke}%
\pgfsetstrokeopacity{0.500000}%
\pgfsetdash{}{0pt}%
\pgfpathmoveto{\pgfqpoint{4.738258in}{0.440000in}}%
\pgfpathlineto{\pgfqpoint{4.902652in}{0.440000in}}%
\pgfpathlineto{\pgfqpoint{4.902652in}{3.148598in}}%
\pgfpathlineto{\pgfqpoint{4.738258in}{3.148598in}}%
\pgfpathlineto{\pgfqpoint{4.738258in}{0.440000in}}%
\pgfpathclose%
\pgfusepath{fill}%
\end{pgfscope}%
\begin{pgfscope}%
\pgfpathrectangle{\pgfqpoint{0.875000in}{0.440000in}}{\pgfqpoint{5.425000in}{3.080000in}}%
\pgfusepath{clip}%
\pgfsetbuttcap%
\pgfsetmiterjoin%
\definecolor{currentfill}{rgb}{0.172549,0.627451,0.172549}%
\pgfsetfillcolor{currentfill}%
\pgfsetfillopacity{0.500000}%
\pgfsetlinewidth{0.000000pt}%
\definecolor{currentstroke}{rgb}{0.000000,0.000000,0.000000}%
\pgfsetstrokecolor{currentstroke}%
\pgfsetstrokeopacity{0.500000}%
\pgfsetdash{}{0pt}%
\pgfpathmoveto{\pgfqpoint{4.902652in}{0.440000in}}%
\pgfpathlineto{\pgfqpoint{5.067045in}{0.440000in}}%
\pgfpathlineto{\pgfqpoint{5.067045in}{2.895772in}}%
\pgfpathlineto{\pgfqpoint{4.902652in}{2.895772in}}%
\pgfpathlineto{\pgfqpoint{4.902652in}{0.440000in}}%
\pgfpathclose%
\pgfusepath{fill}%
\end{pgfscope}%
\begin{pgfscope}%
\pgfpathrectangle{\pgfqpoint{0.875000in}{0.440000in}}{\pgfqpoint{5.425000in}{3.080000in}}%
\pgfusepath{clip}%
\pgfsetbuttcap%
\pgfsetmiterjoin%
\definecolor{currentfill}{rgb}{0.172549,0.627451,0.172549}%
\pgfsetfillcolor{currentfill}%
\pgfsetfillopacity{0.500000}%
\pgfsetlinewidth{0.000000pt}%
\definecolor{currentstroke}{rgb}{0.000000,0.000000,0.000000}%
\pgfsetstrokecolor{currentstroke}%
\pgfsetstrokeopacity{0.500000}%
\pgfsetdash{}{0pt}%
\pgfpathmoveto{\pgfqpoint{5.067045in}{0.440000in}}%
\pgfpathlineto{\pgfqpoint{5.231439in}{0.440000in}}%
\pgfpathlineto{\pgfqpoint{5.231439in}{2.745344in}}%
\pgfpathlineto{\pgfqpoint{5.067045in}{2.745344in}}%
\pgfpathlineto{\pgfqpoint{5.067045in}{0.440000in}}%
\pgfpathclose%
\pgfusepath{fill}%
\end{pgfscope}%
\begin{pgfscope}%
\pgfpathrectangle{\pgfqpoint{0.875000in}{0.440000in}}{\pgfqpoint{5.425000in}{3.080000in}}%
\pgfusepath{clip}%
\pgfsetbuttcap%
\pgfsetmiterjoin%
\definecolor{currentfill}{rgb}{0.172549,0.627451,0.172549}%
\pgfsetfillcolor{currentfill}%
\pgfsetfillopacity{0.500000}%
\pgfsetlinewidth{0.000000pt}%
\definecolor{currentstroke}{rgb}{0.000000,0.000000,0.000000}%
\pgfsetstrokecolor{currentstroke}%
\pgfsetstrokeopacity{0.500000}%
\pgfsetdash{}{0pt}%
\pgfpathmoveto{\pgfqpoint{5.231439in}{0.440000in}}%
\pgfpathlineto{\pgfqpoint{5.395833in}{0.440000in}}%
\pgfpathlineto{\pgfqpoint{5.395833in}{2.299500in}}%
\pgfpathlineto{\pgfqpoint{5.231439in}{2.299500in}}%
\pgfpathlineto{\pgfqpoint{5.231439in}{0.440000in}}%
\pgfpathclose%
\pgfusepath{fill}%
\end{pgfscope}%
\begin{pgfscope}%
\pgfpathrectangle{\pgfqpoint{0.875000in}{0.440000in}}{\pgfqpoint{5.425000in}{3.080000in}}%
\pgfusepath{clip}%
\pgfsetbuttcap%
\pgfsetmiterjoin%
\definecolor{currentfill}{rgb}{0.172549,0.627451,0.172549}%
\pgfsetfillcolor{currentfill}%
\pgfsetfillopacity{0.500000}%
\pgfsetlinewidth{0.000000pt}%
\definecolor{currentstroke}{rgb}{0.000000,0.000000,0.000000}%
\pgfsetstrokecolor{currentstroke}%
\pgfsetstrokeopacity{0.500000}%
\pgfsetdash{}{0pt}%
\pgfpathmoveto{\pgfqpoint{5.395833in}{0.440000in}}%
\pgfpathlineto{\pgfqpoint{5.560227in}{0.440000in}}%
\pgfpathlineto{\pgfqpoint{5.560227in}{2.070234in}}%
\pgfpathlineto{\pgfqpoint{5.395833in}{2.070234in}}%
\pgfpathlineto{\pgfqpoint{5.395833in}{0.440000in}}%
\pgfpathclose%
\pgfusepath{fill}%
\end{pgfscope}%
\begin{pgfscope}%
\pgfpathrectangle{\pgfqpoint{0.875000in}{0.440000in}}{\pgfqpoint{5.425000in}{3.080000in}}%
\pgfusepath{clip}%
\pgfsetbuttcap%
\pgfsetmiterjoin%
\definecolor{currentfill}{rgb}{0.172549,0.627451,0.172549}%
\pgfsetfillcolor{currentfill}%
\pgfsetfillopacity{0.500000}%
\pgfsetlinewidth{0.000000pt}%
\definecolor{currentstroke}{rgb}{0.000000,0.000000,0.000000}%
\pgfsetstrokecolor{currentstroke}%
\pgfsetstrokeopacity{0.500000}%
\pgfsetdash{}{0pt}%
\pgfpathmoveto{\pgfqpoint{5.560227in}{0.440000in}}%
\pgfpathlineto{\pgfqpoint{5.724621in}{0.440000in}}%
\pgfpathlineto{\pgfqpoint{5.724621in}{1.657918in}}%
\pgfpathlineto{\pgfqpoint{5.560227in}{1.657918in}}%
\pgfpathlineto{\pgfqpoint{5.560227in}{0.440000in}}%
\pgfpathclose%
\pgfusepath{fill}%
\end{pgfscope}%
\begin{pgfscope}%
\pgfpathrectangle{\pgfqpoint{0.875000in}{0.440000in}}{\pgfqpoint{5.425000in}{3.080000in}}%
\pgfusepath{clip}%
\pgfsetbuttcap%
\pgfsetmiterjoin%
\definecolor{currentfill}{rgb}{0.172549,0.627451,0.172549}%
\pgfsetfillcolor{currentfill}%
\pgfsetfillopacity{0.500000}%
\pgfsetlinewidth{0.000000pt}%
\definecolor{currentstroke}{rgb}{0.000000,0.000000,0.000000}%
\pgfsetstrokecolor{currentstroke}%
\pgfsetstrokeopacity{0.500000}%
\pgfsetdash{}{0pt}%
\pgfpathmoveto{\pgfqpoint{5.724621in}{0.440000in}}%
\pgfpathlineto{\pgfqpoint{5.889015in}{0.440000in}}%
\pgfpathlineto{\pgfqpoint{5.889015in}{1.237446in}}%
\pgfpathlineto{\pgfqpoint{5.724621in}{1.237446in}}%
\pgfpathlineto{\pgfqpoint{5.724621in}{0.440000in}}%
\pgfpathclose%
\pgfusepath{fill}%
\end{pgfscope}%
\begin{pgfscope}%
\pgfpathrectangle{\pgfqpoint{0.875000in}{0.440000in}}{\pgfqpoint{5.425000in}{3.080000in}}%
\pgfusepath{clip}%
\pgfsetbuttcap%
\pgfsetmiterjoin%
\definecolor{currentfill}{rgb}{0.172549,0.627451,0.172549}%
\pgfsetfillcolor{currentfill}%
\pgfsetfillopacity{0.500000}%
\pgfsetlinewidth{0.000000pt}%
\definecolor{currentstroke}{rgb}{0.000000,0.000000,0.000000}%
\pgfsetstrokecolor{currentstroke}%
\pgfsetstrokeopacity{0.500000}%
\pgfsetdash{}{0pt}%
\pgfpathmoveto{\pgfqpoint{5.889015in}{0.440000in}}%
\pgfpathlineto{\pgfqpoint{6.053409in}{0.440000in}}%
\pgfpathlineto{\pgfqpoint{6.053409in}{0.709138in}}%
\pgfpathlineto{\pgfqpoint{5.889015in}{0.709138in}}%
\pgfpathlineto{\pgfqpoint{5.889015in}{0.440000in}}%
\pgfpathclose%
\pgfusepath{fill}%
\end{pgfscope}%
\begin{pgfscope}%
\pgfsetrectcap%
\pgfsetmiterjoin%
\pgfsetlinewidth{0.000000pt}%
\definecolor{currentstroke}{rgb}{1.000000,1.000000,1.000000}%
\pgfsetstrokecolor{currentstroke}%
\pgfsetdash{}{0pt}%
\pgfpathmoveto{\pgfqpoint{0.875000in}{0.440000in}}%
\pgfpathlineto{\pgfqpoint{0.875000in}{3.520000in}}%
\pgfusepath{}%
\end{pgfscope}%
\begin{pgfscope}%
\pgfsetrectcap%
\pgfsetmiterjoin%
\pgfsetlinewidth{0.000000pt}%
\definecolor{currentstroke}{rgb}{1.000000,1.000000,1.000000}%
\pgfsetstrokecolor{currentstroke}%
\pgfsetdash{}{0pt}%
\pgfpathmoveto{\pgfqpoint{6.300000in}{0.440000in}}%
\pgfpathlineto{\pgfqpoint{6.300000in}{3.520000in}}%
\pgfusepath{}%
\end{pgfscope}%
\begin{pgfscope}%
\pgfsetrectcap%
\pgfsetmiterjoin%
\pgfsetlinewidth{0.000000pt}%
\definecolor{currentstroke}{rgb}{1.000000,1.000000,1.000000}%
\pgfsetstrokecolor{currentstroke}%
\pgfsetdash{}{0pt}%
\pgfpathmoveto{\pgfqpoint{0.875000in}{0.440000in}}%
\pgfpathlineto{\pgfqpoint{6.300000in}{0.440000in}}%
\pgfusepath{}%
\end{pgfscope}%
\begin{pgfscope}%
\pgfsetrectcap%
\pgfsetmiterjoin%
\pgfsetlinewidth{0.000000pt}%
\definecolor{currentstroke}{rgb}{1.000000,1.000000,1.000000}%
\pgfsetstrokecolor{currentstroke}%
\pgfsetdash{}{0pt}%
\pgfpathmoveto{\pgfqpoint{0.875000in}{3.520000in}}%
\pgfpathlineto{\pgfqpoint{6.300000in}{3.520000in}}%
\pgfusepath{}%
\end{pgfscope}%
\begin{pgfscope}%
\definecolor{textcolor}{rgb}{0.150000,0.150000,0.150000}%
\pgfsetstrokecolor{textcolor}%
\pgfsetfillcolor{textcolor}%
\pgftext[x=3.500000in,y=3.920000in,,top]{\color{textcolor}\rmfamily\fontsize{12.000000}{14.400000}\selectfont Histograma para la muestra con propuesta \(\displaystyle U(0,1)\) y \(\displaystyle n=5, r=\)3}%
\end{pgfscope}%
\end{pgfpicture}%
\makeatother%
\endgroup%

        %% Creator: Matplotlib, PGF backend
%%
%% To include the figure in your LaTeX document, write
%%   \input{<filename>.pgf}
%%
%% Make sure the required packages are loaded in your preamble
%%   \usepackage{pgf}
%%
%% Also ensure that all the required font packages are loaded; for instance,
%% the lmodern package is sometimes necessary when using math font.
%%   \usepackage{lmodern}
%%
%% Figures using additional raster images can only be included by \input if
%% they are in the same directory as the main LaTeX file. For loading figures
%% from other directories you can use the `import` package
%%   \usepackage{import}
%%
%% and then include the figures with
%%   \import{<path to file>}{<filename>.pgf}
%%
%% Matplotlib used the following preamble
%%   
%%   \makeatletter\@ifpackageloaded{underscore}{}{\usepackage[strings]{underscore}}\makeatother
%%
\begingroup%
\makeatletter%
\begin{pgfpicture}%
\pgfpathrectangle{\pgfpointorigin}{\pgfqpoint{7.000000in}{4.000000in}}%
\pgfusepath{use as bounding box, clip}%
\begin{pgfscope}%
\pgfsetbuttcap%
\pgfsetmiterjoin%
\definecolor{currentfill}{rgb}{1.000000,1.000000,1.000000}%
\pgfsetfillcolor{currentfill}%
\pgfsetlinewidth{0.000000pt}%
\definecolor{currentstroke}{rgb}{1.000000,1.000000,1.000000}%
\pgfsetstrokecolor{currentstroke}%
\pgfsetdash{}{0pt}%
\pgfpathmoveto{\pgfqpoint{0.000000in}{0.000000in}}%
\pgfpathlineto{\pgfqpoint{7.000000in}{0.000000in}}%
\pgfpathlineto{\pgfqpoint{7.000000in}{4.000000in}}%
\pgfpathlineto{\pgfqpoint{0.000000in}{4.000000in}}%
\pgfpathlineto{\pgfqpoint{0.000000in}{0.000000in}}%
\pgfpathclose%
\pgfusepath{fill}%
\end{pgfscope}%
\begin{pgfscope}%
\pgfsetbuttcap%
\pgfsetmiterjoin%
\definecolor{currentfill}{rgb}{0.917647,0.917647,0.949020}%
\pgfsetfillcolor{currentfill}%
\pgfsetlinewidth{0.000000pt}%
\definecolor{currentstroke}{rgb}{0.000000,0.000000,0.000000}%
\pgfsetstrokecolor{currentstroke}%
\pgfsetstrokeopacity{0.000000}%
\pgfsetdash{}{0pt}%
\pgfpathmoveto{\pgfqpoint{0.875000in}{0.440000in}}%
\pgfpathlineto{\pgfqpoint{6.300000in}{0.440000in}}%
\pgfpathlineto{\pgfqpoint{6.300000in}{3.520000in}}%
\pgfpathlineto{\pgfqpoint{0.875000in}{3.520000in}}%
\pgfpathlineto{\pgfqpoint{0.875000in}{0.440000in}}%
\pgfpathclose%
\pgfusepath{fill}%
\end{pgfscope}%
\begin{pgfscope}%
\pgfpathrectangle{\pgfqpoint{0.875000in}{0.440000in}}{\pgfqpoint{5.425000in}{3.080000in}}%
\pgfusepath{clip}%
\pgfsetroundcap%
\pgfsetroundjoin%
\pgfsetlinewidth{1.003750pt}%
\definecolor{currentstroke}{rgb}{1.000000,1.000000,1.000000}%
\pgfsetstrokecolor{currentstroke}%
\pgfsetdash{}{0pt}%
\pgfpathmoveto{\pgfqpoint{1.433874in}{0.440000in}}%
\pgfpathlineto{\pgfqpoint{1.433874in}{3.520000in}}%
\pgfusepath{stroke}%
\end{pgfscope}%
\begin{pgfscope}%
\definecolor{textcolor}{rgb}{0.150000,0.150000,0.150000}%
\pgfsetstrokecolor{textcolor}%
\pgfsetfillcolor{textcolor}%
\pgftext[x=1.433874in,y=0.342778in,,top]{\color{textcolor}\rmfamily\fontsize{10.000000}{12.000000}\selectfont \(\displaystyle {0.15}\)}%
\end{pgfscope}%
\begin{pgfscope}%
\pgfpathrectangle{\pgfqpoint{0.875000in}{0.440000in}}{\pgfqpoint{5.425000in}{3.080000in}}%
\pgfusepath{clip}%
\pgfsetroundcap%
\pgfsetroundjoin%
\pgfsetlinewidth{1.003750pt}%
\definecolor{currentstroke}{rgb}{1.000000,1.000000,1.000000}%
\pgfsetstrokecolor{currentstroke}%
\pgfsetdash{}{0pt}%
\pgfpathmoveto{\pgfqpoint{2.097030in}{0.440000in}}%
\pgfpathlineto{\pgfqpoint{2.097030in}{3.520000in}}%
\pgfusepath{stroke}%
\end{pgfscope}%
\begin{pgfscope}%
\definecolor{textcolor}{rgb}{0.150000,0.150000,0.150000}%
\pgfsetstrokecolor{textcolor}%
\pgfsetfillcolor{textcolor}%
\pgftext[x=2.097030in,y=0.342778in,,top]{\color{textcolor}\rmfamily\fontsize{10.000000}{12.000000}\selectfont \(\displaystyle {0.20}\)}%
\end{pgfscope}%
\begin{pgfscope}%
\pgfpathrectangle{\pgfqpoint{0.875000in}{0.440000in}}{\pgfqpoint{5.425000in}{3.080000in}}%
\pgfusepath{clip}%
\pgfsetroundcap%
\pgfsetroundjoin%
\pgfsetlinewidth{1.003750pt}%
\definecolor{currentstroke}{rgb}{1.000000,1.000000,1.000000}%
\pgfsetstrokecolor{currentstroke}%
\pgfsetdash{}{0pt}%
\pgfpathmoveto{\pgfqpoint{2.760187in}{0.440000in}}%
\pgfpathlineto{\pgfqpoint{2.760187in}{3.520000in}}%
\pgfusepath{stroke}%
\end{pgfscope}%
\begin{pgfscope}%
\definecolor{textcolor}{rgb}{0.150000,0.150000,0.150000}%
\pgfsetstrokecolor{textcolor}%
\pgfsetfillcolor{textcolor}%
\pgftext[x=2.760187in,y=0.342778in,,top]{\color{textcolor}\rmfamily\fontsize{10.000000}{12.000000}\selectfont \(\displaystyle {0.25}\)}%
\end{pgfscope}%
\begin{pgfscope}%
\pgfpathrectangle{\pgfqpoint{0.875000in}{0.440000in}}{\pgfqpoint{5.425000in}{3.080000in}}%
\pgfusepath{clip}%
\pgfsetroundcap%
\pgfsetroundjoin%
\pgfsetlinewidth{1.003750pt}%
\definecolor{currentstroke}{rgb}{1.000000,1.000000,1.000000}%
\pgfsetstrokecolor{currentstroke}%
\pgfsetdash{}{0pt}%
\pgfpathmoveto{\pgfqpoint{3.423343in}{0.440000in}}%
\pgfpathlineto{\pgfqpoint{3.423343in}{3.520000in}}%
\pgfusepath{stroke}%
\end{pgfscope}%
\begin{pgfscope}%
\definecolor{textcolor}{rgb}{0.150000,0.150000,0.150000}%
\pgfsetstrokecolor{textcolor}%
\pgfsetfillcolor{textcolor}%
\pgftext[x=3.423343in,y=0.342778in,,top]{\color{textcolor}\rmfamily\fontsize{10.000000}{12.000000}\selectfont \(\displaystyle {0.30}\)}%
\end{pgfscope}%
\begin{pgfscope}%
\pgfpathrectangle{\pgfqpoint{0.875000in}{0.440000in}}{\pgfqpoint{5.425000in}{3.080000in}}%
\pgfusepath{clip}%
\pgfsetroundcap%
\pgfsetroundjoin%
\pgfsetlinewidth{1.003750pt}%
\definecolor{currentstroke}{rgb}{1.000000,1.000000,1.000000}%
\pgfsetstrokecolor{currentstroke}%
\pgfsetdash{}{0pt}%
\pgfpathmoveto{\pgfqpoint{4.086499in}{0.440000in}}%
\pgfpathlineto{\pgfqpoint{4.086499in}{3.520000in}}%
\pgfusepath{stroke}%
\end{pgfscope}%
\begin{pgfscope}%
\definecolor{textcolor}{rgb}{0.150000,0.150000,0.150000}%
\pgfsetstrokecolor{textcolor}%
\pgfsetfillcolor{textcolor}%
\pgftext[x=4.086499in,y=0.342778in,,top]{\color{textcolor}\rmfamily\fontsize{10.000000}{12.000000}\selectfont \(\displaystyle {0.35}\)}%
\end{pgfscope}%
\begin{pgfscope}%
\pgfpathrectangle{\pgfqpoint{0.875000in}{0.440000in}}{\pgfqpoint{5.425000in}{3.080000in}}%
\pgfusepath{clip}%
\pgfsetroundcap%
\pgfsetroundjoin%
\pgfsetlinewidth{1.003750pt}%
\definecolor{currentstroke}{rgb}{1.000000,1.000000,1.000000}%
\pgfsetstrokecolor{currentstroke}%
\pgfsetdash{}{0pt}%
\pgfpathmoveto{\pgfqpoint{4.749656in}{0.440000in}}%
\pgfpathlineto{\pgfqpoint{4.749656in}{3.520000in}}%
\pgfusepath{stroke}%
\end{pgfscope}%
\begin{pgfscope}%
\definecolor{textcolor}{rgb}{0.150000,0.150000,0.150000}%
\pgfsetstrokecolor{textcolor}%
\pgfsetfillcolor{textcolor}%
\pgftext[x=4.749656in,y=0.342778in,,top]{\color{textcolor}\rmfamily\fontsize{10.000000}{12.000000}\selectfont \(\displaystyle {0.40}\)}%
\end{pgfscope}%
\begin{pgfscope}%
\pgfpathrectangle{\pgfqpoint{0.875000in}{0.440000in}}{\pgfqpoint{5.425000in}{3.080000in}}%
\pgfusepath{clip}%
\pgfsetroundcap%
\pgfsetroundjoin%
\pgfsetlinewidth{1.003750pt}%
\definecolor{currentstroke}{rgb}{1.000000,1.000000,1.000000}%
\pgfsetstrokecolor{currentstroke}%
\pgfsetdash{}{0pt}%
\pgfpathmoveto{\pgfqpoint{5.412812in}{0.440000in}}%
\pgfpathlineto{\pgfqpoint{5.412812in}{3.520000in}}%
\pgfusepath{stroke}%
\end{pgfscope}%
\begin{pgfscope}%
\definecolor{textcolor}{rgb}{0.150000,0.150000,0.150000}%
\pgfsetstrokecolor{textcolor}%
\pgfsetfillcolor{textcolor}%
\pgftext[x=5.412812in,y=0.342778in,,top]{\color{textcolor}\rmfamily\fontsize{10.000000}{12.000000}\selectfont \(\displaystyle {0.45}\)}%
\end{pgfscope}%
\begin{pgfscope}%
\pgfpathrectangle{\pgfqpoint{0.875000in}{0.440000in}}{\pgfqpoint{5.425000in}{3.080000in}}%
\pgfusepath{clip}%
\pgfsetroundcap%
\pgfsetroundjoin%
\pgfsetlinewidth{1.003750pt}%
\definecolor{currentstroke}{rgb}{1.000000,1.000000,1.000000}%
\pgfsetstrokecolor{currentstroke}%
\pgfsetdash{}{0pt}%
\pgfpathmoveto{\pgfqpoint{6.075968in}{0.440000in}}%
\pgfpathlineto{\pgfqpoint{6.075968in}{3.520000in}}%
\pgfusepath{stroke}%
\end{pgfscope}%
\begin{pgfscope}%
\definecolor{textcolor}{rgb}{0.150000,0.150000,0.150000}%
\pgfsetstrokecolor{textcolor}%
\pgfsetfillcolor{textcolor}%
\pgftext[x=6.075968in,y=0.342778in,,top]{\color{textcolor}\rmfamily\fontsize{10.000000}{12.000000}\selectfont \(\displaystyle {0.50}\)}%
\end{pgfscope}%
\begin{pgfscope}%
\definecolor{textcolor}{rgb}{0.150000,0.150000,0.150000}%
\pgfsetstrokecolor{textcolor}%
\pgfsetfillcolor{textcolor}%
\pgftext[x=3.587500in,y=0.163766in,,top]{\color{textcolor}\rmfamily\fontsize{11.000000}{13.200000}\selectfont Valor}%
\end{pgfscope}%
\begin{pgfscope}%
\pgfpathrectangle{\pgfqpoint{0.875000in}{0.440000in}}{\pgfqpoint{5.425000in}{3.080000in}}%
\pgfusepath{clip}%
\pgfsetroundcap%
\pgfsetroundjoin%
\pgfsetlinewidth{1.003750pt}%
\definecolor{currentstroke}{rgb}{1.000000,1.000000,1.000000}%
\pgfsetstrokecolor{currentstroke}%
\pgfsetdash{}{0pt}%
\pgfpathmoveto{\pgfqpoint{0.875000in}{0.440000in}}%
\pgfpathlineto{\pgfqpoint{6.300000in}{0.440000in}}%
\pgfusepath{stroke}%
\end{pgfscope}%
\begin{pgfscope}%
\definecolor{textcolor}{rgb}{0.150000,0.150000,0.150000}%
\pgfsetstrokecolor{textcolor}%
\pgfsetfillcolor{textcolor}%
\pgftext[x=0.708333in, y=0.391775in, left, base]{\color{textcolor}\rmfamily\fontsize{10.000000}{12.000000}\selectfont \(\displaystyle {0}\)}%
\end{pgfscope}%
\begin{pgfscope}%
\pgfpathrectangle{\pgfqpoint{0.875000in}{0.440000in}}{\pgfqpoint{5.425000in}{3.080000in}}%
\pgfusepath{clip}%
\pgfsetroundcap%
\pgfsetroundjoin%
\pgfsetlinewidth{1.003750pt}%
\definecolor{currentstroke}{rgb}{1.000000,1.000000,1.000000}%
\pgfsetstrokecolor{currentstroke}%
\pgfsetdash{}{0pt}%
\pgfpathmoveto{\pgfqpoint{0.875000in}{0.862679in}}%
\pgfpathlineto{\pgfqpoint{6.300000in}{0.862679in}}%
\pgfusepath{stroke}%
\end{pgfscope}%
\begin{pgfscope}%
\definecolor{textcolor}{rgb}{0.150000,0.150000,0.150000}%
\pgfsetstrokecolor{textcolor}%
\pgfsetfillcolor{textcolor}%
\pgftext[x=0.708333in, y=0.814454in, left, base]{\color{textcolor}\rmfamily\fontsize{10.000000}{12.000000}\selectfont \(\displaystyle {1}\)}%
\end{pgfscope}%
\begin{pgfscope}%
\pgfpathrectangle{\pgfqpoint{0.875000in}{0.440000in}}{\pgfqpoint{5.425000in}{3.080000in}}%
\pgfusepath{clip}%
\pgfsetroundcap%
\pgfsetroundjoin%
\pgfsetlinewidth{1.003750pt}%
\definecolor{currentstroke}{rgb}{1.000000,1.000000,1.000000}%
\pgfsetstrokecolor{currentstroke}%
\pgfsetdash{}{0pt}%
\pgfpathmoveto{\pgfqpoint{0.875000in}{1.285358in}}%
\pgfpathlineto{\pgfqpoint{6.300000in}{1.285358in}}%
\pgfusepath{stroke}%
\end{pgfscope}%
\begin{pgfscope}%
\definecolor{textcolor}{rgb}{0.150000,0.150000,0.150000}%
\pgfsetstrokecolor{textcolor}%
\pgfsetfillcolor{textcolor}%
\pgftext[x=0.708333in, y=1.237132in, left, base]{\color{textcolor}\rmfamily\fontsize{10.000000}{12.000000}\selectfont \(\displaystyle {2}\)}%
\end{pgfscope}%
\begin{pgfscope}%
\pgfpathrectangle{\pgfqpoint{0.875000in}{0.440000in}}{\pgfqpoint{5.425000in}{3.080000in}}%
\pgfusepath{clip}%
\pgfsetroundcap%
\pgfsetroundjoin%
\pgfsetlinewidth{1.003750pt}%
\definecolor{currentstroke}{rgb}{1.000000,1.000000,1.000000}%
\pgfsetstrokecolor{currentstroke}%
\pgfsetdash{}{0pt}%
\pgfpathmoveto{\pgfqpoint{0.875000in}{1.708037in}}%
\pgfpathlineto{\pgfqpoint{6.300000in}{1.708037in}}%
\pgfusepath{stroke}%
\end{pgfscope}%
\begin{pgfscope}%
\definecolor{textcolor}{rgb}{0.150000,0.150000,0.150000}%
\pgfsetstrokecolor{textcolor}%
\pgfsetfillcolor{textcolor}%
\pgftext[x=0.708333in, y=1.659811in, left, base]{\color{textcolor}\rmfamily\fontsize{10.000000}{12.000000}\selectfont \(\displaystyle {3}\)}%
\end{pgfscope}%
\begin{pgfscope}%
\pgfpathrectangle{\pgfqpoint{0.875000in}{0.440000in}}{\pgfqpoint{5.425000in}{3.080000in}}%
\pgfusepath{clip}%
\pgfsetroundcap%
\pgfsetroundjoin%
\pgfsetlinewidth{1.003750pt}%
\definecolor{currentstroke}{rgb}{1.000000,1.000000,1.000000}%
\pgfsetstrokecolor{currentstroke}%
\pgfsetdash{}{0pt}%
\pgfpathmoveto{\pgfqpoint{0.875000in}{2.130715in}}%
\pgfpathlineto{\pgfqpoint{6.300000in}{2.130715in}}%
\pgfusepath{stroke}%
\end{pgfscope}%
\begin{pgfscope}%
\definecolor{textcolor}{rgb}{0.150000,0.150000,0.150000}%
\pgfsetstrokecolor{textcolor}%
\pgfsetfillcolor{textcolor}%
\pgftext[x=0.708333in, y=2.082490in, left, base]{\color{textcolor}\rmfamily\fontsize{10.000000}{12.000000}\selectfont \(\displaystyle {4}\)}%
\end{pgfscope}%
\begin{pgfscope}%
\pgfpathrectangle{\pgfqpoint{0.875000in}{0.440000in}}{\pgfqpoint{5.425000in}{3.080000in}}%
\pgfusepath{clip}%
\pgfsetroundcap%
\pgfsetroundjoin%
\pgfsetlinewidth{1.003750pt}%
\definecolor{currentstroke}{rgb}{1.000000,1.000000,1.000000}%
\pgfsetstrokecolor{currentstroke}%
\pgfsetdash{}{0pt}%
\pgfpathmoveto{\pgfqpoint{0.875000in}{2.553394in}}%
\pgfpathlineto{\pgfqpoint{6.300000in}{2.553394in}}%
\pgfusepath{stroke}%
\end{pgfscope}%
\begin{pgfscope}%
\definecolor{textcolor}{rgb}{0.150000,0.150000,0.150000}%
\pgfsetstrokecolor{textcolor}%
\pgfsetfillcolor{textcolor}%
\pgftext[x=0.708333in, y=2.505169in, left, base]{\color{textcolor}\rmfamily\fontsize{10.000000}{12.000000}\selectfont \(\displaystyle {5}\)}%
\end{pgfscope}%
\begin{pgfscope}%
\pgfpathrectangle{\pgfqpoint{0.875000in}{0.440000in}}{\pgfqpoint{5.425000in}{3.080000in}}%
\pgfusepath{clip}%
\pgfsetroundcap%
\pgfsetroundjoin%
\pgfsetlinewidth{1.003750pt}%
\definecolor{currentstroke}{rgb}{1.000000,1.000000,1.000000}%
\pgfsetstrokecolor{currentstroke}%
\pgfsetdash{}{0pt}%
\pgfpathmoveto{\pgfqpoint{0.875000in}{2.976073in}}%
\pgfpathlineto{\pgfqpoint{6.300000in}{2.976073in}}%
\pgfusepath{stroke}%
\end{pgfscope}%
\begin{pgfscope}%
\definecolor{textcolor}{rgb}{0.150000,0.150000,0.150000}%
\pgfsetstrokecolor{textcolor}%
\pgfsetfillcolor{textcolor}%
\pgftext[x=0.708333in, y=2.927848in, left, base]{\color{textcolor}\rmfamily\fontsize{10.000000}{12.000000}\selectfont \(\displaystyle {6}\)}%
\end{pgfscope}%
\begin{pgfscope}%
\pgfpathrectangle{\pgfqpoint{0.875000in}{0.440000in}}{\pgfqpoint{5.425000in}{3.080000in}}%
\pgfusepath{clip}%
\pgfsetroundcap%
\pgfsetroundjoin%
\pgfsetlinewidth{1.003750pt}%
\definecolor{currentstroke}{rgb}{1.000000,1.000000,1.000000}%
\pgfsetstrokecolor{currentstroke}%
\pgfsetdash{}{0pt}%
\pgfpathmoveto{\pgfqpoint{0.875000in}{3.398752in}}%
\pgfpathlineto{\pgfqpoint{6.300000in}{3.398752in}}%
\pgfusepath{stroke}%
\end{pgfscope}%
\begin{pgfscope}%
\definecolor{textcolor}{rgb}{0.150000,0.150000,0.150000}%
\pgfsetstrokecolor{textcolor}%
\pgfsetfillcolor{textcolor}%
\pgftext[x=0.708333in, y=3.350527in, left, base]{\color{textcolor}\rmfamily\fontsize{10.000000}{12.000000}\selectfont \(\displaystyle {7}\)}%
\end{pgfscope}%
\begin{pgfscope}%
\definecolor{textcolor}{rgb}{0.150000,0.150000,0.150000}%
\pgfsetstrokecolor{textcolor}%
\pgfsetfillcolor{textcolor}%
\pgftext[x=0.652778in,y=1.980000in,,bottom,rotate=90.000000]{\color{textcolor}\rmfamily\fontsize{11.000000}{13.200000}\selectfont Frecuencia}%
\end{pgfscope}%
\begin{pgfscope}%
\pgfpathrectangle{\pgfqpoint{0.875000in}{0.440000in}}{\pgfqpoint{5.425000in}{3.080000in}}%
\pgfusepath{clip}%
\pgfsetbuttcap%
\pgfsetmiterjoin%
\definecolor{currentfill}{rgb}{0.172549,0.627451,0.172549}%
\pgfsetfillcolor{currentfill}%
\pgfsetfillopacity{0.500000}%
\pgfsetlinewidth{0.000000pt}%
\definecolor{currentstroke}{rgb}{0.000000,0.000000,0.000000}%
\pgfsetstrokecolor{currentstroke}%
\pgfsetstrokeopacity{0.500000}%
\pgfsetdash{}{0pt}%
\pgfpathmoveto{\pgfqpoint{1.121591in}{0.440000in}}%
\pgfpathlineto{\pgfqpoint{1.285985in}{0.440000in}}%
\pgfpathlineto{\pgfqpoint{1.285985in}{0.441364in}}%
\pgfpathlineto{\pgfqpoint{1.121591in}{0.441364in}}%
\pgfpathlineto{\pgfqpoint{1.121591in}{0.440000in}}%
\pgfpathclose%
\pgfusepath{fill}%
\end{pgfscope}%
\begin{pgfscope}%
\pgfpathrectangle{\pgfqpoint{0.875000in}{0.440000in}}{\pgfqpoint{5.425000in}{3.080000in}}%
\pgfusepath{clip}%
\pgfsetbuttcap%
\pgfsetmiterjoin%
\definecolor{currentfill}{rgb}{0.172549,0.627451,0.172549}%
\pgfsetfillcolor{currentfill}%
\pgfsetfillopacity{0.500000}%
\pgfsetlinewidth{0.000000pt}%
\definecolor{currentstroke}{rgb}{0.000000,0.000000,0.000000}%
\pgfsetstrokecolor{currentstroke}%
\pgfsetstrokeopacity{0.500000}%
\pgfsetdash{}{0pt}%
\pgfpathmoveto{\pgfqpoint{1.285985in}{0.440000in}}%
\pgfpathlineto{\pgfqpoint{1.450379in}{0.440000in}}%
\pgfpathlineto{\pgfqpoint{1.450379in}{0.444092in}}%
\pgfpathlineto{\pgfqpoint{1.285985in}{0.444092in}}%
\pgfpathlineto{\pgfqpoint{1.285985in}{0.440000in}}%
\pgfpathclose%
\pgfusepath{fill}%
\end{pgfscope}%
\begin{pgfscope}%
\pgfpathrectangle{\pgfqpoint{0.875000in}{0.440000in}}{\pgfqpoint{5.425000in}{3.080000in}}%
\pgfusepath{clip}%
\pgfsetbuttcap%
\pgfsetmiterjoin%
\definecolor{currentfill}{rgb}{0.172549,0.627451,0.172549}%
\pgfsetfillcolor{currentfill}%
\pgfsetfillopacity{0.500000}%
\pgfsetlinewidth{0.000000pt}%
\definecolor{currentstroke}{rgb}{0.000000,0.000000,0.000000}%
\pgfsetstrokecolor{currentstroke}%
\pgfsetstrokeopacity{0.500000}%
\pgfsetdash{}{0pt}%
\pgfpathmoveto{\pgfqpoint{1.450379in}{0.440000in}}%
\pgfpathlineto{\pgfqpoint{1.614773in}{0.440000in}}%
\pgfpathlineto{\pgfqpoint{1.614773in}{0.454322in}}%
\pgfpathlineto{\pgfqpoint{1.450379in}{0.454322in}}%
\pgfpathlineto{\pgfqpoint{1.450379in}{0.440000in}}%
\pgfpathclose%
\pgfusepath{fill}%
\end{pgfscope}%
\begin{pgfscope}%
\pgfpathrectangle{\pgfqpoint{0.875000in}{0.440000in}}{\pgfqpoint{5.425000in}{3.080000in}}%
\pgfusepath{clip}%
\pgfsetbuttcap%
\pgfsetmiterjoin%
\definecolor{currentfill}{rgb}{0.172549,0.627451,0.172549}%
\pgfsetfillcolor{currentfill}%
\pgfsetfillopacity{0.500000}%
\pgfsetlinewidth{0.000000pt}%
\definecolor{currentstroke}{rgb}{0.000000,0.000000,0.000000}%
\pgfsetstrokecolor{currentstroke}%
\pgfsetstrokeopacity{0.500000}%
\pgfsetdash{}{0pt}%
\pgfpathmoveto{\pgfqpoint{1.614773in}{0.440000in}}%
\pgfpathlineto{\pgfqpoint{1.779167in}{0.440000in}}%
\pgfpathlineto{\pgfqpoint{1.779167in}{0.484331in}}%
\pgfpathlineto{\pgfqpoint{1.614773in}{0.484331in}}%
\pgfpathlineto{\pgfqpoint{1.614773in}{0.440000in}}%
\pgfpathclose%
\pgfusepath{fill}%
\end{pgfscope}%
\begin{pgfscope}%
\pgfpathrectangle{\pgfqpoint{0.875000in}{0.440000in}}{\pgfqpoint{5.425000in}{3.080000in}}%
\pgfusepath{clip}%
\pgfsetbuttcap%
\pgfsetmiterjoin%
\definecolor{currentfill}{rgb}{0.172549,0.627451,0.172549}%
\pgfsetfillcolor{currentfill}%
\pgfsetfillopacity{0.500000}%
\pgfsetlinewidth{0.000000pt}%
\definecolor{currentstroke}{rgb}{0.000000,0.000000,0.000000}%
\pgfsetstrokecolor{currentstroke}%
\pgfsetstrokeopacity{0.500000}%
\pgfsetdash{}{0pt}%
\pgfpathmoveto{\pgfqpoint{1.779167in}{0.440000in}}%
\pgfpathlineto{\pgfqpoint{1.943561in}{0.440000in}}%
\pgfpathlineto{\pgfqpoint{1.943561in}{0.524569in}}%
\pgfpathlineto{\pgfqpoint{1.779167in}{0.524569in}}%
\pgfpathlineto{\pgfqpoint{1.779167in}{0.440000in}}%
\pgfpathclose%
\pgfusepath{fill}%
\end{pgfscope}%
\begin{pgfscope}%
\pgfpathrectangle{\pgfqpoint{0.875000in}{0.440000in}}{\pgfqpoint{5.425000in}{3.080000in}}%
\pgfusepath{clip}%
\pgfsetbuttcap%
\pgfsetmiterjoin%
\definecolor{currentfill}{rgb}{0.172549,0.627451,0.172549}%
\pgfsetfillcolor{currentfill}%
\pgfsetfillopacity{0.500000}%
\pgfsetlinewidth{0.000000pt}%
\definecolor{currentstroke}{rgb}{0.000000,0.000000,0.000000}%
\pgfsetstrokecolor{currentstroke}%
\pgfsetstrokeopacity{0.500000}%
\pgfsetdash{}{0pt}%
\pgfpathmoveto{\pgfqpoint{1.943561in}{0.440000in}}%
\pgfpathlineto{\pgfqpoint{2.107955in}{0.440000in}}%
\pgfpathlineto{\pgfqpoint{2.107955in}{0.587315in}}%
\pgfpathlineto{\pgfqpoint{1.943561in}{0.587315in}}%
\pgfpathlineto{\pgfqpoint{1.943561in}{0.440000in}}%
\pgfpathclose%
\pgfusepath{fill}%
\end{pgfscope}%
\begin{pgfscope}%
\pgfpathrectangle{\pgfqpoint{0.875000in}{0.440000in}}{\pgfqpoint{5.425000in}{3.080000in}}%
\pgfusepath{clip}%
\pgfsetbuttcap%
\pgfsetmiterjoin%
\definecolor{currentfill}{rgb}{0.172549,0.627451,0.172549}%
\pgfsetfillcolor{currentfill}%
\pgfsetfillopacity{0.500000}%
\pgfsetlinewidth{0.000000pt}%
\definecolor{currentstroke}{rgb}{0.000000,0.000000,0.000000}%
\pgfsetstrokecolor{currentstroke}%
\pgfsetstrokeopacity{0.500000}%
\pgfsetdash{}{0pt}%
\pgfpathmoveto{\pgfqpoint{2.107955in}{0.440000in}}%
\pgfpathlineto{\pgfqpoint{2.272348in}{0.440000in}}%
\pgfpathlineto{\pgfqpoint{2.272348in}{0.697118in}}%
\pgfpathlineto{\pgfqpoint{2.107955in}{0.697118in}}%
\pgfpathlineto{\pgfqpoint{2.107955in}{0.440000in}}%
\pgfpathclose%
\pgfusepath{fill}%
\end{pgfscope}%
\begin{pgfscope}%
\pgfpathrectangle{\pgfqpoint{0.875000in}{0.440000in}}{\pgfqpoint{5.425000in}{3.080000in}}%
\pgfusepath{clip}%
\pgfsetbuttcap%
\pgfsetmiterjoin%
\definecolor{currentfill}{rgb}{0.172549,0.627451,0.172549}%
\pgfsetfillcolor{currentfill}%
\pgfsetfillopacity{0.500000}%
\pgfsetlinewidth{0.000000pt}%
\definecolor{currentstroke}{rgb}{0.000000,0.000000,0.000000}%
\pgfsetstrokecolor{currentstroke}%
\pgfsetstrokeopacity{0.500000}%
\pgfsetdash{}{0pt}%
\pgfpathmoveto{\pgfqpoint{2.272348in}{0.440000in}}%
\pgfpathlineto{\pgfqpoint{2.436742in}{0.440000in}}%
\pgfpathlineto{\pgfqpoint{2.436742in}{0.858755in}}%
\pgfpathlineto{\pgfqpoint{2.272348in}{0.858755in}}%
\pgfpathlineto{\pgfqpoint{2.272348in}{0.440000in}}%
\pgfpathclose%
\pgfusepath{fill}%
\end{pgfscope}%
\begin{pgfscope}%
\pgfpathrectangle{\pgfqpoint{0.875000in}{0.440000in}}{\pgfqpoint{5.425000in}{3.080000in}}%
\pgfusepath{clip}%
\pgfsetbuttcap%
\pgfsetmiterjoin%
\definecolor{currentfill}{rgb}{0.172549,0.627451,0.172549}%
\pgfsetfillcolor{currentfill}%
\pgfsetfillopacity{0.500000}%
\pgfsetlinewidth{0.000000pt}%
\definecolor{currentstroke}{rgb}{0.000000,0.000000,0.000000}%
\pgfsetstrokecolor{currentstroke}%
\pgfsetstrokeopacity{0.500000}%
\pgfsetdash{}{0pt}%
\pgfpathmoveto{\pgfqpoint{2.436742in}{0.440000in}}%
\pgfpathlineto{\pgfqpoint{2.601136in}{0.440000in}}%
\pgfpathlineto{\pgfqpoint{2.601136in}{1.019028in}}%
\pgfpathlineto{\pgfqpoint{2.436742in}{1.019028in}}%
\pgfpathlineto{\pgfqpoint{2.436742in}{0.440000in}}%
\pgfpathclose%
\pgfusepath{fill}%
\end{pgfscope}%
\begin{pgfscope}%
\pgfpathrectangle{\pgfqpoint{0.875000in}{0.440000in}}{\pgfqpoint{5.425000in}{3.080000in}}%
\pgfusepath{clip}%
\pgfsetbuttcap%
\pgfsetmiterjoin%
\definecolor{currentfill}{rgb}{0.172549,0.627451,0.172549}%
\pgfsetfillcolor{currentfill}%
\pgfsetfillopacity{0.500000}%
\pgfsetlinewidth{0.000000pt}%
\definecolor{currentstroke}{rgb}{0.000000,0.000000,0.000000}%
\pgfsetstrokecolor{currentstroke}%
\pgfsetstrokeopacity{0.500000}%
\pgfsetdash{}{0pt}%
\pgfpathmoveto{\pgfqpoint{2.601136in}{0.440000in}}%
\pgfpathlineto{\pgfqpoint{2.765530in}{0.440000in}}%
\pgfpathlineto{\pgfqpoint{2.765530in}{1.250230in}}%
\pgfpathlineto{\pgfqpoint{2.601136in}{1.250230in}}%
\pgfpathlineto{\pgfqpoint{2.601136in}{0.440000in}}%
\pgfpathclose%
\pgfusepath{fill}%
\end{pgfscope}%
\begin{pgfscope}%
\pgfpathrectangle{\pgfqpoint{0.875000in}{0.440000in}}{\pgfqpoint{5.425000in}{3.080000in}}%
\pgfusepath{clip}%
\pgfsetbuttcap%
\pgfsetmiterjoin%
\definecolor{currentfill}{rgb}{0.172549,0.627451,0.172549}%
\pgfsetfillcolor{currentfill}%
\pgfsetfillopacity{0.500000}%
\pgfsetlinewidth{0.000000pt}%
\definecolor{currentstroke}{rgb}{0.000000,0.000000,0.000000}%
\pgfsetstrokecolor{currentstroke}%
\pgfsetstrokeopacity{0.500000}%
\pgfsetdash{}{0pt}%
\pgfpathmoveto{\pgfqpoint{2.765530in}{0.440000in}}%
\pgfpathlineto{\pgfqpoint{2.929924in}{0.440000in}}%
\pgfpathlineto{\pgfqpoint{2.929924in}{1.495072in}}%
\pgfpathlineto{\pgfqpoint{2.765530in}{1.495072in}}%
\pgfpathlineto{\pgfqpoint{2.765530in}{0.440000in}}%
\pgfpathclose%
\pgfusepath{fill}%
\end{pgfscope}%
\begin{pgfscope}%
\pgfpathrectangle{\pgfqpoint{0.875000in}{0.440000in}}{\pgfqpoint{5.425000in}{3.080000in}}%
\pgfusepath{clip}%
\pgfsetbuttcap%
\pgfsetmiterjoin%
\definecolor{currentfill}{rgb}{0.172549,0.627451,0.172549}%
\pgfsetfillcolor{currentfill}%
\pgfsetfillopacity{0.500000}%
\pgfsetlinewidth{0.000000pt}%
\definecolor{currentstroke}{rgb}{0.000000,0.000000,0.000000}%
\pgfsetstrokecolor{currentstroke}%
\pgfsetstrokeopacity{0.500000}%
\pgfsetdash{}{0pt}%
\pgfpathmoveto{\pgfqpoint{2.929924in}{0.440000in}}%
\pgfpathlineto{\pgfqpoint{3.094318in}{0.440000in}}%
\pgfpathlineto{\pgfqpoint{3.094318in}{1.916556in}}%
\pgfpathlineto{\pgfqpoint{2.929924in}{1.916556in}}%
\pgfpathlineto{\pgfqpoint{2.929924in}{0.440000in}}%
\pgfpathclose%
\pgfusepath{fill}%
\end{pgfscope}%
\begin{pgfscope}%
\pgfpathrectangle{\pgfqpoint{0.875000in}{0.440000in}}{\pgfqpoint{5.425000in}{3.080000in}}%
\pgfusepath{clip}%
\pgfsetbuttcap%
\pgfsetmiterjoin%
\definecolor{currentfill}{rgb}{0.172549,0.627451,0.172549}%
\pgfsetfillcolor{currentfill}%
\pgfsetfillopacity{0.500000}%
\pgfsetlinewidth{0.000000pt}%
\definecolor{currentstroke}{rgb}{0.000000,0.000000,0.000000}%
\pgfsetstrokecolor{currentstroke}%
\pgfsetstrokeopacity{0.500000}%
\pgfsetdash{}{0pt}%
\pgfpathmoveto{\pgfqpoint{3.094318in}{0.440000in}}%
\pgfpathlineto{\pgfqpoint{3.258712in}{0.440000in}}%
\pgfpathlineto{\pgfqpoint{3.258712in}{2.224143in}}%
\pgfpathlineto{\pgfqpoint{3.094318in}{2.224143in}}%
\pgfpathlineto{\pgfqpoint{3.094318in}{0.440000in}}%
\pgfpathclose%
\pgfusepath{fill}%
\end{pgfscope}%
\begin{pgfscope}%
\pgfpathrectangle{\pgfqpoint{0.875000in}{0.440000in}}{\pgfqpoint{5.425000in}{3.080000in}}%
\pgfusepath{clip}%
\pgfsetbuttcap%
\pgfsetmiterjoin%
\definecolor{currentfill}{rgb}{0.172549,0.627451,0.172549}%
\pgfsetfillcolor{currentfill}%
\pgfsetfillopacity{0.500000}%
\pgfsetlinewidth{0.000000pt}%
\definecolor{currentstroke}{rgb}{0.000000,0.000000,0.000000}%
\pgfsetstrokecolor{currentstroke}%
\pgfsetstrokeopacity{0.500000}%
\pgfsetdash{}{0pt}%
\pgfpathmoveto{\pgfqpoint{3.258712in}{0.440000in}}%
\pgfpathlineto{\pgfqpoint{3.423106in}{0.440000in}}%
\pgfpathlineto{\pgfqpoint{3.423106in}{2.453299in}}%
\pgfpathlineto{\pgfqpoint{3.258712in}{2.453299in}}%
\pgfpathlineto{\pgfqpoint{3.258712in}{0.440000in}}%
\pgfpathclose%
\pgfusepath{fill}%
\end{pgfscope}%
\begin{pgfscope}%
\pgfpathrectangle{\pgfqpoint{0.875000in}{0.440000in}}{\pgfqpoint{5.425000in}{3.080000in}}%
\pgfusepath{clip}%
\pgfsetbuttcap%
\pgfsetmiterjoin%
\definecolor{currentfill}{rgb}{0.172549,0.627451,0.172549}%
\pgfsetfillcolor{currentfill}%
\pgfsetfillopacity{0.500000}%
\pgfsetlinewidth{0.000000pt}%
\definecolor{currentstroke}{rgb}{0.000000,0.000000,0.000000}%
\pgfsetstrokecolor{currentstroke}%
\pgfsetstrokeopacity{0.500000}%
\pgfsetdash{}{0pt}%
\pgfpathmoveto{\pgfqpoint{3.423106in}{0.440000in}}%
\pgfpathlineto{\pgfqpoint{3.587500in}{0.440000in}}%
\pgfpathlineto{\pgfqpoint{3.587500in}{2.742472in}}%
\pgfpathlineto{\pgfqpoint{3.423106in}{2.742472in}}%
\pgfpathlineto{\pgfqpoint{3.423106in}{0.440000in}}%
\pgfpathclose%
\pgfusepath{fill}%
\end{pgfscope}%
\begin{pgfscope}%
\pgfpathrectangle{\pgfqpoint{0.875000in}{0.440000in}}{\pgfqpoint{5.425000in}{3.080000in}}%
\pgfusepath{clip}%
\pgfsetbuttcap%
\pgfsetmiterjoin%
\definecolor{currentfill}{rgb}{0.172549,0.627451,0.172549}%
\pgfsetfillcolor{currentfill}%
\pgfsetfillopacity{0.500000}%
\pgfsetlinewidth{0.000000pt}%
\definecolor{currentstroke}{rgb}{0.000000,0.000000,0.000000}%
\pgfsetstrokecolor{currentstroke}%
\pgfsetstrokeopacity{0.500000}%
\pgfsetdash{}{0pt}%
\pgfpathmoveto{\pgfqpoint{3.587500in}{0.440000in}}%
\pgfpathlineto{\pgfqpoint{3.751894in}{0.440000in}}%
\pgfpathlineto{\pgfqpoint{3.751894in}{2.824996in}}%
\pgfpathlineto{\pgfqpoint{3.587500in}{2.824996in}}%
\pgfpathlineto{\pgfqpoint{3.587500in}{0.440000in}}%
\pgfpathclose%
\pgfusepath{fill}%
\end{pgfscope}%
\begin{pgfscope}%
\pgfpathrectangle{\pgfqpoint{0.875000in}{0.440000in}}{\pgfqpoint{5.425000in}{3.080000in}}%
\pgfusepath{clip}%
\pgfsetbuttcap%
\pgfsetmiterjoin%
\definecolor{currentfill}{rgb}{0.172549,0.627451,0.172549}%
\pgfsetfillcolor{currentfill}%
\pgfsetfillopacity{0.500000}%
\pgfsetlinewidth{0.000000pt}%
\definecolor{currentstroke}{rgb}{0.000000,0.000000,0.000000}%
\pgfsetstrokecolor{currentstroke}%
\pgfsetstrokeopacity{0.500000}%
\pgfsetdash{}{0pt}%
\pgfpathmoveto{\pgfqpoint{3.751894in}{0.440000in}}%
\pgfpathlineto{\pgfqpoint{3.916288in}{0.440000in}}%
\pgfpathlineto{\pgfqpoint{3.916288in}{3.373333in}}%
\pgfpathlineto{\pgfqpoint{3.751894in}{3.373333in}}%
\pgfpathlineto{\pgfqpoint{3.751894in}{0.440000in}}%
\pgfpathclose%
\pgfusepath{fill}%
\end{pgfscope}%
\begin{pgfscope}%
\pgfpathrectangle{\pgfqpoint{0.875000in}{0.440000in}}{\pgfqpoint{5.425000in}{3.080000in}}%
\pgfusepath{clip}%
\pgfsetbuttcap%
\pgfsetmiterjoin%
\definecolor{currentfill}{rgb}{0.172549,0.627451,0.172549}%
\pgfsetfillcolor{currentfill}%
\pgfsetfillopacity{0.500000}%
\pgfsetlinewidth{0.000000pt}%
\definecolor{currentstroke}{rgb}{0.000000,0.000000,0.000000}%
\pgfsetstrokecolor{currentstroke}%
\pgfsetstrokeopacity{0.500000}%
\pgfsetdash{}{0pt}%
\pgfpathmoveto{\pgfqpoint{3.916288in}{0.440000in}}%
\pgfpathlineto{\pgfqpoint{4.080682in}{0.440000in}}%
\pgfpathlineto{\pgfqpoint{4.080682in}{3.059608in}}%
\pgfpathlineto{\pgfqpoint{3.916288in}{3.059608in}}%
\pgfpathlineto{\pgfqpoint{3.916288in}{0.440000in}}%
\pgfpathclose%
\pgfusepath{fill}%
\end{pgfscope}%
\begin{pgfscope}%
\pgfpathrectangle{\pgfqpoint{0.875000in}{0.440000in}}{\pgfqpoint{5.425000in}{3.080000in}}%
\pgfusepath{clip}%
\pgfsetbuttcap%
\pgfsetmiterjoin%
\definecolor{currentfill}{rgb}{0.172549,0.627451,0.172549}%
\pgfsetfillcolor{currentfill}%
\pgfsetfillopacity{0.500000}%
\pgfsetlinewidth{0.000000pt}%
\definecolor{currentstroke}{rgb}{0.000000,0.000000,0.000000}%
\pgfsetstrokecolor{currentstroke}%
\pgfsetstrokeopacity{0.500000}%
\pgfsetdash{}{0pt}%
\pgfpathmoveto{\pgfqpoint{4.080682in}{0.440000in}}%
\pgfpathlineto{\pgfqpoint{4.245076in}{0.440000in}}%
\pgfpathlineto{\pgfqpoint{4.245076in}{3.001637in}}%
\pgfpathlineto{\pgfqpoint{4.080682in}{3.001637in}}%
\pgfpathlineto{\pgfqpoint{4.080682in}{0.440000in}}%
\pgfpathclose%
\pgfusepath{fill}%
\end{pgfscope}%
\begin{pgfscope}%
\pgfpathrectangle{\pgfqpoint{0.875000in}{0.440000in}}{\pgfqpoint{5.425000in}{3.080000in}}%
\pgfusepath{clip}%
\pgfsetbuttcap%
\pgfsetmiterjoin%
\definecolor{currentfill}{rgb}{0.172549,0.627451,0.172549}%
\pgfsetfillcolor{currentfill}%
\pgfsetfillopacity{0.500000}%
\pgfsetlinewidth{0.000000pt}%
\definecolor{currentstroke}{rgb}{0.000000,0.000000,0.000000}%
\pgfsetstrokecolor{currentstroke}%
\pgfsetstrokeopacity{0.500000}%
\pgfsetdash{}{0pt}%
\pgfpathmoveto{\pgfqpoint{4.245076in}{0.440000in}}%
\pgfpathlineto{\pgfqpoint{4.409470in}{0.440000in}}%
\pgfpathlineto{\pgfqpoint{4.409470in}{2.767707in}}%
\pgfpathlineto{\pgfqpoint{4.245076in}{2.767707in}}%
\pgfpathlineto{\pgfqpoint{4.245076in}{0.440000in}}%
\pgfpathclose%
\pgfusepath{fill}%
\end{pgfscope}%
\begin{pgfscope}%
\pgfpathrectangle{\pgfqpoint{0.875000in}{0.440000in}}{\pgfqpoint{5.425000in}{3.080000in}}%
\pgfusepath{clip}%
\pgfsetbuttcap%
\pgfsetmiterjoin%
\definecolor{currentfill}{rgb}{0.172549,0.627451,0.172549}%
\pgfsetfillcolor{currentfill}%
\pgfsetfillopacity{0.500000}%
\pgfsetlinewidth{0.000000pt}%
\definecolor{currentstroke}{rgb}{0.000000,0.000000,0.000000}%
\pgfsetstrokecolor{currentstroke}%
\pgfsetstrokeopacity{0.500000}%
\pgfsetdash{}{0pt}%
\pgfpathmoveto{\pgfqpoint{4.409470in}{0.440000in}}%
\pgfpathlineto{\pgfqpoint{4.573864in}{0.440000in}}%
\pgfpathlineto{\pgfqpoint{4.573864in}{2.912293in}}%
\pgfpathlineto{\pgfqpoint{4.409470in}{2.912293in}}%
\pgfpathlineto{\pgfqpoint{4.409470in}{0.440000in}}%
\pgfpathclose%
\pgfusepath{fill}%
\end{pgfscope}%
\begin{pgfscope}%
\pgfpathrectangle{\pgfqpoint{0.875000in}{0.440000in}}{\pgfqpoint{5.425000in}{3.080000in}}%
\pgfusepath{clip}%
\pgfsetbuttcap%
\pgfsetmiterjoin%
\definecolor{currentfill}{rgb}{0.172549,0.627451,0.172549}%
\pgfsetfillcolor{currentfill}%
\pgfsetfillopacity{0.500000}%
\pgfsetlinewidth{0.000000pt}%
\definecolor{currentstroke}{rgb}{0.000000,0.000000,0.000000}%
\pgfsetstrokecolor{currentstroke}%
\pgfsetstrokeopacity{0.500000}%
\pgfsetdash{}{0pt}%
\pgfpathmoveto{\pgfqpoint{4.573864in}{0.440000in}}%
\pgfpathlineto{\pgfqpoint{4.738258in}{0.440000in}}%
\pgfpathlineto{\pgfqpoint{4.738258in}{2.413061in}}%
\pgfpathlineto{\pgfqpoint{4.573864in}{2.413061in}}%
\pgfpathlineto{\pgfqpoint{4.573864in}{0.440000in}}%
\pgfpathclose%
\pgfusepath{fill}%
\end{pgfscope}%
\begin{pgfscope}%
\pgfpathrectangle{\pgfqpoint{0.875000in}{0.440000in}}{\pgfqpoint{5.425000in}{3.080000in}}%
\pgfusepath{clip}%
\pgfsetbuttcap%
\pgfsetmiterjoin%
\definecolor{currentfill}{rgb}{0.172549,0.627451,0.172549}%
\pgfsetfillcolor{currentfill}%
\pgfsetfillopacity{0.500000}%
\pgfsetlinewidth{0.000000pt}%
\definecolor{currentstroke}{rgb}{0.000000,0.000000,0.000000}%
\pgfsetstrokecolor{currentstroke}%
\pgfsetstrokeopacity{0.500000}%
\pgfsetdash{}{0pt}%
\pgfpathmoveto{\pgfqpoint{4.738258in}{0.440000in}}%
\pgfpathlineto{\pgfqpoint{4.902652in}{0.440000in}}%
\pgfpathlineto{\pgfqpoint{4.902652in}{2.111611in}}%
\pgfpathlineto{\pgfqpoint{4.738258in}{2.111611in}}%
\pgfpathlineto{\pgfqpoint{4.738258in}{0.440000in}}%
\pgfpathclose%
\pgfusepath{fill}%
\end{pgfscope}%
\begin{pgfscope}%
\pgfpathrectangle{\pgfqpoint{0.875000in}{0.440000in}}{\pgfqpoint{5.425000in}{3.080000in}}%
\pgfusepath{clip}%
\pgfsetbuttcap%
\pgfsetmiterjoin%
\definecolor{currentfill}{rgb}{0.172549,0.627451,0.172549}%
\pgfsetfillcolor{currentfill}%
\pgfsetfillopacity{0.500000}%
\pgfsetlinewidth{0.000000pt}%
\definecolor{currentstroke}{rgb}{0.000000,0.000000,0.000000}%
\pgfsetstrokecolor{currentstroke}%
\pgfsetstrokeopacity{0.500000}%
\pgfsetdash{}{0pt}%
\pgfpathmoveto{\pgfqpoint{4.902652in}{0.440000in}}%
\pgfpathlineto{\pgfqpoint{5.067045in}{0.440000in}}%
\pgfpathlineto{\pgfqpoint{5.067045in}{1.747417in}}%
\pgfpathlineto{\pgfqpoint{4.902652in}{1.747417in}}%
\pgfpathlineto{\pgfqpoint{4.902652in}{0.440000in}}%
\pgfpathclose%
\pgfusepath{fill}%
\end{pgfscope}%
\begin{pgfscope}%
\pgfpathrectangle{\pgfqpoint{0.875000in}{0.440000in}}{\pgfqpoint{5.425000in}{3.080000in}}%
\pgfusepath{clip}%
\pgfsetbuttcap%
\pgfsetmiterjoin%
\definecolor{currentfill}{rgb}{0.172549,0.627451,0.172549}%
\pgfsetfillcolor{currentfill}%
\pgfsetfillopacity{0.500000}%
\pgfsetlinewidth{0.000000pt}%
\definecolor{currentstroke}{rgb}{0.000000,0.000000,0.000000}%
\pgfsetstrokecolor{currentstroke}%
\pgfsetstrokeopacity{0.500000}%
\pgfsetdash{}{0pt}%
\pgfpathmoveto{\pgfqpoint{5.067045in}{0.440000in}}%
\pgfpathlineto{\pgfqpoint{5.231439in}{0.440000in}}%
\pgfpathlineto{\pgfqpoint{5.231439in}{1.502575in}}%
\pgfpathlineto{\pgfqpoint{5.067045in}{1.502575in}}%
\pgfpathlineto{\pgfqpoint{5.067045in}{0.440000in}}%
\pgfpathclose%
\pgfusepath{fill}%
\end{pgfscope}%
\begin{pgfscope}%
\pgfpathrectangle{\pgfqpoint{0.875000in}{0.440000in}}{\pgfqpoint{5.425000in}{3.080000in}}%
\pgfusepath{clip}%
\pgfsetbuttcap%
\pgfsetmiterjoin%
\definecolor{currentfill}{rgb}{0.172549,0.627451,0.172549}%
\pgfsetfillcolor{currentfill}%
\pgfsetfillopacity{0.500000}%
\pgfsetlinewidth{0.000000pt}%
\definecolor{currentstroke}{rgb}{0.000000,0.000000,0.000000}%
\pgfsetstrokecolor{currentstroke}%
\pgfsetstrokeopacity{0.500000}%
\pgfsetdash{}{0pt}%
\pgfpathmoveto{\pgfqpoint{5.231439in}{0.440000in}}%
\pgfpathlineto{\pgfqpoint{5.395833in}{0.440000in}}%
\pgfpathlineto{\pgfqpoint{5.395833in}{1.163615in}}%
\pgfpathlineto{\pgfqpoint{5.231439in}{1.163615in}}%
\pgfpathlineto{\pgfqpoint{5.231439in}{0.440000in}}%
\pgfpathclose%
\pgfusepath{fill}%
\end{pgfscope}%
\begin{pgfscope}%
\pgfpathrectangle{\pgfqpoint{0.875000in}{0.440000in}}{\pgfqpoint{5.425000in}{3.080000in}}%
\pgfusepath{clip}%
\pgfsetbuttcap%
\pgfsetmiterjoin%
\definecolor{currentfill}{rgb}{0.172549,0.627451,0.172549}%
\pgfsetfillcolor{currentfill}%
\pgfsetfillopacity{0.500000}%
\pgfsetlinewidth{0.000000pt}%
\definecolor{currentstroke}{rgb}{0.000000,0.000000,0.000000}%
\pgfsetstrokecolor{currentstroke}%
\pgfsetstrokeopacity{0.500000}%
\pgfsetdash{}{0pt}%
\pgfpathmoveto{\pgfqpoint{5.395833in}{0.440000in}}%
\pgfpathlineto{\pgfqpoint{5.560227in}{0.440000in}}%
\pgfpathlineto{\pgfqpoint{5.560227in}{0.996522in}}%
\pgfpathlineto{\pgfqpoint{5.395833in}{0.996522in}}%
\pgfpathlineto{\pgfqpoint{5.395833in}{0.440000in}}%
\pgfpathclose%
\pgfusepath{fill}%
\end{pgfscope}%
\begin{pgfscope}%
\pgfpathrectangle{\pgfqpoint{0.875000in}{0.440000in}}{\pgfqpoint{5.425000in}{3.080000in}}%
\pgfusepath{clip}%
\pgfsetbuttcap%
\pgfsetmiterjoin%
\definecolor{currentfill}{rgb}{0.172549,0.627451,0.172549}%
\pgfsetfillcolor{currentfill}%
\pgfsetfillopacity{0.500000}%
\pgfsetlinewidth{0.000000pt}%
\definecolor{currentstroke}{rgb}{0.000000,0.000000,0.000000}%
\pgfsetstrokecolor{currentstroke}%
\pgfsetstrokeopacity{0.500000}%
\pgfsetdash{}{0pt}%
\pgfpathmoveto{\pgfqpoint{5.560227in}{0.440000in}}%
\pgfpathlineto{\pgfqpoint{5.724621in}{0.440000in}}%
\pgfpathlineto{\pgfqpoint{5.724621in}{0.748269in}}%
\pgfpathlineto{\pgfqpoint{5.560227in}{0.748269in}}%
\pgfpathlineto{\pgfqpoint{5.560227in}{0.440000in}}%
\pgfpathclose%
\pgfusepath{fill}%
\end{pgfscope}%
\begin{pgfscope}%
\pgfpathrectangle{\pgfqpoint{0.875000in}{0.440000in}}{\pgfqpoint{5.425000in}{3.080000in}}%
\pgfusepath{clip}%
\pgfsetbuttcap%
\pgfsetmiterjoin%
\definecolor{currentfill}{rgb}{0.172549,0.627451,0.172549}%
\pgfsetfillcolor{currentfill}%
\pgfsetfillopacity{0.500000}%
\pgfsetlinewidth{0.000000pt}%
\definecolor{currentstroke}{rgb}{0.000000,0.000000,0.000000}%
\pgfsetstrokecolor{currentstroke}%
\pgfsetstrokeopacity{0.500000}%
\pgfsetdash{}{0pt}%
\pgfpathmoveto{\pgfqpoint{5.724621in}{0.440000in}}%
\pgfpathlineto{\pgfqpoint{5.889015in}{0.440000in}}%
\pgfpathlineto{\pgfqpoint{5.889015in}{0.600273in}}%
\pgfpathlineto{\pgfqpoint{5.724621in}{0.600273in}}%
\pgfpathlineto{\pgfqpoint{5.724621in}{0.440000in}}%
\pgfpathclose%
\pgfusepath{fill}%
\end{pgfscope}%
\begin{pgfscope}%
\pgfpathrectangle{\pgfqpoint{0.875000in}{0.440000in}}{\pgfqpoint{5.425000in}{3.080000in}}%
\pgfusepath{clip}%
\pgfsetbuttcap%
\pgfsetmiterjoin%
\definecolor{currentfill}{rgb}{0.172549,0.627451,0.172549}%
\pgfsetfillcolor{currentfill}%
\pgfsetfillopacity{0.500000}%
\pgfsetlinewidth{0.000000pt}%
\definecolor{currentstroke}{rgb}{0.000000,0.000000,0.000000}%
\pgfsetstrokecolor{currentstroke}%
\pgfsetstrokeopacity{0.500000}%
\pgfsetdash{}{0pt}%
\pgfpathmoveto{\pgfqpoint{5.889015in}{0.440000in}}%
\pgfpathlineto{\pgfqpoint{6.053409in}{0.440000in}}%
\pgfpathlineto{\pgfqpoint{6.053409in}{0.485695in}}%
\pgfpathlineto{\pgfqpoint{5.889015in}{0.485695in}}%
\pgfpathlineto{\pgfqpoint{5.889015in}{0.440000in}}%
\pgfpathclose%
\pgfusepath{fill}%
\end{pgfscope}%
\begin{pgfscope}%
\pgfsetrectcap%
\pgfsetmiterjoin%
\pgfsetlinewidth{0.000000pt}%
\definecolor{currentstroke}{rgb}{1.000000,1.000000,1.000000}%
\pgfsetstrokecolor{currentstroke}%
\pgfsetdash{}{0pt}%
\pgfpathmoveto{\pgfqpoint{0.875000in}{0.440000in}}%
\pgfpathlineto{\pgfqpoint{0.875000in}{3.520000in}}%
\pgfusepath{}%
\end{pgfscope}%
\begin{pgfscope}%
\pgfsetrectcap%
\pgfsetmiterjoin%
\pgfsetlinewidth{0.000000pt}%
\definecolor{currentstroke}{rgb}{1.000000,1.000000,1.000000}%
\pgfsetstrokecolor{currentstroke}%
\pgfsetdash{}{0pt}%
\pgfpathmoveto{\pgfqpoint{6.300000in}{0.440000in}}%
\pgfpathlineto{\pgfqpoint{6.300000in}{3.520000in}}%
\pgfusepath{}%
\end{pgfscope}%
\begin{pgfscope}%
\pgfsetrectcap%
\pgfsetmiterjoin%
\pgfsetlinewidth{0.000000pt}%
\definecolor{currentstroke}{rgb}{1.000000,1.000000,1.000000}%
\pgfsetstrokecolor{currentstroke}%
\pgfsetdash{}{0pt}%
\pgfpathmoveto{\pgfqpoint{0.875000in}{0.440000in}}%
\pgfpathlineto{\pgfqpoint{6.300000in}{0.440000in}}%
\pgfusepath{}%
\end{pgfscope}%
\begin{pgfscope}%
\pgfsetrectcap%
\pgfsetmiterjoin%
\pgfsetlinewidth{0.000000pt}%
\definecolor{currentstroke}{rgb}{1.000000,1.000000,1.000000}%
\pgfsetstrokecolor{currentstroke}%
\pgfsetdash{}{0pt}%
\pgfpathmoveto{\pgfqpoint{0.875000in}{3.520000in}}%
\pgfpathlineto{\pgfqpoint{6.300000in}{3.520000in}}%
\pgfusepath{}%
\end{pgfscope}%
\begin{pgfscope}%
\definecolor{textcolor}{rgb}{0.150000,0.150000,0.150000}%
\pgfsetstrokecolor{textcolor}%
\pgfsetfillcolor{textcolor}%
\pgftext[x=3.500000in,y=3.920000in,,top]{\color{textcolor}\rmfamily\fontsize{12.000000}{14.400000}\selectfont Histograma para la muestra con propuesta \(\displaystyle U(0,1)\) y \(\displaystyle n=40, r=\)15}%
\end{pgfscope}%
\end{pgfpicture}%
\makeatother%
\endgroup%

    \end{center}

    Ambos histogramas parecen acercarse al resultado deseado. Las gráficas para el logaritmo 
    de la densidad evaluadas en $X_t$ son las siguientes

    \begin{center}
        %% Creator: Matplotlib, PGF backend
%%
%% To include the figure in your LaTeX document, write
%%   \input{<filename>.pgf}
%%
%% Make sure the required packages are loaded in your preamble
%%   \usepackage{pgf}
%%
%% Also ensure that all the required font packages are loaded; for instance,
%% the lmodern package is sometimes necessary when using math font.
%%   \usepackage{lmodern}
%%
%% Figures using additional raster images can only be included by \input if
%% they are in the same directory as the main LaTeX file. For loading figures
%% from other directories you can use the `import` package
%%   \usepackage{import}
%%
%% and then include the figures with
%%   \import{<path to file>}{<filename>.pgf}
%%
%% Matplotlib used the following preamble
%%   
%%   \makeatletter\@ifpackageloaded{underscore}{}{\usepackage[strings]{underscore}}\makeatother
%%
\begingroup%
\makeatletter%
\begin{pgfpicture}%
\pgfpathrectangle{\pgfpointorigin}{\pgfqpoint{7.000000in}{4.000000in}}%
\pgfusepath{use as bounding box, clip}%
\begin{pgfscope}%
\pgfsetbuttcap%
\pgfsetmiterjoin%
\definecolor{currentfill}{rgb}{1.000000,1.000000,1.000000}%
\pgfsetfillcolor{currentfill}%
\pgfsetlinewidth{0.000000pt}%
\definecolor{currentstroke}{rgb}{1.000000,1.000000,1.000000}%
\pgfsetstrokecolor{currentstroke}%
\pgfsetdash{}{0pt}%
\pgfpathmoveto{\pgfqpoint{0.000000in}{0.000000in}}%
\pgfpathlineto{\pgfqpoint{7.000000in}{0.000000in}}%
\pgfpathlineto{\pgfqpoint{7.000000in}{4.000000in}}%
\pgfpathlineto{\pgfqpoint{0.000000in}{4.000000in}}%
\pgfpathlineto{\pgfqpoint{0.000000in}{0.000000in}}%
\pgfpathclose%
\pgfusepath{fill}%
\end{pgfscope}%
\begin{pgfscope}%
\pgfsetbuttcap%
\pgfsetmiterjoin%
\definecolor{currentfill}{rgb}{0.917647,0.917647,0.949020}%
\pgfsetfillcolor{currentfill}%
\pgfsetlinewidth{0.000000pt}%
\definecolor{currentstroke}{rgb}{0.000000,0.000000,0.000000}%
\pgfsetstrokecolor{currentstroke}%
\pgfsetstrokeopacity{0.000000}%
\pgfsetdash{}{0pt}%
\pgfpathmoveto{\pgfqpoint{0.875000in}{0.440000in}}%
\pgfpathlineto{\pgfqpoint{6.300000in}{0.440000in}}%
\pgfpathlineto{\pgfqpoint{6.300000in}{3.520000in}}%
\pgfpathlineto{\pgfqpoint{0.875000in}{3.520000in}}%
\pgfpathlineto{\pgfqpoint{0.875000in}{0.440000in}}%
\pgfpathclose%
\pgfusepath{fill}%
\end{pgfscope}%
\begin{pgfscope}%
\pgfpathrectangle{\pgfqpoint{0.875000in}{0.440000in}}{\pgfqpoint{5.425000in}{3.080000in}}%
\pgfusepath{clip}%
\pgfsetroundcap%
\pgfsetroundjoin%
\pgfsetlinewidth{1.003750pt}%
\definecolor{currentstroke}{rgb}{1.000000,1.000000,1.000000}%
\pgfsetstrokecolor{currentstroke}%
\pgfsetdash{}{0pt}%
\pgfpathmoveto{\pgfqpoint{1.121591in}{0.440000in}}%
\pgfpathlineto{\pgfqpoint{1.121591in}{3.520000in}}%
\pgfusepath{stroke}%
\end{pgfscope}%
\begin{pgfscope}%
\definecolor{textcolor}{rgb}{0.150000,0.150000,0.150000}%
\pgfsetstrokecolor{textcolor}%
\pgfsetfillcolor{textcolor}%
\pgftext[x=1.121591in,y=0.342778in,,top]{\color{textcolor}\rmfamily\fontsize{10.000000}{12.000000}\selectfont \(\displaystyle {0}\)}%
\end{pgfscope}%
\begin{pgfscope}%
\pgfpathrectangle{\pgfqpoint{0.875000in}{0.440000in}}{\pgfqpoint{5.425000in}{3.080000in}}%
\pgfusepath{clip}%
\pgfsetroundcap%
\pgfsetroundjoin%
\pgfsetlinewidth{1.003750pt}%
\definecolor{currentstroke}{rgb}{1.000000,1.000000,1.000000}%
\pgfsetstrokecolor{currentstroke}%
\pgfsetdash{}{0pt}%
\pgfpathmoveto{\pgfqpoint{2.108053in}{0.440000in}}%
\pgfpathlineto{\pgfqpoint{2.108053in}{3.520000in}}%
\pgfusepath{stroke}%
\end{pgfscope}%
\begin{pgfscope}%
\definecolor{textcolor}{rgb}{0.150000,0.150000,0.150000}%
\pgfsetstrokecolor{textcolor}%
\pgfsetfillcolor{textcolor}%
\pgftext[x=2.108053in,y=0.342778in,,top]{\color{textcolor}\rmfamily\fontsize{10.000000}{12.000000}\selectfont \(\displaystyle {2000}\)}%
\end{pgfscope}%
\begin{pgfscope}%
\pgfpathrectangle{\pgfqpoint{0.875000in}{0.440000in}}{\pgfqpoint{5.425000in}{3.080000in}}%
\pgfusepath{clip}%
\pgfsetroundcap%
\pgfsetroundjoin%
\pgfsetlinewidth{1.003750pt}%
\definecolor{currentstroke}{rgb}{1.000000,1.000000,1.000000}%
\pgfsetstrokecolor{currentstroke}%
\pgfsetdash{}{0pt}%
\pgfpathmoveto{\pgfqpoint{3.094515in}{0.440000in}}%
\pgfpathlineto{\pgfqpoint{3.094515in}{3.520000in}}%
\pgfusepath{stroke}%
\end{pgfscope}%
\begin{pgfscope}%
\definecolor{textcolor}{rgb}{0.150000,0.150000,0.150000}%
\pgfsetstrokecolor{textcolor}%
\pgfsetfillcolor{textcolor}%
\pgftext[x=3.094515in,y=0.342778in,,top]{\color{textcolor}\rmfamily\fontsize{10.000000}{12.000000}\selectfont \(\displaystyle {4000}\)}%
\end{pgfscope}%
\begin{pgfscope}%
\pgfpathrectangle{\pgfqpoint{0.875000in}{0.440000in}}{\pgfqpoint{5.425000in}{3.080000in}}%
\pgfusepath{clip}%
\pgfsetroundcap%
\pgfsetroundjoin%
\pgfsetlinewidth{1.003750pt}%
\definecolor{currentstroke}{rgb}{1.000000,1.000000,1.000000}%
\pgfsetstrokecolor{currentstroke}%
\pgfsetdash{}{0pt}%
\pgfpathmoveto{\pgfqpoint{4.080978in}{0.440000in}}%
\pgfpathlineto{\pgfqpoint{4.080978in}{3.520000in}}%
\pgfusepath{stroke}%
\end{pgfscope}%
\begin{pgfscope}%
\definecolor{textcolor}{rgb}{0.150000,0.150000,0.150000}%
\pgfsetstrokecolor{textcolor}%
\pgfsetfillcolor{textcolor}%
\pgftext[x=4.080978in,y=0.342778in,,top]{\color{textcolor}\rmfamily\fontsize{10.000000}{12.000000}\selectfont \(\displaystyle {6000}\)}%
\end{pgfscope}%
\begin{pgfscope}%
\pgfpathrectangle{\pgfqpoint{0.875000in}{0.440000in}}{\pgfqpoint{5.425000in}{3.080000in}}%
\pgfusepath{clip}%
\pgfsetroundcap%
\pgfsetroundjoin%
\pgfsetlinewidth{1.003750pt}%
\definecolor{currentstroke}{rgb}{1.000000,1.000000,1.000000}%
\pgfsetstrokecolor{currentstroke}%
\pgfsetdash{}{0pt}%
\pgfpathmoveto{\pgfqpoint{5.067440in}{0.440000in}}%
\pgfpathlineto{\pgfqpoint{5.067440in}{3.520000in}}%
\pgfusepath{stroke}%
\end{pgfscope}%
\begin{pgfscope}%
\definecolor{textcolor}{rgb}{0.150000,0.150000,0.150000}%
\pgfsetstrokecolor{textcolor}%
\pgfsetfillcolor{textcolor}%
\pgftext[x=5.067440in,y=0.342778in,,top]{\color{textcolor}\rmfamily\fontsize{10.000000}{12.000000}\selectfont \(\displaystyle {8000}\)}%
\end{pgfscope}%
\begin{pgfscope}%
\pgfpathrectangle{\pgfqpoint{0.875000in}{0.440000in}}{\pgfqpoint{5.425000in}{3.080000in}}%
\pgfusepath{clip}%
\pgfsetroundcap%
\pgfsetroundjoin%
\pgfsetlinewidth{1.003750pt}%
\definecolor{currentstroke}{rgb}{1.000000,1.000000,1.000000}%
\pgfsetstrokecolor{currentstroke}%
\pgfsetdash{}{0pt}%
\pgfpathmoveto{\pgfqpoint{6.053902in}{0.440000in}}%
\pgfpathlineto{\pgfqpoint{6.053902in}{3.520000in}}%
\pgfusepath{stroke}%
\end{pgfscope}%
\begin{pgfscope}%
\definecolor{textcolor}{rgb}{0.150000,0.150000,0.150000}%
\pgfsetstrokecolor{textcolor}%
\pgfsetfillcolor{textcolor}%
\pgftext[x=6.053902in,y=0.342778in,,top]{\color{textcolor}\rmfamily\fontsize{10.000000}{12.000000}\selectfont \(\displaystyle {10000}\)}%
\end{pgfscope}%
\begin{pgfscope}%
\pgfpathrectangle{\pgfqpoint{0.875000in}{0.440000in}}{\pgfqpoint{5.425000in}{3.080000in}}%
\pgfusepath{clip}%
\pgfsetroundcap%
\pgfsetroundjoin%
\pgfsetlinewidth{1.003750pt}%
\definecolor{currentstroke}{rgb}{1.000000,1.000000,1.000000}%
\pgfsetstrokecolor{currentstroke}%
\pgfsetdash{}{0pt}%
\pgfpathmoveto{\pgfqpoint{0.875000in}{0.529461in}}%
\pgfpathlineto{\pgfqpoint{6.300000in}{0.529461in}}%
\pgfusepath{stroke}%
\end{pgfscope}%
\begin{pgfscope}%
\definecolor{textcolor}{rgb}{0.150000,0.150000,0.150000}%
\pgfsetstrokecolor{textcolor}%
\pgfsetfillcolor{textcolor}%
\pgftext[x=0.530863in, y=0.481235in, left, base]{\color{textcolor}\rmfamily\fontsize{10.000000}{12.000000}\selectfont \(\displaystyle {\ensuremath{-}11}\)}%
\end{pgfscope}%
\begin{pgfscope}%
\pgfpathrectangle{\pgfqpoint{0.875000in}{0.440000in}}{\pgfqpoint{5.425000in}{3.080000in}}%
\pgfusepath{clip}%
\pgfsetroundcap%
\pgfsetroundjoin%
\pgfsetlinewidth{1.003750pt}%
\definecolor{currentstroke}{rgb}{1.000000,1.000000,1.000000}%
\pgfsetstrokecolor{currentstroke}%
\pgfsetdash{}{0pt}%
\pgfpathmoveto{\pgfqpoint{0.875000in}{0.989058in}}%
\pgfpathlineto{\pgfqpoint{6.300000in}{0.989058in}}%
\pgfusepath{stroke}%
\end{pgfscope}%
\begin{pgfscope}%
\definecolor{textcolor}{rgb}{0.150000,0.150000,0.150000}%
\pgfsetstrokecolor{textcolor}%
\pgfsetfillcolor{textcolor}%
\pgftext[x=0.530863in, y=0.940832in, left, base]{\color{textcolor}\rmfamily\fontsize{10.000000}{12.000000}\selectfont \(\displaystyle {\ensuremath{-}10}\)}%
\end{pgfscope}%
\begin{pgfscope}%
\pgfpathrectangle{\pgfqpoint{0.875000in}{0.440000in}}{\pgfqpoint{5.425000in}{3.080000in}}%
\pgfusepath{clip}%
\pgfsetroundcap%
\pgfsetroundjoin%
\pgfsetlinewidth{1.003750pt}%
\definecolor{currentstroke}{rgb}{1.000000,1.000000,1.000000}%
\pgfsetstrokecolor{currentstroke}%
\pgfsetdash{}{0pt}%
\pgfpathmoveto{\pgfqpoint{0.875000in}{1.448654in}}%
\pgfpathlineto{\pgfqpoint{6.300000in}{1.448654in}}%
\pgfusepath{stroke}%
\end{pgfscope}%
\begin{pgfscope}%
\definecolor{textcolor}{rgb}{0.150000,0.150000,0.150000}%
\pgfsetstrokecolor{textcolor}%
\pgfsetfillcolor{textcolor}%
\pgftext[x=0.600308in, y=1.400429in, left, base]{\color{textcolor}\rmfamily\fontsize{10.000000}{12.000000}\selectfont \(\displaystyle {\ensuremath{-}9}\)}%
\end{pgfscope}%
\begin{pgfscope}%
\pgfpathrectangle{\pgfqpoint{0.875000in}{0.440000in}}{\pgfqpoint{5.425000in}{3.080000in}}%
\pgfusepath{clip}%
\pgfsetroundcap%
\pgfsetroundjoin%
\pgfsetlinewidth{1.003750pt}%
\definecolor{currentstroke}{rgb}{1.000000,1.000000,1.000000}%
\pgfsetstrokecolor{currentstroke}%
\pgfsetdash{}{0pt}%
\pgfpathmoveto{\pgfqpoint{0.875000in}{1.908251in}}%
\pgfpathlineto{\pgfqpoint{6.300000in}{1.908251in}}%
\pgfusepath{stroke}%
\end{pgfscope}%
\begin{pgfscope}%
\definecolor{textcolor}{rgb}{0.150000,0.150000,0.150000}%
\pgfsetstrokecolor{textcolor}%
\pgfsetfillcolor{textcolor}%
\pgftext[x=0.600308in, y=1.860026in, left, base]{\color{textcolor}\rmfamily\fontsize{10.000000}{12.000000}\selectfont \(\displaystyle {\ensuremath{-}8}\)}%
\end{pgfscope}%
\begin{pgfscope}%
\pgfpathrectangle{\pgfqpoint{0.875000in}{0.440000in}}{\pgfqpoint{5.425000in}{3.080000in}}%
\pgfusepath{clip}%
\pgfsetroundcap%
\pgfsetroundjoin%
\pgfsetlinewidth{1.003750pt}%
\definecolor{currentstroke}{rgb}{1.000000,1.000000,1.000000}%
\pgfsetstrokecolor{currentstroke}%
\pgfsetdash{}{0pt}%
\pgfpathmoveto{\pgfqpoint{0.875000in}{2.367848in}}%
\pgfpathlineto{\pgfqpoint{6.300000in}{2.367848in}}%
\pgfusepath{stroke}%
\end{pgfscope}%
\begin{pgfscope}%
\definecolor{textcolor}{rgb}{0.150000,0.150000,0.150000}%
\pgfsetstrokecolor{textcolor}%
\pgfsetfillcolor{textcolor}%
\pgftext[x=0.600308in, y=2.319623in, left, base]{\color{textcolor}\rmfamily\fontsize{10.000000}{12.000000}\selectfont \(\displaystyle {\ensuremath{-}7}\)}%
\end{pgfscope}%
\begin{pgfscope}%
\pgfpathrectangle{\pgfqpoint{0.875000in}{0.440000in}}{\pgfqpoint{5.425000in}{3.080000in}}%
\pgfusepath{clip}%
\pgfsetroundcap%
\pgfsetroundjoin%
\pgfsetlinewidth{1.003750pt}%
\definecolor{currentstroke}{rgb}{1.000000,1.000000,1.000000}%
\pgfsetstrokecolor{currentstroke}%
\pgfsetdash{}{0pt}%
\pgfpathmoveto{\pgfqpoint{0.875000in}{2.827445in}}%
\pgfpathlineto{\pgfqpoint{6.300000in}{2.827445in}}%
\pgfusepath{stroke}%
\end{pgfscope}%
\begin{pgfscope}%
\definecolor{textcolor}{rgb}{0.150000,0.150000,0.150000}%
\pgfsetstrokecolor{textcolor}%
\pgfsetfillcolor{textcolor}%
\pgftext[x=0.600308in, y=2.779220in, left, base]{\color{textcolor}\rmfamily\fontsize{10.000000}{12.000000}\selectfont \(\displaystyle {\ensuremath{-}6}\)}%
\end{pgfscope}%
\begin{pgfscope}%
\pgfpathrectangle{\pgfqpoint{0.875000in}{0.440000in}}{\pgfqpoint{5.425000in}{3.080000in}}%
\pgfusepath{clip}%
\pgfsetroundcap%
\pgfsetroundjoin%
\pgfsetlinewidth{1.003750pt}%
\definecolor{currentstroke}{rgb}{1.000000,1.000000,1.000000}%
\pgfsetstrokecolor{currentstroke}%
\pgfsetdash{}{0pt}%
\pgfpathmoveto{\pgfqpoint{0.875000in}{3.287042in}}%
\pgfpathlineto{\pgfqpoint{6.300000in}{3.287042in}}%
\pgfusepath{stroke}%
\end{pgfscope}%
\begin{pgfscope}%
\definecolor{textcolor}{rgb}{0.150000,0.150000,0.150000}%
\pgfsetstrokecolor{textcolor}%
\pgfsetfillcolor{textcolor}%
\pgftext[x=0.600308in, y=3.238817in, left, base]{\color{textcolor}\rmfamily\fontsize{10.000000}{12.000000}\selectfont \(\displaystyle {\ensuremath{-}5}\)}%
\end{pgfscope}%
\begin{pgfscope}%
\pgfpathrectangle{\pgfqpoint{0.875000in}{0.440000in}}{\pgfqpoint{5.425000in}{3.080000in}}%
\pgfusepath{clip}%
\pgfsetroundcap%
\pgfsetroundjoin%
\pgfsetlinewidth{1.756562pt}%
\definecolor{currentstroke}{rgb}{0.298039,0.447059,0.690196}%
\pgfsetstrokecolor{currentstroke}%
\pgfsetdash{}{0pt}%
\pgfpathmoveto{\pgfqpoint{1.121591in}{2.467691in}}%
\pgfpathlineto{\pgfqpoint{1.123071in}{3.352378in}}%
\pgfpathlineto{\pgfqpoint{1.124550in}{3.258769in}}%
\pgfpathlineto{\pgfqpoint{1.126523in}{3.258769in}}%
\pgfpathlineto{\pgfqpoint{1.128003in}{2.667723in}}%
\pgfpathlineto{\pgfqpoint{1.129483in}{3.372217in}}%
\pgfpathlineto{\pgfqpoint{1.129976in}{3.372217in}}%
\pgfpathlineto{\pgfqpoint{1.130962in}{3.107263in}}%
\pgfpathlineto{\pgfqpoint{1.131949in}{3.377241in}}%
\pgfpathlineto{\pgfqpoint{1.132442in}{3.359613in}}%
\pgfpathlineto{\pgfqpoint{1.133922in}{2.776796in}}%
\pgfpathlineto{\pgfqpoint{1.134908in}{2.776796in}}%
\pgfpathlineto{\pgfqpoint{1.136388in}{2.065251in}}%
\pgfpathlineto{\pgfqpoint{1.137868in}{3.374168in}}%
\pgfpathlineto{\pgfqpoint{1.138361in}{3.374168in}}%
\pgfpathlineto{\pgfqpoint{1.138854in}{3.363089in}}%
\pgfpathlineto{\pgfqpoint{1.139840in}{2.964510in}}%
\pgfpathlineto{\pgfqpoint{1.141320in}{3.378013in}}%
\pgfpathlineto{\pgfqpoint{1.141813in}{3.378013in}}%
\pgfpathlineto{\pgfqpoint{1.142307in}{3.348536in}}%
\pgfpathlineto{\pgfqpoint{1.143293in}{2.624719in}}%
\pgfpathlineto{\pgfqpoint{1.144773in}{3.379936in}}%
\pgfpathlineto{\pgfqpoint{1.145266in}{3.234243in}}%
\pgfpathlineto{\pgfqpoint{1.145759in}{3.347606in}}%
\pgfpathlineto{\pgfqpoint{1.146746in}{3.347606in}}%
\pgfpathlineto{\pgfqpoint{1.147732in}{3.274520in}}%
\pgfpathlineto{\pgfqpoint{1.149212in}{3.367902in}}%
\pgfpathlineto{\pgfqpoint{1.150198in}{3.379774in}}%
\pgfpathlineto{\pgfqpoint{1.150692in}{2.717142in}}%
\pgfpathlineto{\pgfqpoint{1.151185in}{3.275789in}}%
\pgfpathlineto{\pgfqpoint{1.152171in}{3.275789in}}%
\pgfpathlineto{\pgfqpoint{1.153651in}{3.379979in}}%
\pgfpathlineto{\pgfqpoint{1.154144in}{3.379979in}}%
\pgfpathlineto{\pgfqpoint{1.155131in}{3.075645in}}%
\pgfpathlineto{\pgfqpoint{1.156117in}{3.061551in}}%
\pgfpathlineto{\pgfqpoint{1.156610in}{3.367340in}}%
\pgfpathlineto{\pgfqpoint{1.157104in}{2.953511in}}%
\pgfpathlineto{\pgfqpoint{1.157597in}{3.228441in}}%
\pgfpathlineto{\pgfqpoint{1.158583in}{3.127873in}}%
\pgfpathlineto{\pgfqpoint{1.159570in}{3.378417in}}%
\pgfpathlineto{\pgfqpoint{1.160063in}{2.802928in}}%
\pgfpathlineto{\pgfqpoint{1.160556in}{3.376072in}}%
\pgfpathlineto{\pgfqpoint{1.161049in}{3.376072in}}%
\pgfpathlineto{\pgfqpoint{1.162529in}{3.379501in}}%
\pgfpathlineto{\pgfqpoint{1.163516in}{3.379501in}}%
\pgfpathlineto{\pgfqpoint{1.164009in}{2.090603in}}%
\pgfpathlineto{\pgfqpoint{1.164502in}{2.674023in}}%
\pgfpathlineto{\pgfqpoint{1.164995in}{2.702269in}}%
\pgfpathlineto{\pgfqpoint{1.165488in}{3.258172in}}%
\pgfpathlineto{\pgfqpoint{1.166475in}{2.513144in}}%
\pgfpathlineto{\pgfqpoint{1.167955in}{3.205105in}}%
\pgfpathlineto{\pgfqpoint{1.168448in}{3.205105in}}%
\pgfpathlineto{\pgfqpoint{1.168941in}{3.288122in}}%
\pgfpathlineto{\pgfqpoint{1.169434in}{3.205869in}}%
\pgfpathlineto{\pgfqpoint{1.170421in}{3.205869in}}%
\pgfpathlineto{\pgfqpoint{1.170914in}{3.056855in}}%
\pgfpathlineto{\pgfqpoint{1.171407in}{3.176894in}}%
\pgfpathlineto{\pgfqpoint{1.171900in}{3.238801in}}%
\pgfpathlineto{\pgfqpoint{1.172394in}{3.021559in}}%
\pgfpathlineto{\pgfqpoint{1.172887in}{3.145873in}}%
\pgfpathlineto{\pgfqpoint{1.173380in}{3.145873in}}%
\pgfpathlineto{\pgfqpoint{1.174860in}{3.298684in}}%
\pgfpathlineto{\pgfqpoint{1.175846in}{3.298684in}}%
\pgfpathlineto{\pgfqpoint{1.176833in}{3.339958in}}%
\pgfpathlineto{\pgfqpoint{1.177819in}{2.219058in}}%
\pgfpathlineto{\pgfqpoint{1.178806in}{3.263485in}}%
\pgfpathlineto{\pgfqpoint{1.179792in}{2.713851in}}%
\pgfpathlineto{\pgfqpoint{1.181765in}{3.359881in}}%
\pgfpathlineto{\pgfqpoint{1.182258in}{3.320389in}}%
\pgfpathlineto{\pgfqpoint{1.183245in}{3.379821in}}%
\pgfpathlineto{\pgfqpoint{1.183738in}{3.118308in}}%
\pgfpathlineto{\pgfqpoint{1.184231in}{3.253840in}}%
\pgfpathlineto{\pgfqpoint{1.185711in}{3.357059in}}%
\pgfpathlineto{\pgfqpoint{1.187191in}{3.357059in}}%
\pgfpathlineto{\pgfqpoint{1.187684in}{3.347621in}}%
\pgfpathlineto{\pgfqpoint{1.188177in}{3.372929in}}%
\pgfpathlineto{\pgfqpoint{1.188670in}{3.352642in}}%
\pgfpathlineto{\pgfqpoint{1.192123in}{3.252668in}}%
\pgfpathlineto{\pgfqpoint{1.192616in}{3.252668in}}%
\pgfpathlineto{\pgfqpoint{1.194096in}{3.365506in}}%
\pgfpathlineto{\pgfqpoint{1.195082in}{3.379729in}}%
\pgfpathlineto{\pgfqpoint{1.196562in}{3.126965in}}%
\pgfpathlineto{\pgfqpoint{1.198042in}{3.253828in}}%
\pgfpathlineto{\pgfqpoint{1.198535in}{2.664826in}}%
\pgfpathlineto{\pgfqpoint{1.199028in}{2.957310in}}%
\pgfpathlineto{\pgfqpoint{1.199521in}{2.957310in}}%
\pgfpathlineto{\pgfqpoint{1.201001in}{3.359733in}}%
\pgfpathlineto{\pgfqpoint{1.201494in}{2.970543in}}%
\pgfpathlineto{\pgfqpoint{1.201988in}{3.091542in}}%
\pgfpathlineto{\pgfqpoint{1.202974in}{3.091542in}}%
\pgfpathlineto{\pgfqpoint{1.204454in}{3.379341in}}%
\pgfpathlineto{\pgfqpoint{1.206427in}{3.379549in}}%
\pgfpathlineto{\pgfqpoint{1.206920in}{3.083761in}}%
\pgfpathlineto{\pgfqpoint{1.207906in}{3.083950in}}%
\pgfpathlineto{\pgfqpoint{1.208400in}{3.083950in}}%
\pgfpathlineto{\pgfqpoint{1.209879in}{3.349194in}}%
\pgfpathlineto{\pgfqpoint{1.211359in}{3.362434in}}%
\pgfpathlineto{\pgfqpoint{1.211852in}{3.362434in}}%
\pgfpathlineto{\pgfqpoint{1.213332in}{3.368708in}}%
\pgfpathlineto{\pgfqpoint{1.213825in}{3.356247in}}%
\pgfpathlineto{\pgfqpoint{1.215305in}{3.304973in}}%
\pgfpathlineto{\pgfqpoint{1.216291in}{3.304973in}}%
\pgfpathlineto{\pgfqpoint{1.217278in}{3.363023in}}%
\pgfpathlineto{\pgfqpoint{1.217771in}{3.311028in}}%
\pgfpathlineto{\pgfqpoint{1.219251in}{3.379974in}}%
\pgfpathlineto{\pgfqpoint{1.219744in}{3.379974in}}%
\pgfpathlineto{\pgfqpoint{1.221224in}{3.154301in}}%
\pgfpathlineto{\pgfqpoint{1.222703in}{3.379586in}}%
\pgfpathlineto{\pgfqpoint{1.223690in}{3.379586in}}%
\pgfpathlineto{\pgfqpoint{1.224183in}{3.179145in}}%
\pgfpathlineto{\pgfqpoint{1.225169in}{3.212202in}}%
\pgfpathlineto{\pgfqpoint{1.226156in}{3.212202in}}%
\pgfpathlineto{\pgfqpoint{1.229609in}{3.377498in}}%
\pgfpathlineto{\pgfqpoint{1.230102in}{3.309038in}}%
\pgfpathlineto{\pgfqpoint{1.230595in}{3.338139in}}%
\pgfpathlineto{\pgfqpoint{1.232568in}{3.338139in}}%
\pgfpathlineto{\pgfqpoint{1.233061in}{3.371833in}}%
\pgfpathlineto{\pgfqpoint{1.234541in}{2.587945in}}%
\pgfpathlineto{\pgfqpoint{1.235034in}{3.360715in}}%
\pgfpathlineto{\pgfqpoint{1.235527in}{3.301021in}}%
\pgfpathlineto{\pgfqpoint{1.236021in}{2.800195in}}%
\pgfpathlineto{\pgfqpoint{1.236514in}{3.361220in}}%
\pgfpathlineto{\pgfqpoint{1.237007in}{3.361220in}}%
\pgfpathlineto{\pgfqpoint{1.237500in}{2.735131in}}%
\pgfpathlineto{\pgfqpoint{1.237993in}{3.161374in}}%
\pgfpathlineto{\pgfqpoint{1.238487in}{3.161374in}}%
\pgfpathlineto{\pgfqpoint{1.238980in}{2.854692in}}%
\pgfpathlineto{\pgfqpoint{1.240460in}{3.380000in}}%
\pgfpathlineto{\pgfqpoint{1.241939in}{3.323611in}}%
\pgfpathlineto{\pgfqpoint{1.242433in}{3.323611in}}%
\pgfpathlineto{\pgfqpoint{1.242926in}{3.286186in}}%
\pgfpathlineto{\pgfqpoint{1.244405in}{3.379861in}}%
\pgfpathlineto{\pgfqpoint{1.244899in}{3.379861in}}%
\pgfpathlineto{\pgfqpoint{1.245392in}{3.360571in}}%
\pgfpathlineto{\pgfqpoint{1.245885in}{3.379775in}}%
\pgfpathlineto{\pgfqpoint{1.247858in}{3.105999in}}%
\pgfpathlineto{\pgfqpoint{1.248351in}{3.105999in}}%
\pgfpathlineto{\pgfqpoint{1.249831in}{3.277287in}}%
\pgfpathlineto{\pgfqpoint{1.250324in}{3.176065in}}%
\pgfpathlineto{\pgfqpoint{1.251804in}{3.348873in}}%
\pgfpathlineto{\pgfqpoint{1.253284in}{3.191227in}}%
\pgfpathlineto{\pgfqpoint{1.253777in}{3.178097in}}%
\pgfpathlineto{\pgfqpoint{1.255257in}{3.342718in}}%
\pgfpathlineto{\pgfqpoint{1.256243in}{2.832522in}}%
\pgfpathlineto{\pgfqpoint{1.256736in}{3.364952in}}%
\pgfpathlineto{\pgfqpoint{1.257229in}{3.228241in}}%
\pgfpathlineto{\pgfqpoint{1.257723in}{3.228241in}}%
\pgfpathlineto{\pgfqpoint{1.258709in}{3.147293in}}%
\pgfpathlineto{\pgfqpoint{1.260189in}{3.378267in}}%
\pgfpathlineto{\pgfqpoint{1.262162in}{3.227448in}}%
\pgfpathlineto{\pgfqpoint{1.263148in}{3.317746in}}%
\pgfpathlineto{\pgfqpoint{1.264135in}{2.676394in}}%
\pgfpathlineto{\pgfqpoint{1.264628in}{2.892119in}}%
\pgfpathlineto{\pgfqpoint{1.265121in}{2.054466in}}%
\pgfpathlineto{\pgfqpoint{1.265614in}{2.936127in}}%
\pgfpathlineto{\pgfqpoint{1.267094in}{3.156190in}}%
\pgfpathlineto{\pgfqpoint{1.269560in}{3.373039in}}%
\pgfpathlineto{\pgfqpoint{1.270547in}{3.350071in}}%
\pgfpathlineto{\pgfqpoint{1.271040in}{3.052857in}}%
\pgfpathlineto{\pgfqpoint{1.271533in}{3.294714in}}%
\pgfpathlineto{\pgfqpoint{1.273013in}{3.378987in}}%
\pgfpathlineto{\pgfqpoint{1.274493in}{3.372472in}}%
\pgfpathlineto{\pgfqpoint{1.275972in}{3.371728in}}%
\pgfpathlineto{\pgfqpoint{1.277452in}{3.335375in}}%
\pgfpathlineto{\pgfqpoint{1.277945in}{3.326939in}}%
\pgfpathlineto{\pgfqpoint{1.279425in}{3.177353in}}%
\pgfpathlineto{\pgfqpoint{1.279918in}{3.378567in}}%
\pgfpathlineto{\pgfqpoint{1.280411in}{2.841209in}}%
\pgfpathlineto{\pgfqpoint{1.280905in}{3.376880in}}%
\pgfpathlineto{\pgfqpoint{1.282877in}{2.952328in}}%
\pgfpathlineto{\pgfqpoint{1.284850in}{3.378714in}}%
\pgfpathlineto{\pgfqpoint{1.285837in}{3.379471in}}%
\pgfpathlineto{\pgfqpoint{1.286823in}{2.804024in}}%
\pgfpathlineto{\pgfqpoint{1.287810in}{3.167765in}}%
\pgfpathlineto{\pgfqpoint{1.288303in}{2.984546in}}%
\pgfpathlineto{\pgfqpoint{1.288796in}{3.203322in}}%
\pgfpathlineto{\pgfqpoint{1.289289in}{3.203322in}}%
\pgfpathlineto{\pgfqpoint{1.289783in}{3.367179in}}%
\pgfpathlineto{\pgfqpoint{1.290769in}{3.039585in}}%
\pgfpathlineto{\pgfqpoint{1.292249in}{3.307492in}}%
\pgfpathlineto{\pgfqpoint{1.294222in}{2.405420in}}%
\pgfpathlineto{\pgfqpoint{1.295702in}{3.379909in}}%
\pgfpathlineto{\pgfqpoint{1.296195in}{3.339248in}}%
\pgfpathlineto{\pgfqpoint{1.296688in}{3.354663in}}%
\pgfpathlineto{\pgfqpoint{1.299154in}{3.354663in}}%
\pgfpathlineto{\pgfqpoint{1.299647in}{3.361965in}}%
\pgfpathlineto{\pgfqpoint{1.300634in}{3.267898in}}%
\pgfpathlineto{\pgfqpoint{1.301127in}{3.283377in}}%
\pgfpathlineto{\pgfqpoint{1.302114in}{3.283377in}}%
\pgfpathlineto{\pgfqpoint{1.302607in}{3.343475in}}%
\pgfpathlineto{\pgfqpoint{1.303100in}{3.242213in}}%
\pgfpathlineto{\pgfqpoint{1.304086in}{2.274561in}}%
\pgfpathlineto{\pgfqpoint{1.306059in}{3.379942in}}%
\pgfpathlineto{\pgfqpoint{1.306553in}{3.232706in}}%
\pgfpathlineto{\pgfqpoint{1.307046in}{3.323215in}}%
\pgfpathlineto{\pgfqpoint{1.308032in}{3.323215in}}%
\pgfpathlineto{\pgfqpoint{1.309512in}{3.379811in}}%
\pgfpathlineto{\pgfqpoint{1.310992in}{3.281152in}}%
\pgfpathlineto{\pgfqpoint{1.311485in}{3.281152in}}%
\pgfpathlineto{\pgfqpoint{1.312471in}{2.902032in}}%
\pgfpathlineto{\pgfqpoint{1.313458in}{3.365753in}}%
\pgfpathlineto{\pgfqpoint{1.313951in}{3.347300in}}%
\pgfpathlineto{\pgfqpoint{1.314444in}{3.366291in}}%
\pgfpathlineto{\pgfqpoint{1.314938in}{3.364007in}}%
\pgfpathlineto{\pgfqpoint{1.315924in}{3.296912in}}%
\pgfpathlineto{\pgfqpoint{1.316417in}{3.379212in}}%
\pgfpathlineto{\pgfqpoint{1.316910in}{1.826136in}}%
\pgfpathlineto{\pgfqpoint{1.317404in}{3.013187in}}%
\pgfpathlineto{\pgfqpoint{1.317897in}{3.073757in}}%
\pgfpathlineto{\pgfqpoint{1.318390in}{2.408543in}}%
\pgfpathlineto{\pgfqpoint{1.318883in}{3.222976in}}%
\pgfpathlineto{\pgfqpoint{1.320363in}{3.378079in}}%
\pgfpathlineto{\pgfqpoint{1.323816in}{3.378079in}}%
\pgfpathlineto{\pgfqpoint{1.325295in}{3.320292in}}%
\pgfpathlineto{\pgfqpoint{1.325789in}{3.320292in}}%
\pgfpathlineto{\pgfqpoint{1.326282in}{3.372132in}}%
\pgfpathlineto{\pgfqpoint{1.326775in}{3.300805in}}%
\pgfpathlineto{\pgfqpoint{1.327268in}{3.352586in}}%
\pgfpathlineto{\pgfqpoint{1.328255in}{3.374103in}}%
\pgfpathlineto{\pgfqpoint{1.328748in}{2.766812in}}%
\pgfpathlineto{\pgfqpoint{1.329241in}{3.310272in}}%
\pgfpathlineto{\pgfqpoint{1.330228in}{3.310272in}}%
\pgfpathlineto{\pgfqpoint{1.331214in}{3.327244in}}%
\pgfpathlineto{\pgfqpoint{1.331707in}{2.696351in}}%
\pgfpathlineto{\pgfqpoint{1.333187in}{3.321108in}}%
\pgfpathlineto{\pgfqpoint{1.333680in}{3.302001in}}%
\pgfpathlineto{\pgfqpoint{1.334174in}{3.353827in}}%
\pgfpathlineto{\pgfqpoint{1.334667in}{3.310312in}}%
\pgfpathlineto{\pgfqpoint{1.336146in}{3.310312in}}%
\pgfpathlineto{\pgfqpoint{1.337626in}{3.372159in}}%
\pgfpathlineto{\pgfqpoint{1.338119in}{2.764369in}}%
\pgfpathlineto{\pgfqpoint{1.338613in}{2.986404in}}%
\pgfpathlineto{\pgfqpoint{1.340092in}{3.306765in}}%
\pgfpathlineto{\pgfqpoint{1.340586in}{3.343654in}}%
\pgfpathlineto{\pgfqpoint{1.342558in}{3.189335in}}%
\pgfpathlineto{\pgfqpoint{1.344038in}{2.763764in}}%
\pgfpathlineto{\pgfqpoint{1.344531in}{2.763764in}}%
\pgfpathlineto{\pgfqpoint{1.346011in}{3.316530in}}%
\pgfpathlineto{\pgfqpoint{1.348477in}{3.346414in}}%
\pgfpathlineto{\pgfqpoint{1.348970in}{3.346414in}}%
\pgfpathlineto{\pgfqpoint{1.349464in}{3.378878in}}%
\pgfpathlineto{\pgfqpoint{1.350943in}{3.233658in}}%
\pgfpathlineto{\pgfqpoint{1.352423in}{3.371329in}}%
\pgfpathlineto{\pgfqpoint{1.352916in}{3.371329in}}%
\pgfpathlineto{\pgfqpoint{1.353410in}{3.218745in}}%
\pgfpathlineto{\pgfqpoint{1.353903in}{3.289789in}}%
\pgfpathlineto{\pgfqpoint{1.354889in}{3.378167in}}%
\pgfpathlineto{\pgfqpoint{1.355382in}{3.361032in}}%
\pgfpathlineto{\pgfqpoint{1.356862in}{3.361032in}}%
\pgfpathlineto{\pgfqpoint{1.357355in}{3.376128in}}%
\pgfpathlineto{\pgfqpoint{1.358835in}{3.353981in}}%
\pgfpathlineto{\pgfqpoint{1.359328in}{3.353981in}}%
\pgfpathlineto{\pgfqpoint{1.359822in}{3.264128in}}%
\pgfpathlineto{\pgfqpoint{1.360315in}{3.376133in}}%
\pgfpathlineto{\pgfqpoint{1.360808in}{3.376133in}}%
\pgfpathlineto{\pgfqpoint{1.361301in}{3.378880in}}%
\pgfpathlineto{\pgfqpoint{1.362781in}{2.917403in}}%
\pgfpathlineto{\pgfqpoint{1.363274in}{2.917403in}}%
\pgfpathlineto{\pgfqpoint{1.364754in}{3.231051in}}%
\pgfpathlineto{\pgfqpoint{1.365247in}{3.231051in}}%
\pgfpathlineto{\pgfqpoint{1.365740in}{3.179902in}}%
\pgfpathlineto{\pgfqpoint{1.366234in}{3.318618in}}%
\pgfpathlineto{\pgfqpoint{1.366727in}{2.326345in}}%
\pgfpathlineto{\pgfqpoint{1.367220in}{3.379997in}}%
\pgfpathlineto{\pgfqpoint{1.368206in}{3.245234in}}%
\pgfpathlineto{\pgfqpoint{1.368700in}{3.121340in}}%
\pgfpathlineto{\pgfqpoint{1.369193in}{3.228068in}}%
\pgfpathlineto{\pgfqpoint{1.369686in}{3.231184in}}%
\pgfpathlineto{\pgfqpoint{1.370179in}{3.338240in}}%
\pgfpathlineto{\pgfqpoint{1.371659in}{2.715159in}}%
\pgfpathlineto{\pgfqpoint{1.372646in}{3.071584in}}%
\pgfpathlineto{\pgfqpoint{1.373139in}{3.043245in}}%
\pgfpathlineto{\pgfqpoint{1.373632in}{3.043245in}}%
\pgfpathlineto{\pgfqpoint{1.374125in}{2.077518in}}%
\pgfpathlineto{\pgfqpoint{1.374618in}{3.076669in}}%
\pgfpathlineto{\pgfqpoint{1.375112in}{2.921270in}}%
\pgfpathlineto{\pgfqpoint{1.375605in}{3.082958in}}%
\pgfpathlineto{\pgfqpoint{1.377085in}{3.317960in}}%
\pgfpathlineto{\pgfqpoint{1.378071in}{3.317960in}}%
\pgfpathlineto{\pgfqpoint{1.379551in}{2.862749in}}%
\pgfpathlineto{\pgfqpoint{1.380044in}{2.862749in}}%
\pgfpathlineto{\pgfqpoint{1.381030in}{2.410328in}}%
\pgfpathlineto{\pgfqpoint{1.381524in}{3.083807in}}%
\pgfpathlineto{\pgfqpoint{1.382017in}{2.743177in}}%
\pgfpathlineto{\pgfqpoint{1.382510in}{2.743177in}}%
\pgfpathlineto{\pgfqpoint{1.384483in}{3.290805in}}%
\pgfpathlineto{\pgfqpoint{1.385470in}{3.290805in}}%
\pgfpathlineto{\pgfqpoint{1.385963in}{3.379342in}}%
\pgfpathlineto{\pgfqpoint{1.386949in}{2.350735in}}%
\pgfpathlineto{\pgfqpoint{1.388429in}{3.099922in}}%
\pgfpathlineto{\pgfqpoint{1.389909in}{3.371979in}}%
\pgfpathlineto{\pgfqpoint{1.391388in}{3.375160in}}%
\pgfpathlineto{\pgfqpoint{1.394841in}{3.375160in}}%
\pgfpathlineto{\pgfqpoint{1.396814in}{2.740085in}}%
\pgfpathlineto{\pgfqpoint{1.397800in}{3.232943in}}%
\pgfpathlineto{\pgfqpoint{1.398294in}{1.953897in}}%
\pgfpathlineto{\pgfqpoint{1.398787in}{3.101807in}}%
\pgfpathlineto{\pgfqpoint{1.400267in}{3.379948in}}%
\pgfpathlineto{\pgfqpoint{1.400760in}{3.379948in}}%
\pgfpathlineto{\pgfqpoint{1.401253in}{3.021051in}}%
\pgfpathlineto{\pgfqpoint{1.401746in}{3.311771in}}%
\pgfpathlineto{\pgfqpoint{1.402239in}{3.311771in}}%
\pgfpathlineto{\pgfqpoint{1.402733in}{3.190485in}}%
\pgfpathlineto{\pgfqpoint{1.403719in}{3.369647in}}%
\pgfpathlineto{\pgfqpoint{1.404706in}{3.106035in}}%
\pgfpathlineto{\pgfqpoint{1.406185in}{3.352579in}}%
\pgfpathlineto{\pgfqpoint{1.407172in}{3.370156in}}%
\pgfpathlineto{\pgfqpoint{1.408158in}{3.148707in}}%
\pgfpathlineto{\pgfqpoint{1.409145in}{3.379566in}}%
\pgfpathlineto{\pgfqpoint{1.409638in}{3.372214in}}%
\pgfpathlineto{\pgfqpoint{1.411118in}{3.324043in}}%
\pgfpathlineto{\pgfqpoint{1.411611in}{3.324043in}}%
\pgfpathlineto{\pgfqpoint{1.412104in}{3.089487in}}%
\pgfpathlineto{\pgfqpoint{1.412597in}{3.173888in}}%
\pgfpathlineto{\pgfqpoint{1.413091in}{3.173888in}}%
\pgfpathlineto{\pgfqpoint{1.414077in}{3.366434in}}%
\pgfpathlineto{\pgfqpoint{1.414570in}{3.236139in}}%
\pgfpathlineto{\pgfqpoint{1.416050in}{3.236139in}}%
\pgfpathlineto{\pgfqpoint{1.417530in}{3.372104in}}%
\pgfpathlineto{\pgfqpoint{1.418516in}{3.083558in}}%
\pgfpathlineto{\pgfqpoint{1.419009in}{3.377054in}}%
\pgfpathlineto{\pgfqpoint{1.419503in}{3.019173in}}%
\pgfpathlineto{\pgfqpoint{1.419996in}{3.304999in}}%
\pgfpathlineto{\pgfqpoint{1.421475in}{3.357404in}}%
\pgfpathlineto{\pgfqpoint{1.422462in}{3.357404in}}%
\pgfpathlineto{\pgfqpoint{1.422955in}{3.377199in}}%
\pgfpathlineto{\pgfqpoint{1.423448in}{3.298096in}}%
\pgfpathlineto{\pgfqpoint{1.424435in}{2.950929in}}%
\pgfpathlineto{\pgfqpoint{1.425421in}{3.162601in}}%
\pgfpathlineto{\pgfqpoint{1.426408in}{2.905715in}}%
\pgfpathlineto{\pgfqpoint{1.426901in}{2.927984in}}%
\pgfpathlineto{\pgfqpoint{1.428381in}{3.356074in}}%
\pgfpathlineto{\pgfqpoint{1.428874in}{3.356074in}}%
\pgfpathlineto{\pgfqpoint{1.430354in}{2.506895in}}%
\pgfpathlineto{\pgfqpoint{1.431340in}{3.329796in}}%
\pgfpathlineto{\pgfqpoint{1.431833in}{3.258049in}}%
\pgfpathlineto{\pgfqpoint{1.432327in}{3.162609in}}%
\pgfpathlineto{\pgfqpoint{1.433313in}{2.831193in}}%
\pgfpathlineto{\pgfqpoint{1.433806in}{3.365998in}}%
\pgfpathlineto{\pgfqpoint{1.434299in}{3.026555in}}%
\pgfpathlineto{\pgfqpoint{1.434793in}{3.026555in}}%
\pgfpathlineto{\pgfqpoint{1.435779in}{3.264171in}}%
\pgfpathlineto{\pgfqpoint{1.436272in}{3.237034in}}%
\pgfpathlineto{\pgfqpoint{1.437752in}{2.971460in}}%
\pgfpathlineto{\pgfqpoint{1.438245in}{2.986230in}}%
\pgfpathlineto{\pgfqpoint{1.438739in}{2.739127in}}%
\pgfpathlineto{\pgfqpoint{1.439232in}{3.371828in}}%
\pgfpathlineto{\pgfqpoint{1.439725in}{2.413774in}}%
\pgfpathlineto{\pgfqpoint{1.440218in}{3.073107in}}%
\pgfpathlineto{\pgfqpoint{1.441698in}{3.353818in}}%
\pgfpathlineto{\pgfqpoint{1.442191in}{3.293471in}}%
\pgfpathlineto{\pgfqpoint{1.442684in}{3.379993in}}%
\pgfpathlineto{\pgfqpoint{1.443178in}{1.973041in}}%
\pgfpathlineto{\pgfqpoint{1.443671in}{1.981191in}}%
\pgfpathlineto{\pgfqpoint{1.445644in}{3.377625in}}%
\pgfpathlineto{\pgfqpoint{1.446630in}{3.377625in}}%
\pgfpathlineto{\pgfqpoint{1.448110in}{3.320552in}}%
\pgfpathlineto{\pgfqpoint{1.449096in}{2.100316in}}%
\pgfpathlineto{\pgfqpoint{1.449590in}{3.311894in}}%
\pgfpathlineto{\pgfqpoint{1.450576in}{3.147585in}}%
\pgfpathlineto{\pgfqpoint{1.451069in}{3.147585in}}%
\pgfpathlineto{\pgfqpoint{1.451563in}{2.895274in}}%
\pgfpathlineto{\pgfqpoint{1.452549in}{3.336500in}}%
\pgfpathlineto{\pgfqpoint{1.453042in}{3.181664in}}%
\pgfpathlineto{\pgfqpoint{1.454522in}{3.377554in}}%
\pgfpathlineto{\pgfqpoint{1.455015in}{3.007622in}}%
\pgfpathlineto{\pgfqpoint{1.455508in}{3.272567in}}%
\pgfpathlineto{\pgfqpoint{1.456002in}{3.285399in}}%
\pgfpathlineto{\pgfqpoint{1.457481in}{3.369444in}}%
\pgfpathlineto{\pgfqpoint{1.457975in}{3.356002in}}%
\pgfpathlineto{\pgfqpoint{1.459454in}{3.088984in}}%
\pgfpathlineto{\pgfqpoint{1.460441in}{3.373264in}}%
\pgfpathlineto{\pgfqpoint{1.460934in}{3.258076in}}%
\pgfpathlineto{\pgfqpoint{1.461427in}{3.364973in}}%
\pgfpathlineto{\pgfqpoint{1.462414in}{3.364973in}}%
\pgfpathlineto{\pgfqpoint{1.462907in}{3.317971in}}%
\pgfpathlineto{\pgfqpoint{1.463893in}{2.916657in}}%
\pgfpathlineto{\pgfqpoint{1.464387in}{3.342657in}}%
\pgfpathlineto{\pgfqpoint{1.465373in}{3.301008in}}%
\pgfpathlineto{\pgfqpoint{1.466359in}{3.216154in}}%
\pgfpathlineto{\pgfqpoint{1.466853in}{3.318025in}}%
\pgfpathlineto{\pgfqpoint{1.467346in}{3.205152in}}%
\pgfpathlineto{\pgfqpoint{1.467839in}{3.312056in}}%
\pgfpathlineto{\pgfqpoint{1.469319in}{3.330674in}}%
\pgfpathlineto{\pgfqpoint{1.469812in}{3.330674in}}%
\pgfpathlineto{\pgfqpoint{1.471292in}{3.338951in}}%
\pgfpathlineto{\pgfqpoint{1.471785in}{3.338951in}}%
\pgfpathlineto{\pgfqpoint{1.472771in}{3.359638in}}%
\pgfpathlineto{\pgfqpoint{1.473758in}{2.398515in}}%
\pgfpathlineto{\pgfqpoint{1.474251in}{3.378106in}}%
\pgfpathlineto{\pgfqpoint{1.475238in}{3.312570in}}%
\pgfpathlineto{\pgfqpoint{1.475731in}{3.312570in}}%
\pgfpathlineto{\pgfqpoint{1.477211in}{3.294927in}}%
\pgfpathlineto{\pgfqpoint{1.477704in}{3.294927in}}%
\pgfpathlineto{\pgfqpoint{1.478197in}{3.378849in}}%
\pgfpathlineto{\pgfqpoint{1.478690in}{3.350410in}}%
\pgfpathlineto{\pgfqpoint{1.479183in}{3.123182in}}%
\pgfpathlineto{\pgfqpoint{1.479677in}{3.266820in}}%
\pgfpathlineto{\pgfqpoint{1.480170in}{3.344701in}}%
\pgfpathlineto{\pgfqpoint{1.480663in}{3.322176in}}%
\pgfpathlineto{\pgfqpoint{1.482636in}{2.085342in}}%
\pgfpathlineto{\pgfqpoint{1.484609in}{3.365561in}}%
\pgfpathlineto{\pgfqpoint{1.488062in}{3.365561in}}%
\pgfpathlineto{\pgfqpoint{1.489541in}{3.363294in}}%
\pgfpathlineto{\pgfqpoint{1.490035in}{3.337146in}}%
\pgfpathlineto{\pgfqpoint{1.491021in}{3.002933in}}%
\pgfpathlineto{\pgfqpoint{1.491514in}{3.097013in}}%
\pgfpathlineto{\pgfqpoint{1.492994in}{3.375539in}}%
\pgfpathlineto{\pgfqpoint{1.494474in}{3.330498in}}%
\pgfpathlineto{\pgfqpoint{1.494967in}{3.330498in}}%
\pgfpathlineto{\pgfqpoint{1.495460in}{3.322053in}}%
\pgfpathlineto{\pgfqpoint{1.496447in}{3.370413in}}%
\pgfpathlineto{\pgfqpoint{1.496940in}{3.320526in}}%
\pgfpathlineto{\pgfqpoint{1.498420in}{2.390195in}}%
\pgfpathlineto{\pgfqpoint{1.499406in}{2.482412in}}%
\pgfpathlineto{\pgfqpoint{1.499899in}{3.252701in}}%
\pgfpathlineto{\pgfqpoint{1.500886in}{3.133618in}}%
\pgfpathlineto{\pgfqpoint{1.501379in}{2.837926in}}%
\pgfpathlineto{\pgfqpoint{1.501872in}{3.352165in}}%
\pgfpathlineto{\pgfqpoint{1.502859in}{3.335936in}}%
\pgfpathlineto{\pgfqpoint{1.504338in}{3.335936in}}%
\pgfpathlineto{\pgfqpoint{1.504832in}{3.068396in}}%
\pgfpathlineto{\pgfqpoint{1.505325in}{3.188943in}}%
\pgfpathlineto{\pgfqpoint{1.505818in}{3.188943in}}%
\pgfpathlineto{\pgfqpoint{1.506311in}{3.201856in}}%
\pgfpathlineto{\pgfqpoint{1.507791in}{3.379959in}}%
\pgfpathlineto{\pgfqpoint{1.509271in}{3.379959in}}%
\pgfpathlineto{\pgfqpoint{1.510750in}{3.140325in}}%
\pgfpathlineto{\pgfqpoint{1.512230in}{3.379940in}}%
\pgfpathlineto{\pgfqpoint{1.512723in}{3.379940in}}%
\pgfpathlineto{\pgfqpoint{1.513710in}{3.221046in}}%
\pgfpathlineto{\pgfqpoint{1.515189in}{3.379999in}}%
\pgfpathlineto{\pgfqpoint{1.515683in}{3.379999in}}%
\pgfpathlineto{\pgfqpoint{1.516176in}{3.344349in}}%
\pgfpathlineto{\pgfqpoint{1.516669in}{3.367406in}}%
\pgfpathlineto{\pgfqpoint{1.517162in}{3.367406in}}%
\pgfpathlineto{\pgfqpoint{1.519135in}{3.001035in}}%
\pgfpathlineto{\pgfqpoint{1.519628in}{3.015684in}}%
\pgfpathlineto{\pgfqpoint{1.521108in}{3.345245in}}%
\pgfpathlineto{\pgfqpoint{1.522588in}{2.878840in}}%
\pgfpathlineto{\pgfqpoint{1.523574in}{3.377619in}}%
\pgfpathlineto{\pgfqpoint{1.524561in}{3.313203in}}%
\pgfpathlineto{\pgfqpoint{1.525547in}{3.313203in}}%
\pgfpathlineto{\pgfqpoint{1.527027in}{3.272039in}}%
\pgfpathlineto{\pgfqpoint{1.527520in}{3.282422in}}%
\pgfpathlineto{\pgfqpoint{1.529000in}{2.781769in}}%
\pgfpathlineto{\pgfqpoint{1.529493in}{2.717463in}}%
\pgfpathlineto{\pgfqpoint{1.530973in}{3.307187in}}%
\pgfpathlineto{\pgfqpoint{1.531959in}{2.909574in}}%
\pgfpathlineto{\pgfqpoint{1.534919in}{3.379912in}}%
\pgfpathlineto{\pgfqpoint{1.535905in}{3.367718in}}%
\pgfpathlineto{\pgfqpoint{1.537385in}{3.230761in}}%
\pgfpathlineto{\pgfqpoint{1.538864in}{3.355113in}}%
\pgfpathlineto{\pgfqpoint{1.539358in}{3.355113in}}%
\pgfpathlineto{\pgfqpoint{1.540344in}{3.365085in}}%
\pgfpathlineto{\pgfqpoint{1.540837in}{3.002161in}}%
\pgfpathlineto{\pgfqpoint{1.541331in}{3.333764in}}%
\pgfpathlineto{\pgfqpoint{1.541824in}{3.360918in}}%
\pgfpathlineto{\pgfqpoint{1.542317in}{3.238820in}}%
\pgfpathlineto{\pgfqpoint{1.542810in}{3.365635in}}%
\pgfpathlineto{\pgfqpoint{1.543304in}{3.365635in}}%
\pgfpathlineto{\pgfqpoint{1.543797in}{3.379963in}}%
\pgfpathlineto{\pgfqpoint{1.544290in}{3.378546in}}%
\pgfpathlineto{\pgfqpoint{1.545276in}{3.142878in}}%
\pgfpathlineto{\pgfqpoint{1.545770in}{3.375826in}}%
\pgfpathlineto{\pgfqpoint{1.546756in}{2.415496in}}%
\pgfpathlineto{\pgfqpoint{1.547743in}{3.138095in}}%
\pgfpathlineto{\pgfqpoint{1.548729in}{2.826890in}}%
\pgfpathlineto{\pgfqpoint{1.549716in}{3.379980in}}%
\pgfpathlineto{\pgfqpoint{1.550209in}{3.365009in}}%
\pgfpathlineto{\pgfqpoint{1.550702in}{3.365009in}}%
\pgfpathlineto{\pgfqpoint{1.552182in}{3.374741in}}%
\pgfpathlineto{\pgfqpoint{1.552675in}{3.374741in}}%
\pgfpathlineto{\pgfqpoint{1.553168in}{1.616736in}}%
\pgfpathlineto{\pgfqpoint{1.553661in}{3.338541in}}%
\pgfpathlineto{\pgfqpoint{1.554648in}{3.358430in}}%
\pgfpathlineto{\pgfqpoint{1.555141in}{3.356611in}}%
\pgfpathlineto{\pgfqpoint{1.556128in}{3.356611in}}%
\pgfpathlineto{\pgfqpoint{1.557607in}{3.378631in}}%
\pgfpathlineto{\pgfqpoint{1.559087in}{3.294240in}}%
\pgfpathlineto{\pgfqpoint{1.559580in}{3.377726in}}%
\pgfpathlineto{\pgfqpoint{1.560073in}{2.853706in}}%
\pgfpathlineto{\pgfqpoint{1.560567in}{3.363593in}}%
\pgfpathlineto{\pgfqpoint{1.561553in}{3.369907in}}%
\pgfpathlineto{\pgfqpoint{1.562046in}{3.324729in}}%
\pgfpathlineto{\pgfqpoint{1.562540in}{3.379245in}}%
\pgfpathlineto{\pgfqpoint{1.564019in}{3.379245in}}%
\pgfpathlineto{\pgfqpoint{1.565006in}{3.241290in}}%
\pgfpathlineto{\pgfqpoint{1.566485in}{3.365216in}}%
\pgfpathlineto{\pgfqpoint{1.568458in}{3.365216in}}%
\pgfpathlineto{\pgfqpoint{1.568952in}{2.718506in}}%
\pgfpathlineto{\pgfqpoint{1.569445in}{3.213066in}}%
\pgfpathlineto{\pgfqpoint{1.571418in}{3.213066in}}%
\pgfpathlineto{\pgfqpoint{1.573391in}{3.309418in}}%
\pgfpathlineto{\pgfqpoint{1.574377in}{3.309418in}}%
\pgfpathlineto{\pgfqpoint{1.574870in}{3.361201in}}%
\pgfpathlineto{\pgfqpoint{1.576350in}{3.119836in}}%
\pgfpathlineto{\pgfqpoint{1.577830in}{3.119836in}}%
\pgfpathlineto{\pgfqpoint{1.579803in}{3.372681in}}%
\pgfpathlineto{\pgfqpoint{1.581776in}{3.238903in}}%
\pgfpathlineto{\pgfqpoint{1.582269in}{2.580706in}}%
\pgfpathlineto{\pgfqpoint{1.582762in}{3.375657in}}%
\pgfpathlineto{\pgfqpoint{1.583748in}{3.375657in}}%
\pgfpathlineto{\pgfqpoint{1.584735in}{2.985695in}}%
\pgfpathlineto{\pgfqpoint{1.586215in}{3.327305in}}%
\pgfpathlineto{\pgfqpoint{1.587201in}{3.327305in}}%
\pgfpathlineto{\pgfqpoint{1.588681in}{3.330405in}}%
\pgfpathlineto{\pgfqpoint{1.589667in}{3.330405in}}%
\pgfpathlineto{\pgfqpoint{1.591640in}{3.269175in}}%
\pgfpathlineto{\pgfqpoint{1.593120in}{3.379685in}}%
\pgfpathlineto{\pgfqpoint{1.594106in}{3.379685in}}%
\pgfpathlineto{\pgfqpoint{1.595586in}{3.115454in}}%
\pgfpathlineto{\pgfqpoint{1.596079in}{3.115454in}}%
\pgfpathlineto{\pgfqpoint{1.596572in}{3.376823in}}%
\pgfpathlineto{\pgfqpoint{1.597066in}{3.353654in}}%
\pgfpathlineto{\pgfqpoint{1.597559in}{2.886467in}}%
\pgfpathlineto{\pgfqpoint{1.598052in}{3.378020in}}%
\pgfpathlineto{\pgfqpoint{1.598545in}{3.378020in}}%
\pgfpathlineto{\pgfqpoint{1.599532in}{3.263039in}}%
\pgfpathlineto{\pgfqpoint{1.600025in}{3.364216in}}%
\pgfpathlineto{\pgfqpoint{1.600518in}{3.351053in}}%
\pgfpathlineto{\pgfqpoint{1.601998in}{3.310969in}}%
\pgfpathlineto{\pgfqpoint{1.602985in}{3.310969in}}%
\pgfpathlineto{\pgfqpoint{1.603971in}{2.576774in}}%
\pgfpathlineto{\pgfqpoint{1.606930in}{3.373626in}}%
\pgfpathlineto{\pgfqpoint{1.607424in}{3.228797in}}%
\pgfpathlineto{\pgfqpoint{1.608410in}{3.242147in}}%
\pgfpathlineto{\pgfqpoint{1.608903in}{3.182410in}}%
\pgfpathlineto{\pgfqpoint{1.610383in}{3.372570in}}%
\pgfpathlineto{\pgfqpoint{1.611369in}{3.372570in}}%
\pgfpathlineto{\pgfqpoint{1.612356in}{3.366562in}}%
\pgfpathlineto{\pgfqpoint{1.612849in}{3.147180in}}%
\pgfpathlineto{\pgfqpoint{1.613342in}{3.299511in}}%
\pgfpathlineto{\pgfqpoint{1.613836in}{3.299511in}}%
\pgfpathlineto{\pgfqpoint{1.615809in}{3.379180in}}%
\pgfpathlineto{\pgfqpoint{1.616302in}{3.206340in}}%
\pgfpathlineto{\pgfqpoint{1.616795in}{3.338254in}}%
\pgfpathlineto{\pgfqpoint{1.617288in}{3.371047in}}%
\pgfpathlineto{\pgfqpoint{1.618275in}{3.364605in}}%
\pgfpathlineto{\pgfqpoint{1.620741in}{3.364605in}}%
\pgfpathlineto{\pgfqpoint{1.621234in}{3.339288in}}%
\pgfpathlineto{\pgfqpoint{1.622221in}{3.379969in}}%
\pgfpathlineto{\pgfqpoint{1.623207in}{2.740748in}}%
\pgfpathlineto{\pgfqpoint{1.625180in}{3.378879in}}%
\pgfpathlineto{\pgfqpoint{1.627153in}{3.379811in}}%
\pgfpathlineto{\pgfqpoint{1.628633in}{2.663001in}}%
\pgfpathlineto{\pgfqpoint{1.630112in}{3.218783in}}%
\pgfpathlineto{\pgfqpoint{1.630605in}{3.372345in}}%
\pgfpathlineto{\pgfqpoint{1.631592in}{2.416118in}}%
\pgfpathlineto{\pgfqpoint{1.632085in}{2.551947in}}%
\pgfpathlineto{\pgfqpoint{1.633565in}{3.289965in}}%
\pgfpathlineto{\pgfqpoint{1.635045in}{3.353297in}}%
\pgfpathlineto{\pgfqpoint{1.635538in}{3.379813in}}%
\pgfpathlineto{\pgfqpoint{1.636031in}{3.377929in}}%
\pgfpathlineto{\pgfqpoint{1.636524in}{3.327226in}}%
\pgfpathlineto{\pgfqpoint{1.637017in}{3.072729in}}%
\pgfpathlineto{\pgfqpoint{1.637511in}{3.329408in}}%
\pgfpathlineto{\pgfqpoint{1.638004in}{3.329408in}}%
\pgfpathlineto{\pgfqpoint{1.638497in}{3.366110in}}%
\pgfpathlineto{\pgfqpoint{1.639484in}{3.363116in}}%
\pgfpathlineto{\pgfqpoint{1.639977in}{3.376243in}}%
\pgfpathlineto{\pgfqpoint{1.640470in}{2.708817in}}%
\pgfpathlineto{\pgfqpoint{1.640963in}{3.367701in}}%
\pgfpathlineto{\pgfqpoint{1.641950in}{3.171128in}}%
\pgfpathlineto{\pgfqpoint{1.642443in}{3.369429in}}%
\pgfpathlineto{\pgfqpoint{1.642936in}{3.318017in}}%
\pgfpathlineto{\pgfqpoint{1.643923in}{3.318017in}}%
\pgfpathlineto{\pgfqpoint{1.644416in}{2.796163in}}%
\pgfpathlineto{\pgfqpoint{1.644909in}{3.184415in}}%
\pgfpathlineto{\pgfqpoint{1.645402in}{3.366576in}}%
\pgfpathlineto{\pgfqpoint{1.645896in}{3.313619in}}%
\pgfpathlineto{\pgfqpoint{1.646882in}{3.232962in}}%
\pgfpathlineto{\pgfqpoint{1.647869in}{3.373188in}}%
\pgfpathlineto{\pgfqpoint{1.650335in}{3.053163in}}%
\pgfpathlineto{\pgfqpoint{1.651321in}{3.370919in}}%
\pgfpathlineto{\pgfqpoint{1.651814in}{3.365070in}}%
\pgfpathlineto{\pgfqpoint{1.652308in}{3.100088in}}%
\pgfpathlineto{\pgfqpoint{1.652801in}{3.379872in}}%
\pgfpathlineto{\pgfqpoint{1.653294in}{3.379872in}}%
\pgfpathlineto{\pgfqpoint{1.653787in}{1.844691in}}%
\pgfpathlineto{\pgfqpoint{1.654281in}{3.373709in}}%
\pgfpathlineto{\pgfqpoint{1.655267in}{3.342973in}}%
\pgfpathlineto{\pgfqpoint{1.656747in}{3.182533in}}%
\pgfpathlineto{\pgfqpoint{1.658226in}{3.132905in}}%
\pgfpathlineto{\pgfqpoint{1.659706in}{3.379546in}}%
\pgfpathlineto{\pgfqpoint{1.662665in}{3.379546in}}%
\pgfpathlineto{\pgfqpoint{1.663652in}{3.338399in}}%
\pgfpathlineto{\pgfqpoint{1.664145in}{3.061590in}}%
\pgfpathlineto{\pgfqpoint{1.664638in}{3.317754in}}%
\pgfpathlineto{\pgfqpoint{1.665132in}{3.317754in}}%
\pgfpathlineto{\pgfqpoint{1.666118in}{3.379303in}}%
\pgfpathlineto{\pgfqpoint{1.669077in}{3.227063in}}%
\pgfpathlineto{\pgfqpoint{1.669571in}{3.227063in}}%
\pgfpathlineto{\pgfqpoint{1.671050in}{2.883901in}}%
\pgfpathlineto{\pgfqpoint{1.672037in}{3.228227in}}%
\pgfpathlineto{\pgfqpoint{1.672530in}{3.042426in}}%
\pgfpathlineto{\pgfqpoint{1.673023in}{3.109908in}}%
\pgfpathlineto{\pgfqpoint{1.673517in}{3.166838in}}%
\pgfpathlineto{\pgfqpoint{1.674010in}{3.373918in}}%
\pgfpathlineto{\pgfqpoint{1.674996in}{3.364119in}}%
\pgfpathlineto{\pgfqpoint{1.675489in}{3.375896in}}%
\pgfpathlineto{\pgfqpoint{1.675983in}{3.373823in}}%
\pgfpathlineto{\pgfqpoint{1.677462in}{3.042644in}}%
\pgfpathlineto{\pgfqpoint{1.678942in}{3.379926in}}%
\pgfpathlineto{\pgfqpoint{1.679435in}{3.379926in}}%
\pgfpathlineto{\pgfqpoint{1.680915in}{3.212645in}}%
\pgfpathlineto{\pgfqpoint{1.681408in}{3.312241in}}%
\pgfpathlineto{\pgfqpoint{1.682395in}{3.109675in}}%
\pgfpathlineto{\pgfqpoint{1.683874in}{3.238451in}}%
\pgfpathlineto{\pgfqpoint{1.684368in}{3.238451in}}%
\pgfpathlineto{\pgfqpoint{1.684861in}{3.371124in}}%
\pgfpathlineto{\pgfqpoint{1.685847in}{2.161246in}}%
\pgfpathlineto{\pgfqpoint{1.686341in}{2.401861in}}%
\pgfpathlineto{\pgfqpoint{1.687327in}{3.356954in}}%
\pgfpathlineto{\pgfqpoint{1.687820in}{3.348333in}}%
\pgfpathlineto{\pgfqpoint{1.688807in}{3.348333in}}%
\pgfpathlineto{\pgfqpoint{1.689300in}{3.060780in}}%
\pgfpathlineto{\pgfqpoint{1.689793in}{3.177254in}}%
\pgfpathlineto{\pgfqpoint{1.690286in}{3.365577in}}%
\pgfpathlineto{\pgfqpoint{1.691273in}{3.355927in}}%
\pgfpathlineto{\pgfqpoint{1.691766in}{3.355927in}}%
\pgfpathlineto{\pgfqpoint{1.693246in}{2.979579in}}%
\pgfpathlineto{\pgfqpoint{1.693739in}{2.979579in}}%
\pgfpathlineto{\pgfqpoint{1.694232in}{3.345295in}}%
\pgfpathlineto{\pgfqpoint{1.695219in}{3.255811in}}%
\pgfpathlineto{\pgfqpoint{1.695712in}{3.250555in}}%
\pgfpathlineto{\pgfqpoint{1.696698in}{3.379993in}}%
\pgfpathlineto{\pgfqpoint{1.697192in}{3.349175in}}%
\pgfpathlineto{\pgfqpoint{1.697685in}{3.349175in}}%
\pgfpathlineto{\pgfqpoint{1.698178in}{3.362661in}}%
\pgfpathlineto{\pgfqpoint{1.698671in}{2.470278in}}%
\pgfpathlineto{\pgfqpoint{1.699165in}{2.929010in}}%
\pgfpathlineto{\pgfqpoint{1.699658in}{2.832755in}}%
\pgfpathlineto{\pgfqpoint{1.700151in}{0.699616in}}%
\pgfpathlineto{\pgfqpoint{1.700644in}{2.912679in}}%
\pgfpathlineto{\pgfqpoint{1.702124in}{3.361490in}}%
\pgfpathlineto{\pgfqpoint{1.702617in}{3.340575in}}%
\pgfpathlineto{\pgfqpoint{1.703110in}{2.703892in}}%
\pgfpathlineto{\pgfqpoint{1.703604in}{3.377112in}}%
\pgfpathlineto{\pgfqpoint{1.704097in}{2.948741in}}%
\pgfpathlineto{\pgfqpoint{1.704590in}{3.169203in}}%
\pgfpathlineto{\pgfqpoint{1.705083in}{3.169203in}}%
\pgfpathlineto{\pgfqpoint{1.705577in}{2.827680in}}%
\pgfpathlineto{\pgfqpoint{1.706070in}{3.056509in}}%
\pgfpathlineto{\pgfqpoint{1.706563in}{2.994053in}}%
\pgfpathlineto{\pgfqpoint{1.707056in}{3.319987in}}%
\pgfpathlineto{\pgfqpoint{1.707550in}{2.499019in}}%
\pgfpathlineto{\pgfqpoint{1.708043in}{3.039586in}}%
\pgfpathlineto{\pgfqpoint{1.709029in}{3.356389in}}%
\pgfpathlineto{\pgfqpoint{1.710016in}{3.051282in}}%
\pgfpathlineto{\pgfqpoint{1.710509in}{3.354645in}}%
\pgfpathlineto{\pgfqpoint{1.711002in}{3.137887in}}%
\pgfpathlineto{\pgfqpoint{1.711495in}{3.137887in}}%
\pgfpathlineto{\pgfqpoint{1.712482in}{3.379772in}}%
\pgfpathlineto{\pgfqpoint{1.712975in}{3.374789in}}%
\pgfpathlineto{\pgfqpoint{1.714948in}{3.306979in}}%
\pgfpathlineto{\pgfqpoint{1.715934in}{2.340920in}}%
\pgfpathlineto{\pgfqpoint{1.716921in}{3.379763in}}%
\pgfpathlineto{\pgfqpoint{1.717414in}{3.100847in}}%
\pgfpathlineto{\pgfqpoint{1.717907in}{3.100847in}}%
\pgfpathlineto{\pgfqpoint{1.718401in}{3.378523in}}%
\pgfpathlineto{\pgfqpoint{1.719387in}{3.375520in}}%
\pgfpathlineto{\pgfqpoint{1.719880in}{3.312933in}}%
\pgfpathlineto{\pgfqpoint{1.720867in}{2.863446in}}%
\pgfpathlineto{\pgfqpoint{1.722346in}{3.379731in}}%
\pgfpathlineto{\pgfqpoint{1.723333in}{3.249077in}}%
\pgfpathlineto{\pgfqpoint{1.723826in}{3.379919in}}%
\pgfpathlineto{\pgfqpoint{1.724319in}{3.246009in}}%
\pgfpathlineto{\pgfqpoint{1.724813in}{3.125512in}}%
\pgfpathlineto{\pgfqpoint{1.726292in}{3.342825in}}%
\pgfpathlineto{\pgfqpoint{1.727279in}{2.809648in}}%
\pgfpathlineto{\pgfqpoint{1.727772in}{2.989736in}}%
\pgfpathlineto{\pgfqpoint{1.728265in}{2.989736in}}%
\pgfpathlineto{\pgfqpoint{1.729252in}{3.007641in}}%
\pgfpathlineto{\pgfqpoint{1.729745in}{2.958979in}}%
\pgfpathlineto{\pgfqpoint{1.731225in}{3.326650in}}%
\pgfpathlineto{\pgfqpoint{1.732704in}{3.350002in}}%
\pgfpathlineto{\pgfqpoint{1.733198in}{3.350002in}}%
\pgfpathlineto{\pgfqpoint{1.734184in}{2.972482in}}%
\pgfpathlineto{\pgfqpoint{1.734677in}{3.042981in}}%
\pgfpathlineto{\pgfqpoint{1.735170in}{3.346097in}}%
\pgfpathlineto{\pgfqpoint{1.735664in}{3.137588in}}%
\pgfpathlineto{\pgfqpoint{1.736157in}{3.137588in}}%
\pgfpathlineto{\pgfqpoint{1.737637in}{3.359155in}}%
\pgfpathlineto{\pgfqpoint{1.738130in}{3.192179in}}%
\pgfpathlineto{\pgfqpoint{1.738623in}{3.251235in}}%
\pgfpathlineto{\pgfqpoint{1.739610in}{3.370844in}}%
\pgfpathlineto{\pgfqpoint{1.740103in}{3.231672in}}%
\pgfpathlineto{\pgfqpoint{1.740596in}{2.721989in}}%
\pgfpathlineto{\pgfqpoint{1.741089in}{2.929644in}}%
\pgfpathlineto{\pgfqpoint{1.742076in}{3.067124in}}%
\pgfpathlineto{\pgfqpoint{1.743062in}{2.968233in}}%
\pgfpathlineto{\pgfqpoint{1.743555in}{3.012076in}}%
\pgfpathlineto{\pgfqpoint{1.744542in}{3.377320in}}%
\pgfpathlineto{\pgfqpoint{1.745035in}{3.371338in}}%
\pgfpathlineto{\pgfqpoint{1.746515in}{3.205801in}}%
\pgfpathlineto{\pgfqpoint{1.747008in}{3.205801in}}%
\pgfpathlineto{\pgfqpoint{1.748488in}{3.376065in}}%
\pgfpathlineto{\pgfqpoint{1.749474in}{3.376065in}}%
\pgfpathlineto{\pgfqpoint{1.749967in}{3.239185in}}%
\pgfpathlineto{\pgfqpoint{1.750461in}{3.289824in}}%
\pgfpathlineto{\pgfqpoint{1.751447in}{3.289824in}}%
\pgfpathlineto{\pgfqpoint{1.752434in}{3.365299in}}%
\pgfpathlineto{\pgfqpoint{1.752927in}{2.796051in}}%
\pgfpathlineto{\pgfqpoint{1.753420in}{3.132325in}}%
\pgfpathlineto{\pgfqpoint{1.754406in}{3.377263in}}%
\pgfpathlineto{\pgfqpoint{1.754900in}{2.669426in}}%
\pgfpathlineto{\pgfqpoint{1.755886in}{2.673261in}}%
\pgfpathlineto{\pgfqpoint{1.756379in}{2.673261in}}%
\pgfpathlineto{\pgfqpoint{1.756873in}{2.500807in}}%
\pgfpathlineto{\pgfqpoint{1.758352in}{3.365068in}}%
\pgfpathlineto{\pgfqpoint{1.758846in}{3.365068in}}%
\pgfpathlineto{\pgfqpoint{1.759339in}{3.324566in}}%
\pgfpathlineto{\pgfqpoint{1.759832in}{3.074189in}}%
\pgfpathlineto{\pgfqpoint{1.760818in}{3.106562in}}%
\pgfpathlineto{\pgfqpoint{1.761312in}{2.995876in}}%
\pgfpathlineto{\pgfqpoint{1.762791in}{3.356894in}}%
\pgfpathlineto{\pgfqpoint{1.764271in}{3.305004in}}%
\pgfpathlineto{\pgfqpoint{1.764764in}{3.378952in}}%
\pgfpathlineto{\pgfqpoint{1.765258in}{3.358838in}}%
\pgfpathlineto{\pgfqpoint{1.765751in}{3.358838in}}%
\pgfpathlineto{\pgfqpoint{1.767230in}{3.085337in}}%
\pgfpathlineto{\pgfqpoint{1.768710in}{3.357538in}}%
\pgfpathlineto{\pgfqpoint{1.769203in}{3.377267in}}%
\pgfpathlineto{\pgfqpoint{1.770190in}{3.133383in}}%
\pgfpathlineto{\pgfqpoint{1.771176in}{3.372652in}}%
\pgfpathlineto{\pgfqpoint{1.771670in}{3.138659in}}%
\pgfpathlineto{\pgfqpoint{1.772656in}{3.161258in}}%
\pgfpathlineto{\pgfqpoint{1.773149in}{3.161258in}}%
\pgfpathlineto{\pgfqpoint{1.773642in}{2.848277in}}%
\pgfpathlineto{\pgfqpoint{1.774629in}{3.379630in}}%
\pgfpathlineto{\pgfqpoint{1.775122in}{3.302234in}}%
\pgfpathlineto{\pgfqpoint{1.776602in}{3.302234in}}%
\pgfpathlineto{\pgfqpoint{1.777588in}{3.378220in}}%
\pgfpathlineto{\pgfqpoint{1.779068in}{3.142249in}}%
\pgfpathlineto{\pgfqpoint{1.779561in}{3.347646in}}%
\pgfpathlineto{\pgfqpoint{1.780054in}{3.105705in}}%
\pgfpathlineto{\pgfqpoint{1.780548in}{3.105705in}}%
\pgfpathlineto{\pgfqpoint{1.782027in}{3.329109in}}%
\pgfpathlineto{\pgfqpoint{1.782521in}{2.936813in}}%
\pgfpathlineto{\pgfqpoint{1.783014in}{3.305547in}}%
\pgfpathlineto{\pgfqpoint{1.783507in}{3.379945in}}%
\pgfpathlineto{\pgfqpoint{1.784494in}{3.366115in}}%
\pgfpathlineto{\pgfqpoint{1.785480in}{3.366115in}}%
\pgfpathlineto{\pgfqpoint{1.786960in}{2.966681in}}%
\pgfpathlineto{\pgfqpoint{1.787946in}{3.379960in}}%
\pgfpathlineto{\pgfqpoint{1.788439in}{2.621435in}}%
\pgfpathlineto{\pgfqpoint{1.788933in}{3.313887in}}%
\pgfpathlineto{\pgfqpoint{1.789426in}{3.238742in}}%
\pgfpathlineto{\pgfqpoint{1.789919in}{3.366256in}}%
\pgfpathlineto{\pgfqpoint{1.790412in}{3.199175in}}%
\pgfpathlineto{\pgfqpoint{1.790906in}{3.371018in}}%
\pgfpathlineto{\pgfqpoint{1.791892in}{3.379728in}}%
\pgfpathlineto{\pgfqpoint{1.792385in}{3.312388in}}%
\pgfpathlineto{\pgfqpoint{1.793372in}{3.313454in}}%
\pgfpathlineto{\pgfqpoint{1.793865in}{2.893209in}}%
\pgfpathlineto{\pgfqpoint{1.794358in}{3.378059in}}%
\pgfpathlineto{\pgfqpoint{1.794851in}{3.349291in}}%
\pgfpathlineto{\pgfqpoint{1.796331in}{3.021571in}}%
\pgfpathlineto{\pgfqpoint{1.797318in}{3.359111in}}%
\pgfpathlineto{\pgfqpoint{1.797811in}{3.074251in}}%
\pgfpathlineto{\pgfqpoint{1.798304in}{3.372749in}}%
\pgfpathlineto{\pgfqpoint{1.799290in}{3.371867in}}%
\pgfpathlineto{\pgfqpoint{1.800277in}{3.243589in}}%
\pgfpathlineto{\pgfqpoint{1.801263in}{3.338419in}}%
\pgfpathlineto{\pgfqpoint{1.801757in}{3.296587in}}%
\pgfpathlineto{\pgfqpoint{1.802250in}{3.372414in}}%
\pgfpathlineto{\pgfqpoint{1.802743in}{3.364970in}}%
\pgfpathlineto{\pgfqpoint{1.804223in}{3.242217in}}%
\pgfpathlineto{\pgfqpoint{1.804716in}{3.318821in}}%
\pgfpathlineto{\pgfqpoint{1.806196in}{2.642851in}}%
\pgfpathlineto{\pgfqpoint{1.807675in}{3.307686in}}%
\pgfpathlineto{\pgfqpoint{1.808169in}{2.671064in}}%
\pgfpathlineto{\pgfqpoint{1.808662in}{2.962016in}}%
\pgfpathlineto{\pgfqpoint{1.809648in}{2.962016in}}%
\pgfpathlineto{\pgfqpoint{1.810142in}{3.305880in}}%
\pgfpathlineto{\pgfqpoint{1.811128in}{3.267335in}}%
\pgfpathlineto{\pgfqpoint{1.811621in}{2.904523in}}%
\pgfpathlineto{\pgfqpoint{1.812115in}{3.038447in}}%
\pgfpathlineto{\pgfqpoint{1.813101in}{3.363464in}}%
\pgfpathlineto{\pgfqpoint{1.814087in}{3.077181in}}%
\pgfpathlineto{\pgfqpoint{1.814581in}{3.368747in}}%
\pgfpathlineto{\pgfqpoint{1.815074in}{3.161338in}}%
\pgfpathlineto{\pgfqpoint{1.815567in}{3.161338in}}%
\pgfpathlineto{\pgfqpoint{1.816060in}{2.469368in}}%
\pgfpathlineto{\pgfqpoint{1.817047in}{3.379600in}}%
\pgfpathlineto{\pgfqpoint{1.817540in}{3.379337in}}%
\pgfpathlineto{\pgfqpoint{1.818033in}{3.379337in}}%
\pgfpathlineto{\pgfqpoint{1.818527in}{3.337746in}}%
\pgfpathlineto{\pgfqpoint{1.819513in}{3.348124in}}%
\pgfpathlineto{\pgfqpoint{1.820006in}{3.354819in}}%
\pgfpathlineto{\pgfqpoint{1.820499in}{3.294733in}}%
\pgfpathlineto{\pgfqpoint{1.820993in}{3.098385in}}%
\pgfpathlineto{\pgfqpoint{1.821486in}{3.186816in}}%
\pgfpathlineto{\pgfqpoint{1.821979in}{3.186816in}}%
\pgfpathlineto{\pgfqpoint{1.822966in}{3.360864in}}%
\pgfpathlineto{\pgfqpoint{1.823459in}{3.301440in}}%
\pgfpathlineto{\pgfqpoint{1.823952in}{3.362141in}}%
\pgfpathlineto{\pgfqpoint{1.824445in}{3.362141in}}%
\pgfpathlineto{\pgfqpoint{1.824939in}{3.374988in}}%
\pgfpathlineto{\pgfqpoint{1.826911in}{3.018702in}}%
\pgfpathlineto{\pgfqpoint{1.827405in}{3.347876in}}%
\pgfpathlineto{\pgfqpoint{1.827898in}{3.106920in}}%
\pgfpathlineto{\pgfqpoint{1.828391in}{3.106920in}}%
\pgfpathlineto{\pgfqpoint{1.829871in}{3.365104in}}%
\pgfpathlineto{\pgfqpoint{1.830364in}{3.371807in}}%
\pgfpathlineto{\pgfqpoint{1.830857in}{3.063670in}}%
\pgfpathlineto{\pgfqpoint{1.831351in}{3.082520in}}%
\pgfpathlineto{\pgfqpoint{1.833323in}{3.377980in}}%
\pgfpathlineto{\pgfqpoint{1.833817in}{3.350093in}}%
\pgfpathlineto{\pgfqpoint{1.834310in}{3.347348in}}%
\pgfpathlineto{\pgfqpoint{1.835296in}{3.246653in}}%
\pgfpathlineto{\pgfqpoint{1.835790in}{3.313991in}}%
\pgfpathlineto{\pgfqpoint{1.836776in}{2.706114in}}%
\pgfpathlineto{\pgfqpoint{1.837763in}{3.248795in}}%
\pgfpathlineto{\pgfqpoint{1.838256in}{2.986531in}}%
\pgfpathlineto{\pgfqpoint{1.839242in}{3.038289in}}%
\pgfpathlineto{\pgfqpoint{1.840229in}{3.038289in}}%
\pgfpathlineto{\pgfqpoint{1.840722in}{3.032842in}}%
\pgfpathlineto{\pgfqpoint{1.841215in}{3.378279in}}%
\pgfpathlineto{\pgfqpoint{1.842202in}{3.342640in}}%
\pgfpathlineto{\pgfqpoint{1.842695in}{3.342640in}}%
\pgfpathlineto{\pgfqpoint{1.844668in}{3.376727in}}%
\pgfpathlineto{\pgfqpoint{1.845161in}{3.337080in}}%
\pgfpathlineto{\pgfqpoint{1.845654in}{3.354466in}}%
\pgfpathlineto{\pgfqpoint{1.846641in}{3.354466in}}%
\pgfpathlineto{\pgfqpoint{1.847134in}{3.091557in}}%
\pgfpathlineto{\pgfqpoint{1.847627in}{3.379297in}}%
\pgfpathlineto{\pgfqpoint{1.849107in}{3.315231in}}%
\pgfpathlineto{\pgfqpoint{1.849600in}{3.315231in}}%
\pgfpathlineto{\pgfqpoint{1.850093in}{3.306310in}}%
\pgfpathlineto{\pgfqpoint{1.851080in}{3.126534in}}%
\pgfpathlineto{\pgfqpoint{1.852066in}{3.373169in}}%
\pgfpathlineto{\pgfqpoint{1.854039in}{2.893242in}}%
\pgfpathlineto{\pgfqpoint{1.854532in}{2.893242in}}%
\pgfpathlineto{\pgfqpoint{1.855026in}{3.280472in}}%
\pgfpathlineto{\pgfqpoint{1.856012in}{3.245022in}}%
\pgfpathlineto{\pgfqpoint{1.856999in}{3.164493in}}%
\pgfpathlineto{\pgfqpoint{1.857492in}{3.216854in}}%
\pgfpathlineto{\pgfqpoint{1.858478in}{2.837327in}}%
\pgfpathlineto{\pgfqpoint{1.859958in}{3.312861in}}%
\pgfpathlineto{\pgfqpoint{1.860451in}{3.312861in}}%
\pgfpathlineto{\pgfqpoint{1.861438in}{3.361830in}}%
\pgfpathlineto{\pgfqpoint{1.861931in}{3.252155in}}%
\pgfpathlineto{\pgfqpoint{1.862424in}{2.810390in}}%
\pgfpathlineto{\pgfqpoint{1.862917in}{3.083340in}}%
\pgfpathlineto{\pgfqpoint{1.864397in}{3.346558in}}%
\pgfpathlineto{\pgfqpoint{1.866370in}{3.346558in}}%
\pgfpathlineto{\pgfqpoint{1.866863in}{3.351530in}}%
\pgfpathlineto{\pgfqpoint{1.867356in}{3.153025in}}%
\pgfpathlineto{\pgfqpoint{1.867850in}{3.376079in}}%
\pgfpathlineto{\pgfqpoint{1.868343in}{3.367120in}}%
\pgfpathlineto{\pgfqpoint{1.868836in}{3.377802in}}%
\pgfpathlineto{\pgfqpoint{1.871795in}{2.975205in}}%
\pgfpathlineto{\pgfqpoint{1.872782in}{3.114485in}}%
\pgfpathlineto{\pgfqpoint{1.874262in}{2.520847in}}%
\pgfpathlineto{\pgfqpoint{1.875741in}{3.379507in}}%
\pgfpathlineto{\pgfqpoint{1.876728in}{3.379507in}}%
\pgfpathlineto{\pgfqpoint{1.877714in}{3.153560in}}%
\pgfpathlineto{\pgfqpoint{1.879194in}{3.373166in}}%
\pgfpathlineto{\pgfqpoint{1.880180in}{3.275596in}}%
\pgfpathlineto{\pgfqpoint{1.881167in}{3.379985in}}%
\pgfpathlineto{\pgfqpoint{1.883633in}{2.320587in}}%
\pgfpathlineto{\pgfqpoint{1.884126in}{3.371450in}}%
\pgfpathlineto{\pgfqpoint{1.884619in}{2.215628in}}%
\pgfpathlineto{\pgfqpoint{1.885606in}{2.215628in}}%
\pgfpathlineto{\pgfqpoint{1.887086in}{3.239556in}}%
\pgfpathlineto{\pgfqpoint{1.888072in}{3.239556in}}%
\pgfpathlineto{\pgfqpoint{1.888565in}{3.330605in}}%
\pgfpathlineto{\pgfqpoint{1.890538in}{2.437623in}}%
\pgfpathlineto{\pgfqpoint{1.891525in}{3.342720in}}%
\pgfpathlineto{\pgfqpoint{1.892511in}{3.335998in}}%
\pgfpathlineto{\pgfqpoint{1.893498in}{3.326098in}}%
\pgfpathlineto{\pgfqpoint{1.894484in}{3.376722in}}%
\pgfpathlineto{\pgfqpoint{1.894977in}{3.374560in}}%
\pgfpathlineto{\pgfqpoint{1.895471in}{3.374560in}}%
\pgfpathlineto{\pgfqpoint{1.896457in}{2.780519in}}%
\pgfpathlineto{\pgfqpoint{1.897937in}{3.322154in}}%
\pgfpathlineto{\pgfqpoint{1.898430in}{3.289156in}}%
\pgfpathlineto{\pgfqpoint{1.899416in}{3.379853in}}%
\pgfpathlineto{\pgfqpoint{1.901389in}{3.012408in}}%
\pgfpathlineto{\pgfqpoint{1.902869in}{3.349772in}}%
\pgfpathlineto{\pgfqpoint{1.903855in}{3.349772in}}%
\pgfpathlineto{\pgfqpoint{1.904842in}{2.977219in}}%
\pgfpathlineto{\pgfqpoint{1.906322in}{3.326212in}}%
\pgfpathlineto{\pgfqpoint{1.907308in}{3.326212in}}%
\pgfpathlineto{\pgfqpoint{1.907801in}{2.936518in}}%
\pgfpathlineto{\pgfqpoint{1.908295in}{3.218867in}}%
\pgfpathlineto{\pgfqpoint{1.908788in}{3.144709in}}%
\pgfpathlineto{\pgfqpoint{1.909281in}{3.378150in}}%
\pgfpathlineto{\pgfqpoint{1.909774in}{3.317307in}}%
\pgfpathlineto{\pgfqpoint{1.911254in}{3.317307in}}%
\pgfpathlineto{\pgfqpoint{1.911747in}{2.706717in}}%
\pgfpathlineto{\pgfqpoint{1.912240in}{3.300982in}}%
\pgfpathlineto{\pgfqpoint{1.912734in}{3.151273in}}%
\pgfpathlineto{\pgfqpoint{1.913227in}{3.296873in}}%
\pgfpathlineto{\pgfqpoint{1.914213in}{3.303063in}}%
\pgfpathlineto{\pgfqpoint{1.914707in}{3.219594in}}%
\pgfpathlineto{\pgfqpoint{1.915200in}{3.328859in}}%
\pgfpathlineto{\pgfqpoint{1.916186in}{3.029718in}}%
\pgfpathlineto{\pgfqpoint{1.916680in}{3.379395in}}%
\pgfpathlineto{\pgfqpoint{1.917173in}{3.342092in}}%
\pgfpathlineto{\pgfqpoint{1.918652in}{3.169486in}}%
\pgfpathlineto{\pgfqpoint{1.919639in}{3.169486in}}%
\pgfpathlineto{\pgfqpoint{1.920625in}{3.197888in}}%
\pgfpathlineto{\pgfqpoint{1.922105in}{3.377732in}}%
\pgfpathlineto{\pgfqpoint{1.922598in}{3.377732in}}%
\pgfpathlineto{\pgfqpoint{1.924078in}{3.195888in}}%
\pgfpathlineto{\pgfqpoint{1.925064in}{3.126793in}}%
\pgfpathlineto{\pgfqpoint{1.926544in}{3.374326in}}%
\pgfpathlineto{\pgfqpoint{1.927037in}{3.021246in}}%
\pgfpathlineto{\pgfqpoint{1.927531in}{3.163775in}}%
\pgfpathlineto{\pgfqpoint{1.928517in}{3.360665in}}%
\pgfpathlineto{\pgfqpoint{1.929010in}{2.960984in}}%
\pgfpathlineto{\pgfqpoint{1.929504in}{3.308122in}}%
\pgfpathlineto{\pgfqpoint{1.930490in}{3.308122in}}%
\pgfpathlineto{\pgfqpoint{1.931476in}{3.359212in}}%
\pgfpathlineto{\pgfqpoint{1.931970in}{2.964908in}}%
\pgfpathlineto{\pgfqpoint{1.932463in}{3.111264in}}%
\pgfpathlineto{\pgfqpoint{1.933449in}{3.111264in}}%
\pgfpathlineto{\pgfqpoint{1.933943in}{2.718341in}}%
\pgfpathlineto{\pgfqpoint{1.934436in}{3.377920in}}%
\pgfpathlineto{\pgfqpoint{1.934929in}{3.034707in}}%
\pgfpathlineto{\pgfqpoint{1.935422in}{3.034707in}}%
\pgfpathlineto{\pgfqpoint{1.935916in}{2.512114in}}%
\pgfpathlineto{\pgfqpoint{1.936409in}{3.147338in}}%
\pgfpathlineto{\pgfqpoint{1.936902in}{3.121148in}}%
\pgfpathlineto{\pgfqpoint{1.937395in}{3.164330in}}%
\pgfpathlineto{\pgfqpoint{1.937888in}{3.363610in}}%
\pgfpathlineto{\pgfqpoint{1.938382in}{3.174174in}}%
\pgfpathlineto{\pgfqpoint{1.938875in}{3.285009in}}%
\pgfpathlineto{\pgfqpoint{1.939368in}{2.815294in}}%
\pgfpathlineto{\pgfqpoint{1.939861in}{3.165271in}}%
\pgfpathlineto{\pgfqpoint{1.940355in}{3.165271in}}%
\pgfpathlineto{\pgfqpoint{1.940848in}{2.871927in}}%
\pgfpathlineto{\pgfqpoint{1.941341in}{3.239429in}}%
\pgfpathlineto{\pgfqpoint{1.941834in}{3.057478in}}%
\pgfpathlineto{\pgfqpoint{1.943314in}{3.082193in}}%
\pgfpathlineto{\pgfqpoint{1.944794in}{3.379002in}}%
\pgfpathlineto{\pgfqpoint{1.946273in}{3.377488in}}%
\pgfpathlineto{\pgfqpoint{1.946767in}{2.429411in}}%
\pgfpathlineto{\pgfqpoint{1.947260in}{3.097997in}}%
\pgfpathlineto{\pgfqpoint{1.947753in}{3.097997in}}%
\pgfpathlineto{\pgfqpoint{1.949233in}{3.014705in}}%
\pgfpathlineto{\pgfqpoint{1.950219in}{3.139566in}}%
\pgfpathlineto{\pgfqpoint{1.950712in}{3.011989in}}%
\pgfpathlineto{\pgfqpoint{1.951206in}{3.064147in}}%
\pgfpathlineto{\pgfqpoint{1.952192in}{3.064147in}}%
\pgfpathlineto{\pgfqpoint{1.953672in}{3.074315in}}%
\pgfpathlineto{\pgfqpoint{1.954165in}{2.697572in}}%
\pgfpathlineto{\pgfqpoint{1.955152in}{3.267205in}}%
\pgfpathlineto{\pgfqpoint{1.955645in}{3.023255in}}%
\pgfpathlineto{\pgfqpoint{1.957124in}{3.373678in}}%
\pgfpathlineto{\pgfqpoint{1.957618in}{3.373678in}}%
\pgfpathlineto{\pgfqpoint{1.958111in}{3.103041in}}%
\pgfpathlineto{\pgfqpoint{1.958604in}{3.316440in}}%
\pgfpathlineto{\pgfqpoint{1.959097in}{3.333227in}}%
\pgfpathlineto{\pgfqpoint{1.959591in}{3.315891in}}%
\pgfpathlineto{\pgfqpoint{1.960084in}{2.611800in}}%
\pgfpathlineto{\pgfqpoint{1.960577in}{2.623092in}}%
\pgfpathlineto{\pgfqpoint{1.961564in}{3.379782in}}%
\pgfpathlineto{\pgfqpoint{1.962057in}{3.253330in}}%
\pgfpathlineto{\pgfqpoint{1.963536in}{2.584607in}}%
\pgfpathlineto{\pgfqpoint{1.965016in}{3.368502in}}%
\pgfpathlineto{\pgfqpoint{1.965509in}{3.368502in}}%
\pgfpathlineto{\pgfqpoint{1.966496in}{3.325097in}}%
\pgfpathlineto{\pgfqpoint{1.967976in}{2.655599in}}%
\pgfpathlineto{\pgfqpoint{1.968469in}{2.655599in}}%
\pgfpathlineto{\pgfqpoint{1.968962in}{3.197128in}}%
\pgfpathlineto{\pgfqpoint{1.969455in}{3.071552in}}%
\pgfpathlineto{\pgfqpoint{1.969948in}{2.412922in}}%
\pgfpathlineto{\pgfqpoint{1.970442in}{3.058661in}}%
\pgfpathlineto{\pgfqpoint{1.970935in}{3.058661in}}%
\pgfpathlineto{\pgfqpoint{1.971921in}{3.379644in}}%
\pgfpathlineto{\pgfqpoint{1.972415in}{3.250040in}}%
\pgfpathlineto{\pgfqpoint{1.972908in}{3.377957in}}%
\pgfpathlineto{\pgfqpoint{1.973894in}{3.358404in}}%
\pgfpathlineto{\pgfqpoint{1.974388in}{3.379779in}}%
\pgfpathlineto{\pgfqpoint{1.974881in}{3.379411in}}%
\pgfpathlineto{\pgfqpoint{1.976360in}{3.211974in}}%
\pgfpathlineto{\pgfqpoint{1.979320in}{3.378298in}}%
\pgfpathlineto{\pgfqpoint{1.980306in}{3.233261in}}%
\pgfpathlineto{\pgfqpoint{1.981786in}{3.379426in}}%
\pgfpathlineto{\pgfqpoint{1.982279in}{3.284231in}}%
\pgfpathlineto{\pgfqpoint{1.983266in}{3.293334in}}%
\pgfpathlineto{\pgfqpoint{1.984252in}{3.293334in}}%
\pgfpathlineto{\pgfqpoint{1.986225in}{3.075960in}}%
\pgfpathlineto{\pgfqpoint{1.986718in}{3.298223in}}%
\pgfpathlineto{\pgfqpoint{1.987212in}{3.179053in}}%
\pgfpathlineto{\pgfqpoint{1.987705in}{2.494040in}}%
\pgfpathlineto{\pgfqpoint{1.988691in}{2.616741in}}%
\pgfpathlineto{\pgfqpoint{1.989184in}{2.422637in}}%
\pgfpathlineto{\pgfqpoint{1.989678in}{3.305400in}}%
\pgfpathlineto{\pgfqpoint{1.990171in}{3.304175in}}%
\pgfpathlineto{\pgfqpoint{1.990664in}{2.892708in}}%
\pgfpathlineto{\pgfqpoint{1.991157in}{3.018625in}}%
\pgfpathlineto{\pgfqpoint{1.992637in}{3.376725in}}%
\pgfpathlineto{\pgfqpoint{1.995103in}{3.376725in}}%
\pgfpathlineto{\pgfqpoint{1.996090in}{3.356932in}}%
\pgfpathlineto{\pgfqpoint{1.996583in}{3.217749in}}%
\pgfpathlineto{\pgfqpoint{1.997076in}{3.362735in}}%
\pgfpathlineto{\pgfqpoint{1.998063in}{3.362735in}}%
\pgfpathlineto{\pgfqpoint{2.000529in}{3.377764in}}%
\pgfpathlineto{\pgfqpoint{2.002008in}{3.316469in}}%
\pgfpathlineto{\pgfqpoint{2.003488in}{3.337986in}}%
\pgfpathlineto{\pgfqpoint{2.004475in}{3.313198in}}%
\pgfpathlineto{\pgfqpoint{2.004968in}{3.379985in}}%
\pgfpathlineto{\pgfqpoint{2.005461in}{3.367441in}}%
\pgfpathlineto{\pgfqpoint{2.005954in}{3.354596in}}%
\pgfpathlineto{\pgfqpoint{2.006448in}{3.003340in}}%
\pgfpathlineto{\pgfqpoint{2.006941in}{3.361433in}}%
\pgfpathlineto{\pgfqpoint{2.009900in}{3.361433in}}%
\pgfpathlineto{\pgfqpoint{2.010393in}{2.894001in}}%
\pgfpathlineto{\pgfqpoint{2.010887in}{3.320093in}}%
\pgfpathlineto{\pgfqpoint{2.012366in}{3.320093in}}%
\pgfpathlineto{\pgfqpoint{2.013846in}{3.376533in}}%
\pgfpathlineto{\pgfqpoint{2.015326in}{3.331911in}}%
\pgfpathlineto{\pgfqpoint{2.016312in}{3.377604in}}%
\pgfpathlineto{\pgfqpoint{2.017792in}{3.203085in}}%
\pgfpathlineto{\pgfqpoint{2.017299in}{3.379125in}}%
\pgfpathlineto{\pgfqpoint{2.018285in}{3.232614in}}%
\pgfpathlineto{\pgfqpoint{2.018778in}{3.232614in}}%
\pgfpathlineto{\pgfqpoint{2.019272in}{3.379224in}}%
\pgfpathlineto{\pgfqpoint{2.020258in}{2.784871in}}%
\pgfpathlineto{\pgfqpoint{2.020751in}{3.371060in}}%
\pgfpathlineto{\pgfqpoint{2.021738in}{3.369264in}}%
\pgfpathlineto{\pgfqpoint{2.023217in}{3.161352in}}%
\pgfpathlineto{\pgfqpoint{2.024204in}{3.373129in}}%
\pgfpathlineto{\pgfqpoint{2.024697in}{3.337670in}}%
\pgfpathlineto{\pgfqpoint{2.026177in}{3.337670in}}%
\pgfpathlineto{\pgfqpoint{2.026670in}{2.908754in}}%
\pgfpathlineto{\pgfqpoint{2.027163in}{3.032924in}}%
\pgfpathlineto{\pgfqpoint{2.028643in}{3.295521in}}%
\pgfpathlineto{\pgfqpoint{2.029136in}{3.366347in}}%
\pgfpathlineto{\pgfqpoint{2.029629in}{3.358961in}}%
\pgfpathlineto{\pgfqpoint{2.031602in}{2.948259in}}%
\pgfpathlineto{\pgfqpoint{2.033082in}{3.288729in}}%
\pgfpathlineto{\pgfqpoint{2.034069in}{3.367787in}}%
\pgfpathlineto{\pgfqpoint{2.036041in}{2.891507in}}%
\pgfpathlineto{\pgfqpoint{2.036535in}{2.919374in}}%
\pgfpathlineto{\pgfqpoint{2.037521in}{3.363559in}}%
\pgfpathlineto{\pgfqpoint{2.038014in}{3.306040in}}%
\pgfpathlineto{\pgfqpoint{2.039001in}{3.302911in}}%
\pgfpathlineto{\pgfqpoint{2.039987in}{2.371885in}}%
\pgfpathlineto{\pgfqpoint{2.041467in}{3.379895in}}%
\pgfpathlineto{\pgfqpoint{2.042453in}{3.379986in}}%
\pgfpathlineto{\pgfqpoint{2.042947in}{2.586261in}}%
\pgfpathlineto{\pgfqpoint{2.043440in}{2.897979in}}%
\pgfpathlineto{\pgfqpoint{2.045413in}{3.336986in}}%
\pgfpathlineto{\pgfqpoint{2.046893in}{2.769625in}}%
\pgfpathlineto{\pgfqpoint{2.049852in}{3.351633in}}%
\pgfpathlineto{\pgfqpoint{2.050838in}{3.351633in}}%
\pgfpathlineto{\pgfqpoint{2.051332in}{2.668790in}}%
\pgfpathlineto{\pgfqpoint{2.051825in}{2.735562in}}%
\pgfpathlineto{\pgfqpoint{2.052811in}{3.366817in}}%
\pgfpathlineto{\pgfqpoint{2.053305in}{2.823848in}}%
\pgfpathlineto{\pgfqpoint{2.053798in}{3.340979in}}%
\pgfpathlineto{\pgfqpoint{2.054291in}{3.340979in}}%
\pgfpathlineto{\pgfqpoint{2.054784in}{2.980371in}}%
\pgfpathlineto{\pgfqpoint{2.055277in}{3.209231in}}%
\pgfpathlineto{\pgfqpoint{2.055771in}{3.209231in}}%
\pgfpathlineto{\pgfqpoint{2.056264in}{3.099188in}}%
\pgfpathlineto{\pgfqpoint{2.056757in}{2.489116in}}%
\pgfpathlineto{\pgfqpoint{2.057250in}{3.196197in}}%
\pgfpathlineto{\pgfqpoint{2.058237in}{3.314918in}}%
\pgfpathlineto{\pgfqpoint{2.058730in}{3.228574in}}%
\pgfpathlineto{\pgfqpoint{2.059223in}{3.263062in}}%
\pgfpathlineto{\pgfqpoint{2.059717in}{3.276252in}}%
\pgfpathlineto{\pgfqpoint{2.060210in}{3.335947in}}%
\pgfpathlineto{\pgfqpoint{2.062676in}{2.772852in}}%
\pgfpathlineto{\pgfqpoint{2.063169in}{2.795454in}}%
\pgfpathlineto{\pgfqpoint{2.064156in}{3.362006in}}%
\pgfpathlineto{\pgfqpoint{2.064649in}{3.288877in}}%
\pgfpathlineto{\pgfqpoint{2.067115in}{3.288877in}}%
\pgfpathlineto{\pgfqpoint{2.067608in}{3.315605in}}%
\pgfpathlineto{\pgfqpoint{2.068101in}{3.249231in}}%
\pgfpathlineto{\pgfqpoint{2.068595in}{3.329017in}}%
\pgfpathlineto{\pgfqpoint{2.069088in}{3.329017in}}%
\pgfpathlineto{\pgfqpoint{2.070568in}{3.074667in}}%
\pgfpathlineto{\pgfqpoint{2.071554in}{3.379702in}}%
\pgfpathlineto{\pgfqpoint{2.072047in}{3.279552in}}%
\pgfpathlineto{\pgfqpoint{2.074020in}{3.279552in}}%
\pgfpathlineto{\pgfqpoint{2.074513in}{2.970298in}}%
\pgfpathlineto{\pgfqpoint{2.075007in}{3.350868in}}%
\pgfpathlineto{\pgfqpoint{2.076980in}{3.350868in}}%
\pgfpathlineto{\pgfqpoint{2.077473in}{2.460878in}}%
\pgfpathlineto{\pgfqpoint{2.077966in}{2.793060in}}%
\pgfpathlineto{\pgfqpoint{2.079446in}{3.308310in}}%
\pgfpathlineto{\pgfqpoint{2.080432in}{3.308310in}}%
\pgfpathlineto{\pgfqpoint{2.081912in}{3.265203in}}%
\pgfpathlineto{\pgfqpoint{2.082405in}{3.265203in}}%
\pgfpathlineto{\pgfqpoint{2.083885in}{3.264017in}}%
\pgfpathlineto{\pgfqpoint{2.084871in}{3.264017in}}%
\pgfpathlineto{\pgfqpoint{2.086351in}{3.312514in}}%
\pgfpathlineto{\pgfqpoint{2.088817in}{2.915700in}}%
\pgfpathlineto{\pgfqpoint{2.089310in}{3.246670in}}%
\pgfpathlineto{\pgfqpoint{2.089804in}{3.026588in}}%
\pgfpathlineto{\pgfqpoint{2.090297in}{2.777587in}}%
\pgfpathlineto{\pgfqpoint{2.091777in}{3.286675in}}%
\pgfpathlineto{\pgfqpoint{2.092763in}{3.111811in}}%
\pgfpathlineto{\pgfqpoint{2.094243in}{3.341125in}}%
\pgfpathlineto{\pgfqpoint{2.095229in}{3.251092in}}%
\pgfpathlineto{\pgfqpoint{2.096709in}{3.374104in}}%
\pgfpathlineto{\pgfqpoint{2.097695in}{3.374104in}}%
\pgfpathlineto{\pgfqpoint{2.098189in}{3.362584in}}%
\pgfpathlineto{\pgfqpoint{2.100655in}{3.049974in}}%
\pgfpathlineto{\pgfqpoint{2.101641in}{3.303214in}}%
\pgfpathlineto{\pgfqpoint{2.102134in}{3.143170in}}%
\pgfpathlineto{\pgfqpoint{2.103121in}{3.379778in}}%
\pgfpathlineto{\pgfqpoint{2.104601in}{2.780938in}}%
\pgfpathlineto{\pgfqpoint{2.106080in}{3.244062in}}%
\pgfpathlineto{\pgfqpoint{2.106573in}{3.244062in}}%
\pgfpathlineto{\pgfqpoint{2.108053in}{2.841499in}}%
\pgfpathlineto{\pgfqpoint{2.108546in}{2.841499in}}%
\pgfpathlineto{\pgfqpoint{2.109533in}{3.345797in}}%
\pgfpathlineto{\pgfqpoint{2.111013in}{1.332739in}}%
\pgfpathlineto{\pgfqpoint{2.112492in}{3.147520in}}%
\pgfpathlineto{\pgfqpoint{2.113972in}{3.286148in}}%
\pgfpathlineto{\pgfqpoint{2.114465in}{2.832416in}}%
\pgfpathlineto{\pgfqpoint{2.114958in}{3.323349in}}%
\pgfpathlineto{\pgfqpoint{2.115452in}{3.323349in}}%
\pgfpathlineto{\pgfqpoint{2.115945in}{3.032702in}}%
\pgfpathlineto{\pgfqpoint{2.116438in}{3.341650in}}%
\pgfpathlineto{\pgfqpoint{2.118411in}{3.191325in}}%
\pgfpathlineto{\pgfqpoint{2.119891in}{3.341346in}}%
\pgfpathlineto{\pgfqpoint{2.120384in}{2.819483in}}%
\pgfpathlineto{\pgfqpoint{2.120877in}{2.957197in}}%
\pgfpathlineto{\pgfqpoint{2.121370in}{3.363808in}}%
\pgfpathlineto{\pgfqpoint{2.121864in}{3.309131in}}%
\pgfpathlineto{\pgfqpoint{2.123343in}{2.987444in}}%
\pgfpathlineto{\pgfqpoint{2.124330in}{3.237931in}}%
\pgfpathlineto{\pgfqpoint{2.125810in}{2.947452in}}%
\pgfpathlineto{\pgfqpoint{2.126303in}{1.375625in}}%
\pgfpathlineto{\pgfqpoint{2.126796in}{2.945522in}}%
\pgfpathlineto{\pgfqpoint{2.127782in}{3.379288in}}%
\pgfpathlineto{\pgfqpoint{2.128276in}{3.376290in}}%
\pgfpathlineto{\pgfqpoint{2.130249in}{3.376290in}}%
\pgfpathlineto{\pgfqpoint{2.130742in}{3.379184in}}%
\pgfpathlineto{\pgfqpoint{2.132222in}{3.371829in}}%
\pgfpathlineto{\pgfqpoint{2.132715in}{3.371829in}}%
\pgfpathlineto{\pgfqpoint{2.133208in}{3.368404in}}%
\pgfpathlineto{\pgfqpoint{2.133701in}{2.764695in}}%
\pgfpathlineto{\pgfqpoint{2.134194in}{3.349870in}}%
\pgfpathlineto{\pgfqpoint{2.134688in}{3.349870in}}%
\pgfpathlineto{\pgfqpoint{2.135674in}{3.271695in}}%
\pgfpathlineto{\pgfqpoint{2.136661in}{3.351960in}}%
\pgfpathlineto{\pgfqpoint{2.137647in}{3.191971in}}%
\pgfpathlineto{\pgfqpoint{2.138634in}{3.378799in}}%
\pgfpathlineto{\pgfqpoint{2.139127in}{3.343387in}}%
\pgfpathlineto{\pgfqpoint{2.141593in}{3.343387in}}%
\pgfpathlineto{\pgfqpoint{2.142579in}{3.160182in}}%
\pgfpathlineto{\pgfqpoint{2.143566in}{3.310937in}}%
\pgfpathlineto{\pgfqpoint{2.144552in}{3.206566in}}%
\pgfpathlineto{\pgfqpoint{2.145046in}{3.213375in}}%
\pgfpathlineto{\pgfqpoint{2.146525in}{3.379995in}}%
\pgfpathlineto{\pgfqpoint{2.147018in}{3.379995in}}%
\pgfpathlineto{\pgfqpoint{2.148498in}{3.034842in}}%
\pgfpathlineto{\pgfqpoint{2.148991in}{3.034842in}}%
\pgfpathlineto{\pgfqpoint{2.149978in}{3.302875in}}%
\pgfpathlineto{\pgfqpoint{2.150471in}{2.955743in}}%
\pgfpathlineto{\pgfqpoint{2.150964in}{3.255772in}}%
\pgfpathlineto{\pgfqpoint{2.151458in}{3.146849in}}%
\pgfpathlineto{\pgfqpoint{2.151951in}{3.344978in}}%
\pgfpathlineto{\pgfqpoint{2.152444in}{3.130501in}}%
\pgfpathlineto{\pgfqpoint{2.153430in}{3.099850in}}%
\pgfpathlineto{\pgfqpoint{2.154910in}{3.366603in}}%
\pgfpathlineto{\pgfqpoint{2.156390in}{3.366603in}}%
\pgfpathlineto{\pgfqpoint{2.158363in}{3.028915in}}%
\pgfpathlineto{\pgfqpoint{2.158856in}{3.339660in}}%
\pgfpathlineto{\pgfqpoint{2.159842in}{3.273186in}}%
\pgfpathlineto{\pgfqpoint{2.160336in}{3.311235in}}%
\pgfpathlineto{\pgfqpoint{2.160829in}{3.191983in}}%
\pgfpathlineto{\pgfqpoint{2.161815in}{3.214909in}}%
\pgfpathlineto{\pgfqpoint{2.163295in}{3.376556in}}%
\pgfpathlineto{\pgfqpoint{2.163788in}{3.376556in}}%
\pgfpathlineto{\pgfqpoint{2.164775in}{3.251582in}}%
\pgfpathlineto{\pgfqpoint{2.166254in}{3.318020in}}%
\pgfpathlineto{\pgfqpoint{2.167241in}{3.341863in}}%
\pgfpathlineto{\pgfqpoint{2.167734in}{3.318086in}}%
\pgfpathlineto{\pgfqpoint{2.169214in}{3.022427in}}%
\pgfpathlineto{\pgfqpoint{2.169707in}{3.022427in}}%
\pgfpathlineto{\pgfqpoint{2.170200in}{2.866397in}}%
\pgfpathlineto{\pgfqpoint{2.170694in}{3.235555in}}%
\pgfpathlineto{\pgfqpoint{2.171187in}{2.330356in}}%
\pgfpathlineto{\pgfqpoint{2.171680in}{3.378251in}}%
\pgfpathlineto{\pgfqpoint{2.172666in}{3.378251in}}%
\pgfpathlineto{\pgfqpoint{2.174146in}{3.218117in}}%
\pgfpathlineto{\pgfqpoint{2.174639in}{3.362303in}}%
\pgfpathlineto{\pgfqpoint{2.175133in}{3.095697in}}%
\pgfpathlineto{\pgfqpoint{2.175626in}{3.279747in}}%
\pgfpathlineto{\pgfqpoint{2.177106in}{3.379999in}}%
\pgfpathlineto{\pgfqpoint{2.178092in}{3.379999in}}%
\pgfpathlineto{\pgfqpoint{2.180065in}{3.362251in}}%
\pgfpathlineto{\pgfqpoint{2.180558in}{3.091315in}}%
\pgfpathlineto{\pgfqpoint{2.181051in}{3.246022in}}%
\pgfpathlineto{\pgfqpoint{2.182038in}{3.246022in}}%
\pgfpathlineto{\pgfqpoint{2.183518in}{3.344620in}}%
\pgfpathlineto{\pgfqpoint{2.184011in}{3.344620in}}%
\pgfpathlineto{\pgfqpoint{2.185490in}{3.305804in}}%
\pgfpathlineto{\pgfqpoint{2.185984in}{3.378662in}}%
\pgfpathlineto{\pgfqpoint{2.187957in}{3.138496in}}%
\pgfpathlineto{\pgfqpoint{2.188450in}{3.138496in}}%
\pgfpathlineto{\pgfqpoint{2.189436in}{3.379778in}}%
\pgfpathlineto{\pgfqpoint{2.189930in}{3.031404in}}%
\pgfpathlineto{\pgfqpoint{2.190423in}{3.277171in}}%
\pgfpathlineto{\pgfqpoint{2.192889in}{3.003728in}}%
\pgfpathlineto{\pgfqpoint{2.193875in}{3.378597in}}%
\pgfpathlineto{\pgfqpoint{2.194862in}{3.025811in}}%
\pgfpathlineto{\pgfqpoint{2.196342in}{3.380000in}}%
\pgfpathlineto{\pgfqpoint{2.198314in}{3.380000in}}%
\pgfpathlineto{\pgfqpoint{2.198808in}{3.316610in}}%
\pgfpathlineto{\pgfqpoint{2.199301in}{3.375346in}}%
\pgfpathlineto{\pgfqpoint{2.199794in}{2.753460in}}%
\pgfpathlineto{\pgfqpoint{2.200287in}{3.315538in}}%
\pgfpathlineto{\pgfqpoint{2.201767in}{3.279874in}}%
\pgfpathlineto{\pgfqpoint{2.202260in}{3.377214in}}%
\pgfpathlineto{\pgfqpoint{2.203247in}{3.376483in}}%
\pgfpathlineto{\pgfqpoint{2.203740in}{2.389490in}}%
\pgfpathlineto{\pgfqpoint{2.204233in}{3.259763in}}%
\pgfpathlineto{\pgfqpoint{2.205220in}{3.259763in}}%
\pgfpathlineto{\pgfqpoint{2.206206in}{3.372186in}}%
\pgfpathlineto{\pgfqpoint{2.207193in}{3.365955in}}%
\pgfpathlineto{\pgfqpoint{2.207686in}{3.365955in}}%
\pgfpathlineto{\pgfqpoint{2.208672in}{3.298498in}}%
\pgfpathlineto{\pgfqpoint{2.209659in}{3.360584in}}%
\pgfpathlineto{\pgfqpoint{2.210152in}{3.356868in}}%
\pgfpathlineto{\pgfqpoint{2.210645in}{3.090027in}}%
\pgfpathlineto{\pgfqpoint{2.211139in}{3.335431in}}%
\pgfpathlineto{\pgfqpoint{2.211632in}{3.335431in}}%
\pgfpathlineto{\pgfqpoint{2.213111in}{3.379848in}}%
\pgfpathlineto{\pgfqpoint{2.215084in}{3.127656in}}%
\pgfpathlineto{\pgfqpoint{2.216071in}{3.377991in}}%
\pgfpathlineto{\pgfqpoint{2.217551in}{2.738821in}}%
\pgfpathlineto{\pgfqpoint{2.219030in}{3.371696in}}%
\pgfpathlineto{\pgfqpoint{2.219523in}{3.216959in}}%
\pgfpathlineto{\pgfqpoint{2.220017in}{2.756792in}}%
\pgfpathlineto{\pgfqpoint{2.220510in}{3.374975in}}%
\pgfpathlineto{\pgfqpoint{2.221003in}{3.356690in}}%
\pgfpathlineto{\pgfqpoint{2.221496in}{2.884204in}}%
\pgfpathlineto{\pgfqpoint{2.221990in}{3.372772in}}%
\pgfpathlineto{\pgfqpoint{2.222483in}{3.362670in}}%
\pgfpathlineto{\pgfqpoint{2.222976in}{3.372083in}}%
\pgfpathlineto{\pgfqpoint{2.224456in}{3.372083in}}%
\pgfpathlineto{\pgfqpoint{2.225442in}{3.330304in}}%
\pgfpathlineto{\pgfqpoint{2.226429in}{2.548563in}}%
\pgfpathlineto{\pgfqpoint{2.227908in}{3.379961in}}%
\pgfpathlineto{\pgfqpoint{2.229388in}{3.226615in}}%
\pgfpathlineto{\pgfqpoint{2.230868in}{3.296977in}}%
\pgfpathlineto{\pgfqpoint{2.231361in}{3.210254in}}%
\pgfpathlineto{\pgfqpoint{2.232841in}{3.379999in}}%
\pgfpathlineto{\pgfqpoint{2.233334in}{3.379999in}}%
\pgfpathlineto{\pgfqpoint{2.234320in}{3.079674in}}%
\pgfpathlineto{\pgfqpoint{2.234814in}{3.376565in}}%
\pgfpathlineto{\pgfqpoint{2.235307in}{3.221159in}}%
\pgfpathlineto{\pgfqpoint{2.235800in}{3.221159in}}%
\pgfpathlineto{\pgfqpoint{2.236293in}{2.714564in}}%
\pgfpathlineto{\pgfqpoint{2.236787in}{3.265470in}}%
\pgfpathlineto{\pgfqpoint{2.237280in}{2.785040in}}%
\pgfpathlineto{\pgfqpoint{2.237773in}{2.785040in}}%
\pgfpathlineto{\pgfqpoint{2.239253in}{3.127929in}}%
\pgfpathlineto{\pgfqpoint{2.239746in}{3.371594in}}%
\pgfpathlineto{\pgfqpoint{2.240239in}{2.323348in}}%
\pgfpathlineto{\pgfqpoint{2.240732in}{3.080048in}}%
\pgfpathlineto{\pgfqpoint{2.241226in}{3.080048in}}%
\pgfpathlineto{\pgfqpoint{2.242705in}{1.840707in}}%
\pgfpathlineto{\pgfqpoint{2.243199in}{3.157223in}}%
\pgfpathlineto{\pgfqpoint{2.243692in}{2.900346in}}%
\pgfpathlineto{\pgfqpoint{2.244185in}{2.651933in}}%
\pgfpathlineto{\pgfqpoint{2.245665in}{3.101507in}}%
\pgfpathlineto{\pgfqpoint{2.246158in}{3.101507in}}%
\pgfpathlineto{\pgfqpoint{2.247144in}{3.320101in}}%
\pgfpathlineto{\pgfqpoint{2.247638in}{3.308896in}}%
\pgfpathlineto{\pgfqpoint{2.248131in}{3.308896in}}%
\pgfpathlineto{\pgfqpoint{2.249611in}{3.364446in}}%
\pgfpathlineto{\pgfqpoint{2.250104in}{3.127472in}}%
\pgfpathlineto{\pgfqpoint{2.250597in}{3.294412in}}%
\pgfpathlineto{\pgfqpoint{2.252077in}{3.368762in}}%
\pgfpathlineto{\pgfqpoint{2.253556in}{3.368762in}}%
\pgfpathlineto{\pgfqpoint{2.255529in}{3.379999in}}%
\pgfpathlineto{\pgfqpoint{2.256516in}{3.336882in}}%
\pgfpathlineto{\pgfqpoint{2.257995in}{3.365426in}}%
\pgfpathlineto{\pgfqpoint{2.258489in}{3.207119in}}%
\pgfpathlineto{\pgfqpoint{2.258982in}{2.450865in}}%
\pgfpathlineto{\pgfqpoint{2.259475in}{3.145966in}}%
\pgfpathlineto{\pgfqpoint{2.260955in}{3.145966in}}%
\pgfpathlineto{\pgfqpoint{2.261448in}{3.032647in}}%
\pgfpathlineto{\pgfqpoint{2.262435in}{3.268381in}}%
\pgfpathlineto{\pgfqpoint{2.262928in}{2.209578in}}%
\pgfpathlineto{\pgfqpoint{2.263914in}{2.256028in}}%
\pgfpathlineto{\pgfqpoint{2.264901in}{3.346545in}}%
\pgfpathlineto{\pgfqpoint{2.265394in}{2.475906in}}%
\pgfpathlineto{\pgfqpoint{2.265887in}{2.800675in}}%
\pgfpathlineto{\pgfqpoint{2.266874in}{3.218895in}}%
\pgfpathlineto{\pgfqpoint{2.268353in}{2.941448in}}%
\pgfpathlineto{\pgfqpoint{2.269340in}{3.353192in}}%
\pgfpathlineto{\pgfqpoint{2.269833in}{3.206625in}}%
\pgfpathlineto{\pgfqpoint{2.270326in}{3.196659in}}%
\pgfpathlineto{\pgfqpoint{2.271806in}{3.355726in}}%
\pgfpathlineto{\pgfqpoint{2.272299in}{3.355726in}}%
\pgfpathlineto{\pgfqpoint{2.273779in}{2.852586in}}%
\pgfpathlineto{\pgfqpoint{2.275752in}{3.369733in}}%
\pgfpathlineto{\pgfqpoint{2.278711in}{3.369733in}}%
\pgfpathlineto{\pgfqpoint{2.279204in}{3.354887in}}%
\pgfpathlineto{\pgfqpoint{2.279698in}{3.377593in}}%
\pgfpathlineto{\pgfqpoint{2.280684in}{3.377494in}}%
\pgfpathlineto{\pgfqpoint{2.281177in}{3.377494in}}%
\pgfpathlineto{\pgfqpoint{2.281671in}{3.264744in}}%
\pgfpathlineto{\pgfqpoint{2.282164in}{2.550257in}}%
\pgfpathlineto{\pgfqpoint{2.282657in}{3.369740in}}%
\pgfpathlineto{\pgfqpoint{2.283150in}{3.359886in}}%
\pgfpathlineto{\pgfqpoint{2.284630in}{3.072016in}}%
\pgfpathlineto{\pgfqpoint{2.286110in}{3.373926in}}%
\pgfpathlineto{\pgfqpoint{2.287096in}{3.255521in}}%
\pgfpathlineto{\pgfqpoint{2.288576in}{3.361807in}}%
\pgfpathlineto{\pgfqpoint{2.289562in}{3.369372in}}%
\pgfpathlineto{\pgfqpoint{2.290549in}{3.365370in}}%
\pgfpathlineto{\pgfqpoint{2.291042in}{3.365821in}}%
\pgfpathlineto{\pgfqpoint{2.291535in}{3.336720in}}%
\pgfpathlineto{\pgfqpoint{2.293015in}{2.616084in}}%
\pgfpathlineto{\pgfqpoint{2.294001in}{3.379797in}}%
\pgfpathlineto{\pgfqpoint{2.294988in}{3.306577in}}%
\pgfpathlineto{\pgfqpoint{2.295481in}{3.369016in}}%
\pgfpathlineto{\pgfqpoint{2.296467in}{3.365156in}}%
\pgfpathlineto{\pgfqpoint{2.297454in}{3.365156in}}%
\pgfpathlineto{\pgfqpoint{2.297947in}{3.224251in}}%
\pgfpathlineto{\pgfqpoint{2.298440in}{3.343551in}}%
\pgfpathlineto{\pgfqpoint{2.299920in}{3.378985in}}%
\pgfpathlineto{\pgfqpoint{2.301893in}{3.346868in}}%
\pgfpathlineto{\pgfqpoint{2.302879in}{3.371634in}}%
\pgfpathlineto{\pgfqpoint{2.304359in}{3.034895in}}%
\pgfpathlineto{\pgfqpoint{2.304852in}{2.883144in}}%
\pgfpathlineto{\pgfqpoint{2.305839in}{3.379823in}}%
\pgfpathlineto{\pgfqpoint{2.306825in}{2.965684in}}%
\pgfpathlineto{\pgfqpoint{2.307319in}{3.050469in}}%
\pgfpathlineto{\pgfqpoint{2.307812in}{3.217892in}}%
\pgfpathlineto{\pgfqpoint{2.308798in}{2.970884in}}%
\pgfpathlineto{\pgfqpoint{2.310771in}{3.299941in}}%
\pgfpathlineto{\pgfqpoint{2.311264in}{2.541124in}}%
\pgfpathlineto{\pgfqpoint{2.311758in}{3.379999in}}%
\pgfpathlineto{\pgfqpoint{2.314224in}{3.379999in}}%
\pgfpathlineto{\pgfqpoint{2.315704in}{3.346315in}}%
\pgfpathlineto{\pgfqpoint{2.316690in}{3.346315in}}%
\pgfpathlineto{\pgfqpoint{2.317183in}{2.554865in}}%
\pgfpathlineto{\pgfqpoint{2.317676in}{3.294870in}}%
\pgfpathlineto{\pgfqpoint{2.318170in}{3.294870in}}%
\pgfpathlineto{\pgfqpoint{2.319649in}{3.377769in}}%
\pgfpathlineto{\pgfqpoint{2.320636in}{3.126414in}}%
\pgfpathlineto{\pgfqpoint{2.321129in}{3.263633in}}%
\pgfpathlineto{\pgfqpoint{2.321622in}{3.306284in}}%
\pgfpathlineto{\pgfqpoint{2.322116in}{3.204618in}}%
\pgfpathlineto{\pgfqpoint{2.322609in}{3.298864in}}%
\pgfpathlineto{\pgfqpoint{2.324088in}{3.352852in}}%
\pgfpathlineto{\pgfqpoint{2.326061in}{3.352852in}}%
\pgfpathlineto{\pgfqpoint{2.327048in}{3.350729in}}%
\pgfpathlineto{\pgfqpoint{2.327541in}{3.379203in}}%
\pgfpathlineto{\pgfqpoint{2.328528in}{2.806621in}}%
\pgfpathlineto{\pgfqpoint{2.330007in}{3.361830in}}%
\pgfpathlineto{\pgfqpoint{2.330500in}{3.361830in}}%
\pgfpathlineto{\pgfqpoint{2.331487in}{2.963486in}}%
\pgfpathlineto{\pgfqpoint{2.331980in}{3.342530in}}%
\pgfpathlineto{\pgfqpoint{2.332967in}{3.337572in}}%
\pgfpathlineto{\pgfqpoint{2.333460in}{3.379888in}}%
\pgfpathlineto{\pgfqpoint{2.333953in}{3.288176in}}%
\pgfpathlineto{\pgfqpoint{2.334446in}{3.359711in}}%
\pgfpathlineto{\pgfqpoint{2.334940in}{3.359711in}}%
\pgfpathlineto{\pgfqpoint{2.335433in}{3.367041in}}%
\pgfpathlineto{\pgfqpoint{2.335926in}{2.907080in}}%
\pgfpathlineto{\pgfqpoint{2.336419in}{3.339788in}}%
\pgfpathlineto{\pgfqpoint{2.337406in}{3.220928in}}%
\pgfpathlineto{\pgfqpoint{2.338885in}{3.374556in}}%
\pgfpathlineto{\pgfqpoint{2.339379in}{3.374556in}}%
\pgfpathlineto{\pgfqpoint{2.340858in}{3.378223in}}%
\pgfpathlineto{\pgfqpoint{2.341352in}{3.075605in}}%
\pgfpathlineto{\pgfqpoint{2.341845in}{3.196373in}}%
\pgfpathlineto{\pgfqpoint{2.342338in}{3.196373in}}%
\pgfpathlineto{\pgfqpoint{2.342831in}{3.017721in}}%
\pgfpathlineto{\pgfqpoint{2.343324in}{1.946064in}}%
\pgfpathlineto{\pgfqpoint{2.344804in}{3.376866in}}%
\pgfpathlineto{\pgfqpoint{2.345297in}{3.376866in}}%
\pgfpathlineto{\pgfqpoint{2.346777in}{3.370175in}}%
\pgfpathlineto{\pgfqpoint{2.347764in}{2.998062in}}%
\pgfpathlineto{\pgfqpoint{2.348750in}{3.379494in}}%
\pgfpathlineto{\pgfqpoint{2.349243in}{3.354944in}}%
\pgfpathlineto{\pgfqpoint{2.350230in}{3.165869in}}%
\pgfpathlineto{\pgfqpoint{2.351216in}{3.367013in}}%
\pgfpathlineto{\pgfqpoint{2.353189in}{2.433027in}}%
\pgfpathlineto{\pgfqpoint{2.354669in}{3.315113in}}%
\pgfpathlineto{\pgfqpoint{2.355655in}{3.315113in}}%
\pgfpathlineto{\pgfqpoint{2.357135in}{3.310171in}}%
\pgfpathlineto{\pgfqpoint{2.357628in}{3.310171in}}%
\pgfpathlineto{\pgfqpoint{2.359108in}{2.949686in}}%
\pgfpathlineto{\pgfqpoint{2.359601in}{2.949686in}}%
\pgfpathlineto{\pgfqpoint{2.360094in}{3.315117in}}%
\pgfpathlineto{\pgfqpoint{2.360588in}{3.211256in}}%
\pgfpathlineto{\pgfqpoint{2.361081in}{3.215588in}}%
\pgfpathlineto{\pgfqpoint{2.361574in}{3.346932in}}%
\pgfpathlineto{\pgfqpoint{2.362067in}{3.202249in}}%
\pgfpathlineto{\pgfqpoint{2.363054in}{2.614352in}}%
\pgfpathlineto{\pgfqpoint{2.364533in}{3.295878in}}%
\pgfpathlineto{\pgfqpoint{2.365027in}{3.342808in}}%
\pgfpathlineto{\pgfqpoint{2.365520in}{3.208997in}}%
\pgfpathlineto{\pgfqpoint{2.366013in}{3.318979in}}%
\pgfpathlineto{\pgfqpoint{2.367493in}{3.341691in}}%
\pgfpathlineto{\pgfqpoint{2.367986in}{2.734555in}}%
\pgfpathlineto{\pgfqpoint{2.368479in}{2.866402in}}%
\pgfpathlineto{\pgfqpoint{2.369959in}{3.379831in}}%
\pgfpathlineto{\pgfqpoint{2.370452in}{3.249117in}}%
\pgfpathlineto{\pgfqpoint{2.371932in}{3.156068in}}%
\pgfpathlineto{\pgfqpoint{2.373412in}{3.356896in}}%
\pgfpathlineto{\pgfqpoint{2.373905in}{3.356896in}}%
\pgfpathlineto{\pgfqpoint{2.374891in}{2.981059in}}%
\pgfpathlineto{\pgfqpoint{2.376371in}{3.202762in}}%
\pgfpathlineto{\pgfqpoint{2.377357in}{3.202762in}}%
\pgfpathlineto{\pgfqpoint{2.378837in}{3.101880in}}%
\pgfpathlineto{\pgfqpoint{2.379824in}{3.371601in}}%
\pgfpathlineto{\pgfqpoint{2.380810in}{2.539207in}}%
\pgfpathlineto{\pgfqpoint{2.382290in}{3.367610in}}%
\pgfpathlineto{\pgfqpoint{2.383276in}{3.322119in}}%
\pgfpathlineto{\pgfqpoint{2.384756in}{3.379526in}}%
\pgfpathlineto{\pgfqpoint{2.386236in}{3.379526in}}%
\pgfpathlineto{\pgfqpoint{2.387715in}{3.226565in}}%
\pgfpathlineto{\pgfqpoint{2.388208in}{3.226565in}}%
\pgfpathlineto{\pgfqpoint{2.389195in}{2.360128in}}%
\pgfpathlineto{\pgfqpoint{2.390181in}{3.338433in}}%
\pgfpathlineto{\pgfqpoint{2.390675in}{2.861622in}}%
\pgfpathlineto{\pgfqpoint{2.391168in}{3.363895in}}%
\pgfpathlineto{\pgfqpoint{2.392154in}{3.363895in}}%
\pgfpathlineto{\pgfqpoint{2.393141in}{3.194942in}}%
\pgfpathlineto{\pgfqpoint{2.393634in}{3.261149in}}%
\pgfpathlineto{\pgfqpoint{2.394127in}{3.339889in}}%
\pgfpathlineto{\pgfqpoint{2.396593in}{2.906072in}}%
\pgfpathlineto{\pgfqpoint{2.399060in}{3.370630in}}%
\pgfpathlineto{\pgfqpoint{2.400539in}{3.141131in}}%
\pgfpathlineto{\pgfqpoint{2.401526in}{3.141131in}}%
\pgfpathlineto{\pgfqpoint{2.403005in}{3.370086in}}%
\pgfpathlineto{\pgfqpoint{2.403499in}{3.370086in}}%
\pgfpathlineto{\pgfqpoint{2.404485in}{3.103618in}}%
\pgfpathlineto{\pgfqpoint{2.405472in}{3.164811in}}%
\pgfpathlineto{\pgfqpoint{2.406951in}{3.378833in}}%
\pgfpathlineto{\pgfqpoint{2.407444in}{3.378833in}}%
\pgfpathlineto{\pgfqpoint{2.408431in}{3.209645in}}%
\pgfpathlineto{\pgfqpoint{2.409911in}{3.353029in}}%
\pgfpathlineto{\pgfqpoint{2.410404in}{3.353714in}}%
\pgfpathlineto{\pgfqpoint{2.411390in}{2.980273in}}%
\pgfpathlineto{\pgfqpoint{2.412377in}{3.378090in}}%
\pgfpathlineto{\pgfqpoint{2.412870in}{3.328941in}}%
\pgfpathlineto{\pgfqpoint{2.413856in}{0.580000in}}%
\pgfpathlineto{\pgfqpoint{2.414350in}{1.945679in}}%
\pgfpathlineto{\pgfqpoint{2.414843in}{2.198369in}}%
\pgfpathlineto{\pgfqpoint{2.415336in}{3.376279in}}%
\pgfpathlineto{\pgfqpoint{2.416323in}{3.341366in}}%
\pgfpathlineto{\pgfqpoint{2.418789in}{3.261802in}}%
\pgfpathlineto{\pgfqpoint{2.419775in}{3.379027in}}%
\pgfpathlineto{\pgfqpoint{2.421255in}{3.275284in}}%
\pgfpathlineto{\pgfqpoint{2.421748in}{3.275284in}}%
\pgfpathlineto{\pgfqpoint{2.422735in}{3.170132in}}%
\pgfpathlineto{\pgfqpoint{2.423228in}{3.376289in}}%
\pgfpathlineto{\pgfqpoint{2.423721in}{3.344670in}}%
\pgfpathlineto{\pgfqpoint{2.425201in}{3.004255in}}%
\pgfpathlineto{\pgfqpoint{2.425694in}{2.873427in}}%
\pgfpathlineto{\pgfqpoint{2.426681in}{3.379612in}}%
\pgfpathlineto{\pgfqpoint{2.427174in}{2.803268in}}%
\pgfpathlineto{\pgfqpoint{2.427667in}{3.359633in}}%
\pgfpathlineto{\pgfqpoint{2.429640in}{3.359633in}}%
\pgfpathlineto{\pgfqpoint{2.430626in}{2.535706in}}%
\pgfpathlineto{\pgfqpoint{2.431613in}{3.374171in}}%
\pgfpathlineto{\pgfqpoint{2.433093in}{2.320730in}}%
\pgfpathlineto{\pgfqpoint{2.434572in}{3.355748in}}%
\pgfpathlineto{\pgfqpoint{2.435559in}{3.355748in}}%
\pgfpathlineto{\pgfqpoint{2.436052in}{2.889145in}}%
\pgfpathlineto{\pgfqpoint{2.436545in}{3.319049in}}%
\pgfpathlineto{\pgfqpoint{2.438025in}{3.205287in}}%
\pgfpathlineto{\pgfqpoint{2.438518in}{3.205287in}}%
\pgfpathlineto{\pgfqpoint{2.439011in}{3.379718in}}%
\pgfpathlineto{\pgfqpoint{2.439505in}{3.347203in}}%
\pgfpathlineto{\pgfqpoint{2.440984in}{3.157134in}}%
\pgfpathlineto{\pgfqpoint{2.441477in}{3.157134in}}%
\pgfpathlineto{\pgfqpoint{2.441971in}{3.236126in}}%
\pgfpathlineto{\pgfqpoint{2.442464in}{3.177136in}}%
\pgfpathlineto{\pgfqpoint{2.442957in}{3.185209in}}%
\pgfpathlineto{\pgfqpoint{2.444437in}{2.107956in}}%
\pgfpathlineto{\pgfqpoint{2.444930in}{3.376510in}}%
\pgfpathlineto{\pgfqpoint{2.445917in}{3.194760in}}%
\pgfpathlineto{\pgfqpoint{2.446903in}{3.323340in}}%
\pgfpathlineto{\pgfqpoint{2.448383in}{3.121065in}}%
\pgfpathlineto{\pgfqpoint{2.448876in}{3.340731in}}%
\pgfpathlineto{\pgfqpoint{2.449862in}{3.295094in}}%
\pgfpathlineto{\pgfqpoint{2.451342in}{3.295094in}}%
\pgfpathlineto{\pgfqpoint{2.451835in}{3.194081in}}%
\pgfpathlineto{\pgfqpoint{2.453315in}{3.374302in}}%
\pgfpathlineto{\pgfqpoint{2.455781in}{3.374302in}}%
\pgfpathlineto{\pgfqpoint{2.457261in}{3.264000in}}%
\pgfpathlineto{\pgfqpoint{2.457754in}{3.280878in}}%
\pgfpathlineto{\pgfqpoint{2.458247in}{3.350035in}}%
\pgfpathlineto{\pgfqpoint{2.459727in}{3.004906in}}%
\pgfpathlineto{\pgfqpoint{2.460220in}{3.011268in}}%
\pgfpathlineto{\pgfqpoint{2.461207in}{3.341565in}}%
\pgfpathlineto{\pgfqpoint{2.462193in}{3.369177in}}%
\pgfpathlineto{\pgfqpoint{2.462686in}{2.788014in}}%
\pgfpathlineto{\pgfqpoint{2.463180in}{3.356205in}}%
\pgfpathlineto{\pgfqpoint{2.464166in}{3.273896in}}%
\pgfpathlineto{\pgfqpoint{2.465153in}{3.273896in}}%
\pgfpathlineto{\pgfqpoint{2.465646in}{3.374897in}}%
\pgfpathlineto{\pgfqpoint{2.466139in}{3.213028in}}%
\pgfpathlineto{\pgfqpoint{2.466632in}{3.376599in}}%
\pgfpathlineto{\pgfqpoint{2.467619in}{3.378886in}}%
\pgfpathlineto{\pgfqpoint{2.468112in}{3.202588in}}%
\pgfpathlineto{\pgfqpoint{2.468605in}{3.218757in}}%
\pgfpathlineto{\pgfqpoint{2.469098in}{2.987188in}}%
\pgfpathlineto{\pgfqpoint{2.469592in}{3.136129in}}%
\pgfpathlineto{\pgfqpoint{2.471071in}{3.347092in}}%
\pgfpathlineto{\pgfqpoint{2.473044in}{3.154454in}}%
\pgfpathlineto{\pgfqpoint{2.474031in}{3.338303in}}%
\pgfpathlineto{\pgfqpoint{2.475510in}{3.235575in}}%
\pgfpathlineto{\pgfqpoint{2.476497in}{3.378743in}}%
\pgfpathlineto{\pgfqpoint{2.476990in}{3.366098in}}%
\pgfpathlineto{\pgfqpoint{2.477977in}{3.366098in}}%
\pgfpathlineto{\pgfqpoint{2.478963in}{3.171372in}}%
\pgfpathlineto{\pgfqpoint{2.479456in}{3.379053in}}%
\pgfpathlineto{\pgfqpoint{2.479949in}{2.650331in}}%
\pgfpathlineto{\pgfqpoint{2.480443in}{3.123745in}}%
\pgfpathlineto{\pgfqpoint{2.480936in}{3.160911in}}%
\pgfpathlineto{\pgfqpoint{2.481429in}{3.359614in}}%
\pgfpathlineto{\pgfqpoint{2.482416in}{3.108633in}}%
\pgfpathlineto{\pgfqpoint{2.483402in}{3.338552in}}%
\pgfpathlineto{\pgfqpoint{2.485375in}{2.962172in}}%
\pgfpathlineto{\pgfqpoint{2.485868in}{2.962172in}}%
\pgfpathlineto{\pgfqpoint{2.486855in}{3.117156in}}%
\pgfpathlineto{\pgfqpoint{2.487348in}{2.645625in}}%
\pgfpathlineto{\pgfqpoint{2.488828in}{3.379944in}}%
\pgfpathlineto{\pgfqpoint{2.489321in}{2.784756in}}%
\pgfpathlineto{\pgfqpoint{2.489814in}{3.148479in}}%
\pgfpathlineto{\pgfqpoint{2.491294in}{3.352500in}}%
\pgfpathlineto{\pgfqpoint{2.492773in}{3.353239in}}%
\pgfpathlineto{\pgfqpoint{2.494253in}{2.848921in}}%
\pgfpathlineto{\pgfqpoint{2.495733in}{3.367035in}}%
\pgfpathlineto{\pgfqpoint{2.496719in}{3.367035in}}%
\pgfpathlineto{\pgfqpoint{2.498199in}{3.351499in}}%
\pgfpathlineto{\pgfqpoint{2.498692in}{3.376658in}}%
\pgfpathlineto{\pgfqpoint{2.499185in}{3.357775in}}%
\pgfpathlineto{\pgfqpoint{2.499679in}{3.357775in}}%
\pgfpathlineto{\pgfqpoint{2.500172in}{3.077356in}}%
\pgfpathlineto{\pgfqpoint{2.500665in}{3.110197in}}%
\pgfpathlineto{\pgfqpoint{2.501158in}{3.277978in}}%
\pgfpathlineto{\pgfqpoint{2.501652in}{2.987890in}}%
\pgfpathlineto{\pgfqpoint{2.502145in}{3.248686in}}%
\pgfpathlineto{\pgfqpoint{2.503131in}{3.248686in}}%
\pgfpathlineto{\pgfqpoint{2.503625in}{3.364608in}}%
\pgfpathlineto{\pgfqpoint{2.504118in}{3.295426in}}%
\pgfpathlineto{\pgfqpoint{2.506091in}{3.295426in}}%
\pgfpathlineto{\pgfqpoint{2.507077in}{3.173157in}}%
\pgfpathlineto{\pgfqpoint{2.508557in}{3.279882in}}%
\pgfpathlineto{\pgfqpoint{2.509050in}{3.092995in}}%
\pgfpathlineto{\pgfqpoint{2.510037in}{3.095997in}}%
\pgfpathlineto{\pgfqpoint{2.510530in}{2.956512in}}%
\pgfpathlineto{\pgfqpoint{2.512503in}{3.284856in}}%
\pgfpathlineto{\pgfqpoint{2.512996in}{3.373024in}}%
\pgfpathlineto{\pgfqpoint{2.513489in}{3.309735in}}%
\pgfpathlineto{\pgfqpoint{2.513982in}{3.309735in}}%
\pgfpathlineto{\pgfqpoint{2.515462in}{3.362472in}}%
\pgfpathlineto{\pgfqpoint{2.516449in}{2.062060in}}%
\pgfpathlineto{\pgfqpoint{2.516942in}{2.795865in}}%
\pgfpathlineto{\pgfqpoint{2.518422in}{2.795865in}}%
\pgfpathlineto{\pgfqpoint{2.519901in}{3.379023in}}%
\pgfpathlineto{\pgfqpoint{2.520394in}{3.379023in}}%
\pgfpathlineto{\pgfqpoint{2.521874in}{3.376849in}}%
\pgfpathlineto{\pgfqpoint{2.523354in}{3.317321in}}%
\pgfpathlineto{\pgfqpoint{2.523847in}{2.234756in}}%
\pgfpathlineto{\pgfqpoint{2.524340in}{2.631489in}}%
\pgfpathlineto{\pgfqpoint{2.524834in}{2.636449in}}%
\pgfpathlineto{\pgfqpoint{2.526313in}{3.310326in}}%
\pgfpathlineto{\pgfqpoint{2.527793in}{3.369427in}}%
\pgfpathlineto{\pgfqpoint{2.528286in}{3.376127in}}%
\pgfpathlineto{\pgfqpoint{2.528779in}{3.012537in}}%
\pgfpathlineto{\pgfqpoint{2.529273in}{3.341820in}}%
\pgfpathlineto{\pgfqpoint{2.529766in}{3.319524in}}%
\pgfpathlineto{\pgfqpoint{2.530752in}{3.351326in}}%
\pgfpathlineto{\pgfqpoint{2.531246in}{3.345607in}}%
\pgfpathlineto{\pgfqpoint{2.531739in}{2.957527in}}%
\pgfpathlineto{\pgfqpoint{2.532232in}{3.166130in}}%
\pgfpathlineto{\pgfqpoint{2.533218in}{3.357607in}}%
\pgfpathlineto{\pgfqpoint{2.534698in}{3.041495in}}%
\pgfpathlineto{\pgfqpoint{2.536178in}{3.041495in}}%
\pgfpathlineto{\pgfqpoint{2.537164in}{3.314207in}}%
\pgfpathlineto{\pgfqpoint{2.537658in}{3.109016in}}%
\pgfpathlineto{\pgfqpoint{2.538151in}{3.285998in}}%
\pgfpathlineto{\pgfqpoint{2.539630in}{3.359321in}}%
\pgfpathlineto{\pgfqpoint{2.543083in}{3.359321in}}%
\pgfpathlineto{\pgfqpoint{2.544070in}{3.210347in}}%
\pgfpathlineto{\pgfqpoint{2.545549in}{3.325967in}}%
\pgfpathlineto{\pgfqpoint{2.547029in}{3.373771in}}%
\pgfpathlineto{\pgfqpoint{2.547522in}{3.373771in}}%
\pgfpathlineto{\pgfqpoint{2.548015in}{3.174871in}}%
\pgfpathlineto{\pgfqpoint{2.548509in}{3.363203in}}%
\pgfpathlineto{\pgfqpoint{2.549495in}{3.375365in}}%
\pgfpathlineto{\pgfqpoint{2.549988in}{3.373918in}}%
\pgfpathlineto{\pgfqpoint{2.553934in}{3.373373in}}%
\pgfpathlineto{\pgfqpoint{2.554427in}{3.351406in}}%
\pgfpathlineto{\pgfqpoint{2.554921in}{3.365865in}}%
\pgfpathlineto{\pgfqpoint{2.555414in}{3.061859in}}%
\pgfpathlineto{\pgfqpoint{2.555907in}{3.089753in}}%
\pgfpathlineto{\pgfqpoint{2.556894in}{3.337823in}}%
\pgfpathlineto{\pgfqpoint{2.557387in}{3.116884in}}%
\pgfpathlineto{\pgfqpoint{2.557880in}{3.278700in}}%
\pgfpathlineto{\pgfqpoint{2.558866in}{3.278700in}}%
\pgfpathlineto{\pgfqpoint{2.560346in}{3.074153in}}%
\pgfpathlineto{\pgfqpoint{2.560839in}{3.074153in}}%
\pgfpathlineto{\pgfqpoint{2.561826in}{3.379734in}}%
\pgfpathlineto{\pgfqpoint{2.562319in}{3.369920in}}%
\pgfpathlineto{\pgfqpoint{2.563306in}{3.369920in}}%
\pgfpathlineto{\pgfqpoint{2.564785in}{3.091393in}}%
\pgfpathlineto{\pgfqpoint{2.566265in}{3.311453in}}%
\pgfpathlineto{\pgfqpoint{2.566758in}{3.379748in}}%
\pgfpathlineto{\pgfqpoint{2.567251in}{3.309291in}}%
\pgfpathlineto{\pgfqpoint{2.567745in}{3.260947in}}%
\pgfpathlineto{\pgfqpoint{2.569224in}{3.365526in}}%
\pgfpathlineto{\pgfqpoint{2.570211in}{3.365526in}}%
\pgfpathlineto{\pgfqpoint{2.570704in}{3.159694in}}%
\pgfpathlineto{\pgfqpoint{2.571197in}{3.355317in}}%
\pgfpathlineto{\pgfqpoint{2.571690in}{3.355317in}}%
\pgfpathlineto{\pgfqpoint{2.572677in}{2.905188in}}%
\pgfpathlineto{\pgfqpoint{2.573663in}{3.373172in}}%
\pgfpathlineto{\pgfqpoint{2.574157in}{3.281909in}}%
\pgfpathlineto{\pgfqpoint{2.574650in}{2.699264in}}%
\pgfpathlineto{\pgfqpoint{2.575143in}{3.372431in}}%
\pgfpathlineto{\pgfqpoint{2.576623in}{3.279121in}}%
\pgfpathlineto{\pgfqpoint{2.577609in}{3.327093in}}%
\pgfpathlineto{\pgfqpoint{2.579089in}{3.275315in}}%
\pgfpathlineto{\pgfqpoint{2.580569in}{3.275315in}}%
\pgfpathlineto{\pgfqpoint{2.584021in}{2.928164in}}%
\pgfpathlineto{\pgfqpoint{2.585501in}{3.127693in}}%
\pgfpathlineto{\pgfqpoint{2.585994in}{3.127693in}}%
\pgfpathlineto{\pgfqpoint{2.587474in}{3.286283in}}%
\pgfpathlineto{\pgfqpoint{2.587967in}{3.366166in}}%
\pgfpathlineto{\pgfqpoint{2.588460in}{2.429725in}}%
\pgfpathlineto{\pgfqpoint{2.588954in}{3.357014in}}%
\pgfpathlineto{\pgfqpoint{2.589447in}{2.806917in}}%
\pgfpathlineto{\pgfqpoint{2.589940in}{3.328533in}}%
\pgfpathlineto{\pgfqpoint{2.590926in}{3.328533in}}%
\pgfpathlineto{\pgfqpoint{2.591420in}{3.302391in}}%
\pgfpathlineto{\pgfqpoint{2.592406in}{3.088131in}}%
\pgfpathlineto{\pgfqpoint{2.593393in}{3.367277in}}%
\pgfpathlineto{\pgfqpoint{2.593886in}{3.339646in}}%
\pgfpathlineto{\pgfqpoint{2.594872in}{2.890134in}}%
\pgfpathlineto{\pgfqpoint{2.596352in}{3.367199in}}%
\pgfpathlineto{\pgfqpoint{2.596845in}{3.367199in}}%
\pgfpathlineto{\pgfqpoint{2.597338in}{3.377953in}}%
\pgfpathlineto{\pgfqpoint{2.598818in}{2.689111in}}%
\pgfpathlineto{\pgfqpoint{2.600298in}{3.302436in}}%
\pgfpathlineto{\pgfqpoint{2.600791in}{3.205012in}}%
\pgfpathlineto{\pgfqpoint{2.603750in}{3.205012in}}%
\pgfpathlineto{\pgfqpoint{2.605230in}{3.142373in}}%
\pgfpathlineto{\pgfqpoint{2.606217in}{3.379873in}}%
\pgfpathlineto{\pgfqpoint{2.607203in}{2.326601in}}%
\pgfpathlineto{\pgfqpoint{2.607696in}{3.030323in}}%
\pgfpathlineto{\pgfqpoint{2.608190in}{3.219003in}}%
\pgfpathlineto{\pgfqpoint{2.609176in}{3.198551in}}%
\pgfpathlineto{\pgfqpoint{2.610162in}{3.198551in}}%
\pgfpathlineto{\pgfqpoint{2.610656in}{3.111019in}}%
\pgfpathlineto{\pgfqpoint{2.611642in}{3.363839in}}%
\pgfpathlineto{\pgfqpoint{2.612135in}{3.206253in}}%
\pgfpathlineto{\pgfqpoint{2.612629in}{3.376098in}}%
\pgfpathlineto{\pgfqpoint{2.614108in}{3.376098in}}%
\pgfpathlineto{\pgfqpoint{2.615588in}{2.931077in}}%
\pgfpathlineto{\pgfqpoint{2.617068in}{3.365277in}}%
\pgfpathlineto{\pgfqpoint{2.617561in}{3.365277in}}%
\pgfpathlineto{\pgfqpoint{2.618547in}{3.355435in}}%
\pgfpathlineto{\pgfqpoint{2.619041in}{3.364417in}}%
\pgfpathlineto{\pgfqpoint{2.621014in}{2.468705in}}%
\pgfpathlineto{\pgfqpoint{2.622000in}{3.146934in}}%
\pgfpathlineto{\pgfqpoint{2.622987in}{2.939627in}}%
\pgfpathlineto{\pgfqpoint{2.623973in}{3.379583in}}%
\pgfpathlineto{\pgfqpoint{2.624466in}{3.348396in}}%
\pgfpathlineto{\pgfqpoint{2.624959in}{3.303278in}}%
\pgfpathlineto{\pgfqpoint{2.625453in}{2.281832in}}%
\pgfpathlineto{\pgfqpoint{2.625946in}{3.368734in}}%
\pgfpathlineto{\pgfqpoint{2.626439in}{3.114656in}}%
\pgfpathlineto{\pgfqpoint{2.626932in}{3.337841in}}%
\pgfpathlineto{\pgfqpoint{2.628905in}{3.379661in}}%
\pgfpathlineto{\pgfqpoint{2.629399in}{2.332262in}}%
\pgfpathlineto{\pgfqpoint{2.629892in}{2.936605in}}%
\pgfpathlineto{\pgfqpoint{2.630385in}{2.936605in}}%
\pgfpathlineto{\pgfqpoint{2.631865in}{3.316284in}}%
\pgfpathlineto{\pgfqpoint{2.632358in}{2.724262in}}%
\pgfpathlineto{\pgfqpoint{2.632851in}{3.167625in}}%
\pgfpathlineto{\pgfqpoint{2.634824in}{2.504785in}}%
\pgfpathlineto{\pgfqpoint{2.635811in}{3.120131in}}%
\pgfpathlineto{\pgfqpoint{2.636304in}{3.069946in}}%
\pgfpathlineto{\pgfqpoint{2.637290in}{3.069946in}}%
\pgfpathlineto{\pgfqpoint{2.638277in}{3.336441in}}%
\pgfpathlineto{\pgfqpoint{2.638770in}{2.086467in}}%
\pgfpathlineto{\pgfqpoint{2.639263in}{3.347399in}}%
\pgfpathlineto{\pgfqpoint{2.640743in}{3.347399in}}%
\pgfpathlineto{\pgfqpoint{2.641236in}{3.257915in}}%
\pgfpathlineto{\pgfqpoint{2.641729in}{3.373401in}}%
\pgfpathlineto{\pgfqpoint{2.642223in}{3.329723in}}%
\pgfpathlineto{\pgfqpoint{2.643702in}{3.309838in}}%
\pgfpathlineto{\pgfqpoint{2.644689in}{3.309838in}}%
\pgfpathlineto{\pgfqpoint{2.645182in}{3.290586in}}%
\pgfpathlineto{\pgfqpoint{2.645675in}{3.377086in}}%
\pgfpathlineto{\pgfqpoint{2.646168in}{3.215054in}}%
\pgfpathlineto{\pgfqpoint{2.646662in}{3.334686in}}%
\pgfpathlineto{\pgfqpoint{2.647648in}{3.301187in}}%
\pgfpathlineto{\pgfqpoint{2.648635in}{3.118421in}}%
\pgfpathlineto{\pgfqpoint{2.650114in}{3.238154in}}%
\pgfpathlineto{\pgfqpoint{2.650607in}{3.238154in}}%
\pgfpathlineto{\pgfqpoint{2.652087in}{3.362527in}}%
\pgfpathlineto{\pgfqpoint{2.652580in}{2.891194in}}%
\pgfpathlineto{\pgfqpoint{2.653074in}{3.318095in}}%
\pgfpathlineto{\pgfqpoint{2.654553in}{3.359028in}}%
\pgfpathlineto{\pgfqpoint{2.655540in}{3.359028in}}%
\pgfpathlineto{\pgfqpoint{2.656526in}{3.331464in}}%
\pgfpathlineto{\pgfqpoint{2.657019in}{3.335696in}}%
\pgfpathlineto{\pgfqpoint{2.658499in}{3.337699in}}%
\pgfpathlineto{\pgfqpoint{2.659979in}{3.088834in}}%
\pgfpathlineto{\pgfqpoint{2.660472in}{3.088834in}}%
\pgfpathlineto{\pgfqpoint{2.661459in}{3.378937in}}%
\pgfpathlineto{\pgfqpoint{2.661952in}{3.059628in}}%
\pgfpathlineto{\pgfqpoint{2.662445in}{3.239058in}}%
\pgfpathlineto{\pgfqpoint{2.662938in}{3.239058in}}%
\pgfpathlineto{\pgfqpoint{2.663925in}{3.360801in}}%
\pgfpathlineto{\pgfqpoint{2.665404in}{2.936670in}}%
\pgfpathlineto{\pgfqpoint{2.666884in}{3.375259in}}%
\pgfpathlineto{\pgfqpoint{2.668364in}{3.375909in}}%
\pgfpathlineto{\pgfqpoint{2.670337in}{2.142673in}}%
\pgfpathlineto{\pgfqpoint{2.671816in}{3.376258in}}%
\pgfpathlineto{\pgfqpoint{2.672310in}{3.376258in}}%
\pgfpathlineto{\pgfqpoint{2.673789in}{2.395232in}}%
\pgfpathlineto{\pgfqpoint{2.674283in}{3.366102in}}%
\pgfpathlineto{\pgfqpoint{2.674776in}{2.882075in}}%
\pgfpathlineto{\pgfqpoint{2.675762in}{2.882075in}}%
\pgfpathlineto{\pgfqpoint{2.676255in}{3.374771in}}%
\pgfpathlineto{\pgfqpoint{2.677242in}{3.285610in}}%
\pgfpathlineto{\pgfqpoint{2.678228in}{3.285610in}}%
\pgfpathlineto{\pgfqpoint{2.678722in}{2.733573in}}%
\pgfpathlineto{\pgfqpoint{2.679215in}{3.091625in}}%
\pgfpathlineto{\pgfqpoint{2.680695in}{3.377756in}}%
\pgfpathlineto{\pgfqpoint{2.681188in}{3.377756in}}%
\pgfpathlineto{\pgfqpoint{2.682667in}{3.218019in}}%
\pgfpathlineto{\pgfqpoint{2.684147in}{3.379853in}}%
\pgfpathlineto{\pgfqpoint{2.684640in}{3.379853in}}%
\pgfpathlineto{\pgfqpoint{2.685134in}{3.338315in}}%
\pgfpathlineto{\pgfqpoint{2.685627in}{3.045994in}}%
\pgfpathlineto{\pgfqpoint{2.686120in}{3.373135in}}%
\pgfpathlineto{\pgfqpoint{2.686613in}{3.373135in}}%
\pgfpathlineto{\pgfqpoint{2.687107in}{3.380000in}}%
\pgfpathlineto{\pgfqpoint{2.688586in}{3.014269in}}%
\pgfpathlineto{\pgfqpoint{2.689079in}{3.209328in}}%
\pgfpathlineto{\pgfqpoint{2.690066in}{3.376469in}}%
\pgfpathlineto{\pgfqpoint{2.690559in}{3.274032in}}%
\pgfpathlineto{\pgfqpoint{2.691052in}{3.354677in}}%
\pgfpathlineto{\pgfqpoint{2.691546in}{3.354677in}}%
\pgfpathlineto{\pgfqpoint{2.692532in}{3.106006in}}%
\pgfpathlineto{\pgfqpoint{2.693519in}{3.303462in}}%
\pgfpathlineto{\pgfqpoint{2.694505in}{3.083556in}}%
\pgfpathlineto{\pgfqpoint{2.695491in}{3.049962in}}%
\pgfpathlineto{\pgfqpoint{2.695985in}{3.357036in}}%
\pgfpathlineto{\pgfqpoint{2.698451in}{3.358009in}}%
\pgfpathlineto{\pgfqpoint{2.699931in}{3.379921in}}%
\pgfpathlineto{\pgfqpoint{2.700424in}{2.831464in}}%
\pgfpathlineto{\pgfqpoint{2.700917in}{2.914732in}}%
\pgfpathlineto{\pgfqpoint{2.702890in}{3.364473in}}%
\pgfpathlineto{\pgfqpoint{2.704370in}{3.375415in}}%
\pgfpathlineto{\pgfqpoint{2.704863in}{3.375415in}}%
\pgfpathlineto{\pgfqpoint{2.705356in}{3.207344in}}%
\pgfpathlineto{\pgfqpoint{2.705849in}{3.210348in}}%
\pgfpathlineto{\pgfqpoint{2.706343in}{3.333552in}}%
\pgfpathlineto{\pgfqpoint{2.707329in}{3.311279in}}%
\pgfpathlineto{\pgfqpoint{2.707822in}{3.311279in}}%
\pgfpathlineto{\pgfqpoint{2.709302in}{3.364582in}}%
\pgfpathlineto{\pgfqpoint{2.711768in}{3.364582in}}%
\pgfpathlineto{\pgfqpoint{2.712261in}{3.313963in}}%
\pgfpathlineto{\pgfqpoint{2.712755in}{3.379956in}}%
\pgfpathlineto{\pgfqpoint{2.714234in}{3.093093in}}%
\pgfpathlineto{\pgfqpoint{2.714727in}{3.093093in}}%
\pgfpathlineto{\pgfqpoint{2.715221in}{3.373588in}}%
\pgfpathlineto{\pgfqpoint{2.716207in}{3.328149in}}%
\pgfpathlineto{\pgfqpoint{2.716700in}{3.288029in}}%
\pgfpathlineto{\pgfqpoint{2.717194in}{3.364135in}}%
\pgfpathlineto{\pgfqpoint{2.718180in}{3.356482in}}%
\pgfpathlineto{\pgfqpoint{2.718673in}{3.356482in}}%
\pgfpathlineto{\pgfqpoint{2.719660in}{3.359139in}}%
\pgfpathlineto{\pgfqpoint{2.720153in}{3.352884in}}%
\pgfpathlineto{\pgfqpoint{2.720646in}{3.230289in}}%
\pgfpathlineto{\pgfqpoint{2.721140in}{3.236543in}}%
\pgfpathlineto{\pgfqpoint{2.722619in}{3.323696in}}%
\pgfpathlineto{\pgfqpoint{2.724099in}{3.358519in}}%
\pgfpathlineto{\pgfqpoint{2.724592in}{3.306934in}}%
\pgfpathlineto{\pgfqpoint{2.726072in}{3.379991in}}%
\pgfpathlineto{\pgfqpoint{2.726565in}{3.003311in}}%
\pgfpathlineto{\pgfqpoint{2.727058in}{3.305504in}}%
\pgfpathlineto{\pgfqpoint{2.727552in}{3.331040in}}%
\pgfpathlineto{\pgfqpoint{2.728045in}{3.246311in}}%
\pgfpathlineto{\pgfqpoint{2.728538in}{3.318952in}}%
\pgfpathlineto{\pgfqpoint{2.730018in}{3.318952in}}%
\pgfpathlineto{\pgfqpoint{2.731004in}{3.329150in}}%
\pgfpathlineto{\pgfqpoint{2.733470in}{2.679309in}}%
\pgfpathlineto{\pgfqpoint{2.734950in}{3.281958in}}%
\pgfpathlineto{\pgfqpoint{2.735443in}{2.827559in}}%
\pgfpathlineto{\pgfqpoint{2.735936in}{3.276081in}}%
\pgfpathlineto{\pgfqpoint{2.736430in}{3.232787in}}%
\pgfpathlineto{\pgfqpoint{2.736923in}{3.043485in}}%
\pgfpathlineto{\pgfqpoint{2.737909in}{3.371747in}}%
\pgfpathlineto{\pgfqpoint{2.738403in}{3.352271in}}%
\pgfpathlineto{\pgfqpoint{2.738896in}{3.352271in}}%
\pgfpathlineto{\pgfqpoint{2.741855in}{3.331837in}}%
\pgfpathlineto{\pgfqpoint{2.743828in}{3.331837in}}%
\pgfpathlineto{\pgfqpoint{2.744815in}{3.043333in}}%
\pgfpathlineto{\pgfqpoint{2.746294in}{3.247913in}}%
\pgfpathlineto{\pgfqpoint{2.746788in}{3.379943in}}%
\pgfpathlineto{\pgfqpoint{2.747281in}{3.346126in}}%
\pgfpathlineto{\pgfqpoint{2.748267in}{3.346126in}}%
\pgfpathlineto{\pgfqpoint{2.748760in}{3.370669in}}%
\pgfpathlineto{\pgfqpoint{2.749254in}{2.871174in}}%
\pgfpathlineto{\pgfqpoint{2.749747in}{3.281909in}}%
\pgfpathlineto{\pgfqpoint{2.750240in}{3.281909in}}%
\pgfpathlineto{\pgfqpoint{2.750733in}{3.026526in}}%
\pgfpathlineto{\pgfqpoint{2.751227in}{3.147210in}}%
\pgfpathlineto{\pgfqpoint{2.752213in}{3.147210in}}%
\pgfpathlineto{\pgfqpoint{2.752706in}{3.292624in}}%
\pgfpathlineto{\pgfqpoint{2.754186in}{3.000766in}}%
\pgfpathlineto{\pgfqpoint{2.755666in}{3.376271in}}%
\pgfpathlineto{\pgfqpoint{2.756652in}{2.225022in}}%
\pgfpathlineto{\pgfqpoint{2.758132in}{3.343733in}}%
\pgfpathlineto{\pgfqpoint{2.758625in}{3.203400in}}%
\pgfpathlineto{\pgfqpoint{2.759118in}{2.669075in}}%
\pgfpathlineto{\pgfqpoint{2.759612in}{3.372340in}}%
\pgfpathlineto{\pgfqpoint{2.760598in}{3.335838in}}%
\pgfpathlineto{\pgfqpoint{2.762078in}{3.207748in}}%
\pgfpathlineto{\pgfqpoint{2.763064in}{3.244858in}}%
\pgfpathlineto{\pgfqpoint{2.763557in}{2.327832in}}%
\pgfpathlineto{\pgfqpoint{2.764051in}{2.798352in}}%
\pgfpathlineto{\pgfqpoint{2.764544in}{2.671570in}}%
\pgfpathlineto{\pgfqpoint{2.765530in}{3.355537in}}%
\pgfpathlineto{\pgfqpoint{2.766024in}{3.285337in}}%
\pgfpathlineto{\pgfqpoint{2.766517in}{3.285337in}}%
\pgfpathlineto{\pgfqpoint{2.767010in}{3.181187in}}%
\pgfpathlineto{\pgfqpoint{2.768490in}{3.365492in}}%
\pgfpathlineto{\pgfqpoint{2.769969in}{3.372919in}}%
\pgfpathlineto{\pgfqpoint{2.770463in}{3.195314in}}%
\pgfpathlineto{\pgfqpoint{2.770956in}{3.344960in}}%
\pgfpathlineto{\pgfqpoint{2.772436in}{3.344960in}}%
\pgfpathlineto{\pgfqpoint{2.773915in}{3.348551in}}%
\pgfpathlineto{\pgfqpoint{2.775395in}{3.376420in}}%
\pgfpathlineto{\pgfqpoint{2.775888in}{3.376420in}}%
\pgfpathlineto{\pgfqpoint{2.776875in}{2.896875in}}%
\pgfpathlineto{\pgfqpoint{2.777368in}{3.152381in}}%
\pgfpathlineto{\pgfqpoint{2.777861in}{3.100929in}}%
\pgfpathlineto{\pgfqpoint{2.779341in}{2.872796in}}%
\pgfpathlineto{\pgfqpoint{2.779834in}{2.872796in}}%
\pgfpathlineto{\pgfqpoint{2.780327in}{3.375951in}}%
\pgfpathlineto{\pgfqpoint{2.781314in}{3.365169in}}%
\pgfpathlineto{\pgfqpoint{2.781807in}{3.365169in}}%
\pgfpathlineto{\pgfqpoint{2.782300in}{2.972699in}}%
\pgfpathlineto{\pgfqpoint{2.782793in}{3.379926in}}%
\pgfpathlineto{\pgfqpoint{2.784273in}{3.357635in}}%
\pgfpathlineto{\pgfqpoint{2.784766in}{3.357635in}}%
\pgfpathlineto{\pgfqpoint{2.786739in}{2.351118in}}%
\pgfpathlineto{\pgfqpoint{2.787726in}{3.376534in}}%
\pgfpathlineto{\pgfqpoint{2.788219in}{3.373192in}}%
\pgfpathlineto{\pgfqpoint{2.789205in}{3.338295in}}%
\pgfpathlineto{\pgfqpoint{2.789699in}{3.098702in}}%
\pgfpathlineto{\pgfqpoint{2.790192in}{3.373677in}}%
\pgfpathlineto{\pgfqpoint{2.790685in}{3.373677in}}%
\pgfpathlineto{\pgfqpoint{2.791178in}{3.205405in}}%
\pgfpathlineto{\pgfqpoint{2.791672in}{3.277358in}}%
\pgfpathlineto{\pgfqpoint{2.792658in}{3.277358in}}%
\pgfpathlineto{\pgfqpoint{2.794138in}{3.150679in}}%
\pgfpathlineto{\pgfqpoint{2.794631in}{3.150679in}}%
\pgfpathlineto{\pgfqpoint{2.795124in}{3.309444in}}%
\pgfpathlineto{\pgfqpoint{2.796111in}{3.030803in}}%
\pgfpathlineto{\pgfqpoint{2.796604in}{3.376410in}}%
\pgfpathlineto{\pgfqpoint{2.797097in}{3.265692in}}%
\pgfpathlineto{\pgfqpoint{2.797590in}{3.265692in}}%
\pgfpathlineto{\pgfqpoint{2.798577in}{3.214726in}}%
\pgfpathlineto{\pgfqpoint{2.799070in}{2.315681in}}%
\pgfpathlineto{\pgfqpoint{2.799563in}{2.472901in}}%
\pgfpathlineto{\pgfqpoint{2.801043in}{3.360049in}}%
\pgfpathlineto{\pgfqpoint{2.802523in}{3.296795in}}%
\pgfpathlineto{\pgfqpoint{2.803016in}{3.371763in}}%
\pgfpathlineto{\pgfqpoint{2.803509in}{3.317266in}}%
\pgfpathlineto{\pgfqpoint{2.806468in}{3.317266in}}%
\pgfpathlineto{\pgfqpoint{2.807948in}{3.378692in}}%
\pgfpathlineto{\pgfqpoint{2.808935in}{3.378692in}}%
\pgfpathlineto{\pgfqpoint{2.810908in}{2.147203in}}%
\pgfpathlineto{\pgfqpoint{2.812387in}{3.352870in}}%
\pgfpathlineto{\pgfqpoint{2.813867in}{2.827060in}}%
\pgfpathlineto{\pgfqpoint{2.815347in}{3.333896in}}%
\pgfpathlineto{\pgfqpoint{2.817320in}{2.332774in}}%
\pgfpathlineto{\pgfqpoint{2.818799in}{3.229406in}}%
\pgfpathlineto{\pgfqpoint{2.820772in}{2.578438in}}%
\pgfpathlineto{\pgfqpoint{2.821265in}{2.578438in}}%
\pgfpathlineto{\pgfqpoint{2.822252in}{2.541837in}}%
\pgfpathlineto{\pgfqpoint{2.823238in}{3.355354in}}%
\pgfpathlineto{\pgfqpoint{2.824225in}{3.232313in}}%
\pgfpathlineto{\pgfqpoint{2.824718in}{3.028110in}}%
\pgfpathlineto{\pgfqpoint{2.825705in}{3.369999in}}%
\pgfpathlineto{\pgfqpoint{2.826691in}{3.048579in}}%
\pgfpathlineto{\pgfqpoint{2.827184in}{3.175429in}}%
\pgfpathlineto{\pgfqpoint{2.828171in}{3.151720in}}%
\pgfpathlineto{\pgfqpoint{2.828664in}{3.151720in}}%
\pgfpathlineto{\pgfqpoint{2.829157in}{3.345556in}}%
\pgfpathlineto{\pgfqpoint{2.830144in}{3.337258in}}%
\pgfpathlineto{\pgfqpoint{2.830637in}{3.370367in}}%
\pgfpathlineto{\pgfqpoint{2.831623in}{3.296355in}}%
\pgfpathlineto{\pgfqpoint{2.832117in}{3.311317in}}%
\pgfpathlineto{\pgfqpoint{2.835076in}{3.132854in}}%
\pgfpathlineto{\pgfqpoint{2.836062in}{3.371390in}}%
\pgfpathlineto{\pgfqpoint{2.837049in}{2.654405in}}%
\pgfpathlineto{\pgfqpoint{2.838035in}{3.379361in}}%
\pgfpathlineto{\pgfqpoint{2.838529in}{3.365274in}}%
\pgfpathlineto{\pgfqpoint{2.839022in}{3.158033in}}%
\pgfpathlineto{\pgfqpoint{2.839515in}{3.379903in}}%
\pgfpathlineto{\pgfqpoint{2.840501in}{3.379903in}}%
\pgfpathlineto{\pgfqpoint{2.841981in}{3.350040in}}%
\pgfpathlineto{\pgfqpoint{2.843461in}{3.353479in}}%
\pgfpathlineto{\pgfqpoint{2.844447in}{3.353479in}}%
\pgfpathlineto{\pgfqpoint{2.844941in}{3.206570in}}%
\pgfpathlineto{\pgfqpoint{2.845434in}{3.318744in}}%
\pgfpathlineto{\pgfqpoint{2.845927in}{3.318744in}}%
\pgfpathlineto{\pgfqpoint{2.848393in}{2.715655in}}%
\pgfpathlineto{\pgfqpoint{2.849873in}{3.378782in}}%
\pgfpathlineto{\pgfqpoint{2.850366in}{3.272054in}}%
\pgfpathlineto{\pgfqpoint{2.850859in}{2.744515in}}%
\pgfpathlineto{\pgfqpoint{2.851353in}{3.164301in}}%
\pgfpathlineto{\pgfqpoint{2.851846in}{3.164301in}}%
\pgfpathlineto{\pgfqpoint{2.852832in}{2.707294in}}%
\pgfpathlineto{\pgfqpoint{2.853819in}{3.370787in}}%
\pgfpathlineto{\pgfqpoint{2.854312in}{2.382540in}}%
\pgfpathlineto{\pgfqpoint{2.854805in}{3.369388in}}%
\pgfpathlineto{\pgfqpoint{2.855792in}{3.111935in}}%
\pgfpathlineto{\pgfqpoint{2.857271in}{3.336329in}}%
\pgfpathlineto{\pgfqpoint{2.858258in}{3.336329in}}%
\pgfpathlineto{\pgfqpoint{2.859244in}{3.376604in}}%
\pgfpathlineto{\pgfqpoint{2.859737in}{3.111049in}}%
\pgfpathlineto{\pgfqpoint{2.860231in}{3.360145in}}%
\pgfpathlineto{\pgfqpoint{2.860724in}{3.360145in}}%
\pgfpathlineto{\pgfqpoint{2.861217in}{3.379722in}}%
\pgfpathlineto{\pgfqpoint{2.861710in}{3.370105in}}%
\pgfpathlineto{\pgfqpoint{2.862204in}{3.370105in}}%
\pgfpathlineto{\pgfqpoint{2.863190in}{2.362622in}}%
\pgfpathlineto{\pgfqpoint{2.864177in}{3.379295in}}%
\pgfpathlineto{\pgfqpoint{2.864670in}{3.241070in}}%
\pgfpathlineto{\pgfqpoint{2.866149in}{3.371967in}}%
\pgfpathlineto{\pgfqpoint{2.866643in}{3.254325in}}%
\pgfpathlineto{\pgfqpoint{2.867136in}{3.354914in}}%
\pgfpathlineto{\pgfqpoint{2.867629in}{3.354914in}}%
\pgfpathlineto{\pgfqpoint{2.869109in}{3.180213in}}%
\pgfpathlineto{\pgfqpoint{2.869602in}{1.979165in}}%
\pgfpathlineto{\pgfqpoint{2.870095in}{3.368144in}}%
\pgfpathlineto{\pgfqpoint{2.871082in}{3.368144in}}%
\pgfpathlineto{\pgfqpoint{2.871575in}{3.342837in}}%
\pgfpathlineto{\pgfqpoint{2.872068in}{3.360215in}}%
\pgfpathlineto{\pgfqpoint{2.873055in}{3.360215in}}%
\pgfpathlineto{\pgfqpoint{2.875028in}{3.168880in}}%
\pgfpathlineto{\pgfqpoint{2.876507in}{3.371560in}}%
\pgfpathlineto{\pgfqpoint{2.877987in}{3.371560in}}%
\pgfpathlineto{\pgfqpoint{2.878973in}{3.320450in}}%
\pgfpathlineto{\pgfqpoint{2.880453in}{2.953707in}}%
\pgfpathlineto{\pgfqpoint{2.882919in}{3.375713in}}%
\pgfpathlineto{\pgfqpoint{2.883413in}{3.375713in}}%
\pgfpathlineto{\pgfqpoint{2.884399in}{3.100630in}}%
\pgfpathlineto{\pgfqpoint{2.885879in}{3.378791in}}%
\pgfpathlineto{\pgfqpoint{2.887358in}{3.342044in}}%
\pgfpathlineto{\pgfqpoint{2.888345in}{3.358249in}}%
\pgfpathlineto{\pgfqpoint{2.889825in}{3.324920in}}%
\pgfpathlineto{\pgfqpoint{2.890811in}{3.379810in}}%
\pgfpathlineto{\pgfqpoint{2.891797in}{3.310000in}}%
\pgfpathlineto{\pgfqpoint{2.892784in}{3.378036in}}%
\pgfpathlineto{\pgfqpoint{2.894264in}{3.312802in}}%
\pgfpathlineto{\pgfqpoint{2.895743in}{3.336016in}}%
\pgfpathlineto{\pgfqpoint{2.896237in}{3.336016in}}%
\pgfpathlineto{\pgfqpoint{2.897716in}{3.318836in}}%
\pgfpathlineto{\pgfqpoint{2.899196in}{2.197104in}}%
\pgfpathlineto{\pgfqpoint{2.899689in}{3.357931in}}%
\pgfpathlineto{\pgfqpoint{2.900182in}{3.347319in}}%
\pgfpathlineto{\pgfqpoint{2.901662in}{2.746811in}}%
\pgfpathlineto{\pgfqpoint{2.902155in}{2.746811in}}%
\pgfpathlineto{\pgfqpoint{2.904128in}{3.225481in}}%
\pgfpathlineto{\pgfqpoint{2.907088in}{3.225481in}}%
\pgfpathlineto{\pgfqpoint{2.908074in}{3.232035in}}%
\pgfpathlineto{\pgfqpoint{2.908567in}{3.328540in}}%
\pgfpathlineto{\pgfqpoint{2.909061in}{3.259129in}}%
\pgfpathlineto{\pgfqpoint{2.909554in}{2.585913in}}%
\pgfpathlineto{\pgfqpoint{2.910047in}{2.895210in}}%
\pgfpathlineto{\pgfqpoint{2.911527in}{3.377726in}}%
\pgfpathlineto{\pgfqpoint{2.913993in}{3.173734in}}%
\pgfpathlineto{\pgfqpoint{2.915473in}{3.343429in}}%
\pgfpathlineto{\pgfqpoint{2.915966in}{3.379933in}}%
\pgfpathlineto{\pgfqpoint{2.916459in}{3.264291in}}%
\pgfpathlineto{\pgfqpoint{2.916952in}{3.287333in}}%
\pgfpathlineto{\pgfqpoint{2.917445in}{3.379602in}}%
\pgfpathlineto{\pgfqpoint{2.918432in}{3.378758in}}%
\pgfpathlineto{\pgfqpoint{2.919418in}{3.259737in}}%
\pgfpathlineto{\pgfqpoint{2.920405in}{3.364896in}}%
\pgfpathlineto{\pgfqpoint{2.920898in}{3.194850in}}%
\pgfpathlineto{\pgfqpoint{2.921391in}{3.364750in}}%
\pgfpathlineto{\pgfqpoint{2.921885in}{3.334242in}}%
\pgfpathlineto{\pgfqpoint{2.922378in}{3.372171in}}%
\pgfpathlineto{\pgfqpoint{2.922871in}{3.379904in}}%
\pgfpathlineto{\pgfqpoint{2.924351in}{3.186832in}}%
\pgfpathlineto{\pgfqpoint{2.925337in}{3.375474in}}%
\pgfpathlineto{\pgfqpoint{2.926324in}{1.849289in}}%
\pgfpathlineto{\pgfqpoint{2.927310in}{3.324140in}}%
\pgfpathlineto{\pgfqpoint{2.927803in}{3.169651in}}%
\pgfpathlineto{\pgfqpoint{2.928297in}{3.343628in}}%
\pgfpathlineto{\pgfqpoint{2.928790in}{3.259521in}}%
\pgfpathlineto{\pgfqpoint{2.930270in}{3.316484in}}%
\pgfpathlineto{\pgfqpoint{2.930763in}{2.958502in}}%
\pgfpathlineto{\pgfqpoint{2.931256in}{3.377497in}}%
\pgfpathlineto{\pgfqpoint{2.932242in}{3.307254in}}%
\pgfpathlineto{\pgfqpoint{2.932736in}{3.364482in}}%
\pgfpathlineto{\pgfqpoint{2.933229in}{3.201949in}}%
\pgfpathlineto{\pgfqpoint{2.933722in}{2.264175in}}%
\pgfpathlineto{\pgfqpoint{2.934215in}{3.377230in}}%
\pgfpathlineto{\pgfqpoint{2.934709in}{3.377230in}}%
\pgfpathlineto{\pgfqpoint{2.935695in}{3.163753in}}%
\pgfpathlineto{\pgfqpoint{2.937175in}{3.352135in}}%
\pgfpathlineto{\pgfqpoint{2.938161in}{3.352135in}}%
\pgfpathlineto{\pgfqpoint{2.939641in}{3.363573in}}%
\pgfpathlineto{\pgfqpoint{2.940134in}{3.363573in}}%
\pgfpathlineto{\pgfqpoint{2.943587in}{3.238698in}}%
\pgfpathlineto{\pgfqpoint{2.945066in}{3.371131in}}%
\pgfpathlineto{\pgfqpoint{2.946053in}{3.031647in}}%
\pgfpathlineto{\pgfqpoint{2.946546in}{3.376864in}}%
\pgfpathlineto{\pgfqpoint{2.947039in}{3.303610in}}%
\pgfpathlineto{\pgfqpoint{2.949999in}{2.951563in}}%
\pgfpathlineto{\pgfqpoint{2.951478in}{3.379857in}}%
\pgfpathlineto{\pgfqpoint{2.953945in}{3.379857in}}%
\pgfpathlineto{\pgfqpoint{2.955424in}{3.360945in}}%
\pgfpathlineto{\pgfqpoint{2.955918in}{2.755963in}}%
\pgfpathlineto{\pgfqpoint{2.956411in}{3.376076in}}%
\pgfpathlineto{\pgfqpoint{2.956904in}{3.276211in}}%
\pgfpathlineto{\pgfqpoint{2.957397in}{3.379018in}}%
\pgfpathlineto{\pgfqpoint{2.958384in}{3.372677in}}%
\pgfpathlineto{\pgfqpoint{2.958877in}{2.387882in}}%
\pgfpathlineto{\pgfqpoint{2.959370in}{2.948241in}}%
\pgfpathlineto{\pgfqpoint{2.959863in}{2.948241in}}%
\pgfpathlineto{\pgfqpoint{2.961343in}{3.372317in}}%
\pgfpathlineto{\pgfqpoint{2.962823in}{3.073327in}}%
\pgfpathlineto{\pgfqpoint{2.963316in}{3.358581in}}%
\pgfpathlineto{\pgfqpoint{2.964302in}{3.348239in}}%
\pgfpathlineto{\pgfqpoint{2.965289in}{3.348239in}}%
\pgfpathlineto{\pgfqpoint{2.966275in}{3.362622in}}%
\pgfpathlineto{\pgfqpoint{2.966769in}{3.299143in}}%
\pgfpathlineto{\pgfqpoint{2.967262in}{3.320801in}}%
\pgfpathlineto{\pgfqpoint{2.968742in}{3.332357in}}%
\pgfpathlineto{\pgfqpoint{2.970714in}{3.332357in}}%
\pgfpathlineto{\pgfqpoint{2.971208in}{2.702083in}}%
\pgfpathlineto{\pgfqpoint{2.971701in}{3.209602in}}%
\pgfpathlineto{\pgfqpoint{2.972194in}{3.209602in}}%
\pgfpathlineto{\pgfqpoint{2.972687in}{3.164460in}}%
\pgfpathlineto{\pgfqpoint{2.973181in}{3.171440in}}%
\pgfpathlineto{\pgfqpoint{2.974660in}{3.379821in}}%
\pgfpathlineto{\pgfqpoint{2.975154in}{3.379821in}}%
\pgfpathlineto{\pgfqpoint{2.976633in}{3.377702in}}%
\pgfpathlineto{\pgfqpoint{2.977126in}{3.377702in}}%
\pgfpathlineto{\pgfqpoint{2.978606in}{3.202620in}}%
\pgfpathlineto{\pgfqpoint{2.980086in}{2.872818in}}%
\pgfpathlineto{\pgfqpoint{2.980579in}{3.379014in}}%
\pgfpathlineto{\pgfqpoint{2.981566in}{3.256530in}}%
\pgfpathlineto{\pgfqpoint{2.982552in}{3.256530in}}%
\pgfpathlineto{\pgfqpoint{2.984032in}{3.310970in}}%
\pgfpathlineto{\pgfqpoint{2.985018in}{3.310970in}}%
\pgfpathlineto{\pgfqpoint{2.986005in}{2.808448in}}%
\pgfpathlineto{\pgfqpoint{2.987484in}{3.367290in}}%
\pgfpathlineto{\pgfqpoint{2.987978in}{3.288024in}}%
\pgfpathlineto{\pgfqpoint{2.988471in}{3.325217in}}%
\pgfpathlineto{\pgfqpoint{2.989457in}{3.332945in}}%
\pgfpathlineto{\pgfqpoint{2.989950in}{3.379340in}}%
\pgfpathlineto{\pgfqpoint{2.991430in}{3.140579in}}%
\pgfpathlineto{\pgfqpoint{2.991923in}{3.140579in}}%
\pgfpathlineto{\pgfqpoint{2.992417in}{3.376237in}}%
\pgfpathlineto{\pgfqpoint{2.992910in}{3.278966in}}%
\pgfpathlineto{\pgfqpoint{2.993403in}{3.278966in}}%
\pgfpathlineto{\pgfqpoint{2.994390in}{3.272151in}}%
\pgfpathlineto{\pgfqpoint{2.994883in}{3.295373in}}%
\pgfpathlineto{\pgfqpoint{2.996362in}{3.377048in}}%
\pgfpathlineto{\pgfqpoint{2.997842in}{3.376884in}}%
\pgfpathlineto{\pgfqpoint{2.999322in}{3.334330in}}%
\pgfpathlineto{\pgfqpoint{2.999815in}{3.377684in}}%
\pgfpathlineto{\pgfqpoint{3.000802in}{3.369650in}}%
\pgfpathlineto{\pgfqpoint{3.001295in}{3.377146in}}%
\pgfpathlineto{\pgfqpoint{3.002774in}{3.364983in}}%
\pgfpathlineto{\pgfqpoint{3.004254in}{3.104544in}}%
\pgfpathlineto{\pgfqpoint{3.006227in}{3.104544in}}%
\pgfpathlineto{\pgfqpoint{3.006720in}{2.947811in}}%
\pgfpathlineto{\pgfqpoint{3.007707in}{3.317482in}}%
\pgfpathlineto{\pgfqpoint{3.008693in}{3.158349in}}%
\pgfpathlineto{\pgfqpoint{3.009186in}{2.309193in}}%
\pgfpathlineto{\pgfqpoint{3.009680in}{2.409160in}}%
\pgfpathlineto{\pgfqpoint{3.010173in}{3.351385in}}%
\pgfpathlineto{\pgfqpoint{3.011159in}{3.348733in}}%
\pgfpathlineto{\pgfqpoint{3.013626in}{3.348733in}}%
\pgfpathlineto{\pgfqpoint{3.014119in}{2.798802in}}%
\pgfpathlineto{\pgfqpoint{3.014612in}{3.217282in}}%
\pgfpathlineto{\pgfqpoint{3.015105in}{3.217282in}}%
\pgfpathlineto{\pgfqpoint{3.015598in}{3.361742in}}%
\pgfpathlineto{\pgfqpoint{3.016092in}{3.279308in}}%
\pgfpathlineto{\pgfqpoint{3.016585in}{3.279308in}}%
\pgfpathlineto{\pgfqpoint{3.017078in}{3.331613in}}%
\pgfpathlineto{\pgfqpoint{3.018065in}{3.324276in}}%
\pgfpathlineto{\pgfqpoint{3.018558in}{3.324276in}}%
\pgfpathlineto{\pgfqpoint{3.019544in}{3.133126in}}%
\pgfpathlineto{\pgfqpoint{3.020038in}{3.377169in}}%
\pgfpathlineto{\pgfqpoint{3.020531in}{3.186756in}}%
\pgfpathlineto{\pgfqpoint{3.021024in}{2.546839in}}%
\pgfpathlineto{\pgfqpoint{3.021517in}{3.377275in}}%
\pgfpathlineto{\pgfqpoint{3.022504in}{3.353978in}}%
\pgfpathlineto{\pgfqpoint{3.022997in}{3.353978in}}%
\pgfpathlineto{\pgfqpoint{3.024477in}{3.309033in}}%
\pgfpathlineto{\pgfqpoint{3.024970in}{3.309033in}}%
\pgfpathlineto{\pgfqpoint{3.025463in}{2.772074in}}%
\pgfpathlineto{\pgfqpoint{3.025956in}{3.301122in}}%
\pgfpathlineto{\pgfqpoint{3.026450in}{3.332807in}}%
\pgfpathlineto{\pgfqpoint{3.026943in}{3.313344in}}%
\pgfpathlineto{\pgfqpoint{3.028423in}{3.313344in}}%
\pgfpathlineto{\pgfqpoint{3.028916in}{3.103657in}}%
\pgfpathlineto{\pgfqpoint{3.029409in}{3.198044in}}%
\pgfpathlineto{\pgfqpoint{3.030395in}{3.224936in}}%
\pgfpathlineto{\pgfqpoint{3.030889in}{3.142772in}}%
\pgfpathlineto{\pgfqpoint{3.031875in}{3.370714in}}%
\pgfpathlineto{\pgfqpoint{3.032862in}{3.301835in}}%
\pgfpathlineto{\pgfqpoint{3.033355in}{3.312475in}}%
\pgfpathlineto{\pgfqpoint{3.033848in}{3.312475in}}%
\pgfpathlineto{\pgfqpoint{3.034341in}{3.198389in}}%
\pgfpathlineto{\pgfqpoint{3.034835in}{3.359991in}}%
\pgfpathlineto{\pgfqpoint{3.036314in}{2.885949in}}%
\pgfpathlineto{\pgfqpoint{3.037794in}{2.885949in}}%
\pgfpathlineto{\pgfqpoint{3.039274in}{3.365915in}}%
\pgfpathlineto{\pgfqpoint{3.041247in}{3.365915in}}%
\pgfpathlineto{\pgfqpoint{3.041740in}{3.311341in}}%
\pgfpathlineto{\pgfqpoint{3.042233in}{3.377668in}}%
\pgfpathlineto{\pgfqpoint{3.043713in}{3.377668in}}%
\pgfpathlineto{\pgfqpoint{3.044206in}{2.453149in}}%
\pgfpathlineto{\pgfqpoint{3.044699in}{3.190452in}}%
\pgfpathlineto{\pgfqpoint{3.046672in}{3.190452in}}%
\pgfpathlineto{\pgfqpoint{3.047659in}{3.379941in}}%
\pgfpathlineto{\pgfqpoint{3.048152in}{3.368777in}}%
\pgfpathlineto{\pgfqpoint{3.050125in}{3.368777in}}%
\pgfpathlineto{\pgfqpoint{3.051111in}{2.896838in}}%
\pgfpathlineto{\pgfqpoint{3.051604in}{2.912444in}}%
\pgfpathlineto{\pgfqpoint{3.052098in}{2.912444in}}%
\pgfpathlineto{\pgfqpoint{3.053084in}{3.336510in}}%
\pgfpathlineto{\pgfqpoint{3.053577in}{3.318217in}}%
\pgfpathlineto{\pgfqpoint{3.055057in}{3.355867in}}%
\pgfpathlineto{\pgfqpoint{3.055550in}{3.355867in}}%
\pgfpathlineto{\pgfqpoint{3.056043in}{3.358619in}}%
\pgfpathlineto{\pgfqpoint{3.056537in}{3.373812in}}%
\pgfpathlineto{\pgfqpoint{3.057523in}{3.194290in}}%
\pgfpathlineto{\pgfqpoint{3.059003in}{3.379991in}}%
\pgfpathlineto{\pgfqpoint{3.059496in}{3.379991in}}%
\pgfpathlineto{\pgfqpoint{3.059989in}{3.232143in}}%
\pgfpathlineto{\pgfqpoint{3.060483in}{3.349639in}}%
\pgfpathlineto{\pgfqpoint{3.061962in}{3.349639in}}%
\pgfpathlineto{\pgfqpoint{3.062949in}{3.249743in}}%
\pgfpathlineto{\pgfqpoint{3.064428in}{3.378331in}}%
\pgfpathlineto{\pgfqpoint{3.064922in}{3.371790in}}%
\pgfpathlineto{\pgfqpoint{3.065415in}{3.379986in}}%
\pgfpathlineto{\pgfqpoint{3.066895in}{3.197376in}}%
\pgfpathlineto{\pgfqpoint{3.069361in}{3.197376in}}%
\pgfpathlineto{\pgfqpoint{3.069854in}{3.313775in}}%
\pgfpathlineto{\pgfqpoint{3.070347in}{3.187343in}}%
\pgfpathlineto{\pgfqpoint{3.071334in}{3.162295in}}%
\pgfpathlineto{\pgfqpoint{3.072813in}{3.196961in}}%
\pgfpathlineto{\pgfqpoint{3.073800in}{3.196961in}}%
\pgfpathlineto{\pgfqpoint{3.075279in}{3.379707in}}%
\pgfpathlineto{\pgfqpoint{3.076266in}{3.379707in}}%
\pgfpathlineto{\pgfqpoint{3.077746in}{3.031849in}}%
\pgfpathlineto{\pgfqpoint{3.078239in}{2.960356in}}%
\pgfpathlineto{\pgfqpoint{3.079225in}{3.190570in}}%
\pgfpathlineto{\pgfqpoint{3.080705in}{3.011195in}}%
\pgfpathlineto{\pgfqpoint{3.081691in}{3.202217in}}%
\pgfpathlineto{\pgfqpoint{3.082185in}{3.195776in}}%
\pgfpathlineto{\pgfqpoint{3.083171in}{3.379947in}}%
\pgfpathlineto{\pgfqpoint{3.085144in}{2.802272in}}%
\pgfpathlineto{\pgfqpoint{3.086624in}{3.240426in}}%
\pgfpathlineto{\pgfqpoint{3.088103in}{3.352948in}}%
\pgfpathlineto{\pgfqpoint{3.089090in}{3.375376in}}%
\pgfpathlineto{\pgfqpoint{3.089583in}{3.264782in}}%
\pgfpathlineto{\pgfqpoint{3.090570in}{2.652490in}}%
\pgfpathlineto{\pgfqpoint{3.092543in}{3.190245in}}%
\pgfpathlineto{\pgfqpoint{3.093036in}{2.656265in}}%
\pgfpathlineto{\pgfqpoint{3.093529in}{3.339646in}}%
\pgfpathlineto{\pgfqpoint{3.094022in}{2.481511in}}%
\pgfpathlineto{\pgfqpoint{3.094515in}{3.245415in}}%
\pgfpathlineto{\pgfqpoint{3.095009in}{3.371705in}}%
\pgfpathlineto{\pgfqpoint{3.095995in}{3.362780in}}%
\pgfpathlineto{\pgfqpoint{3.096488in}{3.004925in}}%
\pgfpathlineto{\pgfqpoint{3.096982in}{3.011449in}}%
\pgfpathlineto{\pgfqpoint{3.098461in}{3.366214in}}%
\pgfpathlineto{\pgfqpoint{3.099448in}{3.314835in}}%
\pgfpathlineto{\pgfqpoint{3.099941in}{3.379883in}}%
\pgfpathlineto{\pgfqpoint{3.100927in}{3.369464in}}%
\pgfpathlineto{\pgfqpoint{3.101421in}{3.369464in}}%
\pgfpathlineto{\pgfqpoint{3.102900in}{3.354547in}}%
\pgfpathlineto{\pgfqpoint{3.103887in}{3.354547in}}%
\pgfpathlineto{\pgfqpoint{3.104380in}{3.368715in}}%
\pgfpathlineto{\pgfqpoint{3.105367in}{3.223615in}}%
\pgfpathlineto{\pgfqpoint{3.106846in}{3.342453in}}%
\pgfpathlineto{\pgfqpoint{3.107339in}{3.231607in}}%
\pgfpathlineto{\pgfqpoint{3.107833in}{3.366629in}}%
\pgfpathlineto{\pgfqpoint{3.108326in}{3.366629in}}%
\pgfpathlineto{\pgfqpoint{3.108819in}{3.365441in}}%
\pgfpathlineto{\pgfqpoint{3.109806in}{3.374777in}}%
\pgfpathlineto{\pgfqpoint{3.110299in}{3.251495in}}%
\pgfpathlineto{\pgfqpoint{3.110792in}{3.363259in}}%
\pgfpathlineto{\pgfqpoint{3.111285in}{3.363259in}}%
\pgfpathlineto{\pgfqpoint{3.111779in}{3.283179in}}%
\pgfpathlineto{\pgfqpoint{3.112272in}{3.379889in}}%
\pgfpathlineto{\pgfqpoint{3.112765in}{3.379889in}}%
\pgfpathlineto{\pgfqpoint{3.114245in}{3.160536in}}%
\pgfpathlineto{\pgfqpoint{3.115724in}{3.380000in}}%
\pgfpathlineto{\pgfqpoint{3.116218in}{3.380000in}}%
\pgfpathlineto{\pgfqpoint{3.117697in}{3.229617in}}%
\pgfpathlineto{\pgfqpoint{3.118191in}{3.229617in}}%
\pgfpathlineto{\pgfqpoint{3.118684in}{3.259367in}}%
\pgfpathlineto{\pgfqpoint{3.120163in}{3.377948in}}%
\pgfpathlineto{\pgfqpoint{3.121643in}{3.052459in}}%
\pgfpathlineto{\pgfqpoint{3.124603in}{3.375874in}}%
\pgfpathlineto{\pgfqpoint{3.125096in}{2.868482in}}%
\pgfpathlineto{\pgfqpoint{3.125589in}{3.373227in}}%
\pgfpathlineto{\pgfqpoint{3.126082in}{3.064361in}}%
\pgfpathlineto{\pgfqpoint{3.126575in}{3.374850in}}%
\pgfpathlineto{\pgfqpoint{3.127562in}{3.374850in}}%
\pgfpathlineto{\pgfqpoint{3.128055in}{3.371572in}}%
\pgfpathlineto{\pgfqpoint{3.129535in}{3.207345in}}%
\pgfpathlineto{\pgfqpoint{3.130521in}{3.379826in}}%
\pgfpathlineto{\pgfqpoint{3.132001in}{3.203727in}}%
\pgfpathlineto{\pgfqpoint{3.132494in}{3.203727in}}%
\pgfpathlineto{\pgfqpoint{3.133974in}{3.168271in}}%
\pgfpathlineto{\pgfqpoint{3.134960in}{2.969402in}}%
\pgfpathlineto{\pgfqpoint{3.136440in}{3.168315in}}%
\pgfpathlineto{\pgfqpoint{3.136933in}{3.207684in}}%
\pgfpathlineto{\pgfqpoint{3.137427in}{3.364528in}}%
\pgfpathlineto{\pgfqpoint{3.137920in}{3.319349in}}%
\pgfpathlineto{\pgfqpoint{3.138413in}{3.319349in}}%
\pgfpathlineto{\pgfqpoint{3.138906in}{3.373920in}}%
\pgfpathlineto{\pgfqpoint{3.139400in}{3.347134in}}%
\pgfpathlineto{\pgfqpoint{3.140879in}{2.921849in}}%
\pgfpathlineto{\pgfqpoint{3.142359in}{3.234049in}}%
\pgfpathlineto{\pgfqpoint{3.144332in}{3.369600in}}%
\pgfpathlineto{\pgfqpoint{3.145812in}{3.083765in}}%
\pgfpathlineto{\pgfqpoint{3.147291in}{3.373924in}}%
\pgfpathlineto{\pgfqpoint{3.147784in}{1.881196in}}%
\pgfpathlineto{\pgfqpoint{3.148278in}{3.378683in}}%
\pgfpathlineto{\pgfqpoint{3.149264in}{3.378683in}}%
\pgfpathlineto{\pgfqpoint{3.150744in}{3.347822in}}%
\pgfpathlineto{\pgfqpoint{3.151237in}{3.347822in}}%
\pgfpathlineto{\pgfqpoint{3.152224in}{3.376624in}}%
\pgfpathlineto{\pgfqpoint{3.153210in}{2.929791in}}%
\pgfpathlineto{\pgfqpoint{3.153703in}{3.267389in}}%
\pgfpathlineto{\pgfqpoint{3.155183in}{3.380000in}}%
\pgfpathlineto{\pgfqpoint{3.157649in}{3.291205in}}%
\pgfpathlineto{\pgfqpoint{3.158142in}{3.287785in}}%
\pgfpathlineto{\pgfqpoint{3.158636in}{3.364871in}}%
\pgfpathlineto{\pgfqpoint{3.159129in}{1.677894in}}%
\pgfpathlineto{\pgfqpoint{3.159622in}{3.014133in}}%
\pgfpathlineto{\pgfqpoint{3.160115in}{2.830402in}}%
\pgfpathlineto{\pgfqpoint{3.160608in}{2.208717in}}%
\pgfpathlineto{\pgfqpoint{3.161102in}{2.855236in}}%
\pgfpathlineto{\pgfqpoint{3.163568in}{3.379236in}}%
\pgfpathlineto{\pgfqpoint{3.164554in}{3.379236in}}%
\pgfpathlineto{\pgfqpoint{3.166034in}{3.318242in}}%
\pgfpathlineto{\pgfqpoint{3.167514in}{3.379090in}}%
\pgfpathlineto{\pgfqpoint{3.168007in}{3.379472in}}%
\pgfpathlineto{\pgfqpoint{3.169487in}{3.250373in}}%
\pgfpathlineto{\pgfqpoint{3.169980in}{3.250373in}}%
\pgfpathlineto{\pgfqpoint{3.170473in}{3.375124in}}%
\pgfpathlineto{\pgfqpoint{3.170966in}{3.241427in}}%
\pgfpathlineto{\pgfqpoint{3.172446in}{3.252948in}}%
\pgfpathlineto{\pgfqpoint{3.172939in}{2.779546in}}%
\pgfpathlineto{\pgfqpoint{3.173432in}{3.313631in}}%
\pgfpathlineto{\pgfqpoint{3.173926in}{3.356568in}}%
\pgfpathlineto{\pgfqpoint{3.175405in}{3.075502in}}%
\pgfpathlineto{\pgfqpoint{3.176392in}{3.379569in}}%
\pgfpathlineto{\pgfqpoint{3.176885in}{3.355977in}}%
\pgfpathlineto{\pgfqpoint{3.177378in}{3.379057in}}%
\pgfpathlineto{\pgfqpoint{3.178365in}{2.687137in}}%
\pgfpathlineto{\pgfqpoint{3.179844in}{3.057756in}}%
\pgfpathlineto{\pgfqpoint{3.180338in}{3.057756in}}%
\pgfpathlineto{\pgfqpoint{3.181324in}{3.020777in}}%
\pgfpathlineto{\pgfqpoint{3.181817in}{3.367551in}}%
\pgfpathlineto{\pgfqpoint{3.182804in}{3.367551in}}%
\pgfpathlineto{\pgfqpoint{3.183297in}{3.364921in}}%
\pgfpathlineto{\pgfqpoint{3.184284in}{2.681969in}}%
\pgfpathlineto{\pgfqpoint{3.184777in}{2.877425in}}%
\pgfpathlineto{\pgfqpoint{3.185763in}{2.877425in}}%
\pgfpathlineto{\pgfqpoint{3.188229in}{3.352679in}}%
\pgfpathlineto{\pgfqpoint{3.189216in}{3.352679in}}%
\pgfpathlineto{\pgfqpoint{3.189709in}{3.349022in}}%
\pgfpathlineto{\pgfqpoint{3.190202in}{3.314917in}}%
\pgfpathlineto{\pgfqpoint{3.190696in}{2.954564in}}%
\pgfpathlineto{\pgfqpoint{3.191189in}{3.353259in}}%
\pgfpathlineto{\pgfqpoint{3.191682in}{3.379123in}}%
\pgfpathlineto{\pgfqpoint{3.192175in}{3.357857in}}%
\pgfpathlineto{\pgfqpoint{3.193655in}{3.345602in}}%
\pgfpathlineto{\pgfqpoint{3.195628in}{3.345602in}}%
\pgfpathlineto{\pgfqpoint{3.197601in}{3.376954in}}%
\pgfpathlineto{\pgfqpoint{3.199080in}{3.343130in}}%
\pgfpathlineto{\pgfqpoint{3.201547in}{0.968361in}}%
\pgfpathlineto{\pgfqpoint{3.203026in}{3.379357in}}%
\pgfpathlineto{\pgfqpoint{3.203520in}{3.321064in}}%
\pgfpathlineto{\pgfqpoint{3.204013in}{3.089163in}}%
\pgfpathlineto{\pgfqpoint{3.204506in}{3.191281in}}%
\pgfpathlineto{\pgfqpoint{3.207465in}{3.376459in}}%
\pgfpathlineto{\pgfqpoint{3.207959in}{3.110917in}}%
\pgfpathlineto{\pgfqpoint{3.208452in}{3.374273in}}%
\pgfpathlineto{\pgfqpoint{3.209438in}{3.298020in}}%
\pgfpathlineto{\pgfqpoint{3.209932in}{2.740608in}}%
\pgfpathlineto{\pgfqpoint{3.210425in}{2.977879in}}%
\pgfpathlineto{\pgfqpoint{3.210918in}{2.977879in}}%
\pgfpathlineto{\pgfqpoint{3.211411in}{3.018969in}}%
\pgfpathlineto{\pgfqpoint{3.211904in}{3.335537in}}%
\pgfpathlineto{\pgfqpoint{3.212398in}{2.971920in}}%
\pgfpathlineto{\pgfqpoint{3.213384in}{3.377220in}}%
\pgfpathlineto{\pgfqpoint{3.214371in}{3.364420in}}%
\pgfpathlineto{\pgfqpoint{3.214864in}{3.157856in}}%
\pgfpathlineto{\pgfqpoint{3.215357in}{3.373374in}}%
\pgfpathlineto{\pgfqpoint{3.216837in}{3.358281in}}%
\pgfpathlineto{\pgfqpoint{3.217330in}{3.358281in}}%
\pgfpathlineto{\pgfqpoint{3.217823in}{3.156239in}}%
\pgfpathlineto{\pgfqpoint{3.218316in}{3.280567in}}%
\pgfpathlineto{\pgfqpoint{3.220289in}{3.355954in}}%
\pgfpathlineto{\pgfqpoint{3.221276in}{2.878102in}}%
\pgfpathlineto{\pgfqpoint{3.222262in}{3.379671in}}%
\pgfpathlineto{\pgfqpoint{3.222756in}{3.226607in}}%
\pgfpathlineto{\pgfqpoint{3.223249in}{3.298444in}}%
\pgfpathlineto{\pgfqpoint{3.224235in}{3.298444in}}%
\pgfpathlineto{\pgfqpoint{3.226701in}{3.371156in}}%
\pgfpathlineto{\pgfqpoint{3.227195in}{3.234497in}}%
\pgfpathlineto{\pgfqpoint{3.227688in}{3.350876in}}%
\pgfpathlineto{\pgfqpoint{3.228181in}{3.316712in}}%
\pgfpathlineto{\pgfqpoint{3.228674in}{3.120969in}}%
\pgfpathlineto{\pgfqpoint{3.229168in}{3.251972in}}%
\pgfpathlineto{\pgfqpoint{3.231141in}{3.164327in}}%
\pgfpathlineto{\pgfqpoint{3.231634in}{2.570120in}}%
\pgfpathlineto{\pgfqpoint{3.232127in}{3.211846in}}%
\pgfpathlineto{\pgfqpoint{3.232620in}{3.211846in}}%
\pgfpathlineto{\pgfqpoint{3.234100in}{3.323641in}}%
\pgfpathlineto{\pgfqpoint{3.235580in}{3.259026in}}%
\pgfpathlineto{\pgfqpoint{3.237059in}{3.379957in}}%
\pgfpathlineto{\pgfqpoint{3.238046in}{3.379957in}}%
\pgfpathlineto{\pgfqpoint{3.239525in}{2.640713in}}%
\pgfpathlineto{\pgfqpoint{3.241005in}{3.378904in}}%
\pgfpathlineto{\pgfqpoint{3.241498in}{3.280785in}}%
\pgfpathlineto{\pgfqpoint{3.241992in}{3.356439in}}%
\pgfpathlineto{\pgfqpoint{3.242485in}{3.319744in}}%
\pgfpathlineto{\pgfqpoint{3.242978in}{3.375340in}}%
\pgfpathlineto{\pgfqpoint{3.243471in}{3.350369in}}%
\pgfpathlineto{\pgfqpoint{3.243965in}{3.350369in}}%
\pgfpathlineto{\pgfqpoint{3.245444in}{3.375074in}}%
\pgfpathlineto{\pgfqpoint{3.245937in}{3.362595in}}%
\pgfpathlineto{\pgfqpoint{3.246431in}{3.371829in}}%
\pgfpathlineto{\pgfqpoint{3.247417in}{3.371829in}}%
\pgfpathlineto{\pgfqpoint{3.248404in}{3.319813in}}%
\pgfpathlineto{\pgfqpoint{3.248897in}{3.119122in}}%
\pgfpathlineto{\pgfqpoint{3.249390in}{3.372962in}}%
\pgfpathlineto{\pgfqpoint{3.249883in}{3.217633in}}%
\pgfpathlineto{\pgfqpoint{3.251856in}{3.379120in}}%
\pgfpathlineto{\pgfqpoint{3.253336in}{3.292720in}}%
\pgfpathlineto{\pgfqpoint{3.253829in}{3.292720in}}%
\pgfpathlineto{\pgfqpoint{3.255309in}{2.935211in}}%
\pgfpathlineto{\pgfqpoint{3.256295in}{3.273307in}}%
\pgfpathlineto{\pgfqpoint{3.257775in}{3.267226in}}%
\pgfpathlineto{\pgfqpoint{3.258761in}{3.337153in}}%
\pgfpathlineto{\pgfqpoint{3.260241in}{3.258421in}}%
\pgfpathlineto{\pgfqpoint{3.260734in}{3.258421in}}%
\pgfpathlineto{\pgfqpoint{3.262214in}{3.060339in}}%
\pgfpathlineto{\pgfqpoint{3.263694in}{3.379761in}}%
\pgfpathlineto{\pgfqpoint{3.264187in}{3.379761in}}%
\pgfpathlineto{\pgfqpoint{3.265173in}{3.112324in}}%
\pgfpathlineto{\pgfqpoint{3.266653in}{3.333963in}}%
\pgfpathlineto{\pgfqpoint{3.268133in}{3.333963in}}%
\pgfpathlineto{\pgfqpoint{3.268626in}{3.121641in}}%
\pgfpathlineto{\pgfqpoint{3.269119in}{3.222176in}}%
\pgfpathlineto{\pgfqpoint{3.269613in}{3.222176in}}%
\pgfpathlineto{\pgfqpoint{3.270599in}{3.379999in}}%
\pgfpathlineto{\pgfqpoint{3.271092in}{3.155537in}}%
\pgfpathlineto{\pgfqpoint{3.271585in}{3.356933in}}%
\pgfpathlineto{\pgfqpoint{3.272079in}{3.319935in}}%
\pgfpathlineto{\pgfqpoint{3.272572in}{3.363950in}}%
\pgfpathlineto{\pgfqpoint{3.274052in}{2.604083in}}%
\pgfpathlineto{\pgfqpoint{3.274545in}{3.365577in}}%
\pgfpathlineto{\pgfqpoint{3.275531in}{3.180569in}}%
\pgfpathlineto{\pgfqpoint{3.276518in}{3.180569in}}%
\pgfpathlineto{\pgfqpoint{3.277011in}{3.316960in}}%
\pgfpathlineto{\pgfqpoint{3.277504in}{3.221304in}}%
\pgfpathlineto{\pgfqpoint{3.277997in}{3.221304in}}%
\pgfpathlineto{\pgfqpoint{3.279477in}{3.379855in}}%
\pgfpathlineto{\pgfqpoint{3.280464in}{2.510404in}}%
\pgfpathlineto{\pgfqpoint{3.280957in}{3.327596in}}%
\pgfpathlineto{\pgfqpoint{3.281943in}{3.310133in}}%
\pgfpathlineto{\pgfqpoint{3.282930in}{3.310133in}}%
\pgfpathlineto{\pgfqpoint{3.283423in}{3.198262in}}%
\pgfpathlineto{\pgfqpoint{3.283916in}{3.348713in}}%
\pgfpathlineto{\pgfqpoint{3.284409in}{3.194938in}}%
\pgfpathlineto{\pgfqpoint{3.284903in}{3.321005in}}%
\pgfpathlineto{\pgfqpoint{3.285396in}{3.321005in}}%
\pgfpathlineto{\pgfqpoint{3.286876in}{3.152173in}}%
\pgfpathlineto{\pgfqpoint{3.287862in}{3.366675in}}%
\pgfpathlineto{\pgfqpoint{3.288355in}{3.078964in}}%
\pgfpathlineto{\pgfqpoint{3.289342in}{3.136392in}}%
\pgfpathlineto{\pgfqpoint{3.290328in}{3.136392in}}%
\pgfpathlineto{\pgfqpoint{3.290821in}{3.061641in}}%
\pgfpathlineto{\pgfqpoint{3.292301in}{3.182968in}}%
\pgfpathlineto{\pgfqpoint{3.293288in}{3.182968in}}%
\pgfpathlineto{\pgfqpoint{3.293781in}{3.377992in}}%
\pgfpathlineto{\pgfqpoint{3.294274in}{3.172069in}}%
\pgfpathlineto{\pgfqpoint{3.295261in}{3.172069in}}%
\pgfpathlineto{\pgfqpoint{3.296247in}{3.145183in}}%
\pgfpathlineto{\pgfqpoint{3.297233in}{2.892892in}}%
\pgfpathlineto{\pgfqpoint{3.298713in}{3.379997in}}%
\pgfpathlineto{\pgfqpoint{3.299700in}{3.379997in}}%
\pgfpathlineto{\pgfqpoint{3.300193in}{3.326785in}}%
\pgfpathlineto{\pgfqpoint{3.300686in}{3.379901in}}%
\pgfpathlineto{\pgfqpoint{3.301179in}{3.379901in}}%
\pgfpathlineto{\pgfqpoint{3.301673in}{2.729045in}}%
\pgfpathlineto{\pgfqpoint{3.302166in}{2.909626in}}%
\pgfpathlineto{\pgfqpoint{3.303152in}{3.354081in}}%
\pgfpathlineto{\pgfqpoint{3.304139in}{3.058921in}}%
\pgfpathlineto{\pgfqpoint{3.305618in}{3.362856in}}%
\pgfpathlineto{\pgfqpoint{3.306605in}{3.362856in}}%
\pgfpathlineto{\pgfqpoint{3.308578in}{2.544139in}}%
\pgfpathlineto{\pgfqpoint{3.309071in}{3.371060in}}%
\pgfpathlineto{\pgfqpoint{3.310057in}{3.183491in}}%
\pgfpathlineto{\pgfqpoint{3.311537in}{3.372269in}}%
\pgfpathlineto{\pgfqpoint{3.312524in}{3.372269in}}%
\pgfpathlineto{\pgfqpoint{3.313017in}{3.100654in}}%
\pgfpathlineto{\pgfqpoint{3.313510in}{3.379961in}}%
\pgfpathlineto{\pgfqpoint{3.314990in}{3.193998in}}%
\pgfpathlineto{\pgfqpoint{3.315483in}{3.379786in}}%
\pgfpathlineto{\pgfqpoint{3.315976in}{3.371036in}}%
\pgfpathlineto{\pgfqpoint{3.317456in}{3.129090in}}%
\pgfpathlineto{\pgfqpoint{3.317949in}{3.129090in}}%
\pgfpathlineto{\pgfqpoint{3.319429in}{3.314436in}}%
\pgfpathlineto{\pgfqpoint{3.319922in}{3.314436in}}%
\pgfpathlineto{\pgfqpoint{3.320415in}{3.259402in}}%
\pgfpathlineto{\pgfqpoint{3.320909in}{3.300207in}}%
\pgfpathlineto{\pgfqpoint{3.322388in}{3.300207in}}%
\pgfpathlineto{\pgfqpoint{3.322881in}{2.205941in}}%
\pgfpathlineto{\pgfqpoint{3.323375in}{3.075628in}}%
\pgfpathlineto{\pgfqpoint{3.324361in}{3.350614in}}%
\pgfpathlineto{\pgfqpoint{3.325841in}{3.245892in}}%
\pgfpathlineto{\pgfqpoint{3.326334in}{3.245892in}}%
\pgfpathlineto{\pgfqpoint{3.327814in}{3.340522in}}%
\pgfpathlineto{\pgfqpoint{3.329787in}{2.067875in}}%
\pgfpathlineto{\pgfqpoint{3.330773in}{2.845348in}}%
\pgfpathlineto{\pgfqpoint{3.331266in}{2.320917in}}%
\pgfpathlineto{\pgfqpoint{3.332746in}{3.375934in}}%
\pgfpathlineto{\pgfqpoint{3.334226in}{3.375934in}}%
\pgfpathlineto{\pgfqpoint{3.335212in}{3.379125in}}%
\pgfpathlineto{\pgfqpoint{3.335706in}{3.249245in}}%
\pgfpathlineto{\pgfqpoint{3.336199in}{3.252776in}}%
\pgfpathlineto{\pgfqpoint{3.336692in}{3.375683in}}%
\pgfpathlineto{\pgfqpoint{3.337185in}{2.903966in}}%
\pgfpathlineto{\pgfqpoint{3.337678in}{3.326258in}}%
\pgfpathlineto{\pgfqpoint{3.338172in}{3.326258in}}%
\pgfpathlineto{\pgfqpoint{3.339651in}{3.375110in}}%
\pgfpathlineto{\pgfqpoint{3.340145in}{3.375110in}}%
\pgfpathlineto{\pgfqpoint{3.340638in}{3.378871in}}%
\pgfpathlineto{\pgfqpoint{3.342118in}{2.700438in}}%
\pgfpathlineto{\pgfqpoint{3.342611in}{3.260019in}}%
\pgfpathlineto{\pgfqpoint{3.343104in}{2.959355in}}%
\pgfpathlineto{\pgfqpoint{3.343597in}{2.915392in}}%
\pgfpathlineto{\pgfqpoint{3.344090in}{3.320791in}}%
\pgfpathlineto{\pgfqpoint{3.344584in}{2.831125in}}%
\pgfpathlineto{\pgfqpoint{3.345077in}{3.221597in}}%
\pgfpathlineto{\pgfqpoint{3.346063in}{3.366644in}}%
\pgfpathlineto{\pgfqpoint{3.347543in}{3.133576in}}%
\pgfpathlineto{\pgfqpoint{3.348036in}{3.133576in}}%
\pgfpathlineto{\pgfqpoint{3.349023in}{2.808941in}}%
\pgfpathlineto{\pgfqpoint{3.350502in}{3.255588in}}%
\pgfpathlineto{\pgfqpoint{3.351489in}{3.255588in}}%
\pgfpathlineto{\pgfqpoint{3.352969in}{3.306157in}}%
\pgfpathlineto{\pgfqpoint{3.353955in}{1.677509in}}%
\pgfpathlineto{\pgfqpoint{3.355435in}{3.317577in}}%
\pgfpathlineto{\pgfqpoint{3.356914in}{3.317577in}}%
\pgfpathlineto{\pgfqpoint{3.357901in}{3.083402in}}%
\pgfpathlineto{\pgfqpoint{3.359381in}{3.376930in}}%
\pgfpathlineto{\pgfqpoint{3.359874in}{3.376930in}}%
\pgfpathlineto{\pgfqpoint{3.360860in}{2.946261in}}%
\pgfpathlineto{\pgfqpoint{3.361354in}{3.167856in}}%
\pgfpathlineto{\pgfqpoint{3.362833in}{1.818056in}}%
\pgfpathlineto{\pgfqpoint{3.364313in}{3.346045in}}%
\pgfpathlineto{\pgfqpoint{3.365793in}{3.353421in}}%
\pgfpathlineto{\pgfqpoint{3.366779in}{3.353421in}}%
\pgfpathlineto{\pgfqpoint{3.367766in}{2.843917in}}%
\pgfpathlineto{\pgfqpoint{3.368752in}{3.243648in}}%
\pgfpathlineto{\pgfqpoint{3.369738in}{2.485729in}}%
\pgfpathlineto{\pgfqpoint{3.371218in}{3.369572in}}%
\pgfpathlineto{\pgfqpoint{3.375164in}{3.369572in}}%
\pgfpathlineto{\pgfqpoint{3.375657in}{3.351193in}}%
\pgfpathlineto{\pgfqpoint{3.377137in}{2.976026in}}%
\pgfpathlineto{\pgfqpoint{3.378617in}{3.379738in}}%
\pgfpathlineto{\pgfqpoint{3.380096in}{3.134957in}}%
\pgfpathlineto{\pgfqpoint{3.381083in}{3.134957in}}%
\pgfpathlineto{\pgfqpoint{3.382562in}{3.324961in}}%
\pgfpathlineto{\pgfqpoint{3.383549in}{3.324961in}}%
\pgfpathlineto{\pgfqpoint{3.385029in}{2.865859in}}%
\pgfpathlineto{\pgfqpoint{3.385522in}{2.332359in}}%
\pgfpathlineto{\pgfqpoint{3.387002in}{3.337804in}}%
\pgfpathlineto{\pgfqpoint{3.387988in}{3.337804in}}%
\pgfpathlineto{\pgfqpoint{3.388974in}{3.295101in}}%
\pgfpathlineto{\pgfqpoint{3.389468in}{3.304456in}}%
\pgfpathlineto{\pgfqpoint{3.389961in}{3.371311in}}%
\pgfpathlineto{\pgfqpoint{3.391441in}{2.652774in}}%
\pgfpathlineto{\pgfqpoint{3.391934in}{3.364748in}}%
\pgfpathlineto{\pgfqpoint{3.392920in}{3.197514in}}%
\pgfpathlineto{\pgfqpoint{3.393414in}{3.321497in}}%
\pgfpathlineto{\pgfqpoint{3.393907in}{3.088636in}}%
\pgfpathlineto{\pgfqpoint{3.394400in}{3.240332in}}%
\pgfpathlineto{\pgfqpoint{3.394893in}{3.240332in}}%
\pgfpathlineto{\pgfqpoint{3.395880in}{3.192479in}}%
\pgfpathlineto{\pgfqpoint{3.396866in}{2.950639in}}%
\pgfpathlineto{\pgfqpoint{3.398346in}{3.355100in}}%
\pgfpathlineto{\pgfqpoint{3.401798in}{3.355100in}}%
\pgfpathlineto{\pgfqpoint{3.402292in}{3.123876in}}%
\pgfpathlineto{\pgfqpoint{3.402785in}{3.279825in}}%
\pgfpathlineto{\pgfqpoint{3.403278in}{3.279825in}}%
\pgfpathlineto{\pgfqpoint{3.404758in}{3.379212in}}%
\pgfpathlineto{\pgfqpoint{3.405251in}{3.190760in}}%
\pgfpathlineto{\pgfqpoint{3.405744in}{3.337223in}}%
\pgfpathlineto{\pgfqpoint{3.406238in}{3.337223in}}%
\pgfpathlineto{\pgfqpoint{3.407224in}{3.363961in}}%
\pgfpathlineto{\pgfqpoint{3.408210in}{3.227378in}}%
\pgfpathlineto{\pgfqpoint{3.409197in}{3.376562in}}%
\pgfpathlineto{\pgfqpoint{3.410677in}{3.245761in}}%
\pgfpathlineto{\pgfqpoint{3.411170in}{3.245761in}}%
\pgfpathlineto{\pgfqpoint{3.412156in}{2.821818in}}%
\pgfpathlineto{\pgfqpoint{3.413636in}{3.231969in}}%
\pgfpathlineto{\pgfqpoint{3.414129in}{3.145066in}}%
\pgfpathlineto{\pgfqpoint{3.414622in}{3.250760in}}%
\pgfpathlineto{\pgfqpoint{3.415116in}{3.250760in}}%
\pgfpathlineto{\pgfqpoint{3.415609in}{3.271026in}}%
\pgfpathlineto{\pgfqpoint{3.416102in}{3.181110in}}%
\pgfpathlineto{\pgfqpoint{3.417582in}{3.353212in}}%
\pgfpathlineto{\pgfqpoint{3.418568in}{3.353212in}}%
\pgfpathlineto{\pgfqpoint{3.419062in}{2.986197in}}%
\pgfpathlineto{\pgfqpoint{3.419555in}{3.353777in}}%
\pgfpathlineto{\pgfqpoint{3.420541in}{3.353777in}}%
\pgfpathlineto{\pgfqpoint{3.421034in}{3.377015in}}%
\pgfpathlineto{\pgfqpoint{3.421528in}{3.256306in}}%
\pgfpathlineto{\pgfqpoint{3.422021in}{3.362602in}}%
\pgfpathlineto{\pgfqpoint{3.422514in}{3.362602in}}%
\pgfpathlineto{\pgfqpoint{3.423007in}{2.388963in}}%
\pgfpathlineto{\pgfqpoint{3.423501in}{2.913541in}}%
\pgfpathlineto{\pgfqpoint{3.423994in}{2.913541in}}%
\pgfpathlineto{\pgfqpoint{3.424487in}{3.291585in}}%
\pgfpathlineto{\pgfqpoint{3.424980in}{3.237399in}}%
\pgfpathlineto{\pgfqpoint{3.426460in}{2.912357in}}%
\pgfpathlineto{\pgfqpoint{3.427446in}{3.358932in}}%
\pgfpathlineto{\pgfqpoint{3.428433in}{3.299311in}}%
\pgfpathlineto{\pgfqpoint{3.428926in}{3.299311in}}%
\pgfpathlineto{\pgfqpoint{3.429913in}{2.986054in}}%
\pgfpathlineto{\pgfqpoint{3.430406in}{3.375385in}}%
\pgfpathlineto{\pgfqpoint{3.431392in}{3.337412in}}%
\pgfpathlineto{\pgfqpoint{3.431886in}{3.343424in}}%
\pgfpathlineto{\pgfqpoint{3.432872in}{3.287094in}}%
\pgfpathlineto{\pgfqpoint{3.434352in}{3.344870in}}%
\pgfpathlineto{\pgfqpoint{3.434845in}{3.289674in}}%
\pgfpathlineto{\pgfqpoint{3.435338in}{3.102115in}}%
\pgfpathlineto{\pgfqpoint{3.435831in}{3.379567in}}%
\pgfpathlineto{\pgfqpoint{3.436818in}{3.318276in}}%
\pgfpathlineto{\pgfqpoint{3.438298in}{3.311737in}}%
\pgfpathlineto{\pgfqpoint{3.438791in}{3.375690in}}%
\pgfpathlineto{\pgfqpoint{3.439284in}{3.324402in}}%
\pgfpathlineto{\pgfqpoint{3.440271in}{3.324402in}}%
\pgfpathlineto{\pgfqpoint{3.441257in}{2.825122in}}%
\pgfpathlineto{\pgfqpoint{3.442243in}{3.364790in}}%
\pgfpathlineto{\pgfqpoint{3.442737in}{3.332660in}}%
\pgfpathlineto{\pgfqpoint{3.443230in}{3.332660in}}%
\pgfpathlineto{\pgfqpoint{3.444710in}{3.208856in}}%
\pgfpathlineto{\pgfqpoint{3.445203in}{3.208856in}}%
\pgfpathlineto{\pgfqpoint{3.446189in}{3.379912in}}%
\pgfpathlineto{\pgfqpoint{3.446683in}{3.346789in}}%
\pgfpathlineto{\pgfqpoint{3.447669in}{2.939179in}}%
\pgfpathlineto{\pgfqpoint{3.449149in}{3.376803in}}%
\pgfpathlineto{\pgfqpoint{3.449642in}{3.313584in}}%
\pgfpathlineto{\pgfqpoint{3.450135in}{3.377754in}}%
\pgfpathlineto{\pgfqpoint{3.450628in}{3.379970in}}%
\pgfpathlineto{\pgfqpoint{3.451122in}{2.814509in}}%
\pgfpathlineto{\pgfqpoint{3.451615in}{3.180012in}}%
\pgfpathlineto{\pgfqpoint{3.452108in}{3.180012in}}%
\pgfpathlineto{\pgfqpoint{3.453095in}{3.378616in}}%
\pgfpathlineto{\pgfqpoint{3.453588in}{3.363674in}}%
\pgfpathlineto{\pgfqpoint{3.454574in}{3.363674in}}%
\pgfpathlineto{\pgfqpoint{3.455067in}{2.850066in}}%
\pgfpathlineto{\pgfqpoint{3.455561in}{3.306599in}}%
\pgfpathlineto{\pgfqpoint{3.457040in}{3.246673in}}%
\pgfpathlineto{\pgfqpoint{3.457534in}{3.246673in}}%
\pgfpathlineto{\pgfqpoint{3.458027in}{3.376916in}}%
\pgfpathlineto{\pgfqpoint{3.458520in}{3.231848in}}%
\pgfpathlineto{\pgfqpoint{3.459013in}{3.231848in}}%
\pgfpathlineto{\pgfqpoint{3.459507in}{3.294919in}}%
\pgfpathlineto{\pgfqpoint{3.460000in}{3.203735in}}%
\pgfpathlineto{\pgfqpoint{3.460493in}{3.379807in}}%
\pgfpathlineto{\pgfqpoint{3.460986in}{3.290467in}}%
\pgfpathlineto{\pgfqpoint{3.461973in}{3.310177in}}%
\pgfpathlineto{\pgfqpoint{3.462466in}{2.683875in}}%
\pgfpathlineto{\pgfqpoint{3.462959in}{3.342966in}}%
\pgfpathlineto{\pgfqpoint{3.463946in}{3.356015in}}%
\pgfpathlineto{\pgfqpoint{3.465919in}{2.905641in}}%
\pgfpathlineto{\pgfqpoint{3.468385in}{3.351196in}}%
\pgfpathlineto{\pgfqpoint{3.468878in}{3.351196in}}%
\pgfpathlineto{\pgfqpoint{3.470358in}{3.369283in}}%
\pgfpathlineto{\pgfqpoint{3.471344in}{3.369283in}}%
\pgfpathlineto{\pgfqpoint{3.472824in}{3.304892in}}%
\pgfpathlineto{\pgfqpoint{3.474303in}{3.377209in}}%
\pgfpathlineto{\pgfqpoint{3.475783in}{3.374209in}}%
\pgfpathlineto{\pgfqpoint{3.476276in}{3.193242in}}%
\pgfpathlineto{\pgfqpoint{3.476770in}{3.250980in}}%
\pgfpathlineto{\pgfqpoint{3.477263in}{3.250980in}}%
\pgfpathlineto{\pgfqpoint{3.477756in}{2.834265in}}%
\pgfpathlineto{\pgfqpoint{3.478249in}{3.325789in}}%
\pgfpathlineto{\pgfqpoint{3.480715in}{3.325789in}}%
\pgfpathlineto{\pgfqpoint{3.483182in}{2.927376in}}%
\pgfpathlineto{\pgfqpoint{3.484168in}{3.360923in}}%
\pgfpathlineto{\pgfqpoint{3.484661in}{3.235417in}}%
\pgfpathlineto{\pgfqpoint{3.486141in}{3.353850in}}%
\pgfpathlineto{\pgfqpoint{3.487127in}{3.334474in}}%
\pgfpathlineto{\pgfqpoint{3.487621in}{3.248522in}}%
\pgfpathlineto{\pgfqpoint{3.488607in}{2.388780in}}%
\pgfpathlineto{\pgfqpoint{3.490087in}{3.328637in}}%
\pgfpathlineto{\pgfqpoint{3.490580in}{3.328637in}}%
\pgfpathlineto{\pgfqpoint{3.492060in}{2.684646in}}%
\pgfpathlineto{\pgfqpoint{3.493539in}{3.154931in}}%
\pgfpathlineto{\pgfqpoint{3.495512in}{3.154931in}}%
\pgfpathlineto{\pgfqpoint{3.496992in}{3.377532in}}%
\pgfpathlineto{\pgfqpoint{3.498472in}{3.312769in}}%
\pgfpathlineto{\pgfqpoint{3.498965in}{3.370557in}}%
\pgfpathlineto{\pgfqpoint{3.499458in}{3.310166in}}%
\pgfpathlineto{\pgfqpoint{3.499951in}{3.271152in}}%
\pgfpathlineto{\pgfqpoint{3.500938in}{3.373145in}}%
\pgfpathlineto{\pgfqpoint{3.501431in}{3.248015in}}%
\pgfpathlineto{\pgfqpoint{3.501924in}{3.349779in}}%
\pgfpathlineto{\pgfqpoint{3.502418in}{3.349779in}}%
\pgfpathlineto{\pgfqpoint{3.503404in}{3.359365in}}%
\pgfpathlineto{\pgfqpoint{3.504391in}{3.280746in}}%
\pgfpathlineto{\pgfqpoint{3.504884in}{3.281005in}}%
\pgfpathlineto{\pgfqpoint{3.505377in}{2.534498in}}%
\pgfpathlineto{\pgfqpoint{3.505870in}{3.115172in}}%
\pgfpathlineto{\pgfqpoint{3.507350in}{3.371281in}}%
\pgfpathlineto{\pgfqpoint{3.507843in}{3.371281in}}%
\pgfpathlineto{\pgfqpoint{3.508336in}{3.278348in}}%
\pgfpathlineto{\pgfqpoint{3.509323in}{2.855065in}}%
\pgfpathlineto{\pgfqpoint{3.510803in}{3.369292in}}%
\pgfpathlineto{\pgfqpoint{3.511789in}{3.369292in}}%
\pgfpathlineto{\pgfqpoint{3.514255in}{3.153833in}}%
\pgfpathlineto{\pgfqpoint{3.514748in}{2.203322in}}%
\pgfpathlineto{\pgfqpoint{3.515242in}{3.039059in}}%
\pgfpathlineto{\pgfqpoint{3.515735in}{3.039059in}}%
\pgfpathlineto{\pgfqpoint{3.516721in}{3.378092in}}%
\pgfpathlineto{\pgfqpoint{3.517215in}{3.347602in}}%
\pgfpathlineto{\pgfqpoint{3.517708in}{3.347602in}}%
\pgfpathlineto{\pgfqpoint{3.518694in}{3.375878in}}%
\pgfpathlineto{\pgfqpoint{3.519187in}{2.240574in}}%
\pgfpathlineto{\pgfqpoint{3.519681in}{3.076474in}}%
\pgfpathlineto{\pgfqpoint{3.520174in}{3.076474in}}%
\pgfpathlineto{\pgfqpoint{3.520667in}{3.151373in}}%
\pgfpathlineto{\pgfqpoint{3.523133in}{2.775538in}}%
\pgfpathlineto{\pgfqpoint{3.525106in}{3.365889in}}%
\pgfpathlineto{\pgfqpoint{3.526093in}{3.379880in}}%
\pgfpathlineto{\pgfqpoint{3.526586in}{2.967496in}}%
\pgfpathlineto{\pgfqpoint{3.528066in}{3.379978in}}%
\pgfpathlineto{\pgfqpoint{3.528559in}{3.379978in}}%
\pgfpathlineto{\pgfqpoint{3.529052in}{3.305580in}}%
\pgfpathlineto{\pgfqpoint{3.529545in}{3.378337in}}%
\pgfpathlineto{\pgfqpoint{3.530039in}{3.378337in}}%
\pgfpathlineto{\pgfqpoint{3.531025in}{3.379991in}}%
\pgfpathlineto{\pgfqpoint{3.532505in}{3.047030in}}%
\pgfpathlineto{\pgfqpoint{3.533984in}{3.376442in}}%
\pgfpathlineto{\pgfqpoint{3.535957in}{3.376442in}}%
\pgfpathlineto{\pgfqpoint{3.537437in}{3.221524in}}%
\pgfpathlineto{\pgfqpoint{3.537930in}{3.377714in}}%
\pgfpathlineto{\pgfqpoint{3.539410in}{2.672918in}}%
\pgfpathlineto{\pgfqpoint{3.540890in}{3.373966in}}%
\pgfpathlineto{\pgfqpoint{3.542369in}{3.373966in}}%
\pgfpathlineto{\pgfqpoint{3.542863in}{3.199417in}}%
\pgfpathlineto{\pgfqpoint{3.543356in}{3.329438in}}%
\pgfpathlineto{\pgfqpoint{3.543849in}{3.329438in}}%
\pgfpathlineto{\pgfqpoint{3.544342in}{2.975302in}}%
\pgfpathlineto{\pgfqpoint{3.544836in}{3.344922in}}%
\pgfpathlineto{\pgfqpoint{3.545329in}{3.344922in}}%
\pgfpathlineto{\pgfqpoint{3.546315in}{3.139571in}}%
\pgfpathlineto{\pgfqpoint{3.547302in}{3.212008in}}%
\pgfpathlineto{\pgfqpoint{3.547795in}{3.078060in}}%
\pgfpathlineto{\pgfqpoint{3.548781in}{3.298707in}}%
\pgfpathlineto{\pgfqpoint{3.550261in}{3.046410in}}%
\pgfpathlineto{\pgfqpoint{3.550754in}{3.046410in}}%
\pgfpathlineto{\pgfqpoint{3.551248in}{2.095467in}}%
\pgfpathlineto{\pgfqpoint{3.552727in}{3.303277in}}%
\pgfpathlineto{\pgfqpoint{3.554207in}{3.275736in}}%
\pgfpathlineto{\pgfqpoint{3.555687in}{3.379589in}}%
\pgfpathlineto{\pgfqpoint{3.556673in}{3.379589in}}%
\pgfpathlineto{\pgfqpoint{3.558646in}{3.250893in}}%
\pgfpathlineto{\pgfqpoint{3.559139in}{3.299648in}}%
\pgfpathlineto{\pgfqpoint{3.559632in}{2.750679in}}%
\pgfpathlineto{\pgfqpoint{3.560126in}{3.162038in}}%
\pgfpathlineto{\pgfqpoint{3.560619in}{3.162038in}}%
\pgfpathlineto{\pgfqpoint{3.561112in}{2.713554in}}%
\pgfpathlineto{\pgfqpoint{3.561605in}{3.154714in}}%
\pgfpathlineto{\pgfqpoint{3.562099in}{3.379189in}}%
\pgfpathlineto{\pgfqpoint{3.563085in}{3.333878in}}%
\pgfpathlineto{\pgfqpoint{3.565058in}{3.183783in}}%
\pgfpathlineto{\pgfqpoint{3.566538in}{3.373525in}}%
\pgfpathlineto{\pgfqpoint{3.567524in}{3.373525in}}%
\pgfpathlineto{\pgfqpoint{3.568017in}{3.235842in}}%
\pgfpathlineto{\pgfqpoint{3.568511in}{3.352642in}}%
\pgfpathlineto{\pgfqpoint{3.569004in}{3.352642in}}%
\pgfpathlineto{\pgfqpoint{3.569990in}{3.124185in}}%
\pgfpathlineto{\pgfqpoint{3.571470in}{3.374922in}}%
\pgfpathlineto{\pgfqpoint{3.571963in}{2.880763in}}%
\pgfpathlineto{\pgfqpoint{3.572456in}{3.379989in}}%
\pgfpathlineto{\pgfqpoint{3.573443in}{3.343819in}}%
\pgfpathlineto{\pgfqpoint{3.573936in}{2.877511in}}%
\pgfpathlineto{\pgfqpoint{3.574429in}{3.272199in}}%
\pgfpathlineto{\pgfqpoint{3.575416in}{3.186158in}}%
\pgfpathlineto{\pgfqpoint{3.575909in}{3.190928in}}%
\pgfpathlineto{\pgfqpoint{3.577389in}{3.158463in}}%
\pgfpathlineto{\pgfqpoint{3.578375in}{3.158463in}}%
\pgfpathlineto{\pgfqpoint{3.578868in}{2.982256in}}%
\pgfpathlineto{\pgfqpoint{3.579362in}{3.366187in}}%
\pgfpathlineto{\pgfqpoint{3.579855in}{3.228347in}}%
\pgfpathlineto{\pgfqpoint{3.580348in}{3.228347in}}%
\pgfpathlineto{\pgfqpoint{3.580841in}{3.379451in}}%
\pgfpathlineto{\pgfqpoint{3.581828in}{3.141914in}}%
\pgfpathlineto{\pgfqpoint{3.582814in}{3.237388in}}%
\pgfpathlineto{\pgfqpoint{3.583801in}{3.209619in}}%
\pgfpathlineto{\pgfqpoint{3.585280in}{3.340823in}}%
\pgfpathlineto{\pgfqpoint{3.585774in}{3.340823in}}%
\pgfpathlineto{\pgfqpoint{3.587253in}{3.332764in}}%
\pgfpathlineto{\pgfqpoint{3.587747in}{3.332764in}}%
\pgfpathlineto{\pgfqpoint{3.588240in}{3.367956in}}%
\pgfpathlineto{\pgfqpoint{3.588733in}{3.264882in}}%
\pgfpathlineto{\pgfqpoint{3.589226in}{3.342037in}}%
\pgfpathlineto{\pgfqpoint{3.589720in}{3.378179in}}%
\pgfpathlineto{\pgfqpoint{3.590706in}{3.376501in}}%
\pgfpathlineto{\pgfqpoint{3.591199in}{3.376501in}}%
\pgfpathlineto{\pgfqpoint{3.592679in}{3.334479in}}%
\pgfpathlineto{\pgfqpoint{3.593172in}{3.334479in}}%
\pgfpathlineto{\pgfqpoint{3.594159in}{3.267645in}}%
\pgfpathlineto{\pgfqpoint{3.595145in}{3.369588in}}%
\pgfpathlineto{\pgfqpoint{3.596625in}{3.226049in}}%
\pgfpathlineto{\pgfqpoint{3.598598in}{3.226049in}}%
\pgfpathlineto{\pgfqpoint{3.599091in}{3.221712in}}%
\pgfpathlineto{\pgfqpoint{3.600077in}{3.236827in}}%
\pgfpathlineto{\pgfqpoint{3.601557in}{3.350818in}}%
\pgfpathlineto{\pgfqpoint{3.602544in}{3.377654in}}%
\pgfpathlineto{\pgfqpoint{3.603037in}{2.909012in}}%
\pgfpathlineto{\pgfqpoint{3.603530in}{3.280003in}}%
\pgfpathlineto{\pgfqpoint{3.605010in}{3.280003in}}%
\pgfpathlineto{\pgfqpoint{3.605996in}{3.158763in}}%
\pgfpathlineto{\pgfqpoint{3.606983in}{2.601931in}}%
\pgfpathlineto{\pgfqpoint{3.609942in}{3.353774in}}%
\pgfpathlineto{\pgfqpoint{3.610928in}{2.997676in}}%
\pgfpathlineto{\pgfqpoint{3.611422in}{3.378474in}}%
\pgfpathlineto{\pgfqpoint{3.612408in}{3.373216in}}%
\pgfpathlineto{\pgfqpoint{3.613888in}{3.378058in}}%
\pgfpathlineto{\pgfqpoint{3.614381in}{3.378058in}}%
\pgfpathlineto{\pgfqpoint{3.615861in}{3.379697in}}%
\pgfpathlineto{\pgfqpoint{3.617340in}{3.379697in}}%
\pgfpathlineto{\pgfqpoint{3.617834in}{3.371206in}}%
\pgfpathlineto{\pgfqpoint{3.618327in}{3.327789in}}%
\pgfpathlineto{\pgfqpoint{3.618820in}{3.364224in}}%
\pgfpathlineto{\pgfqpoint{3.620300in}{3.379273in}}%
\pgfpathlineto{\pgfqpoint{3.620793in}{3.308224in}}%
\pgfpathlineto{\pgfqpoint{3.621286in}{2.781911in}}%
\pgfpathlineto{\pgfqpoint{3.621780in}{3.343059in}}%
\pgfpathlineto{\pgfqpoint{3.623259in}{3.379165in}}%
\pgfpathlineto{\pgfqpoint{3.624739in}{3.364711in}}%
\pgfpathlineto{\pgfqpoint{3.627698in}{3.364711in}}%
\pgfpathlineto{\pgfqpoint{3.629178in}{3.357481in}}%
\pgfpathlineto{\pgfqpoint{3.629671in}{2.853687in}}%
\pgfpathlineto{\pgfqpoint{3.630164in}{3.225532in}}%
\pgfpathlineto{\pgfqpoint{3.631644in}{3.325796in}}%
\pgfpathlineto{\pgfqpoint{3.632137in}{3.374784in}}%
\pgfpathlineto{\pgfqpoint{3.633124in}{3.160635in}}%
\pgfpathlineto{\pgfqpoint{3.634604in}{3.353567in}}%
\pgfpathlineto{\pgfqpoint{3.635097in}{3.353567in}}%
\pgfpathlineto{\pgfqpoint{3.637070in}{3.221143in}}%
\pgfpathlineto{\pgfqpoint{3.637563in}{3.221143in}}%
\pgfpathlineto{\pgfqpoint{3.638549in}{3.377711in}}%
\pgfpathlineto{\pgfqpoint{3.639536in}{3.348665in}}%
\pgfpathlineto{\pgfqpoint{3.640029in}{3.045527in}}%
\pgfpathlineto{\pgfqpoint{3.640522in}{3.374434in}}%
\pgfpathlineto{\pgfqpoint{3.642495in}{3.374434in}}%
\pgfpathlineto{\pgfqpoint{3.643975in}{3.181923in}}%
\pgfpathlineto{\pgfqpoint{3.644468in}{3.181923in}}%
\pgfpathlineto{\pgfqpoint{3.645948in}{3.377239in}}%
\pgfpathlineto{\pgfqpoint{3.646441in}{3.041069in}}%
\pgfpathlineto{\pgfqpoint{3.646934in}{3.218941in}}%
\pgfpathlineto{\pgfqpoint{3.647921in}{3.013276in}}%
\pgfpathlineto{\pgfqpoint{3.649401in}{3.329849in}}%
\pgfpathlineto{\pgfqpoint{3.649894in}{3.329849in}}%
\pgfpathlineto{\pgfqpoint{3.651373in}{3.180041in}}%
\pgfpathlineto{\pgfqpoint{3.652360in}{3.180041in}}%
\pgfpathlineto{\pgfqpoint{3.652853in}{3.123960in}}%
\pgfpathlineto{\pgfqpoint{3.653840in}{3.318765in}}%
\pgfpathlineto{\pgfqpoint{3.654826in}{3.122893in}}%
\pgfpathlineto{\pgfqpoint{3.655319in}{3.132560in}}%
\pgfpathlineto{\pgfqpoint{3.655813in}{3.132560in}}%
\pgfpathlineto{\pgfqpoint{3.656306in}{3.020160in}}%
\pgfpathlineto{\pgfqpoint{3.657292in}{3.241176in}}%
\pgfpathlineto{\pgfqpoint{3.657785in}{2.923441in}}%
\pgfpathlineto{\pgfqpoint{3.658279in}{3.256845in}}%
\pgfpathlineto{\pgfqpoint{3.658772in}{3.038710in}}%
\pgfpathlineto{\pgfqpoint{3.659265in}{3.374541in}}%
\pgfpathlineto{\pgfqpoint{3.659758in}{3.279372in}}%
\pgfpathlineto{\pgfqpoint{3.660252in}{3.279372in}}%
\pgfpathlineto{\pgfqpoint{3.660745in}{2.789045in}}%
\pgfpathlineto{\pgfqpoint{3.661238in}{3.135238in}}%
\pgfpathlineto{\pgfqpoint{3.661731in}{3.362871in}}%
\pgfpathlineto{\pgfqpoint{3.662225in}{3.162320in}}%
\pgfpathlineto{\pgfqpoint{3.662718in}{3.162320in}}%
\pgfpathlineto{\pgfqpoint{3.664197in}{3.201541in}}%
\pgfpathlineto{\pgfqpoint{3.664691in}{3.201541in}}%
\pgfpathlineto{\pgfqpoint{3.666664in}{3.337935in}}%
\pgfpathlineto{\pgfqpoint{3.667157in}{3.337935in}}%
\pgfpathlineto{\pgfqpoint{3.668143in}{3.137108in}}%
\pgfpathlineto{\pgfqpoint{3.669623in}{3.379211in}}%
\pgfpathlineto{\pgfqpoint{3.671103in}{3.378387in}}%
\pgfpathlineto{\pgfqpoint{3.673076in}{2.765656in}}%
\pgfpathlineto{\pgfqpoint{3.674555in}{3.364319in}}%
\pgfpathlineto{\pgfqpoint{3.675049in}{3.349227in}}%
\pgfpathlineto{\pgfqpoint{3.675542in}{3.349227in}}%
\pgfpathlineto{\pgfqpoint{3.676035in}{2.814510in}}%
\pgfpathlineto{\pgfqpoint{3.676528in}{3.228976in}}%
\pgfpathlineto{\pgfqpoint{3.678008in}{3.379303in}}%
\pgfpathlineto{\pgfqpoint{3.678501in}{3.324063in}}%
\pgfpathlineto{\pgfqpoint{3.678994in}{3.342715in}}%
\pgfpathlineto{\pgfqpoint{3.679981in}{3.342715in}}%
\pgfpathlineto{\pgfqpoint{3.680967in}{3.363122in}}%
\pgfpathlineto{\pgfqpoint{3.681461in}{3.313732in}}%
\pgfpathlineto{\pgfqpoint{3.681954in}{3.320587in}}%
\pgfpathlineto{\pgfqpoint{3.682447in}{3.353921in}}%
\pgfpathlineto{\pgfqpoint{3.683927in}{3.271538in}}%
\pgfpathlineto{\pgfqpoint{3.685406in}{3.378441in}}%
\pgfpathlineto{\pgfqpoint{3.685900in}{3.093025in}}%
\pgfpathlineto{\pgfqpoint{3.686393in}{3.369556in}}%
\pgfpathlineto{\pgfqpoint{3.688859in}{3.369556in}}%
\pgfpathlineto{\pgfqpoint{3.689845in}{3.377710in}}%
\pgfpathlineto{\pgfqpoint{3.692312in}{3.302058in}}%
\pgfpathlineto{\pgfqpoint{3.692805in}{3.322259in}}%
\pgfpathlineto{\pgfqpoint{3.693298in}{3.219541in}}%
\pgfpathlineto{\pgfqpoint{3.693791in}{3.345076in}}%
\pgfpathlineto{\pgfqpoint{3.694778in}{3.358436in}}%
\pgfpathlineto{\pgfqpoint{3.695764in}{3.342114in}}%
\pgfpathlineto{\pgfqpoint{3.697737in}{3.059175in}}%
\pgfpathlineto{\pgfqpoint{3.698724in}{3.059175in}}%
\pgfpathlineto{\pgfqpoint{3.699217in}{2.265595in}}%
\pgfpathlineto{\pgfqpoint{3.700697in}{3.379314in}}%
\pgfpathlineto{\pgfqpoint{3.701190in}{2.992584in}}%
\pgfpathlineto{\pgfqpoint{3.701683in}{3.376340in}}%
\pgfpathlineto{\pgfqpoint{3.703163in}{3.376340in}}%
\pgfpathlineto{\pgfqpoint{3.704642in}{3.171472in}}%
\pgfpathlineto{\pgfqpoint{3.705136in}{3.171472in}}%
\pgfpathlineto{\pgfqpoint{3.706615in}{3.378117in}}%
\pgfpathlineto{\pgfqpoint{3.707109in}{3.327174in}}%
\pgfpathlineto{\pgfqpoint{3.708095in}{3.327174in}}%
\pgfpathlineto{\pgfqpoint{3.708588in}{2.689940in}}%
\pgfpathlineto{\pgfqpoint{3.709081in}{3.343562in}}%
\pgfpathlineto{\pgfqpoint{3.710068in}{3.343562in}}%
\pgfpathlineto{\pgfqpoint{3.710561in}{3.356034in}}%
\pgfpathlineto{\pgfqpoint{3.711548in}{3.204494in}}%
\pgfpathlineto{\pgfqpoint{3.712534in}{3.379976in}}%
\pgfpathlineto{\pgfqpoint{3.714014in}{3.323254in}}%
\pgfpathlineto{\pgfqpoint{3.714507in}{3.318290in}}%
\pgfpathlineto{\pgfqpoint{3.715000in}{2.605599in}}%
\pgfpathlineto{\pgfqpoint{3.715493in}{3.347659in}}%
\pgfpathlineto{\pgfqpoint{3.716973in}{3.347659in}}%
\pgfpathlineto{\pgfqpoint{3.717466in}{3.374307in}}%
\pgfpathlineto{\pgfqpoint{3.717960in}{3.361394in}}%
\pgfpathlineto{\pgfqpoint{3.718946in}{3.361394in}}%
\pgfpathlineto{\pgfqpoint{3.719439in}{3.359149in}}%
\pgfpathlineto{\pgfqpoint{3.719933in}{3.063713in}}%
\pgfpathlineto{\pgfqpoint{3.720426in}{3.332938in}}%
\pgfpathlineto{\pgfqpoint{3.721905in}{3.379345in}}%
\pgfpathlineto{\pgfqpoint{3.722399in}{3.321592in}}%
\pgfpathlineto{\pgfqpoint{3.722892in}{1.242469in}}%
\pgfpathlineto{\pgfqpoint{3.723385in}{2.888565in}}%
\pgfpathlineto{\pgfqpoint{3.724865in}{3.379404in}}%
\pgfpathlineto{\pgfqpoint{3.725851in}{2.680288in}}%
\pgfpathlineto{\pgfqpoint{3.726838in}{3.346877in}}%
\pgfpathlineto{\pgfqpoint{3.727331in}{3.191219in}}%
\pgfpathlineto{\pgfqpoint{3.727824in}{3.177211in}}%
\pgfpathlineto{\pgfqpoint{3.728811in}{1.907369in}}%
\pgfpathlineto{\pgfqpoint{3.729797in}{3.327893in}}%
\pgfpathlineto{\pgfqpoint{3.730290in}{2.971235in}}%
\pgfpathlineto{\pgfqpoint{3.731277in}{2.971235in}}%
\pgfpathlineto{\pgfqpoint{3.732757in}{3.377633in}}%
\pgfpathlineto{\pgfqpoint{3.734236in}{2.780161in}}%
\pgfpathlineto{\pgfqpoint{3.735716in}{3.362346in}}%
\pgfpathlineto{\pgfqpoint{3.736209in}{3.362346in}}%
\pgfpathlineto{\pgfqpoint{3.737196in}{2.935161in}}%
\pgfpathlineto{\pgfqpoint{3.738675in}{3.373903in}}%
\pgfpathlineto{\pgfqpoint{3.739169in}{3.373903in}}%
\pgfpathlineto{\pgfqpoint{3.740155in}{3.379838in}}%
\pgfpathlineto{\pgfqpoint{3.741635in}{3.045914in}}%
\pgfpathlineto{\pgfqpoint{3.742621in}{3.379994in}}%
\pgfpathlineto{\pgfqpoint{3.743114in}{3.374159in}}%
\pgfpathlineto{\pgfqpoint{3.743608in}{2.981675in}}%
\pgfpathlineto{\pgfqpoint{3.744101in}{3.202675in}}%
\pgfpathlineto{\pgfqpoint{3.745581in}{3.202675in}}%
\pgfpathlineto{\pgfqpoint{3.747060in}{3.318204in}}%
\pgfpathlineto{\pgfqpoint{3.748540in}{3.133067in}}%
\pgfpathlineto{\pgfqpoint{3.749526in}{3.053286in}}%
\pgfpathlineto{\pgfqpoint{3.751499in}{3.353978in}}%
\pgfpathlineto{\pgfqpoint{3.752979in}{3.004836in}}%
\pgfpathlineto{\pgfqpoint{3.753966in}{3.376158in}}%
\pgfpathlineto{\pgfqpoint{3.754459in}{2.378201in}}%
\pgfpathlineto{\pgfqpoint{3.754952in}{3.302080in}}%
\pgfpathlineto{\pgfqpoint{3.755445in}{3.302080in}}%
\pgfpathlineto{\pgfqpoint{3.756925in}{2.601550in}}%
\pgfpathlineto{\pgfqpoint{3.758405in}{3.379228in}}%
\pgfpathlineto{\pgfqpoint{3.759884in}{2.660577in}}%
\pgfpathlineto{\pgfqpoint{3.760378in}{2.660577in}}%
\pgfpathlineto{\pgfqpoint{3.761857in}{3.364349in}}%
\pgfpathlineto{\pgfqpoint{3.763830in}{3.364349in}}%
\pgfpathlineto{\pgfqpoint{3.765310in}{3.226158in}}%
\pgfpathlineto{\pgfqpoint{3.766296in}{3.266115in}}%
\pgfpathlineto{\pgfqpoint{3.767776in}{3.164991in}}%
\pgfpathlineto{\pgfqpoint{3.768269in}{3.377702in}}%
\pgfpathlineto{\pgfqpoint{3.768762in}{3.303562in}}%
\pgfpathlineto{\pgfqpoint{3.769256in}{3.184911in}}%
\pgfpathlineto{\pgfqpoint{3.770735in}{3.379942in}}%
\pgfpathlineto{\pgfqpoint{3.772215in}{3.378705in}}%
\pgfpathlineto{\pgfqpoint{3.772708in}{3.378705in}}%
\pgfpathlineto{\pgfqpoint{3.773202in}{3.344034in}}%
\pgfpathlineto{\pgfqpoint{3.773695in}{3.055116in}}%
\pgfpathlineto{\pgfqpoint{3.774188in}{3.204680in}}%
\pgfpathlineto{\pgfqpoint{3.774681in}{3.204680in}}%
\pgfpathlineto{\pgfqpoint{3.775174in}{3.347589in}}%
\pgfpathlineto{\pgfqpoint{3.775668in}{3.074039in}}%
\pgfpathlineto{\pgfqpoint{3.776161in}{3.143313in}}%
\pgfpathlineto{\pgfqpoint{3.776654in}{3.143313in}}%
\pgfpathlineto{\pgfqpoint{3.777641in}{3.379206in}}%
\pgfpathlineto{\pgfqpoint{3.778627in}{3.005304in}}%
\pgfpathlineto{\pgfqpoint{3.781093in}{3.379989in}}%
\pgfpathlineto{\pgfqpoint{3.782573in}{3.129289in}}%
\pgfpathlineto{\pgfqpoint{3.783559in}{3.110669in}}%
\pgfpathlineto{\pgfqpoint{3.785039in}{3.379353in}}%
\pgfpathlineto{\pgfqpoint{3.786026in}{3.379353in}}%
\pgfpathlineto{\pgfqpoint{3.787505in}{3.218195in}}%
\pgfpathlineto{\pgfqpoint{3.787998in}{3.320271in}}%
\pgfpathlineto{\pgfqpoint{3.788492in}{3.141286in}}%
\pgfpathlineto{\pgfqpoint{3.788985in}{3.295780in}}%
\pgfpathlineto{\pgfqpoint{3.790958in}{3.119564in}}%
\pgfpathlineto{\pgfqpoint{3.791451in}{3.348893in}}%
\pgfpathlineto{\pgfqpoint{3.791944in}{2.833165in}}%
\pgfpathlineto{\pgfqpoint{3.792438in}{3.244595in}}%
\pgfpathlineto{\pgfqpoint{3.792931in}{3.238178in}}%
\pgfpathlineto{\pgfqpoint{3.794410in}{3.379299in}}%
\pgfpathlineto{\pgfqpoint{3.794904in}{3.232866in}}%
\pgfpathlineto{\pgfqpoint{3.795397in}{2.732109in}}%
\pgfpathlineto{\pgfqpoint{3.795890in}{3.119024in}}%
\pgfpathlineto{\pgfqpoint{3.798356in}{3.379559in}}%
\pgfpathlineto{\pgfqpoint{3.798850in}{3.375016in}}%
\pgfpathlineto{\pgfqpoint{3.799343in}{3.375016in}}%
\pgfpathlineto{\pgfqpoint{3.799836in}{3.335336in}}%
\pgfpathlineto{\pgfqpoint{3.800329in}{3.353216in}}%
\pgfpathlineto{\pgfqpoint{3.800822in}{3.371207in}}%
\pgfpathlineto{\pgfqpoint{3.801809in}{3.371088in}}%
\pgfpathlineto{\pgfqpoint{3.803289in}{1.026805in}}%
\pgfpathlineto{\pgfqpoint{3.803782in}{1.026805in}}%
\pgfpathlineto{\pgfqpoint{3.804768in}{2.983893in}}%
\pgfpathlineto{\pgfqpoint{3.805262in}{2.978386in}}%
\pgfpathlineto{\pgfqpoint{3.806741in}{3.130032in}}%
\pgfpathlineto{\pgfqpoint{3.807728in}{3.198521in}}%
\pgfpathlineto{\pgfqpoint{3.808221in}{2.554475in}}%
\pgfpathlineto{\pgfqpoint{3.808714in}{3.285776in}}%
\pgfpathlineto{\pgfqpoint{3.809701in}{3.372910in}}%
\pgfpathlineto{\pgfqpoint{3.810194in}{3.360323in}}%
\pgfpathlineto{\pgfqpoint{3.810687in}{3.360323in}}%
\pgfpathlineto{\pgfqpoint{3.811180in}{3.371697in}}%
\pgfpathlineto{\pgfqpoint{3.811674in}{3.002569in}}%
\pgfpathlineto{\pgfqpoint{3.812167in}{3.247680in}}%
\pgfpathlineto{\pgfqpoint{3.813153in}{3.375097in}}%
\pgfpathlineto{\pgfqpoint{3.813646in}{3.374633in}}%
\pgfpathlineto{\pgfqpoint{3.815126in}{3.311529in}}%
\pgfpathlineto{\pgfqpoint{3.816606in}{3.373682in}}%
\pgfpathlineto{\pgfqpoint{3.817099in}{3.373682in}}%
\pgfpathlineto{\pgfqpoint{3.818579in}{3.379602in}}%
\pgfpathlineto{\pgfqpoint{3.819565in}{3.191399in}}%
\pgfpathlineto{\pgfqpoint{3.820058in}{3.222425in}}%
\pgfpathlineto{\pgfqpoint{3.821538in}{3.276323in}}%
\pgfpathlineto{\pgfqpoint{3.823018in}{3.138981in}}%
\pgfpathlineto{\pgfqpoint{3.823511in}{3.373308in}}%
\pgfpathlineto{\pgfqpoint{3.824004in}{2.876941in}}%
\pgfpathlineto{\pgfqpoint{3.824498in}{3.356514in}}%
\pgfpathlineto{\pgfqpoint{3.825484in}{3.037548in}}%
\pgfpathlineto{\pgfqpoint{3.825977in}{3.224458in}}%
\pgfpathlineto{\pgfqpoint{3.827457in}{3.109708in}}%
\pgfpathlineto{\pgfqpoint{3.828443in}{3.379908in}}%
\pgfpathlineto{\pgfqpoint{3.829923in}{2.889638in}}%
\pgfpathlineto{\pgfqpoint{3.831403in}{3.317689in}}%
\pgfpathlineto{\pgfqpoint{3.832389in}{3.375721in}}%
\pgfpathlineto{\pgfqpoint{3.832882in}{3.228895in}}%
\pgfpathlineto{\pgfqpoint{3.833376in}{3.365042in}}%
\pgfpathlineto{\pgfqpoint{3.834855in}{3.377203in}}%
\pgfpathlineto{\pgfqpoint{3.835349in}{3.377203in}}%
\pgfpathlineto{\pgfqpoint{3.835842in}{3.356489in}}%
\pgfpathlineto{\pgfqpoint{3.836335in}{3.252442in}}%
\pgfpathlineto{\pgfqpoint{3.836828in}{3.362493in}}%
\pgfpathlineto{\pgfqpoint{3.837322in}{3.378272in}}%
\pgfpathlineto{\pgfqpoint{3.838801in}{2.963713in}}%
\pgfpathlineto{\pgfqpoint{3.839294in}{3.368146in}}%
\pgfpathlineto{\pgfqpoint{3.839788in}{3.205437in}}%
\pgfpathlineto{\pgfqpoint{3.840774in}{3.205437in}}%
\pgfpathlineto{\pgfqpoint{3.841267in}{3.372602in}}%
\pgfpathlineto{\pgfqpoint{3.842254in}{3.352793in}}%
\pgfpathlineto{\pgfqpoint{3.842747in}{3.349542in}}%
\pgfpathlineto{\pgfqpoint{3.843240in}{2.701144in}}%
\pgfpathlineto{\pgfqpoint{3.844227in}{2.779277in}}%
\pgfpathlineto{\pgfqpoint{3.844720in}{2.429108in}}%
\pgfpathlineto{\pgfqpoint{3.846200in}{3.356892in}}%
\pgfpathlineto{\pgfqpoint{3.847186in}{3.356892in}}%
\pgfpathlineto{\pgfqpoint{3.847679in}{3.282047in}}%
\pgfpathlineto{\pgfqpoint{3.848173in}{3.373665in}}%
\pgfpathlineto{\pgfqpoint{3.849652in}{2.937357in}}%
\pgfpathlineto{\pgfqpoint{3.851132in}{3.286853in}}%
\pgfpathlineto{\pgfqpoint{3.852119in}{3.286853in}}%
\pgfpathlineto{\pgfqpoint{3.853105in}{2.920372in}}%
\pgfpathlineto{\pgfqpoint{3.853598in}{2.968310in}}%
\pgfpathlineto{\pgfqpoint{3.855078in}{3.323831in}}%
\pgfpathlineto{\pgfqpoint{3.856064in}{3.323831in}}%
\pgfpathlineto{\pgfqpoint{3.857544in}{3.175398in}}%
\pgfpathlineto{\pgfqpoint{3.859024in}{3.271610in}}%
\pgfpathlineto{\pgfqpoint{3.859517in}{2.788147in}}%
\pgfpathlineto{\pgfqpoint{3.860010in}{3.154684in}}%
\pgfpathlineto{\pgfqpoint{3.861490in}{3.095024in}}%
\pgfpathlineto{\pgfqpoint{3.862476in}{3.095024in}}%
\pgfpathlineto{\pgfqpoint{3.863463in}{3.379987in}}%
\pgfpathlineto{\pgfqpoint{3.863956in}{3.064039in}}%
\pgfpathlineto{\pgfqpoint{3.864943in}{3.091275in}}%
\pgfpathlineto{\pgfqpoint{3.865436in}{3.371152in}}%
\pgfpathlineto{\pgfqpoint{3.866422in}{3.321740in}}%
\pgfpathlineto{\pgfqpoint{3.867902in}{3.329721in}}%
\pgfpathlineto{\pgfqpoint{3.868395in}{3.329721in}}%
\pgfpathlineto{\pgfqpoint{3.870861in}{3.224776in}}%
\pgfpathlineto{\pgfqpoint{3.871355in}{3.224776in}}%
\pgfpathlineto{\pgfqpoint{3.872834in}{2.923503in}}%
\pgfpathlineto{\pgfqpoint{3.874314in}{3.375693in}}%
\pgfpathlineto{\pgfqpoint{3.875300in}{2.799249in}}%
\pgfpathlineto{\pgfqpoint{3.876287in}{3.376931in}}%
\pgfpathlineto{\pgfqpoint{3.876780in}{3.263184in}}%
\pgfpathlineto{\pgfqpoint{3.877273in}{3.220045in}}%
\pgfpathlineto{\pgfqpoint{3.878753in}{3.375401in}}%
\pgfpathlineto{\pgfqpoint{3.879246in}{3.375401in}}%
\pgfpathlineto{\pgfqpoint{3.879739in}{3.379564in}}%
\pgfpathlineto{\pgfqpoint{3.880233in}{2.725785in}}%
\pgfpathlineto{\pgfqpoint{3.880726in}{3.130790in}}%
\pgfpathlineto{\pgfqpoint{3.882206in}{3.130790in}}%
\pgfpathlineto{\pgfqpoint{3.882699in}{3.380000in}}%
\pgfpathlineto{\pgfqpoint{3.883685in}{3.368989in}}%
\pgfpathlineto{\pgfqpoint{3.884179in}{3.379997in}}%
\pgfpathlineto{\pgfqpoint{3.885165in}{2.972123in}}%
\pgfpathlineto{\pgfqpoint{3.886645in}{3.308932in}}%
\pgfpathlineto{\pgfqpoint{3.887631in}{3.308932in}}%
\pgfpathlineto{\pgfqpoint{3.888124in}{2.732470in}}%
\pgfpathlineto{\pgfqpoint{3.888618in}{3.322395in}}%
\pgfpathlineto{\pgfqpoint{3.892070in}{3.322395in}}%
\pgfpathlineto{\pgfqpoint{3.893550in}{3.269521in}}%
\pgfpathlineto{\pgfqpoint{3.894043in}{3.269521in}}%
\pgfpathlineto{\pgfqpoint{3.895030in}{3.280761in}}%
\pgfpathlineto{\pgfqpoint{3.896016in}{2.819180in}}%
\pgfpathlineto{\pgfqpoint{3.897496in}{3.378539in}}%
\pgfpathlineto{\pgfqpoint{3.899469in}{2.532131in}}%
\pgfpathlineto{\pgfqpoint{3.900948in}{2.553895in}}%
\pgfpathlineto{\pgfqpoint{3.902428in}{3.377203in}}%
\pgfpathlineto{\pgfqpoint{3.903908in}{3.377203in}}%
\pgfpathlineto{\pgfqpoint{3.904894in}{3.305142in}}%
\pgfpathlineto{\pgfqpoint{3.905881in}{3.379550in}}%
\pgfpathlineto{\pgfqpoint{3.906867in}{3.160625in}}%
\pgfpathlineto{\pgfqpoint{3.908347in}{3.352398in}}%
\pgfpathlineto{\pgfqpoint{3.909333in}{3.352398in}}%
\pgfpathlineto{\pgfqpoint{3.910320in}{3.378826in}}%
\pgfpathlineto{\pgfqpoint{3.910813in}{2.840054in}}%
\pgfpathlineto{\pgfqpoint{3.911306in}{3.355095in}}%
\pgfpathlineto{\pgfqpoint{3.912293in}{3.355095in}}%
\pgfpathlineto{\pgfqpoint{3.913279in}{3.308044in}}%
\pgfpathlineto{\pgfqpoint{3.914759in}{3.363093in}}%
\pgfpathlineto{\pgfqpoint{3.915745in}{2.323803in}}%
\pgfpathlineto{\pgfqpoint{3.916239in}{3.032660in}}%
\pgfpathlineto{\pgfqpoint{3.916732in}{3.032660in}}%
\pgfpathlineto{\pgfqpoint{3.917225in}{2.974940in}}%
\pgfpathlineto{\pgfqpoint{3.917718in}{3.209187in}}%
\pgfpathlineto{\pgfqpoint{3.918211in}{2.973560in}}%
\pgfpathlineto{\pgfqpoint{3.919198in}{2.454755in}}%
\pgfpathlineto{\pgfqpoint{3.920184in}{3.370299in}}%
\pgfpathlineto{\pgfqpoint{3.921664in}{3.047369in}}%
\pgfpathlineto{\pgfqpoint{3.922651in}{2.958778in}}%
\pgfpathlineto{\pgfqpoint{3.925117in}{3.329232in}}%
\pgfpathlineto{\pgfqpoint{3.925610in}{3.329232in}}%
\pgfpathlineto{\pgfqpoint{3.926103in}{3.282462in}}%
\pgfpathlineto{\pgfqpoint{3.926596in}{3.359799in}}%
\pgfpathlineto{\pgfqpoint{3.927090in}{2.952139in}}%
\pgfpathlineto{\pgfqpoint{3.927583in}{3.373561in}}%
\pgfpathlineto{\pgfqpoint{3.928076in}{3.373561in}}%
\pgfpathlineto{\pgfqpoint{3.929063in}{2.450536in}}%
\pgfpathlineto{\pgfqpoint{3.930049in}{2.796266in}}%
\pgfpathlineto{\pgfqpoint{3.930542in}{2.796266in}}%
\pgfpathlineto{\pgfqpoint{3.931035in}{2.767857in}}%
\pgfpathlineto{\pgfqpoint{3.932515in}{3.310276in}}%
\pgfpathlineto{\pgfqpoint{3.933008in}{3.310276in}}%
\pgfpathlineto{\pgfqpoint{3.933502in}{3.379410in}}%
\pgfpathlineto{\pgfqpoint{3.933995in}{3.277442in}}%
\pgfpathlineto{\pgfqpoint{3.934981in}{3.297823in}}%
\pgfpathlineto{\pgfqpoint{3.936461in}{3.242463in}}%
\pgfpathlineto{\pgfqpoint{3.936954in}{3.242463in}}%
\pgfpathlineto{\pgfqpoint{3.937941in}{3.091078in}}%
\pgfpathlineto{\pgfqpoint{3.938434in}{1.772985in}}%
\pgfpathlineto{\pgfqpoint{3.939420in}{2.004574in}}%
\pgfpathlineto{\pgfqpoint{3.939914in}{1.878331in}}%
\pgfpathlineto{\pgfqpoint{3.940900in}{3.288583in}}%
\pgfpathlineto{\pgfqpoint{3.941393in}{3.169078in}}%
\pgfpathlineto{\pgfqpoint{3.942380in}{3.036377in}}%
\pgfpathlineto{\pgfqpoint{3.943859in}{3.195453in}}%
\pgfpathlineto{\pgfqpoint{3.944353in}{3.017476in}}%
\pgfpathlineto{\pgfqpoint{3.945832in}{3.379955in}}%
\pgfpathlineto{\pgfqpoint{3.946326in}{3.379955in}}%
\pgfpathlineto{\pgfqpoint{3.946819in}{3.128016in}}%
\pgfpathlineto{\pgfqpoint{3.947312in}{3.356377in}}%
\pgfpathlineto{\pgfqpoint{3.948792in}{3.203396in}}%
\pgfpathlineto{\pgfqpoint{3.949285in}{3.311670in}}%
\pgfpathlineto{\pgfqpoint{3.949778in}{3.195009in}}%
\pgfpathlineto{\pgfqpoint{3.950765in}{3.195009in}}%
\pgfpathlineto{\pgfqpoint{3.952244in}{3.217008in}}%
\pgfpathlineto{\pgfqpoint{3.953724in}{2.785094in}}%
\pgfpathlineto{\pgfqpoint{3.955697in}{2.785094in}}%
\pgfpathlineto{\pgfqpoint{3.956190in}{3.378786in}}%
\pgfpathlineto{\pgfqpoint{3.957177in}{3.245498in}}%
\pgfpathlineto{\pgfqpoint{3.958163in}{3.379580in}}%
\pgfpathlineto{\pgfqpoint{3.959643in}{3.294434in}}%
\pgfpathlineto{\pgfqpoint{3.960136in}{3.352825in}}%
\pgfpathlineto{\pgfqpoint{3.960629in}{3.238438in}}%
\pgfpathlineto{\pgfqpoint{3.961123in}{3.376506in}}%
\pgfpathlineto{\pgfqpoint{3.962109in}{3.376506in}}%
\pgfpathlineto{\pgfqpoint{3.963096in}{2.577524in}}%
\pgfpathlineto{\pgfqpoint{3.964575in}{3.336260in}}%
\pgfpathlineto{\pgfqpoint{3.965068in}{3.355982in}}%
\pgfpathlineto{\pgfqpoint{3.966055in}{3.131377in}}%
\pgfpathlineto{\pgfqpoint{3.966548in}{3.233716in}}%
\pgfpathlineto{\pgfqpoint{3.968028in}{3.214069in}}%
\pgfpathlineto{\pgfqpoint{3.969014in}{3.214069in}}%
\pgfpathlineto{\pgfqpoint{3.970494in}{3.372280in}}%
\pgfpathlineto{\pgfqpoint{3.973453in}{3.372280in}}%
\pgfpathlineto{\pgfqpoint{3.974933in}{2.964760in}}%
\pgfpathlineto{\pgfqpoint{3.976906in}{2.964760in}}%
\pgfpathlineto{\pgfqpoint{3.978386in}{3.377168in}}%
\pgfpathlineto{\pgfqpoint{3.978879in}{3.276425in}}%
\pgfpathlineto{\pgfqpoint{3.979372in}{3.297585in}}%
\pgfpathlineto{\pgfqpoint{3.980359in}{2.822208in}}%
\pgfpathlineto{\pgfqpoint{3.981838in}{3.368318in}}%
\pgfpathlineto{\pgfqpoint{3.982332in}{2.207285in}}%
\pgfpathlineto{\pgfqpoint{3.982825in}{3.360104in}}%
\pgfpathlineto{\pgfqpoint{3.984304in}{3.360104in}}%
\pgfpathlineto{\pgfqpoint{3.985291in}{3.225771in}}%
\pgfpathlineto{\pgfqpoint{3.986771in}{3.355087in}}%
\pgfpathlineto{\pgfqpoint{3.988250in}{3.355087in}}%
\pgfpathlineto{\pgfqpoint{3.989237in}{3.057296in}}%
\pgfpathlineto{\pgfqpoint{3.990716in}{3.371301in}}%
\pgfpathlineto{\pgfqpoint{3.991703in}{2.611220in}}%
\pgfpathlineto{\pgfqpoint{3.993183in}{3.288466in}}%
\pgfpathlineto{\pgfqpoint{3.994169in}{3.288466in}}%
\pgfpathlineto{\pgfqpoint{3.994662in}{3.187820in}}%
\pgfpathlineto{\pgfqpoint{3.995156in}{3.203379in}}%
\pgfpathlineto{\pgfqpoint{3.996635in}{3.354539in}}%
\pgfpathlineto{\pgfqpoint{3.997128in}{3.354539in}}%
\pgfpathlineto{\pgfqpoint{3.997622in}{3.377896in}}%
\pgfpathlineto{\pgfqpoint{3.998608in}{2.966997in}}%
\pgfpathlineto{\pgfqpoint{3.999101in}{3.297110in}}%
\pgfpathlineto{\pgfqpoint{3.999595in}{1.985772in}}%
\pgfpathlineto{\pgfqpoint{4.000088in}{3.333933in}}%
\pgfpathlineto{\pgfqpoint{4.000581in}{3.333933in}}%
\pgfpathlineto{\pgfqpoint{4.001074in}{3.200564in}}%
\pgfpathlineto{\pgfqpoint{4.001568in}{3.214999in}}%
\pgfpathlineto{\pgfqpoint{4.002554in}{3.374894in}}%
\pgfpathlineto{\pgfqpoint{4.003540in}{2.710626in}}%
\pgfpathlineto{\pgfqpoint{4.005020in}{3.115318in}}%
\pgfpathlineto{\pgfqpoint{4.006007in}{3.379749in}}%
\pgfpathlineto{\pgfqpoint{4.007486in}{3.184932in}}%
\pgfpathlineto{\pgfqpoint{4.009459in}{3.362093in}}%
\pgfpathlineto{\pgfqpoint{4.009952in}{3.362093in}}%
\pgfpathlineto{\pgfqpoint{4.010446in}{2.941698in}}%
\pgfpathlineto{\pgfqpoint{4.011432in}{2.953192in}}%
\pgfpathlineto{\pgfqpoint{4.012912in}{2.849539in}}%
\pgfpathlineto{\pgfqpoint{4.013898in}{3.295508in}}%
\pgfpathlineto{\pgfqpoint{4.014392in}{3.223611in}}%
\pgfpathlineto{\pgfqpoint{4.015378in}{3.329952in}}%
\pgfpathlineto{\pgfqpoint{4.016858in}{3.261138in}}%
\pgfpathlineto{\pgfqpoint{4.017844in}{3.261138in}}%
\pgfpathlineto{\pgfqpoint{4.018831in}{3.375924in}}%
\pgfpathlineto{\pgfqpoint{4.019324in}{3.171747in}}%
\pgfpathlineto{\pgfqpoint{4.019817in}{3.323815in}}%
\pgfpathlineto{\pgfqpoint{4.021297in}{3.053193in}}%
\pgfpathlineto{\pgfqpoint{4.021790in}{2.507061in}}%
\pgfpathlineto{\pgfqpoint{4.022283in}{3.036245in}}%
\pgfpathlineto{\pgfqpoint{4.023270in}{2.845729in}}%
\pgfpathlineto{\pgfqpoint{4.023763in}{3.154167in}}%
\pgfpathlineto{\pgfqpoint{4.024749in}{2.231027in}}%
\pgfpathlineto{\pgfqpoint{4.026229in}{3.379807in}}%
\pgfpathlineto{\pgfqpoint{4.027216in}{3.379807in}}%
\pgfpathlineto{\pgfqpoint{4.028202in}{3.045816in}}%
\pgfpathlineto{\pgfqpoint{4.028695in}{3.084349in}}%
\pgfpathlineto{\pgfqpoint{4.029188in}{3.084349in}}%
\pgfpathlineto{\pgfqpoint{4.029682in}{3.024402in}}%
\pgfpathlineto{\pgfqpoint{4.030175in}{3.037824in}}%
\pgfpathlineto{\pgfqpoint{4.031655in}{3.378863in}}%
\pgfpathlineto{\pgfqpoint{4.032148in}{3.378863in}}%
\pgfpathlineto{\pgfqpoint{4.035107in}{3.141144in}}%
\pgfpathlineto{\pgfqpoint{4.035600in}{3.141144in}}%
\pgfpathlineto{\pgfqpoint{4.037080in}{3.379983in}}%
\pgfpathlineto{\pgfqpoint{4.037573in}{3.379983in}}%
\pgfpathlineto{\pgfqpoint{4.040533in}{3.063206in}}%
\pgfpathlineto{\pgfqpoint{4.041026in}{3.210532in}}%
\pgfpathlineto{\pgfqpoint{4.041519in}{3.067207in}}%
\pgfpathlineto{\pgfqpoint{4.042012in}{3.067207in}}%
\pgfpathlineto{\pgfqpoint{4.043492in}{2.660175in}}%
\pgfpathlineto{\pgfqpoint{4.043985in}{2.631578in}}%
\pgfpathlineto{\pgfqpoint{4.044479in}{2.195374in}}%
\pgfpathlineto{\pgfqpoint{4.044972in}{3.343203in}}%
\pgfpathlineto{\pgfqpoint{4.045958in}{3.325296in}}%
\pgfpathlineto{\pgfqpoint{4.046452in}{2.820955in}}%
\pgfpathlineto{\pgfqpoint{4.047438in}{2.881376in}}%
\pgfpathlineto{\pgfqpoint{4.047931in}{2.881376in}}%
\pgfpathlineto{\pgfqpoint{4.048425in}{3.358186in}}%
\pgfpathlineto{\pgfqpoint{4.049411in}{3.326348in}}%
\pgfpathlineto{\pgfqpoint{4.050397in}{3.326348in}}%
\pgfpathlineto{\pgfqpoint{4.050891in}{3.377920in}}%
\pgfpathlineto{\pgfqpoint{4.051384in}{3.336674in}}%
\pgfpathlineto{\pgfqpoint{4.052864in}{3.356032in}}%
\pgfpathlineto{\pgfqpoint{4.053850in}{3.356032in}}%
\pgfpathlineto{\pgfqpoint{4.054343in}{3.377543in}}%
\pgfpathlineto{\pgfqpoint{4.054837in}{3.364645in}}%
\pgfpathlineto{\pgfqpoint{4.055330in}{3.364645in}}%
\pgfpathlineto{\pgfqpoint{4.055823in}{3.241318in}}%
\pgfpathlineto{\pgfqpoint{4.056316in}{3.265156in}}%
\pgfpathlineto{\pgfqpoint{4.057303in}{1.650622in}}%
\pgfpathlineto{\pgfqpoint{4.057796in}{2.422496in}}%
\pgfpathlineto{\pgfqpoint{4.059769in}{3.376321in}}%
\pgfpathlineto{\pgfqpoint{4.060755in}{2.464632in}}%
\pgfpathlineto{\pgfqpoint{4.061249in}{3.197105in}}%
\pgfpathlineto{\pgfqpoint{4.062235in}{3.152445in}}%
\pgfpathlineto{\pgfqpoint{4.062728in}{3.152445in}}%
\pgfpathlineto{\pgfqpoint{4.063221in}{3.164908in}}%
\pgfpathlineto{\pgfqpoint{4.063715in}{3.379117in}}%
\pgfpathlineto{\pgfqpoint{4.064208in}{3.225140in}}%
\pgfpathlineto{\pgfqpoint{4.065194in}{3.292041in}}%
\pgfpathlineto{\pgfqpoint{4.065688in}{2.417773in}}%
\pgfpathlineto{\pgfqpoint{4.066181in}{3.377040in}}%
\pgfpathlineto{\pgfqpoint{4.066674in}{3.377040in}}%
\pgfpathlineto{\pgfqpoint{4.067167in}{3.154007in}}%
\pgfpathlineto{\pgfqpoint{4.067661in}{3.379355in}}%
\pgfpathlineto{\pgfqpoint{4.068647in}{3.379355in}}%
\pgfpathlineto{\pgfqpoint{4.070127in}{3.357122in}}%
\pgfpathlineto{\pgfqpoint{4.071113in}{2.884042in}}%
\pgfpathlineto{\pgfqpoint{4.072593in}{3.328595in}}%
\pgfpathlineto{\pgfqpoint{4.073086in}{3.373457in}}%
\pgfpathlineto{\pgfqpoint{4.074073in}{2.999281in}}%
\pgfpathlineto{\pgfqpoint{4.075552in}{3.329012in}}%
\pgfpathlineto{\pgfqpoint{4.076045in}{3.273110in}}%
\pgfpathlineto{\pgfqpoint{4.076539in}{3.298573in}}%
\pgfpathlineto{\pgfqpoint{4.077032in}{2.786896in}}%
\pgfpathlineto{\pgfqpoint{4.077525in}{3.092616in}}%
\pgfpathlineto{\pgfqpoint{4.078018in}{3.367955in}}%
\pgfpathlineto{\pgfqpoint{4.078512in}{3.274521in}}%
\pgfpathlineto{\pgfqpoint{4.079498in}{3.274521in}}%
\pgfpathlineto{\pgfqpoint{4.080978in}{3.088711in}}%
\pgfpathlineto{\pgfqpoint{4.081471in}{3.279692in}}%
\pgfpathlineto{\pgfqpoint{4.082457in}{2.164098in}}%
\pgfpathlineto{\pgfqpoint{4.083444in}{3.379685in}}%
\pgfpathlineto{\pgfqpoint{4.083937in}{3.361461in}}%
\pgfpathlineto{\pgfqpoint{4.084430in}{3.011599in}}%
\pgfpathlineto{\pgfqpoint{4.084924in}{3.149816in}}%
\pgfpathlineto{\pgfqpoint{4.085417in}{3.149816in}}%
\pgfpathlineto{\pgfqpoint{4.085910in}{3.190216in}}%
\pgfpathlineto{\pgfqpoint{4.086403in}{3.376552in}}%
\pgfpathlineto{\pgfqpoint{4.086897in}{3.355954in}}%
\pgfpathlineto{\pgfqpoint{4.087390in}{3.174859in}}%
\pgfpathlineto{\pgfqpoint{4.087883in}{3.375307in}}%
\pgfpathlineto{\pgfqpoint{4.089363in}{3.271685in}}%
\pgfpathlineto{\pgfqpoint{4.089856in}{3.378881in}}%
\pgfpathlineto{\pgfqpoint{4.090349in}{3.372012in}}%
\pgfpathlineto{\pgfqpoint{4.090842in}{2.637769in}}%
\pgfpathlineto{\pgfqpoint{4.091336in}{2.911728in}}%
\pgfpathlineto{\pgfqpoint{4.092815in}{3.336460in}}%
\pgfpathlineto{\pgfqpoint{4.093309in}{3.336460in}}%
\pgfpathlineto{\pgfqpoint{4.093802in}{2.755971in}}%
\pgfpathlineto{\pgfqpoint{4.094295in}{3.378088in}}%
\pgfpathlineto{\pgfqpoint{4.095281in}{3.285502in}}%
\pgfpathlineto{\pgfqpoint{4.096761in}{3.370138in}}%
\pgfpathlineto{\pgfqpoint{4.097254in}{3.370138in}}%
\pgfpathlineto{\pgfqpoint{4.097748in}{0.799267in}}%
\pgfpathlineto{\pgfqpoint{4.098241in}{2.210845in}}%
\pgfpathlineto{\pgfqpoint{4.098734in}{3.376459in}}%
\pgfpathlineto{\pgfqpoint{4.099721in}{3.338157in}}%
\pgfpathlineto{\pgfqpoint{4.101200in}{3.338157in}}%
\pgfpathlineto{\pgfqpoint{4.102187in}{3.295730in}}%
\pgfpathlineto{\pgfqpoint{4.103666in}{3.354024in}}%
\pgfpathlineto{\pgfqpoint{4.105146in}{3.379918in}}%
\pgfpathlineto{\pgfqpoint{4.106133in}{3.379918in}}%
\pgfpathlineto{\pgfqpoint{4.106626in}{3.378235in}}%
\pgfpathlineto{\pgfqpoint{4.108105in}{3.346492in}}%
\pgfpathlineto{\pgfqpoint{4.108599in}{2.547443in}}%
\pgfpathlineto{\pgfqpoint{4.109092in}{3.379989in}}%
\pgfpathlineto{\pgfqpoint{4.109585in}{3.346742in}}%
\pgfpathlineto{\pgfqpoint{4.110078in}{3.376696in}}%
\pgfpathlineto{\pgfqpoint{4.112545in}{3.376696in}}%
\pgfpathlineto{\pgfqpoint{4.113038in}{3.180982in}}%
\pgfpathlineto{\pgfqpoint{4.113531in}{3.300908in}}%
\pgfpathlineto{\pgfqpoint{4.114024in}{3.320532in}}%
\pgfpathlineto{\pgfqpoint{4.114517in}{2.891409in}}%
\pgfpathlineto{\pgfqpoint{4.115011in}{3.378096in}}%
\pgfpathlineto{\pgfqpoint{4.115997in}{3.378096in}}%
\pgfpathlineto{\pgfqpoint{4.117970in}{2.438628in}}%
\pgfpathlineto{\pgfqpoint{4.118463in}{2.503123in}}%
\pgfpathlineto{\pgfqpoint{4.119450in}{3.379882in}}%
\pgfpathlineto{\pgfqpoint{4.119943in}{3.363356in}}%
\pgfpathlineto{\pgfqpoint{4.120929in}{3.238955in}}%
\pgfpathlineto{\pgfqpoint{4.121423in}{3.378631in}}%
\pgfpathlineto{\pgfqpoint{4.122409in}{3.097556in}}%
\pgfpathlineto{\pgfqpoint{4.122902in}{3.164104in}}%
\pgfpathlineto{\pgfqpoint{4.123396in}{3.198566in}}%
\pgfpathlineto{\pgfqpoint{4.124382in}{3.377988in}}%
\pgfpathlineto{\pgfqpoint{4.124875in}{3.309387in}}%
\pgfpathlineto{\pgfqpoint{4.125369in}{3.378795in}}%
\pgfpathlineto{\pgfqpoint{4.126355in}{3.378795in}}%
\pgfpathlineto{\pgfqpoint{4.127835in}{3.333883in}}%
\pgfpathlineto{\pgfqpoint{4.128821in}{2.793297in}}%
\pgfpathlineto{\pgfqpoint{4.129314in}{2.921238in}}%
\pgfpathlineto{\pgfqpoint{4.130301in}{2.921238in}}%
\pgfpathlineto{\pgfqpoint{4.131781in}{3.369915in}}%
\pgfpathlineto{\pgfqpoint{4.132274in}{3.294756in}}%
\pgfpathlineto{\pgfqpoint{4.132767in}{3.372922in}}%
\pgfpathlineto{\pgfqpoint{4.133260in}{3.376035in}}%
\pgfpathlineto{\pgfqpoint{4.135726in}{2.076552in}}%
\pgfpathlineto{\pgfqpoint{4.137206in}{3.194884in}}%
\pgfpathlineto{\pgfqpoint{4.138686in}{3.019854in}}%
\pgfpathlineto{\pgfqpoint{4.139672in}{3.092601in}}%
\pgfpathlineto{\pgfqpoint{4.140165in}{3.378581in}}%
\pgfpathlineto{\pgfqpoint{4.141152in}{3.369430in}}%
\pgfpathlineto{\pgfqpoint{4.141645in}{3.343856in}}%
\pgfpathlineto{\pgfqpoint{4.142632in}{2.702664in}}%
\pgfpathlineto{\pgfqpoint{4.143125in}{3.364292in}}%
\pgfpathlineto{\pgfqpoint{4.144111in}{3.304573in}}%
\pgfpathlineto{\pgfqpoint{4.144605in}{2.247918in}}%
\pgfpathlineto{\pgfqpoint{4.145098in}{3.327428in}}%
\pgfpathlineto{\pgfqpoint{4.146577in}{3.327428in}}%
\pgfpathlineto{\pgfqpoint{4.147071in}{3.366652in}}%
\pgfpathlineto{\pgfqpoint{4.148550in}{3.286313in}}%
\pgfpathlineto{\pgfqpoint{4.149044in}{3.379194in}}%
\pgfpathlineto{\pgfqpoint{4.149537in}{2.916930in}}%
\pgfpathlineto{\pgfqpoint{4.150030in}{2.975671in}}%
\pgfpathlineto{\pgfqpoint{4.151017in}{3.377461in}}%
\pgfpathlineto{\pgfqpoint{4.151510in}{3.357570in}}%
\pgfpathlineto{\pgfqpoint{4.152003in}{2.585261in}}%
\pgfpathlineto{\pgfqpoint{4.152496in}{3.063749in}}%
\pgfpathlineto{\pgfqpoint{4.152990in}{3.247341in}}%
\pgfpathlineto{\pgfqpoint{4.153483in}{3.100557in}}%
\pgfpathlineto{\pgfqpoint{4.153976in}{3.100557in}}%
\pgfpathlineto{\pgfqpoint{4.154962in}{3.377460in}}%
\pgfpathlineto{\pgfqpoint{4.155456in}{3.015991in}}%
\pgfpathlineto{\pgfqpoint{4.155949in}{3.378601in}}%
\pgfpathlineto{\pgfqpoint{4.156935in}{3.378601in}}%
\pgfpathlineto{\pgfqpoint{4.157922in}{3.227780in}}%
\pgfpathlineto{\pgfqpoint{4.158908in}{3.378907in}}%
\pgfpathlineto{\pgfqpoint{4.159402in}{3.378457in}}%
\pgfpathlineto{\pgfqpoint{4.159895in}{2.120945in}}%
\pgfpathlineto{\pgfqpoint{4.160388in}{2.820693in}}%
\pgfpathlineto{\pgfqpoint{4.160881in}{2.820693in}}%
\pgfpathlineto{\pgfqpoint{4.162361in}{3.368511in}}%
\pgfpathlineto{\pgfqpoint{4.162854in}{2.081831in}}%
\pgfpathlineto{\pgfqpoint{4.163347in}{3.369164in}}%
\pgfpathlineto{\pgfqpoint{4.164334in}{3.369164in}}%
\pgfpathlineto{\pgfqpoint{4.164827in}{3.335986in}}%
\pgfpathlineto{\pgfqpoint{4.166307in}{2.761610in}}%
\pgfpathlineto{\pgfqpoint{4.166800in}{2.843023in}}%
\pgfpathlineto{\pgfqpoint{4.167293in}{3.361914in}}%
\pgfpathlineto{\pgfqpoint{4.167786in}{2.396729in}}%
\pgfpathlineto{\pgfqpoint{4.168280in}{3.107314in}}%
\pgfpathlineto{\pgfqpoint{4.169759in}{3.377179in}}%
\pgfpathlineto{\pgfqpoint{4.170253in}{3.377179in}}%
\pgfpathlineto{\pgfqpoint{4.171732in}{3.343869in}}%
\pgfpathlineto{\pgfqpoint{4.172719in}{2.821699in}}%
\pgfpathlineto{\pgfqpoint{4.173212in}{3.378903in}}%
\pgfpathlineto{\pgfqpoint{4.174198in}{3.282267in}}%
\pgfpathlineto{\pgfqpoint{4.175185in}{3.282267in}}%
\pgfpathlineto{\pgfqpoint{4.176665in}{3.292674in}}%
\pgfpathlineto{\pgfqpoint{4.177158in}{3.292674in}}%
\pgfpathlineto{\pgfqpoint{4.178144in}{2.287371in}}%
\pgfpathlineto{\pgfqpoint{4.178638in}{2.600590in}}%
\pgfpathlineto{\pgfqpoint{4.180117in}{3.086332in}}%
\pgfpathlineto{\pgfqpoint{4.180610in}{3.086332in}}%
\pgfpathlineto{\pgfqpoint{4.181104in}{3.282034in}}%
\pgfpathlineto{\pgfqpoint{4.181597in}{3.009819in}}%
\pgfpathlineto{\pgfqpoint{4.182090in}{3.340912in}}%
\pgfpathlineto{\pgfqpoint{4.183570in}{2.976175in}}%
\pgfpathlineto{\pgfqpoint{4.185050in}{3.221400in}}%
\pgfpathlineto{\pgfqpoint{4.185543in}{3.221400in}}%
\pgfpathlineto{\pgfqpoint{4.186529in}{3.368782in}}%
\pgfpathlineto{\pgfqpoint{4.187516in}{2.598381in}}%
\pgfpathlineto{\pgfqpoint{4.188995in}{3.343291in}}%
\pgfpathlineto{\pgfqpoint{4.190475in}{3.356625in}}%
\pgfpathlineto{\pgfqpoint{4.190968in}{3.171936in}}%
\pgfpathlineto{\pgfqpoint{4.191462in}{3.277680in}}%
\pgfpathlineto{\pgfqpoint{4.191955in}{3.277680in}}%
\pgfpathlineto{\pgfqpoint{4.192941in}{3.374130in}}%
\pgfpathlineto{\pgfqpoint{4.193434in}{3.373781in}}%
\pgfpathlineto{\pgfqpoint{4.195901in}{3.373781in}}%
\pgfpathlineto{\pgfqpoint{4.196394in}{3.349917in}}%
\pgfpathlineto{\pgfqpoint{4.196887in}{3.377705in}}%
\pgfpathlineto{\pgfqpoint{4.198367in}{3.377705in}}%
\pgfpathlineto{\pgfqpoint{4.199353in}{3.379899in}}%
\pgfpathlineto{\pgfqpoint{4.200340in}{3.184462in}}%
\pgfpathlineto{\pgfqpoint{4.201326in}{3.325190in}}%
\pgfpathlineto{\pgfqpoint{4.201819in}{3.155280in}}%
\pgfpathlineto{\pgfqpoint{4.202313in}{3.364278in}}%
\pgfpathlineto{\pgfqpoint{4.203792in}{3.364278in}}%
\pgfpathlineto{\pgfqpoint{4.204286in}{3.072960in}}%
\pgfpathlineto{\pgfqpoint{4.204779in}{3.379199in}}%
\pgfpathlineto{\pgfqpoint{4.205765in}{3.378896in}}%
\pgfpathlineto{\pgfqpoint{4.206258in}{2.813795in}}%
\pgfpathlineto{\pgfqpoint{4.206752in}{3.340488in}}%
\pgfpathlineto{\pgfqpoint{4.207245in}{3.339341in}}%
\pgfpathlineto{\pgfqpoint{4.208231in}{2.565252in}}%
\pgfpathlineto{\pgfqpoint{4.209218in}{3.318296in}}%
\pgfpathlineto{\pgfqpoint{4.209711in}{3.182908in}}%
\pgfpathlineto{\pgfqpoint{4.210698in}{3.273124in}}%
\pgfpathlineto{\pgfqpoint{4.211191in}{3.263368in}}%
\pgfpathlineto{\pgfqpoint{4.212177in}{3.263368in}}%
\pgfpathlineto{\pgfqpoint{4.214150in}{2.950719in}}%
\pgfpathlineto{\pgfqpoint{4.215630in}{3.323694in}}%
\pgfpathlineto{\pgfqpoint{4.217110in}{3.376671in}}%
\pgfpathlineto{\pgfqpoint{4.217603in}{3.379965in}}%
\pgfpathlineto{\pgfqpoint{4.219576in}{3.098888in}}%
\pgfpathlineto{\pgfqpoint{4.220069in}{3.071995in}}%
\pgfpathlineto{\pgfqpoint{4.221549in}{3.379687in}}%
\pgfpathlineto{\pgfqpoint{4.223028in}{3.379687in}}%
\pgfpathlineto{\pgfqpoint{4.224508in}{3.213805in}}%
\pgfpathlineto{\pgfqpoint{4.225494in}{3.202040in}}%
\pgfpathlineto{\pgfqpoint{4.225988in}{3.131467in}}%
\pgfpathlineto{\pgfqpoint{4.227467in}{3.354912in}}%
\pgfpathlineto{\pgfqpoint{4.227961in}{3.354912in}}%
\pgfpathlineto{\pgfqpoint{4.228947in}{3.359666in}}%
\pgfpathlineto{\pgfqpoint{4.229440in}{2.461445in}}%
\pgfpathlineto{\pgfqpoint{4.229934in}{3.372317in}}%
\pgfpathlineto{\pgfqpoint{4.230427in}{3.372317in}}%
\pgfpathlineto{\pgfqpoint{4.231906in}{3.379945in}}%
\pgfpathlineto{\pgfqpoint{4.232893in}{3.379945in}}%
\pgfpathlineto{\pgfqpoint{4.233386in}{2.982701in}}%
\pgfpathlineto{\pgfqpoint{4.233879in}{3.153547in}}%
\pgfpathlineto{\pgfqpoint{4.234866in}{3.343813in}}%
\pgfpathlineto{\pgfqpoint{4.235359in}{3.342235in}}%
\pgfpathlineto{\pgfqpoint{4.236839in}{3.379816in}}%
\pgfpathlineto{\pgfqpoint{4.238318in}{3.379816in}}%
\pgfpathlineto{\pgfqpoint{4.241278in}{3.200676in}}%
\pgfpathlineto{\pgfqpoint{4.242264in}{3.367389in}}%
\pgfpathlineto{\pgfqpoint{4.242758in}{3.346242in}}%
\pgfpathlineto{\pgfqpoint{4.243251in}{3.346242in}}%
\pgfpathlineto{\pgfqpoint{4.244237in}{3.364858in}}%
\pgfpathlineto{\pgfqpoint{4.245717in}{3.322580in}}%
\pgfpathlineto{\pgfqpoint{4.246210in}{3.322580in}}%
\pgfpathlineto{\pgfqpoint{4.247197in}{3.367198in}}%
\pgfpathlineto{\pgfqpoint{4.248183in}{3.272027in}}%
\pgfpathlineto{\pgfqpoint{4.248676in}{3.379956in}}%
\pgfpathlineto{\pgfqpoint{4.249663in}{3.377297in}}%
\pgfpathlineto{\pgfqpoint{4.250156in}{3.377297in}}%
\pgfpathlineto{\pgfqpoint{4.250649in}{3.194942in}}%
\pgfpathlineto{\pgfqpoint{4.251143in}{3.276035in}}%
\pgfpathlineto{\pgfqpoint{4.251636in}{3.276035in}}%
\pgfpathlineto{\pgfqpoint{4.252129in}{2.869963in}}%
\pgfpathlineto{\pgfqpoint{4.252622in}{3.184470in}}%
\pgfpathlineto{\pgfqpoint{4.254102in}{3.368563in}}%
\pgfpathlineto{\pgfqpoint{4.254595in}{2.364212in}}%
\pgfpathlineto{\pgfqpoint{4.255088in}{3.373867in}}%
\pgfpathlineto{\pgfqpoint{4.256568in}{2.649463in}}%
\pgfpathlineto{\pgfqpoint{4.258048in}{3.379995in}}%
\pgfpathlineto{\pgfqpoint{4.258541in}{3.379995in}}%
\pgfpathlineto{\pgfqpoint{4.260514in}{3.250656in}}%
\pgfpathlineto{\pgfqpoint{4.261994in}{3.378979in}}%
\pgfpathlineto{\pgfqpoint{4.265446in}{3.378979in}}%
\pgfpathlineto{\pgfqpoint{4.266926in}{3.321636in}}%
\pgfpathlineto{\pgfqpoint{4.267912in}{3.348489in}}%
\pgfpathlineto{\pgfqpoint{4.269392in}{2.960512in}}%
\pgfpathlineto{\pgfqpoint{4.269885in}{3.099335in}}%
\pgfpathlineto{\pgfqpoint{4.270379in}{2.874861in}}%
\pgfpathlineto{\pgfqpoint{4.271858in}{3.379562in}}%
\pgfpathlineto{\pgfqpoint{4.274818in}{3.142504in}}%
\pgfpathlineto{\pgfqpoint{4.275804in}{3.217478in}}%
\pgfpathlineto{\pgfqpoint{4.276297in}{2.413039in}}%
\pgfpathlineto{\pgfqpoint{4.277284in}{3.379816in}}%
\pgfpathlineto{\pgfqpoint{4.277777in}{3.376973in}}%
\pgfpathlineto{\pgfqpoint{4.278270in}{3.376973in}}%
\pgfpathlineto{\pgfqpoint{4.279257in}{3.371692in}}%
\pgfpathlineto{\pgfqpoint{4.280243in}{3.240798in}}%
\pgfpathlineto{\pgfqpoint{4.281723in}{3.356828in}}%
\pgfpathlineto{\pgfqpoint{4.282709in}{3.193239in}}%
\pgfpathlineto{\pgfqpoint{4.283696in}{3.373751in}}%
\pgfpathlineto{\pgfqpoint{4.284682in}{2.909952in}}%
\pgfpathlineto{\pgfqpoint{4.286162in}{3.370044in}}%
\pgfpathlineto{\pgfqpoint{4.287642in}{3.245258in}}%
\pgfpathlineto{\pgfqpoint{4.288135in}{3.245258in}}%
\pgfpathlineto{\pgfqpoint{4.289615in}{3.352128in}}%
\pgfpathlineto{\pgfqpoint{4.290108in}{3.352128in}}%
\pgfpathlineto{\pgfqpoint{4.290601in}{2.872329in}}%
\pgfpathlineto{\pgfqpoint{4.291094in}{3.145737in}}%
\pgfpathlineto{\pgfqpoint{4.291587in}{3.365127in}}%
\pgfpathlineto{\pgfqpoint{4.292574in}{3.347437in}}%
\pgfpathlineto{\pgfqpoint{4.295533in}{3.347437in}}%
\pgfpathlineto{\pgfqpoint{4.296520in}{1.818978in}}%
\pgfpathlineto{\pgfqpoint{4.297013in}{1.972318in}}%
\pgfpathlineto{\pgfqpoint{4.298493in}{3.379021in}}%
\pgfpathlineto{\pgfqpoint{4.300466in}{3.379021in}}%
\pgfpathlineto{\pgfqpoint{4.300959in}{3.373270in}}%
\pgfpathlineto{\pgfqpoint{4.301945in}{3.230788in}}%
\pgfpathlineto{\pgfqpoint{4.303425in}{3.366451in}}%
\pgfpathlineto{\pgfqpoint{4.303918in}{2.630269in}}%
\pgfpathlineto{\pgfqpoint{4.304411in}{3.137222in}}%
\pgfpathlineto{\pgfqpoint{4.305891in}{3.232139in}}%
\pgfpathlineto{\pgfqpoint{4.307371in}{2.495668in}}%
\pgfpathlineto{\pgfqpoint{4.307864in}{3.200724in}}%
\pgfpathlineto{\pgfqpoint{4.308357in}{3.009833in}}%
\pgfpathlineto{\pgfqpoint{4.308851in}{2.456035in}}%
\pgfpathlineto{\pgfqpoint{4.309344in}{2.755842in}}%
\pgfpathlineto{\pgfqpoint{4.309837in}{3.360394in}}%
\pgfpathlineto{\pgfqpoint{4.310823in}{3.358208in}}%
\pgfpathlineto{\pgfqpoint{4.313290in}{2.760814in}}%
\pgfpathlineto{\pgfqpoint{4.314769in}{3.339736in}}%
\pgfpathlineto{\pgfqpoint{4.315263in}{3.371563in}}%
\pgfpathlineto{\pgfqpoint{4.315756in}{2.698407in}}%
\pgfpathlineto{\pgfqpoint{4.316742in}{2.860929in}}%
\pgfpathlineto{\pgfqpoint{4.317235in}{3.318343in}}%
\pgfpathlineto{\pgfqpoint{4.318222in}{3.265406in}}%
\pgfpathlineto{\pgfqpoint{4.318715in}{3.265406in}}%
\pgfpathlineto{\pgfqpoint{4.319208in}{3.376221in}}%
\pgfpathlineto{\pgfqpoint{4.320195in}{3.358965in}}%
\pgfpathlineto{\pgfqpoint{4.320688in}{3.353948in}}%
\pgfpathlineto{\pgfqpoint{4.321675in}{3.232624in}}%
\pgfpathlineto{\pgfqpoint{4.322168in}{3.255480in}}%
\pgfpathlineto{\pgfqpoint{4.323647in}{3.367835in}}%
\pgfpathlineto{\pgfqpoint{4.325620in}{3.328773in}}%
\pgfpathlineto{\pgfqpoint{4.327593in}{3.328773in}}%
\pgfpathlineto{\pgfqpoint{4.329073in}{2.998881in}}%
\pgfpathlineto{\pgfqpoint{4.330059in}{3.341279in}}%
\pgfpathlineto{\pgfqpoint{4.330553in}{3.337624in}}%
\pgfpathlineto{\pgfqpoint{4.332032in}{3.248729in}}%
\pgfpathlineto{\pgfqpoint{4.332526in}{3.248729in}}%
\pgfpathlineto{\pgfqpoint{4.334005in}{3.371676in}}%
\pgfpathlineto{\pgfqpoint{4.335485in}{3.379908in}}%
\pgfpathlineto{\pgfqpoint{4.336965in}{3.379946in}}%
\pgfpathlineto{\pgfqpoint{4.337458in}{2.942794in}}%
\pgfpathlineto{\pgfqpoint{4.337951in}{3.316979in}}%
\pgfpathlineto{\pgfqpoint{4.338444in}{3.316979in}}%
\pgfpathlineto{\pgfqpoint{4.338938in}{3.117508in}}%
\pgfpathlineto{\pgfqpoint{4.339431in}{3.369742in}}%
\pgfpathlineto{\pgfqpoint{4.340417in}{3.363067in}}%
\pgfpathlineto{\pgfqpoint{4.340911in}{3.363067in}}%
\pgfpathlineto{\pgfqpoint{4.341404in}{3.206032in}}%
\pgfpathlineto{\pgfqpoint{4.341897in}{3.379519in}}%
\pgfpathlineto{\pgfqpoint{4.342883in}{3.379519in}}%
\pgfpathlineto{\pgfqpoint{4.343870in}{2.538702in}}%
\pgfpathlineto{\pgfqpoint{4.345350in}{3.198188in}}%
\pgfpathlineto{\pgfqpoint{4.345843in}{3.198188in}}%
\pgfpathlineto{\pgfqpoint{4.346336in}{3.379803in}}%
\pgfpathlineto{\pgfqpoint{4.347816in}{3.080420in}}%
\pgfpathlineto{\pgfqpoint{4.349295in}{3.369542in}}%
\pgfpathlineto{\pgfqpoint{4.349789in}{3.369542in}}%
\pgfpathlineto{\pgfqpoint{4.350282in}{2.044297in}}%
\pgfpathlineto{\pgfqpoint{4.350775in}{2.374105in}}%
\pgfpathlineto{\pgfqpoint{4.352255in}{3.345048in}}%
\pgfpathlineto{\pgfqpoint{4.353241in}{3.345048in}}%
\pgfpathlineto{\pgfqpoint{4.353735in}{2.981937in}}%
\pgfpathlineto{\pgfqpoint{4.354228in}{3.358748in}}%
\pgfpathlineto{\pgfqpoint{4.354721in}{3.368930in}}%
\pgfpathlineto{\pgfqpoint{4.355214in}{2.924208in}}%
\pgfpathlineto{\pgfqpoint{4.355708in}{3.363586in}}%
\pgfpathlineto{\pgfqpoint{4.356201in}{3.363586in}}%
\pgfpathlineto{\pgfqpoint{4.356694in}{1.125142in}}%
\pgfpathlineto{\pgfqpoint{4.357187in}{2.971001in}}%
\pgfpathlineto{\pgfqpoint{4.357680in}{2.971001in}}%
\pgfpathlineto{\pgfqpoint{4.358174in}{3.379080in}}%
\pgfpathlineto{\pgfqpoint{4.358667in}{3.078427in}}%
\pgfpathlineto{\pgfqpoint{4.360147in}{3.042101in}}%
\pgfpathlineto{\pgfqpoint{4.360640in}{2.701300in}}%
\pgfpathlineto{\pgfqpoint{4.361133in}{3.289561in}}%
\pgfpathlineto{\pgfqpoint{4.362120in}{3.209874in}}%
\pgfpathlineto{\pgfqpoint{4.362613in}{3.209874in}}%
\pgfpathlineto{\pgfqpoint{4.364092in}{3.367037in}}%
\pgfpathlineto{\pgfqpoint{4.364586in}{3.321955in}}%
\pgfpathlineto{\pgfqpoint{4.365079in}{3.338071in}}%
\pgfpathlineto{\pgfqpoint{4.366065in}{3.338071in}}%
\pgfpathlineto{\pgfqpoint{4.366559in}{3.002375in}}%
\pgfpathlineto{\pgfqpoint{4.367052in}{3.135660in}}%
\pgfpathlineto{\pgfqpoint{4.367545in}{3.187596in}}%
\pgfpathlineto{\pgfqpoint{4.368532in}{2.338891in}}%
\pgfpathlineto{\pgfqpoint{4.370011in}{3.315570in}}%
\pgfpathlineto{\pgfqpoint{4.371491in}{3.379987in}}%
\pgfpathlineto{\pgfqpoint{4.372477in}{3.326899in}}%
\pgfpathlineto{\pgfqpoint{4.373464in}{2.736836in}}%
\pgfpathlineto{\pgfqpoint{4.374944in}{3.362847in}}%
\pgfpathlineto{\pgfqpoint{4.376423in}{3.367826in}}%
\pgfpathlineto{\pgfqpoint{4.377410in}{3.330619in}}%
\pgfpathlineto{\pgfqpoint{4.378396in}{2.952694in}}%
\pgfpathlineto{\pgfqpoint{4.380369in}{3.294682in}}%
\pgfpathlineto{\pgfqpoint{4.380862in}{3.338505in}}%
\pgfpathlineto{\pgfqpoint{4.381356in}{3.038247in}}%
\pgfpathlineto{\pgfqpoint{4.381849in}{3.331676in}}%
\pgfpathlineto{\pgfqpoint{4.383822in}{3.042263in}}%
\pgfpathlineto{\pgfqpoint{4.385301in}{3.318432in}}%
\pgfpathlineto{\pgfqpoint{4.386781in}{3.318432in}}%
\pgfpathlineto{\pgfqpoint{4.388261in}{3.378923in}}%
\pgfpathlineto{\pgfqpoint{4.388754in}{2.770473in}}%
\pgfpathlineto{\pgfqpoint{4.389247in}{3.028916in}}%
\pgfpathlineto{\pgfqpoint{4.389740in}{3.375906in}}%
\pgfpathlineto{\pgfqpoint{4.390727in}{3.304476in}}%
\pgfpathlineto{\pgfqpoint{4.391220in}{3.153034in}}%
\pgfpathlineto{\pgfqpoint{4.392700in}{3.380000in}}%
\pgfpathlineto{\pgfqpoint{4.393193in}{3.380000in}}%
\pgfpathlineto{\pgfqpoint{4.393686in}{3.336322in}}%
\pgfpathlineto{\pgfqpoint{4.394180in}{3.353165in}}%
\pgfpathlineto{\pgfqpoint{4.396646in}{3.353165in}}%
\pgfpathlineto{\pgfqpoint{4.399605in}{3.012448in}}%
\pgfpathlineto{\pgfqpoint{4.400098in}{3.012448in}}%
\pgfpathlineto{\pgfqpoint{4.401578in}{3.328474in}}%
\pgfpathlineto{\pgfqpoint{4.402564in}{2.925071in}}%
\pgfpathlineto{\pgfqpoint{4.403551in}{3.211230in}}%
\pgfpathlineto{\pgfqpoint{4.404537in}{2.952722in}}%
\pgfpathlineto{\pgfqpoint{4.405031in}{3.004676in}}%
\pgfpathlineto{\pgfqpoint{4.405524in}{3.004676in}}%
\pgfpathlineto{\pgfqpoint{4.406017in}{2.999631in}}%
\pgfpathlineto{\pgfqpoint{4.406510in}{3.324591in}}%
\pgfpathlineto{\pgfqpoint{4.407497in}{3.302676in}}%
\pgfpathlineto{\pgfqpoint{4.409470in}{3.219046in}}%
\pgfpathlineto{\pgfqpoint{4.410456in}{3.219046in}}%
\pgfpathlineto{\pgfqpoint{4.411936in}{3.354224in}}%
\pgfpathlineto{\pgfqpoint{4.412429in}{3.354224in}}%
\pgfpathlineto{\pgfqpoint{4.413416in}{3.316427in}}%
\pgfpathlineto{\pgfqpoint{4.414895in}{2.763365in}}%
\pgfpathlineto{\pgfqpoint{4.415882in}{3.108308in}}%
\pgfpathlineto{\pgfqpoint{4.416868in}{2.976920in}}%
\pgfpathlineto{\pgfqpoint{4.418841in}{3.363501in}}%
\pgfpathlineto{\pgfqpoint{4.419334in}{3.331518in}}%
\pgfpathlineto{\pgfqpoint{4.419828in}{3.361628in}}%
\pgfpathlineto{\pgfqpoint{4.420321in}{3.361628in}}%
\pgfpathlineto{\pgfqpoint{4.420814in}{3.066231in}}%
\pgfpathlineto{\pgfqpoint{4.421307in}{3.345326in}}%
\pgfpathlineto{\pgfqpoint{4.422294in}{3.344811in}}%
\pgfpathlineto{\pgfqpoint{4.422787in}{3.065716in}}%
\pgfpathlineto{\pgfqpoint{4.423280in}{3.301626in}}%
\pgfpathlineto{\pgfqpoint{4.423773in}{3.377386in}}%
\pgfpathlineto{\pgfqpoint{4.424267in}{3.269357in}}%
\pgfpathlineto{\pgfqpoint{4.424760in}{3.323805in}}%
\pgfpathlineto{\pgfqpoint{4.426240in}{2.658725in}}%
\pgfpathlineto{\pgfqpoint{4.426733in}{2.658725in}}%
\pgfpathlineto{\pgfqpoint{4.427719in}{3.379820in}}%
\pgfpathlineto{\pgfqpoint{4.428212in}{3.227029in}}%
\pgfpathlineto{\pgfqpoint{4.429692in}{3.214642in}}%
\pgfpathlineto{\pgfqpoint{4.430185in}{2.368134in}}%
\pgfpathlineto{\pgfqpoint{4.430679in}{2.445193in}}%
\pgfpathlineto{\pgfqpoint{4.431172in}{3.316057in}}%
\pgfpathlineto{\pgfqpoint{4.432158in}{3.285583in}}%
\pgfpathlineto{\pgfqpoint{4.433145in}{3.285583in}}%
\pgfpathlineto{\pgfqpoint{4.433638in}{3.356217in}}%
\pgfpathlineto{\pgfqpoint{4.434131in}{3.344082in}}%
\pgfpathlineto{\pgfqpoint{4.434624in}{3.264719in}}%
\pgfpathlineto{\pgfqpoint{4.435118in}{3.282936in}}%
\pgfpathlineto{\pgfqpoint{4.436597in}{3.300060in}}%
\pgfpathlineto{\pgfqpoint{4.437091in}{3.300060in}}%
\pgfpathlineto{\pgfqpoint{4.438077in}{3.321138in}}%
\pgfpathlineto{\pgfqpoint{4.439557in}{2.844732in}}%
\pgfpathlineto{\pgfqpoint{4.440050in}{2.844732in}}%
\pgfpathlineto{\pgfqpoint{4.440543in}{3.331600in}}%
\pgfpathlineto{\pgfqpoint{4.441530in}{3.294732in}}%
\pgfpathlineto{\pgfqpoint{4.443009in}{3.294732in}}%
\pgfpathlineto{\pgfqpoint{4.443503in}{3.307035in}}%
\pgfpathlineto{\pgfqpoint{4.444982in}{3.376256in}}%
\pgfpathlineto{\pgfqpoint{4.445476in}{3.376256in}}%
\pgfpathlineto{\pgfqpoint{4.446462in}{3.370538in}}%
\pgfpathlineto{\pgfqpoint{4.447448in}{3.321501in}}%
\pgfpathlineto{\pgfqpoint{4.447942in}{2.854087in}}%
\pgfpathlineto{\pgfqpoint{4.448435in}{3.259514in}}%
\pgfpathlineto{\pgfqpoint{4.449421in}{3.226179in}}%
\pgfpathlineto{\pgfqpoint{4.450901in}{3.377514in}}%
\pgfpathlineto{\pgfqpoint{4.451888in}{3.377951in}}%
\pgfpathlineto{\pgfqpoint{4.452381in}{3.326381in}}%
\pgfpathlineto{\pgfqpoint{4.454847in}{3.373293in}}%
\pgfpathlineto{\pgfqpoint{4.455340in}{3.242114in}}%
\pgfpathlineto{\pgfqpoint{4.455833in}{3.328288in}}%
\pgfpathlineto{\pgfqpoint{4.456327in}{3.346345in}}%
\pgfpathlineto{\pgfqpoint{4.457806in}{3.104722in}}%
\pgfpathlineto{\pgfqpoint{4.459286in}{3.366826in}}%
\pgfpathlineto{\pgfqpoint{4.461752in}{3.366826in}}%
\pgfpathlineto{\pgfqpoint{4.462245in}{3.101471in}}%
\pgfpathlineto{\pgfqpoint{4.462739in}{3.342840in}}%
\pgfpathlineto{\pgfqpoint{4.463725in}{3.342840in}}%
\pgfpathlineto{\pgfqpoint{4.465698in}{2.916341in}}%
\pgfpathlineto{\pgfqpoint{4.466191in}{3.240529in}}%
\pgfpathlineto{\pgfqpoint{4.466685in}{3.233049in}}%
\pgfpathlineto{\pgfqpoint{4.467671in}{3.043808in}}%
\pgfpathlineto{\pgfqpoint{4.468657in}{3.378030in}}%
\pgfpathlineto{\pgfqpoint{4.470137in}{3.111142in}}%
\pgfpathlineto{\pgfqpoint{4.471124in}{3.002060in}}%
\pgfpathlineto{\pgfqpoint{4.472110in}{3.349570in}}%
\pgfpathlineto{\pgfqpoint{4.472603in}{3.305881in}}%
\pgfpathlineto{\pgfqpoint{4.474083in}{3.365342in}}%
\pgfpathlineto{\pgfqpoint{4.475069in}{3.365342in}}%
\pgfpathlineto{\pgfqpoint{4.476549in}{3.377135in}}%
\pgfpathlineto{\pgfqpoint{4.477536in}{3.089543in}}%
\pgfpathlineto{\pgfqpoint{4.478522in}{3.372406in}}%
\pgfpathlineto{\pgfqpoint{4.479015in}{3.264755in}}%
\pgfpathlineto{\pgfqpoint{4.479509in}{2.871429in}}%
\pgfpathlineto{\pgfqpoint{4.480002in}{3.152496in}}%
\pgfpathlineto{\pgfqpoint{4.480495in}{3.152496in}}%
\pgfpathlineto{\pgfqpoint{4.480988in}{2.951312in}}%
\pgfpathlineto{\pgfqpoint{4.481481in}{3.212216in}}%
\pgfpathlineto{\pgfqpoint{4.481975in}{2.888911in}}%
\pgfpathlineto{\pgfqpoint{4.482961in}{2.537667in}}%
\pgfpathlineto{\pgfqpoint{4.484441in}{3.251045in}}%
\pgfpathlineto{\pgfqpoint{4.485427in}{3.379725in}}%
\pgfpathlineto{\pgfqpoint{4.486907in}{3.301774in}}%
\pgfpathlineto{\pgfqpoint{4.487400in}{3.301774in}}%
\pgfpathlineto{\pgfqpoint{4.488880in}{3.174922in}}%
\pgfpathlineto{\pgfqpoint{4.489373in}{3.174922in}}%
\pgfpathlineto{\pgfqpoint{4.489866in}{3.256087in}}%
\pgfpathlineto{\pgfqpoint{4.490360in}{3.230884in}}%
\pgfpathlineto{\pgfqpoint{4.490853in}{2.592003in}}%
\pgfpathlineto{\pgfqpoint{4.491346in}{3.263826in}}%
\pgfpathlineto{\pgfqpoint{4.492826in}{3.205631in}}%
\pgfpathlineto{\pgfqpoint{4.493319in}{3.375862in}}%
\pgfpathlineto{\pgfqpoint{4.493812in}{3.297723in}}%
\pgfpathlineto{\pgfqpoint{4.494305in}{3.234690in}}%
\pgfpathlineto{\pgfqpoint{4.494799in}{3.378983in}}%
\pgfpathlineto{\pgfqpoint{4.495292in}{3.378932in}}%
\pgfpathlineto{\pgfqpoint{4.496772in}{3.057933in}}%
\pgfpathlineto{\pgfqpoint{4.497265in}{3.239111in}}%
\pgfpathlineto{\pgfqpoint{4.497758in}{2.963082in}}%
\pgfpathlineto{\pgfqpoint{4.498745in}{3.032136in}}%
\pgfpathlineto{\pgfqpoint{4.500224in}{3.335914in}}%
\pgfpathlineto{\pgfqpoint{4.500717in}{3.322725in}}%
\pgfpathlineto{\pgfqpoint{4.501704in}{3.079399in}}%
\pgfpathlineto{\pgfqpoint{4.503184in}{3.262727in}}%
\pgfpathlineto{\pgfqpoint{4.503677in}{3.245577in}}%
\pgfpathlineto{\pgfqpoint{4.505157in}{3.376216in}}%
\pgfpathlineto{\pgfqpoint{4.505650in}{3.372951in}}%
\pgfpathlineto{\pgfqpoint{4.506143in}{3.279463in}}%
\pgfpathlineto{\pgfqpoint{4.506636in}{3.334694in}}%
\pgfpathlineto{\pgfqpoint{4.507129in}{3.334694in}}%
\pgfpathlineto{\pgfqpoint{4.509102in}{3.377777in}}%
\pgfpathlineto{\pgfqpoint{4.510089in}{3.351834in}}%
\pgfpathlineto{\pgfqpoint{4.510582in}{2.890131in}}%
\pgfpathlineto{\pgfqpoint{4.511075in}{3.371690in}}%
\pgfpathlineto{\pgfqpoint{4.512062in}{3.371690in}}%
\pgfpathlineto{\pgfqpoint{4.512555in}{3.358784in}}%
\pgfpathlineto{\pgfqpoint{4.514035in}{3.379971in}}%
\pgfpathlineto{\pgfqpoint{4.516008in}{3.378719in}}%
\pgfpathlineto{\pgfqpoint{4.516501in}{3.003586in}}%
\pgfpathlineto{\pgfqpoint{4.516994in}{3.353587in}}%
\pgfpathlineto{\pgfqpoint{4.517487in}{3.353587in}}%
\pgfpathlineto{\pgfqpoint{4.517981in}{2.672130in}}%
\pgfpathlineto{\pgfqpoint{4.518474in}{3.376869in}}%
\pgfpathlineto{\pgfqpoint{4.519953in}{3.317053in}}%
\pgfpathlineto{\pgfqpoint{4.520447in}{3.308457in}}%
\pgfpathlineto{\pgfqpoint{4.520940in}{2.666544in}}%
\pgfpathlineto{\pgfqpoint{4.521433in}{2.801997in}}%
\pgfpathlineto{\pgfqpoint{4.521926in}{3.319275in}}%
\pgfpathlineto{\pgfqpoint{4.522913in}{3.285794in}}%
\pgfpathlineto{\pgfqpoint{4.523406in}{3.340165in}}%
\pgfpathlineto{\pgfqpoint{4.524886in}{2.690878in}}%
\pgfpathlineto{\pgfqpoint{4.526365in}{3.371112in}}%
\pgfpathlineto{\pgfqpoint{4.526859in}{3.371112in}}%
\pgfpathlineto{\pgfqpoint{4.527352in}{3.376420in}}%
\pgfpathlineto{\pgfqpoint{4.528832in}{3.300257in}}%
\pgfpathlineto{\pgfqpoint{4.529325in}{3.300257in}}%
\pgfpathlineto{\pgfqpoint{4.529818in}{2.944344in}}%
\pgfpathlineto{\pgfqpoint{4.530805in}{2.996450in}}%
\pgfpathlineto{\pgfqpoint{4.532284in}{3.347541in}}%
\pgfpathlineto{\pgfqpoint{4.534257in}{3.289935in}}%
\pgfpathlineto{\pgfqpoint{4.535737in}{2.654797in}}%
\pgfpathlineto{\pgfqpoint{4.536230in}{3.366060in}}%
\pgfpathlineto{\pgfqpoint{4.537217in}{3.359277in}}%
\pgfpathlineto{\pgfqpoint{4.537710in}{3.298814in}}%
\pgfpathlineto{\pgfqpoint{4.538203in}{3.360909in}}%
\pgfpathlineto{\pgfqpoint{4.539683in}{3.360909in}}%
\pgfpathlineto{\pgfqpoint{4.540176in}{3.210952in}}%
\pgfpathlineto{\pgfqpoint{4.540669in}{3.319625in}}%
\pgfpathlineto{\pgfqpoint{4.541162in}{2.351923in}}%
\pgfpathlineto{\pgfqpoint{4.541656in}{3.344447in}}%
\pgfpathlineto{\pgfqpoint{4.542642in}{3.329354in}}%
\pgfpathlineto{\pgfqpoint{4.543629in}{2.940179in}}%
\pgfpathlineto{\pgfqpoint{4.545108in}{3.359399in}}%
\pgfpathlineto{\pgfqpoint{4.546588in}{3.350413in}}%
\pgfpathlineto{\pgfqpoint{4.547574in}{3.350413in}}%
\pgfpathlineto{\pgfqpoint{4.548068in}{3.377956in}}%
\pgfpathlineto{\pgfqpoint{4.549054in}{3.252336in}}%
\pgfpathlineto{\pgfqpoint{4.550534in}{3.379000in}}%
\pgfpathlineto{\pgfqpoint{4.551520in}{3.360099in}}%
\pgfpathlineto{\pgfqpoint{4.552013in}{2.864863in}}%
\pgfpathlineto{\pgfqpoint{4.552507in}{3.047782in}}%
\pgfpathlineto{\pgfqpoint{4.553986in}{3.345123in}}%
\pgfpathlineto{\pgfqpoint{4.554973in}{3.379877in}}%
\pgfpathlineto{\pgfqpoint{4.555466in}{3.121027in}}%
\pgfpathlineto{\pgfqpoint{4.555959in}{3.379008in}}%
\pgfpathlineto{\pgfqpoint{4.556453in}{3.379008in}}%
\pgfpathlineto{\pgfqpoint{4.557932in}{3.370869in}}%
\pgfpathlineto{\pgfqpoint{4.559412in}{3.370869in}}%
\pgfpathlineto{\pgfqpoint{4.559905in}{3.042262in}}%
\pgfpathlineto{\pgfqpoint{4.560398in}{3.275838in}}%
\pgfpathlineto{\pgfqpoint{4.561878in}{3.377267in}}%
\pgfpathlineto{\pgfqpoint{4.563358in}{3.322590in}}%
\pgfpathlineto{\pgfqpoint{4.564344in}{3.347841in}}%
\pgfpathlineto{\pgfqpoint{4.565824in}{3.304624in}}%
\pgfpathlineto{\pgfqpoint{4.566810in}{1.700254in}}%
\pgfpathlineto{\pgfqpoint{4.568783in}{3.264838in}}%
\pgfpathlineto{\pgfqpoint{4.569770in}{3.264838in}}%
\pgfpathlineto{\pgfqpoint{4.570263in}{3.300026in}}%
\pgfpathlineto{\pgfqpoint{4.570756in}{3.162632in}}%
\pgfpathlineto{\pgfqpoint{4.571250in}{3.370623in}}%
\pgfpathlineto{\pgfqpoint{4.572236in}{3.331490in}}%
\pgfpathlineto{\pgfqpoint{4.573716in}{3.331490in}}%
\pgfpathlineto{\pgfqpoint{4.574209in}{3.352519in}}%
\pgfpathlineto{\pgfqpoint{4.574702in}{3.324636in}}%
\pgfpathlineto{\pgfqpoint{4.576675in}{2.780751in}}%
\pgfpathlineto{\pgfqpoint{4.578155in}{3.379717in}}%
\pgfpathlineto{\pgfqpoint{4.579141in}{2.954993in}}%
\pgfpathlineto{\pgfqpoint{4.579634in}{2.964631in}}%
\pgfpathlineto{\pgfqpoint{4.581114in}{3.333458in}}%
\pgfpathlineto{\pgfqpoint{4.582594in}{3.374562in}}%
\pgfpathlineto{\pgfqpoint{4.584074in}{2.857631in}}%
\pgfpathlineto{\pgfqpoint{4.585553in}{2.821643in}}%
\pgfpathlineto{\pgfqpoint{4.587033in}{3.379970in}}%
\pgfpathlineto{\pgfqpoint{4.588513in}{3.286093in}}%
\pgfpathlineto{\pgfqpoint{4.589006in}{3.377534in}}%
\pgfpathlineto{\pgfqpoint{4.589499in}{3.229224in}}%
\pgfpathlineto{\pgfqpoint{4.589992in}{3.360660in}}%
\pgfpathlineto{\pgfqpoint{4.591472in}{3.360660in}}%
\pgfpathlineto{\pgfqpoint{4.592952in}{3.227672in}}%
\pgfpathlineto{\pgfqpoint{4.594431in}{3.362460in}}%
\pgfpathlineto{\pgfqpoint{4.595911in}{3.356936in}}%
\pgfpathlineto{\pgfqpoint{4.596898in}{3.356936in}}%
\pgfpathlineto{\pgfqpoint{4.597884in}{3.323939in}}%
\pgfpathlineto{\pgfqpoint{4.599364in}{3.374665in}}%
\pgfpathlineto{\pgfqpoint{4.600350in}{3.050553in}}%
\pgfpathlineto{\pgfqpoint{4.600843in}{3.355481in}}%
\pgfpathlineto{\pgfqpoint{4.601337in}{2.126459in}}%
\pgfpathlineto{\pgfqpoint{4.601830in}{3.351858in}}%
\pgfpathlineto{\pgfqpoint{4.603803in}{3.351858in}}%
\pgfpathlineto{\pgfqpoint{4.605282in}{3.227683in}}%
\pgfpathlineto{\pgfqpoint{4.606762in}{3.222607in}}%
\pgfpathlineto{\pgfqpoint{4.607255in}{3.222607in}}%
\pgfpathlineto{\pgfqpoint{4.608735in}{3.348806in}}%
\pgfpathlineto{\pgfqpoint{4.609228in}{3.370900in}}%
\pgfpathlineto{\pgfqpoint{4.611201in}{3.212718in}}%
\pgfpathlineto{\pgfqpoint{4.611694in}{2.291750in}}%
\pgfpathlineto{\pgfqpoint{4.612188in}{3.296678in}}%
\pgfpathlineto{\pgfqpoint{4.613174in}{3.296678in}}%
\pgfpathlineto{\pgfqpoint{4.613667in}{3.223745in}}%
\pgfpathlineto{\pgfqpoint{4.614161in}{3.311669in}}%
\pgfpathlineto{\pgfqpoint{4.614654in}{3.311669in}}%
\pgfpathlineto{\pgfqpoint{4.616134in}{3.170201in}}%
\pgfpathlineto{\pgfqpoint{4.616627in}{3.376990in}}%
\pgfpathlineto{\pgfqpoint{4.617613in}{3.345298in}}%
\pgfpathlineto{\pgfqpoint{4.618106in}{3.271651in}}%
\pgfpathlineto{\pgfqpoint{4.619586in}{3.375822in}}%
\pgfpathlineto{\pgfqpoint{4.621066in}{3.192625in}}%
\pgfpathlineto{\pgfqpoint{4.622052in}{3.376331in}}%
\pgfpathlineto{\pgfqpoint{4.622546in}{3.305261in}}%
\pgfpathlineto{\pgfqpoint{4.623039in}{3.362306in}}%
\pgfpathlineto{\pgfqpoint{4.624025in}{3.362306in}}%
\pgfpathlineto{\pgfqpoint{4.625012in}{3.260101in}}%
\pgfpathlineto{\pgfqpoint{4.626491in}{3.358380in}}%
\pgfpathlineto{\pgfqpoint{4.626985in}{3.358380in}}%
\pgfpathlineto{\pgfqpoint{4.627478in}{3.026685in}}%
\pgfpathlineto{\pgfqpoint{4.627971in}{3.272705in}}%
\pgfpathlineto{\pgfqpoint{4.628958in}{3.272705in}}%
\pgfpathlineto{\pgfqpoint{4.630437in}{1.387359in}}%
\pgfpathlineto{\pgfqpoint{4.631917in}{3.340289in}}%
\pgfpathlineto{\pgfqpoint{4.635370in}{3.340289in}}%
\pgfpathlineto{\pgfqpoint{4.635863in}{3.262707in}}%
\pgfpathlineto{\pgfqpoint{4.636356in}{3.329420in}}%
\pgfpathlineto{\pgfqpoint{4.637342in}{3.329420in}}%
\pgfpathlineto{\pgfqpoint{4.638329in}{3.096885in}}%
\pgfpathlineto{\pgfqpoint{4.639809in}{3.248238in}}%
\pgfpathlineto{\pgfqpoint{4.640302in}{3.248238in}}%
\pgfpathlineto{\pgfqpoint{4.640795in}{3.379526in}}%
\pgfpathlineto{\pgfqpoint{4.642275in}{2.869948in}}%
\pgfpathlineto{\pgfqpoint{4.642768in}{3.173267in}}%
\pgfpathlineto{\pgfqpoint{4.643261in}{2.871287in}}%
\pgfpathlineto{\pgfqpoint{4.645727in}{3.378856in}}%
\pgfpathlineto{\pgfqpoint{4.646221in}{3.143406in}}%
\pgfpathlineto{\pgfqpoint{4.646714in}{3.261952in}}%
\pgfpathlineto{\pgfqpoint{4.648194in}{3.358287in}}%
\pgfpathlineto{\pgfqpoint{4.648687in}{3.284394in}}%
\pgfpathlineto{\pgfqpoint{4.649180in}{3.359325in}}%
\pgfpathlineto{\pgfqpoint{4.649673in}{3.359325in}}%
\pgfpathlineto{\pgfqpoint{4.650660in}{3.302923in}}%
\pgfpathlineto{\pgfqpoint{4.652139in}{3.359224in}}%
\pgfpathlineto{\pgfqpoint{4.653126in}{3.378774in}}%
\pgfpathlineto{\pgfqpoint{4.654606in}{3.017220in}}%
\pgfpathlineto{\pgfqpoint{4.656578in}{3.355699in}}%
\pgfpathlineto{\pgfqpoint{4.657072in}{3.346705in}}%
\pgfpathlineto{\pgfqpoint{4.658551in}{3.346705in}}%
\pgfpathlineto{\pgfqpoint{4.660031in}{3.333435in}}%
\pgfpathlineto{\pgfqpoint{4.661018in}{3.114235in}}%
\pgfpathlineto{\pgfqpoint{4.661511in}{3.282053in}}%
\pgfpathlineto{\pgfqpoint{4.662991in}{2.836472in}}%
\pgfpathlineto{\pgfqpoint{4.663484in}{3.365733in}}%
\pgfpathlineto{\pgfqpoint{4.664470in}{3.267652in}}%
\pgfpathlineto{\pgfqpoint{4.664963in}{3.267652in}}%
\pgfpathlineto{\pgfqpoint{4.665457in}{2.802960in}}%
\pgfpathlineto{\pgfqpoint{4.665950in}{3.282972in}}%
\pgfpathlineto{\pgfqpoint{4.666443in}{3.282972in}}%
\pgfpathlineto{\pgfqpoint{4.666936in}{2.677516in}}%
\pgfpathlineto{\pgfqpoint{4.667430in}{3.377657in}}%
\pgfpathlineto{\pgfqpoint{4.668909in}{2.940881in}}%
\pgfpathlineto{\pgfqpoint{4.669403in}{3.356218in}}%
\pgfpathlineto{\pgfqpoint{4.670389in}{3.342334in}}%
\pgfpathlineto{\pgfqpoint{4.671375in}{3.368311in}}%
\pgfpathlineto{\pgfqpoint{4.672362in}{3.233790in}}%
\pgfpathlineto{\pgfqpoint{4.673348in}{3.378616in}}%
\pgfpathlineto{\pgfqpoint{4.673842in}{3.354488in}}%
\pgfpathlineto{\pgfqpoint{4.674828in}{3.341665in}}%
\pgfpathlineto{\pgfqpoint{4.675321in}{2.910292in}}%
\pgfpathlineto{\pgfqpoint{4.676308in}{2.963615in}}%
\pgfpathlineto{\pgfqpoint{4.677294in}{2.963615in}}%
\pgfpathlineto{\pgfqpoint{4.678774in}{3.285793in}}%
\pgfpathlineto{\pgfqpoint{4.680254in}{2.825701in}}%
\pgfpathlineto{\pgfqpoint{4.681733in}{3.379220in}}%
\pgfpathlineto{\pgfqpoint{4.682720in}{3.368455in}}%
\pgfpathlineto{\pgfqpoint{4.684199in}{3.377359in}}%
\pgfpathlineto{\pgfqpoint{4.685186in}{3.377359in}}%
\pgfpathlineto{\pgfqpoint{4.686172in}{3.149635in}}%
\pgfpathlineto{\pgfqpoint{4.687652in}{3.230659in}}%
\pgfpathlineto{\pgfqpoint{4.688639in}{3.249018in}}%
\pgfpathlineto{\pgfqpoint{4.689132in}{3.003858in}}%
\pgfpathlineto{\pgfqpoint{4.689625in}{3.375994in}}%
\pgfpathlineto{\pgfqpoint{4.690611in}{3.371928in}}%
\pgfpathlineto{\pgfqpoint{4.692584in}{3.372606in}}%
\pgfpathlineto{\pgfqpoint{4.693571in}{3.377300in}}%
\pgfpathlineto{\pgfqpoint{4.694064in}{3.336494in}}%
\pgfpathlineto{\pgfqpoint{4.695051in}{3.336494in}}%
\pgfpathlineto{\pgfqpoint{4.695544in}{2.135710in}}%
\pgfpathlineto{\pgfqpoint{4.696037in}{3.107053in}}%
\pgfpathlineto{\pgfqpoint{4.696530in}{2.948376in}}%
\pgfpathlineto{\pgfqpoint{4.698010in}{3.360584in}}%
\pgfpathlineto{\pgfqpoint{4.698503in}{3.325705in}}%
\pgfpathlineto{\pgfqpoint{4.698996in}{3.352344in}}%
\pgfpathlineto{\pgfqpoint{4.699490in}{3.352344in}}%
\pgfpathlineto{\pgfqpoint{4.699983in}{2.741399in}}%
\pgfpathlineto{\pgfqpoint{4.700476in}{3.145793in}}%
\pgfpathlineto{\pgfqpoint{4.701956in}{3.344041in}}%
\pgfpathlineto{\pgfqpoint{4.703435in}{3.344041in}}%
\pgfpathlineto{\pgfqpoint{4.704915in}{3.315808in}}%
\pgfpathlineto{\pgfqpoint{4.705408in}{3.363035in}}%
\pgfpathlineto{\pgfqpoint{4.705902in}{2.958786in}}%
\pgfpathlineto{\pgfqpoint{4.706395in}{3.312014in}}%
\pgfpathlineto{\pgfqpoint{4.706888in}{3.378318in}}%
\pgfpathlineto{\pgfqpoint{4.707875in}{3.371848in}}%
\pgfpathlineto{\pgfqpoint{4.709354in}{2.437460in}}%
\pgfpathlineto{\pgfqpoint{4.710341in}{3.370864in}}%
\pgfpathlineto{\pgfqpoint{4.710834in}{3.354907in}}%
\pgfpathlineto{\pgfqpoint{4.711820in}{3.354907in}}%
\pgfpathlineto{\pgfqpoint{4.713300in}{3.378207in}}%
\pgfpathlineto{\pgfqpoint{4.714287in}{3.068708in}}%
\pgfpathlineto{\pgfqpoint{4.715766in}{3.342919in}}%
\pgfpathlineto{\pgfqpoint{4.717246in}{3.371608in}}%
\pgfpathlineto{\pgfqpoint{4.717739in}{3.155547in}}%
\pgfpathlineto{\pgfqpoint{4.718232in}{3.367541in}}%
\pgfpathlineto{\pgfqpoint{4.718726in}{3.367541in}}%
\pgfpathlineto{\pgfqpoint{4.720205in}{3.297349in}}%
\pgfpathlineto{\pgfqpoint{4.721192in}{3.372930in}}%
\pgfpathlineto{\pgfqpoint{4.721685in}{2.998542in}}%
\pgfpathlineto{\pgfqpoint{4.722178in}{3.378732in}}%
\pgfpathlineto{\pgfqpoint{4.722671in}{3.244375in}}%
\pgfpathlineto{\pgfqpoint{4.723165in}{3.283075in}}%
\pgfpathlineto{\pgfqpoint{4.723658in}{3.379012in}}%
\pgfpathlineto{\pgfqpoint{4.725631in}{2.748677in}}%
\pgfpathlineto{\pgfqpoint{4.726617in}{3.276086in}}%
\pgfpathlineto{\pgfqpoint{4.727111in}{1.967031in}}%
\pgfpathlineto{\pgfqpoint{4.727604in}{3.261073in}}%
\pgfpathlineto{\pgfqpoint{4.728590in}{3.081951in}}%
\pgfpathlineto{\pgfqpoint{4.729083in}{3.379977in}}%
\pgfpathlineto{\pgfqpoint{4.730070in}{3.310346in}}%
\pgfpathlineto{\pgfqpoint{4.731056in}{3.329085in}}%
\pgfpathlineto{\pgfqpoint{4.732043in}{3.098469in}}%
\pgfpathlineto{\pgfqpoint{4.733523in}{3.379901in}}%
\pgfpathlineto{\pgfqpoint{4.735002in}{3.379901in}}%
\pgfpathlineto{\pgfqpoint{4.736482in}{3.371609in}}%
\pgfpathlineto{\pgfqpoint{4.736975in}{3.371609in}}%
\pgfpathlineto{\pgfqpoint{4.737962in}{3.259588in}}%
\pgfpathlineto{\pgfqpoint{4.738948in}{3.379764in}}%
\pgfpathlineto{\pgfqpoint{4.739441in}{3.356740in}}%
\pgfpathlineto{\pgfqpoint{4.739935in}{3.356740in}}%
\pgfpathlineto{\pgfqpoint{4.740428in}{2.891116in}}%
\pgfpathlineto{\pgfqpoint{4.740921in}{3.355760in}}%
\pgfpathlineto{\pgfqpoint{4.741414in}{3.370093in}}%
\pgfpathlineto{\pgfqpoint{4.741907in}{3.202562in}}%
\pgfpathlineto{\pgfqpoint{4.742401in}{3.268239in}}%
\pgfpathlineto{\pgfqpoint{4.742894in}{3.268239in}}%
\pgfpathlineto{\pgfqpoint{4.743387in}{3.378635in}}%
\pgfpathlineto{\pgfqpoint{4.744867in}{3.213734in}}%
\pgfpathlineto{\pgfqpoint{4.745853in}{3.213734in}}%
\pgfpathlineto{\pgfqpoint{4.746840in}{3.024441in}}%
\pgfpathlineto{\pgfqpoint{4.747333in}{3.052484in}}%
\pgfpathlineto{\pgfqpoint{4.747826in}{3.052484in}}%
\pgfpathlineto{\pgfqpoint{4.748813in}{2.975703in}}%
\pgfpathlineto{\pgfqpoint{4.750292in}{3.324904in}}%
\pgfpathlineto{\pgfqpoint{4.750786in}{3.269513in}}%
\pgfpathlineto{\pgfqpoint{4.751279in}{3.374167in}}%
\pgfpathlineto{\pgfqpoint{4.751772in}{3.301319in}}%
\pgfpathlineto{\pgfqpoint{4.752759in}{3.301319in}}%
\pgfpathlineto{\pgfqpoint{4.753252in}{3.264931in}}%
\pgfpathlineto{\pgfqpoint{4.754731in}{3.364584in}}%
\pgfpathlineto{\pgfqpoint{4.755225in}{3.045517in}}%
\pgfpathlineto{\pgfqpoint{4.755718in}{3.090790in}}%
\pgfpathlineto{\pgfqpoint{4.757198in}{3.379853in}}%
\pgfpathlineto{\pgfqpoint{4.758184in}{3.379853in}}%
\pgfpathlineto{\pgfqpoint{4.759664in}{2.623568in}}%
\pgfpathlineto{\pgfqpoint{4.762130in}{3.311444in}}%
\pgfpathlineto{\pgfqpoint{4.763610in}{3.335807in}}%
\pgfpathlineto{\pgfqpoint{4.765583in}{3.379787in}}%
\pgfpathlineto{\pgfqpoint{4.767062in}{3.371133in}}%
\pgfpathlineto{\pgfqpoint{4.768542in}{3.371133in}}%
\pgfpathlineto{\pgfqpoint{4.770022in}{3.298220in}}%
\pgfpathlineto{\pgfqpoint{4.771008in}{3.379993in}}%
\pgfpathlineto{\pgfqpoint{4.771995in}{2.652133in}}%
\pgfpathlineto{\pgfqpoint{4.773474in}{3.216511in}}%
\pgfpathlineto{\pgfqpoint{4.773968in}{3.216511in}}%
\pgfpathlineto{\pgfqpoint{4.774954in}{2.916378in}}%
\pgfpathlineto{\pgfqpoint{4.776434in}{3.290774in}}%
\pgfpathlineto{\pgfqpoint{4.777913in}{3.290774in}}%
\pgfpathlineto{\pgfqpoint{4.778407in}{2.804472in}}%
\pgfpathlineto{\pgfqpoint{4.778900in}{3.354134in}}%
\pgfpathlineto{\pgfqpoint{4.779886in}{3.372732in}}%
\pgfpathlineto{\pgfqpoint{4.781366in}{3.339110in}}%
\pgfpathlineto{\pgfqpoint{4.782352in}{3.367030in}}%
\pgfpathlineto{\pgfqpoint{4.782846in}{2.883331in}}%
\pgfpathlineto{\pgfqpoint{4.783339in}{3.373281in}}%
\pgfpathlineto{\pgfqpoint{4.784819in}{3.314411in}}%
\pgfpathlineto{\pgfqpoint{4.785312in}{3.315948in}}%
\pgfpathlineto{\pgfqpoint{4.785805in}{3.083708in}}%
\pgfpathlineto{\pgfqpoint{4.786792in}{3.378330in}}%
\pgfpathlineto{\pgfqpoint{4.787778in}{3.188024in}}%
\pgfpathlineto{\pgfqpoint{4.790244in}{3.375997in}}%
\pgfpathlineto{\pgfqpoint{4.790737in}{2.691376in}}%
\pgfpathlineto{\pgfqpoint{4.791231in}{3.371133in}}%
\pgfpathlineto{\pgfqpoint{4.791724in}{3.371133in}}%
\pgfpathlineto{\pgfqpoint{4.793204in}{3.174616in}}%
\pgfpathlineto{\pgfqpoint{4.794683in}{3.174616in}}%
\pgfpathlineto{\pgfqpoint{4.795176in}{2.841670in}}%
\pgfpathlineto{\pgfqpoint{4.796656in}{3.359904in}}%
\pgfpathlineto{\pgfqpoint{4.797643in}{3.371970in}}%
\pgfpathlineto{\pgfqpoint{4.799122in}{3.039477in}}%
\pgfpathlineto{\pgfqpoint{4.800109in}{3.039477in}}%
\pgfpathlineto{\pgfqpoint{4.801095in}{2.955421in}}%
\pgfpathlineto{\pgfqpoint{4.801588in}{3.255413in}}%
\pgfpathlineto{\pgfqpoint{4.802575in}{3.254521in}}%
\pgfpathlineto{\pgfqpoint{4.803068in}{3.070213in}}%
\pgfpathlineto{\pgfqpoint{4.803561in}{3.261846in}}%
\pgfpathlineto{\pgfqpoint{4.804055in}{3.376767in}}%
\pgfpathlineto{\pgfqpoint{4.804548in}{3.256315in}}%
\pgfpathlineto{\pgfqpoint{4.805041in}{3.375599in}}%
\pgfpathlineto{\pgfqpoint{4.805534in}{3.135711in}}%
\pgfpathlineto{\pgfqpoint{4.806028in}{3.284103in}}%
\pgfpathlineto{\pgfqpoint{4.807014in}{3.324754in}}%
\pgfpathlineto{\pgfqpoint{4.808987in}{2.372378in}}%
\pgfpathlineto{\pgfqpoint{4.810467in}{3.203122in}}%
\pgfpathlineto{\pgfqpoint{4.811453in}{3.121676in}}%
\pgfpathlineto{\pgfqpoint{4.812933in}{3.346750in}}%
\pgfpathlineto{\pgfqpoint{4.813426in}{3.006962in}}%
\pgfpathlineto{\pgfqpoint{4.813919in}{3.317919in}}%
\pgfpathlineto{\pgfqpoint{4.814906in}{3.361689in}}%
\pgfpathlineto{\pgfqpoint{4.816879in}{2.971854in}}%
\pgfpathlineto{\pgfqpoint{4.818358in}{3.313454in}}%
\pgfpathlineto{\pgfqpoint{4.820824in}{3.313454in}}%
\pgfpathlineto{\pgfqpoint{4.821318in}{3.277359in}}%
\pgfpathlineto{\pgfqpoint{4.822304in}{3.126316in}}%
\pgfpathlineto{\pgfqpoint{4.822797in}{2.196897in}}%
\pgfpathlineto{\pgfqpoint{4.823291in}{2.602831in}}%
\pgfpathlineto{\pgfqpoint{4.824770in}{3.377700in}}%
\pgfpathlineto{\pgfqpoint{4.826250in}{3.137875in}}%
\pgfpathlineto{\pgfqpoint{4.827236in}{2.968180in}}%
\pgfpathlineto{\pgfqpoint{4.827730in}{3.329342in}}%
\pgfpathlineto{\pgfqpoint{4.828223in}{3.132327in}}%
\pgfpathlineto{\pgfqpoint{4.828716in}{3.157840in}}%
\pgfpathlineto{\pgfqpoint{4.830196in}{3.371680in}}%
\pgfpathlineto{\pgfqpoint{4.830689in}{3.290415in}}%
\pgfpathlineto{\pgfqpoint{4.831182in}{3.308629in}}%
\pgfpathlineto{\pgfqpoint{4.831676in}{2.388964in}}%
\pgfpathlineto{\pgfqpoint{4.832169in}{3.379078in}}%
\pgfpathlineto{\pgfqpoint{4.832662in}{3.379078in}}%
\pgfpathlineto{\pgfqpoint{4.833155in}{3.180934in}}%
\pgfpathlineto{\pgfqpoint{4.833648in}{3.368837in}}%
\pgfpathlineto{\pgfqpoint{4.834635in}{3.239463in}}%
\pgfpathlineto{\pgfqpoint{4.835128in}{3.309125in}}%
\pgfpathlineto{\pgfqpoint{4.835621in}{3.309125in}}%
\pgfpathlineto{\pgfqpoint{4.837101in}{3.358417in}}%
\pgfpathlineto{\pgfqpoint{4.838088in}{3.358417in}}%
\pgfpathlineto{\pgfqpoint{4.838581in}{2.530349in}}%
\pgfpathlineto{\pgfqpoint{4.839074in}{2.960853in}}%
\pgfpathlineto{\pgfqpoint{4.839567in}{2.960853in}}%
\pgfpathlineto{\pgfqpoint{4.840554in}{3.295460in}}%
\pgfpathlineto{\pgfqpoint{4.841047in}{3.277163in}}%
\pgfpathlineto{\pgfqpoint{4.841540in}{3.072722in}}%
\pgfpathlineto{\pgfqpoint{4.842033in}{3.245926in}}%
\pgfpathlineto{\pgfqpoint{4.842527in}{3.245926in}}%
\pgfpathlineto{\pgfqpoint{4.843513in}{3.376413in}}%
\pgfpathlineto{\pgfqpoint{4.844500in}{3.246865in}}%
\pgfpathlineto{\pgfqpoint{4.845979in}{3.357199in}}%
\pgfpathlineto{\pgfqpoint{4.846472in}{3.348301in}}%
\pgfpathlineto{\pgfqpoint{4.846966in}{3.355284in}}%
\pgfpathlineto{\pgfqpoint{4.848445in}{3.360857in}}%
\pgfpathlineto{\pgfqpoint{4.848939in}{3.360857in}}%
\pgfpathlineto{\pgfqpoint{4.849432in}{3.354192in}}%
\pgfpathlineto{\pgfqpoint{4.850418in}{3.355896in}}%
\pgfpathlineto{\pgfqpoint{4.850912in}{2.022287in}}%
\pgfpathlineto{\pgfqpoint{4.851405in}{3.289700in}}%
\pgfpathlineto{\pgfqpoint{4.852391in}{3.373058in}}%
\pgfpathlineto{\pgfqpoint{4.852884in}{3.071151in}}%
\pgfpathlineto{\pgfqpoint{4.853378in}{3.378821in}}%
\pgfpathlineto{\pgfqpoint{4.853871in}{3.304904in}}%
\pgfpathlineto{\pgfqpoint{4.854364in}{3.339230in}}%
\pgfpathlineto{\pgfqpoint{4.855844in}{3.344828in}}%
\pgfpathlineto{\pgfqpoint{4.856830in}{3.344828in}}%
\pgfpathlineto{\pgfqpoint{4.857324in}{3.337837in}}%
\pgfpathlineto{\pgfqpoint{4.857817in}{3.379926in}}%
\pgfpathlineto{\pgfqpoint{4.858310in}{3.341435in}}%
\pgfpathlineto{\pgfqpoint{4.858803in}{3.341435in}}%
\pgfpathlineto{\pgfqpoint{4.859790in}{3.319228in}}%
\pgfpathlineto{\pgfqpoint{4.860776in}{3.326711in}}%
\pgfpathlineto{\pgfqpoint{4.862256in}{3.086519in}}%
\pgfpathlineto{\pgfqpoint{4.863736in}{3.258041in}}%
\pgfpathlineto{\pgfqpoint{4.864229in}{3.258041in}}%
\pgfpathlineto{\pgfqpoint{4.864722in}{3.340140in}}%
\pgfpathlineto{\pgfqpoint{4.866202in}{3.024089in}}%
\pgfpathlineto{\pgfqpoint{4.866695in}{2.984569in}}%
\pgfpathlineto{\pgfqpoint{4.867681in}{3.345601in}}%
\pgfpathlineto{\pgfqpoint{4.868175in}{3.022216in}}%
\pgfpathlineto{\pgfqpoint{4.868668in}{3.253448in}}%
\pgfpathlineto{\pgfqpoint{4.870148in}{3.287715in}}%
\pgfpathlineto{\pgfqpoint{4.870641in}{3.287715in}}%
\pgfpathlineto{\pgfqpoint{4.871134in}{3.268344in}}%
\pgfpathlineto{\pgfqpoint{4.872121in}{3.145041in}}%
\pgfpathlineto{\pgfqpoint{4.873600in}{3.241847in}}%
\pgfpathlineto{\pgfqpoint{4.875080in}{3.319111in}}%
\pgfpathlineto{\pgfqpoint{4.875573in}{3.319111in}}%
\pgfpathlineto{\pgfqpoint{4.876066in}{3.300070in}}%
\pgfpathlineto{\pgfqpoint{4.876560in}{2.590557in}}%
\pgfpathlineto{\pgfqpoint{4.877053in}{2.886473in}}%
\pgfpathlineto{\pgfqpoint{4.878039in}{2.886473in}}%
\pgfpathlineto{\pgfqpoint{4.879026in}{3.273289in}}%
\pgfpathlineto{\pgfqpoint{4.879519in}{3.188786in}}%
\pgfpathlineto{\pgfqpoint{4.880999in}{3.188786in}}%
\pgfpathlineto{\pgfqpoint{4.882972in}{3.374907in}}%
\pgfpathlineto{\pgfqpoint{4.883465in}{3.047413in}}%
\pgfpathlineto{\pgfqpoint{4.883958in}{3.098718in}}%
\pgfpathlineto{\pgfqpoint{4.885438in}{3.292863in}}%
\pgfpathlineto{\pgfqpoint{4.887411in}{2.871833in}}%
\pgfpathlineto{\pgfqpoint{4.887904in}{3.378947in}}%
\pgfpathlineto{\pgfqpoint{4.888397in}{3.333591in}}%
\pgfpathlineto{\pgfqpoint{4.889877in}{2.419939in}}%
\pgfpathlineto{\pgfqpoint{4.891850in}{3.379988in}}%
\pgfpathlineto{\pgfqpoint{4.892343in}{3.282514in}}%
\pgfpathlineto{\pgfqpoint{4.892836in}{3.282514in}}%
\pgfpathlineto{\pgfqpoint{4.893329in}{3.316038in}}%
\pgfpathlineto{\pgfqpoint{4.894316in}{3.227326in}}%
\pgfpathlineto{\pgfqpoint{4.894809in}{3.357661in}}%
\pgfpathlineto{\pgfqpoint{4.895302in}{3.243236in}}%
\pgfpathlineto{\pgfqpoint{4.895796in}{3.243236in}}%
\pgfpathlineto{\pgfqpoint{4.897275in}{3.253517in}}%
\pgfpathlineto{\pgfqpoint{4.898755in}{3.253517in}}%
\pgfpathlineto{\pgfqpoint{4.899741in}{3.294737in}}%
\pgfpathlineto{\pgfqpoint{4.900235in}{3.276190in}}%
\pgfpathlineto{\pgfqpoint{4.900728in}{3.132111in}}%
\pgfpathlineto{\pgfqpoint{4.901221in}{2.083505in}}%
\pgfpathlineto{\pgfqpoint{4.901714in}{3.344127in}}%
\pgfpathlineto{\pgfqpoint{4.902208in}{3.379669in}}%
\pgfpathlineto{\pgfqpoint{4.902701in}{3.174481in}}%
\pgfpathlineto{\pgfqpoint{4.903194in}{3.230827in}}%
\pgfpathlineto{\pgfqpoint{4.903687in}{3.230827in}}%
\pgfpathlineto{\pgfqpoint{4.904674in}{3.079726in}}%
\pgfpathlineto{\pgfqpoint{4.906153in}{3.337978in}}%
\pgfpathlineto{\pgfqpoint{4.907140in}{2.092784in}}%
\pgfpathlineto{\pgfqpoint{4.907633in}{2.093497in}}%
\pgfpathlineto{\pgfqpoint{4.909113in}{3.379364in}}%
\pgfpathlineto{\pgfqpoint{4.910099in}{2.552014in}}%
\pgfpathlineto{\pgfqpoint{4.911579in}{3.379421in}}%
\pgfpathlineto{\pgfqpoint{4.912072in}{3.362352in}}%
\pgfpathlineto{\pgfqpoint{4.913059in}{2.844086in}}%
\pgfpathlineto{\pgfqpoint{4.914045in}{2.954014in}}%
\pgfpathlineto{\pgfqpoint{4.915032in}{3.336794in}}%
\pgfpathlineto{\pgfqpoint{4.916018in}{3.362579in}}%
\pgfpathlineto{\pgfqpoint{4.916511in}{3.080737in}}%
\pgfpathlineto{\pgfqpoint{4.917005in}{3.379632in}}%
\pgfpathlineto{\pgfqpoint{4.917991in}{3.322210in}}%
\pgfpathlineto{\pgfqpoint{4.918484in}{3.322210in}}%
\pgfpathlineto{\pgfqpoint{4.918977in}{3.328740in}}%
\pgfpathlineto{\pgfqpoint{4.919471in}{2.319805in}}%
\pgfpathlineto{\pgfqpoint{4.920457in}{2.449460in}}%
\pgfpathlineto{\pgfqpoint{4.920950in}{2.449460in}}%
\pgfpathlineto{\pgfqpoint{4.922430in}{3.339313in}}%
\pgfpathlineto{\pgfqpoint{4.922923in}{3.329239in}}%
\pgfpathlineto{\pgfqpoint{4.923417in}{3.329239in}}%
\pgfpathlineto{\pgfqpoint{4.924403in}{3.353749in}}%
\pgfpathlineto{\pgfqpoint{4.924896in}{3.229794in}}%
\pgfpathlineto{\pgfqpoint{4.925389in}{3.378681in}}%
\pgfpathlineto{\pgfqpoint{4.925883in}{3.378681in}}%
\pgfpathlineto{\pgfqpoint{4.926376in}{3.167755in}}%
\pgfpathlineto{\pgfqpoint{4.926869in}{3.306029in}}%
\pgfpathlineto{\pgfqpoint{4.928349in}{3.306029in}}%
\pgfpathlineto{\pgfqpoint{4.928842in}{3.277311in}}%
\pgfpathlineto{\pgfqpoint{4.929335in}{3.309416in}}%
\pgfpathlineto{\pgfqpoint{4.930815in}{3.379514in}}%
\pgfpathlineto{\pgfqpoint{4.932295in}{3.362795in}}%
\pgfpathlineto{\pgfqpoint{4.933774in}{3.369700in}}%
\pgfpathlineto{\pgfqpoint{4.934268in}{3.262585in}}%
\pgfpathlineto{\pgfqpoint{4.934761in}{2.088494in}}%
\pgfpathlineto{\pgfqpoint{4.935254in}{2.432850in}}%
\pgfpathlineto{\pgfqpoint{4.937227in}{3.363668in}}%
\pgfpathlineto{\pgfqpoint{4.937720in}{3.233577in}}%
\pgfpathlineto{\pgfqpoint{4.938707in}{3.233577in}}%
\pgfpathlineto{\pgfqpoint{4.939200in}{3.146345in}}%
\pgfpathlineto{\pgfqpoint{4.939693in}{3.366949in}}%
\pgfpathlineto{\pgfqpoint{4.940186in}{2.757378in}}%
\pgfpathlineto{\pgfqpoint{4.940680in}{3.144951in}}%
\pgfpathlineto{\pgfqpoint{4.941173in}{3.144951in}}%
\pgfpathlineto{\pgfqpoint{4.941666in}{2.693703in}}%
\pgfpathlineto{\pgfqpoint{4.942159in}{2.873509in}}%
\pgfpathlineto{\pgfqpoint{4.942653in}{2.873509in}}%
\pgfpathlineto{\pgfqpoint{4.943146in}{2.846146in}}%
\pgfpathlineto{\pgfqpoint{4.944625in}{2.532554in}}%
\pgfpathlineto{\pgfqpoint{4.945612in}{3.228050in}}%
\pgfpathlineto{\pgfqpoint{4.947092in}{2.893494in}}%
\pgfpathlineto{\pgfqpoint{4.948571in}{3.279294in}}%
\pgfpathlineto{\pgfqpoint{4.950051in}{3.266086in}}%
\pgfpathlineto{\pgfqpoint{4.950544in}{3.356978in}}%
\pgfpathlineto{\pgfqpoint{4.951531in}{2.866506in}}%
\pgfpathlineto{\pgfqpoint{4.952517in}{3.352767in}}%
\pgfpathlineto{\pgfqpoint{4.953010in}{3.318822in}}%
\pgfpathlineto{\pgfqpoint{4.953997in}{3.318822in}}%
\pgfpathlineto{\pgfqpoint{4.954490in}{2.991679in}}%
\pgfpathlineto{\pgfqpoint{4.954983in}{3.379979in}}%
\pgfpathlineto{\pgfqpoint{4.955477in}{3.375436in}}%
\pgfpathlineto{\pgfqpoint{4.955970in}{3.376653in}}%
\pgfpathlineto{\pgfqpoint{4.956463in}{3.376653in}}%
\pgfpathlineto{\pgfqpoint{4.956956in}{3.282859in}}%
\pgfpathlineto{\pgfqpoint{4.957449in}{3.377032in}}%
\pgfpathlineto{\pgfqpoint{4.959422in}{3.307874in}}%
\pgfpathlineto{\pgfqpoint{4.960902in}{3.371084in}}%
\pgfpathlineto{\pgfqpoint{4.961395in}{3.361870in}}%
\pgfpathlineto{\pgfqpoint{4.961889in}{3.367928in}}%
\pgfpathlineto{\pgfqpoint{4.962382in}{3.375409in}}%
\pgfpathlineto{\pgfqpoint{4.964848in}{3.092346in}}%
\pgfpathlineto{\pgfqpoint{4.966328in}{3.351042in}}%
\pgfpathlineto{\pgfqpoint{4.967314in}{3.351042in}}%
\pgfpathlineto{\pgfqpoint{4.968301in}{3.379902in}}%
\pgfpathlineto{\pgfqpoint{4.968794in}{3.071625in}}%
\pgfpathlineto{\pgfqpoint{4.969287in}{3.071625in}}%
\pgfpathlineto{\pgfqpoint{4.969780in}{3.370680in}}%
\pgfpathlineto{\pgfqpoint{4.970767in}{3.349412in}}%
\pgfpathlineto{\pgfqpoint{4.971260in}{3.305909in}}%
\pgfpathlineto{\pgfqpoint{4.972740in}{3.370709in}}%
\pgfpathlineto{\pgfqpoint{4.974219in}{3.370709in}}%
\pgfpathlineto{\pgfqpoint{4.975206in}{3.379243in}}%
\pgfpathlineto{\pgfqpoint{4.977179in}{2.557033in}}%
\pgfpathlineto{\pgfqpoint{4.979152in}{3.314568in}}%
\pgfpathlineto{\pgfqpoint{4.980138in}{2.966536in}}%
\pgfpathlineto{\pgfqpoint{4.982111in}{3.379633in}}%
\pgfpathlineto{\pgfqpoint{4.982604in}{3.232277in}}%
\pgfpathlineto{\pgfqpoint{4.983098in}{3.371169in}}%
\pgfpathlineto{\pgfqpoint{4.983591in}{3.371169in}}%
\pgfpathlineto{\pgfqpoint{4.984084in}{3.095869in}}%
\pgfpathlineto{\pgfqpoint{4.984577in}{3.369243in}}%
\pgfpathlineto{\pgfqpoint{4.986057in}{3.369243in}}%
\pgfpathlineto{\pgfqpoint{4.986550in}{3.374391in}}%
\pgfpathlineto{\pgfqpoint{4.988030in}{3.252004in}}%
\pgfpathlineto{\pgfqpoint{4.988523in}{2.103081in}}%
\pgfpathlineto{\pgfqpoint{4.989016in}{3.329113in}}%
\pgfpathlineto{\pgfqpoint{4.989510in}{2.563723in}}%
\pgfpathlineto{\pgfqpoint{4.990496in}{2.711088in}}%
\pgfpathlineto{\pgfqpoint{4.991482in}{3.277774in}}%
\pgfpathlineto{\pgfqpoint{4.991976in}{3.268391in}}%
\pgfpathlineto{\pgfqpoint{4.992469in}{3.268391in}}%
\pgfpathlineto{\pgfqpoint{4.993949in}{3.370858in}}%
\pgfpathlineto{\pgfqpoint{4.994442in}{2.827740in}}%
\pgfpathlineto{\pgfqpoint{4.994935in}{3.366715in}}%
\pgfpathlineto{\pgfqpoint{4.995428in}{3.366715in}}%
\pgfpathlineto{\pgfqpoint{4.995922in}{2.820653in}}%
\pgfpathlineto{\pgfqpoint{4.996415in}{3.361570in}}%
\pgfpathlineto{\pgfqpoint{4.996908in}{3.361570in}}%
\pgfpathlineto{\pgfqpoint{4.998388in}{3.373307in}}%
\pgfpathlineto{\pgfqpoint{4.999374in}{3.373307in}}%
\pgfpathlineto{\pgfqpoint{5.000854in}{3.366510in}}%
\pgfpathlineto{\pgfqpoint{5.001347in}{3.366510in}}%
\pgfpathlineto{\pgfqpoint{5.001840in}{3.369560in}}%
\pgfpathlineto{\pgfqpoint{5.003320in}{3.211206in}}%
\pgfpathlineto{\pgfqpoint{5.003813in}{3.211206in}}%
\pgfpathlineto{\pgfqpoint{5.004800in}{3.282412in}}%
\pgfpathlineto{\pgfqpoint{5.005293in}{2.967012in}}%
\pgfpathlineto{\pgfqpoint{5.005786in}{3.170312in}}%
\pgfpathlineto{\pgfqpoint{5.006773in}{3.170312in}}%
\pgfpathlineto{\pgfqpoint{5.007266in}{3.379457in}}%
\pgfpathlineto{\pgfqpoint{5.007759in}{3.326051in}}%
\pgfpathlineto{\pgfqpoint{5.008746in}{3.326051in}}%
\pgfpathlineto{\pgfqpoint{5.009239in}{3.357226in}}%
\pgfpathlineto{\pgfqpoint{5.011212in}{3.251884in}}%
\pgfpathlineto{\pgfqpoint{5.012691in}{3.368518in}}%
\pgfpathlineto{\pgfqpoint{5.013678in}{2.973297in}}%
\pgfpathlineto{\pgfqpoint{5.015158in}{3.343470in}}%
\pgfpathlineto{\pgfqpoint{5.015651in}{3.343470in}}%
\pgfpathlineto{\pgfqpoint{5.016637in}{3.379208in}}%
\pgfpathlineto{\pgfqpoint{5.017624in}{3.139475in}}%
\pgfpathlineto{\pgfqpoint{5.018117in}{3.374543in}}%
\pgfpathlineto{\pgfqpoint{5.018610in}{3.344918in}}%
\pgfpathlineto{\pgfqpoint{5.019103in}{3.295603in}}%
\pgfpathlineto{\pgfqpoint{5.019597in}{3.317690in}}%
\pgfpathlineto{\pgfqpoint{5.020583in}{3.317690in}}%
\pgfpathlineto{\pgfqpoint{5.021076in}{3.379976in}}%
\pgfpathlineto{\pgfqpoint{5.021570in}{3.333331in}}%
\pgfpathlineto{\pgfqpoint{5.022063in}{3.333331in}}%
\pgfpathlineto{\pgfqpoint{5.023049in}{3.213882in}}%
\pgfpathlineto{\pgfqpoint{5.023542in}{3.362336in}}%
\pgfpathlineto{\pgfqpoint{5.024036in}{3.052082in}}%
\pgfpathlineto{\pgfqpoint{5.024529in}{3.336172in}}%
\pgfpathlineto{\pgfqpoint{5.026502in}{3.336172in}}%
\pgfpathlineto{\pgfqpoint{5.026995in}{3.370582in}}%
\pgfpathlineto{\pgfqpoint{5.027982in}{3.278304in}}%
\pgfpathlineto{\pgfqpoint{5.028968in}{3.372612in}}%
\pgfpathlineto{\pgfqpoint{5.029954in}{2.960028in}}%
\pgfpathlineto{\pgfqpoint{5.030448in}{2.969017in}}%
\pgfpathlineto{\pgfqpoint{5.031434in}{2.969017in}}%
\pgfpathlineto{\pgfqpoint{5.031927in}{2.951397in}}%
\pgfpathlineto{\pgfqpoint{5.032914in}{3.379998in}}%
\pgfpathlineto{\pgfqpoint{5.033407in}{3.313390in}}%
\pgfpathlineto{\pgfqpoint{5.033900in}{3.313390in}}%
\pgfpathlineto{\pgfqpoint{5.034394in}{3.208728in}}%
\pgfpathlineto{\pgfqpoint{5.034887in}{3.336648in}}%
\pgfpathlineto{\pgfqpoint{5.035380in}{3.345424in}}%
\pgfpathlineto{\pgfqpoint{5.035873in}{2.803173in}}%
\pgfpathlineto{\pgfqpoint{5.036366in}{2.907186in}}%
\pgfpathlineto{\pgfqpoint{5.037846in}{3.160434in}}%
\pgfpathlineto{\pgfqpoint{5.039326in}{3.239411in}}%
\pgfpathlineto{\pgfqpoint{5.039819in}{3.239411in}}%
\pgfpathlineto{\pgfqpoint{5.040312in}{3.171581in}}%
\pgfpathlineto{\pgfqpoint{5.041299in}{2.503755in}}%
\pgfpathlineto{\pgfqpoint{5.041792in}{2.651349in}}%
\pgfpathlineto{\pgfqpoint{5.043272in}{3.358717in}}%
\pgfpathlineto{\pgfqpoint{5.044258in}{3.358717in}}%
\pgfpathlineto{\pgfqpoint{5.045245in}{2.904580in}}%
\pgfpathlineto{\pgfqpoint{5.046724in}{3.357205in}}%
\pgfpathlineto{\pgfqpoint{5.047218in}{3.094725in}}%
\pgfpathlineto{\pgfqpoint{5.047711in}{3.308846in}}%
\pgfpathlineto{\pgfqpoint{5.048697in}{3.344530in}}%
\pgfpathlineto{\pgfqpoint{5.050670in}{2.548193in}}%
\pgfpathlineto{\pgfqpoint{5.052150in}{3.365513in}}%
\pgfpathlineto{\pgfqpoint{5.052643in}{3.372791in}}%
\pgfpathlineto{\pgfqpoint{5.053136in}{3.332244in}}%
\pgfpathlineto{\pgfqpoint{5.054616in}{2.897746in}}%
\pgfpathlineto{\pgfqpoint{5.056096in}{3.363937in}}%
\pgfpathlineto{\pgfqpoint{5.056589in}{3.377428in}}%
\pgfpathlineto{\pgfqpoint{5.057082in}{3.373477in}}%
\pgfpathlineto{\pgfqpoint{5.059548in}{3.373477in}}%
\pgfpathlineto{\pgfqpoint{5.061028in}{3.344319in}}%
\pgfpathlineto{\pgfqpoint{5.062508in}{2.894406in}}%
\pgfpathlineto{\pgfqpoint{5.063001in}{2.817083in}}%
\pgfpathlineto{\pgfqpoint{5.064481in}{3.091234in}}%
\pgfpathlineto{\pgfqpoint{5.065467in}{2.736055in}}%
\pgfpathlineto{\pgfqpoint{5.066454in}{3.154115in}}%
\pgfpathlineto{\pgfqpoint{5.066947in}{2.664363in}}%
\pgfpathlineto{\pgfqpoint{5.068427in}{3.377451in}}%
\pgfpathlineto{\pgfqpoint{5.069906in}{3.377451in}}%
\pgfpathlineto{\pgfqpoint{5.072372in}{3.210467in}}%
\pgfpathlineto{\pgfqpoint{5.072866in}{3.210467in}}%
\pgfpathlineto{\pgfqpoint{5.074345in}{3.249469in}}%
\pgfpathlineto{\pgfqpoint{5.076811in}{2.971750in}}%
\pgfpathlineto{\pgfqpoint{5.078291in}{3.379986in}}%
\pgfpathlineto{\pgfqpoint{5.078784in}{3.359845in}}%
\pgfpathlineto{\pgfqpoint{5.080264in}{3.260167in}}%
\pgfpathlineto{\pgfqpoint{5.080757in}{2.496240in}}%
\pgfpathlineto{\pgfqpoint{5.081251in}{3.174001in}}%
\pgfpathlineto{\pgfqpoint{5.082730in}{3.366983in}}%
\pgfpathlineto{\pgfqpoint{5.084210in}{3.366983in}}%
\pgfpathlineto{\pgfqpoint{5.085690in}{3.315104in}}%
\pgfpathlineto{\pgfqpoint{5.086676in}{3.377798in}}%
\pgfpathlineto{\pgfqpoint{5.087169in}{2.830968in}}%
\pgfpathlineto{\pgfqpoint{5.087663in}{3.360402in}}%
\pgfpathlineto{\pgfqpoint{5.088156in}{3.061107in}}%
\pgfpathlineto{\pgfqpoint{5.088649in}{3.061107in}}%
\pgfpathlineto{\pgfqpoint{5.089142in}{3.354359in}}%
\pgfpathlineto{\pgfqpoint{5.089635in}{3.168473in}}%
\pgfpathlineto{\pgfqpoint{5.090622in}{2.839195in}}%
\pgfpathlineto{\pgfqpoint{5.091608in}{3.349377in}}%
\pgfpathlineto{\pgfqpoint{5.093088in}{2.708839in}}%
\pgfpathlineto{\pgfqpoint{5.094568in}{3.365039in}}%
\pgfpathlineto{\pgfqpoint{5.095061in}{3.365039in}}%
\pgfpathlineto{\pgfqpoint{5.095554in}{3.378234in}}%
\pgfpathlineto{\pgfqpoint{5.096047in}{3.285718in}}%
\pgfpathlineto{\pgfqpoint{5.096541in}{3.376725in}}%
\pgfpathlineto{\pgfqpoint{5.098020in}{3.376725in}}%
\pgfpathlineto{\pgfqpoint{5.099007in}{2.444488in}}%
\pgfpathlineto{\pgfqpoint{5.099500in}{3.378905in}}%
\pgfpathlineto{\pgfqpoint{5.100487in}{3.376224in}}%
\pgfpathlineto{\pgfqpoint{5.101473in}{3.246910in}}%
\pgfpathlineto{\pgfqpoint{5.102459in}{3.378527in}}%
\pgfpathlineto{\pgfqpoint{5.102953in}{3.377812in}}%
\pgfpathlineto{\pgfqpoint{5.103446in}{3.377812in}}%
\pgfpathlineto{\pgfqpoint{5.104432in}{3.379959in}}%
\pgfpathlineto{\pgfqpoint{5.104926in}{3.375603in}}%
\pgfpathlineto{\pgfqpoint{5.106405in}{3.305907in}}%
\pgfpathlineto{\pgfqpoint{5.106899in}{2.332866in}}%
\pgfpathlineto{\pgfqpoint{5.107392in}{3.278408in}}%
\pgfpathlineto{\pgfqpoint{5.107885in}{3.278408in}}%
\pgfpathlineto{\pgfqpoint{5.109365in}{3.362149in}}%
\pgfpathlineto{\pgfqpoint{5.110351in}{3.368966in}}%
\pgfpathlineto{\pgfqpoint{5.111831in}{3.314790in}}%
\pgfpathlineto{\pgfqpoint{5.113311in}{3.377559in}}%
\pgfpathlineto{\pgfqpoint{5.113804in}{3.280539in}}%
\pgfpathlineto{\pgfqpoint{5.114790in}{3.304860in}}%
\pgfpathlineto{\pgfqpoint{5.115283in}{2.490554in}}%
\pgfpathlineto{\pgfqpoint{5.115777in}{3.306293in}}%
\pgfpathlineto{\pgfqpoint{5.116270in}{3.363794in}}%
\pgfpathlineto{\pgfqpoint{5.116763in}{3.296756in}}%
\pgfpathlineto{\pgfqpoint{5.117256in}{3.301548in}}%
\pgfpathlineto{\pgfqpoint{5.119229in}{2.820690in}}%
\pgfpathlineto{\pgfqpoint{5.120709in}{2.998241in}}%
\pgfpathlineto{\pgfqpoint{5.121695in}{3.121217in}}%
\pgfpathlineto{\pgfqpoint{5.122189in}{2.995458in}}%
\pgfpathlineto{\pgfqpoint{5.122682in}{3.018730in}}%
\pgfpathlineto{\pgfqpoint{5.123175in}{3.081018in}}%
\pgfpathlineto{\pgfqpoint{5.124162in}{3.378721in}}%
\pgfpathlineto{\pgfqpoint{5.124655in}{3.349888in}}%
\pgfpathlineto{\pgfqpoint{5.125148in}{3.349888in}}%
\pgfpathlineto{\pgfqpoint{5.125641in}{3.262276in}}%
\pgfpathlineto{\pgfqpoint{5.126135in}{2.324776in}}%
\pgfpathlineto{\pgfqpoint{5.126628in}{3.273408in}}%
\pgfpathlineto{\pgfqpoint{5.127121in}{3.273408in}}%
\pgfpathlineto{\pgfqpoint{5.128601in}{3.315857in}}%
\pgfpathlineto{\pgfqpoint{5.129094in}{3.315857in}}%
\pgfpathlineto{\pgfqpoint{5.130080in}{3.370930in}}%
\pgfpathlineto{\pgfqpoint{5.131067in}{3.304338in}}%
\pgfpathlineto{\pgfqpoint{5.132053in}{3.379930in}}%
\pgfpathlineto{\pgfqpoint{5.132547in}{2.970101in}}%
\pgfpathlineto{\pgfqpoint{5.133533in}{3.065430in}}%
\pgfpathlineto{\pgfqpoint{5.134026in}{3.379961in}}%
\pgfpathlineto{\pgfqpoint{5.134519in}{3.360753in}}%
\pgfpathlineto{\pgfqpoint{5.135506in}{2.850372in}}%
\pgfpathlineto{\pgfqpoint{5.136492in}{3.290776in}}%
\pgfpathlineto{\pgfqpoint{5.137972in}{1.926091in}}%
\pgfpathlineto{\pgfqpoint{5.139452in}{3.296862in}}%
\pgfpathlineto{\pgfqpoint{5.139945in}{3.296862in}}%
\pgfpathlineto{\pgfqpoint{5.140438in}{3.244737in}}%
\pgfpathlineto{\pgfqpoint{5.140931in}{3.299318in}}%
\pgfpathlineto{\pgfqpoint{5.142411in}{3.299318in}}%
\pgfpathlineto{\pgfqpoint{5.143891in}{3.378729in}}%
\pgfpathlineto{\pgfqpoint{5.144384in}{3.378729in}}%
\pgfpathlineto{\pgfqpoint{5.145864in}{3.293080in}}%
\pgfpathlineto{\pgfqpoint{5.146357in}{3.293080in}}%
\pgfpathlineto{\pgfqpoint{5.147343in}{3.369798in}}%
\pgfpathlineto{\pgfqpoint{5.147837in}{3.309271in}}%
\pgfpathlineto{\pgfqpoint{5.148330in}{3.360308in}}%
\pgfpathlineto{\pgfqpoint{5.149810in}{3.374467in}}%
\pgfpathlineto{\pgfqpoint{5.150796in}{3.374467in}}%
\pgfpathlineto{\pgfqpoint{5.152769in}{2.772129in}}%
\pgfpathlineto{\pgfqpoint{5.154249in}{2.772129in}}%
\pgfpathlineto{\pgfqpoint{5.155235in}{2.095573in}}%
\pgfpathlineto{\pgfqpoint{5.156222in}{3.244479in}}%
\pgfpathlineto{\pgfqpoint{5.156715in}{3.111988in}}%
\pgfpathlineto{\pgfqpoint{5.157208in}{3.111988in}}%
\pgfpathlineto{\pgfqpoint{5.159674in}{3.277764in}}%
\pgfpathlineto{\pgfqpoint{5.160661in}{3.277764in}}%
\pgfpathlineto{\pgfqpoint{5.161154in}{2.779472in}}%
\pgfpathlineto{\pgfqpoint{5.161647in}{3.160528in}}%
\pgfpathlineto{\pgfqpoint{5.163127in}{3.160528in}}%
\pgfpathlineto{\pgfqpoint{5.164113in}{3.341173in}}%
\pgfpathlineto{\pgfqpoint{5.164607in}{3.299943in}}%
\pgfpathlineto{\pgfqpoint{5.166086in}{3.172619in}}%
\pgfpathlineto{\pgfqpoint{5.166579in}{3.172619in}}%
\pgfpathlineto{\pgfqpoint{5.167566in}{2.879909in}}%
\pgfpathlineto{\pgfqpoint{5.169046in}{3.195345in}}%
\pgfpathlineto{\pgfqpoint{5.169539in}{3.195345in}}%
\pgfpathlineto{\pgfqpoint{5.170032in}{2.404971in}}%
\pgfpathlineto{\pgfqpoint{5.170525in}{3.367825in}}%
\pgfpathlineto{\pgfqpoint{5.171512in}{3.379949in}}%
\pgfpathlineto{\pgfqpoint{5.172005in}{2.874605in}}%
\pgfpathlineto{\pgfqpoint{5.172498in}{3.376437in}}%
\pgfpathlineto{\pgfqpoint{5.173485in}{3.334137in}}%
\pgfpathlineto{\pgfqpoint{5.173978in}{3.367232in}}%
\pgfpathlineto{\pgfqpoint{5.175458in}{3.256405in}}%
\pgfpathlineto{\pgfqpoint{5.175951in}{3.343831in}}%
\pgfpathlineto{\pgfqpoint{5.176444in}{3.134706in}}%
\pgfpathlineto{\pgfqpoint{5.176937in}{3.208853in}}%
\pgfpathlineto{\pgfqpoint{5.177431in}{3.208853in}}%
\pgfpathlineto{\pgfqpoint{5.177924in}{2.748889in}}%
\pgfpathlineto{\pgfqpoint{5.178417in}{3.321653in}}%
\pgfpathlineto{\pgfqpoint{5.180390in}{3.321653in}}%
\pgfpathlineto{\pgfqpoint{5.180883in}{3.372558in}}%
\pgfpathlineto{\pgfqpoint{5.181376in}{2.720221in}}%
\pgfpathlineto{\pgfqpoint{5.181870in}{3.366485in}}%
\pgfpathlineto{\pgfqpoint{5.183349in}{3.370757in}}%
\pgfpathlineto{\pgfqpoint{5.184829in}{3.359772in}}%
\pgfpathlineto{\pgfqpoint{5.185322in}{3.366021in}}%
\pgfpathlineto{\pgfqpoint{5.185816in}{3.351182in}}%
\pgfpathlineto{\pgfqpoint{5.186309in}{3.357138in}}%
\pgfpathlineto{\pgfqpoint{5.186802in}{3.357138in}}%
\pgfpathlineto{\pgfqpoint{5.188282in}{3.371517in}}%
\pgfpathlineto{\pgfqpoint{5.189268in}{2.691824in}}%
\pgfpathlineto{\pgfqpoint{5.190748in}{3.336247in}}%
\pgfpathlineto{\pgfqpoint{5.191241in}{3.336247in}}%
\pgfpathlineto{\pgfqpoint{5.192228in}{3.378625in}}%
\pgfpathlineto{\pgfqpoint{5.193214in}{2.846016in}}%
\pgfpathlineto{\pgfqpoint{5.193707in}{3.379192in}}%
\pgfpathlineto{\pgfqpoint{5.194694in}{3.317792in}}%
\pgfpathlineto{\pgfqpoint{5.195187in}{3.317792in}}%
\pgfpathlineto{\pgfqpoint{5.195680in}{3.378391in}}%
\pgfpathlineto{\pgfqpoint{5.196667in}{3.133513in}}%
\pgfpathlineto{\pgfqpoint{5.198146in}{3.292110in}}%
\pgfpathlineto{\pgfqpoint{5.198640in}{3.337862in}}%
\pgfpathlineto{\pgfqpoint{5.199626in}{2.969612in}}%
\pgfpathlineto{\pgfqpoint{5.201599in}{3.379016in}}%
\pgfpathlineto{\pgfqpoint{5.202585in}{3.379016in}}%
\pgfpathlineto{\pgfqpoint{5.203079in}{3.093376in}}%
\pgfpathlineto{\pgfqpoint{5.203572in}{3.233948in}}%
\pgfpathlineto{\pgfqpoint{5.204065in}{3.246559in}}%
\pgfpathlineto{\pgfqpoint{5.204558in}{3.361513in}}%
\pgfpathlineto{\pgfqpoint{5.205052in}{2.774053in}}%
\pgfpathlineto{\pgfqpoint{5.205545in}{3.125202in}}%
\pgfpathlineto{\pgfqpoint{5.207024in}{3.375293in}}%
\pgfpathlineto{\pgfqpoint{5.208997in}{2.804054in}}%
\pgfpathlineto{\pgfqpoint{5.209984in}{3.356645in}}%
\pgfpathlineto{\pgfqpoint{5.210477in}{3.230076in}}%
\pgfpathlineto{\pgfqpoint{5.210970in}{3.230076in}}%
\pgfpathlineto{\pgfqpoint{5.211464in}{3.374424in}}%
\pgfpathlineto{\pgfqpoint{5.212450in}{3.362030in}}%
\pgfpathlineto{\pgfqpoint{5.212943in}{3.254351in}}%
\pgfpathlineto{\pgfqpoint{5.213436in}{2.892678in}}%
\pgfpathlineto{\pgfqpoint{5.213930in}{3.339775in}}%
\pgfpathlineto{\pgfqpoint{5.214423in}{3.339775in}}%
\pgfpathlineto{\pgfqpoint{5.214916in}{2.801657in}}%
\pgfpathlineto{\pgfqpoint{5.215409in}{3.366430in}}%
\pgfpathlineto{\pgfqpoint{5.216396in}{3.366430in}}%
\pgfpathlineto{\pgfqpoint{5.216889in}{3.369004in}}%
\pgfpathlineto{\pgfqpoint{5.218369in}{2.965267in}}%
\pgfpathlineto{\pgfqpoint{5.219848in}{3.358138in}}%
\pgfpathlineto{\pgfqpoint{5.220342in}{3.379849in}}%
\pgfpathlineto{\pgfqpoint{5.221328in}{3.118866in}}%
\pgfpathlineto{\pgfqpoint{5.222808in}{3.312916in}}%
\pgfpathlineto{\pgfqpoint{5.224288in}{3.172419in}}%
\pgfpathlineto{\pgfqpoint{5.225767in}{3.308672in}}%
\pgfpathlineto{\pgfqpoint{5.226754in}{2.768610in}}%
\pgfpathlineto{\pgfqpoint{5.228233in}{3.344278in}}%
\pgfpathlineto{\pgfqpoint{5.228727in}{3.344278in}}%
\pgfpathlineto{\pgfqpoint{5.229220in}{3.332763in}}%
\pgfpathlineto{\pgfqpoint{5.230700in}{2.936624in}}%
\pgfpathlineto{\pgfqpoint{5.231686in}{3.379976in}}%
\pgfpathlineto{\pgfqpoint{5.232179in}{3.377167in}}%
\pgfpathlineto{\pgfqpoint{5.233659in}{2.753067in}}%
\pgfpathlineto{\pgfqpoint{5.234645in}{3.379236in}}%
\pgfpathlineto{\pgfqpoint{5.235139in}{3.349793in}}%
\pgfpathlineto{\pgfqpoint{5.235632in}{3.349793in}}%
\pgfpathlineto{\pgfqpoint{5.236125in}{3.305976in}}%
\pgfpathlineto{\pgfqpoint{5.237112in}{3.364975in}}%
\pgfpathlineto{\pgfqpoint{5.238098in}{3.074272in}}%
\pgfpathlineto{\pgfqpoint{5.238591in}{3.379856in}}%
\pgfpathlineto{\pgfqpoint{5.239084in}{2.999990in}}%
\pgfpathlineto{\pgfqpoint{5.239578in}{2.680447in}}%
\pgfpathlineto{\pgfqpoint{5.241057in}{3.306598in}}%
\pgfpathlineto{\pgfqpoint{5.242537in}{3.379524in}}%
\pgfpathlineto{\pgfqpoint{5.243030in}{2.901903in}}%
\pgfpathlineto{\pgfqpoint{5.243524in}{3.290343in}}%
\pgfpathlineto{\pgfqpoint{5.244017in}{3.290343in}}%
\pgfpathlineto{\pgfqpoint{5.245496in}{3.380000in}}%
\pgfpathlineto{\pgfqpoint{5.246483in}{3.380000in}}%
\pgfpathlineto{\pgfqpoint{5.248456in}{3.285822in}}%
\pgfpathlineto{\pgfqpoint{5.248949in}{3.101607in}}%
\pgfpathlineto{\pgfqpoint{5.250429in}{3.376947in}}%
\pgfpathlineto{\pgfqpoint{5.250922in}{3.376947in}}%
\pgfpathlineto{\pgfqpoint{5.251415in}{3.115074in}}%
\pgfpathlineto{\pgfqpoint{5.251908in}{3.255094in}}%
\pgfpathlineto{\pgfqpoint{5.253388in}{3.280631in}}%
\pgfpathlineto{\pgfqpoint{5.253881in}{3.280631in}}%
\pgfpathlineto{\pgfqpoint{5.254375in}{2.031030in}}%
\pgfpathlineto{\pgfqpoint{5.254868in}{2.381168in}}%
\pgfpathlineto{\pgfqpoint{5.255361in}{2.381168in}}%
\pgfpathlineto{\pgfqpoint{5.256348in}{3.291420in}}%
\pgfpathlineto{\pgfqpoint{5.256841in}{3.217889in}}%
\pgfpathlineto{\pgfqpoint{5.257334in}{3.276910in}}%
\pgfpathlineto{\pgfqpoint{5.257827in}{3.242168in}}%
\pgfpathlineto{\pgfqpoint{5.258814in}{3.207360in}}%
\pgfpathlineto{\pgfqpoint{5.260293in}{3.363084in}}%
\pgfpathlineto{\pgfqpoint{5.261773in}{3.363084in}}%
\pgfpathlineto{\pgfqpoint{5.262266in}{3.096790in}}%
\pgfpathlineto{\pgfqpoint{5.262760in}{3.376039in}}%
\pgfpathlineto{\pgfqpoint{5.263746in}{3.251156in}}%
\pgfpathlineto{\pgfqpoint{5.264239in}{3.343002in}}%
\pgfpathlineto{\pgfqpoint{5.265226in}{3.334704in}}%
\pgfpathlineto{\pgfqpoint{5.265719in}{3.334704in}}%
\pgfpathlineto{\pgfqpoint{5.267199in}{3.208649in}}%
\pgfpathlineto{\pgfqpoint{5.267692in}{3.375948in}}%
\pgfpathlineto{\pgfqpoint{5.268185in}{3.262181in}}%
\pgfpathlineto{\pgfqpoint{5.269172in}{3.379823in}}%
\pgfpathlineto{\pgfqpoint{5.270158in}{3.228233in}}%
\pgfpathlineto{\pgfqpoint{5.271638in}{3.379119in}}%
\pgfpathlineto{\pgfqpoint{5.273117in}{3.370219in}}%
\pgfpathlineto{\pgfqpoint{5.273611in}{3.370219in}}%
\pgfpathlineto{\pgfqpoint{5.275090in}{3.374807in}}%
\pgfpathlineto{\pgfqpoint{5.276570in}{3.226397in}}%
\pgfpathlineto{\pgfqpoint{5.278050in}{3.226397in}}%
\pgfpathlineto{\pgfqpoint{5.279529in}{3.336855in}}%
\pgfpathlineto{\pgfqpoint{5.281009in}{3.336855in}}%
\pgfpathlineto{\pgfqpoint{5.281996in}{3.353745in}}%
\pgfpathlineto{\pgfqpoint{5.282489in}{3.342944in}}%
\pgfpathlineto{\pgfqpoint{5.282982in}{3.352659in}}%
\pgfpathlineto{\pgfqpoint{5.284462in}{2.969930in}}%
\pgfpathlineto{\pgfqpoint{5.285941in}{3.303646in}}%
\pgfpathlineto{\pgfqpoint{5.286435in}{3.303646in}}%
\pgfpathlineto{\pgfqpoint{5.286928in}{3.270369in}}%
\pgfpathlineto{\pgfqpoint{5.288901in}{3.353583in}}%
\pgfpathlineto{\pgfqpoint{5.289394in}{3.353583in}}%
\pgfpathlineto{\pgfqpoint{5.289887in}{3.370219in}}%
\pgfpathlineto{\pgfqpoint{5.290381in}{3.353633in}}%
\pgfpathlineto{\pgfqpoint{5.290874in}{3.353633in}}%
\pgfpathlineto{\pgfqpoint{5.291367in}{3.283235in}}%
\pgfpathlineto{\pgfqpoint{5.291860in}{3.356373in}}%
\pgfpathlineto{\pgfqpoint{5.292847in}{2.337519in}}%
\pgfpathlineto{\pgfqpoint{5.293833in}{3.310327in}}%
\pgfpathlineto{\pgfqpoint{5.295313in}{2.850233in}}%
\pgfpathlineto{\pgfqpoint{5.296793in}{3.378996in}}%
\pgfpathlineto{\pgfqpoint{5.297286in}{3.275249in}}%
\pgfpathlineto{\pgfqpoint{5.297779in}{3.234264in}}%
\pgfpathlineto{\pgfqpoint{5.299259in}{3.379106in}}%
\pgfpathlineto{\pgfqpoint{5.299752in}{3.323777in}}%
\pgfpathlineto{\pgfqpoint{5.301232in}{3.037866in}}%
\pgfpathlineto{\pgfqpoint{5.301725in}{3.027567in}}%
\pgfpathlineto{\pgfqpoint{5.303698in}{3.328886in}}%
\pgfpathlineto{\pgfqpoint{5.304191in}{3.328886in}}%
\pgfpathlineto{\pgfqpoint{5.305177in}{3.353436in}}%
\pgfpathlineto{\pgfqpoint{5.306657in}{3.316196in}}%
\pgfpathlineto{\pgfqpoint{5.307150in}{3.316196in}}%
\pgfpathlineto{\pgfqpoint{5.307644in}{3.355004in}}%
\pgfpathlineto{\pgfqpoint{5.309123in}{2.902405in}}%
\pgfpathlineto{\pgfqpoint{5.310603in}{3.329015in}}%
\pgfpathlineto{\pgfqpoint{5.311096in}{3.379077in}}%
\pgfpathlineto{\pgfqpoint{5.312576in}{2.493094in}}%
\pgfpathlineto{\pgfqpoint{5.314056in}{3.319993in}}%
\pgfpathlineto{\pgfqpoint{5.314549in}{3.319993in}}%
\pgfpathlineto{\pgfqpoint{5.315042in}{3.379287in}}%
\pgfpathlineto{\pgfqpoint{5.316522in}{3.248410in}}%
\pgfpathlineto{\pgfqpoint{5.318495in}{3.375139in}}%
\pgfpathlineto{\pgfqpoint{5.319481in}{3.375139in}}%
\pgfpathlineto{\pgfqpoint{5.319974in}{3.350740in}}%
\pgfpathlineto{\pgfqpoint{5.320961in}{2.960904in}}%
\pgfpathlineto{\pgfqpoint{5.321454in}{3.077587in}}%
\pgfpathlineto{\pgfqpoint{5.322934in}{3.373511in}}%
\pgfpathlineto{\pgfqpoint{5.324413in}{3.373511in}}%
\pgfpathlineto{\pgfqpoint{5.325400in}{3.379998in}}%
\pgfpathlineto{\pgfqpoint{5.326386in}{3.368213in}}%
\pgfpathlineto{\pgfqpoint{5.326880in}{3.370666in}}%
\pgfpathlineto{\pgfqpoint{5.327866in}{3.241600in}}%
\pgfpathlineto{\pgfqpoint{5.328853in}{3.370749in}}%
\pgfpathlineto{\pgfqpoint{5.331319in}{2.909522in}}%
\pgfpathlineto{\pgfqpoint{5.332305in}{3.359342in}}%
\pgfpathlineto{\pgfqpoint{5.332798in}{3.246353in}}%
\pgfpathlineto{\pgfqpoint{5.333785in}{3.246353in}}%
\pgfpathlineto{\pgfqpoint{5.334278in}{3.230792in}}%
\pgfpathlineto{\pgfqpoint{5.335265in}{3.137527in}}%
\pgfpathlineto{\pgfqpoint{5.336744in}{3.329849in}}%
\pgfpathlineto{\pgfqpoint{5.337731in}{3.329849in}}%
\pgfpathlineto{\pgfqpoint{5.338717in}{3.103547in}}%
\pgfpathlineto{\pgfqpoint{5.340197in}{3.376852in}}%
\pgfpathlineto{\pgfqpoint{5.343156in}{3.376852in}}%
\pgfpathlineto{\pgfqpoint{5.344636in}{3.361818in}}%
\pgfpathlineto{\pgfqpoint{5.345622in}{3.361818in}}%
\pgfpathlineto{\pgfqpoint{5.346116in}{3.375832in}}%
\pgfpathlineto{\pgfqpoint{5.346609in}{3.369473in}}%
\pgfpathlineto{\pgfqpoint{5.347102in}{3.369473in}}%
\pgfpathlineto{\pgfqpoint{5.347595in}{3.271084in}}%
\pgfpathlineto{\pgfqpoint{5.348089in}{3.366532in}}%
\pgfpathlineto{\pgfqpoint{5.348582in}{3.366532in}}%
\pgfpathlineto{\pgfqpoint{5.349075in}{3.204573in}}%
\pgfpathlineto{\pgfqpoint{5.349568in}{3.343408in}}%
\pgfpathlineto{\pgfqpoint{5.350555in}{3.343408in}}%
\pgfpathlineto{\pgfqpoint{5.352034in}{3.379229in}}%
\pgfpathlineto{\pgfqpoint{5.353514in}{3.379229in}}%
\pgfpathlineto{\pgfqpoint{5.354007in}{3.376900in}}%
\pgfpathlineto{\pgfqpoint{5.355487in}{3.316838in}}%
\pgfpathlineto{\pgfqpoint{5.356967in}{3.288611in}}%
\pgfpathlineto{\pgfqpoint{5.358446in}{3.365517in}}%
\pgfpathlineto{\pgfqpoint{5.359433in}{3.282080in}}%
\pgfpathlineto{\pgfqpoint{5.359926in}{2.926911in}}%
\pgfpathlineto{\pgfqpoint{5.360419in}{3.232140in}}%
\pgfpathlineto{\pgfqpoint{5.361899in}{3.315695in}}%
\pgfpathlineto{\pgfqpoint{5.362392in}{3.315695in}}%
\pgfpathlineto{\pgfqpoint{5.362885in}{2.702145in}}%
\pgfpathlineto{\pgfqpoint{5.363379in}{3.003711in}}%
\pgfpathlineto{\pgfqpoint{5.364858in}{3.369938in}}%
\pgfpathlineto{\pgfqpoint{5.366338in}{2.992552in}}%
\pgfpathlineto{\pgfqpoint{5.367818in}{3.245424in}}%
\pgfpathlineto{\pgfqpoint{5.368311in}{3.378777in}}%
\pgfpathlineto{\pgfqpoint{5.368804in}{3.244191in}}%
\pgfpathlineto{\pgfqpoint{5.370284in}{3.244191in}}%
\pgfpathlineto{\pgfqpoint{5.370777in}{2.076575in}}%
\pgfpathlineto{\pgfqpoint{5.371270in}{3.315953in}}%
\pgfpathlineto{\pgfqpoint{5.371764in}{3.319118in}}%
\pgfpathlineto{\pgfqpoint{5.372750in}{3.372650in}}%
\pgfpathlineto{\pgfqpoint{5.375710in}{3.219164in}}%
\pgfpathlineto{\pgfqpoint{5.376203in}{3.219164in}}%
\pgfpathlineto{\pgfqpoint{5.376696in}{2.246003in}}%
\pgfpathlineto{\pgfqpoint{5.377189in}{2.399418in}}%
\pgfpathlineto{\pgfqpoint{5.378669in}{3.371101in}}%
\pgfpathlineto{\pgfqpoint{5.381628in}{3.371101in}}%
\pgfpathlineto{\pgfqpoint{5.382615in}{3.251090in}}%
\pgfpathlineto{\pgfqpoint{5.383108in}{3.275863in}}%
\pgfpathlineto{\pgfqpoint{5.384094in}{3.379385in}}%
\pgfpathlineto{\pgfqpoint{5.384588in}{3.322199in}}%
\pgfpathlineto{\pgfqpoint{5.385081in}{3.370446in}}%
\pgfpathlineto{\pgfqpoint{5.386067in}{3.370446in}}%
\pgfpathlineto{\pgfqpoint{5.386561in}{2.639333in}}%
\pgfpathlineto{\pgfqpoint{5.387054in}{3.295378in}}%
\pgfpathlineto{\pgfqpoint{5.388040in}{3.295378in}}%
\pgfpathlineto{\pgfqpoint{5.388534in}{3.363161in}}%
\pgfpathlineto{\pgfqpoint{5.389027in}{2.951742in}}%
\pgfpathlineto{\pgfqpoint{5.389520in}{3.043621in}}%
\pgfpathlineto{\pgfqpoint{5.390506in}{3.347873in}}%
\pgfpathlineto{\pgfqpoint{5.391986in}{3.169966in}}%
\pgfpathlineto{\pgfqpoint{5.392973in}{3.178224in}}%
\pgfpathlineto{\pgfqpoint{5.393959in}{3.344617in}}%
\pgfpathlineto{\pgfqpoint{5.394452in}{3.344192in}}%
\pgfpathlineto{\pgfqpoint{5.394946in}{3.344192in}}%
\pgfpathlineto{\pgfqpoint{5.395439in}{2.893305in}}%
\pgfpathlineto{\pgfqpoint{5.395932in}{3.112719in}}%
\pgfpathlineto{\pgfqpoint{5.396918in}{3.343419in}}%
\pgfpathlineto{\pgfqpoint{5.397905in}{3.338934in}}%
\pgfpathlineto{\pgfqpoint{5.399878in}{3.338934in}}%
\pgfpathlineto{\pgfqpoint{5.400371in}{3.343721in}}%
\pgfpathlineto{\pgfqpoint{5.400864in}{2.942377in}}%
\pgfpathlineto{\pgfqpoint{5.401358in}{3.275787in}}%
\pgfpathlineto{\pgfqpoint{5.402837in}{3.379167in}}%
\pgfpathlineto{\pgfqpoint{5.404317in}{3.135946in}}%
\pgfpathlineto{\pgfqpoint{5.403824in}{3.379751in}}%
\pgfpathlineto{\pgfqpoint{5.404810in}{3.185985in}}%
\pgfpathlineto{\pgfqpoint{5.406290in}{3.295367in}}%
\pgfpathlineto{\pgfqpoint{5.406783in}{3.199118in}}%
\pgfpathlineto{\pgfqpoint{5.407276in}{3.225672in}}%
\pgfpathlineto{\pgfqpoint{5.408756in}{3.375495in}}%
\pgfpathlineto{\pgfqpoint{5.411715in}{2.997033in}}%
\pgfpathlineto{\pgfqpoint{5.413195in}{3.296718in}}%
\pgfpathlineto{\pgfqpoint{5.413688in}{3.304517in}}%
\pgfpathlineto{\pgfqpoint{5.414182in}{3.347833in}}%
\pgfpathlineto{\pgfqpoint{5.414675in}{3.310149in}}%
\pgfpathlineto{\pgfqpoint{5.415168in}{3.281418in}}%
\pgfpathlineto{\pgfqpoint{5.416648in}{3.379777in}}%
\pgfpathlineto{\pgfqpoint{5.417141in}{3.379777in}}%
\pgfpathlineto{\pgfqpoint{5.418621in}{3.007225in}}%
\pgfpathlineto{\pgfqpoint{5.420100in}{3.372156in}}%
\pgfpathlineto{\pgfqpoint{5.420594in}{3.372156in}}%
\pgfpathlineto{\pgfqpoint{5.421087in}{3.235688in}}%
\pgfpathlineto{\pgfqpoint{5.421580in}{3.366929in}}%
\pgfpathlineto{\pgfqpoint{5.422566in}{3.341562in}}%
\pgfpathlineto{\pgfqpoint{5.424046in}{3.021675in}}%
\pgfpathlineto{\pgfqpoint{5.425033in}{3.021675in}}%
\pgfpathlineto{\pgfqpoint{5.426019in}{2.692065in}}%
\pgfpathlineto{\pgfqpoint{5.427006in}{3.361511in}}%
\pgfpathlineto{\pgfqpoint{5.427499in}{3.355369in}}%
\pgfpathlineto{\pgfqpoint{5.429472in}{3.354715in}}%
\pgfpathlineto{\pgfqpoint{5.430951in}{3.379763in}}%
\pgfpathlineto{\pgfqpoint{5.431938in}{3.197386in}}%
\pgfpathlineto{\pgfqpoint{5.433418in}{3.362118in}}%
\pgfpathlineto{\pgfqpoint{5.433911in}{3.014315in}}%
\pgfpathlineto{\pgfqpoint{5.434404in}{3.118524in}}%
\pgfpathlineto{\pgfqpoint{5.436377in}{3.370528in}}%
\pgfpathlineto{\pgfqpoint{5.437857in}{3.296771in}}%
\pgfpathlineto{\pgfqpoint{5.439336in}{3.356330in}}%
\pgfpathlineto{\pgfqpoint{5.439830in}{3.356330in}}%
\pgfpathlineto{\pgfqpoint{5.440323in}{3.323003in}}%
\pgfpathlineto{\pgfqpoint{5.440816in}{3.376318in}}%
\pgfpathlineto{\pgfqpoint{5.441309in}{3.178337in}}%
\pgfpathlineto{\pgfqpoint{5.441802in}{3.347828in}}%
\pgfpathlineto{\pgfqpoint{5.442296in}{3.347828in}}%
\pgfpathlineto{\pgfqpoint{5.442789in}{3.293480in}}%
\pgfpathlineto{\pgfqpoint{5.443282in}{3.357130in}}%
\pgfpathlineto{\pgfqpoint{5.444269in}{3.357130in}}%
\pgfpathlineto{\pgfqpoint{5.444762in}{3.160764in}}%
\pgfpathlineto{\pgfqpoint{5.445255in}{3.374927in}}%
\pgfpathlineto{\pgfqpoint{5.446242in}{3.182494in}}%
\pgfpathlineto{\pgfqpoint{5.447228in}{3.376628in}}%
\pgfpathlineto{\pgfqpoint{5.447721in}{3.309528in}}%
\pgfpathlineto{\pgfqpoint{5.448708in}{3.373639in}}%
\pgfpathlineto{\pgfqpoint{5.449201in}{3.372868in}}%
\pgfpathlineto{\pgfqpoint{5.449694in}{3.368389in}}%
\pgfpathlineto{\pgfqpoint{5.451174in}{3.377517in}}%
\pgfpathlineto{\pgfqpoint{5.451667in}{3.377517in}}%
\pgfpathlineto{\pgfqpoint{5.452654in}{3.247680in}}%
\pgfpathlineto{\pgfqpoint{5.453147in}{3.350774in}}%
\pgfpathlineto{\pgfqpoint{5.453640in}{3.287137in}}%
\pgfpathlineto{\pgfqpoint{5.454133in}{3.303415in}}%
\pgfpathlineto{\pgfqpoint{5.454626in}{3.179239in}}%
\pgfpathlineto{\pgfqpoint{5.455613in}{3.367424in}}%
\pgfpathlineto{\pgfqpoint{5.456599in}{2.881879in}}%
\pgfpathlineto{\pgfqpoint{5.458079in}{3.288872in}}%
\pgfpathlineto{\pgfqpoint{5.459559in}{3.337108in}}%
\pgfpathlineto{\pgfqpoint{5.460545in}{2.997723in}}%
\pgfpathlineto{\pgfqpoint{5.461038in}{3.012467in}}%
\pgfpathlineto{\pgfqpoint{5.462518in}{3.378849in}}%
\pgfpathlineto{\pgfqpoint{5.463505in}{3.367528in}}%
\pgfpathlineto{\pgfqpoint{5.463998in}{3.159714in}}%
\pgfpathlineto{\pgfqpoint{5.464491in}{3.251035in}}%
\pgfpathlineto{\pgfqpoint{5.465971in}{3.347152in}}%
\pgfpathlineto{\pgfqpoint{5.467450in}{3.234999in}}%
\pgfpathlineto{\pgfqpoint{5.469917in}{3.160167in}}%
\pgfpathlineto{\pgfqpoint{5.471396in}{3.376373in}}%
\pgfpathlineto{\pgfqpoint{5.471890in}{3.376373in}}%
\pgfpathlineto{\pgfqpoint{5.473369in}{3.370083in}}%
\pgfpathlineto{\pgfqpoint{5.474356in}{3.370083in}}%
\pgfpathlineto{\pgfqpoint{5.475342in}{3.297316in}}%
\pgfpathlineto{\pgfqpoint{5.475835in}{3.353813in}}%
\pgfpathlineto{\pgfqpoint{5.476329in}{2.906412in}}%
\pgfpathlineto{\pgfqpoint{5.476822in}{3.297536in}}%
\pgfpathlineto{\pgfqpoint{5.477315in}{3.297536in}}%
\pgfpathlineto{\pgfqpoint{5.478302in}{3.346582in}}%
\pgfpathlineto{\pgfqpoint{5.478795in}{3.339876in}}%
\pgfpathlineto{\pgfqpoint{5.479288in}{3.072224in}}%
\pgfpathlineto{\pgfqpoint{5.479781in}{3.306049in}}%
\pgfpathlineto{\pgfqpoint{5.480275in}{3.306049in}}%
\pgfpathlineto{\pgfqpoint{5.480768in}{3.313447in}}%
\pgfpathlineto{\pgfqpoint{5.481261in}{3.071447in}}%
\pgfpathlineto{\pgfqpoint{5.481754in}{3.319547in}}%
\pgfpathlineto{\pgfqpoint{5.482247in}{3.333658in}}%
\pgfpathlineto{\pgfqpoint{5.482741in}{3.321464in}}%
\pgfpathlineto{\pgfqpoint{5.486193in}{3.321464in}}%
\pgfpathlineto{\pgfqpoint{5.487180in}{3.318038in}}%
\pgfpathlineto{\pgfqpoint{5.488166in}{2.877644in}}%
\pgfpathlineto{\pgfqpoint{5.488659in}{3.370999in}}%
\pgfpathlineto{\pgfqpoint{5.489153in}{3.351330in}}%
\pgfpathlineto{\pgfqpoint{5.491126in}{3.030815in}}%
\pgfpathlineto{\pgfqpoint{5.491619in}{3.057572in}}%
\pgfpathlineto{\pgfqpoint{5.493099in}{3.350398in}}%
\pgfpathlineto{\pgfqpoint{5.494578in}{3.330356in}}%
\pgfpathlineto{\pgfqpoint{5.495071in}{3.330356in}}%
\pgfpathlineto{\pgfqpoint{5.495565in}{3.379387in}}%
\pgfpathlineto{\pgfqpoint{5.497538in}{3.134852in}}%
\pgfpathlineto{\pgfqpoint{5.499017in}{3.377998in}}%
\pgfpathlineto{\pgfqpoint{5.499511in}{2.987471in}}%
\pgfpathlineto{\pgfqpoint{5.500004in}{3.329488in}}%
\pgfpathlineto{\pgfqpoint{5.502470in}{3.329488in}}%
\pgfpathlineto{\pgfqpoint{5.502963in}{3.195913in}}%
\pgfpathlineto{\pgfqpoint{5.503456in}{3.360935in}}%
\pgfpathlineto{\pgfqpoint{5.504936in}{3.379796in}}%
\pgfpathlineto{\pgfqpoint{5.505923in}{3.341748in}}%
\pgfpathlineto{\pgfqpoint{5.506909in}{3.379711in}}%
\pgfpathlineto{\pgfqpoint{5.508389in}{3.241272in}}%
\pgfpathlineto{\pgfqpoint{5.508882in}{3.241272in}}%
\pgfpathlineto{\pgfqpoint{5.510362in}{3.297829in}}%
\pgfpathlineto{\pgfqpoint{5.510855in}{3.297829in}}%
\pgfpathlineto{\pgfqpoint{5.511348in}{3.375095in}}%
\pgfpathlineto{\pgfqpoint{5.512335in}{3.359572in}}%
\pgfpathlineto{\pgfqpoint{5.513321in}{3.359572in}}%
\pgfpathlineto{\pgfqpoint{5.514307in}{3.362387in}}%
\pgfpathlineto{\pgfqpoint{5.514801in}{3.312593in}}%
\pgfpathlineto{\pgfqpoint{5.515294in}{3.322309in}}%
\pgfpathlineto{\pgfqpoint{5.516774in}{3.303425in}}%
\pgfpathlineto{\pgfqpoint{5.518253in}{3.358690in}}%
\pgfpathlineto{\pgfqpoint{5.519733in}{3.358690in}}%
\pgfpathlineto{\pgfqpoint{5.520719in}{3.063484in}}%
\pgfpathlineto{\pgfqpoint{5.521213in}{3.375270in}}%
\pgfpathlineto{\pgfqpoint{5.522199in}{3.341176in}}%
\pgfpathlineto{\pgfqpoint{5.522692in}{3.307868in}}%
\pgfpathlineto{\pgfqpoint{5.523186in}{3.350023in}}%
\pgfpathlineto{\pgfqpoint{5.525159in}{2.872032in}}%
\pgfpathlineto{\pgfqpoint{5.525652in}{3.274000in}}%
\pgfpathlineto{\pgfqpoint{5.526145in}{3.128944in}}%
\pgfpathlineto{\pgfqpoint{5.527131in}{3.128944in}}%
\pgfpathlineto{\pgfqpoint{5.528118in}{3.370396in}}%
\pgfpathlineto{\pgfqpoint{5.528611in}{3.309117in}}%
\pgfpathlineto{\pgfqpoint{5.529104in}{3.260074in}}%
\pgfpathlineto{\pgfqpoint{5.530091in}{2.668657in}}%
\pgfpathlineto{\pgfqpoint{5.530584in}{3.379967in}}%
\pgfpathlineto{\pgfqpoint{5.531077in}{3.232190in}}%
\pgfpathlineto{\pgfqpoint{5.532064in}{3.031502in}}%
\pgfpathlineto{\pgfqpoint{5.533050in}{3.366158in}}%
\pgfpathlineto{\pgfqpoint{5.534037in}{3.105211in}}%
\pgfpathlineto{\pgfqpoint{5.534530in}{3.281349in}}%
\pgfpathlineto{\pgfqpoint{5.536503in}{3.360579in}}%
\pgfpathlineto{\pgfqpoint{5.537983in}{3.306306in}}%
\pgfpathlineto{\pgfqpoint{5.538969in}{3.358123in}}%
\pgfpathlineto{\pgfqpoint{5.540449in}{3.089639in}}%
\pgfpathlineto{\pgfqpoint{5.540942in}{3.089639in}}%
\pgfpathlineto{\pgfqpoint{5.541435in}{1.951236in}}%
\pgfpathlineto{\pgfqpoint{5.541928in}{3.361678in}}%
\pgfpathlineto{\pgfqpoint{5.542422in}{3.361678in}}%
\pgfpathlineto{\pgfqpoint{5.542915in}{2.537087in}}%
\pgfpathlineto{\pgfqpoint{5.543408in}{3.002418in}}%
\pgfpathlineto{\pgfqpoint{5.543901in}{3.002418in}}%
\pgfpathlineto{\pgfqpoint{5.544395in}{2.358003in}}%
\pgfpathlineto{\pgfqpoint{5.544888in}{3.019215in}}%
\pgfpathlineto{\pgfqpoint{5.545381in}{3.019215in}}%
\pgfpathlineto{\pgfqpoint{5.545874in}{3.036411in}}%
\pgfpathlineto{\pgfqpoint{5.546367in}{3.267993in}}%
\pgfpathlineto{\pgfqpoint{5.546861in}{3.033218in}}%
\pgfpathlineto{\pgfqpoint{5.547847in}{3.379643in}}%
\pgfpathlineto{\pgfqpoint{5.548340in}{3.302054in}}%
\pgfpathlineto{\pgfqpoint{5.548834in}{2.760857in}}%
\pgfpathlineto{\pgfqpoint{5.549327in}{3.330223in}}%
\pgfpathlineto{\pgfqpoint{5.551793in}{3.330223in}}%
\pgfpathlineto{\pgfqpoint{5.552286in}{3.164046in}}%
\pgfpathlineto{\pgfqpoint{5.553273in}{2.490141in}}%
\pgfpathlineto{\pgfqpoint{5.553766in}{3.356448in}}%
\pgfpathlineto{\pgfqpoint{5.554752in}{3.255267in}}%
\pgfpathlineto{\pgfqpoint{5.555246in}{3.255267in}}%
\pgfpathlineto{\pgfqpoint{5.556725in}{2.072455in}}%
\pgfpathlineto{\pgfqpoint{5.557219in}{1.823882in}}%
\pgfpathlineto{\pgfqpoint{5.557712in}{2.127730in}}%
\pgfpathlineto{\pgfqpoint{5.559191in}{3.379723in}}%
\pgfpathlineto{\pgfqpoint{5.562644in}{3.379723in}}%
\pgfpathlineto{\pgfqpoint{5.563137in}{3.243567in}}%
\pgfpathlineto{\pgfqpoint{5.563631in}{3.324737in}}%
\pgfpathlineto{\pgfqpoint{5.564617in}{3.324737in}}%
\pgfpathlineto{\pgfqpoint{5.566097in}{3.282440in}}%
\pgfpathlineto{\pgfqpoint{5.566590in}{3.282440in}}%
\pgfpathlineto{\pgfqpoint{5.567576in}{3.355223in}}%
\pgfpathlineto{\pgfqpoint{5.569056in}{3.264004in}}%
\pgfpathlineto{\pgfqpoint{5.570043in}{3.338533in}}%
\pgfpathlineto{\pgfqpoint{5.570536in}{3.151095in}}%
\pgfpathlineto{\pgfqpoint{5.571029in}{3.344530in}}%
\pgfpathlineto{\pgfqpoint{5.572015in}{3.344530in}}%
\pgfpathlineto{\pgfqpoint{5.573002in}{3.379863in}}%
\pgfpathlineto{\pgfqpoint{5.573495in}{2.803055in}}%
\pgfpathlineto{\pgfqpoint{5.573988in}{2.803055in}}%
\pgfpathlineto{\pgfqpoint{5.574975in}{3.343173in}}%
\pgfpathlineto{\pgfqpoint{5.575468in}{3.079975in}}%
\pgfpathlineto{\pgfqpoint{5.575961in}{3.312896in}}%
\pgfpathlineto{\pgfqpoint{5.576455in}{3.312896in}}%
\pgfpathlineto{\pgfqpoint{5.577934in}{3.379932in}}%
\pgfpathlineto{\pgfqpoint{5.579414in}{3.363852in}}%
\pgfpathlineto{\pgfqpoint{5.580400in}{3.363852in}}%
\pgfpathlineto{\pgfqpoint{5.580894in}{3.191297in}}%
\pgfpathlineto{\pgfqpoint{5.581387in}{3.329510in}}%
\pgfpathlineto{\pgfqpoint{5.581880in}{3.329510in}}%
\pgfpathlineto{\pgfqpoint{5.583853in}{3.379011in}}%
\pgfpathlineto{\pgfqpoint{5.585826in}{3.379011in}}%
\pgfpathlineto{\pgfqpoint{5.587306in}{3.294009in}}%
\pgfpathlineto{\pgfqpoint{5.587799in}{3.294009in}}%
\pgfpathlineto{\pgfqpoint{5.588785in}{3.364212in}}%
\pgfpathlineto{\pgfqpoint{5.589772in}{3.107811in}}%
\pgfpathlineto{\pgfqpoint{5.590265in}{3.124200in}}%
\pgfpathlineto{\pgfqpoint{5.590758in}{3.124200in}}%
\pgfpathlineto{\pgfqpoint{5.591252in}{3.379750in}}%
\pgfpathlineto{\pgfqpoint{5.591745in}{3.114938in}}%
\pgfpathlineto{\pgfqpoint{5.592238in}{3.374092in}}%
\pgfpathlineto{\pgfqpoint{5.592731in}{3.374092in}}%
\pgfpathlineto{\pgfqpoint{5.593718in}{3.378755in}}%
\pgfpathlineto{\pgfqpoint{5.594211in}{3.122494in}}%
\pgfpathlineto{\pgfqpoint{5.594704in}{3.378885in}}%
\pgfpathlineto{\pgfqpoint{5.595197in}{3.172427in}}%
\pgfpathlineto{\pgfqpoint{5.597170in}{3.379999in}}%
\pgfpathlineto{\pgfqpoint{5.598650in}{3.373868in}}%
\pgfpathlineto{\pgfqpoint{5.600130in}{3.378780in}}%
\pgfpathlineto{\pgfqpoint{5.601116in}{3.287695in}}%
\pgfpathlineto{\pgfqpoint{5.601609in}{2.875159in}}%
\pgfpathlineto{\pgfqpoint{5.602103in}{3.378737in}}%
\pgfpathlineto{\pgfqpoint{5.603089in}{3.378737in}}%
\pgfpathlineto{\pgfqpoint{5.603582in}{3.301075in}}%
\pgfpathlineto{\pgfqpoint{5.604076in}{3.375503in}}%
\pgfpathlineto{\pgfqpoint{5.604569in}{3.375503in}}%
\pgfpathlineto{\pgfqpoint{5.605062in}{3.379467in}}%
\pgfpathlineto{\pgfqpoint{5.605555in}{2.446067in}}%
\pgfpathlineto{\pgfqpoint{5.606048in}{3.327453in}}%
\pgfpathlineto{\pgfqpoint{5.606542in}{3.327453in}}%
\pgfpathlineto{\pgfqpoint{5.607528in}{3.373506in}}%
\pgfpathlineto{\pgfqpoint{5.608515in}{3.285100in}}%
\pgfpathlineto{\pgfqpoint{5.609008in}{3.344067in}}%
\pgfpathlineto{\pgfqpoint{5.609994in}{3.339037in}}%
\pgfpathlineto{\pgfqpoint{5.610488in}{3.339037in}}%
\pgfpathlineto{\pgfqpoint{5.611474in}{3.132191in}}%
\pgfpathlineto{\pgfqpoint{5.612954in}{2.923358in}}%
\pgfpathlineto{\pgfqpoint{5.613940in}{3.374066in}}%
\pgfpathlineto{\pgfqpoint{5.614433in}{3.360238in}}%
\pgfpathlineto{\pgfqpoint{5.616900in}{2.711546in}}%
\pgfpathlineto{\pgfqpoint{5.618379in}{3.317474in}}%
\pgfpathlineto{\pgfqpoint{5.619859in}{3.316651in}}%
\pgfpathlineto{\pgfqpoint{5.621339in}{2.606995in}}%
\pgfpathlineto{\pgfqpoint{5.622818in}{3.365208in}}%
\pgfpathlineto{\pgfqpoint{5.623312in}{3.365208in}}%
\pgfpathlineto{\pgfqpoint{5.624298in}{3.378782in}}%
\pgfpathlineto{\pgfqpoint{5.624791in}{3.369051in}}%
\pgfpathlineto{\pgfqpoint{5.625778in}{2.869009in}}%
\pgfpathlineto{\pgfqpoint{5.627257in}{3.352052in}}%
\pgfpathlineto{\pgfqpoint{5.627751in}{3.183230in}}%
\pgfpathlineto{\pgfqpoint{5.628244in}{3.230787in}}%
\pgfpathlineto{\pgfqpoint{5.629724in}{3.378602in}}%
\pgfpathlineto{\pgfqpoint{5.630217in}{3.310286in}}%
\pgfpathlineto{\pgfqpoint{5.631696in}{3.171128in}}%
\pgfpathlineto{\pgfqpoint{5.632190in}{3.171128in}}%
\pgfpathlineto{\pgfqpoint{5.633176in}{3.322068in}}%
\pgfpathlineto{\pgfqpoint{5.634163in}{3.277935in}}%
\pgfpathlineto{\pgfqpoint{5.635149in}{3.367050in}}%
\pgfpathlineto{\pgfqpoint{5.636136in}{3.188995in}}%
\pgfpathlineto{\pgfqpoint{5.637615in}{3.360338in}}%
\pgfpathlineto{\pgfqpoint{5.638108in}{3.375716in}}%
\pgfpathlineto{\pgfqpoint{5.638602in}{3.365418in}}%
\pgfpathlineto{\pgfqpoint{5.640081in}{2.030913in}}%
\pgfpathlineto{\pgfqpoint{5.640575in}{2.237111in}}%
\pgfpathlineto{\pgfqpoint{5.642054in}{3.325911in}}%
\pgfpathlineto{\pgfqpoint{5.643041in}{3.325911in}}%
\pgfpathlineto{\pgfqpoint{5.644520in}{2.839518in}}%
\pgfpathlineto{\pgfqpoint{5.645507in}{3.379908in}}%
\pgfpathlineto{\pgfqpoint{5.646000in}{3.316849in}}%
\pgfpathlineto{\pgfqpoint{5.647973in}{3.075015in}}%
\pgfpathlineto{\pgfqpoint{5.648466in}{3.075015in}}%
\pgfpathlineto{\pgfqpoint{5.649946in}{3.349816in}}%
\pgfpathlineto{\pgfqpoint{5.650439in}{3.377473in}}%
\pgfpathlineto{\pgfqpoint{5.651919in}{3.172254in}}%
\pgfpathlineto{\pgfqpoint{5.652905in}{3.079075in}}%
\pgfpathlineto{\pgfqpoint{5.654385in}{3.368899in}}%
\pgfpathlineto{\pgfqpoint{5.655372in}{3.368899in}}%
\pgfpathlineto{\pgfqpoint{5.656358in}{3.102996in}}%
\pgfpathlineto{\pgfqpoint{5.657838in}{3.379906in}}%
\pgfpathlineto{\pgfqpoint{5.658824in}{3.365081in}}%
\pgfpathlineto{\pgfqpoint{5.659811in}{2.969296in}}%
\pgfpathlineto{\pgfqpoint{5.660304in}{3.025554in}}%
\pgfpathlineto{\pgfqpoint{5.660797in}{3.379534in}}%
\pgfpathlineto{\pgfqpoint{5.661784in}{3.373093in}}%
\pgfpathlineto{\pgfqpoint{5.663263in}{3.237773in}}%
\pgfpathlineto{\pgfqpoint{5.663756in}{3.201428in}}%
\pgfpathlineto{\pgfqpoint{5.664250in}{3.354853in}}%
\pgfpathlineto{\pgfqpoint{5.664743in}{3.262500in}}%
\pgfpathlineto{\pgfqpoint{5.665729in}{3.161574in}}%
\pgfpathlineto{\pgfqpoint{5.667702in}{3.377997in}}%
\pgfpathlineto{\pgfqpoint{5.668196in}{3.225991in}}%
\pgfpathlineto{\pgfqpoint{5.668689in}{3.250605in}}%
\pgfpathlineto{\pgfqpoint{5.669182in}{3.335431in}}%
\pgfpathlineto{\pgfqpoint{5.670168in}{2.979644in}}%
\pgfpathlineto{\pgfqpoint{5.670662in}{3.361736in}}%
\pgfpathlineto{\pgfqpoint{5.671648in}{3.276627in}}%
\pgfpathlineto{\pgfqpoint{5.673128in}{3.361246in}}%
\pgfpathlineto{\pgfqpoint{5.674114in}{3.378065in}}%
\pgfpathlineto{\pgfqpoint{5.674608in}{2.966725in}}%
\pgfpathlineto{\pgfqpoint{5.675101in}{3.201769in}}%
\pgfpathlineto{\pgfqpoint{5.675594in}{3.201769in}}%
\pgfpathlineto{\pgfqpoint{5.676580in}{3.200758in}}%
\pgfpathlineto{\pgfqpoint{5.677074in}{3.379727in}}%
\pgfpathlineto{\pgfqpoint{5.677567in}{3.372200in}}%
\pgfpathlineto{\pgfqpoint{5.678060in}{2.929219in}}%
\pgfpathlineto{\pgfqpoint{5.678553in}{3.379829in}}%
\pgfpathlineto{\pgfqpoint{5.680033in}{3.372224in}}%
\pgfpathlineto{\pgfqpoint{5.680526in}{3.372224in}}%
\pgfpathlineto{\pgfqpoint{5.681020in}{3.317976in}}%
\pgfpathlineto{\pgfqpoint{5.682006in}{2.836558in}}%
\pgfpathlineto{\pgfqpoint{5.682993in}{3.324888in}}%
\pgfpathlineto{\pgfqpoint{5.683486in}{3.085616in}}%
\pgfpathlineto{\pgfqpoint{5.683979in}{3.342488in}}%
\pgfpathlineto{\pgfqpoint{5.685459in}{3.378396in}}%
\pgfpathlineto{\pgfqpoint{5.687432in}{2.805387in}}%
\pgfpathlineto{\pgfqpoint{5.687925in}{3.322161in}}%
\pgfpathlineto{\pgfqpoint{5.688911in}{3.317565in}}%
\pgfpathlineto{\pgfqpoint{5.692364in}{3.317565in}}%
\pgfpathlineto{\pgfqpoint{5.692857in}{3.298785in}}%
\pgfpathlineto{\pgfqpoint{5.693350in}{3.152490in}}%
\pgfpathlineto{\pgfqpoint{5.693844in}{3.220915in}}%
\pgfpathlineto{\pgfqpoint{5.694337in}{3.379525in}}%
\pgfpathlineto{\pgfqpoint{5.694830in}{3.374662in}}%
\pgfpathlineto{\pgfqpoint{5.695817in}{2.608724in}}%
\pgfpathlineto{\pgfqpoint{5.697296in}{3.339521in}}%
\pgfpathlineto{\pgfqpoint{5.697789in}{3.294643in}}%
\pgfpathlineto{\pgfqpoint{5.698283in}{3.379995in}}%
\pgfpathlineto{\pgfqpoint{5.699269in}{2.726245in}}%
\pgfpathlineto{\pgfqpoint{5.700749in}{3.303556in}}%
\pgfpathlineto{\pgfqpoint{5.701242in}{3.084317in}}%
\pgfpathlineto{\pgfqpoint{5.701735in}{3.294746in}}%
\pgfpathlineto{\pgfqpoint{5.702229in}{3.294746in}}%
\pgfpathlineto{\pgfqpoint{5.703215in}{3.348310in}}%
\pgfpathlineto{\pgfqpoint{5.703708in}{3.057171in}}%
\pgfpathlineto{\pgfqpoint{5.704201in}{3.374953in}}%
\pgfpathlineto{\pgfqpoint{5.705681in}{3.302119in}}%
\pgfpathlineto{\pgfqpoint{5.707161in}{3.317131in}}%
\pgfpathlineto{\pgfqpoint{5.708147in}{3.317131in}}%
\pgfpathlineto{\pgfqpoint{5.709134in}{3.354595in}}%
\pgfpathlineto{\pgfqpoint{5.710120in}{3.256690in}}%
\pgfpathlineto{\pgfqpoint{5.711600in}{3.339129in}}%
\pgfpathlineto{\pgfqpoint{5.712586in}{3.339129in}}%
\pgfpathlineto{\pgfqpoint{5.713080in}{3.372315in}}%
\pgfpathlineto{\pgfqpoint{5.714066in}{3.158672in}}%
\pgfpathlineto{\pgfqpoint{5.714559in}{3.369736in}}%
\pgfpathlineto{\pgfqpoint{5.715546in}{3.338946in}}%
\pgfpathlineto{\pgfqpoint{5.716532in}{3.338946in}}%
\pgfpathlineto{\pgfqpoint{5.717025in}{3.375683in}}%
\pgfpathlineto{\pgfqpoint{5.717519in}{2.848171in}}%
\pgfpathlineto{\pgfqpoint{5.718012in}{3.274125in}}%
\pgfpathlineto{\pgfqpoint{5.719492in}{3.274125in}}%
\pgfpathlineto{\pgfqpoint{5.720478in}{2.789627in}}%
\pgfpathlineto{\pgfqpoint{5.721958in}{3.215907in}}%
\pgfpathlineto{\pgfqpoint{5.722944in}{3.215907in}}%
\pgfpathlineto{\pgfqpoint{5.723437in}{3.319476in}}%
\pgfpathlineto{\pgfqpoint{5.723931in}{3.196931in}}%
\pgfpathlineto{\pgfqpoint{5.725410in}{3.128173in}}%
\pgfpathlineto{\pgfqpoint{5.726890in}{3.128173in}}%
\pgfpathlineto{\pgfqpoint{5.727877in}{3.343833in}}%
\pgfpathlineto{\pgfqpoint{5.728370in}{3.246741in}}%
\pgfpathlineto{\pgfqpoint{5.728863in}{3.365219in}}%
\pgfpathlineto{\pgfqpoint{5.729356in}{3.368749in}}%
\pgfpathlineto{\pgfqpoint{5.729849in}{3.029917in}}%
\pgfpathlineto{\pgfqpoint{5.730343in}{3.298715in}}%
\pgfpathlineto{\pgfqpoint{5.731822in}{3.298715in}}%
\pgfpathlineto{\pgfqpoint{5.733302in}{3.377716in}}%
\pgfpathlineto{\pgfqpoint{5.734782in}{3.377716in}}%
\pgfpathlineto{\pgfqpoint{5.735768in}{3.358137in}}%
\pgfpathlineto{\pgfqpoint{5.736261in}{3.358381in}}%
\pgfpathlineto{\pgfqpoint{5.737248in}{3.358381in}}%
\pgfpathlineto{\pgfqpoint{5.737741in}{3.221185in}}%
\pgfpathlineto{\pgfqpoint{5.738234in}{3.229502in}}%
\pgfpathlineto{\pgfqpoint{5.738728in}{3.341304in}}%
\pgfpathlineto{\pgfqpoint{5.739221in}{2.763895in}}%
\pgfpathlineto{\pgfqpoint{5.739714in}{3.266863in}}%
\pgfpathlineto{\pgfqpoint{5.740207in}{3.377604in}}%
\pgfpathlineto{\pgfqpoint{5.740701in}{3.292176in}}%
\pgfpathlineto{\pgfqpoint{5.741687in}{3.292176in}}%
\pgfpathlineto{\pgfqpoint{5.742180in}{2.780696in}}%
\pgfpathlineto{\pgfqpoint{5.742673in}{3.379686in}}%
\pgfpathlineto{\pgfqpoint{5.743660in}{3.379686in}}%
\pgfpathlineto{\pgfqpoint{5.744153in}{3.287074in}}%
\pgfpathlineto{\pgfqpoint{5.744646in}{3.355865in}}%
\pgfpathlineto{\pgfqpoint{5.745633in}{3.361789in}}%
\pgfpathlineto{\pgfqpoint{5.746126in}{3.128207in}}%
\pgfpathlineto{\pgfqpoint{5.746619in}{3.195521in}}%
\pgfpathlineto{\pgfqpoint{5.748099in}{3.307698in}}%
\pgfpathlineto{\pgfqpoint{5.748592in}{3.340042in}}%
\pgfpathlineto{\pgfqpoint{5.749085in}{3.024293in}}%
\pgfpathlineto{\pgfqpoint{5.749579in}{3.378314in}}%
\pgfpathlineto{\pgfqpoint{5.751058in}{3.378314in}}%
\pgfpathlineto{\pgfqpoint{5.752538in}{3.365376in}}%
\pgfpathlineto{\pgfqpoint{5.753031in}{3.365376in}}%
\pgfpathlineto{\pgfqpoint{5.753525in}{2.873412in}}%
\pgfpathlineto{\pgfqpoint{5.754018in}{3.147021in}}%
\pgfpathlineto{\pgfqpoint{5.755004in}{3.147021in}}%
\pgfpathlineto{\pgfqpoint{5.755497in}{3.114790in}}%
\pgfpathlineto{\pgfqpoint{5.756977in}{3.369609in}}%
\pgfpathlineto{\pgfqpoint{5.758457in}{3.340055in}}%
\pgfpathlineto{\pgfqpoint{5.758950in}{3.340055in}}%
\pgfpathlineto{\pgfqpoint{5.759443in}{3.357529in}}%
\pgfpathlineto{\pgfqpoint{5.760430in}{2.488470in}}%
\pgfpathlineto{\pgfqpoint{5.760923in}{2.985382in}}%
\pgfpathlineto{\pgfqpoint{5.761416in}{2.776567in}}%
\pgfpathlineto{\pgfqpoint{5.762403in}{1.919749in}}%
\pgfpathlineto{\pgfqpoint{5.763389in}{3.306287in}}%
\pgfpathlineto{\pgfqpoint{5.763882in}{3.146485in}}%
\pgfpathlineto{\pgfqpoint{5.765362in}{3.373038in}}%
\pgfpathlineto{\pgfqpoint{5.766842in}{3.373038in}}%
\pgfpathlineto{\pgfqpoint{5.767335in}{3.332957in}}%
\pgfpathlineto{\pgfqpoint{5.768321in}{3.333028in}}%
\pgfpathlineto{\pgfqpoint{5.768815in}{3.000161in}}%
\pgfpathlineto{\pgfqpoint{5.769308in}{3.379686in}}%
\pgfpathlineto{\pgfqpoint{5.769801in}{3.318764in}}%
\pgfpathlineto{\pgfqpoint{5.770294in}{3.332638in}}%
\pgfpathlineto{\pgfqpoint{5.771281in}{2.783583in}}%
\pgfpathlineto{\pgfqpoint{5.772761in}{3.376029in}}%
\pgfpathlineto{\pgfqpoint{5.774733in}{3.376029in}}%
\pgfpathlineto{\pgfqpoint{5.777200in}{3.062185in}}%
\pgfpathlineto{\pgfqpoint{5.778679in}{3.242968in}}%
\pgfpathlineto{\pgfqpoint{5.779666in}{3.372324in}}%
\pgfpathlineto{\pgfqpoint{5.780652in}{3.275777in}}%
\pgfpathlineto{\pgfqpoint{5.782132in}{3.377289in}}%
\pgfpathlineto{\pgfqpoint{5.782625in}{3.294071in}}%
\pgfpathlineto{\pgfqpoint{5.783118in}{2.574185in}}%
\pgfpathlineto{\pgfqpoint{5.783612in}{3.341097in}}%
\pgfpathlineto{\pgfqpoint{5.784105in}{3.341097in}}%
\pgfpathlineto{\pgfqpoint{5.785585in}{3.375168in}}%
\pgfpathlineto{\pgfqpoint{5.786078in}{2.867254in}}%
\pgfpathlineto{\pgfqpoint{5.786571in}{3.274397in}}%
\pgfpathlineto{\pgfqpoint{5.789037in}{3.274397in}}%
\pgfpathlineto{\pgfqpoint{5.789530in}{3.379746in}}%
\pgfpathlineto{\pgfqpoint{5.790517in}{3.354300in}}%
\pgfpathlineto{\pgfqpoint{5.791503in}{3.354300in}}%
\pgfpathlineto{\pgfqpoint{5.792983in}{2.769088in}}%
\pgfpathlineto{\pgfqpoint{5.793476in}{3.378081in}}%
\pgfpathlineto{\pgfqpoint{5.793970in}{3.155543in}}%
\pgfpathlineto{\pgfqpoint{5.794463in}{3.155543in}}%
\pgfpathlineto{\pgfqpoint{5.794956in}{3.234045in}}%
\pgfpathlineto{\pgfqpoint{5.795449in}{3.192625in}}%
\pgfpathlineto{\pgfqpoint{5.795942in}{2.219980in}}%
\pgfpathlineto{\pgfqpoint{5.796436in}{3.305368in}}%
\pgfpathlineto{\pgfqpoint{5.797422in}{2.892275in}}%
\pgfpathlineto{\pgfqpoint{5.798409in}{3.379914in}}%
\pgfpathlineto{\pgfqpoint{5.798902in}{3.289263in}}%
\pgfpathlineto{\pgfqpoint{5.799395in}{3.362925in}}%
\pgfpathlineto{\pgfqpoint{5.801368in}{3.010781in}}%
\pgfpathlineto{\pgfqpoint{5.802848in}{3.353833in}}%
\pgfpathlineto{\pgfqpoint{5.803834in}{3.371874in}}%
\pgfpathlineto{\pgfqpoint{5.804821in}{3.323649in}}%
\pgfpathlineto{\pgfqpoint{5.805807in}{3.379965in}}%
\pgfpathlineto{\pgfqpoint{5.806300in}{2.803147in}}%
\pgfpathlineto{\pgfqpoint{5.806794in}{3.333674in}}%
\pgfpathlineto{\pgfqpoint{5.807780in}{3.378878in}}%
\pgfpathlineto{\pgfqpoint{5.808273in}{3.280075in}}%
\pgfpathlineto{\pgfqpoint{5.808766in}{3.356336in}}%
\pgfpathlineto{\pgfqpoint{5.810246in}{3.356336in}}%
\pgfpathlineto{\pgfqpoint{5.810739in}{3.030023in}}%
\pgfpathlineto{\pgfqpoint{5.811233in}{3.149876in}}%
\pgfpathlineto{\pgfqpoint{5.811726in}{3.358834in}}%
\pgfpathlineto{\pgfqpoint{5.812219in}{3.270687in}}%
\pgfpathlineto{\pgfqpoint{5.812712in}{2.826697in}}%
\pgfpathlineto{\pgfqpoint{5.813206in}{3.102246in}}%
\pgfpathlineto{\pgfqpoint{5.813699in}{3.102246in}}%
\pgfpathlineto{\pgfqpoint{5.815178in}{3.371242in}}%
\pgfpathlineto{\pgfqpoint{5.815672in}{3.338522in}}%
\pgfpathlineto{\pgfqpoint{5.817151in}{3.338522in}}%
\pgfpathlineto{\pgfqpoint{5.817645in}{2.224291in}}%
\pgfpathlineto{\pgfqpoint{5.818138in}{3.266062in}}%
\pgfpathlineto{\pgfqpoint{5.818631in}{3.266062in}}%
\pgfpathlineto{\pgfqpoint{5.819124in}{3.268949in}}%
\pgfpathlineto{\pgfqpoint{5.819618in}{3.378278in}}%
\pgfpathlineto{\pgfqpoint{5.820604in}{3.105225in}}%
\pgfpathlineto{\pgfqpoint{5.821097in}{3.359235in}}%
\pgfpathlineto{\pgfqpoint{5.821590in}{3.090581in}}%
\pgfpathlineto{\pgfqpoint{5.822084in}{3.090581in}}%
\pgfpathlineto{\pgfqpoint{5.822577in}{3.373670in}}%
\pgfpathlineto{\pgfqpoint{5.823070in}{2.325896in}}%
\pgfpathlineto{\pgfqpoint{5.823563in}{2.943938in}}%
\pgfpathlineto{\pgfqpoint{5.826030in}{3.270322in}}%
\pgfpathlineto{\pgfqpoint{5.826523in}{3.083318in}}%
\pgfpathlineto{\pgfqpoint{5.828002in}{3.370112in}}%
\pgfpathlineto{\pgfqpoint{5.828496in}{3.331978in}}%
\pgfpathlineto{\pgfqpoint{5.828989in}{3.336167in}}%
\pgfpathlineto{\pgfqpoint{5.829482in}{3.163454in}}%
\pgfpathlineto{\pgfqpoint{5.829975in}{3.351290in}}%
\pgfpathlineto{\pgfqpoint{5.830469in}{3.351290in}}%
\pgfpathlineto{\pgfqpoint{5.830962in}{3.090780in}}%
\pgfpathlineto{\pgfqpoint{5.831455in}{1.730093in}}%
\pgfpathlineto{\pgfqpoint{5.831948in}{2.826189in}}%
\pgfpathlineto{\pgfqpoint{5.834414in}{3.379912in}}%
\pgfpathlineto{\pgfqpoint{5.835401in}{3.379912in}}%
\pgfpathlineto{\pgfqpoint{5.835894in}{3.348874in}}%
\pgfpathlineto{\pgfqpoint{5.836387in}{3.208225in}}%
\pgfpathlineto{\pgfqpoint{5.836881in}{1.566007in}}%
\pgfpathlineto{\pgfqpoint{5.837374in}{3.323582in}}%
\pgfpathlineto{\pgfqpoint{5.837867in}{2.568874in}}%
\pgfpathlineto{\pgfqpoint{5.838360in}{3.311514in}}%
\pgfpathlineto{\pgfqpoint{5.839347in}{3.311514in}}%
\pgfpathlineto{\pgfqpoint{5.840333in}{3.310248in}}%
\pgfpathlineto{\pgfqpoint{5.841320in}{3.284186in}}%
\pgfpathlineto{\pgfqpoint{5.842306in}{3.377382in}}%
\pgfpathlineto{\pgfqpoint{5.842799in}{3.372788in}}%
\pgfpathlineto{\pgfqpoint{5.843293in}{2.887290in}}%
\pgfpathlineto{\pgfqpoint{5.844279in}{2.985550in}}%
\pgfpathlineto{\pgfqpoint{5.845759in}{3.296178in}}%
\pgfpathlineto{\pgfqpoint{5.847238in}{3.296178in}}%
\pgfpathlineto{\pgfqpoint{5.847732in}{2.831358in}}%
\pgfpathlineto{\pgfqpoint{5.848225in}{3.332296in}}%
\pgfpathlineto{\pgfqpoint{5.849705in}{3.023234in}}%
\pgfpathlineto{\pgfqpoint{5.850198in}{3.366015in}}%
\pgfpathlineto{\pgfqpoint{5.850691in}{3.243846in}}%
\pgfpathlineto{\pgfqpoint{5.851678in}{3.243846in}}%
\pgfpathlineto{\pgfqpoint{5.852171in}{2.759393in}}%
\pgfpathlineto{\pgfqpoint{5.852664in}{3.133982in}}%
\pgfpathlineto{\pgfqpoint{5.854144in}{3.376765in}}%
\pgfpathlineto{\pgfqpoint{5.854637in}{3.376765in}}%
\pgfpathlineto{\pgfqpoint{5.856117in}{3.012551in}}%
\pgfpathlineto{\pgfqpoint{5.856610in}{3.134898in}}%
\pgfpathlineto{\pgfqpoint{5.857596in}{3.214265in}}%
\pgfpathlineto{\pgfqpoint{5.859569in}{2.881197in}}%
\pgfpathlineto{\pgfqpoint{5.861049in}{3.126952in}}%
\pgfpathlineto{\pgfqpoint{5.862035in}{2.868102in}}%
\pgfpathlineto{\pgfqpoint{5.862529in}{2.950235in}}%
\pgfpathlineto{\pgfqpoint{5.863022in}{3.374303in}}%
\pgfpathlineto{\pgfqpoint{5.864008in}{3.270317in}}%
\pgfpathlineto{\pgfqpoint{5.865488in}{3.304003in}}%
\pgfpathlineto{\pgfqpoint{5.865981in}{3.208995in}}%
\pgfpathlineto{\pgfqpoint{5.867461in}{3.370364in}}%
\pgfpathlineto{\pgfqpoint{5.867954in}{3.370364in}}%
\pgfpathlineto{\pgfqpoint{5.868941in}{3.368833in}}%
\pgfpathlineto{\pgfqpoint{5.870420in}{3.370365in}}%
\pgfpathlineto{\pgfqpoint{5.870914in}{3.379911in}}%
\pgfpathlineto{\pgfqpoint{5.871900in}{3.203344in}}%
\pgfpathlineto{\pgfqpoint{5.872393in}{3.258235in}}%
\pgfpathlineto{\pgfqpoint{5.872886in}{3.258235in}}%
\pgfpathlineto{\pgfqpoint{5.873380in}{3.093388in}}%
\pgfpathlineto{\pgfqpoint{5.873873in}{3.146295in}}%
\pgfpathlineto{\pgfqpoint{5.874859in}{3.377844in}}%
\pgfpathlineto{\pgfqpoint{5.875353in}{2.949746in}}%
\pgfpathlineto{\pgfqpoint{5.875846in}{3.098775in}}%
\pgfpathlineto{\pgfqpoint{5.876832in}{3.098775in}}%
\pgfpathlineto{\pgfqpoint{5.877326in}{2.979076in}}%
\pgfpathlineto{\pgfqpoint{5.878805in}{3.373383in}}%
\pgfpathlineto{\pgfqpoint{5.879298in}{3.373383in}}%
\pgfpathlineto{\pgfqpoint{5.880285in}{3.338108in}}%
\pgfpathlineto{\pgfqpoint{5.881271in}{3.070268in}}%
\pgfpathlineto{\pgfqpoint{5.881765in}{3.208481in}}%
\pgfpathlineto{\pgfqpoint{5.882258in}{2.701300in}}%
\pgfpathlineto{\pgfqpoint{5.882751in}{3.314504in}}%
\pgfpathlineto{\pgfqpoint{5.883244in}{2.953644in}}%
\pgfpathlineto{\pgfqpoint{5.883738in}{3.254049in}}%
\pgfpathlineto{\pgfqpoint{5.885217in}{3.371526in}}%
\pgfpathlineto{\pgfqpoint{5.886697in}{2.794120in}}%
\pgfpathlineto{\pgfqpoint{5.888177in}{3.236742in}}%
\pgfpathlineto{\pgfqpoint{5.889656in}{3.167416in}}%
\pgfpathlineto{\pgfqpoint{5.891629in}{3.167416in}}%
\pgfpathlineto{\pgfqpoint{5.893109in}{3.348728in}}%
\pgfpathlineto{\pgfqpoint{5.893602in}{3.348728in}}%
\pgfpathlineto{\pgfqpoint{5.896068in}{3.033925in}}%
\pgfpathlineto{\pgfqpoint{5.899028in}{3.360201in}}%
\pgfpathlineto{\pgfqpoint{5.900507in}{3.147402in}}%
\pgfpathlineto{\pgfqpoint{5.901001in}{3.350975in}}%
\pgfpathlineto{\pgfqpoint{5.901987in}{3.343572in}}%
\pgfpathlineto{\pgfqpoint{5.903467in}{3.379298in}}%
\pgfpathlineto{\pgfqpoint{5.904453in}{3.379298in}}%
\pgfpathlineto{\pgfqpoint{5.905440in}{3.169218in}}%
\pgfpathlineto{\pgfqpoint{5.905933in}{3.323302in}}%
\pgfpathlineto{\pgfqpoint{5.906426in}{3.323302in}}%
\pgfpathlineto{\pgfqpoint{5.907413in}{3.355122in}}%
\pgfpathlineto{\pgfqpoint{5.908399in}{2.929646in}}%
\pgfpathlineto{\pgfqpoint{5.909879in}{3.376992in}}%
\pgfpathlineto{\pgfqpoint{5.910372in}{3.376992in}}%
\pgfpathlineto{\pgfqpoint{5.911359in}{3.077996in}}%
\pgfpathlineto{\pgfqpoint{5.912345in}{3.358606in}}%
\pgfpathlineto{\pgfqpoint{5.913331in}{3.222477in}}%
\pgfpathlineto{\pgfqpoint{5.913825in}{3.308385in}}%
\pgfpathlineto{\pgfqpoint{5.914811in}{3.002012in}}%
\pgfpathlineto{\pgfqpoint{5.916291in}{3.293274in}}%
\pgfpathlineto{\pgfqpoint{5.916784in}{3.293274in}}%
\pgfpathlineto{\pgfqpoint{5.918264in}{3.258039in}}%
\pgfpathlineto{\pgfqpoint{5.919250in}{3.258039in}}%
\pgfpathlineto{\pgfqpoint{5.919743in}{2.943430in}}%
\pgfpathlineto{\pgfqpoint{5.920237in}{3.188692in}}%
\pgfpathlineto{\pgfqpoint{5.921716in}{3.367771in}}%
\pgfpathlineto{\pgfqpoint{5.922210in}{3.367771in}}%
\pgfpathlineto{\pgfqpoint{5.923689in}{2.975508in}}%
\pgfpathlineto{\pgfqpoint{5.925169in}{3.038382in}}%
\pgfpathlineto{\pgfqpoint{5.925662in}{3.038382in}}%
\pgfpathlineto{\pgfqpoint{5.926649in}{2.687326in}}%
\pgfpathlineto{\pgfqpoint{5.928128in}{3.377752in}}%
\pgfpathlineto{\pgfqpoint{5.929608in}{3.377752in}}%
\pgfpathlineto{\pgfqpoint{5.931088in}{3.342071in}}%
\pgfpathlineto{\pgfqpoint{5.931581in}{3.356279in}}%
\pgfpathlineto{\pgfqpoint{5.932567in}{3.379320in}}%
\pgfpathlineto{\pgfqpoint{5.933061in}{2.279045in}}%
\pgfpathlineto{\pgfqpoint{5.933554in}{2.876406in}}%
\pgfpathlineto{\pgfqpoint{5.934047in}{0.894427in}}%
\pgfpathlineto{\pgfqpoint{5.934540in}{3.151463in}}%
\pgfpathlineto{\pgfqpoint{5.935034in}{3.151463in}}%
\pgfpathlineto{\pgfqpoint{5.936513in}{3.379894in}}%
\pgfpathlineto{\pgfqpoint{5.937993in}{3.379894in}}%
\pgfpathlineto{\pgfqpoint{5.938486in}{2.773364in}}%
\pgfpathlineto{\pgfqpoint{5.938979in}{3.281608in}}%
\pgfpathlineto{\pgfqpoint{5.940459in}{3.281608in}}%
\pgfpathlineto{\pgfqpoint{5.940952in}{3.196374in}}%
\pgfpathlineto{\pgfqpoint{5.941446in}{3.225534in}}%
\pgfpathlineto{\pgfqpoint{5.941939in}{3.225534in}}%
\pgfpathlineto{\pgfqpoint{5.943419in}{3.377218in}}%
\pgfpathlineto{\pgfqpoint{5.943912in}{2.346413in}}%
\pgfpathlineto{\pgfqpoint{5.944405in}{3.338592in}}%
\pgfpathlineto{\pgfqpoint{5.945391in}{3.265639in}}%
\pgfpathlineto{\pgfqpoint{5.945885in}{3.321125in}}%
\pgfpathlineto{\pgfqpoint{5.947364in}{3.136818in}}%
\pgfpathlineto{\pgfqpoint{5.947858in}{3.027644in}}%
\pgfpathlineto{\pgfqpoint{5.948351in}{2.295199in}}%
\pgfpathlineto{\pgfqpoint{5.948844in}{2.621276in}}%
\pgfpathlineto{\pgfqpoint{5.949337in}{2.621276in}}%
\pgfpathlineto{\pgfqpoint{5.950817in}{3.364586in}}%
\pgfpathlineto{\pgfqpoint{5.951310in}{3.364586in}}%
\pgfpathlineto{\pgfqpoint{5.951803in}{3.298344in}}%
\pgfpathlineto{\pgfqpoint{5.952297in}{2.872869in}}%
\pgfpathlineto{\pgfqpoint{5.952790in}{3.373922in}}%
\pgfpathlineto{\pgfqpoint{5.953283in}{3.137304in}}%
\pgfpathlineto{\pgfqpoint{5.953776in}{3.378633in}}%
\pgfpathlineto{\pgfqpoint{5.954763in}{3.378633in}}%
\pgfpathlineto{\pgfqpoint{5.955256in}{3.316941in}}%
\pgfpathlineto{\pgfqpoint{5.956736in}{3.035818in}}%
\pgfpathlineto{\pgfqpoint{5.957229in}{3.035818in}}%
\pgfpathlineto{\pgfqpoint{5.957722in}{2.490523in}}%
\pgfpathlineto{\pgfqpoint{5.958215in}{2.713496in}}%
\pgfpathlineto{\pgfqpoint{5.959695in}{3.311722in}}%
\pgfpathlineto{\pgfqpoint{5.961175in}{2.911453in}}%
\pgfpathlineto{\pgfqpoint{5.961668in}{2.911453in}}%
\pgfpathlineto{\pgfqpoint{5.962161in}{2.821268in}}%
\pgfpathlineto{\pgfqpoint{5.963641in}{2.975786in}}%
\pgfpathlineto{\pgfqpoint{5.964627in}{2.601742in}}%
\pgfpathlineto{\pgfqpoint{5.966107in}{3.374940in}}%
\pgfpathlineto{\pgfqpoint{5.967587in}{3.379493in}}%
\pgfpathlineto{\pgfqpoint{5.968080in}{3.379493in}}%
\pgfpathlineto{\pgfqpoint{5.969560in}{2.766225in}}%
\pgfpathlineto{\pgfqpoint{5.971039in}{3.227521in}}%
\pgfpathlineto{\pgfqpoint{5.971533in}{3.227521in}}%
\pgfpathlineto{\pgfqpoint{5.973012in}{2.559622in}}%
\pgfpathlineto{\pgfqpoint{5.974492in}{3.367773in}}%
\pgfpathlineto{\pgfqpoint{5.975972in}{3.307468in}}%
\pgfpathlineto{\pgfqpoint{5.976465in}{3.307468in}}%
\pgfpathlineto{\pgfqpoint{5.977451in}{3.283240in}}%
\pgfpathlineto{\pgfqpoint{5.978438in}{2.788964in}}%
\pgfpathlineto{\pgfqpoint{5.978931in}{3.101765in}}%
\pgfpathlineto{\pgfqpoint{5.979918in}{3.073767in}}%
\pgfpathlineto{\pgfqpoint{5.980904in}{3.379512in}}%
\pgfpathlineto{\pgfqpoint{5.981397in}{3.373596in}}%
\pgfpathlineto{\pgfqpoint{5.982877in}{3.186014in}}%
\pgfpathlineto{\pgfqpoint{5.983863in}{3.368741in}}%
\pgfpathlineto{\pgfqpoint{5.984357in}{3.236645in}}%
\pgfpathlineto{\pgfqpoint{5.984850in}{3.379309in}}%
\pgfpathlineto{\pgfqpoint{5.985836in}{2.229986in}}%
\pgfpathlineto{\pgfqpoint{5.986330in}{3.379895in}}%
\pgfpathlineto{\pgfqpoint{5.987316in}{3.204887in}}%
\pgfpathlineto{\pgfqpoint{5.987809in}{3.204887in}}%
\pgfpathlineto{\pgfqpoint{5.989289in}{3.343597in}}%
\pgfpathlineto{\pgfqpoint{5.991262in}{3.367607in}}%
\pgfpathlineto{\pgfqpoint{5.991755in}{3.372269in}}%
\pgfpathlineto{\pgfqpoint{5.992742in}{2.566003in}}%
\pgfpathlineto{\pgfqpoint{5.994221in}{3.364849in}}%
\pgfpathlineto{\pgfqpoint{5.995208in}{3.348019in}}%
\pgfpathlineto{\pgfqpoint{5.996688in}{3.364937in}}%
\pgfpathlineto{\pgfqpoint{5.997181in}{2.468010in}}%
\pgfpathlineto{\pgfqpoint{5.997674in}{3.315884in}}%
\pgfpathlineto{\pgfqpoint{5.999647in}{3.315884in}}%
\pgfpathlineto{\pgfqpoint{6.000140in}{3.342436in}}%
\pgfpathlineto{\pgfqpoint{6.000633in}{2.995169in}}%
\pgfpathlineto{\pgfqpoint{6.001620in}{3.065826in}}%
\pgfpathlineto{\pgfqpoint{6.002113in}{3.065826in}}%
\pgfpathlineto{\pgfqpoint{6.003100in}{3.359939in}}%
\pgfpathlineto{\pgfqpoint{6.003593in}{3.344688in}}%
\pgfpathlineto{\pgfqpoint{6.004086in}{3.356408in}}%
\pgfpathlineto{\pgfqpoint{6.004579in}{2.698822in}}%
\pgfpathlineto{\pgfqpoint{6.005072in}{3.339075in}}%
\pgfpathlineto{\pgfqpoint{6.005566in}{3.121057in}}%
\pgfpathlineto{\pgfqpoint{6.006059in}{3.349593in}}%
\pgfpathlineto{\pgfqpoint{6.006552in}{3.349593in}}%
\pgfpathlineto{\pgfqpoint{6.007539in}{2.714403in}}%
\pgfpathlineto{\pgfqpoint{6.009018in}{3.370473in}}%
\pgfpathlineto{\pgfqpoint{6.009512in}{3.370473in}}%
\pgfpathlineto{\pgfqpoint{6.010498in}{3.022798in}}%
\pgfpathlineto{\pgfqpoint{6.011484in}{3.302267in}}%
\pgfpathlineto{\pgfqpoint{6.012471in}{2.875734in}}%
\pgfpathlineto{\pgfqpoint{6.012964in}{3.188846in}}%
\pgfpathlineto{\pgfqpoint{6.013457in}{3.188846in}}%
\pgfpathlineto{\pgfqpoint{6.013951in}{2.822428in}}%
\pgfpathlineto{\pgfqpoint{6.014444in}{0.611283in}}%
\pgfpathlineto{\pgfqpoint{6.014937in}{3.253507in}}%
\pgfpathlineto{\pgfqpoint{6.015924in}{3.253507in}}%
\pgfpathlineto{\pgfqpoint{6.016910in}{3.368312in}}%
\pgfpathlineto{\pgfqpoint{6.019376in}{2.683859in}}%
\pgfpathlineto{\pgfqpoint{6.020856in}{3.370393in}}%
\pgfpathlineto{\pgfqpoint{6.022336in}{3.370393in}}%
\pgfpathlineto{\pgfqpoint{6.022829in}{3.110951in}}%
\pgfpathlineto{\pgfqpoint{6.023322in}{3.278342in}}%
\pgfpathlineto{\pgfqpoint{6.023815in}{3.364087in}}%
\pgfpathlineto{\pgfqpoint{6.024308in}{3.284510in}}%
\pgfpathlineto{\pgfqpoint{6.024802in}{3.318942in}}%
\pgfpathlineto{\pgfqpoint{6.025788in}{2.975699in}}%
\pgfpathlineto{\pgfqpoint{6.027268in}{3.339658in}}%
\pgfpathlineto{\pgfqpoint{6.027761in}{3.339658in}}%
\pgfpathlineto{\pgfqpoint{6.028254in}{3.295961in}}%
\pgfpathlineto{\pgfqpoint{6.029241in}{3.351816in}}%
\pgfpathlineto{\pgfqpoint{6.031214in}{2.487405in}}%
\pgfpathlineto{\pgfqpoint{6.032693in}{3.376036in}}%
\pgfpathlineto{\pgfqpoint{6.033187in}{3.328010in}}%
\pgfpathlineto{\pgfqpoint{6.035160in}{2.791631in}}%
\pgfpathlineto{\pgfqpoint{6.036146in}{3.364978in}}%
\pgfpathlineto{\pgfqpoint{6.036639in}{3.228392in}}%
\pgfpathlineto{\pgfqpoint{6.038119in}{3.228392in}}%
\pgfpathlineto{\pgfqpoint{6.039599in}{3.340997in}}%
\pgfpathlineto{\pgfqpoint{6.040585in}{3.340997in}}%
\pgfpathlineto{\pgfqpoint{6.042065in}{3.378273in}}%
\pgfpathlineto{\pgfqpoint{6.042558in}{3.378273in}}%
\pgfpathlineto{\pgfqpoint{6.044038in}{3.379868in}}%
\pgfpathlineto{\pgfqpoint{6.045517in}{3.017707in}}%
\pgfpathlineto{\pgfqpoint{6.046011in}{3.361425in}}%
\pgfpathlineto{\pgfqpoint{6.046997in}{3.325033in}}%
\pgfpathlineto{\pgfqpoint{6.047490in}{3.053383in}}%
\pgfpathlineto{\pgfqpoint{6.047984in}{3.378301in}}%
\pgfpathlineto{\pgfqpoint{6.048477in}{3.378301in}}%
\pgfpathlineto{\pgfqpoint{6.049956in}{2.926693in}}%
\pgfpathlineto{\pgfqpoint{6.051436in}{3.343132in}}%
\pgfpathlineto{\pgfqpoint{6.052916in}{3.165088in}}%
\pgfpathlineto{\pgfqpoint{6.052423in}{3.357510in}}%
\pgfpathlineto{\pgfqpoint{6.053409in}{3.196021in}}%
\pgfusepath{stroke}%
\end{pgfscope}%
\begin{pgfscope}%
\pgfsetrectcap%
\pgfsetmiterjoin%
\pgfsetlinewidth{0.000000pt}%
\definecolor{currentstroke}{rgb}{1.000000,1.000000,1.000000}%
\pgfsetstrokecolor{currentstroke}%
\pgfsetdash{}{0pt}%
\pgfpathmoveto{\pgfqpoint{0.875000in}{0.440000in}}%
\pgfpathlineto{\pgfqpoint{0.875000in}{3.520000in}}%
\pgfusepath{}%
\end{pgfscope}%
\begin{pgfscope}%
\pgfsetrectcap%
\pgfsetmiterjoin%
\pgfsetlinewidth{0.000000pt}%
\definecolor{currentstroke}{rgb}{1.000000,1.000000,1.000000}%
\pgfsetstrokecolor{currentstroke}%
\pgfsetdash{}{0pt}%
\pgfpathmoveto{\pgfqpoint{6.300000in}{0.440000in}}%
\pgfpathlineto{\pgfqpoint{6.300000in}{3.520000in}}%
\pgfusepath{}%
\end{pgfscope}%
\begin{pgfscope}%
\pgfsetrectcap%
\pgfsetmiterjoin%
\pgfsetlinewidth{0.000000pt}%
\definecolor{currentstroke}{rgb}{1.000000,1.000000,1.000000}%
\pgfsetstrokecolor{currentstroke}%
\pgfsetdash{}{0pt}%
\pgfpathmoveto{\pgfqpoint{0.875000in}{0.440000in}}%
\pgfpathlineto{\pgfqpoint{6.300000in}{0.440000in}}%
\pgfusepath{}%
\end{pgfscope}%
\begin{pgfscope}%
\pgfsetrectcap%
\pgfsetmiterjoin%
\pgfsetlinewidth{0.000000pt}%
\definecolor{currentstroke}{rgb}{1.000000,1.000000,1.000000}%
\pgfsetstrokecolor{currentstroke}%
\pgfsetdash{}{0pt}%
\pgfpathmoveto{\pgfqpoint{0.875000in}{3.520000in}}%
\pgfpathlineto{\pgfqpoint{6.300000in}{3.520000in}}%
\pgfusepath{}%
\end{pgfscope}%
\begin{pgfscope}%
\definecolor{textcolor}{rgb}{0.150000,0.150000,0.150000}%
\pgfsetstrokecolor{textcolor}%
\pgfsetfillcolor{textcolor}%
\pgftext[x=3.500000in,y=3.920000in,,top]{\color{textcolor}\rmfamily\fontsize{12.000000}{14.400000}\selectfont Gráfica de \(\displaystyle \log f(X_t)\) para propuesta \(\displaystyle U(0,1)\), \(\displaystyle n=5, r=\)3}%
\end{pgfscope}%
\end{pgfpicture}%
\makeatother%
\endgroup%

        %% Creator: Matplotlib, PGF backend
%%
%% To include the figure in your LaTeX document, write
%%   \input{<filename>.pgf}
%%
%% Make sure the required packages are loaded in your preamble
%%   \usepackage{pgf}
%%
%% Also ensure that all the required font packages are loaded; for instance,
%% the lmodern package is sometimes necessary when using math font.
%%   \usepackage{lmodern}
%%
%% Figures using additional raster images can only be included by \input if
%% they are in the same directory as the main LaTeX file. For loading figures
%% from other directories you can use the `import` package
%%   \usepackage{import}
%%
%% and then include the figures with
%%   \import{<path to file>}{<filename>.pgf}
%%
%% Matplotlib used the following preamble
%%   
%%   \makeatletter\@ifpackageloaded{underscore}{}{\usepackage[strings]{underscore}}\makeatother
%%
\begingroup%
\makeatletter%
\begin{pgfpicture}%
\pgfpathrectangle{\pgfpointorigin}{\pgfqpoint{7.000000in}{4.000000in}}%
\pgfusepath{use as bounding box, clip}%
\begin{pgfscope}%
\pgfsetbuttcap%
\pgfsetmiterjoin%
\definecolor{currentfill}{rgb}{1.000000,1.000000,1.000000}%
\pgfsetfillcolor{currentfill}%
\pgfsetlinewidth{0.000000pt}%
\definecolor{currentstroke}{rgb}{1.000000,1.000000,1.000000}%
\pgfsetstrokecolor{currentstroke}%
\pgfsetdash{}{0pt}%
\pgfpathmoveto{\pgfqpoint{0.000000in}{0.000000in}}%
\pgfpathlineto{\pgfqpoint{7.000000in}{0.000000in}}%
\pgfpathlineto{\pgfqpoint{7.000000in}{4.000000in}}%
\pgfpathlineto{\pgfqpoint{0.000000in}{4.000000in}}%
\pgfpathlineto{\pgfqpoint{0.000000in}{0.000000in}}%
\pgfpathclose%
\pgfusepath{fill}%
\end{pgfscope}%
\begin{pgfscope}%
\pgfsetbuttcap%
\pgfsetmiterjoin%
\definecolor{currentfill}{rgb}{0.917647,0.917647,0.949020}%
\pgfsetfillcolor{currentfill}%
\pgfsetlinewidth{0.000000pt}%
\definecolor{currentstroke}{rgb}{0.000000,0.000000,0.000000}%
\pgfsetstrokecolor{currentstroke}%
\pgfsetstrokeopacity{0.000000}%
\pgfsetdash{}{0pt}%
\pgfpathmoveto{\pgfqpoint{0.875000in}{0.440000in}}%
\pgfpathlineto{\pgfqpoint{6.300000in}{0.440000in}}%
\pgfpathlineto{\pgfqpoint{6.300000in}{3.520000in}}%
\pgfpathlineto{\pgfqpoint{0.875000in}{3.520000in}}%
\pgfpathlineto{\pgfqpoint{0.875000in}{0.440000in}}%
\pgfpathclose%
\pgfusepath{fill}%
\end{pgfscope}%
\begin{pgfscope}%
\pgfpathrectangle{\pgfqpoint{0.875000in}{0.440000in}}{\pgfqpoint{5.425000in}{3.080000in}}%
\pgfusepath{clip}%
\pgfsetroundcap%
\pgfsetroundjoin%
\pgfsetlinewidth{1.003750pt}%
\definecolor{currentstroke}{rgb}{1.000000,1.000000,1.000000}%
\pgfsetstrokecolor{currentstroke}%
\pgfsetdash{}{0pt}%
\pgfpathmoveto{\pgfqpoint{1.121591in}{0.440000in}}%
\pgfpathlineto{\pgfqpoint{1.121591in}{3.520000in}}%
\pgfusepath{stroke}%
\end{pgfscope}%
\begin{pgfscope}%
\definecolor{textcolor}{rgb}{0.150000,0.150000,0.150000}%
\pgfsetstrokecolor{textcolor}%
\pgfsetfillcolor{textcolor}%
\pgftext[x=1.121591in,y=0.342778in,,top]{\color{textcolor}\rmfamily\fontsize{10.000000}{12.000000}\selectfont \(\displaystyle {0}\)}%
\end{pgfscope}%
\begin{pgfscope}%
\pgfpathrectangle{\pgfqpoint{0.875000in}{0.440000in}}{\pgfqpoint{5.425000in}{3.080000in}}%
\pgfusepath{clip}%
\pgfsetroundcap%
\pgfsetroundjoin%
\pgfsetlinewidth{1.003750pt}%
\definecolor{currentstroke}{rgb}{1.000000,1.000000,1.000000}%
\pgfsetstrokecolor{currentstroke}%
\pgfsetdash{}{0pt}%
\pgfpathmoveto{\pgfqpoint{2.108053in}{0.440000in}}%
\pgfpathlineto{\pgfqpoint{2.108053in}{3.520000in}}%
\pgfusepath{stroke}%
\end{pgfscope}%
\begin{pgfscope}%
\definecolor{textcolor}{rgb}{0.150000,0.150000,0.150000}%
\pgfsetstrokecolor{textcolor}%
\pgfsetfillcolor{textcolor}%
\pgftext[x=2.108053in,y=0.342778in,,top]{\color{textcolor}\rmfamily\fontsize{10.000000}{12.000000}\selectfont \(\displaystyle {2000}\)}%
\end{pgfscope}%
\begin{pgfscope}%
\pgfpathrectangle{\pgfqpoint{0.875000in}{0.440000in}}{\pgfqpoint{5.425000in}{3.080000in}}%
\pgfusepath{clip}%
\pgfsetroundcap%
\pgfsetroundjoin%
\pgfsetlinewidth{1.003750pt}%
\definecolor{currentstroke}{rgb}{1.000000,1.000000,1.000000}%
\pgfsetstrokecolor{currentstroke}%
\pgfsetdash{}{0pt}%
\pgfpathmoveto{\pgfqpoint{3.094515in}{0.440000in}}%
\pgfpathlineto{\pgfqpoint{3.094515in}{3.520000in}}%
\pgfusepath{stroke}%
\end{pgfscope}%
\begin{pgfscope}%
\definecolor{textcolor}{rgb}{0.150000,0.150000,0.150000}%
\pgfsetstrokecolor{textcolor}%
\pgfsetfillcolor{textcolor}%
\pgftext[x=3.094515in,y=0.342778in,,top]{\color{textcolor}\rmfamily\fontsize{10.000000}{12.000000}\selectfont \(\displaystyle {4000}\)}%
\end{pgfscope}%
\begin{pgfscope}%
\pgfpathrectangle{\pgfqpoint{0.875000in}{0.440000in}}{\pgfqpoint{5.425000in}{3.080000in}}%
\pgfusepath{clip}%
\pgfsetroundcap%
\pgfsetroundjoin%
\pgfsetlinewidth{1.003750pt}%
\definecolor{currentstroke}{rgb}{1.000000,1.000000,1.000000}%
\pgfsetstrokecolor{currentstroke}%
\pgfsetdash{}{0pt}%
\pgfpathmoveto{\pgfqpoint{4.080978in}{0.440000in}}%
\pgfpathlineto{\pgfqpoint{4.080978in}{3.520000in}}%
\pgfusepath{stroke}%
\end{pgfscope}%
\begin{pgfscope}%
\definecolor{textcolor}{rgb}{0.150000,0.150000,0.150000}%
\pgfsetstrokecolor{textcolor}%
\pgfsetfillcolor{textcolor}%
\pgftext[x=4.080978in,y=0.342778in,,top]{\color{textcolor}\rmfamily\fontsize{10.000000}{12.000000}\selectfont \(\displaystyle {6000}\)}%
\end{pgfscope}%
\begin{pgfscope}%
\pgfpathrectangle{\pgfqpoint{0.875000in}{0.440000in}}{\pgfqpoint{5.425000in}{3.080000in}}%
\pgfusepath{clip}%
\pgfsetroundcap%
\pgfsetroundjoin%
\pgfsetlinewidth{1.003750pt}%
\definecolor{currentstroke}{rgb}{1.000000,1.000000,1.000000}%
\pgfsetstrokecolor{currentstroke}%
\pgfsetdash{}{0pt}%
\pgfpathmoveto{\pgfqpoint{5.067440in}{0.440000in}}%
\pgfpathlineto{\pgfqpoint{5.067440in}{3.520000in}}%
\pgfusepath{stroke}%
\end{pgfscope}%
\begin{pgfscope}%
\definecolor{textcolor}{rgb}{0.150000,0.150000,0.150000}%
\pgfsetstrokecolor{textcolor}%
\pgfsetfillcolor{textcolor}%
\pgftext[x=5.067440in,y=0.342778in,,top]{\color{textcolor}\rmfamily\fontsize{10.000000}{12.000000}\selectfont \(\displaystyle {8000}\)}%
\end{pgfscope}%
\begin{pgfscope}%
\pgfpathrectangle{\pgfqpoint{0.875000in}{0.440000in}}{\pgfqpoint{5.425000in}{3.080000in}}%
\pgfusepath{clip}%
\pgfsetroundcap%
\pgfsetroundjoin%
\pgfsetlinewidth{1.003750pt}%
\definecolor{currentstroke}{rgb}{1.000000,1.000000,1.000000}%
\pgfsetstrokecolor{currentstroke}%
\pgfsetdash{}{0pt}%
\pgfpathmoveto{\pgfqpoint{6.053902in}{0.440000in}}%
\pgfpathlineto{\pgfqpoint{6.053902in}{3.520000in}}%
\pgfusepath{stroke}%
\end{pgfscope}%
\begin{pgfscope}%
\definecolor{textcolor}{rgb}{0.150000,0.150000,0.150000}%
\pgfsetstrokecolor{textcolor}%
\pgfsetfillcolor{textcolor}%
\pgftext[x=6.053902in,y=0.342778in,,top]{\color{textcolor}\rmfamily\fontsize{10.000000}{12.000000}\selectfont \(\displaystyle {10000}\)}%
\end{pgfscope}%
\begin{pgfscope}%
\pgfpathrectangle{\pgfqpoint{0.875000in}{0.440000in}}{\pgfqpoint{5.425000in}{3.080000in}}%
\pgfusepath{clip}%
\pgfsetroundcap%
\pgfsetroundjoin%
\pgfsetlinewidth{1.003750pt}%
\definecolor{currentstroke}{rgb}{1.000000,1.000000,1.000000}%
\pgfsetstrokecolor{currentstroke}%
\pgfsetdash{}{0pt}%
\pgfpathmoveto{\pgfqpoint{0.875000in}{0.684679in}}%
\pgfpathlineto{\pgfqpoint{6.300000in}{0.684679in}}%
\pgfusepath{stroke}%
\end{pgfscope}%
\begin{pgfscope}%
\definecolor{textcolor}{rgb}{0.150000,0.150000,0.150000}%
\pgfsetstrokecolor{textcolor}%
\pgfsetfillcolor{textcolor}%
\pgftext[x=0.530863in, y=0.636454in, left, base]{\color{textcolor}\rmfamily\fontsize{10.000000}{12.000000}\selectfont \(\displaystyle {\ensuremath{-}34}\)}%
\end{pgfscope}%
\begin{pgfscope}%
\pgfpathrectangle{\pgfqpoint{0.875000in}{0.440000in}}{\pgfqpoint{5.425000in}{3.080000in}}%
\pgfusepath{clip}%
\pgfsetroundcap%
\pgfsetroundjoin%
\pgfsetlinewidth{1.003750pt}%
\definecolor{currentstroke}{rgb}{1.000000,1.000000,1.000000}%
\pgfsetstrokecolor{currentstroke}%
\pgfsetdash{}{0pt}%
\pgfpathmoveto{\pgfqpoint{0.875000in}{1.086951in}}%
\pgfpathlineto{\pgfqpoint{6.300000in}{1.086951in}}%
\pgfusepath{stroke}%
\end{pgfscope}%
\begin{pgfscope}%
\definecolor{textcolor}{rgb}{0.150000,0.150000,0.150000}%
\pgfsetstrokecolor{textcolor}%
\pgfsetfillcolor{textcolor}%
\pgftext[x=0.530863in, y=1.038726in, left, base]{\color{textcolor}\rmfamily\fontsize{10.000000}{12.000000}\selectfont \(\displaystyle {\ensuremath{-}33}\)}%
\end{pgfscope}%
\begin{pgfscope}%
\pgfpathrectangle{\pgfqpoint{0.875000in}{0.440000in}}{\pgfqpoint{5.425000in}{3.080000in}}%
\pgfusepath{clip}%
\pgfsetroundcap%
\pgfsetroundjoin%
\pgfsetlinewidth{1.003750pt}%
\definecolor{currentstroke}{rgb}{1.000000,1.000000,1.000000}%
\pgfsetstrokecolor{currentstroke}%
\pgfsetdash{}{0pt}%
\pgfpathmoveto{\pgfqpoint{0.875000in}{1.489223in}}%
\pgfpathlineto{\pgfqpoint{6.300000in}{1.489223in}}%
\pgfusepath{stroke}%
\end{pgfscope}%
\begin{pgfscope}%
\definecolor{textcolor}{rgb}{0.150000,0.150000,0.150000}%
\pgfsetstrokecolor{textcolor}%
\pgfsetfillcolor{textcolor}%
\pgftext[x=0.530863in, y=1.440998in, left, base]{\color{textcolor}\rmfamily\fontsize{10.000000}{12.000000}\selectfont \(\displaystyle {\ensuremath{-}32}\)}%
\end{pgfscope}%
\begin{pgfscope}%
\pgfpathrectangle{\pgfqpoint{0.875000in}{0.440000in}}{\pgfqpoint{5.425000in}{3.080000in}}%
\pgfusepath{clip}%
\pgfsetroundcap%
\pgfsetroundjoin%
\pgfsetlinewidth{1.003750pt}%
\definecolor{currentstroke}{rgb}{1.000000,1.000000,1.000000}%
\pgfsetstrokecolor{currentstroke}%
\pgfsetdash{}{0pt}%
\pgfpathmoveto{\pgfqpoint{0.875000in}{1.891495in}}%
\pgfpathlineto{\pgfqpoint{6.300000in}{1.891495in}}%
\pgfusepath{stroke}%
\end{pgfscope}%
\begin{pgfscope}%
\definecolor{textcolor}{rgb}{0.150000,0.150000,0.150000}%
\pgfsetstrokecolor{textcolor}%
\pgfsetfillcolor{textcolor}%
\pgftext[x=0.530863in, y=1.843270in, left, base]{\color{textcolor}\rmfamily\fontsize{10.000000}{12.000000}\selectfont \(\displaystyle {\ensuremath{-}31}\)}%
\end{pgfscope}%
\begin{pgfscope}%
\pgfpathrectangle{\pgfqpoint{0.875000in}{0.440000in}}{\pgfqpoint{5.425000in}{3.080000in}}%
\pgfusepath{clip}%
\pgfsetroundcap%
\pgfsetroundjoin%
\pgfsetlinewidth{1.003750pt}%
\definecolor{currentstroke}{rgb}{1.000000,1.000000,1.000000}%
\pgfsetstrokecolor{currentstroke}%
\pgfsetdash{}{0pt}%
\pgfpathmoveto{\pgfqpoint{0.875000in}{2.293767in}}%
\pgfpathlineto{\pgfqpoint{6.300000in}{2.293767in}}%
\pgfusepath{stroke}%
\end{pgfscope}%
\begin{pgfscope}%
\definecolor{textcolor}{rgb}{0.150000,0.150000,0.150000}%
\pgfsetstrokecolor{textcolor}%
\pgfsetfillcolor{textcolor}%
\pgftext[x=0.530863in, y=2.245542in, left, base]{\color{textcolor}\rmfamily\fontsize{10.000000}{12.000000}\selectfont \(\displaystyle {\ensuremath{-}30}\)}%
\end{pgfscope}%
\begin{pgfscope}%
\pgfpathrectangle{\pgfqpoint{0.875000in}{0.440000in}}{\pgfqpoint{5.425000in}{3.080000in}}%
\pgfusepath{clip}%
\pgfsetroundcap%
\pgfsetroundjoin%
\pgfsetlinewidth{1.003750pt}%
\definecolor{currentstroke}{rgb}{1.000000,1.000000,1.000000}%
\pgfsetstrokecolor{currentstroke}%
\pgfsetdash{}{0pt}%
\pgfpathmoveto{\pgfqpoint{0.875000in}{2.696039in}}%
\pgfpathlineto{\pgfqpoint{6.300000in}{2.696039in}}%
\pgfusepath{stroke}%
\end{pgfscope}%
\begin{pgfscope}%
\definecolor{textcolor}{rgb}{0.150000,0.150000,0.150000}%
\pgfsetstrokecolor{textcolor}%
\pgfsetfillcolor{textcolor}%
\pgftext[x=0.530863in, y=2.647814in, left, base]{\color{textcolor}\rmfamily\fontsize{10.000000}{12.000000}\selectfont \(\displaystyle {\ensuremath{-}29}\)}%
\end{pgfscope}%
\begin{pgfscope}%
\pgfpathrectangle{\pgfqpoint{0.875000in}{0.440000in}}{\pgfqpoint{5.425000in}{3.080000in}}%
\pgfusepath{clip}%
\pgfsetroundcap%
\pgfsetroundjoin%
\pgfsetlinewidth{1.003750pt}%
\definecolor{currentstroke}{rgb}{1.000000,1.000000,1.000000}%
\pgfsetstrokecolor{currentstroke}%
\pgfsetdash{}{0pt}%
\pgfpathmoveto{\pgfqpoint{0.875000in}{3.098311in}}%
\pgfpathlineto{\pgfqpoint{6.300000in}{3.098311in}}%
\pgfusepath{stroke}%
\end{pgfscope}%
\begin{pgfscope}%
\definecolor{textcolor}{rgb}{0.150000,0.150000,0.150000}%
\pgfsetstrokecolor{textcolor}%
\pgfsetfillcolor{textcolor}%
\pgftext[x=0.530863in, y=3.050086in, left, base]{\color{textcolor}\rmfamily\fontsize{10.000000}{12.000000}\selectfont \(\displaystyle {\ensuremath{-}28}\)}%
\end{pgfscope}%
\begin{pgfscope}%
\pgfpathrectangle{\pgfqpoint{0.875000in}{0.440000in}}{\pgfqpoint{5.425000in}{3.080000in}}%
\pgfusepath{clip}%
\pgfsetroundcap%
\pgfsetroundjoin%
\pgfsetlinewidth{1.003750pt}%
\definecolor{currentstroke}{rgb}{1.000000,1.000000,1.000000}%
\pgfsetstrokecolor{currentstroke}%
\pgfsetdash{}{0pt}%
\pgfpathmoveto{\pgfqpoint{0.875000in}{3.500583in}}%
\pgfpathlineto{\pgfqpoint{6.300000in}{3.500583in}}%
\pgfusepath{stroke}%
\end{pgfscope}%
\begin{pgfscope}%
\definecolor{textcolor}{rgb}{0.150000,0.150000,0.150000}%
\pgfsetstrokecolor{textcolor}%
\pgfsetfillcolor{textcolor}%
\pgftext[x=0.530863in, y=3.452358in, left, base]{\color{textcolor}\rmfamily\fontsize{10.000000}{12.000000}\selectfont \(\displaystyle {\ensuremath{-}27}\)}%
\end{pgfscope}%
\begin{pgfscope}%
\pgfpathrectangle{\pgfqpoint{0.875000in}{0.440000in}}{\pgfqpoint{5.425000in}{3.080000in}}%
\pgfusepath{clip}%
\pgfsetroundcap%
\pgfsetroundjoin%
\pgfsetlinewidth{1.756562pt}%
\definecolor{currentstroke}{rgb}{0.298039,0.447059,0.690196}%
\pgfsetstrokecolor{currentstroke}%
\pgfsetdash{}{0pt}%
\pgfpathmoveto{\pgfqpoint{1.121591in}{3.169539in}}%
\pgfpathlineto{\pgfqpoint{1.123071in}{3.169539in}}%
\pgfpathlineto{\pgfqpoint{1.124550in}{3.378500in}}%
\pgfpathlineto{\pgfqpoint{1.125537in}{3.378500in}}%
\pgfpathlineto{\pgfqpoint{1.127016in}{2.759901in}}%
\pgfpathlineto{\pgfqpoint{1.128496in}{2.985431in}}%
\pgfpathlineto{\pgfqpoint{1.128989in}{2.985431in}}%
\pgfpathlineto{\pgfqpoint{1.130469in}{3.364214in}}%
\pgfpathlineto{\pgfqpoint{1.132442in}{3.364214in}}%
\pgfpathlineto{\pgfqpoint{1.133922in}{3.007228in}}%
\pgfpathlineto{\pgfqpoint{1.135401in}{3.372034in}}%
\pgfpathlineto{\pgfqpoint{1.136881in}{3.124453in}}%
\pgfpathlineto{\pgfqpoint{1.138361in}{3.124453in}}%
\pgfpathlineto{\pgfqpoint{1.138854in}{3.082868in}}%
\pgfpathlineto{\pgfqpoint{1.140827in}{2.835159in}}%
\pgfpathlineto{\pgfqpoint{1.142307in}{3.306906in}}%
\pgfpathlineto{\pgfqpoint{1.143293in}{3.306906in}}%
\pgfpathlineto{\pgfqpoint{1.143786in}{2.904166in}}%
\pgfpathlineto{\pgfqpoint{1.144280in}{3.379638in}}%
\pgfpathlineto{\pgfqpoint{1.145266in}{3.377092in}}%
\pgfpathlineto{\pgfqpoint{1.145759in}{3.138518in}}%
\pgfpathlineto{\pgfqpoint{1.146252in}{3.222408in}}%
\pgfpathlineto{\pgfqpoint{1.146746in}{3.222408in}}%
\pgfpathlineto{\pgfqpoint{1.147732in}{3.247134in}}%
\pgfpathlineto{\pgfqpoint{1.148225in}{3.377879in}}%
\pgfpathlineto{\pgfqpoint{1.149212in}{3.367273in}}%
\pgfpathlineto{\pgfqpoint{1.149705in}{3.367273in}}%
\pgfpathlineto{\pgfqpoint{1.152664in}{2.826390in}}%
\pgfpathlineto{\pgfqpoint{1.153158in}{2.826390in}}%
\pgfpathlineto{\pgfqpoint{1.154637in}{3.347530in}}%
\pgfpathlineto{\pgfqpoint{1.155131in}{3.324367in}}%
\pgfpathlineto{\pgfqpoint{1.155624in}{3.359249in}}%
\pgfpathlineto{\pgfqpoint{1.156117in}{3.101867in}}%
\pgfpathlineto{\pgfqpoint{1.156610in}{3.357507in}}%
\pgfpathlineto{\pgfqpoint{1.158090in}{3.357507in}}%
\pgfpathlineto{\pgfqpoint{1.159076in}{3.366614in}}%
\pgfpathlineto{\pgfqpoint{1.161543in}{2.758944in}}%
\pgfpathlineto{\pgfqpoint{1.163022in}{3.360178in}}%
\pgfpathlineto{\pgfqpoint{1.164502in}{3.360178in}}%
\pgfpathlineto{\pgfqpoint{1.164995in}{3.303502in}}%
\pgfpathlineto{\pgfqpoint{1.165488in}{3.330485in}}%
\pgfpathlineto{\pgfqpoint{1.166475in}{3.330485in}}%
\pgfpathlineto{\pgfqpoint{1.167955in}{3.156536in}}%
\pgfpathlineto{\pgfqpoint{1.169928in}{3.156536in}}%
\pgfpathlineto{\pgfqpoint{1.170421in}{3.130910in}}%
\pgfpathlineto{\pgfqpoint{1.171900in}{3.379870in}}%
\pgfpathlineto{\pgfqpoint{1.173380in}{3.311653in}}%
\pgfpathlineto{\pgfqpoint{1.174860in}{3.370805in}}%
\pgfpathlineto{\pgfqpoint{1.176340in}{3.370805in}}%
\pgfpathlineto{\pgfqpoint{1.177326in}{3.021587in}}%
\pgfpathlineto{\pgfqpoint{1.178312in}{3.361179in}}%
\pgfpathlineto{\pgfqpoint{1.179792in}{3.175145in}}%
\pgfpathlineto{\pgfqpoint{1.180779in}{3.175145in}}%
\pgfpathlineto{\pgfqpoint{1.182258in}{3.163410in}}%
\pgfpathlineto{\pgfqpoint{1.182752in}{3.249243in}}%
\pgfpathlineto{\pgfqpoint{1.183738in}{2.740749in}}%
\pgfpathlineto{\pgfqpoint{1.184231in}{3.312705in}}%
\pgfpathlineto{\pgfqpoint{1.184724in}{3.139501in}}%
\pgfpathlineto{\pgfqpoint{1.186697in}{3.139501in}}%
\pgfpathlineto{\pgfqpoint{1.187191in}{2.807688in}}%
\pgfpathlineto{\pgfqpoint{1.187684in}{3.155061in}}%
\pgfpathlineto{\pgfqpoint{1.189164in}{3.379171in}}%
\pgfpathlineto{\pgfqpoint{1.190150in}{3.379171in}}%
\pgfpathlineto{\pgfqpoint{1.192123in}{3.103258in}}%
\pgfpathlineto{\pgfqpoint{1.193603in}{3.103258in}}%
\pgfpathlineto{\pgfqpoint{1.196069in}{3.323125in}}%
\pgfpathlineto{\pgfqpoint{1.196562in}{3.323125in}}%
\pgfpathlineto{\pgfqpoint{1.197055in}{3.232761in}}%
\pgfpathlineto{\pgfqpoint{1.197549in}{3.306107in}}%
\pgfpathlineto{\pgfqpoint{1.199028in}{3.341890in}}%
\pgfpathlineto{\pgfqpoint{1.199521in}{3.349472in}}%
\pgfpathlineto{\pgfqpoint{1.200508in}{3.375196in}}%
\pgfpathlineto{\pgfqpoint{1.201494in}{2.954244in}}%
\pgfpathlineto{\pgfqpoint{1.202974in}{3.286609in}}%
\pgfpathlineto{\pgfqpoint{1.206427in}{3.286609in}}%
\pgfpathlineto{\pgfqpoint{1.207906in}{3.313193in}}%
\pgfpathlineto{\pgfqpoint{1.210866in}{3.313193in}}%
\pgfpathlineto{\pgfqpoint{1.211852in}{3.378434in}}%
\pgfpathlineto{\pgfqpoint{1.212839in}{3.328389in}}%
\pgfpathlineto{\pgfqpoint{1.214318in}{3.369661in}}%
\pgfpathlineto{\pgfqpoint{1.219744in}{3.369661in}}%
\pgfpathlineto{\pgfqpoint{1.221224in}{3.222080in}}%
\pgfpathlineto{\pgfqpoint{1.223690in}{3.222080in}}%
\pgfpathlineto{\pgfqpoint{1.224676in}{3.176421in}}%
\pgfpathlineto{\pgfqpoint{1.226156in}{3.346638in}}%
\pgfpathlineto{\pgfqpoint{1.227142in}{3.346638in}}%
\pgfpathlineto{\pgfqpoint{1.228129in}{2.783616in}}%
\pgfpathlineto{\pgfqpoint{1.229609in}{3.376250in}}%
\pgfpathlineto{\pgfqpoint{1.230102in}{3.376250in}}%
\pgfpathlineto{\pgfqpoint{1.231581in}{3.277892in}}%
\pgfpathlineto{\pgfqpoint{1.233061in}{3.336822in}}%
\pgfpathlineto{\pgfqpoint{1.233554in}{3.270060in}}%
\pgfpathlineto{\pgfqpoint{1.234541in}{3.379767in}}%
\pgfpathlineto{\pgfqpoint{1.235034in}{2.260340in}}%
\pgfpathlineto{\pgfqpoint{1.235527in}{3.379948in}}%
\pgfpathlineto{\pgfqpoint{1.236514in}{3.376699in}}%
\pgfpathlineto{\pgfqpoint{1.237007in}{2.973895in}}%
\pgfpathlineto{\pgfqpoint{1.237500in}{3.324631in}}%
\pgfpathlineto{\pgfqpoint{1.238980in}{3.324631in}}%
\pgfpathlineto{\pgfqpoint{1.239966in}{3.162691in}}%
\pgfpathlineto{\pgfqpoint{1.240953in}{2.475551in}}%
\pgfpathlineto{\pgfqpoint{1.241446in}{3.259122in}}%
\pgfpathlineto{\pgfqpoint{1.241939in}{3.119685in}}%
\pgfpathlineto{\pgfqpoint{1.242433in}{1.686187in}}%
\pgfpathlineto{\pgfqpoint{1.242926in}{3.357664in}}%
\pgfpathlineto{\pgfqpoint{1.244405in}{3.357664in}}%
\pgfpathlineto{\pgfqpoint{1.245885in}{3.364551in}}%
\pgfpathlineto{\pgfqpoint{1.247365in}{3.364551in}}%
\pgfpathlineto{\pgfqpoint{1.248351in}{2.782082in}}%
\pgfpathlineto{\pgfqpoint{1.248845in}{3.257128in}}%
\pgfpathlineto{\pgfqpoint{1.249338in}{3.100168in}}%
\pgfpathlineto{\pgfqpoint{1.249831in}{3.100168in}}%
\pgfpathlineto{\pgfqpoint{1.251311in}{3.373025in}}%
\pgfpathlineto{\pgfqpoint{1.252790in}{3.373025in}}%
\pgfpathlineto{\pgfqpoint{1.253284in}{3.317119in}}%
\pgfpathlineto{\pgfqpoint{1.254270in}{2.909463in}}%
\pgfpathlineto{\pgfqpoint{1.255750in}{3.377449in}}%
\pgfpathlineto{\pgfqpoint{1.256736in}{3.377449in}}%
\pgfpathlineto{\pgfqpoint{1.257229in}{3.352769in}}%
\pgfpathlineto{\pgfqpoint{1.257723in}{3.368126in}}%
\pgfpathlineto{\pgfqpoint{1.258709in}{3.217720in}}%
\pgfpathlineto{\pgfqpoint{1.260189in}{3.359796in}}%
\pgfpathlineto{\pgfqpoint{1.263641in}{3.359796in}}%
\pgfpathlineto{\pgfqpoint{1.264135in}{3.371316in}}%
\pgfpathlineto{\pgfqpoint{1.265121in}{2.687185in}}%
\pgfpathlineto{\pgfqpoint{1.265614in}{3.126966in}}%
\pgfpathlineto{\pgfqpoint{1.266601in}{3.126966in}}%
\pgfpathlineto{\pgfqpoint{1.267094in}{2.839368in}}%
\pgfpathlineto{\pgfqpoint{1.267587in}{3.377814in}}%
\pgfpathlineto{\pgfqpoint{1.268081in}{3.343455in}}%
\pgfpathlineto{\pgfqpoint{1.269067in}{2.981415in}}%
\pgfpathlineto{\pgfqpoint{1.270547in}{3.370997in}}%
\pgfpathlineto{\pgfqpoint{1.271040in}{3.370997in}}%
\pgfpathlineto{\pgfqpoint{1.272520in}{3.133255in}}%
\pgfpathlineto{\pgfqpoint{1.273506in}{2.968742in}}%
\pgfpathlineto{\pgfqpoint{1.274493in}{3.379120in}}%
\pgfpathlineto{\pgfqpoint{1.274986in}{3.296526in}}%
\pgfpathlineto{\pgfqpoint{1.275479in}{3.296526in}}%
\pgfpathlineto{\pgfqpoint{1.276959in}{3.230843in}}%
\pgfpathlineto{\pgfqpoint{1.279918in}{3.230843in}}%
\pgfpathlineto{\pgfqpoint{1.281398in}{3.378183in}}%
\pgfpathlineto{\pgfqpoint{1.282877in}{3.007127in}}%
\pgfpathlineto{\pgfqpoint{1.283864in}{3.379421in}}%
\pgfpathlineto{\pgfqpoint{1.284357in}{3.356875in}}%
\pgfpathlineto{\pgfqpoint{1.286823in}{3.356875in}}%
\pgfpathlineto{\pgfqpoint{1.287810in}{3.368553in}}%
\pgfpathlineto{\pgfqpoint{1.289289in}{3.345495in}}%
\pgfpathlineto{\pgfqpoint{1.290769in}{3.117864in}}%
\pgfpathlineto{\pgfqpoint{1.292249in}{3.117864in}}%
\pgfpathlineto{\pgfqpoint{1.292742in}{2.738369in}}%
\pgfpathlineto{\pgfqpoint{1.294222in}{3.379597in}}%
\pgfpathlineto{\pgfqpoint{1.298168in}{3.379597in}}%
\pgfpathlineto{\pgfqpoint{1.298661in}{3.373131in}}%
\pgfpathlineto{\pgfqpoint{1.300141in}{3.290703in}}%
\pgfpathlineto{\pgfqpoint{1.301620in}{2.860827in}}%
\pgfpathlineto{\pgfqpoint{1.303593in}{3.302548in}}%
\pgfpathlineto{\pgfqpoint{1.305566in}{2.075075in}}%
\pgfpathlineto{\pgfqpoint{1.307046in}{3.378997in}}%
\pgfpathlineto{\pgfqpoint{1.309019in}{3.379457in}}%
\pgfpathlineto{\pgfqpoint{1.310498in}{2.855101in}}%
\pgfpathlineto{\pgfqpoint{1.311485in}{3.108970in}}%
\pgfpathlineto{\pgfqpoint{1.311978in}{3.075066in}}%
\pgfpathlineto{\pgfqpoint{1.312471in}{3.075066in}}%
\pgfpathlineto{\pgfqpoint{1.313951in}{2.854073in}}%
\pgfpathlineto{\pgfqpoint{1.314938in}{2.602692in}}%
\pgfpathlineto{\pgfqpoint{1.316417in}{3.286184in}}%
\pgfpathlineto{\pgfqpoint{1.317404in}{3.286184in}}%
\pgfpathlineto{\pgfqpoint{1.317897in}{2.584987in}}%
\pgfpathlineto{\pgfqpoint{1.318883in}{2.655175in}}%
\pgfpathlineto{\pgfqpoint{1.319377in}{2.655175in}}%
\pgfpathlineto{\pgfqpoint{1.320856in}{2.768571in}}%
\pgfpathlineto{\pgfqpoint{1.322336in}{2.768571in}}%
\pgfpathlineto{\pgfqpoint{1.323816in}{3.326388in}}%
\pgfpathlineto{\pgfqpoint{1.325295in}{3.378819in}}%
\pgfpathlineto{\pgfqpoint{1.326775in}{3.378819in}}%
\pgfpathlineto{\pgfqpoint{1.328748in}{2.233179in}}%
\pgfpathlineto{\pgfqpoint{1.329734in}{3.123081in}}%
\pgfpathlineto{\pgfqpoint{1.330228in}{3.115266in}}%
\pgfpathlineto{\pgfqpoint{1.331707in}{3.368402in}}%
\pgfpathlineto{\pgfqpoint{1.333187in}{3.371672in}}%
\pgfpathlineto{\pgfqpoint{1.334174in}{3.371672in}}%
\pgfpathlineto{\pgfqpoint{1.334667in}{3.111767in}}%
\pgfpathlineto{\pgfqpoint{1.335160in}{3.334934in}}%
\pgfpathlineto{\pgfqpoint{1.338613in}{3.334934in}}%
\pgfpathlineto{\pgfqpoint{1.339599in}{3.376491in}}%
\pgfpathlineto{\pgfqpoint{1.340092in}{3.065297in}}%
\pgfpathlineto{\pgfqpoint{1.341079in}{3.065297in}}%
\pgfpathlineto{\pgfqpoint{1.342558in}{3.260701in}}%
\pgfpathlineto{\pgfqpoint{1.343052in}{3.260701in}}%
\pgfpathlineto{\pgfqpoint{1.344531in}{3.236416in}}%
\pgfpathlineto{\pgfqpoint{1.345518in}{3.236416in}}%
\pgfpathlineto{\pgfqpoint{1.346011in}{3.092335in}}%
\pgfpathlineto{\pgfqpoint{1.346504in}{3.361574in}}%
\pgfpathlineto{\pgfqpoint{1.346998in}{2.961656in}}%
\pgfpathlineto{\pgfqpoint{1.347491in}{3.343707in}}%
\pgfpathlineto{\pgfqpoint{1.348970in}{3.379363in}}%
\pgfpathlineto{\pgfqpoint{1.349464in}{2.605693in}}%
\pgfpathlineto{\pgfqpoint{1.349957in}{3.086549in}}%
\pgfpathlineto{\pgfqpoint{1.351437in}{3.086549in}}%
\pgfpathlineto{\pgfqpoint{1.352423in}{3.078576in}}%
\pgfpathlineto{\pgfqpoint{1.353903in}{3.298260in}}%
\pgfpathlineto{\pgfqpoint{1.355382in}{3.379648in}}%
\pgfpathlineto{\pgfqpoint{1.357355in}{3.379648in}}%
\pgfpathlineto{\pgfqpoint{1.358835in}{3.339270in}}%
\pgfpathlineto{\pgfqpoint{1.359822in}{3.339270in}}%
\pgfpathlineto{\pgfqpoint{1.360315in}{3.207224in}}%
\pgfpathlineto{\pgfqpoint{1.360808in}{3.309163in}}%
\pgfpathlineto{\pgfqpoint{1.361301in}{3.309163in}}%
\pgfpathlineto{\pgfqpoint{1.361794in}{3.192366in}}%
\pgfpathlineto{\pgfqpoint{1.362288in}{3.323603in}}%
\pgfpathlineto{\pgfqpoint{1.363274in}{3.364654in}}%
\pgfpathlineto{\pgfqpoint{1.363767in}{3.356831in}}%
\pgfpathlineto{\pgfqpoint{1.365247in}{3.346232in}}%
\pgfpathlineto{\pgfqpoint{1.365740in}{2.796598in}}%
\pgfpathlineto{\pgfqpoint{1.366234in}{3.315945in}}%
\pgfpathlineto{\pgfqpoint{1.367220in}{3.229868in}}%
\pgfpathlineto{\pgfqpoint{1.368700in}{3.349814in}}%
\pgfpathlineto{\pgfqpoint{1.370179in}{3.357989in}}%
\pgfpathlineto{\pgfqpoint{1.373632in}{3.357989in}}%
\pgfpathlineto{\pgfqpoint{1.374618in}{3.133946in}}%
\pgfpathlineto{\pgfqpoint{1.376098in}{3.378190in}}%
\pgfpathlineto{\pgfqpoint{1.376591in}{3.378190in}}%
\pgfpathlineto{\pgfqpoint{1.377578in}{3.348839in}}%
\pgfpathlineto{\pgfqpoint{1.378071in}{3.158317in}}%
\pgfpathlineto{\pgfqpoint{1.379058in}{3.169619in}}%
\pgfpathlineto{\pgfqpoint{1.380537in}{3.182617in}}%
\pgfpathlineto{\pgfqpoint{1.382017in}{3.182617in}}%
\pgfpathlineto{\pgfqpoint{1.382510in}{3.328420in}}%
\pgfpathlineto{\pgfqpoint{1.383990in}{3.128627in}}%
\pgfpathlineto{\pgfqpoint{1.385963in}{3.128627in}}%
\pgfpathlineto{\pgfqpoint{1.386456in}{2.968601in}}%
\pgfpathlineto{\pgfqpoint{1.386949in}{2.983243in}}%
\pgfpathlineto{\pgfqpoint{1.388429in}{3.343211in}}%
\pgfpathlineto{\pgfqpoint{1.389909in}{3.087129in}}%
\pgfpathlineto{\pgfqpoint{1.390402in}{3.087129in}}%
\pgfpathlineto{\pgfqpoint{1.391388in}{3.370706in}}%
\pgfpathlineto{\pgfqpoint{1.391882in}{3.322543in}}%
\pgfpathlineto{\pgfqpoint{1.392375in}{2.639257in}}%
\pgfpathlineto{\pgfqpoint{1.392868in}{3.331847in}}%
\pgfpathlineto{\pgfqpoint{1.394348in}{3.373396in}}%
\pgfpathlineto{\pgfqpoint{1.395334in}{3.373396in}}%
\pgfpathlineto{\pgfqpoint{1.396814in}{3.356568in}}%
\pgfpathlineto{\pgfqpoint{1.397800in}{3.032924in}}%
\pgfpathlineto{\pgfqpoint{1.399280in}{3.355601in}}%
\pgfpathlineto{\pgfqpoint{1.399773in}{3.355601in}}%
\pgfpathlineto{\pgfqpoint{1.401253in}{3.364494in}}%
\pgfpathlineto{\pgfqpoint{1.404706in}{3.364494in}}%
\pgfpathlineto{\pgfqpoint{1.405692in}{3.369740in}}%
\pgfpathlineto{\pgfqpoint{1.406185in}{2.877756in}}%
\pgfpathlineto{\pgfqpoint{1.407665in}{3.377268in}}%
\pgfpathlineto{\pgfqpoint{1.408158in}{2.742755in}}%
\pgfpathlineto{\pgfqpoint{1.408651in}{3.076288in}}%
\pgfpathlineto{\pgfqpoint{1.409145in}{3.076288in}}%
\pgfpathlineto{\pgfqpoint{1.410624in}{3.319031in}}%
\pgfpathlineto{\pgfqpoint{1.413091in}{3.319031in}}%
\pgfpathlineto{\pgfqpoint{1.414077in}{3.082353in}}%
\pgfpathlineto{\pgfqpoint{1.415557in}{3.377715in}}%
\pgfpathlineto{\pgfqpoint{1.416050in}{3.377715in}}%
\pgfpathlineto{\pgfqpoint{1.417530in}{3.272854in}}%
\pgfpathlineto{\pgfqpoint{1.418023in}{3.272854in}}%
\pgfpathlineto{\pgfqpoint{1.419009in}{2.852380in}}%
\pgfpathlineto{\pgfqpoint{1.420489in}{3.228846in}}%
\pgfpathlineto{\pgfqpoint{1.420982in}{3.310730in}}%
\pgfpathlineto{\pgfqpoint{1.422462in}{3.101994in}}%
\pgfpathlineto{\pgfqpoint{1.425421in}{3.101994in}}%
\pgfpathlineto{\pgfqpoint{1.425915in}{2.436767in}}%
\pgfpathlineto{\pgfqpoint{1.426408in}{3.042732in}}%
\pgfpathlineto{\pgfqpoint{1.427394in}{3.042732in}}%
\pgfpathlineto{\pgfqpoint{1.428381in}{3.365385in}}%
\pgfpathlineto{\pgfqpoint{1.428874in}{3.326413in}}%
\pgfpathlineto{\pgfqpoint{1.429367in}{3.326413in}}%
\pgfpathlineto{\pgfqpoint{1.430847in}{3.377432in}}%
\pgfpathlineto{\pgfqpoint{1.431340in}{3.377432in}}%
\pgfpathlineto{\pgfqpoint{1.431833in}{3.369042in}}%
\pgfpathlineto{\pgfqpoint{1.433313in}{3.313115in}}%
\pgfpathlineto{\pgfqpoint{1.434299in}{3.313115in}}%
\pgfpathlineto{\pgfqpoint{1.435779in}{3.321851in}}%
\pgfpathlineto{\pgfqpoint{1.437752in}{3.321851in}}%
\pgfpathlineto{\pgfqpoint{1.438245in}{3.379061in}}%
\pgfpathlineto{\pgfqpoint{1.438739in}{3.343641in}}%
\pgfpathlineto{\pgfqpoint{1.440218in}{3.343641in}}%
\pgfpathlineto{\pgfqpoint{1.440711in}{2.996428in}}%
\pgfpathlineto{\pgfqpoint{1.441205in}{3.157495in}}%
\pgfpathlineto{\pgfqpoint{1.441698in}{3.157495in}}%
\pgfpathlineto{\pgfqpoint{1.442684in}{3.352903in}}%
\pgfpathlineto{\pgfqpoint{1.443178in}{3.317820in}}%
\pgfpathlineto{\pgfqpoint{1.443671in}{3.317820in}}%
\pgfpathlineto{\pgfqpoint{1.445151in}{3.275891in}}%
\pgfpathlineto{\pgfqpoint{1.448110in}{3.275891in}}%
\pgfpathlineto{\pgfqpoint{1.448603in}{3.102289in}}%
\pgfpathlineto{\pgfqpoint{1.449096in}{1.499015in}}%
\pgfpathlineto{\pgfqpoint{1.449590in}{3.376214in}}%
\pgfpathlineto{\pgfqpoint{1.450083in}{3.376214in}}%
\pgfpathlineto{\pgfqpoint{1.450576in}{3.373640in}}%
\pgfpathlineto{\pgfqpoint{1.452056in}{3.284320in}}%
\pgfpathlineto{\pgfqpoint{1.454029in}{3.284320in}}%
\pgfpathlineto{\pgfqpoint{1.455015in}{2.463946in}}%
\pgfpathlineto{\pgfqpoint{1.455508in}{3.212515in}}%
\pgfpathlineto{\pgfqpoint{1.456495in}{3.183848in}}%
\pgfpathlineto{\pgfqpoint{1.457975in}{3.312927in}}%
\pgfpathlineto{\pgfqpoint{1.461920in}{3.312927in}}%
\pgfpathlineto{\pgfqpoint{1.462414in}{3.378551in}}%
\pgfpathlineto{\pgfqpoint{1.462907in}{3.305190in}}%
\pgfpathlineto{\pgfqpoint{1.463400in}{3.305190in}}%
\pgfpathlineto{\pgfqpoint{1.463893in}{3.196969in}}%
\pgfpathlineto{\pgfqpoint{1.464387in}{3.361983in}}%
\pgfpathlineto{\pgfqpoint{1.465373in}{3.358816in}}%
\pgfpathlineto{\pgfqpoint{1.466359in}{3.272973in}}%
\pgfpathlineto{\pgfqpoint{1.466853in}{3.378078in}}%
\pgfpathlineto{\pgfqpoint{1.467839in}{3.025929in}}%
\pgfpathlineto{\pgfqpoint{1.468826in}{3.367950in}}%
\pgfpathlineto{\pgfqpoint{1.470305in}{2.543779in}}%
\pgfpathlineto{\pgfqpoint{1.471785in}{2.543779in}}%
\pgfpathlineto{\pgfqpoint{1.472278in}{2.507033in}}%
\pgfpathlineto{\pgfqpoint{1.472771in}{3.249490in}}%
\pgfpathlineto{\pgfqpoint{1.473758in}{3.204702in}}%
\pgfpathlineto{\pgfqpoint{1.474251in}{3.204702in}}%
\pgfpathlineto{\pgfqpoint{1.474744in}{2.995133in}}%
\pgfpathlineto{\pgfqpoint{1.476224in}{3.369144in}}%
\pgfpathlineto{\pgfqpoint{1.476717in}{3.369144in}}%
\pgfpathlineto{\pgfqpoint{1.478690in}{3.310240in}}%
\pgfpathlineto{\pgfqpoint{1.480170in}{3.310240in}}%
\pgfpathlineto{\pgfqpoint{1.480663in}{3.290820in}}%
\pgfpathlineto{\pgfqpoint{1.481156in}{3.162115in}}%
\pgfpathlineto{\pgfqpoint{1.482636in}{3.373741in}}%
\pgfpathlineto{\pgfqpoint{1.484116in}{3.373741in}}%
\pgfpathlineto{\pgfqpoint{1.486089in}{3.320371in}}%
\pgfpathlineto{\pgfqpoint{1.487568in}{3.326933in}}%
\pgfpathlineto{\pgfqpoint{1.489048in}{3.326933in}}%
\pgfpathlineto{\pgfqpoint{1.489541in}{2.921908in}}%
\pgfpathlineto{\pgfqpoint{1.490035in}{3.356871in}}%
\pgfpathlineto{\pgfqpoint{1.491514in}{3.379042in}}%
\pgfpathlineto{\pgfqpoint{1.493487in}{3.379042in}}%
\pgfpathlineto{\pgfqpoint{1.494967in}{3.223484in}}%
\pgfpathlineto{\pgfqpoint{1.496447in}{3.367056in}}%
\pgfpathlineto{\pgfqpoint{1.496940in}{3.367056in}}%
\pgfpathlineto{\pgfqpoint{1.497433in}{2.681923in}}%
\pgfpathlineto{\pgfqpoint{1.497926in}{3.273474in}}%
\pgfpathlineto{\pgfqpoint{1.498913in}{3.273474in}}%
\pgfpathlineto{\pgfqpoint{1.499406in}{3.375348in}}%
\pgfpathlineto{\pgfqpoint{1.499899in}{3.229112in}}%
\pgfpathlineto{\pgfqpoint{1.500392in}{3.303651in}}%
\pgfpathlineto{\pgfqpoint{1.502859in}{3.303651in}}%
\pgfpathlineto{\pgfqpoint{1.503845in}{3.349743in}}%
\pgfpathlineto{\pgfqpoint{1.506311in}{2.971873in}}%
\pgfpathlineto{\pgfqpoint{1.507791in}{3.365331in}}%
\pgfpathlineto{\pgfqpoint{1.508777in}{3.350725in}}%
\pgfpathlineto{\pgfqpoint{1.510257in}{3.379359in}}%
\pgfpathlineto{\pgfqpoint{1.510750in}{3.379359in}}%
\pgfpathlineto{\pgfqpoint{1.511244in}{2.318314in}}%
\pgfpathlineto{\pgfqpoint{1.512230in}{2.384490in}}%
\pgfpathlineto{\pgfqpoint{1.514203in}{3.351402in}}%
\pgfpathlineto{\pgfqpoint{1.515189in}{3.351402in}}%
\pgfpathlineto{\pgfqpoint{1.516176in}{3.095938in}}%
\pgfpathlineto{\pgfqpoint{1.517162in}{3.374361in}}%
\pgfpathlineto{\pgfqpoint{1.517656in}{3.370574in}}%
\pgfpathlineto{\pgfqpoint{1.518149in}{3.370574in}}%
\pgfpathlineto{\pgfqpoint{1.518642in}{3.365896in}}%
\pgfpathlineto{\pgfqpoint{1.520615in}{3.318489in}}%
\pgfpathlineto{\pgfqpoint{1.521108in}{3.318489in}}%
\pgfpathlineto{\pgfqpoint{1.522588in}{2.698815in}}%
\pgfpathlineto{\pgfqpoint{1.523081in}{2.701498in}}%
\pgfpathlineto{\pgfqpoint{1.523574in}{2.643429in}}%
\pgfpathlineto{\pgfqpoint{1.525054in}{3.373361in}}%
\pgfpathlineto{\pgfqpoint{1.525547in}{3.373361in}}%
\pgfpathlineto{\pgfqpoint{1.527027in}{2.590347in}}%
\pgfpathlineto{\pgfqpoint{1.527520in}{3.315492in}}%
\pgfpathlineto{\pgfqpoint{1.528507in}{3.205671in}}%
\pgfpathlineto{\pgfqpoint{1.529986in}{3.374963in}}%
\pgfpathlineto{\pgfqpoint{1.530480in}{3.374963in}}%
\pgfpathlineto{\pgfqpoint{1.531959in}{2.923768in}}%
\pgfpathlineto{\pgfqpoint{1.532452in}{3.333492in}}%
\pgfpathlineto{\pgfqpoint{1.533439in}{3.291486in}}%
\pgfpathlineto{\pgfqpoint{1.534919in}{3.074864in}}%
\pgfpathlineto{\pgfqpoint{1.535412in}{3.010595in}}%
\pgfpathlineto{\pgfqpoint{1.537385in}{2.097870in}}%
\pgfpathlineto{\pgfqpoint{1.538864in}{3.191318in}}%
\pgfpathlineto{\pgfqpoint{1.540344in}{3.379777in}}%
\pgfpathlineto{\pgfqpoint{1.541824in}{3.294872in}}%
\pgfpathlineto{\pgfqpoint{1.542317in}{3.294872in}}%
\pgfpathlineto{\pgfqpoint{1.543304in}{3.323987in}}%
\pgfpathlineto{\pgfqpoint{1.544290in}{3.164803in}}%
\pgfpathlineto{\pgfqpoint{1.544783in}{3.324219in}}%
\pgfpathlineto{\pgfqpoint{1.546263in}{3.078455in}}%
\pgfpathlineto{\pgfqpoint{1.547249in}{3.078455in}}%
\pgfpathlineto{\pgfqpoint{1.549716in}{3.333863in}}%
\pgfpathlineto{\pgfqpoint{1.550702in}{3.333863in}}%
\pgfpathlineto{\pgfqpoint{1.551688in}{3.379921in}}%
\pgfpathlineto{\pgfqpoint{1.553168in}{3.225274in}}%
\pgfpathlineto{\pgfqpoint{1.554155in}{3.225274in}}%
\pgfpathlineto{\pgfqpoint{1.554648in}{3.313945in}}%
\pgfpathlineto{\pgfqpoint{1.555141in}{3.259955in}}%
\pgfpathlineto{\pgfqpoint{1.556128in}{3.259955in}}%
\pgfpathlineto{\pgfqpoint{1.556621in}{2.828992in}}%
\pgfpathlineto{\pgfqpoint{1.557114in}{3.032376in}}%
\pgfpathlineto{\pgfqpoint{1.558594in}{3.310663in}}%
\pgfpathlineto{\pgfqpoint{1.559087in}{2.310330in}}%
\pgfpathlineto{\pgfqpoint{1.559580in}{2.577640in}}%
\pgfpathlineto{\pgfqpoint{1.561553in}{3.326988in}}%
\pgfpathlineto{\pgfqpoint{1.562540in}{3.326988in}}%
\pgfpathlineto{\pgfqpoint{1.564019in}{2.790188in}}%
\pgfpathlineto{\pgfqpoint{1.564512in}{2.764558in}}%
\pgfpathlineto{\pgfqpoint{1.565499in}{3.376081in}}%
\pgfpathlineto{\pgfqpoint{1.566485in}{2.929312in}}%
\pgfpathlineto{\pgfqpoint{1.568458in}{1.839081in}}%
\pgfpathlineto{\pgfqpoint{1.570431in}{3.367193in}}%
\pgfpathlineto{\pgfqpoint{1.573391in}{3.367193in}}%
\pgfpathlineto{\pgfqpoint{1.574870in}{3.370127in}}%
\pgfpathlineto{\pgfqpoint{1.576843in}{3.370127in}}%
\pgfpathlineto{\pgfqpoint{1.577336in}{2.825685in}}%
\pgfpathlineto{\pgfqpoint{1.578323in}{2.860189in}}%
\pgfpathlineto{\pgfqpoint{1.579309in}{3.335258in}}%
\pgfpathlineto{\pgfqpoint{1.579803in}{3.199092in}}%
\pgfpathlineto{\pgfqpoint{1.580296in}{3.362920in}}%
\pgfpathlineto{\pgfqpoint{1.580789in}{3.362920in}}%
\pgfpathlineto{\pgfqpoint{1.582269in}{3.379250in}}%
\pgfpathlineto{\pgfqpoint{1.583748in}{3.379250in}}%
\pgfpathlineto{\pgfqpoint{1.585228in}{3.347188in}}%
\pgfpathlineto{\pgfqpoint{1.587201in}{3.347188in}}%
\pgfpathlineto{\pgfqpoint{1.588188in}{2.839905in}}%
\pgfpathlineto{\pgfqpoint{1.589667in}{3.283110in}}%
\pgfpathlineto{\pgfqpoint{1.591147in}{3.283110in}}%
\pgfpathlineto{\pgfqpoint{1.592627in}{3.250692in}}%
\pgfpathlineto{\pgfqpoint{1.593613in}{3.250692in}}%
\pgfpathlineto{\pgfqpoint{1.595093in}{3.359436in}}%
\pgfpathlineto{\pgfqpoint{1.597066in}{3.359436in}}%
\pgfpathlineto{\pgfqpoint{1.598052in}{3.087664in}}%
\pgfpathlineto{\pgfqpoint{1.599039in}{3.373041in}}%
\pgfpathlineto{\pgfqpoint{1.599532in}{3.367937in}}%
\pgfpathlineto{\pgfqpoint{1.601505in}{3.367937in}}%
\pgfpathlineto{\pgfqpoint{1.602491in}{2.453160in}}%
\pgfpathlineto{\pgfqpoint{1.602985in}{3.378779in}}%
\pgfpathlineto{\pgfqpoint{1.603971in}{3.217895in}}%
\pgfpathlineto{\pgfqpoint{1.604464in}{3.217895in}}%
\pgfpathlineto{\pgfqpoint{1.605944in}{3.332565in}}%
\pgfpathlineto{\pgfqpoint{1.606930in}{3.332565in}}%
\pgfpathlineto{\pgfqpoint{1.607917in}{3.210816in}}%
\pgfpathlineto{\pgfqpoint{1.608410in}{3.228492in}}%
\pgfpathlineto{\pgfqpoint{1.609890in}{3.228492in}}%
\pgfpathlineto{\pgfqpoint{1.611369in}{3.296373in}}%
\pgfpathlineto{\pgfqpoint{1.613836in}{3.296373in}}%
\pgfpathlineto{\pgfqpoint{1.614329in}{3.134748in}}%
\pgfpathlineto{\pgfqpoint{1.615809in}{3.377886in}}%
\pgfpathlineto{\pgfqpoint{1.616795in}{3.268749in}}%
\pgfpathlineto{\pgfqpoint{1.618275in}{3.378119in}}%
\pgfpathlineto{\pgfqpoint{1.618768in}{3.302281in}}%
\pgfpathlineto{\pgfqpoint{1.620248in}{2.857099in}}%
\pgfpathlineto{\pgfqpoint{1.621727in}{3.179775in}}%
\pgfpathlineto{\pgfqpoint{1.623207in}{2.635167in}}%
\pgfpathlineto{\pgfqpoint{1.624687in}{3.327275in}}%
\pgfpathlineto{\pgfqpoint{1.626166in}{3.375634in}}%
\pgfpathlineto{\pgfqpoint{1.626660in}{3.375634in}}%
\pgfpathlineto{\pgfqpoint{1.628139in}{2.925332in}}%
\pgfpathlineto{\pgfqpoint{1.628633in}{2.925332in}}%
\pgfpathlineto{\pgfqpoint{1.629126in}{3.376686in}}%
\pgfpathlineto{\pgfqpoint{1.629619in}{3.245507in}}%
\pgfpathlineto{\pgfqpoint{1.630112in}{3.245507in}}%
\pgfpathlineto{\pgfqpoint{1.631592in}{3.356867in}}%
\pgfpathlineto{\pgfqpoint{1.632085in}{3.356867in}}%
\pgfpathlineto{\pgfqpoint{1.632578in}{3.379985in}}%
\pgfpathlineto{\pgfqpoint{1.634058in}{3.204090in}}%
\pgfpathlineto{\pgfqpoint{1.635045in}{3.204090in}}%
\pgfpathlineto{\pgfqpoint{1.635538in}{3.194046in}}%
\pgfpathlineto{\pgfqpoint{1.637017in}{3.343762in}}%
\pgfpathlineto{\pgfqpoint{1.638497in}{3.328032in}}%
\pgfpathlineto{\pgfqpoint{1.641457in}{3.328032in}}%
\pgfpathlineto{\pgfqpoint{1.642936in}{2.764684in}}%
\pgfpathlineto{\pgfqpoint{1.643429in}{2.764684in}}%
\pgfpathlineto{\pgfqpoint{1.644909in}{3.370634in}}%
\pgfpathlineto{\pgfqpoint{1.646389in}{3.370634in}}%
\pgfpathlineto{\pgfqpoint{1.646882in}{3.234377in}}%
\pgfpathlineto{\pgfqpoint{1.647375in}{3.365377in}}%
\pgfpathlineto{\pgfqpoint{1.647869in}{3.346586in}}%
\pgfpathlineto{\pgfqpoint{1.648362in}{3.063732in}}%
\pgfpathlineto{\pgfqpoint{1.649348in}{3.120134in}}%
\pgfpathlineto{\pgfqpoint{1.650828in}{3.219506in}}%
\pgfpathlineto{\pgfqpoint{1.651814in}{3.221737in}}%
\pgfpathlineto{\pgfqpoint{1.653294in}{3.370495in}}%
\pgfpathlineto{\pgfqpoint{1.655760in}{2.663456in}}%
\pgfpathlineto{\pgfqpoint{1.657240in}{2.902406in}}%
\pgfpathlineto{\pgfqpoint{1.659213in}{2.902406in}}%
\pgfpathlineto{\pgfqpoint{1.659706in}{2.909067in}}%
\pgfpathlineto{\pgfqpoint{1.661186in}{3.293467in}}%
\pgfpathlineto{\pgfqpoint{1.661679in}{2.786646in}}%
\pgfpathlineto{\pgfqpoint{1.662172in}{3.029331in}}%
\pgfpathlineto{\pgfqpoint{1.664145in}{3.345471in}}%
\pgfpathlineto{\pgfqpoint{1.665625in}{2.831395in}}%
\pgfpathlineto{\pgfqpoint{1.666118in}{2.831395in}}%
\pgfpathlineto{\pgfqpoint{1.667598in}{3.287574in}}%
\pgfpathlineto{\pgfqpoint{1.668584in}{3.272091in}}%
\pgfpathlineto{\pgfqpoint{1.669571in}{3.343693in}}%
\pgfpathlineto{\pgfqpoint{1.671050in}{3.032836in}}%
\pgfpathlineto{\pgfqpoint{1.672037in}{3.369600in}}%
\pgfpathlineto{\pgfqpoint{1.672530in}{3.294605in}}%
\pgfpathlineto{\pgfqpoint{1.673023in}{3.294605in}}%
\pgfpathlineto{\pgfqpoint{1.675489in}{3.339289in}}%
\pgfpathlineto{\pgfqpoint{1.676476in}{2.828353in}}%
\pgfpathlineto{\pgfqpoint{1.677462in}{3.129388in}}%
\pgfpathlineto{\pgfqpoint{1.678942in}{2.950745in}}%
\pgfpathlineto{\pgfqpoint{1.679929in}{2.950745in}}%
\pgfpathlineto{\pgfqpoint{1.680422in}{3.033779in}}%
\pgfpathlineto{\pgfqpoint{1.681408in}{2.922580in}}%
\pgfpathlineto{\pgfqpoint{1.682395in}{3.363436in}}%
\pgfpathlineto{\pgfqpoint{1.682888in}{3.223103in}}%
\pgfpathlineto{\pgfqpoint{1.683874in}{3.242032in}}%
\pgfpathlineto{\pgfqpoint{1.684368in}{3.161346in}}%
\pgfpathlineto{\pgfqpoint{1.684861in}{3.343740in}}%
\pgfpathlineto{\pgfqpoint{1.685847in}{3.321398in}}%
\pgfpathlineto{\pgfqpoint{1.687327in}{3.374889in}}%
\pgfpathlineto{\pgfqpoint{1.688313in}{3.363077in}}%
\pgfpathlineto{\pgfqpoint{1.688807in}{3.180988in}}%
\pgfpathlineto{\pgfqpoint{1.689300in}{3.300052in}}%
\pgfpathlineto{\pgfqpoint{1.692753in}{3.300052in}}%
\pgfpathlineto{\pgfqpoint{1.693739in}{3.048995in}}%
\pgfpathlineto{\pgfqpoint{1.695219in}{3.299947in}}%
\pgfpathlineto{\pgfqpoint{1.696205in}{3.299947in}}%
\pgfpathlineto{\pgfqpoint{1.696698in}{3.244274in}}%
\pgfpathlineto{\pgfqpoint{1.697192in}{3.283363in}}%
\pgfpathlineto{\pgfqpoint{1.698671in}{3.283363in}}%
\pgfpathlineto{\pgfqpoint{1.700151in}{3.117103in}}%
\pgfpathlineto{\pgfqpoint{1.701631in}{3.370256in}}%
\pgfpathlineto{\pgfqpoint{1.702124in}{3.370256in}}%
\pgfpathlineto{\pgfqpoint{1.702617in}{2.309441in}}%
\pgfpathlineto{\pgfqpoint{1.703110in}{3.325963in}}%
\pgfpathlineto{\pgfqpoint{1.704097in}{3.325963in}}%
\pgfpathlineto{\pgfqpoint{1.705577in}{3.109296in}}%
\pgfpathlineto{\pgfqpoint{1.707550in}{3.367194in}}%
\pgfpathlineto{\pgfqpoint{1.709522in}{3.126030in}}%
\pgfpathlineto{\pgfqpoint{1.711002in}{3.126030in}}%
\pgfpathlineto{\pgfqpoint{1.712482in}{3.307203in}}%
\pgfpathlineto{\pgfqpoint{1.713468in}{3.307203in}}%
\pgfpathlineto{\pgfqpoint{1.713962in}{2.924123in}}%
\pgfpathlineto{\pgfqpoint{1.714948in}{2.981999in}}%
\pgfpathlineto{\pgfqpoint{1.715441in}{3.151385in}}%
\pgfpathlineto{\pgfqpoint{1.715934in}{3.049810in}}%
\pgfpathlineto{\pgfqpoint{1.716921in}{3.049810in}}%
\pgfpathlineto{\pgfqpoint{1.717414in}{3.350613in}}%
\pgfpathlineto{\pgfqpoint{1.717907in}{2.765808in}}%
\pgfpathlineto{\pgfqpoint{1.718401in}{3.177611in}}%
\pgfpathlineto{\pgfqpoint{1.718894in}{3.177611in}}%
\pgfpathlineto{\pgfqpoint{1.719880in}{3.351770in}}%
\pgfpathlineto{\pgfqpoint{1.720867in}{3.256554in}}%
\pgfpathlineto{\pgfqpoint{1.721360in}{3.324503in}}%
\pgfpathlineto{\pgfqpoint{1.723333in}{2.901399in}}%
\pgfpathlineto{\pgfqpoint{1.723826in}{2.270522in}}%
\pgfpathlineto{\pgfqpoint{1.724813in}{2.347920in}}%
\pgfpathlineto{\pgfqpoint{1.725306in}{2.347920in}}%
\pgfpathlineto{\pgfqpoint{1.727772in}{3.372936in}}%
\pgfpathlineto{\pgfqpoint{1.729252in}{3.365096in}}%
\pgfpathlineto{\pgfqpoint{1.730238in}{3.365096in}}%
\pgfpathlineto{\pgfqpoint{1.730731in}{2.681010in}}%
\pgfpathlineto{\pgfqpoint{1.731225in}{3.357905in}}%
\pgfpathlineto{\pgfqpoint{1.731718in}{3.357905in}}%
\pgfpathlineto{\pgfqpoint{1.732704in}{3.064821in}}%
\pgfpathlineto{\pgfqpoint{1.734184in}{3.371100in}}%
\pgfpathlineto{\pgfqpoint{1.735664in}{3.371100in}}%
\pgfpathlineto{\pgfqpoint{1.737143in}{3.361087in}}%
\pgfpathlineto{\pgfqpoint{1.739116in}{3.361087in}}%
\pgfpathlineto{\pgfqpoint{1.739610in}{3.168078in}}%
\pgfpathlineto{\pgfqpoint{1.740103in}{3.224432in}}%
\pgfpathlineto{\pgfqpoint{1.741089in}{3.224432in}}%
\pgfpathlineto{\pgfqpoint{1.741582in}{3.157981in}}%
\pgfpathlineto{\pgfqpoint{1.743555in}{2.552170in}}%
\pgfpathlineto{\pgfqpoint{1.744049in}{3.347341in}}%
\pgfpathlineto{\pgfqpoint{1.745035in}{3.344049in}}%
\pgfpathlineto{\pgfqpoint{1.746022in}{3.252784in}}%
\pgfpathlineto{\pgfqpoint{1.747501in}{3.379929in}}%
\pgfpathlineto{\pgfqpoint{1.748488in}{3.255452in}}%
\pgfpathlineto{\pgfqpoint{1.749967in}{3.344210in}}%
\pgfpathlineto{\pgfqpoint{1.750461in}{3.344210in}}%
\pgfpathlineto{\pgfqpoint{1.750954in}{3.010882in}}%
\pgfpathlineto{\pgfqpoint{1.751447in}{3.376398in}}%
\pgfpathlineto{\pgfqpoint{1.754900in}{3.376580in}}%
\pgfpathlineto{\pgfqpoint{1.756379in}{3.331153in}}%
\pgfpathlineto{\pgfqpoint{1.757366in}{3.331153in}}%
\pgfpathlineto{\pgfqpoint{1.757859in}{3.375569in}}%
\pgfpathlineto{\pgfqpoint{1.758352in}{3.342761in}}%
\pgfpathlineto{\pgfqpoint{1.759339in}{3.342761in}}%
\pgfpathlineto{\pgfqpoint{1.759832in}{3.162819in}}%
\pgfpathlineto{\pgfqpoint{1.760325in}{3.366863in}}%
\pgfpathlineto{\pgfqpoint{1.760818in}{3.366863in}}%
\pgfpathlineto{\pgfqpoint{1.762298in}{3.363070in}}%
\pgfpathlineto{\pgfqpoint{1.764764in}{3.362984in}}%
\pgfpathlineto{\pgfqpoint{1.765751in}{3.064639in}}%
\pgfpathlineto{\pgfqpoint{1.767230in}{3.169538in}}%
\pgfpathlineto{\pgfqpoint{1.767724in}{3.199885in}}%
\pgfpathlineto{\pgfqpoint{1.768217in}{3.377130in}}%
\pgfpathlineto{\pgfqpoint{1.768710in}{3.238945in}}%
\pgfpathlineto{\pgfqpoint{1.769203in}{3.238945in}}%
\pgfpathlineto{\pgfqpoint{1.770190in}{3.234373in}}%
\pgfpathlineto{\pgfqpoint{1.771176in}{3.332076in}}%
\pgfpathlineto{\pgfqpoint{1.773149in}{2.906235in}}%
\pgfpathlineto{\pgfqpoint{1.774136in}{3.379020in}}%
\pgfpathlineto{\pgfqpoint{1.775615in}{2.907150in}}%
\pgfpathlineto{\pgfqpoint{1.776109in}{3.368345in}}%
\pgfpathlineto{\pgfqpoint{1.776602in}{2.699725in}}%
\pgfpathlineto{\pgfqpoint{1.777095in}{3.217660in}}%
\pgfpathlineto{\pgfqpoint{1.778575in}{3.379016in}}%
\pgfpathlineto{\pgfqpoint{1.781041in}{3.379882in}}%
\pgfpathlineto{\pgfqpoint{1.781534in}{3.281508in}}%
\pgfpathlineto{\pgfqpoint{1.782027in}{3.372561in}}%
\pgfpathlineto{\pgfqpoint{1.784000in}{3.347463in}}%
\pgfpathlineto{\pgfqpoint{1.786466in}{3.347463in}}%
\pgfpathlineto{\pgfqpoint{1.787946in}{3.339496in}}%
\pgfpathlineto{\pgfqpoint{1.788439in}{3.339496in}}%
\pgfpathlineto{\pgfqpoint{1.789919in}{2.629619in}}%
\pgfpathlineto{\pgfqpoint{1.790412in}{2.629619in}}%
\pgfpathlineto{\pgfqpoint{1.791892in}{3.376599in}}%
\pgfpathlineto{\pgfqpoint{1.793372in}{3.376599in}}%
\pgfpathlineto{\pgfqpoint{1.794358in}{3.358982in}}%
\pgfpathlineto{\pgfqpoint{1.796331in}{1.792112in}}%
\pgfpathlineto{\pgfqpoint{1.798797in}{3.374553in}}%
\pgfpathlineto{\pgfqpoint{1.800277in}{3.363922in}}%
\pgfpathlineto{\pgfqpoint{1.800770in}{3.363922in}}%
\pgfpathlineto{\pgfqpoint{1.802250in}{3.290287in}}%
\pgfpathlineto{\pgfqpoint{1.803236in}{3.290287in}}%
\pgfpathlineto{\pgfqpoint{1.803730in}{3.293116in}}%
\pgfpathlineto{\pgfqpoint{1.805209in}{3.379980in}}%
\pgfpathlineto{\pgfqpoint{1.805703in}{3.234016in}}%
\pgfpathlineto{\pgfqpoint{1.806196in}{3.302514in}}%
\pgfpathlineto{\pgfqpoint{1.808169in}{3.302514in}}%
\pgfpathlineto{\pgfqpoint{1.809155in}{3.379899in}}%
\pgfpathlineto{\pgfqpoint{1.811128in}{3.054060in}}%
\pgfpathlineto{\pgfqpoint{1.811621in}{3.356091in}}%
\pgfpathlineto{\pgfqpoint{1.812115in}{3.112680in}}%
\pgfpathlineto{\pgfqpoint{1.813101in}{3.123756in}}%
\pgfpathlineto{\pgfqpoint{1.813594in}{3.199836in}}%
\pgfpathlineto{\pgfqpoint{1.814087in}{3.156938in}}%
\pgfpathlineto{\pgfqpoint{1.814581in}{2.930289in}}%
\pgfpathlineto{\pgfqpoint{1.816060in}{3.376647in}}%
\pgfpathlineto{\pgfqpoint{1.820006in}{3.376647in}}%
\pgfpathlineto{\pgfqpoint{1.821486in}{2.411265in}}%
\pgfpathlineto{\pgfqpoint{1.821979in}{2.725138in}}%
\pgfpathlineto{\pgfqpoint{1.822472in}{2.725138in}}%
\pgfpathlineto{\pgfqpoint{1.823459in}{3.059358in}}%
\pgfpathlineto{\pgfqpoint{1.823952in}{2.797078in}}%
\pgfpathlineto{\pgfqpoint{1.825432in}{3.374510in}}%
\pgfpathlineto{\pgfqpoint{1.826911in}{3.379914in}}%
\pgfpathlineto{\pgfqpoint{1.827405in}{3.333845in}}%
\pgfpathlineto{\pgfqpoint{1.828391in}{3.032284in}}%
\pgfpathlineto{\pgfqpoint{1.828884in}{3.045280in}}%
\pgfpathlineto{\pgfqpoint{1.830364in}{3.312409in}}%
\pgfpathlineto{\pgfqpoint{1.830857in}{3.144571in}}%
\pgfpathlineto{\pgfqpoint{1.831351in}{3.322032in}}%
\pgfpathlineto{\pgfqpoint{1.832337in}{3.307364in}}%
\pgfpathlineto{\pgfqpoint{1.833817in}{3.373335in}}%
\pgfpathlineto{\pgfqpoint{1.834310in}{3.240124in}}%
\pgfpathlineto{\pgfqpoint{1.834803in}{2.574317in}}%
\pgfpathlineto{\pgfqpoint{1.835296in}{3.356505in}}%
\pgfpathlineto{\pgfqpoint{1.838749in}{3.356505in}}%
\pgfpathlineto{\pgfqpoint{1.839242in}{2.558311in}}%
\pgfpathlineto{\pgfqpoint{1.839735in}{3.258123in}}%
\pgfpathlineto{\pgfqpoint{1.841708in}{3.258123in}}%
\pgfpathlineto{\pgfqpoint{1.843188in}{3.357071in}}%
\pgfpathlineto{\pgfqpoint{1.843681in}{3.357071in}}%
\pgfpathlineto{\pgfqpoint{1.845161in}{3.106907in}}%
\pgfpathlineto{\pgfqpoint{1.846147in}{3.026234in}}%
\pgfpathlineto{\pgfqpoint{1.847627in}{3.229967in}}%
\pgfpathlineto{\pgfqpoint{1.848120in}{3.229967in}}%
\pgfpathlineto{\pgfqpoint{1.849107in}{2.380633in}}%
\pgfpathlineto{\pgfqpoint{1.850587in}{3.376352in}}%
\pgfpathlineto{\pgfqpoint{1.852066in}{3.045852in}}%
\pgfpathlineto{\pgfqpoint{1.852559in}{3.045852in}}%
\pgfpathlineto{\pgfqpoint{1.854039in}{3.298461in}}%
\pgfpathlineto{\pgfqpoint{1.855519in}{3.347332in}}%
\pgfpathlineto{\pgfqpoint{1.856505in}{3.347332in}}%
\pgfpathlineto{\pgfqpoint{1.857492in}{3.062239in}}%
\pgfpathlineto{\pgfqpoint{1.857985in}{3.369943in}}%
\pgfpathlineto{\pgfqpoint{1.858478in}{3.237347in}}%
\pgfpathlineto{\pgfqpoint{1.860944in}{3.237347in}}%
\pgfpathlineto{\pgfqpoint{1.861931in}{3.076363in}}%
\pgfpathlineto{\pgfqpoint{1.862917in}{2.112787in}}%
\pgfpathlineto{\pgfqpoint{1.864397in}{3.368393in}}%
\pgfpathlineto{\pgfqpoint{1.864890in}{3.368393in}}%
\pgfpathlineto{\pgfqpoint{1.865877in}{2.960391in}}%
\pgfpathlineto{\pgfqpoint{1.866370in}{3.053634in}}%
\pgfpathlineto{\pgfqpoint{1.867850in}{3.369761in}}%
\pgfpathlineto{\pgfqpoint{1.868343in}{3.369761in}}%
\pgfpathlineto{\pgfqpoint{1.869329in}{2.795381in}}%
\pgfpathlineto{\pgfqpoint{1.869823in}{3.112524in}}%
\pgfpathlineto{\pgfqpoint{1.870316in}{3.112524in}}%
\pgfpathlineto{\pgfqpoint{1.870809in}{2.784333in}}%
\pgfpathlineto{\pgfqpoint{1.871302in}{2.913152in}}%
\pgfpathlineto{\pgfqpoint{1.872782in}{3.191502in}}%
\pgfpathlineto{\pgfqpoint{1.874262in}{3.191502in}}%
\pgfpathlineto{\pgfqpoint{1.875741in}{3.379954in}}%
\pgfpathlineto{\pgfqpoint{1.879194in}{3.379954in}}%
\pgfpathlineto{\pgfqpoint{1.881167in}{3.103268in}}%
\pgfpathlineto{\pgfqpoint{1.881660in}{3.103268in}}%
\pgfpathlineto{\pgfqpoint{1.883140in}{3.364521in}}%
\pgfpathlineto{\pgfqpoint{1.883633in}{3.364521in}}%
\pgfpathlineto{\pgfqpoint{1.884126in}{2.794962in}}%
\pgfpathlineto{\pgfqpoint{1.884619in}{3.352073in}}%
\pgfpathlineto{\pgfqpoint{1.891031in}{3.352073in}}%
\pgfpathlineto{\pgfqpoint{1.891525in}{3.162229in}}%
\pgfpathlineto{\pgfqpoint{1.892018in}{3.214953in}}%
\pgfpathlineto{\pgfqpoint{1.893004in}{3.214953in}}%
\pgfpathlineto{\pgfqpoint{1.893991in}{3.378308in}}%
\pgfpathlineto{\pgfqpoint{1.895471in}{2.826288in}}%
\pgfpathlineto{\pgfqpoint{1.896950in}{3.347716in}}%
\pgfpathlineto{\pgfqpoint{1.898923in}{3.347716in}}%
\pgfpathlineto{\pgfqpoint{1.899416in}{3.239786in}}%
\pgfpathlineto{\pgfqpoint{1.900403in}{3.379995in}}%
\pgfpathlineto{\pgfqpoint{1.900896in}{3.171934in}}%
\pgfpathlineto{\pgfqpoint{1.901389in}{3.369183in}}%
\pgfpathlineto{\pgfqpoint{1.902376in}{3.369183in}}%
\pgfpathlineto{\pgfqpoint{1.903855in}{3.358790in}}%
\pgfpathlineto{\pgfqpoint{1.905335in}{3.379995in}}%
\pgfpathlineto{\pgfqpoint{1.907308in}{3.273349in}}%
\pgfpathlineto{\pgfqpoint{1.908295in}{2.269702in}}%
\pgfpathlineto{\pgfqpoint{1.910268in}{3.377637in}}%
\pgfpathlineto{\pgfqpoint{1.910761in}{3.377637in}}%
\pgfpathlineto{\pgfqpoint{1.912240in}{3.111235in}}%
\pgfpathlineto{\pgfqpoint{1.912734in}{3.111235in}}%
\pgfpathlineto{\pgfqpoint{1.914213in}{3.213492in}}%
\pgfpathlineto{\pgfqpoint{1.914707in}{3.213492in}}%
\pgfpathlineto{\pgfqpoint{1.915200in}{2.739290in}}%
\pgfpathlineto{\pgfqpoint{1.915693in}{3.290714in}}%
\pgfpathlineto{\pgfqpoint{1.916186in}{3.290714in}}%
\pgfpathlineto{\pgfqpoint{1.917666in}{3.195329in}}%
\pgfpathlineto{\pgfqpoint{1.918159in}{3.375853in}}%
\pgfpathlineto{\pgfqpoint{1.919146in}{3.344959in}}%
\pgfpathlineto{\pgfqpoint{1.919639in}{3.344959in}}%
\pgfpathlineto{\pgfqpoint{1.920625in}{3.211199in}}%
\pgfpathlineto{\pgfqpoint{1.922105in}{3.370923in}}%
\pgfpathlineto{\pgfqpoint{1.923092in}{3.069548in}}%
\pgfpathlineto{\pgfqpoint{1.924571in}{3.369692in}}%
\pgfpathlineto{\pgfqpoint{1.925064in}{3.369692in}}%
\pgfpathlineto{\pgfqpoint{1.926544in}{3.379110in}}%
\pgfpathlineto{\pgfqpoint{1.927531in}{3.379110in}}%
\pgfpathlineto{\pgfqpoint{1.928024in}{2.788599in}}%
\pgfpathlineto{\pgfqpoint{1.929010in}{2.918781in}}%
\pgfpathlineto{\pgfqpoint{1.929504in}{2.918781in}}%
\pgfpathlineto{\pgfqpoint{1.929997in}{2.778001in}}%
\pgfpathlineto{\pgfqpoint{1.931476in}{3.316451in}}%
\pgfpathlineto{\pgfqpoint{1.931970in}{2.145626in}}%
\pgfpathlineto{\pgfqpoint{1.932463in}{3.375971in}}%
\pgfpathlineto{\pgfqpoint{1.932956in}{3.375971in}}%
\pgfpathlineto{\pgfqpoint{1.933943in}{2.667723in}}%
\pgfpathlineto{\pgfqpoint{1.934436in}{3.322765in}}%
\pgfpathlineto{\pgfqpoint{1.934929in}{2.854478in}}%
\pgfpathlineto{\pgfqpoint{1.937888in}{3.302256in}}%
\pgfpathlineto{\pgfqpoint{1.938875in}{3.056115in}}%
\pgfpathlineto{\pgfqpoint{1.939368in}{3.368815in}}%
\pgfpathlineto{\pgfqpoint{1.940355in}{3.363971in}}%
\pgfpathlineto{\pgfqpoint{1.945287in}{3.363971in}}%
\pgfpathlineto{\pgfqpoint{1.945780in}{2.584926in}}%
\pgfpathlineto{\pgfqpoint{1.946273in}{3.270732in}}%
\pgfpathlineto{\pgfqpoint{1.946767in}{3.074825in}}%
\pgfpathlineto{\pgfqpoint{1.947260in}{3.284229in}}%
\pgfpathlineto{\pgfqpoint{1.947753in}{3.284229in}}%
\pgfpathlineto{\pgfqpoint{1.949726in}{3.361555in}}%
\pgfpathlineto{\pgfqpoint{1.951206in}{3.007750in}}%
\pgfpathlineto{\pgfqpoint{1.952685in}{3.358418in}}%
\pgfpathlineto{\pgfqpoint{1.958604in}{3.358418in}}%
\pgfpathlineto{\pgfqpoint{1.960084in}{3.004309in}}%
\pgfpathlineto{\pgfqpoint{1.960577in}{3.004309in}}%
\pgfpathlineto{\pgfqpoint{1.962057in}{3.191468in}}%
\pgfpathlineto{\pgfqpoint{1.963536in}{3.102612in}}%
\pgfpathlineto{\pgfqpoint{1.964030in}{3.102612in}}%
\pgfpathlineto{\pgfqpoint{1.964523in}{3.293879in}}%
\pgfpathlineto{\pgfqpoint{1.965016in}{2.972911in}}%
\pgfpathlineto{\pgfqpoint{1.965509in}{3.191745in}}%
\pgfpathlineto{\pgfqpoint{1.966003in}{3.191745in}}%
\pgfpathlineto{\pgfqpoint{1.967482in}{3.292045in}}%
\pgfpathlineto{\pgfqpoint{1.968962in}{3.274746in}}%
\pgfpathlineto{\pgfqpoint{1.971428in}{3.274746in}}%
\pgfpathlineto{\pgfqpoint{1.971921in}{2.134800in}}%
\pgfpathlineto{\pgfqpoint{1.972415in}{3.213830in}}%
\pgfpathlineto{\pgfqpoint{1.975374in}{3.213830in}}%
\pgfpathlineto{\pgfqpoint{1.975867in}{2.407190in}}%
\pgfpathlineto{\pgfqpoint{1.976360in}{2.904262in}}%
\pgfpathlineto{\pgfqpoint{1.976854in}{2.904262in}}%
\pgfpathlineto{\pgfqpoint{1.977347in}{2.853721in}}%
\pgfpathlineto{\pgfqpoint{1.978333in}{3.377319in}}%
\pgfpathlineto{\pgfqpoint{1.979320in}{3.175525in}}%
\pgfpathlineto{\pgfqpoint{1.979813in}{3.210795in}}%
\pgfpathlineto{\pgfqpoint{1.980800in}{3.062565in}}%
\pgfpathlineto{\pgfqpoint{1.981293in}{3.066796in}}%
\pgfpathlineto{\pgfqpoint{1.981786in}{3.066796in}}%
\pgfpathlineto{\pgfqpoint{1.983266in}{3.378573in}}%
\pgfpathlineto{\pgfqpoint{1.983759in}{3.378573in}}%
\pgfpathlineto{\pgfqpoint{1.984252in}{3.304066in}}%
\pgfpathlineto{\pgfqpoint{1.984745in}{3.379051in}}%
\pgfpathlineto{\pgfqpoint{1.985239in}{3.379051in}}%
\pgfpathlineto{\pgfqpoint{1.986718in}{3.366966in}}%
\pgfpathlineto{\pgfqpoint{1.987212in}{3.366966in}}%
\pgfpathlineto{\pgfqpoint{1.989184in}{3.125791in}}%
\pgfpathlineto{\pgfqpoint{1.991157in}{3.369813in}}%
\pgfpathlineto{\pgfqpoint{1.991651in}{3.369813in}}%
\pgfpathlineto{\pgfqpoint{1.993130in}{3.376875in}}%
\pgfpathlineto{\pgfqpoint{1.994117in}{3.376875in}}%
\pgfpathlineto{\pgfqpoint{1.995596in}{2.560520in}}%
\pgfpathlineto{\pgfqpoint{1.996090in}{2.560520in}}%
\pgfpathlineto{\pgfqpoint{1.997569in}{2.968196in}}%
\pgfpathlineto{\pgfqpoint{1.998556in}{2.968196in}}%
\pgfpathlineto{\pgfqpoint{2.000036in}{3.377763in}}%
\pgfpathlineto{\pgfqpoint{2.000529in}{3.377763in}}%
\pgfpathlineto{\pgfqpoint{2.002008in}{3.376510in}}%
\pgfpathlineto{\pgfqpoint{2.002502in}{3.376510in}}%
\pgfpathlineto{\pgfqpoint{2.003981in}{3.328278in}}%
\pgfpathlineto{\pgfqpoint{2.004475in}{3.328278in}}%
\pgfpathlineto{\pgfqpoint{2.004968in}{3.353633in}}%
\pgfpathlineto{\pgfqpoint{2.005461in}{3.341991in}}%
\pgfpathlineto{\pgfqpoint{2.009900in}{3.341991in}}%
\pgfpathlineto{\pgfqpoint{2.011380in}{3.320291in}}%
\pgfpathlineto{\pgfqpoint{2.014339in}{3.320291in}}%
\pgfpathlineto{\pgfqpoint{2.015326in}{2.754303in}}%
\pgfpathlineto{\pgfqpoint{2.017792in}{3.379964in}}%
\pgfpathlineto{\pgfqpoint{2.023217in}{3.379969in}}%
\pgfpathlineto{\pgfqpoint{2.024204in}{3.369035in}}%
\pgfpathlineto{\pgfqpoint{2.025684in}{3.373662in}}%
\pgfpathlineto{\pgfqpoint{2.027163in}{3.215047in}}%
\pgfpathlineto{\pgfqpoint{2.028643in}{3.348963in}}%
\pgfpathlineto{\pgfqpoint{2.029629in}{3.348963in}}%
\pgfpathlineto{\pgfqpoint{2.030123in}{3.185669in}}%
\pgfpathlineto{\pgfqpoint{2.031109in}{3.214760in}}%
\pgfpathlineto{\pgfqpoint{2.032096in}{3.214760in}}%
\pgfpathlineto{\pgfqpoint{2.034069in}{2.602337in}}%
\pgfpathlineto{\pgfqpoint{2.034562in}{3.363173in}}%
\pgfpathlineto{\pgfqpoint{2.035055in}{3.074328in}}%
\pgfpathlineto{\pgfqpoint{2.037028in}{3.074328in}}%
\pgfpathlineto{\pgfqpoint{2.038508in}{3.366948in}}%
\pgfpathlineto{\pgfqpoint{2.039987in}{3.070919in}}%
\pgfpathlineto{\pgfqpoint{2.040481in}{3.070919in}}%
\pgfpathlineto{\pgfqpoint{2.040974in}{2.933812in}}%
\pgfpathlineto{\pgfqpoint{2.042453in}{3.374743in}}%
\pgfpathlineto{\pgfqpoint{2.043933in}{3.348506in}}%
\pgfpathlineto{\pgfqpoint{2.044426in}{3.348506in}}%
\pgfpathlineto{\pgfqpoint{2.045906in}{3.369121in}}%
\pgfpathlineto{\pgfqpoint{2.046399in}{3.369121in}}%
\pgfpathlineto{\pgfqpoint{2.046893in}{3.236234in}}%
\pgfpathlineto{\pgfqpoint{2.047386in}{3.379646in}}%
\pgfpathlineto{\pgfqpoint{2.048865in}{3.379646in}}%
\pgfpathlineto{\pgfqpoint{2.050345in}{3.341829in}}%
\pgfpathlineto{\pgfqpoint{2.050838in}{3.341829in}}%
\pgfpathlineto{\pgfqpoint{2.052318in}{2.946766in}}%
\pgfpathlineto{\pgfqpoint{2.052811in}{3.305106in}}%
\pgfpathlineto{\pgfqpoint{2.053305in}{2.941647in}}%
\pgfpathlineto{\pgfqpoint{2.053798in}{2.941647in}}%
\pgfpathlineto{\pgfqpoint{2.055771in}{3.379615in}}%
\pgfpathlineto{\pgfqpoint{2.057744in}{2.578431in}}%
\pgfpathlineto{\pgfqpoint{2.058730in}{2.578431in}}%
\pgfpathlineto{\pgfqpoint{2.060210in}{3.378875in}}%
\pgfpathlineto{\pgfqpoint{2.061689in}{3.378875in}}%
\pgfpathlineto{\pgfqpoint{2.063169in}{2.985946in}}%
\pgfpathlineto{\pgfqpoint{2.065635in}{2.985946in}}%
\pgfpathlineto{\pgfqpoint{2.067115in}{3.350001in}}%
\pgfpathlineto{\pgfqpoint{2.068595in}{3.048588in}}%
\pgfpathlineto{\pgfqpoint{2.069088in}{3.378146in}}%
\pgfpathlineto{\pgfqpoint{2.070074in}{3.353676in}}%
\pgfpathlineto{\pgfqpoint{2.071554in}{3.340206in}}%
\pgfpathlineto{\pgfqpoint{2.073034in}{3.340206in}}%
\pgfpathlineto{\pgfqpoint{2.073527in}{3.240152in}}%
\pgfpathlineto{\pgfqpoint{2.074020in}{3.306535in}}%
\pgfpathlineto{\pgfqpoint{2.075007in}{3.306535in}}%
\pgfpathlineto{\pgfqpoint{2.075500in}{3.335689in}}%
\pgfpathlineto{\pgfqpoint{2.077473in}{3.014921in}}%
\pgfpathlineto{\pgfqpoint{2.078953in}{3.014921in}}%
\pgfpathlineto{\pgfqpoint{2.079939in}{3.307357in}}%
\pgfpathlineto{\pgfqpoint{2.080432in}{2.256348in}}%
\pgfpathlineto{\pgfqpoint{2.080925in}{2.593153in}}%
\pgfpathlineto{\pgfqpoint{2.081419in}{2.593153in}}%
\pgfpathlineto{\pgfqpoint{2.081912in}{2.204843in}}%
\pgfpathlineto{\pgfqpoint{2.083392in}{3.350067in}}%
\pgfpathlineto{\pgfqpoint{2.084871in}{3.350067in}}%
\pgfpathlineto{\pgfqpoint{2.085365in}{3.291434in}}%
\pgfpathlineto{\pgfqpoint{2.085858in}{3.369817in}}%
\pgfpathlineto{\pgfqpoint{2.086844in}{3.020782in}}%
\pgfpathlineto{\pgfqpoint{2.087337in}{3.251046in}}%
\pgfpathlineto{\pgfqpoint{2.087831in}{3.014839in}}%
\pgfpathlineto{\pgfqpoint{2.088324in}{3.014839in}}%
\pgfpathlineto{\pgfqpoint{2.091283in}{3.379732in}}%
\pgfpathlineto{\pgfqpoint{2.092270in}{3.379732in}}%
\pgfpathlineto{\pgfqpoint{2.093749in}{3.281547in}}%
\pgfpathlineto{\pgfqpoint{2.094243in}{3.281547in}}%
\pgfpathlineto{\pgfqpoint{2.094736in}{3.362591in}}%
\pgfpathlineto{\pgfqpoint{2.095229in}{3.283362in}}%
\pgfpathlineto{\pgfqpoint{2.095722in}{3.283362in}}%
\pgfpathlineto{\pgfqpoint{2.097202in}{3.375940in}}%
\pgfpathlineto{\pgfqpoint{2.101148in}{3.375940in}}%
\pgfpathlineto{\pgfqpoint{2.101641in}{3.358497in}}%
\pgfpathlineto{\pgfqpoint{2.103121in}{3.242588in}}%
\pgfpathlineto{\pgfqpoint{2.103614in}{3.242588in}}%
\pgfpathlineto{\pgfqpoint{2.104601in}{3.195686in}}%
\pgfpathlineto{\pgfqpoint{2.106080in}{3.378307in}}%
\pgfpathlineto{\pgfqpoint{2.106573in}{3.310837in}}%
\pgfpathlineto{\pgfqpoint{2.107067in}{3.376612in}}%
\pgfpathlineto{\pgfqpoint{2.108546in}{3.376612in}}%
\pgfpathlineto{\pgfqpoint{2.110026in}{3.121792in}}%
\pgfpathlineto{\pgfqpoint{2.111506in}{3.345791in}}%
\pgfpathlineto{\pgfqpoint{2.111999in}{3.345791in}}%
\pgfpathlineto{\pgfqpoint{2.112492in}{3.300259in}}%
\pgfpathlineto{\pgfqpoint{2.113972in}{3.375514in}}%
\pgfpathlineto{\pgfqpoint{2.114465in}{3.375514in}}%
\pgfpathlineto{\pgfqpoint{2.115945in}{3.207323in}}%
\pgfpathlineto{\pgfqpoint{2.117425in}{3.368601in}}%
\pgfpathlineto{\pgfqpoint{2.118411in}{3.368601in}}%
\pgfpathlineto{\pgfqpoint{2.119891in}{3.047214in}}%
\pgfpathlineto{\pgfqpoint{2.120384in}{3.047214in}}%
\pgfpathlineto{\pgfqpoint{2.120877in}{3.318897in}}%
\pgfpathlineto{\pgfqpoint{2.121370in}{3.001968in}}%
\pgfpathlineto{\pgfqpoint{2.121864in}{3.306057in}}%
\pgfpathlineto{\pgfqpoint{2.123837in}{3.306057in}}%
\pgfpathlineto{\pgfqpoint{2.125316in}{3.372555in}}%
\pgfpathlineto{\pgfqpoint{2.126303in}{3.372555in}}%
\pgfpathlineto{\pgfqpoint{2.127782in}{3.223274in}}%
\pgfpathlineto{\pgfqpoint{2.128769in}{3.223274in}}%
\pgfpathlineto{\pgfqpoint{2.130249in}{3.376372in}}%
\pgfpathlineto{\pgfqpoint{2.131235in}{3.364679in}}%
\pgfpathlineto{\pgfqpoint{2.133208in}{3.364679in}}%
\pgfpathlineto{\pgfqpoint{2.133701in}{3.206142in}}%
\pgfpathlineto{\pgfqpoint{2.134194in}{3.284233in}}%
\pgfpathlineto{\pgfqpoint{2.135181in}{3.284233in}}%
\pgfpathlineto{\pgfqpoint{2.135674in}{3.376727in}}%
\pgfpathlineto{\pgfqpoint{2.136167in}{3.279123in}}%
\pgfpathlineto{\pgfqpoint{2.141593in}{3.279123in}}%
\pgfpathlineto{\pgfqpoint{2.143073in}{3.359628in}}%
\pgfpathlineto{\pgfqpoint{2.143566in}{3.359628in}}%
\pgfpathlineto{\pgfqpoint{2.145046in}{3.277238in}}%
\pgfpathlineto{\pgfqpoint{2.145539in}{3.277238in}}%
\pgfpathlineto{\pgfqpoint{2.146032in}{3.287862in}}%
\pgfpathlineto{\pgfqpoint{2.147018in}{3.373598in}}%
\pgfpathlineto{\pgfqpoint{2.147512in}{3.063348in}}%
\pgfpathlineto{\pgfqpoint{2.148005in}{3.370820in}}%
\pgfpathlineto{\pgfqpoint{2.148498in}{3.370820in}}%
\pgfpathlineto{\pgfqpoint{2.149485in}{3.324061in}}%
\pgfpathlineto{\pgfqpoint{2.149978in}{2.220250in}}%
\pgfpathlineto{\pgfqpoint{2.150471in}{2.902925in}}%
\pgfpathlineto{\pgfqpoint{2.151951in}{3.379812in}}%
\pgfpathlineto{\pgfqpoint{2.153430in}{3.379812in}}%
\pgfpathlineto{\pgfqpoint{2.154910in}{3.170104in}}%
\pgfpathlineto{\pgfqpoint{2.156390in}{3.330252in}}%
\pgfpathlineto{\pgfqpoint{2.156883in}{3.141620in}}%
\pgfpathlineto{\pgfqpoint{2.157376in}{3.313671in}}%
\pgfpathlineto{\pgfqpoint{2.157870in}{1.672480in}}%
\pgfpathlineto{\pgfqpoint{2.158363in}{3.239505in}}%
\pgfpathlineto{\pgfqpoint{2.159349in}{3.239505in}}%
\pgfpathlineto{\pgfqpoint{2.160336in}{3.379349in}}%
\pgfpathlineto{\pgfqpoint{2.160829in}{3.371067in}}%
\pgfpathlineto{\pgfqpoint{2.162802in}{2.769222in}}%
\pgfpathlineto{\pgfqpoint{2.163295in}{2.862599in}}%
\pgfpathlineto{\pgfqpoint{2.164775in}{3.343740in}}%
\pgfpathlineto{\pgfqpoint{2.166254in}{3.343740in}}%
\pgfpathlineto{\pgfqpoint{2.167734in}{3.299604in}}%
\pgfpathlineto{\pgfqpoint{2.169214in}{3.299604in}}%
\pgfpathlineto{\pgfqpoint{2.169707in}{3.234172in}}%
\pgfpathlineto{\pgfqpoint{2.170200in}{2.320189in}}%
\pgfpathlineto{\pgfqpoint{2.170694in}{3.227361in}}%
\pgfpathlineto{\pgfqpoint{2.171187in}{3.246843in}}%
\pgfpathlineto{\pgfqpoint{2.172173in}{3.375633in}}%
\pgfpathlineto{\pgfqpoint{2.173160in}{3.313075in}}%
\pgfpathlineto{\pgfqpoint{2.173653in}{3.375471in}}%
\pgfpathlineto{\pgfqpoint{2.175133in}{3.286595in}}%
\pgfpathlineto{\pgfqpoint{2.175626in}{3.256967in}}%
\pgfpathlineto{\pgfqpoint{2.176119in}{2.957054in}}%
\pgfpathlineto{\pgfqpoint{2.177599in}{3.379469in}}%
\pgfpathlineto{\pgfqpoint{2.178092in}{3.379469in}}%
\pgfpathlineto{\pgfqpoint{2.178585in}{3.094802in}}%
\pgfpathlineto{\pgfqpoint{2.179078in}{3.242474in}}%
\pgfpathlineto{\pgfqpoint{2.180558in}{3.321628in}}%
\pgfpathlineto{\pgfqpoint{2.182531in}{3.031188in}}%
\pgfpathlineto{\pgfqpoint{2.183024in}{3.031188in}}%
\pgfpathlineto{\pgfqpoint{2.183518in}{3.136243in}}%
\pgfpathlineto{\pgfqpoint{2.184011in}{3.107082in}}%
\pgfpathlineto{\pgfqpoint{2.184997in}{3.107082in}}%
\pgfpathlineto{\pgfqpoint{2.185984in}{3.375822in}}%
\pgfpathlineto{\pgfqpoint{2.186477in}{3.334114in}}%
\pgfpathlineto{\pgfqpoint{2.187957in}{3.183157in}}%
\pgfpathlineto{\pgfqpoint{2.188450in}{3.183157in}}%
\pgfpathlineto{\pgfqpoint{2.189930in}{3.322443in}}%
\pgfpathlineto{\pgfqpoint{2.190916in}{3.359569in}}%
\pgfpathlineto{\pgfqpoint{2.192396in}{3.295760in}}%
\pgfpathlineto{\pgfqpoint{2.193875in}{3.295760in}}%
\pgfpathlineto{\pgfqpoint{2.195355in}{3.343499in}}%
\pgfpathlineto{\pgfqpoint{2.195848in}{2.668583in}}%
\pgfpathlineto{\pgfqpoint{2.196342in}{3.330754in}}%
\pgfpathlineto{\pgfqpoint{2.198314in}{2.810859in}}%
\pgfpathlineto{\pgfqpoint{2.198808in}{2.779485in}}%
\pgfpathlineto{\pgfqpoint{2.200287in}{2.949153in}}%
\pgfpathlineto{\pgfqpoint{2.202260in}{2.949153in}}%
\pgfpathlineto{\pgfqpoint{2.203247in}{2.830911in}}%
\pgfpathlineto{\pgfqpoint{2.204726in}{3.276869in}}%
\pgfpathlineto{\pgfqpoint{2.206206in}{3.320793in}}%
\pgfpathlineto{\pgfqpoint{2.207686in}{3.299524in}}%
\pgfpathlineto{\pgfqpoint{2.208179in}{2.859103in}}%
\pgfpathlineto{\pgfqpoint{2.208672in}{3.305186in}}%
\pgfpathlineto{\pgfqpoint{2.209166in}{3.305186in}}%
\pgfpathlineto{\pgfqpoint{2.210152in}{3.341658in}}%
\pgfpathlineto{\pgfqpoint{2.211632in}{3.304272in}}%
\pgfpathlineto{\pgfqpoint{2.213111in}{3.355430in}}%
\pgfpathlineto{\pgfqpoint{2.214591in}{3.078958in}}%
\pgfpathlineto{\pgfqpoint{2.215578in}{3.078958in}}%
\pgfpathlineto{\pgfqpoint{2.217057in}{3.369701in}}%
\pgfpathlineto{\pgfqpoint{2.217551in}{3.083558in}}%
\pgfpathlineto{\pgfqpoint{2.218537in}{3.129212in}}%
\pgfpathlineto{\pgfqpoint{2.219030in}{3.129212in}}%
\pgfpathlineto{\pgfqpoint{2.219523in}{2.690302in}}%
\pgfpathlineto{\pgfqpoint{2.220017in}{3.131732in}}%
\pgfpathlineto{\pgfqpoint{2.221003in}{3.170945in}}%
\pgfpathlineto{\pgfqpoint{2.221990in}{3.079424in}}%
\pgfpathlineto{\pgfqpoint{2.222976in}{3.317275in}}%
\pgfpathlineto{\pgfqpoint{2.224456in}{3.100522in}}%
\pgfpathlineto{\pgfqpoint{2.225935in}{3.352467in}}%
\pgfpathlineto{\pgfqpoint{2.226922in}{3.352467in}}%
\pgfpathlineto{\pgfqpoint{2.227415in}{3.313526in}}%
\pgfpathlineto{\pgfqpoint{2.228895in}{3.374787in}}%
\pgfpathlineto{\pgfqpoint{2.229881in}{3.374787in}}%
\pgfpathlineto{\pgfqpoint{2.231361in}{2.955593in}}%
\pgfpathlineto{\pgfqpoint{2.232347in}{2.955593in}}%
\pgfpathlineto{\pgfqpoint{2.233827in}{3.324178in}}%
\pgfpathlineto{\pgfqpoint{2.234320in}{3.324178in}}%
\pgfpathlineto{\pgfqpoint{2.235307in}{3.129078in}}%
\pgfpathlineto{\pgfqpoint{2.236787in}{3.375254in}}%
\pgfpathlineto{\pgfqpoint{2.237280in}{3.375254in}}%
\pgfpathlineto{\pgfqpoint{2.237773in}{3.368155in}}%
\pgfpathlineto{\pgfqpoint{2.239253in}{3.329209in}}%
\pgfpathlineto{\pgfqpoint{2.239746in}{3.329209in}}%
\pgfpathlineto{\pgfqpoint{2.240732in}{3.110604in}}%
\pgfpathlineto{\pgfqpoint{2.242212in}{3.352434in}}%
\pgfpathlineto{\pgfqpoint{2.243692in}{2.665983in}}%
\pgfpathlineto{\pgfqpoint{2.244678in}{3.044022in}}%
\pgfpathlineto{\pgfqpoint{2.245665in}{1.648312in}}%
\pgfpathlineto{\pgfqpoint{2.246158in}{3.123331in}}%
\pgfpathlineto{\pgfqpoint{2.247144in}{2.985483in}}%
\pgfpathlineto{\pgfqpoint{2.248624in}{3.306703in}}%
\pgfpathlineto{\pgfqpoint{2.249117in}{3.306703in}}%
\pgfpathlineto{\pgfqpoint{2.250597in}{3.332063in}}%
\pgfpathlineto{\pgfqpoint{2.253063in}{3.332063in}}%
\pgfpathlineto{\pgfqpoint{2.254543in}{3.351293in}}%
\pgfpathlineto{\pgfqpoint{2.255529in}{3.351293in}}%
\pgfpathlineto{\pgfqpoint{2.256516in}{3.379912in}}%
\pgfpathlineto{\pgfqpoint{2.257009in}{3.021063in}}%
\pgfpathlineto{\pgfqpoint{2.257502in}{3.257584in}}%
\pgfpathlineto{\pgfqpoint{2.257995in}{3.257584in}}%
\pgfpathlineto{\pgfqpoint{2.259475in}{3.284592in}}%
\pgfpathlineto{\pgfqpoint{2.259968in}{2.936796in}}%
\pgfpathlineto{\pgfqpoint{2.260462in}{3.292355in}}%
\pgfpathlineto{\pgfqpoint{2.261448in}{3.292355in}}%
\pgfpathlineto{\pgfqpoint{2.262435in}{3.059537in}}%
\pgfpathlineto{\pgfqpoint{2.263914in}{3.355067in}}%
\pgfpathlineto{\pgfqpoint{2.264901in}{2.928153in}}%
\pgfpathlineto{\pgfqpoint{2.266380in}{3.379890in}}%
\pgfpathlineto{\pgfqpoint{2.266874in}{3.379890in}}%
\pgfpathlineto{\pgfqpoint{2.268353in}{2.548228in}}%
\pgfpathlineto{\pgfqpoint{2.268847in}{2.548228in}}%
\pgfpathlineto{\pgfqpoint{2.270326in}{3.339423in}}%
\pgfpathlineto{\pgfqpoint{2.270819in}{3.339423in}}%
\pgfpathlineto{\pgfqpoint{2.271806in}{2.780467in}}%
\pgfpathlineto{\pgfqpoint{2.273286in}{3.268344in}}%
\pgfpathlineto{\pgfqpoint{2.273779in}{3.268344in}}%
\pgfpathlineto{\pgfqpoint{2.274272in}{3.361546in}}%
\pgfpathlineto{\pgfqpoint{2.274765in}{3.135478in}}%
\pgfpathlineto{\pgfqpoint{2.275259in}{3.362234in}}%
\pgfpathlineto{\pgfqpoint{2.276245in}{3.379612in}}%
\pgfpathlineto{\pgfqpoint{2.277725in}{3.349783in}}%
\pgfpathlineto{\pgfqpoint{2.279698in}{3.348757in}}%
\pgfpathlineto{\pgfqpoint{2.280191in}{3.329948in}}%
\pgfpathlineto{\pgfqpoint{2.280684in}{3.346706in}}%
\pgfpathlineto{\pgfqpoint{2.281177in}{3.346706in}}%
\pgfpathlineto{\pgfqpoint{2.281671in}{3.033454in}}%
\pgfpathlineto{\pgfqpoint{2.282164in}{3.374421in}}%
\pgfpathlineto{\pgfqpoint{2.284630in}{3.374421in}}%
\pgfpathlineto{\pgfqpoint{2.285616in}{3.141777in}}%
\pgfpathlineto{\pgfqpoint{2.287096in}{3.359529in}}%
\pgfpathlineto{\pgfqpoint{2.287589in}{3.346966in}}%
\pgfpathlineto{\pgfqpoint{2.288083in}{3.357976in}}%
\pgfpathlineto{\pgfqpoint{2.289562in}{3.357976in}}%
\pgfpathlineto{\pgfqpoint{2.290549in}{3.110553in}}%
\pgfpathlineto{\pgfqpoint{2.291535in}{3.379920in}}%
\pgfpathlineto{\pgfqpoint{2.293508in}{2.899112in}}%
\pgfpathlineto{\pgfqpoint{2.294001in}{3.091836in}}%
\pgfpathlineto{\pgfqpoint{2.294495in}{2.875235in}}%
\pgfpathlineto{\pgfqpoint{2.294988in}{2.875235in}}%
\pgfpathlineto{\pgfqpoint{2.295481in}{3.379286in}}%
\pgfpathlineto{\pgfqpoint{2.296467in}{3.378499in}}%
\pgfpathlineto{\pgfqpoint{2.297947in}{3.373474in}}%
\pgfpathlineto{\pgfqpoint{2.299427in}{3.379975in}}%
\pgfpathlineto{\pgfqpoint{2.302386in}{3.379975in}}%
\pgfpathlineto{\pgfqpoint{2.303866in}{3.318093in}}%
\pgfpathlineto{\pgfqpoint{2.306332in}{3.318093in}}%
\pgfpathlineto{\pgfqpoint{2.306825in}{3.378708in}}%
\pgfpathlineto{\pgfqpoint{2.307319in}{3.358583in}}%
\pgfpathlineto{\pgfqpoint{2.310771in}{3.358583in}}%
\pgfpathlineto{\pgfqpoint{2.312251in}{3.255714in}}%
\pgfpathlineto{\pgfqpoint{2.312744in}{3.255714in}}%
\pgfpathlineto{\pgfqpoint{2.314224in}{3.377282in}}%
\pgfpathlineto{\pgfqpoint{2.314717in}{3.377282in}}%
\pgfpathlineto{\pgfqpoint{2.315210in}{3.074604in}}%
\pgfpathlineto{\pgfqpoint{2.315704in}{3.360606in}}%
\pgfpathlineto{\pgfqpoint{2.316197in}{3.293751in}}%
\pgfpathlineto{\pgfqpoint{2.316690in}{3.338482in}}%
\pgfpathlineto{\pgfqpoint{2.317183in}{3.338482in}}%
\pgfpathlineto{\pgfqpoint{2.319156in}{3.115598in}}%
\pgfpathlineto{\pgfqpoint{2.320636in}{3.352410in}}%
\pgfpathlineto{\pgfqpoint{2.325075in}{3.352410in}}%
\pgfpathlineto{\pgfqpoint{2.326555in}{3.342698in}}%
\pgfpathlineto{\pgfqpoint{2.327048in}{3.342698in}}%
\pgfpathlineto{\pgfqpoint{2.328528in}{3.375825in}}%
\pgfpathlineto{\pgfqpoint{2.329021in}{3.375825in}}%
\pgfpathlineto{\pgfqpoint{2.330500in}{2.824720in}}%
\pgfpathlineto{\pgfqpoint{2.331980in}{3.355473in}}%
\pgfpathlineto{\pgfqpoint{2.333460in}{3.355473in}}%
\pgfpathlineto{\pgfqpoint{2.334446in}{3.371897in}}%
\pgfpathlineto{\pgfqpoint{2.335433in}{2.659725in}}%
\pgfpathlineto{\pgfqpoint{2.336912in}{3.355146in}}%
\pgfpathlineto{\pgfqpoint{2.341845in}{3.355146in}}%
\pgfpathlineto{\pgfqpoint{2.342338in}{3.379983in}}%
\pgfpathlineto{\pgfqpoint{2.344311in}{3.163524in}}%
\pgfpathlineto{\pgfqpoint{2.344804in}{3.378779in}}%
\pgfpathlineto{\pgfqpoint{2.346284in}{2.307609in}}%
\pgfpathlineto{\pgfqpoint{2.346777in}{2.307609in}}%
\pgfpathlineto{\pgfqpoint{2.347270in}{3.317043in}}%
\pgfpathlineto{\pgfqpoint{2.347764in}{3.030718in}}%
\pgfpathlineto{\pgfqpoint{2.348257in}{3.030718in}}%
\pgfpathlineto{\pgfqpoint{2.349736in}{3.227244in}}%
\pgfpathlineto{\pgfqpoint{2.350723in}{3.227244in}}%
\pgfpathlineto{\pgfqpoint{2.351216in}{3.379958in}}%
\pgfpathlineto{\pgfqpoint{2.351709in}{3.378075in}}%
\pgfpathlineto{\pgfqpoint{2.353189in}{3.211117in}}%
\pgfpathlineto{\pgfqpoint{2.353682in}{3.211117in}}%
\pgfpathlineto{\pgfqpoint{2.354669in}{3.375648in}}%
\pgfpathlineto{\pgfqpoint{2.355162in}{3.364420in}}%
\pgfpathlineto{\pgfqpoint{2.357628in}{3.364420in}}%
\pgfpathlineto{\pgfqpoint{2.358121in}{3.106512in}}%
\pgfpathlineto{\pgfqpoint{2.358615in}{3.374039in}}%
\pgfpathlineto{\pgfqpoint{2.359108in}{3.374039in}}%
\pgfpathlineto{\pgfqpoint{2.361081in}{3.126909in}}%
\pgfpathlineto{\pgfqpoint{2.361574in}{3.375803in}}%
\pgfpathlineto{\pgfqpoint{2.362560in}{3.323967in}}%
\pgfpathlineto{\pgfqpoint{2.364040in}{3.323967in}}%
\pgfpathlineto{\pgfqpoint{2.365027in}{3.318335in}}%
\pgfpathlineto{\pgfqpoint{2.366506in}{2.940152in}}%
\pgfpathlineto{\pgfqpoint{2.369959in}{2.940152in}}%
\pgfpathlineto{\pgfqpoint{2.370452in}{2.186172in}}%
\pgfpathlineto{\pgfqpoint{2.371932in}{3.354075in}}%
\pgfpathlineto{\pgfqpoint{2.372425in}{3.332143in}}%
\pgfpathlineto{\pgfqpoint{2.372918in}{2.971102in}}%
\pgfpathlineto{\pgfqpoint{2.373412in}{3.283055in}}%
\pgfpathlineto{\pgfqpoint{2.375384in}{3.283055in}}%
\pgfpathlineto{\pgfqpoint{2.376864in}{2.716199in}}%
\pgfpathlineto{\pgfqpoint{2.377357in}{2.716199in}}%
\pgfpathlineto{\pgfqpoint{2.377851in}{2.541995in}}%
\pgfpathlineto{\pgfqpoint{2.379330in}{3.378935in}}%
\pgfpathlineto{\pgfqpoint{2.381796in}{3.378935in}}%
\pgfpathlineto{\pgfqpoint{2.382783in}{2.996149in}}%
\pgfpathlineto{\pgfqpoint{2.383276in}{3.060222in}}%
\pgfpathlineto{\pgfqpoint{2.383769in}{3.060222in}}%
\pgfpathlineto{\pgfqpoint{2.385249in}{3.086713in}}%
\pgfpathlineto{\pgfqpoint{2.386729in}{3.097814in}}%
\pgfpathlineto{\pgfqpoint{2.387715in}{3.097814in}}%
\pgfpathlineto{\pgfqpoint{2.388702in}{3.374614in}}%
\pgfpathlineto{\pgfqpoint{2.389195in}{3.367286in}}%
\pgfpathlineto{\pgfqpoint{2.389688in}{3.367286in}}%
\pgfpathlineto{\pgfqpoint{2.390181in}{2.994248in}}%
\pgfpathlineto{\pgfqpoint{2.390675in}{3.181471in}}%
\pgfpathlineto{\pgfqpoint{2.391661in}{3.181471in}}%
\pgfpathlineto{\pgfqpoint{2.393141in}{3.016644in}}%
\pgfpathlineto{\pgfqpoint{2.393634in}{3.016644in}}%
\pgfpathlineto{\pgfqpoint{2.395114in}{2.922051in}}%
\pgfpathlineto{\pgfqpoint{2.396593in}{3.327234in}}%
\pgfpathlineto{\pgfqpoint{2.397087in}{3.327234in}}%
\pgfpathlineto{\pgfqpoint{2.398073in}{3.374304in}}%
\pgfpathlineto{\pgfqpoint{2.398566in}{3.362696in}}%
\pgfpathlineto{\pgfqpoint{2.401032in}{3.362696in}}%
\pgfpathlineto{\pgfqpoint{2.402512in}{3.115045in}}%
\pgfpathlineto{\pgfqpoint{2.403005in}{3.115045in}}%
\pgfpathlineto{\pgfqpoint{2.403499in}{2.549130in}}%
\pgfpathlineto{\pgfqpoint{2.404978in}{3.372409in}}%
\pgfpathlineto{\pgfqpoint{2.406951in}{3.242518in}}%
\pgfpathlineto{\pgfqpoint{2.407444in}{3.242518in}}%
\pgfpathlineto{\pgfqpoint{2.408924in}{3.270757in}}%
\pgfpathlineto{\pgfqpoint{2.409417in}{3.270757in}}%
\pgfpathlineto{\pgfqpoint{2.410897in}{3.371767in}}%
\pgfpathlineto{\pgfqpoint{2.412377in}{3.371767in}}%
\pgfpathlineto{\pgfqpoint{2.413363in}{3.290041in}}%
\pgfpathlineto{\pgfqpoint{2.414843in}{3.373717in}}%
\pgfpathlineto{\pgfqpoint{2.416816in}{3.373717in}}%
\pgfpathlineto{\pgfqpoint{2.418296in}{3.368804in}}%
\pgfpathlineto{\pgfqpoint{2.422241in}{3.368804in}}%
\pgfpathlineto{\pgfqpoint{2.423721in}{3.331943in}}%
\pgfpathlineto{\pgfqpoint{2.425694in}{3.331943in}}%
\pgfpathlineto{\pgfqpoint{2.426187in}{3.320708in}}%
\pgfpathlineto{\pgfqpoint{2.427667in}{3.350649in}}%
\pgfpathlineto{\pgfqpoint{2.429640in}{3.350649in}}%
\pgfpathlineto{\pgfqpoint{2.431120in}{3.265049in}}%
\pgfpathlineto{\pgfqpoint{2.432106in}{3.265049in}}%
\pgfpathlineto{\pgfqpoint{2.433093in}{3.193388in}}%
\pgfpathlineto{\pgfqpoint{2.434572in}{3.326913in}}%
\pgfpathlineto{\pgfqpoint{2.435559in}{2.976960in}}%
\pgfpathlineto{\pgfqpoint{2.436545in}{3.320839in}}%
\pgfpathlineto{\pgfqpoint{2.437038in}{1.890282in}}%
\pgfpathlineto{\pgfqpoint{2.437532in}{2.665094in}}%
\pgfpathlineto{\pgfqpoint{2.439505in}{2.665094in}}%
\pgfpathlineto{\pgfqpoint{2.440491in}{3.372257in}}%
\pgfpathlineto{\pgfqpoint{2.441477in}{2.877661in}}%
\pgfpathlineto{\pgfqpoint{2.442464in}{3.363820in}}%
\pgfpathlineto{\pgfqpoint{2.443450in}{3.052358in}}%
\pgfpathlineto{\pgfqpoint{2.443944in}{3.264896in}}%
\pgfpathlineto{\pgfqpoint{2.444437in}{3.264896in}}%
\pgfpathlineto{\pgfqpoint{2.445423in}{3.300990in}}%
\pgfpathlineto{\pgfqpoint{2.445917in}{3.014370in}}%
\pgfpathlineto{\pgfqpoint{2.446410in}{3.355212in}}%
\pgfpathlineto{\pgfqpoint{2.447396in}{3.355212in}}%
\pgfpathlineto{\pgfqpoint{2.448876in}{3.139868in}}%
\pgfpathlineto{\pgfqpoint{2.449369in}{3.139868in}}%
\pgfpathlineto{\pgfqpoint{2.450849in}{3.376454in}}%
\pgfpathlineto{\pgfqpoint{2.452822in}{3.376454in}}%
\pgfpathlineto{\pgfqpoint{2.453808in}{3.059581in}}%
\pgfpathlineto{\pgfqpoint{2.454301in}{3.278430in}}%
\pgfpathlineto{\pgfqpoint{2.455781in}{3.278430in}}%
\pgfpathlineto{\pgfqpoint{2.456274in}{3.259483in}}%
\pgfpathlineto{\pgfqpoint{2.457261in}{3.317333in}}%
\pgfpathlineto{\pgfqpoint{2.457754in}{2.813245in}}%
\pgfpathlineto{\pgfqpoint{2.458741in}{2.912477in}}%
\pgfpathlineto{\pgfqpoint{2.459234in}{2.912477in}}%
\pgfpathlineto{\pgfqpoint{2.460713in}{3.097288in}}%
\pgfpathlineto{\pgfqpoint{2.461207in}{3.097288in}}%
\pgfpathlineto{\pgfqpoint{2.462686in}{2.839038in}}%
\pgfpathlineto{\pgfqpoint{2.464166in}{3.276741in}}%
\pgfpathlineto{\pgfqpoint{2.465646in}{3.276741in}}%
\pgfpathlineto{\pgfqpoint{2.467125in}{3.359559in}}%
\pgfpathlineto{\pgfqpoint{2.468605in}{3.359559in}}%
\pgfpathlineto{\pgfqpoint{2.470085in}{3.339656in}}%
\pgfpathlineto{\pgfqpoint{2.471565in}{3.339656in}}%
\pgfpathlineto{\pgfqpoint{2.473044in}{3.257363in}}%
\pgfpathlineto{\pgfqpoint{2.474524in}{3.257363in}}%
\pgfpathlineto{\pgfqpoint{2.475017in}{3.327599in}}%
\pgfpathlineto{\pgfqpoint{2.476004in}{3.166868in}}%
\pgfpathlineto{\pgfqpoint{2.476990in}{3.286776in}}%
\pgfpathlineto{\pgfqpoint{2.479456in}{2.871747in}}%
\pgfpathlineto{\pgfqpoint{2.479949in}{2.871747in}}%
\pgfpathlineto{\pgfqpoint{2.481429in}{2.883719in}}%
\pgfpathlineto{\pgfqpoint{2.481922in}{3.291024in}}%
\pgfpathlineto{\pgfqpoint{2.482416in}{2.627939in}}%
\pgfpathlineto{\pgfqpoint{2.482909in}{3.039903in}}%
\pgfpathlineto{\pgfqpoint{2.483895in}{2.936742in}}%
\pgfpathlineto{\pgfqpoint{2.484389in}{3.373683in}}%
\pgfpathlineto{\pgfqpoint{2.485375in}{3.343924in}}%
\pgfpathlineto{\pgfqpoint{2.487841in}{3.343924in}}%
\pgfpathlineto{\pgfqpoint{2.489321in}{3.084913in}}%
\pgfpathlineto{\pgfqpoint{2.490307in}{3.084913in}}%
\pgfpathlineto{\pgfqpoint{2.491294in}{3.363340in}}%
\pgfpathlineto{\pgfqpoint{2.491787in}{3.323101in}}%
\pgfpathlineto{\pgfqpoint{2.492773in}{3.379215in}}%
\pgfpathlineto{\pgfqpoint{2.493267in}{3.370696in}}%
\pgfpathlineto{\pgfqpoint{2.494746in}{3.117657in}}%
\pgfpathlineto{\pgfqpoint{2.496226in}{3.292961in}}%
\pgfpathlineto{\pgfqpoint{2.496719in}{3.292961in}}%
\pgfpathlineto{\pgfqpoint{2.497706in}{3.378247in}}%
\pgfpathlineto{\pgfqpoint{2.498199in}{3.093148in}}%
\pgfpathlineto{\pgfqpoint{2.498692in}{3.149104in}}%
\pgfpathlineto{\pgfqpoint{2.499679in}{3.314379in}}%
\pgfpathlineto{\pgfqpoint{2.500172in}{3.301201in}}%
\pgfpathlineto{\pgfqpoint{2.501652in}{3.372177in}}%
\pgfpathlineto{\pgfqpoint{2.503131in}{3.331407in}}%
\pgfpathlineto{\pgfqpoint{2.504118in}{3.331407in}}%
\pgfpathlineto{\pgfqpoint{2.505597in}{3.307552in}}%
\pgfpathlineto{\pgfqpoint{2.506091in}{3.307552in}}%
\pgfpathlineto{\pgfqpoint{2.507570in}{3.316848in}}%
\pgfpathlineto{\pgfqpoint{2.508557in}{3.316848in}}%
\pgfpathlineto{\pgfqpoint{2.509543in}{3.246534in}}%
\pgfpathlineto{\pgfqpoint{2.511023in}{2.543171in}}%
\pgfpathlineto{\pgfqpoint{2.512996in}{2.543171in}}%
\pgfpathlineto{\pgfqpoint{2.514969in}{3.376113in}}%
\pgfpathlineto{\pgfqpoint{2.515955in}{3.316408in}}%
\pgfpathlineto{\pgfqpoint{2.516449in}{2.840950in}}%
\pgfpathlineto{\pgfqpoint{2.516942in}{3.231272in}}%
\pgfpathlineto{\pgfqpoint{2.517928in}{3.231272in}}%
\pgfpathlineto{\pgfqpoint{2.518915in}{3.146961in}}%
\pgfpathlineto{\pgfqpoint{2.520394in}{3.351235in}}%
\pgfpathlineto{\pgfqpoint{2.523354in}{3.351235in}}%
\pgfpathlineto{\pgfqpoint{2.524834in}{3.296378in}}%
\pgfpathlineto{\pgfqpoint{2.525327in}{1.864566in}}%
\pgfpathlineto{\pgfqpoint{2.525820in}{3.353309in}}%
\pgfpathlineto{\pgfqpoint{2.527300in}{3.364602in}}%
\pgfpathlineto{\pgfqpoint{2.528286in}{3.371392in}}%
\pgfpathlineto{\pgfqpoint{2.529766in}{3.348750in}}%
\pgfpathlineto{\pgfqpoint{2.530752in}{2.751244in}}%
\pgfpathlineto{\pgfqpoint{2.531739in}{3.375476in}}%
\pgfpathlineto{\pgfqpoint{2.532232in}{3.306290in}}%
\pgfpathlineto{\pgfqpoint{2.532725in}{3.306290in}}%
\pgfpathlineto{\pgfqpoint{2.534205in}{3.350731in}}%
\pgfpathlineto{\pgfqpoint{2.538644in}{3.350731in}}%
\pgfpathlineto{\pgfqpoint{2.540124in}{3.230728in}}%
\pgfpathlineto{\pgfqpoint{2.540617in}{2.164593in}}%
\pgfpathlineto{\pgfqpoint{2.541110in}{3.346408in}}%
\pgfpathlineto{\pgfqpoint{2.542097in}{3.165048in}}%
\pgfpathlineto{\pgfqpoint{2.543083in}{3.361064in}}%
\pgfpathlineto{\pgfqpoint{2.544070in}{3.338166in}}%
\pgfpathlineto{\pgfqpoint{2.544563in}{3.338166in}}%
\pgfpathlineto{\pgfqpoint{2.545549in}{3.379769in}}%
\pgfpathlineto{\pgfqpoint{2.547029in}{3.362324in}}%
\pgfpathlineto{\pgfqpoint{2.548015in}{3.362324in}}%
\pgfpathlineto{\pgfqpoint{2.548509in}{2.634572in}}%
\pgfpathlineto{\pgfqpoint{2.549002in}{3.273324in}}%
\pgfpathlineto{\pgfqpoint{2.550482in}{3.374593in}}%
\pgfpathlineto{\pgfqpoint{2.551468in}{3.374593in}}%
\pgfpathlineto{\pgfqpoint{2.551961in}{3.366889in}}%
\pgfpathlineto{\pgfqpoint{2.552454in}{3.371895in}}%
\pgfpathlineto{\pgfqpoint{2.553934in}{3.246388in}}%
\pgfpathlineto{\pgfqpoint{2.555414in}{3.246388in}}%
\pgfpathlineto{\pgfqpoint{2.556894in}{3.298638in}}%
\pgfpathlineto{\pgfqpoint{2.558373in}{3.298638in}}%
\pgfpathlineto{\pgfqpoint{2.558866in}{2.177639in}}%
\pgfpathlineto{\pgfqpoint{2.559360in}{3.369311in}}%
\pgfpathlineto{\pgfqpoint{2.559853in}{3.369311in}}%
\pgfpathlineto{\pgfqpoint{2.561333in}{3.073492in}}%
\pgfpathlineto{\pgfqpoint{2.562319in}{3.379802in}}%
\pgfpathlineto{\pgfqpoint{2.562812in}{3.375883in}}%
\pgfpathlineto{\pgfqpoint{2.563306in}{3.375883in}}%
\pgfpathlineto{\pgfqpoint{2.564785in}{3.123030in}}%
\pgfpathlineto{\pgfqpoint{2.565278in}{2.491733in}}%
\pgfpathlineto{\pgfqpoint{2.565772in}{3.066962in}}%
\pgfpathlineto{\pgfqpoint{2.567251in}{3.248419in}}%
\pgfpathlineto{\pgfqpoint{2.568731in}{3.248419in}}%
\pgfpathlineto{\pgfqpoint{2.570211in}{2.718570in}}%
\pgfpathlineto{\pgfqpoint{2.571197in}{2.718570in}}%
\pgfpathlineto{\pgfqpoint{2.571690in}{3.237631in}}%
\pgfpathlineto{\pgfqpoint{2.572677in}{3.146852in}}%
\pgfpathlineto{\pgfqpoint{2.573663in}{3.358963in}}%
\pgfpathlineto{\pgfqpoint{2.574157in}{3.352909in}}%
\pgfpathlineto{\pgfqpoint{2.575636in}{3.367298in}}%
\pgfpathlineto{\pgfqpoint{2.577116in}{2.936284in}}%
\pgfpathlineto{\pgfqpoint{2.577609in}{3.174096in}}%
\pgfpathlineto{\pgfqpoint{2.578596in}{3.138806in}}%
\pgfpathlineto{\pgfqpoint{2.579582in}{2.306420in}}%
\pgfpathlineto{\pgfqpoint{2.581062in}{3.370140in}}%
\pgfpathlineto{\pgfqpoint{2.582542in}{3.370140in}}%
\pgfpathlineto{\pgfqpoint{2.584021in}{3.323344in}}%
\pgfpathlineto{\pgfqpoint{2.585008in}{3.323344in}}%
\pgfpathlineto{\pgfqpoint{2.585994in}{3.206301in}}%
\pgfpathlineto{\pgfqpoint{2.586487in}{3.346993in}}%
\pgfpathlineto{\pgfqpoint{2.586981in}{3.134995in}}%
\pgfpathlineto{\pgfqpoint{2.587474in}{3.233379in}}%
\pgfpathlineto{\pgfqpoint{2.588954in}{3.233379in}}%
\pgfpathlineto{\pgfqpoint{2.589940in}{3.379270in}}%
\pgfpathlineto{\pgfqpoint{2.591420in}{2.772034in}}%
\pgfpathlineto{\pgfqpoint{2.591913in}{2.772034in}}%
\pgfpathlineto{\pgfqpoint{2.592899in}{3.373495in}}%
\pgfpathlineto{\pgfqpoint{2.594379in}{3.033261in}}%
\pgfpathlineto{\pgfqpoint{2.594872in}{3.203253in}}%
\pgfpathlineto{\pgfqpoint{2.595366in}{3.140958in}}%
\pgfpathlineto{\pgfqpoint{2.596352in}{3.140958in}}%
\pgfpathlineto{\pgfqpoint{2.598325in}{3.378370in}}%
\pgfpathlineto{\pgfqpoint{2.599805in}{3.378370in}}%
\pgfpathlineto{\pgfqpoint{2.600298in}{2.561923in}}%
\pgfpathlineto{\pgfqpoint{2.600791in}{3.373665in}}%
\pgfpathlineto{\pgfqpoint{2.601778in}{3.373665in}}%
\pgfpathlineto{\pgfqpoint{2.603257in}{3.204167in}}%
\pgfpathlineto{\pgfqpoint{2.605230in}{3.204167in}}%
\pgfpathlineto{\pgfqpoint{2.606710in}{3.330744in}}%
\pgfpathlineto{\pgfqpoint{2.608190in}{3.330744in}}%
\pgfpathlineto{\pgfqpoint{2.609669in}{3.375207in}}%
\pgfpathlineto{\pgfqpoint{2.612629in}{3.375207in}}%
\pgfpathlineto{\pgfqpoint{2.614108in}{3.279172in}}%
\pgfpathlineto{\pgfqpoint{2.614602in}{3.279172in}}%
\pgfpathlineto{\pgfqpoint{2.616081in}{3.366965in}}%
\pgfpathlineto{\pgfqpoint{2.618054in}{3.366965in}}%
\pgfpathlineto{\pgfqpoint{2.619534in}{3.325165in}}%
\pgfpathlineto{\pgfqpoint{2.620027in}{3.325165in}}%
\pgfpathlineto{\pgfqpoint{2.621014in}{2.792082in}}%
\pgfpathlineto{\pgfqpoint{2.622493in}{3.258299in}}%
\pgfpathlineto{\pgfqpoint{2.623480in}{3.327097in}}%
\pgfpathlineto{\pgfqpoint{2.623973in}{2.568704in}}%
\pgfpathlineto{\pgfqpoint{2.625453in}{3.378046in}}%
\pgfpathlineto{\pgfqpoint{2.626439in}{3.378046in}}%
\pgfpathlineto{\pgfqpoint{2.627919in}{3.304931in}}%
\pgfpathlineto{\pgfqpoint{2.628412in}{3.304931in}}%
\pgfpathlineto{\pgfqpoint{2.628905in}{3.376746in}}%
\pgfpathlineto{\pgfqpoint{2.629399in}{3.178206in}}%
\pgfpathlineto{\pgfqpoint{2.629892in}{3.351212in}}%
\pgfpathlineto{\pgfqpoint{2.630878in}{3.351212in}}%
\pgfpathlineto{\pgfqpoint{2.632851in}{3.379483in}}%
\pgfpathlineto{\pgfqpoint{2.634331in}{3.374221in}}%
\pgfpathlineto{\pgfqpoint{2.635317in}{2.809412in}}%
\pgfpathlineto{\pgfqpoint{2.636304in}{1.959543in}}%
\pgfpathlineto{\pgfqpoint{2.637290in}{3.173678in}}%
\pgfpathlineto{\pgfqpoint{2.637783in}{2.954887in}}%
\pgfpathlineto{\pgfqpoint{2.638770in}{2.954887in}}%
\pgfpathlineto{\pgfqpoint{2.640250in}{3.293002in}}%
\pgfpathlineto{\pgfqpoint{2.641236in}{3.212232in}}%
\pgfpathlineto{\pgfqpoint{2.641729in}{3.356342in}}%
\pgfpathlineto{\pgfqpoint{2.642223in}{3.043729in}}%
\pgfpathlineto{\pgfqpoint{2.642716in}{3.303443in}}%
\pgfpathlineto{\pgfqpoint{2.643209in}{3.303443in}}%
\pgfpathlineto{\pgfqpoint{2.643702in}{3.344077in}}%
\pgfpathlineto{\pgfqpoint{2.644195in}{3.289558in}}%
\pgfpathlineto{\pgfqpoint{2.645182in}{3.378410in}}%
\pgfpathlineto{\pgfqpoint{2.647648in}{3.167465in}}%
\pgfpathlineto{\pgfqpoint{2.648141in}{2.846915in}}%
\pgfpathlineto{\pgfqpoint{2.649128in}{3.293302in}}%
\pgfpathlineto{\pgfqpoint{2.649621in}{3.204570in}}%
\pgfpathlineto{\pgfqpoint{2.650607in}{3.226857in}}%
\pgfpathlineto{\pgfqpoint{2.652087in}{3.184319in}}%
\pgfpathlineto{\pgfqpoint{2.652580in}{3.184319in}}%
\pgfpathlineto{\pgfqpoint{2.653074in}{3.328868in}}%
\pgfpathlineto{\pgfqpoint{2.653567in}{3.284405in}}%
\pgfpathlineto{\pgfqpoint{2.654553in}{3.284405in}}%
\pgfpathlineto{\pgfqpoint{2.656033in}{3.197219in}}%
\pgfpathlineto{\pgfqpoint{2.658006in}{3.197219in}}%
\pgfpathlineto{\pgfqpoint{2.659486in}{3.348904in}}%
\pgfpathlineto{\pgfqpoint{2.661459in}{3.348904in}}%
\pgfpathlineto{\pgfqpoint{2.662938in}{3.329724in}}%
\pgfpathlineto{\pgfqpoint{2.664911in}{3.329724in}}%
\pgfpathlineto{\pgfqpoint{2.666391in}{3.245818in}}%
\pgfpathlineto{\pgfqpoint{2.669350in}{3.245818in}}%
\pgfpathlineto{\pgfqpoint{2.670830in}{3.097372in}}%
\pgfpathlineto{\pgfqpoint{2.671816in}{3.097372in}}%
\pgfpathlineto{\pgfqpoint{2.672310in}{3.371325in}}%
\pgfpathlineto{\pgfqpoint{2.673296in}{3.335667in}}%
\pgfpathlineto{\pgfqpoint{2.673789in}{3.335667in}}%
\pgfpathlineto{\pgfqpoint{2.675269in}{3.280469in}}%
\pgfpathlineto{\pgfqpoint{2.675762in}{3.379634in}}%
\pgfpathlineto{\pgfqpoint{2.676749in}{3.359786in}}%
\pgfpathlineto{\pgfqpoint{2.678228in}{3.359786in}}%
\pgfpathlineto{\pgfqpoint{2.679708in}{2.930158in}}%
\pgfpathlineto{\pgfqpoint{2.681188in}{3.365781in}}%
\pgfpathlineto{\pgfqpoint{2.682667in}{3.261047in}}%
\pgfpathlineto{\pgfqpoint{2.684640in}{3.261047in}}%
\pgfpathlineto{\pgfqpoint{2.686613in}{3.359135in}}%
\pgfpathlineto{\pgfqpoint{2.687600in}{3.260954in}}%
\pgfpathlineto{\pgfqpoint{2.688093in}{3.357073in}}%
\pgfpathlineto{\pgfqpoint{2.689079in}{3.043605in}}%
\pgfpathlineto{\pgfqpoint{2.690559in}{3.338268in}}%
\pgfpathlineto{\pgfqpoint{2.691546in}{2.382716in}}%
\pgfpathlineto{\pgfqpoint{2.692039in}{3.379635in}}%
\pgfpathlineto{\pgfqpoint{2.693025in}{3.349115in}}%
\pgfpathlineto{\pgfqpoint{2.694012in}{3.349115in}}%
\pgfpathlineto{\pgfqpoint{2.695491in}{3.300002in}}%
\pgfpathlineto{\pgfqpoint{2.696971in}{3.378302in}}%
\pgfpathlineto{\pgfqpoint{2.699437in}{3.378302in}}%
\pgfpathlineto{\pgfqpoint{2.700917in}{3.269435in}}%
\pgfpathlineto{\pgfqpoint{2.701410in}{3.379614in}}%
\pgfpathlineto{\pgfqpoint{2.701903in}{3.324783in}}%
\pgfpathlineto{\pgfqpoint{2.704370in}{3.324783in}}%
\pgfpathlineto{\pgfqpoint{2.704863in}{3.097382in}}%
\pgfpathlineto{\pgfqpoint{2.705356in}{3.299325in}}%
\pgfpathlineto{\pgfqpoint{2.707822in}{3.299325in}}%
\pgfpathlineto{\pgfqpoint{2.709302in}{3.324458in}}%
\pgfpathlineto{\pgfqpoint{2.709795in}{3.322830in}}%
\pgfpathlineto{\pgfqpoint{2.711275in}{3.223391in}}%
\pgfpathlineto{\pgfqpoint{2.712261in}{3.223391in}}%
\pgfpathlineto{\pgfqpoint{2.713741in}{3.037657in}}%
\pgfpathlineto{\pgfqpoint{2.715221in}{3.379979in}}%
\pgfpathlineto{\pgfqpoint{2.715714in}{3.379979in}}%
\pgfpathlineto{\pgfqpoint{2.717194in}{2.881136in}}%
\pgfpathlineto{\pgfqpoint{2.717687in}{3.081486in}}%
\pgfpathlineto{\pgfqpoint{2.718673in}{3.081486in}}%
\pgfpathlineto{\pgfqpoint{2.719167in}{3.033752in}}%
\pgfpathlineto{\pgfqpoint{2.721140in}{3.355589in}}%
\pgfpathlineto{\pgfqpoint{2.723112in}{3.355589in}}%
\pgfpathlineto{\pgfqpoint{2.724592in}{3.254777in}}%
\pgfpathlineto{\pgfqpoint{2.725085in}{3.254777in}}%
\pgfpathlineto{\pgfqpoint{2.726565in}{3.376939in}}%
\pgfpathlineto{\pgfqpoint{2.728045in}{3.376939in}}%
\pgfpathlineto{\pgfqpoint{2.729524in}{3.208173in}}%
\pgfpathlineto{\pgfqpoint{2.730018in}{3.208173in}}%
\pgfpathlineto{\pgfqpoint{2.730511in}{3.352528in}}%
\pgfpathlineto{\pgfqpoint{2.731004in}{2.988133in}}%
\pgfpathlineto{\pgfqpoint{2.731497in}{3.299059in}}%
\pgfpathlineto{\pgfqpoint{2.733964in}{3.299059in}}%
\pgfpathlineto{\pgfqpoint{2.734950in}{3.265653in}}%
\pgfpathlineto{\pgfqpoint{2.736430in}{3.037544in}}%
\pgfpathlineto{\pgfqpoint{2.736923in}{3.037544in}}%
\pgfpathlineto{\pgfqpoint{2.738403in}{3.377843in}}%
\pgfpathlineto{\pgfqpoint{2.738896in}{3.377843in}}%
\pgfpathlineto{\pgfqpoint{2.739389in}{3.327928in}}%
\pgfpathlineto{\pgfqpoint{2.739882in}{3.350078in}}%
\pgfpathlineto{\pgfqpoint{2.740869in}{3.350078in}}%
\pgfpathlineto{\pgfqpoint{2.741362in}{3.328312in}}%
\pgfpathlineto{\pgfqpoint{2.742842in}{2.893023in}}%
\pgfpathlineto{\pgfqpoint{2.743828in}{3.379955in}}%
\pgfpathlineto{\pgfqpoint{2.744321in}{3.352363in}}%
\pgfpathlineto{\pgfqpoint{2.746788in}{3.352742in}}%
\pgfpathlineto{\pgfqpoint{2.747774in}{3.234910in}}%
\pgfpathlineto{\pgfqpoint{2.748760in}{2.544899in}}%
\pgfpathlineto{\pgfqpoint{2.749254in}{3.339557in}}%
\pgfpathlineto{\pgfqpoint{2.750240in}{3.225530in}}%
\pgfpathlineto{\pgfqpoint{2.751720in}{3.295457in}}%
\pgfpathlineto{\pgfqpoint{2.753200in}{3.255848in}}%
\pgfpathlineto{\pgfqpoint{2.754186in}{0.580000in}}%
\pgfpathlineto{\pgfqpoint{2.754679in}{2.571676in}}%
\pgfpathlineto{\pgfqpoint{2.755666in}{2.241788in}}%
\pgfpathlineto{\pgfqpoint{2.757145in}{3.336534in}}%
\pgfpathlineto{\pgfqpoint{2.758625in}{3.107319in}}%
\pgfpathlineto{\pgfqpoint{2.759612in}{3.107319in}}%
\pgfpathlineto{\pgfqpoint{2.761091in}{3.369810in}}%
\pgfpathlineto{\pgfqpoint{2.763557in}{3.369810in}}%
\pgfpathlineto{\pgfqpoint{2.764051in}{2.584902in}}%
\pgfpathlineto{\pgfqpoint{2.764544in}{2.721838in}}%
\pgfpathlineto{\pgfqpoint{2.766024in}{3.271078in}}%
\pgfpathlineto{\pgfqpoint{2.767010in}{3.271078in}}%
\pgfpathlineto{\pgfqpoint{2.767996in}{3.379849in}}%
\pgfpathlineto{\pgfqpoint{2.768490in}{3.360599in}}%
\pgfpathlineto{\pgfqpoint{2.768983in}{3.360599in}}%
\pgfpathlineto{\pgfqpoint{2.770463in}{3.319163in}}%
\pgfpathlineto{\pgfqpoint{2.772929in}{3.319163in}}%
\pgfpathlineto{\pgfqpoint{2.773915in}{3.374240in}}%
\pgfpathlineto{\pgfqpoint{2.774902in}{3.212227in}}%
\pgfpathlineto{\pgfqpoint{2.775395in}{3.285400in}}%
\pgfpathlineto{\pgfqpoint{2.776381in}{3.281901in}}%
\pgfpathlineto{\pgfqpoint{2.778848in}{3.281901in}}%
\pgfpathlineto{\pgfqpoint{2.779834in}{3.369326in}}%
\pgfpathlineto{\pgfqpoint{2.780327in}{2.535123in}}%
\pgfpathlineto{\pgfqpoint{2.780820in}{3.295359in}}%
\pgfpathlineto{\pgfqpoint{2.781314in}{3.295359in}}%
\pgfpathlineto{\pgfqpoint{2.781807in}{3.004777in}}%
\pgfpathlineto{\pgfqpoint{2.782300in}{3.343935in}}%
\pgfpathlineto{\pgfqpoint{2.785260in}{3.343935in}}%
\pgfpathlineto{\pgfqpoint{2.785753in}{2.544569in}}%
\pgfpathlineto{\pgfqpoint{2.786246in}{3.110364in}}%
\pgfpathlineto{\pgfqpoint{2.787726in}{3.094138in}}%
\pgfpathlineto{\pgfqpoint{2.788219in}{3.094138in}}%
\pgfpathlineto{\pgfqpoint{2.789699in}{3.291851in}}%
\pgfpathlineto{\pgfqpoint{2.790192in}{2.347922in}}%
\pgfpathlineto{\pgfqpoint{2.790685in}{3.379623in}}%
\pgfpathlineto{\pgfqpoint{2.791178in}{3.379623in}}%
\pgfpathlineto{\pgfqpoint{2.791672in}{2.832764in}}%
\pgfpathlineto{\pgfqpoint{2.792165in}{3.347985in}}%
\pgfpathlineto{\pgfqpoint{2.793644in}{3.347985in}}%
\pgfpathlineto{\pgfqpoint{2.794138in}{3.200900in}}%
\pgfpathlineto{\pgfqpoint{2.794631in}{3.331191in}}%
\pgfpathlineto{\pgfqpoint{2.796604in}{3.331191in}}%
\pgfpathlineto{\pgfqpoint{2.798084in}{3.361817in}}%
\pgfpathlineto{\pgfqpoint{2.799070in}{3.307053in}}%
\pgfpathlineto{\pgfqpoint{2.799563in}{2.818585in}}%
\pgfpathlineto{\pgfqpoint{2.800056in}{3.181034in}}%
\pgfpathlineto{\pgfqpoint{2.801043in}{3.376231in}}%
\pgfpathlineto{\pgfqpoint{2.801536in}{3.128752in}}%
\pgfpathlineto{\pgfqpoint{2.802029in}{3.227314in}}%
\pgfpathlineto{\pgfqpoint{2.804496in}{3.227314in}}%
\pgfpathlineto{\pgfqpoint{2.805975in}{3.313923in}}%
\pgfpathlineto{\pgfqpoint{2.808935in}{3.313923in}}%
\pgfpathlineto{\pgfqpoint{2.810414in}{3.379837in}}%
\pgfpathlineto{\pgfqpoint{2.811894in}{3.202996in}}%
\pgfpathlineto{\pgfqpoint{2.813374in}{3.202996in}}%
\pgfpathlineto{\pgfqpoint{2.813867in}{3.375576in}}%
\pgfpathlineto{\pgfqpoint{2.814360in}{2.960447in}}%
\pgfpathlineto{\pgfqpoint{2.814853in}{3.148420in}}%
\pgfpathlineto{\pgfqpoint{2.815347in}{3.148420in}}%
\pgfpathlineto{\pgfqpoint{2.816826in}{3.355148in}}%
\pgfpathlineto{\pgfqpoint{2.817813in}{3.352670in}}%
\pgfpathlineto{\pgfqpoint{2.819292in}{3.379837in}}%
\pgfpathlineto{\pgfqpoint{2.819786in}{3.296528in}}%
\pgfpathlineto{\pgfqpoint{2.820279in}{3.328075in}}%
\pgfpathlineto{\pgfqpoint{2.821265in}{3.328075in}}%
\pgfpathlineto{\pgfqpoint{2.821759in}{3.017529in}}%
\pgfpathlineto{\pgfqpoint{2.822252in}{3.295302in}}%
\pgfpathlineto{\pgfqpoint{2.824718in}{3.295302in}}%
\pgfpathlineto{\pgfqpoint{2.825705in}{3.223166in}}%
\pgfpathlineto{\pgfqpoint{2.826691in}{3.376218in}}%
\pgfpathlineto{\pgfqpoint{2.827184in}{3.358305in}}%
\pgfpathlineto{\pgfqpoint{2.828664in}{3.365750in}}%
\pgfpathlineto{\pgfqpoint{2.829650in}{3.365750in}}%
\pgfpathlineto{\pgfqpoint{2.830637in}{3.189288in}}%
\pgfpathlineto{\pgfqpoint{2.832117in}{3.356631in}}%
\pgfpathlineto{\pgfqpoint{2.834583in}{2.892168in}}%
\pgfpathlineto{\pgfqpoint{2.835076in}{2.892168in}}%
\pgfpathlineto{\pgfqpoint{2.836556in}{2.727480in}}%
\pgfpathlineto{\pgfqpoint{2.838035in}{3.345744in}}%
\pgfpathlineto{\pgfqpoint{2.839515in}{3.345744in}}%
\pgfpathlineto{\pgfqpoint{2.840995in}{3.375073in}}%
\pgfpathlineto{\pgfqpoint{2.841488in}{3.375073in}}%
\pgfpathlineto{\pgfqpoint{2.842474in}{3.358534in}}%
\pgfpathlineto{\pgfqpoint{2.843954in}{3.376216in}}%
\pgfpathlineto{\pgfqpoint{2.844941in}{3.376216in}}%
\pgfpathlineto{\pgfqpoint{2.845927in}{3.071438in}}%
\pgfpathlineto{\pgfqpoint{2.847407in}{3.292297in}}%
\pgfpathlineto{\pgfqpoint{2.847900in}{3.292297in}}%
\pgfpathlineto{\pgfqpoint{2.848393in}{3.379996in}}%
\pgfpathlineto{\pgfqpoint{2.849873in}{3.236233in}}%
\pgfpathlineto{\pgfqpoint{2.850366in}{3.336686in}}%
\pgfpathlineto{\pgfqpoint{2.852339in}{2.152562in}}%
\pgfpathlineto{\pgfqpoint{2.853819in}{2.983748in}}%
\pgfpathlineto{\pgfqpoint{2.855792in}{2.983748in}}%
\pgfpathlineto{\pgfqpoint{2.857271in}{3.379953in}}%
\pgfpathlineto{\pgfqpoint{2.859244in}{3.379953in}}%
\pgfpathlineto{\pgfqpoint{2.860724in}{3.171811in}}%
\pgfpathlineto{\pgfqpoint{2.861217in}{3.374212in}}%
\pgfpathlineto{\pgfqpoint{2.861710in}{3.319518in}}%
\pgfpathlineto{\pgfqpoint{2.863190in}{3.319518in}}%
\pgfpathlineto{\pgfqpoint{2.864670in}{3.121447in}}%
\pgfpathlineto{\pgfqpoint{2.865656in}{3.053116in}}%
\pgfpathlineto{\pgfqpoint{2.867136in}{3.354579in}}%
\pgfpathlineto{\pgfqpoint{2.868616in}{2.528549in}}%
\pgfpathlineto{\pgfqpoint{2.869602in}{2.528549in}}%
\pgfpathlineto{\pgfqpoint{2.871082in}{3.278855in}}%
\pgfpathlineto{\pgfqpoint{2.873548in}{3.278855in}}%
\pgfpathlineto{\pgfqpoint{2.875028in}{3.347590in}}%
\pgfpathlineto{\pgfqpoint{2.876014in}{3.287103in}}%
\pgfpathlineto{\pgfqpoint{2.877987in}{3.366583in}}%
\pgfpathlineto{\pgfqpoint{2.879467in}{1.888345in}}%
\pgfpathlineto{\pgfqpoint{2.881440in}{3.373209in}}%
\pgfpathlineto{\pgfqpoint{2.883906in}{3.373209in}}%
\pgfpathlineto{\pgfqpoint{2.884892in}{3.341845in}}%
\pgfpathlineto{\pgfqpoint{2.885879in}{3.379760in}}%
\pgfpathlineto{\pgfqpoint{2.886372in}{3.307493in}}%
\pgfpathlineto{\pgfqpoint{2.887852in}{2.952689in}}%
\pgfpathlineto{\pgfqpoint{2.889331in}{3.207158in}}%
\pgfpathlineto{\pgfqpoint{2.890318in}{3.207158in}}%
\pgfpathlineto{\pgfqpoint{2.891797in}{3.375282in}}%
\pgfpathlineto{\pgfqpoint{2.892784in}{3.375282in}}%
\pgfpathlineto{\pgfqpoint{2.893770in}{2.791192in}}%
\pgfpathlineto{\pgfqpoint{2.895250in}{3.285196in}}%
\pgfpathlineto{\pgfqpoint{2.896237in}{3.285196in}}%
\pgfpathlineto{\pgfqpoint{2.896730in}{3.367303in}}%
\pgfpathlineto{\pgfqpoint{2.897716in}{3.356879in}}%
\pgfpathlineto{\pgfqpoint{2.899196in}{3.136239in}}%
\pgfpathlineto{\pgfqpoint{2.899689in}{3.136239in}}%
\pgfpathlineto{\pgfqpoint{2.900182in}{3.097498in}}%
\pgfpathlineto{\pgfqpoint{2.901662in}{3.343184in}}%
\pgfpathlineto{\pgfqpoint{2.902155in}{3.343184in}}%
\pgfpathlineto{\pgfqpoint{2.903635in}{3.254993in}}%
\pgfpathlineto{\pgfqpoint{2.904128in}{3.254993in}}%
\pgfpathlineto{\pgfqpoint{2.905115in}{3.377036in}}%
\pgfpathlineto{\pgfqpoint{2.906594in}{3.201512in}}%
\pgfpathlineto{\pgfqpoint{2.907088in}{3.201512in}}%
\pgfpathlineto{\pgfqpoint{2.908567in}{3.362705in}}%
\pgfpathlineto{\pgfqpoint{2.909061in}{3.362705in}}%
\pgfpathlineto{\pgfqpoint{2.910047in}{2.975986in}}%
\pgfpathlineto{\pgfqpoint{2.910540in}{3.274249in}}%
\pgfpathlineto{\pgfqpoint{2.911527in}{3.239219in}}%
\pgfpathlineto{\pgfqpoint{2.915473in}{3.239219in}}%
\pgfpathlineto{\pgfqpoint{2.915966in}{3.345226in}}%
\pgfpathlineto{\pgfqpoint{2.916459in}{3.263911in}}%
\pgfpathlineto{\pgfqpoint{2.918925in}{3.263911in}}%
\pgfpathlineto{\pgfqpoint{2.919418in}{2.703006in}}%
\pgfpathlineto{\pgfqpoint{2.919912in}{3.136853in}}%
\pgfpathlineto{\pgfqpoint{2.920405in}{3.150845in}}%
\pgfpathlineto{\pgfqpoint{2.921885in}{3.327942in}}%
\pgfpathlineto{\pgfqpoint{2.923364in}{3.379400in}}%
\pgfpathlineto{\pgfqpoint{2.923857in}{3.160492in}}%
\pgfpathlineto{\pgfqpoint{2.924351in}{3.297042in}}%
\pgfpathlineto{\pgfqpoint{2.924844in}{3.297042in}}%
\pgfpathlineto{\pgfqpoint{2.925830in}{3.298373in}}%
\pgfpathlineto{\pgfqpoint{2.927310in}{3.136332in}}%
\pgfpathlineto{\pgfqpoint{2.927803in}{3.136332in}}%
\pgfpathlineto{\pgfqpoint{2.928790in}{2.841111in}}%
\pgfpathlineto{\pgfqpoint{2.930270in}{3.358902in}}%
\pgfpathlineto{\pgfqpoint{2.930763in}{3.358902in}}%
\pgfpathlineto{\pgfqpoint{2.931256in}{3.379969in}}%
\pgfpathlineto{\pgfqpoint{2.932242in}{3.257365in}}%
\pgfpathlineto{\pgfqpoint{2.933722in}{3.318740in}}%
\pgfpathlineto{\pgfqpoint{2.934215in}{3.318740in}}%
\pgfpathlineto{\pgfqpoint{2.934709in}{3.206172in}}%
\pgfpathlineto{\pgfqpoint{2.935695in}{3.229258in}}%
\pgfpathlineto{\pgfqpoint{2.937175in}{3.229258in}}%
\pgfpathlineto{\pgfqpoint{2.938161in}{3.252451in}}%
\pgfpathlineto{\pgfqpoint{2.940134in}{2.810173in}}%
\pgfpathlineto{\pgfqpoint{2.941614in}{3.346974in}}%
\pgfpathlineto{\pgfqpoint{2.943094in}{3.334638in}}%
\pgfpathlineto{\pgfqpoint{2.947533in}{3.334638in}}%
\pgfpathlineto{\pgfqpoint{2.949012in}{3.303387in}}%
\pgfpathlineto{\pgfqpoint{2.950985in}{3.303387in}}%
\pgfpathlineto{\pgfqpoint{2.951972in}{3.378282in}}%
\pgfpathlineto{\pgfqpoint{2.953451in}{2.978106in}}%
\pgfpathlineto{\pgfqpoint{2.953945in}{2.978106in}}%
\pgfpathlineto{\pgfqpoint{2.955424in}{3.377503in}}%
\pgfpathlineto{\pgfqpoint{2.959863in}{3.377503in}}%
\pgfpathlineto{\pgfqpoint{2.961343in}{3.177068in}}%
\pgfpathlineto{\pgfqpoint{2.962330in}{3.197894in}}%
\pgfpathlineto{\pgfqpoint{2.962823in}{3.130331in}}%
\pgfpathlineto{\pgfqpoint{2.963809in}{2.725907in}}%
\pgfpathlineto{\pgfqpoint{2.964302in}{3.108003in}}%
\pgfpathlineto{\pgfqpoint{2.965289in}{3.041447in}}%
\pgfpathlineto{\pgfqpoint{2.966769in}{3.041447in}}%
\pgfpathlineto{\pgfqpoint{2.967755in}{2.939135in}}%
\pgfpathlineto{\pgfqpoint{2.969235in}{3.335161in}}%
\pgfpathlineto{\pgfqpoint{2.970714in}{3.335161in}}%
\pgfpathlineto{\pgfqpoint{2.972194in}{3.356400in}}%
\pgfpathlineto{\pgfqpoint{2.973181in}{3.106660in}}%
\pgfpathlineto{\pgfqpoint{2.973674in}{3.380000in}}%
\pgfpathlineto{\pgfqpoint{2.974167in}{3.220742in}}%
\pgfpathlineto{\pgfqpoint{2.975154in}{3.220742in}}%
\pgfpathlineto{\pgfqpoint{2.976140in}{3.321634in}}%
\pgfpathlineto{\pgfqpoint{2.976633in}{3.238988in}}%
\pgfpathlineto{\pgfqpoint{2.978113in}{3.379307in}}%
\pgfpathlineto{\pgfqpoint{2.982552in}{3.379307in}}%
\pgfpathlineto{\pgfqpoint{2.984032in}{3.365582in}}%
\pgfpathlineto{\pgfqpoint{2.985018in}{3.365582in}}%
\pgfpathlineto{\pgfqpoint{2.985511in}{3.379930in}}%
\pgfpathlineto{\pgfqpoint{2.986991in}{3.219542in}}%
\pgfpathlineto{\pgfqpoint{2.987484in}{3.200853in}}%
\pgfpathlineto{\pgfqpoint{2.987978in}{3.042534in}}%
\pgfpathlineto{\pgfqpoint{2.988471in}{3.359035in}}%
\pgfpathlineto{\pgfqpoint{2.988964in}{3.292718in}}%
\pgfpathlineto{\pgfqpoint{2.990444in}{3.240049in}}%
\pgfpathlineto{\pgfqpoint{2.992417in}{3.240049in}}%
\pgfpathlineto{\pgfqpoint{2.993896in}{3.265045in}}%
\pgfpathlineto{\pgfqpoint{2.995869in}{3.265045in}}%
\pgfpathlineto{\pgfqpoint{2.997349in}{3.198145in}}%
\pgfpathlineto{\pgfqpoint{2.997842in}{3.198145in}}%
\pgfpathlineto{\pgfqpoint{2.999322in}{3.377251in}}%
\pgfpathlineto{\pgfqpoint{3.002774in}{3.377251in}}%
\pgfpathlineto{\pgfqpoint{3.004254in}{3.293633in}}%
\pgfpathlineto{\pgfqpoint{3.005734in}{3.379376in}}%
\pgfpathlineto{\pgfqpoint{3.007707in}{3.379376in}}%
\pgfpathlineto{\pgfqpoint{3.008693in}{3.373152in}}%
\pgfpathlineto{\pgfqpoint{3.009680in}{2.549309in}}%
\pgfpathlineto{\pgfqpoint{3.010666in}{3.282218in}}%
\pgfpathlineto{\pgfqpoint{3.011159in}{3.153942in}}%
\pgfpathlineto{\pgfqpoint{3.012146in}{3.153942in}}%
\pgfpathlineto{\pgfqpoint{3.013626in}{2.638117in}}%
\pgfpathlineto{\pgfqpoint{3.015105in}{3.331192in}}%
\pgfpathlineto{\pgfqpoint{3.015598in}{2.971114in}}%
\pgfpathlineto{\pgfqpoint{3.016092in}{3.122917in}}%
\pgfpathlineto{\pgfqpoint{3.017571in}{3.379647in}}%
\pgfpathlineto{\pgfqpoint{3.019051in}{3.377492in}}%
\pgfpathlineto{\pgfqpoint{3.020038in}{3.377492in}}%
\pgfpathlineto{\pgfqpoint{3.022010in}{2.975032in}}%
\pgfpathlineto{\pgfqpoint{3.023490in}{3.068797in}}%
\pgfpathlineto{\pgfqpoint{3.023983in}{3.068797in}}%
\pgfpathlineto{\pgfqpoint{3.025463in}{3.234986in}}%
\pgfpathlineto{\pgfqpoint{3.026450in}{2.489436in}}%
\pgfpathlineto{\pgfqpoint{3.027929in}{3.379959in}}%
\pgfpathlineto{\pgfqpoint{3.030395in}{3.379959in}}%
\pgfpathlineto{\pgfqpoint{3.030889in}{3.374767in}}%
\pgfpathlineto{\pgfqpoint{3.032368in}{3.346331in}}%
\pgfpathlineto{\pgfqpoint{3.032862in}{3.346331in}}%
\pgfpathlineto{\pgfqpoint{3.033355in}{3.377197in}}%
\pgfpathlineto{\pgfqpoint{3.034835in}{2.922901in}}%
\pgfpathlineto{\pgfqpoint{3.036314in}{3.379229in}}%
\pgfpathlineto{\pgfqpoint{3.036807in}{3.379229in}}%
\pgfpathlineto{\pgfqpoint{3.037301in}{3.316946in}}%
\pgfpathlineto{\pgfqpoint{3.037794in}{2.789828in}}%
\pgfpathlineto{\pgfqpoint{3.038287in}{3.376781in}}%
\pgfpathlineto{\pgfqpoint{3.038780in}{3.376781in}}%
\pgfpathlineto{\pgfqpoint{3.040260in}{3.374974in}}%
\pgfpathlineto{\pgfqpoint{3.045192in}{3.374974in}}%
\pgfpathlineto{\pgfqpoint{3.046179in}{2.694137in}}%
\pgfpathlineto{\pgfqpoint{3.047165in}{3.293814in}}%
\pgfpathlineto{\pgfqpoint{3.047659in}{3.115986in}}%
\pgfpathlineto{\pgfqpoint{3.048152in}{3.347176in}}%
\pgfpathlineto{\pgfqpoint{3.048645in}{3.224103in}}%
\pgfpathlineto{\pgfqpoint{3.049631in}{3.224103in}}%
\pgfpathlineto{\pgfqpoint{3.051111in}{3.378567in}}%
\pgfpathlineto{\pgfqpoint{3.052591in}{3.375392in}}%
\pgfpathlineto{\pgfqpoint{3.053084in}{3.342967in}}%
\pgfpathlineto{\pgfqpoint{3.053577in}{3.088343in}}%
\pgfpathlineto{\pgfqpoint{3.054071in}{3.347069in}}%
\pgfpathlineto{\pgfqpoint{3.057523in}{3.347069in}}%
\pgfpathlineto{\pgfqpoint{3.058016in}{2.800085in}}%
\pgfpathlineto{\pgfqpoint{3.058510in}{3.185168in}}%
\pgfpathlineto{\pgfqpoint{3.060483in}{3.185168in}}%
\pgfpathlineto{\pgfqpoint{3.061962in}{2.948631in}}%
\pgfpathlineto{\pgfqpoint{3.062455in}{2.948631in}}%
\pgfpathlineto{\pgfqpoint{3.063935in}{3.363931in}}%
\pgfpathlineto{\pgfqpoint{3.064922in}{3.054553in}}%
\pgfpathlineto{\pgfqpoint{3.066401in}{3.369450in}}%
\pgfpathlineto{\pgfqpoint{3.066895in}{3.369450in}}%
\pgfpathlineto{\pgfqpoint{3.068867in}{2.912079in}}%
\pgfpathlineto{\pgfqpoint{3.069361in}{2.912079in}}%
\pgfpathlineto{\pgfqpoint{3.069854in}{3.016250in}}%
\pgfpathlineto{\pgfqpoint{3.071334in}{3.363289in}}%
\pgfpathlineto{\pgfqpoint{3.072320in}{3.375954in}}%
\pgfpathlineto{\pgfqpoint{3.073800in}{2.796662in}}%
\pgfpathlineto{\pgfqpoint{3.074293in}{2.796662in}}%
\pgfpathlineto{\pgfqpoint{3.076266in}{3.020494in}}%
\pgfpathlineto{\pgfqpoint{3.076759in}{2.984006in}}%
\pgfpathlineto{\pgfqpoint{3.078239in}{3.318043in}}%
\pgfpathlineto{\pgfqpoint{3.080212in}{3.318043in}}%
\pgfpathlineto{\pgfqpoint{3.080705in}{3.085425in}}%
\pgfpathlineto{\pgfqpoint{3.081198in}{3.119252in}}%
\pgfpathlineto{\pgfqpoint{3.082678in}{3.374417in}}%
\pgfpathlineto{\pgfqpoint{3.085144in}{3.374417in}}%
\pgfpathlineto{\pgfqpoint{3.085637in}{3.017554in}}%
\pgfpathlineto{\pgfqpoint{3.086131in}{3.261270in}}%
\pgfpathlineto{\pgfqpoint{3.087610in}{3.201309in}}%
\pgfpathlineto{\pgfqpoint{3.088103in}{2.866348in}}%
\pgfpathlineto{\pgfqpoint{3.089583in}{3.358508in}}%
\pgfpathlineto{\pgfqpoint{3.090570in}{3.358508in}}%
\pgfpathlineto{\pgfqpoint{3.092049in}{3.328183in}}%
\pgfpathlineto{\pgfqpoint{3.092543in}{3.328183in}}%
\pgfpathlineto{\pgfqpoint{3.094022in}{3.265135in}}%
\pgfpathlineto{\pgfqpoint{3.095009in}{3.265135in}}%
\pgfpathlineto{\pgfqpoint{3.096488in}{3.162324in}}%
\pgfpathlineto{\pgfqpoint{3.096982in}{3.162324in}}%
\pgfpathlineto{\pgfqpoint{3.097968in}{3.150154in}}%
\pgfpathlineto{\pgfqpoint{3.099448in}{2.957125in}}%
\pgfpathlineto{\pgfqpoint{3.100927in}{3.378458in}}%
\pgfpathlineto{\pgfqpoint{3.102407in}{3.276332in}}%
\pgfpathlineto{\pgfqpoint{3.103887in}{3.276332in}}%
\pgfpathlineto{\pgfqpoint{3.105367in}{3.208459in}}%
\pgfpathlineto{\pgfqpoint{3.107339in}{3.208459in}}%
\pgfpathlineto{\pgfqpoint{3.108326in}{3.275853in}}%
\pgfpathlineto{\pgfqpoint{3.108819in}{3.267510in}}%
\pgfpathlineto{\pgfqpoint{3.109312in}{3.267510in}}%
\pgfpathlineto{\pgfqpoint{3.110792in}{3.370888in}}%
\pgfpathlineto{\pgfqpoint{3.112272in}{3.370888in}}%
\pgfpathlineto{\pgfqpoint{3.113258in}{3.378609in}}%
\pgfpathlineto{\pgfqpoint{3.114738in}{3.346375in}}%
\pgfpathlineto{\pgfqpoint{3.115231in}{2.672055in}}%
\pgfpathlineto{\pgfqpoint{3.115724in}{3.378306in}}%
\pgfpathlineto{\pgfqpoint{3.117204in}{3.339130in}}%
\pgfpathlineto{\pgfqpoint{3.118684in}{3.339130in}}%
\pgfpathlineto{\pgfqpoint{3.120163in}{3.350498in}}%
\pgfpathlineto{\pgfqpoint{3.124603in}{3.350498in}}%
\pgfpathlineto{\pgfqpoint{3.125589in}{2.878522in}}%
\pgfpathlineto{\pgfqpoint{3.127069in}{3.337546in}}%
\pgfpathlineto{\pgfqpoint{3.128055in}{3.337546in}}%
\pgfpathlineto{\pgfqpoint{3.129535in}{2.993652in}}%
\pgfpathlineto{\pgfqpoint{3.132001in}{3.370577in}}%
\pgfpathlineto{\pgfqpoint{3.132988in}{3.370577in}}%
\pgfpathlineto{\pgfqpoint{3.133974in}{3.380000in}}%
\pgfpathlineto{\pgfqpoint{3.135454in}{3.137844in}}%
\pgfpathlineto{\pgfqpoint{3.136933in}{3.346253in}}%
\pgfpathlineto{\pgfqpoint{3.138413in}{3.346253in}}%
\pgfpathlineto{\pgfqpoint{3.138906in}{3.260313in}}%
\pgfpathlineto{\pgfqpoint{3.139400in}{2.906823in}}%
\pgfpathlineto{\pgfqpoint{3.139893in}{3.335133in}}%
\pgfpathlineto{\pgfqpoint{3.141372in}{3.292740in}}%
\pgfpathlineto{\pgfqpoint{3.143345in}{3.292740in}}%
\pgfpathlineto{\pgfqpoint{3.144825in}{2.873304in}}%
\pgfpathlineto{\pgfqpoint{3.145318in}{2.873304in}}%
\pgfpathlineto{\pgfqpoint{3.146798in}{3.360885in}}%
\pgfpathlineto{\pgfqpoint{3.148278in}{3.360885in}}%
\pgfpathlineto{\pgfqpoint{3.149264in}{3.333136in}}%
\pgfpathlineto{\pgfqpoint{3.149757in}{2.586567in}}%
\pgfpathlineto{\pgfqpoint{3.150251in}{3.359870in}}%
\pgfpathlineto{\pgfqpoint{3.150744in}{3.359870in}}%
\pgfpathlineto{\pgfqpoint{3.151730in}{3.374224in}}%
\pgfpathlineto{\pgfqpoint{3.152717in}{3.370436in}}%
\pgfpathlineto{\pgfqpoint{3.154196in}{3.088462in}}%
\pgfpathlineto{\pgfqpoint{3.155676in}{3.245427in}}%
\pgfpathlineto{\pgfqpoint{3.156169in}{3.202812in}}%
\pgfpathlineto{\pgfqpoint{3.157156in}{3.212042in}}%
\pgfpathlineto{\pgfqpoint{3.158142in}{3.212042in}}%
\pgfpathlineto{\pgfqpoint{3.159129in}{3.379215in}}%
\pgfpathlineto{\pgfqpoint{3.160115in}{3.237966in}}%
\pgfpathlineto{\pgfqpoint{3.161595in}{3.325567in}}%
\pgfpathlineto{\pgfqpoint{3.163075in}{2.824934in}}%
\pgfpathlineto{\pgfqpoint{3.165048in}{3.220700in}}%
\pgfpathlineto{\pgfqpoint{3.165541in}{3.109279in}}%
\pgfpathlineto{\pgfqpoint{3.166034in}{3.237422in}}%
\pgfpathlineto{\pgfqpoint{3.166527in}{3.237422in}}%
\pgfpathlineto{\pgfqpoint{3.168007in}{3.348237in}}%
\pgfpathlineto{\pgfqpoint{3.168500in}{3.348237in}}%
\pgfpathlineto{\pgfqpoint{3.169980in}{3.360188in}}%
\pgfpathlineto{\pgfqpoint{3.170473in}{3.324204in}}%
\pgfpathlineto{\pgfqpoint{3.170966in}{2.605668in}}%
\pgfpathlineto{\pgfqpoint{3.171460in}{2.984075in}}%
\pgfpathlineto{\pgfqpoint{3.172446in}{3.379867in}}%
\pgfpathlineto{\pgfqpoint{3.172939in}{3.371623in}}%
\pgfpathlineto{\pgfqpoint{3.173926in}{3.347572in}}%
\pgfpathlineto{\pgfqpoint{3.175405in}{3.361372in}}%
\pgfpathlineto{\pgfqpoint{3.176885in}{3.181299in}}%
\pgfpathlineto{\pgfqpoint{3.177872in}{3.181299in}}%
\pgfpathlineto{\pgfqpoint{3.178365in}{3.360939in}}%
\pgfpathlineto{\pgfqpoint{3.178858in}{3.296389in}}%
\pgfpathlineto{\pgfqpoint{3.179351in}{3.296389in}}%
\pgfpathlineto{\pgfqpoint{3.180831in}{3.378627in}}%
\pgfpathlineto{\pgfqpoint{3.181817in}{3.378627in}}%
\pgfpathlineto{\pgfqpoint{3.183297in}{3.314443in}}%
\pgfpathlineto{\pgfqpoint{3.184777in}{3.314443in}}%
\pgfpathlineto{\pgfqpoint{3.185270in}{3.375222in}}%
\pgfpathlineto{\pgfqpoint{3.186750in}{3.018557in}}%
\pgfpathlineto{\pgfqpoint{3.187736in}{3.018557in}}%
\pgfpathlineto{\pgfqpoint{3.188723in}{3.269745in}}%
\pgfpathlineto{\pgfqpoint{3.189216in}{2.995626in}}%
\pgfpathlineto{\pgfqpoint{3.189709in}{3.048156in}}%
\pgfpathlineto{\pgfqpoint{3.190202in}{3.367828in}}%
\pgfpathlineto{\pgfqpoint{3.191189in}{3.340498in}}%
\pgfpathlineto{\pgfqpoint{3.192668in}{3.340498in}}%
\pgfpathlineto{\pgfqpoint{3.193162in}{1.579553in}}%
\pgfpathlineto{\pgfqpoint{3.193655in}{2.906287in}}%
\pgfpathlineto{\pgfqpoint{3.194148in}{2.801349in}}%
\pgfpathlineto{\pgfqpoint{3.195135in}{1.704265in}}%
\pgfpathlineto{\pgfqpoint{3.196614in}{3.285698in}}%
\pgfpathlineto{\pgfqpoint{3.197601in}{3.272511in}}%
\pgfpathlineto{\pgfqpoint{3.199080in}{3.362282in}}%
\pgfpathlineto{\pgfqpoint{3.200560in}{2.957175in}}%
\pgfpathlineto{\pgfqpoint{3.201053in}{3.302318in}}%
\pgfpathlineto{\pgfqpoint{3.201547in}{2.950188in}}%
\pgfpathlineto{\pgfqpoint{3.202040in}{2.950188in}}%
\pgfpathlineto{\pgfqpoint{3.203520in}{3.154563in}}%
\pgfpathlineto{\pgfqpoint{3.204013in}{3.151263in}}%
\pgfpathlineto{\pgfqpoint{3.204506in}{2.943538in}}%
\pgfpathlineto{\pgfqpoint{3.204999in}{3.183729in}}%
\pgfpathlineto{\pgfqpoint{3.206479in}{3.361620in}}%
\pgfpathlineto{\pgfqpoint{3.207465in}{3.361620in}}%
\pgfpathlineto{\pgfqpoint{3.207959in}{2.772621in}}%
\pgfpathlineto{\pgfqpoint{3.208452in}{3.245351in}}%
\pgfpathlineto{\pgfqpoint{3.209438in}{3.245351in}}%
\pgfpathlineto{\pgfqpoint{3.210918in}{3.345031in}}%
\pgfpathlineto{\pgfqpoint{3.214371in}{3.345031in}}%
\pgfpathlineto{\pgfqpoint{3.215357in}{1.926589in}}%
\pgfpathlineto{\pgfqpoint{3.216837in}{3.204929in}}%
\pgfpathlineto{\pgfqpoint{3.217330in}{3.379985in}}%
\pgfpathlineto{\pgfqpoint{3.217823in}{3.187842in}}%
\pgfpathlineto{\pgfqpoint{3.218316in}{3.187842in}}%
\pgfpathlineto{\pgfqpoint{3.219303in}{3.317507in}}%
\pgfpathlineto{\pgfqpoint{3.220783in}{3.050089in}}%
\pgfpathlineto{\pgfqpoint{3.222756in}{3.050089in}}%
\pgfpathlineto{\pgfqpoint{3.223742in}{3.278526in}}%
\pgfpathlineto{\pgfqpoint{3.224235in}{3.210545in}}%
\pgfpathlineto{\pgfqpoint{3.224728in}{3.289261in}}%
\pgfpathlineto{\pgfqpoint{3.225222in}{3.289261in}}%
\pgfpathlineto{\pgfqpoint{3.226208in}{3.243811in}}%
\pgfpathlineto{\pgfqpoint{3.227688in}{3.350923in}}%
\pgfpathlineto{\pgfqpoint{3.228181in}{3.350923in}}%
\pgfpathlineto{\pgfqpoint{3.229661in}{3.370095in}}%
\pgfpathlineto{\pgfqpoint{3.230647in}{3.300361in}}%
\pgfpathlineto{\pgfqpoint{3.232127in}{3.379442in}}%
\pgfpathlineto{\pgfqpoint{3.232620in}{3.379442in}}%
\pgfpathlineto{\pgfqpoint{3.233607in}{3.379694in}}%
\pgfpathlineto{\pgfqpoint{3.234100in}{3.188546in}}%
\pgfpathlineto{\pgfqpoint{3.234593in}{3.285819in}}%
\pgfpathlineto{\pgfqpoint{3.235086in}{2.128182in}}%
\pgfpathlineto{\pgfqpoint{3.235580in}{2.392343in}}%
\pgfpathlineto{\pgfqpoint{3.237553in}{3.183332in}}%
\pgfpathlineto{\pgfqpoint{3.238539in}{3.183332in}}%
\pgfpathlineto{\pgfqpoint{3.239032in}{3.280940in}}%
\pgfpathlineto{\pgfqpoint{3.239525in}{3.203709in}}%
\pgfpathlineto{\pgfqpoint{3.243965in}{3.203709in}}%
\pgfpathlineto{\pgfqpoint{3.245444in}{3.307304in}}%
\pgfpathlineto{\pgfqpoint{3.246431in}{3.307304in}}%
\pgfpathlineto{\pgfqpoint{3.246924in}{3.361879in}}%
\pgfpathlineto{\pgfqpoint{3.247417in}{2.571010in}}%
\pgfpathlineto{\pgfqpoint{3.247910in}{3.056536in}}%
\pgfpathlineto{\pgfqpoint{3.249390in}{3.372774in}}%
\pgfpathlineto{\pgfqpoint{3.249883in}{3.372774in}}%
\pgfpathlineto{\pgfqpoint{3.250870in}{3.379922in}}%
\pgfpathlineto{\pgfqpoint{3.251363in}{3.262114in}}%
\pgfpathlineto{\pgfqpoint{3.252843in}{2.872186in}}%
\pgfpathlineto{\pgfqpoint{3.254322in}{3.372303in}}%
\pgfpathlineto{\pgfqpoint{3.255309in}{3.016828in}}%
\pgfpathlineto{\pgfqpoint{3.256789in}{3.198960in}}%
\pgfpathlineto{\pgfqpoint{3.258268in}{3.224682in}}%
\pgfpathlineto{\pgfqpoint{3.259255in}{3.224682in}}%
\pgfpathlineto{\pgfqpoint{3.260734in}{3.297061in}}%
\pgfpathlineto{\pgfqpoint{3.261228in}{3.297061in}}%
\pgfpathlineto{\pgfqpoint{3.262707in}{3.176069in}}%
\pgfpathlineto{\pgfqpoint{3.263201in}{3.319822in}}%
\pgfpathlineto{\pgfqpoint{3.263694in}{3.311804in}}%
\pgfpathlineto{\pgfqpoint{3.264187in}{3.130407in}}%
\pgfpathlineto{\pgfqpoint{3.264680in}{3.318838in}}%
\pgfpathlineto{\pgfqpoint{3.270599in}{3.318838in}}%
\pgfpathlineto{\pgfqpoint{3.271585in}{2.811441in}}%
\pgfpathlineto{\pgfqpoint{3.273065in}{3.372832in}}%
\pgfpathlineto{\pgfqpoint{3.274545in}{2.674658in}}%
\pgfpathlineto{\pgfqpoint{3.275038in}{2.674658in}}%
\pgfpathlineto{\pgfqpoint{3.276518in}{3.376015in}}%
\pgfpathlineto{\pgfqpoint{3.277504in}{3.354111in}}%
\pgfpathlineto{\pgfqpoint{3.277997in}{3.354111in}}%
\pgfpathlineto{\pgfqpoint{3.278491in}{3.110144in}}%
\pgfpathlineto{\pgfqpoint{3.278984in}{3.318241in}}%
\pgfpathlineto{\pgfqpoint{3.281450in}{3.318241in}}%
\pgfpathlineto{\pgfqpoint{3.282930in}{3.302829in}}%
\pgfpathlineto{\pgfqpoint{3.283916in}{3.302829in}}%
\pgfpathlineto{\pgfqpoint{3.285396in}{3.371926in}}%
\pgfpathlineto{\pgfqpoint{3.286876in}{2.453094in}}%
\pgfpathlineto{\pgfqpoint{3.288355in}{3.368646in}}%
\pgfpathlineto{\pgfqpoint{3.289835in}{2.440372in}}%
\pgfpathlineto{\pgfqpoint{3.291315in}{3.074684in}}%
\pgfpathlineto{\pgfqpoint{3.291808in}{2.890041in}}%
\pgfpathlineto{\pgfqpoint{3.293288in}{3.379965in}}%
\pgfpathlineto{\pgfqpoint{3.293781in}{2.536071in}}%
\pgfpathlineto{\pgfqpoint{3.294274in}{3.323096in}}%
\pgfpathlineto{\pgfqpoint{3.296740in}{3.323096in}}%
\pgfpathlineto{\pgfqpoint{3.297233in}{3.374910in}}%
\pgfpathlineto{\pgfqpoint{3.297727in}{3.358730in}}%
\pgfpathlineto{\pgfqpoint{3.298220in}{3.358730in}}%
\pgfpathlineto{\pgfqpoint{3.298713in}{3.249844in}}%
\pgfpathlineto{\pgfqpoint{3.299206in}{3.321120in}}%
\pgfpathlineto{\pgfqpoint{3.299700in}{3.352863in}}%
\pgfpathlineto{\pgfqpoint{3.300193in}{3.349217in}}%
\pgfpathlineto{\pgfqpoint{3.301179in}{3.216054in}}%
\pgfpathlineto{\pgfqpoint{3.302659in}{3.329561in}}%
\pgfpathlineto{\pgfqpoint{3.305125in}{3.329561in}}%
\pgfpathlineto{\pgfqpoint{3.306605in}{3.361866in}}%
\pgfpathlineto{\pgfqpoint{3.308085in}{3.361866in}}%
\pgfpathlineto{\pgfqpoint{3.308578in}{3.341316in}}%
\pgfpathlineto{\pgfqpoint{3.309564in}{3.189275in}}%
\pgfpathlineto{\pgfqpoint{3.310057in}{3.193280in}}%
\pgfpathlineto{\pgfqpoint{3.310551in}{3.193280in}}%
\pgfpathlineto{\pgfqpoint{3.312030in}{3.102921in}}%
\pgfpathlineto{\pgfqpoint{3.312524in}{3.102921in}}%
\pgfpathlineto{\pgfqpoint{3.314003in}{3.371194in}}%
\pgfpathlineto{\pgfqpoint{3.317456in}{3.371194in}}%
\pgfpathlineto{\pgfqpoint{3.317949in}{3.172422in}}%
\pgfpathlineto{\pgfqpoint{3.318936in}{3.192202in}}%
\pgfpathlineto{\pgfqpoint{3.320909in}{3.192202in}}%
\pgfpathlineto{\pgfqpoint{3.321402in}{2.567126in}}%
\pgfpathlineto{\pgfqpoint{3.321895in}{3.199666in}}%
\pgfpathlineto{\pgfqpoint{3.322881in}{3.199666in}}%
\pgfpathlineto{\pgfqpoint{3.323868in}{3.369929in}}%
\pgfpathlineto{\pgfqpoint{3.325841in}{2.949889in}}%
\pgfpathlineto{\pgfqpoint{3.326334in}{2.949889in}}%
\pgfpathlineto{\pgfqpoint{3.327814in}{3.378680in}}%
\pgfpathlineto{\pgfqpoint{3.331266in}{3.378680in}}%
\pgfpathlineto{\pgfqpoint{3.332253in}{2.916474in}}%
\pgfpathlineto{\pgfqpoint{3.333733in}{3.375834in}}%
\pgfpathlineto{\pgfqpoint{3.334226in}{3.375834in}}%
\pgfpathlineto{\pgfqpoint{3.334719in}{3.070949in}}%
\pgfpathlineto{\pgfqpoint{3.335212in}{3.378035in}}%
\pgfpathlineto{\pgfqpoint{3.336199in}{3.378035in}}%
\pgfpathlineto{\pgfqpoint{3.337678in}{3.375291in}}%
\pgfpathlineto{\pgfqpoint{3.338665in}{3.375291in}}%
\pgfpathlineto{\pgfqpoint{3.339158in}{3.332082in}}%
\pgfpathlineto{\pgfqpoint{3.340638in}{3.047302in}}%
\pgfpathlineto{\pgfqpoint{3.342118in}{3.047302in}}%
\pgfpathlineto{\pgfqpoint{3.343597in}{3.265959in}}%
\pgfpathlineto{\pgfqpoint{3.345077in}{3.379253in}}%
\pgfpathlineto{\pgfqpoint{3.345570in}{3.379253in}}%
\pgfpathlineto{\pgfqpoint{3.346557in}{3.105326in}}%
\pgfpathlineto{\pgfqpoint{3.348036in}{3.341957in}}%
\pgfpathlineto{\pgfqpoint{3.349516in}{3.341957in}}%
\pgfpathlineto{\pgfqpoint{3.350996in}{3.319720in}}%
\pgfpathlineto{\pgfqpoint{3.352969in}{3.319720in}}%
\pgfpathlineto{\pgfqpoint{3.353462in}{3.379874in}}%
\pgfpathlineto{\pgfqpoint{3.353955in}{3.255224in}}%
\pgfpathlineto{\pgfqpoint{3.354448in}{3.373690in}}%
\pgfpathlineto{\pgfqpoint{3.354942in}{3.373690in}}%
\pgfpathlineto{\pgfqpoint{3.355928in}{2.983312in}}%
\pgfpathlineto{\pgfqpoint{3.356421in}{3.075226in}}%
\pgfpathlineto{\pgfqpoint{3.358887in}{3.075226in}}%
\pgfpathlineto{\pgfqpoint{3.360367in}{3.128751in}}%
\pgfpathlineto{\pgfqpoint{3.361354in}{3.128751in}}%
\pgfpathlineto{\pgfqpoint{3.362340in}{2.743712in}}%
\pgfpathlineto{\pgfqpoint{3.363820in}{3.344229in}}%
\pgfpathlineto{\pgfqpoint{3.364806in}{3.371704in}}%
\pgfpathlineto{\pgfqpoint{3.366286in}{3.252943in}}%
\pgfpathlineto{\pgfqpoint{3.366779in}{3.252943in}}%
\pgfpathlineto{\pgfqpoint{3.368259in}{2.767410in}}%
\pgfpathlineto{\pgfqpoint{3.369245in}{3.379668in}}%
\pgfpathlineto{\pgfqpoint{3.369738in}{3.315981in}}%
\pgfpathlineto{\pgfqpoint{3.370232in}{3.315981in}}%
\pgfpathlineto{\pgfqpoint{3.370725in}{3.370624in}}%
\pgfpathlineto{\pgfqpoint{3.371711in}{3.104118in}}%
\pgfpathlineto{\pgfqpoint{3.373191in}{3.224653in}}%
\pgfpathlineto{\pgfqpoint{3.375164in}{3.224653in}}%
\pgfpathlineto{\pgfqpoint{3.376644in}{2.883233in}}%
\pgfpathlineto{\pgfqpoint{3.378123in}{3.344365in}}%
\pgfpathlineto{\pgfqpoint{3.378617in}{3.344365in}}%
\pgfpathlineto{\pgfqpoint{3.379110in}{3.133274in}}%
\pgfpathlineto{\pgfqpoint{3.379603in}{3.378284in}}%
\pgfpathlineto{\pgfqpoint{3.381083in}{3.378284in}}%
\pgfpathlineto{\pgfqpoint{3.381576in}{3.004059in}}%
\pgfpathlineto{\pgfqpoint{3.382069in}{3.132932in}}%
\pgfpathlineto{\pgfqpoint{3.382562in}{3.132932in}}%
\pgfpathlineto{\pgfqpoint{3.385029in}{3.376801in}}%
\pgfpathlineto{\pgfqpoint{3.386508in}{3.376700in}}%
\pgfpathlineto{\pgfqpoint{3.387495in}{3.271354in}}%
\pgfpathlineto{\pgfqpoint{3.387988in}{3.358754in}}%
\pgfpathlineto{\pgfqpoint{3.389468in}{2.452784in}}%
\pgfpathlineto{\pgfqpoint{3.390947in}{3.312261in}}%
\pgfpathlineto{\pgfqpoint{3.391934in}{3.312261in}}%
\pgfpathlineto{\pgfqpoint{3.392427in}{3.261412in}}%
\pgfpathlineto{\pgfqpoint{3.392920in}{3.295969in}}%
\pgfpathlineto{\pgfqpoint{3.393907in}{2.513482in}}%
\pgfpathlineto{\pgfqpoint{3.395386in}{3.376250in}}%
\pgfpathlineto{\pgfqpoint{3.395880in}{3.261457in}}%
\pgfpathlineto{\pgfqpoint{3.396373in}{3.313110in}}%
\pgfpathlineto{\pgfqpoint{3.397853in}{3.313110in}}%
\pgfpathlineto{\pgfqpoint{3.398346in}{3.379955in}}%
\pgfpathlineto{\pgfqpoint{3.400319in}{2.892446in}}%
\pgfpathlineto{\pgfqpoint{3.400812in}{3.377582in}}%
\pgfpathlineto{\pgfqpoint{3.401798in}{3.324536in}}%
\pgfpathlineto{\pgfqpoint{3.402292in}{3.324536in}}%
\pgfpathlineto{\pgfqpoint{3.403771in}{3.355369in}}%
\pgfpathlineto{\pgfqpoint{3.405744in}{3.355369in}}%
\pgfpathlineto{\pgfqpoint{3.408704in}{2.699456in}}%
\pgfpathlineto{\pgfqpoint{3.409690in}{2.699456in}}%
\pgfpathlineto{\pgfqpoint{3.411663in}{3.344379in}}%
\pgfpathlineto{\pgfqpoint{3.414622in}{3.344379in}}%
\pgfpathlineto{\pgfqpoint{3.416102in}{3.292431in}}%
\pgfpathlineto{\pgfqpoint{3.417089in}{3.292431in}}%
\pgfpathlineto{\pgfqpoint{3.418568in}{2.669245in}}%
\pgfpathlineto{\pgfqpoint{3.419555in}{2.782154in}}%
\pgfpathlineto{\pgfqpoint{3.420048in}{2.782154in}}%
\pgfpathlineto{\pgfqpoint{3.423007in}{3.272427in}}%
\pgfpathlineto{\pgfqpoint{3.423501in}{3.272427in}}%
\pgfpathlineto{\pgfqpoint{3.424487in}{3.360144in}}%
\pgfpathlineto{\pgfqpoint{3.424980in}{2.899237in}}%
\pgfpathlineto{\pgfqpoint{3.425474in}{3.379738in}}%
\pgfpathlineto{\pgfqpoint{3.425967in}{3.379738in}}%
\pgfpathlineto{\pgfqpoint{3.427940in}{3.036328in}}%
\pgfpathlineto{\pgfqpoint{3.428433in}{3.036328in}}%
\pgfpathlineto{\pgfqpoint{3.428926in}{2.536816in}}%
\pgfpathlineto{\pgfqpoint{3.430406in}{3.300106in}}%
\pgfpathlineto{\pgfqpoint{3.432379in}{3.300106in}}%
\pgfpathlineto{\pgfqpoint{3.432872in}{3.283832in}}%
\pgfpathlineto{\pgfqpoint{3.433365in}{3.210283in}}%
\pgfpathlineto{\pgfqpoint{3.434845in}{3.377112in}}%
\pgfpathlineto{\pgfqpoint{3.435338in}{3.377112in}}%
\pgfpathlineto{\pgfqpoint{3.436818in}{3.273018in}}%
\pgfpathlineto{\pgfqpoint{3.437804in}{1.868625in}}%
\pgfpathlineto{\pgfqpoint{3.439284in}{3.356505in}}%
\pgfpathlineto{\pgfqpoint{3.439777in}{3.356505in}}%
\pgfpathlineto{\pgfqpoint{3.441257in}{3.372725in}}%
\pgfpathlineto{\pgfqpoint{3.443230in}{3.372725in}}%
\pgfpathlineto{\pgfqpoint{3.444710in}{3.341962in}}%
\pgfpathlineto{\pgfqpoint{3.446683in}{3.341962in}}%
\pgfpathlineto{\pgfqpoint{3.447669in}{3.058123in}}%
\pgfpathlineto{\pgfqpoint{3.448162in}{3.218930in}}%
\pgfpathlineto{\pgfqpoint{3.449642in}{3.176529in}}%
\pgfpathlineto{\pgfqpoint{3.450135in}{3.176529in}}%
\pgfpathlineto{\pgfqpoint{3.451615in}{3.082966in}}%
\pgfpathlineto{\pgfqpoint{3.452601in}{3.082966in}}%
\pgfpathlineto{\pgfqpoint{3.455561in}{3.378191in}}%
\pgfpathlineto{\pgfqpoint{3.458520in}{3.341217in}}%
\pgfpathlineto{\pgfqpoint{3.461479in}{3.341217in}}%
\pgfpathlineto{\pgfqpoint{3.461973in}{3.378818in}}%
\pgfpathlineto{\pgfqpoint{3.464439in}{3.187926in}}%
\pgfpathlineto{\pgfqpoint{3.464932in}{3.187926in}}%
\pgfpathlineto{\pgfqpoint{3.466412in}{3.100215in}}%
\pgfpathlineto{\pgfqpoint{3.466905in}{3.100215in}}%
\pgfpathlineto{\pgfqpoint{3.468385in}{3.361578in}}%
\pgfpathlineto{\pgfqpoint{3.470851in}{3.259423in}}%
\pgfpathlineto{\pgfqpoint{3.472331in}{2.710024in}}%
\pgfpathlineto{\pgfqpoint{3.473317in}{2.710024in}}%
\pgfpathlineto{\pgfqpoint{3.474797in}{3.376862in}}%
\pgfpathlineto{\pgfqpoint{3.477263in}{3.376862in}}%
\pgfpathlineto{\pgfqpoint{3.478249in}{3.250921in}}%
\pgfpathlineto{\pgfqpoint{3.479729in}{3.344405in}}%
\pgfpathlineto{\pgfqpoint{3.483675in}{3.344405in}}%
\pgfpathlineto{\pgfqpoint{3.485155in}{3.010418in}}%
\pgfpathlineto{\pgfqpoint{3.486141in}{3.056984in}}%
\pgfpathlineto{\pgfqpoint{3.487621in}{3.353289in}}%
\pgfpathlineto{\pgfqpoint{3.489100in}{3.379999in}}%
\pgfpathlineto{\pgfqpoint{3.490087in}{3.379999in}}%
\pgfpathlineto{\pgfqpoint{3.494033in}{3.164254in}}%
\pgfpathlineto{\pgfqpoint{3.494526in}{3.167414in}}%
\pgfpathlineto{\pgfqpoint{3.496006in}{3.379689in}}%
\pgfpathlineto{\pgfqpoint{3.497979in}{3.131527in}}%
\pgfpathlineto{\pgfqpoint{3.498965in}{3.131731in}}%
\pgfpathlineto{\pgfqpoint{3.500445in}{3.131731in}}%
\pgfpathlineto{\pgfqpoint{3.501924in}{3.348148in}}%
\pgfpathlineto{\pgfqpoint{3.502418in}{3.339452in}}%
\pgfpathlineto{\pgfqpoint{3.503897in}{3.165903in}}%
\pgfpathlineto{\pgfqpoint{3.505377in}{3.165903in}}%
\pgfpathlineto{\pgfqpoint{3.506363in}{3.366232in}}%
\pgfpathlineto{\pgfqpoint{3.506857in}{3.112805in}}%
\pgfpathlineto{\pgfqpoint{3.507843in}{3.169687in}}%
\pgfpathlineto{\pgfqpoint{3.509323in}{3.169687in}}%
\pgfpathlineto{\pgfqpoint{3.510309in}{3.371143in}}%
\pgfpathlineto{\pgfqpoint{3.510803in}{3.321913in}}%
\pgfpathlineto{\pgfqpoint{3.512282in}{3.365144in}}%
\pgfpathlineto{\pgfqpoint{3.513762in}{3.365144in}}%
\pgfpathlineto{\pgfqpoint{3.514255in}{3.024987in}}%
\pgfpathlineto{\pgfqpoint{3.514748in}{3.349245in}}%
\pgfpathlineto{\pgfqpoint{3.515242in}{3.196173in}}%
\pgfpathlineto{\pgfqpoint{3.515735in}{3.235561in}}%
\pgfpathlineto{\pgfqpoint{3.516721in}{3.235561in}}%
\pgfpathlineto{\pgfqpoint{3.518201in}{2.776995in}}%
\pgfpathlineto{\pgfqpoint{3.519681in}{3.297118in}}%
\pgfpathlineto{\pgfqpoint{3.520667in}{3.297955in}}%
\pgfpathlineto{\pgfqpoint{3.522147in}{3.127798in}}%
\pgfpathlineto{\pgfqpoint{3.523627in}{3.127798in}}%
\pgfpathlineto{\pgfqpoint{3.524613in}{2.932644in}}%
\pgfpathlineto{\pgfqpoint{3.526093in}{3.359498in}}%
\pgfpathlineto{\pgfqpoint{3.527079in}{3.367122in}}%
\pgfpathlineto{\pgfqpoint{3.528559in}{3.246368in}}%
\pgfpathlineto{\pgfqpoint{3.529052in}{3.246368in}}%
\pgfpathlineto{\pgfqpoint{3.530532in}{3.379811in}}%
\pgfpathlineto{\pgfqpoint{3.532011in}{3.378090in}}%
\pgfpathlineto{\pgfqpoint{3.532998in}{2.902843in}}%
\pgfpathlineto{\pgfqpoint{3.533984in}{3.135987in}}%
\pgfpathlineto{\pgfqpoint{3.535464in}{2.976400in}}%
\pgfpathlineto{\pgfqpoint{3.536944in}{3.232837in}}%
\pgfpathlineto{\pgfqpoint{3.538424in}{3.194907in}}%
\pgfpathlineto{\pgfqpoint{3.540396in}{3.377251in}}%
\pgfpathlineto{\pgfqpoint{3.540890in}{3.377251in}}%
\pgfpathlineto{\pgfqpoint{3.542369in}{3.146167in}}%
\pgfpathlineto{\pgfqpoint{3.542863in}{3.146167in}}%
\pgfpathlineto{\pgfqpoint{3.543849in}{2.771506in}}%
\pgfpathlineto{\pgfqpoint{3.545329in}{3.370504in}}%
\pgfpathlineto{\pgfqpoint{3.547795in}{3.370504in}}%
\pgfpathlineto{\pgfqpoint{3.548781in}{2.792050in}}%
\pgfpathlineto{\pgfqpoint{3.550261in}{3.379807in}}%
\pgfpathlineto{\pgfqpoint{3.552234in}{3.350676in}}%
\pgfpathlineto{\pgfqpoint{3.555193in}{3.350676in}}%
\pgfpathlineto{\pgfqpoint{3.556673in}{3.238394in}}%
\pgfpathlineto{\pgfqpoint{3.557166in}{3.380000in}}%
\pgfpathlineto{\pgfqpoint{3.558153in}{3.344548in}}%
\pgfpathlineto{\pgfqpoint{3.559139in}{3.344548in}}%
\pgfpathlineto{\pgfqpoint{3.559632in}{3.136805in}}%
\pgfpathlineto{\pgfqpoint{3.560126in}{3.239137in}}%
\pgfpathlineto{\pgfqpoint{3.562099in}{3.124146in}}%
\pgfpathlineto{\pgfqpoint{3.562592in}{2.800598in}}%
\pgfpathlineto{\pgfqpoint{3.564072in}{3.327495in}}%
\pgfpathlineto{\pgfqpoint{3.565058in}{3.235789in}}%
\pgfpathlineto{\pgfqpoint{3.566538in}{3.372425in}}%
\pgfpathlineto{\pgfqpoint{3.568017in}{3.372425in}}%
\pgfpathlineto{\pgfqpoint{3.568511in}{3.298924in}}%
\pgfpathlineto{\pgfqpoint{3.569990in}{2.765623in}}%
\pgfpathlineto{\pgfqpoint{3.570484in}{2.765623in}}%
\pgfpathlineto{\pgfqpoint{3.571963in}{3.376930in}}%
\pgfpathlineto{\pgfqpoint{3.572456in}{3.376930in}}%
\pgfpathlineto{\pgfqpoint{3.573936in}{3.188680in}}%
\pgfpathlineto{\pgfqpoint{3.575909in}{3.188680in}}%
\pgfpathlineto{\pgfqpoint{3.576402in}{3.018428in}}%
\pgfpathlineto{\pgfqpoint{3.576896in}{3.372920in}}%
\pgfpathlineto{\pgfqpoint{3.577389in}{3.273290in}}%
\pgfpathlineto{\pgfqpoint{3.578375in}{3.273290in}}%
\pgfpathlineto{\pgfqpoint{3.578868in}{3.095083in}}%
\pgfpathlineto{\pgfqpoint{3.579855in}{3.348493in}}%
\pgfpathlineto{\pgfqpoint{3.581335in}{3.282557in}}%
\pgfpathlineto{\pgfqpoint{3.581828in}{3.282557in}}%
\pgfpathlineto{\pgfqpoint{3.582321in}{3.034947in}}%
\pgfpathlineto{\pgfqpoint{3.582814in}{3.315071in}}%
\pgfpathlineto{\pgfqpoint{3.584787in}{3.315071in}}%
\pgfpathlineto{\pgfqpoint{3.586267in}{3.136213in}}%
\pgfpathlineto{\pgfqpoint{3.586760in}{3.136213in}}%
\pgfpathlineto{\pgfqpoint{3.587747in}{3.169454in}}%
\pgfpathlineto{\pgfqpoint{3.588240in}{2.467294in}}%
\pgfpathlineto{\pgfqpoint{3.588733in}{3.263850in}}%
\pgfpathlineto{\pgfqpoint{3.590706in}{2.707650in}}%
\pgfpathlineto{\pgfqpoint{3.592186in}{3.165328in}}%
\pgfpathlineto{\pgfqpoint{3.594652in}{3.165328in}}%
\pgfpathlineto{\pgfqpoint{3.597118in}{3.301065in}}%
\pgfpathlineto{\pgfqpoint{3.598598in}{3.301065in}}%
\pgfpathlineto{\pgfqpoint{3.600077in}{3.302646in}}%
\pgfpathlineto{\pgfqpoint{3.600571in}{3.302646in}}%
\pgfpathlineto{\pgfqpoint{3.602050in}{3.269969in}}%
\pgfpathlineto{\pgfqpoint{3.602544in}{3.269969in}}%
\pgfpathlineto{\pgfqpoint{3.604023in}{3.337818in}}%
\pgfpathlineto{\pgfqpoint{3.605503in}{3.337818in}}%
\pgfpathlineto{\pgfqpoint{3.605996in}{2.826112in}}%
\pgfpathlineto{\pgfqpoint{3.606489in}{3.373882in}}%
\pgfpathlineto{\pgfqpoint{3.609942in}{3.373882in}}%
\pgfpathlineto{\pgfqpoint{3.610435in}{2.826453in}}%
\pgfpathlineto{\pgfqpoint{3.611422in}{2.904133in}}%
\pgfpathlineto{\pgfqpoint{3.612408in}{2.904133in}}%
\pgfpathlineto{\pgfqpoint{3.613888in}{3.300852in}}%
\pgfpathlineto{\pgfqpoint{3.614874in}{3.300852in}}%
\pgfpathlineto{\pgfqpoint{3.615368in}{3.371476in}}%
\pgfpathlineto{\pgfqpoint{3.615861in}{3.318052in}}%
\pgfpathlineto{\pgfqpoint{3.616354in}{3.189219in}}%
\pgfpathlineto{\pgfqpoint{3.617340in}{3.213786in}}%
\pgfpathlineto{\pgfqpoint{3.620793in}{3.213786in}}%
\pgfpathlineto{\pgfqpoint{3.621286in}{2.747705in}}%
\pgfpathlineto{\pgfqpoint{3.621780in}{3.268498in}}%
\pgfpathlineto{\pgfqpoint{3.623752in}{3.129394in}}%
\pgfpathlineto{\pgfqpoint{3.624739in}{3.129394in}}%
\pgfpathlineto{\pgfqpoint{3.626219in}{3.354498in}}%
\pgfpathlineto{\pgfqpoint{3.626712in}{3.354498in}}%
\pgfpathlineto{\pgfqpoint{3.627205in}{3.296769in}}%
\pgfpathlineto{\pgfqpoint{3.628685in}{3.378000in}}%
\pgfpathlineto{\pgfqpoint{3.629178in}{2.846687in}}%
\pgfpathlineto{\pgfqpoint{3.629671in}{3.339994in}}%
\pgfpathlineto{\pgfqpoint{3.632137in}{3.339994in}}%
\pgfpathlineto{\pgfqpoint{3.634604in}{3.364161in}}%
\pgfpathlineto{\pgfqpoint{3.636576in}{3.364161in}}%
\pgfpathlineto{\pgfqpoint{3.637563in}{3.284591in}}%
\pgfpathlineto{\pgfqpoint{3.639043in}{3.372246in}}%
\pgfpathlineto{\pgfqpoint{3.640522in}{3.337135in}}%
\pgfpathlineto{\pgfqpoint{3.642002in}{3.337135in}}%
\pgfpathlineto{\pgfqpoint{3.642495in}{3.260244in}}%
\pgfpathlineto{\pgfqpoint{3.642989in}{3.297081in}}%
\pgfpathlineto{\pgfqpoint{3.643482in}{3.303239in}}%
\pgfpathlineto{\pgfqpoint{3.643975in}{2.858845in}}%
\pgfpathlineto{\pgfqpoint{3.644468in}{3.114656in}}%
\pgfpathlineto{\pgfqpoint{3.645455in}{3.114656in}}%
\pgfpathlineto{\pgfqpoint{3.646934in}{3.376247in}}%
\pgfpathlineto{\pgfqpoint{3.647428in}{3.376247in}}%
\pgfpathlineto{\pgfqpoint{3.648907in}{2.531685in}}%
\pgfpathlineto{\pgfqpoint{3.649401in}{2.531685in}}%
\pgfpathlineto{\pgfqpoint{3.650880in}{3.378142in}}%
\pgfpathlineto{\pgfqpoint{3.653840in}{3.378142in}}%
\pgfpathlineto{\pgfqpoint{3.655319in}{3.373794in}}%
\pgfpathlineto{\pgfqpoint{3.655813in}{3.373794in}}%
\pgfpathlineto{\pgfqpoint{3.656799in}{2.819088in}}%
\pgfpathlineto{\pgfqpoint{3.657292in}{3.379000in}}%
\pgfpathlineto{\pgfqpoint{3.658279in}{3.362050in}}%
\pgfpathlineto{\pgfqpoint{3.658772in}{3.358147in}}%
\pgfpathlineto{\pgfqpoint{3.661731in}{3.255844in}}%
\pgfpathlineto{\pgfqpoint{3.662225in}{3.307786in}}%
\pgfpathlineto{\pgfqpoint{3.664197in}{3.077120in}}%
\pgfpathlineto{\pgfqpoint{3.665677in}{3.339905in}}%
\pgfpathlineto{\pgfqpoint{3.667157in}{3.379787in}}%
\pgfpathlineto{\pgfqpoint{3.667650in}{3.365326in}}%
\pgfpathlineto{\pgfqpoint{3.671103in}{1.675373in}}%
\pgfpathlineto{\pgfqpoint{3.671596in}{3.370979in}}%
\pgfpathlineto{\pgfqpoint{3.672582in}{3.266932in}}%
\pgfpathlineto{\pgfqpoint{3.673569in}{3.266932in}}%
\pgfpathlineto{\pgfqpoint{3.675049in}{3.378580in}}%
\pgfpathlineto{\pgfqpoint{3.678008in}{3.378580in}}%
\pgfpathlineto{\pgfqpoint{3.679488in}{2.976113in}}%
\pgfpathlineto{\pgfqpoint{3.679981in}{2.976113in}}%
\pgfpathlineto{\pgfqpoint{3.680474in}{2.741324in}}%
\pgfpathlineto{\pgfqpoint{3.681461in}{2.775668in}}%
\pgfpathlineto{\pgfqpoint{3.682940in}{2.912827in}}%
\pgfpathlineto{\pgfqpoint{3.683927in}{2.912827in}}%
\pgfpathlineto{\pgfqpoint{3.684913in}{3.367335in}}%
\pgfpathlineto{\pgfqpoint{3.686393in}{3.112300in}}%
\pgfpathlineto{\pgfqpoint{3.687379in}{3.112300in}}%
\pgfpathlineto{\pgfqpoint{3.687873in}{3.362615in}}%
\pgfpathlineto{\pgfqpoint{3.688366in}{3.168974in}}%
\pgfpathlineto{\pgfqpoint{3.689352in}{3.168974in}}%
\pgfpathlineto{\pgfqpoint{3.690339in}{3.377615in}}%
\pgfpathlineto{\pgfqpoint{3.690832in}{3.331270in}}%
\pgfpathlineto{\pgfqpoint{3.692312in}{2.994719in}}%
\pgfpathlineto{\pgfqpoint{3.693791in}{2.994719in}}%
\pgfpathlineto{\pgfqpoint{3.695271in}{2.483899in}}%
\pgfpathlineto{\pgfqpoint{3.696751in}{3.284639in}}%
\pgfpathlineto{\pgfqpoint{3.698724in}{3.284639in}}%
\pgfpathlineto{\pgfqpoint{3.699217in}{2.817651in}}%
\pgfpathlineto{\pgfqpoint{3.699710in}{3.377794in}}%
\pgfpathlineto{\pgfqpoint{3.701190in}{3.377794in}}%
\pgfpathlineto{\pgfqpoint{3.702669in}{3.064703in}}%
\pgfpathlineto{\pgfqpoint{3.703656in}{3.064703in}}%
\pgfpathlineto{\pgfqpoint{3.705136in}{3.314851in}}%
\pgfpathlineto{\pgfqpoint{3.706615in}{2.974670in}}%
\pgfpathlineto{\pgfqpoint{3.708095in}{3.334363in}}%
\pgfpathlineto{\pgfqpoint{3.710561in}{3.244538in}}%
\pgfpathlineto{\pgfqpoint{3.711548in}{3.244538in}}%
\pgfpathlineto{\pgfqpoint{3.714014in}{2.613451in}}%
\pgfpathlineto{\pgfqpoint{3.715493in}{3.353312in}}%
\pgfpathlineto{\pgfqpoint{3.715987in}{3.353312in}}%
\pgfpathlineto{\pgfqpoint{3.717466in}{3.230028in}}%
\pgfpathlineto{\pgfqpoint{3.718946in}{3.371062in}}%
\pgfpathlineto{\pgfqpoint{3.719439in}{3.371657in}}%
\pgfpathlineto{\pgfqpoint{3.719933in}{3.361429in}}%
\pgfpathlineto{\pgfqpoint{3.720919in}{3.378593in}}%
\pgfpathlineto{\pgfqpoint{3.721412in}{3.359177in}}%
\pgfpathlineto{\pgfqpoint{3.722399in}{3.157958in}}%
\pgfpathlineto{\pgfqpoint{3.723878in}{3.325293in}}%
\pgfpathlineto{\pgfqpoint{3.724865in}{3.325293in}}%
\pgfpathlineto{\pgfqpoint{3.726345in}{3.379838in}}%
\pgfpathlineto{\pgfqpoint{3.727331in}{3.379838in}}%
\pgfpathlineto{\pgfqpoint{3.728811in}{3.012519in}}%
\pgfpathlineto{\pgfqpoint{3.730290in}{3.336917in}}%
\pgfpathlineto{\pgfqpoint{3.730784in}{3.336917in}}%
\pgfpathlineto{\pgfqpoint{3.732263in}{3.376853in}}%
\pgfpathlineto{\pgfqpoint{3.732757in}{3.376853in}}%
\pgfpathlineto{\pgfqpoint{3.733250in}{3.368938in}}%
\pgfpathlineto{\pgfqpoint{3.733743in}{2.474947in}}%
\pgfpathlineto{\pgfqpoint{3.734236in}{3.277331in}}%
\pgfpathlineto{\pgfqpoint{3.735223in}{3.277331in}}%
\pgfpathlineto{\pgfqpoint{3.735716in}{2.277064in}}%
\pgfpathlineto{\pgfqpoint{3.736209in}{2.562001in}}%
\pgfpathlineto{\pgfqpoint{3.736702in}{2.562001in}}%
\pgfpathlineto{\pgfqpoint{3.738182in}{3.357462in}}%
\pgfpathlineto{\pgfqpoint{3.741142in}{3.357462in}}%
\pgfpathlineto{\pgfqpoint{3.743114in}{2.879512in}}%
\pgfpathlineto{\pgfqpoint{3.744101in}{3.235016in}}%
\pgfpathlineto{\pgfqpoint{3.744594in}{3.197512in}}%
\pgfpathlineto{\pgfqpoint{3.745087in}{3.197512in}}%
\pgfpathlineto{\pgfqpoint{3.746567in}{3.366643in}}%
\pgfpathlineto{\pgfqpoint{3.748047in}{3.366643in}}%
\pgfpathlineto{\pgfqpoint{3.749526in}{2.971484in}}%
\pgfpathlineto{\pgfqpoint{3.751006in}{3.363382in}}%
\pgfpathlineto{\pgfqpoint{3.751499in}{3.363382in}}%
\pgfpathlineto{\pgfqpoint{3.753966in}{3.168480in}}%
\pgfpathlineto{\pgfqpoint{3.754952in}{3.168480in}}%
\pgfpathlineto{\pgfqpoint{3.756432in}{3.222650in}}%
\pgfpathlineto{\pgfqpoint{3.756925in}{3.136535in}}%
\pgfpathlineto{\pgfqpoint{3.758405in}{3.341349in}}%
\pgfpathlineto{\pgfqpoint{3.758898in}{3.341349in}}%
\pgfpathlineto{\pgfqpoint{3.759391in}{2.869855in}}%
\pgfpathlineto{\pgfqpoint{3.759884in}{3.233259in}}%
\pgfpathlineto{\pgfqpoint{3.761857in}{3.233259in}}%
\pgfpathlineto{\pgfqpoint{3.762350in}{2.894343in}}%
\pgfpathlineto{\pgfqpoint{3.762844in}{3.288609in}}%
\pgfpathlineto{\pgfqpoint{3.763337in}{3.288609in}}%
\pgfpathlineto{\pgfqpoint{3.764817in}{3.373762in}}%
\pgfpathlineto{\pgfqpoint{3.769749in}{3.373762in}}%
\pgfpathlineto{\pgfqpoint{3.770735in}{3.375100in}}%
\pgfpathlineto{\pgfqpoint{3.772215in}{3.279142in}}%
\pgfpathlineto{\pgfqpoint{3.773695in}{3.330733in}}%
\pgfpathlineto{\pgfqpoint{3.777641in}{3.330733in}}%
\pgfpathlineto{\pgfqpoint{3.778134in}{3.369691in}}%
\pgfpathlineto{\pgfqpoint{3.779120in}{3.363301in}}%
\pgfpathlineto{\pgfqpoint{3.780600in}{3.363301in}}%
\pgfpathlineto{\pgfqpoint{3.781093in}{2.437568in}}%
\pgfpathlineto{\pgfqpoint{3.781586in}{2.987879in}}%
\pgfpathlineto{\pgfqpoint{3.782080in}{2.987879in}}%
\pgfpathlineto{\pgfqpoint{3.783559in}{3.287039in}}%
\pgfpathlineto{\pgfqpoint{3.784546in}{3.287039in}}%
\pgfpathlineto{\pgfqpoint{3.786026in}{2.790710in}}%
\pgfpathlineto{\pgfqpoint{3.787012in}{3.379968in}}%
\pgfpathlineto{\pgfqpoint{3.787505in}{2.552330in}}%
\pgfpathlineto{\pgfqpoint{3.787998in}{2.766672in}}%
\pgfpathlineto{\pgfqpoint{3.788985in}{2.766672in}}%
\pgfpathlineto{\pgfqpoint{3.789478in}{2.747716in}}%
\pgfpathlineto{\pgfqpoint{3.790465in}{3.272871in}}%
\pgfpathlineto{\pgfqpoint{3.790958in}{3.246418in}}%
\pgfpathlineto{\pgfqpoint{3.791451in}{2.264402in}}%
\pgfpathlineto{\pgfqpoint{3.791944in}{3.016537in}}%
\pgfpathlineto{\pgfqpoint{3.792438in}{3.062656in}}%
\pgfpathlineto{\pgfqpoint{3.792931in}{2.833661in}}%
\pgfpathlineto{\pgfqpoint{3.793917in}{3.369558in}}%
\pgfpathlineto{\pgfqpoint{3.794410in}{3.292573in}}%
\pgfpathlineto{\pgfqpoint{3.795397in}{2.881983in}}%
\pgfpathlineto{\pgfqpoint{3.796877in}{3.366960in}}%
\pgfpathlineto{\pgfqpoint{3.798850in}{3.366960in}}%
\pgfpathlineto{\pgfqpoint{3.799343in}{3.362059in}}%
\pgfpathlineto{\pgfqpoint{3.800329in}{3.342793in}}%
\pgfpathlineto{\pgfqpoint{3.801809in}{3.253339in}}%
\pgfpathlineto{\pgfqpoint{3.803782in}{3.253339in}}%
\pgfpathlineto{\pgfqpoint{3.804275in}{3.336371in}}%
\pgfpathlineto{\pgfqpoint{3.804768in}{3.309120in}}%
\pgfpathlineto{\pgfqpoint{3.805262in}{3.309120in}}%
\pgfpathlineto{\pgfqpoint{3.805755in}{3.300317in}}%
\pgfpathlineto{\pgfqpoint{3.807234in}{3.377125in}}%
\pgfpathlineto{\pgfqpoint{3.808714in}{3.326789in}}%
\pgfpathlineto{\pgfqpoint{3.810194in}{3.326789in}}%
\pgfpathlineto{\pgfqpoint{3.811674in}{3.310697in}}%
\pgfpathlineto{\pgfqpoint{3.812167in}{2.990788in}}%
\pgfpathlineto{\pgfqpoint{3.812660in}{3.378490in}}%
\pgfpathlineto{\pgfqpoint{3.813153in}{3.379775in}}%
\pgfpathlineto{\pgfqpoint{3.814633in}{3.261394in}}%
\pgfpathlineto{\pgfqpoint{3.816606in}{3.376004in}}%
\pgfpathlineto{\pgfqpoint{3.817592in}{3.379599in}}%
\pgfpathlineto{\pgfqpoint{3.819072in}{3.120268in}}%
\pgfpathlineto{\pgfqpoint{3.819565in}{3.120268in}}%
\pgfpathlineto{\pgfqpoint{3.821045in}{3.370633in}}%
\pgfpathlineto{\pgfqpoint{3.821538in}{3.121281in}}%
\pgfpathlineto{\pgfqpoint{3.822031in}{3.352888in}}%
\pgfpathlineto{\pgfqpoint{3.822525in}{3.352888in}}%
\pgfpathlineto{\pgfqpoint{3.823511in}{3.354838in}}%
\pgfpathlineto{\pgfqpoint{3.824991in}{3.374243in}}%
\pgfpathlineto{\pgfqpoint{3.827457in}{3.374243in}}%
\pgfpathlineto{\pgfqpoint{3.829923in}{3.251120in}}%
\pgfpathlineto{\pgfqpoint{3.831896in}{3.251120in}}%
\pgfpathlineto{\pgfqpoint{3.832389in}{3.144862in}}%
\pgfpathlineto{\pgfqpoint{3.832882in}{3.189909in}}%
\pgfpathlineto{\pgfqpoint{3.834362in}{3.189909in}}%
\pgfpathlineto{\pgfqpoint{3.835349in}{3.319149in}}%
\pgfpathlineto{\pgfqpoint{3.836335in}{3.055552in}}%
\pgfpathlineto{\pgfqpoint{3.837815in}{3.368371in}}%
\pgfpathlineto{\pgfqpoint{3.838308in}{3.368371in}}%
\pgfpathlineto{\pgfqpoint{3.838801in}{3.144013in}}%
\pgfpathlineto{\pgfqpoint{3.839294in}{3.352612in}}%
\pgfpathlineto{\pgfqpoint{3.840281in}{3.352612in}}%
\pgfpathlineto{\pgfqpoint{3.841761in}{3.129428in}}%
\pgfpathlineto{\pgfqpoint{3.843240in}{3.375149in}}%
\pgfpathlineto{\pgfqpoint{3.845213in}{3.196312in}}%
\pgfpathlineto{\pgfqpoint{3.845707in}{3.379033in}}%
\pgfpathlineto{\pgfqpoint{3.846200in}{3.268421in}}%
\pgfpathlineto{\pgfqpoint{3.846693in}{3.268421in}}%
\pgfpathlineto{\pgfqpoint{3.848173in}{3.373845in}}%
\pgfpathlineto{\pgfqpoint{3.849159in}{3.373845in}}%
\pgfpathlineto{\pgfqpoint{3.850639in}{3.379914in}}%
\pgfpathlineto{\pgfqpoint{3.851132in}{3.379914in}}%
\pgfpathlineto{\pgfqpoint{3.852612in}{3.241223in}}%
\pgfpathlineto{\pgfqpoint{3.853105in}{3.241223in}}%
\pgfpathlineto{\pgfqpoint{3.854091in}{2.921162in}}%
\pgfpathlineto{\pgfqpoint{3.855571in}{3.367324in}}%
\pgfpathlineto{\pgfqpoint{3.856558in}{3.367324in}}%
\pgfpathlineto{\pgfqpoint{3.857544in}{3.297700in}}%
\pgfpathlineto{\pgfqpoint{3.858037in}{3.330566in}}%
\pgfpathlineto{\pgfqpoint{3.859517in}{3.256492in}}%
\pgfpathlineto{\pgfqpoint{3.861983in}{3.256492in}}%
\pgfpathlineto{\pgfqpoint{3.862476in}{3.214395in}}%
\pgfpathlineto{\pgfqpoint{3.863956in}{3.368555in}}%
\pgfpathlineto{\pgfqpoint{3.866422in}{3.368555in}}%
\pgfpathlineto{\pgfqpoint{3.867902in}{3.153376in}}%
\pgfpathlineto{\pgfqpoint{3.868395in}{3.153376in}}%
\pgfpathlineto{\pgfqpoint{3.869382in}{3.334689in}}%
\pgfpathlineto{\pgfqpoint{3.869875in}{3.299439in}}%
\pgfpathlineto{\pgfqpoint{3.870368in}{3.299439in}}%
\pgfpathlineto{\pgfqpoint{3.871355in}{3.302060in}}%
\pgfpathlineto{\pgfqpoint{3.872341in}{3.047917in}}%
\pgfpathlineto{\pgfqpoint{3.873821in}{3.378366in}}%
\pgfpathlineto{\pgfqpoint{3.874314in}{3.371809in}}%
\pgfpathlineto{\pgfqpoint{3.874807in}{3.378412in}}%
\pgfpathlineto{\pgfqpoint{3.876287in}{3.378412in}}%
\pgfpathlineto{\pgfqpoint{3.877767in}{2.817390in}}%
\pgfpathlineto{\pgfqpoint{3.878753in}{3.353703in}}%
\pgfpathlineto{\pgfqpoint{3.879246in}{3.078379in}}%
\pgfpathlineto{\pgfqpoint{3.879739in}{3.192033in}}%
\pgfpathlineto{\pgfqpoint{3.880233in}{3.192033in}}%
\pgfpathlineto{\pgfqpoint{3.881712in}{2.764136in}}%
\pgfpathlineto{\pgfqpoint{3.883685in}{3.366871in}}%
\pgfpathlineto{\pgfqpoint{3.885165in}{3.366871in}}%
\pgfpathlineto{\pgfqpoint{3.886645in}{3.377615in}}%
\pgfpathlineto{\pgfqpoint{3.890097in}{3.377615in}}%
\pgfpathlineto{\pgfqpoint{3.891577in}{3.195741in}}%
\pgfpathlineto{\pgfqpoint{3.895030in}{3.195741in}}%
\pgfpathlineto{\pgfqpoint{3.896509in}{3.103210in}}%
\pgfpathlineto{\pgfqpoint{3.897989in}{3.046638in}}%
\pgfpathlineto{\pgfqpoint{3.898975in}{3.046638in}}%
\pgfpathlineto{\pgfqpoint{3.899469in}{3.364902in}}%
\pgfpathlineto{\pgfqpoint{3.899962in}{3.208400in}}%
\pgfpathlineto{\pgfqpoint{3.900455in}{3.208400in}}%
\pgfpathlineto{\pgfqpoint{3.900948in}{2.180516in}}%
\pgfpathlineto{\pgfqpoint{3.901442in}{3.053643in}}%
\pgfpathlineto{\pgfqpoint{3.902921in}{2.446737in}}%
\pgfpathlineto{\pgfqpoint{3.903415in}{2.507114in}}%
\pgfpathlineto{\pgfqpoint{3.904894in}{3.209123in}}%
\pgfpathlineto{\pgfqpoint{3.905881in}{3.209123in}}%
\pgfpathlineto{\pgfqpoint{3.907360in}{2.960766in}}%
\pgfpathlineto{\pgfqpoint{3.908840in}{3.378008in}}%
\pgfpathlineto{\pgfqpoint{3.910320in}{3.378176in}}%
\pgfpathlineto{\pgfqpoint{3.911306in}{3.373957in}}%
\pgfpathlineto{\pgfqpoint{3.912293in}{2.328983in}}%
\pgfpathlineto{\pgfqpoint{3.913772in}{3.070120in}}%
\pgfpathlineto{\pgfqpoint{3.914266in}{3.070120in}}%
\pgfpathlineto{\pgfqpoint{3.914759in}{2.927572in}}%
\pgfpathlineto{\pgfqpoint{3.915252in}{3.367983in}}%
\pgfpathlineto{\pgfqpoint{3.915745in}{2.707967in}}%
\pgfpathlineto{\pgfqpoint{3.916239in}{3.298616in}}%
\pgfpathlineto{\pgfqpoint{3.917718in}{3.377425in}}%
\pgfpathlineto{\pgfqpoint{3.918211in}{3.377425in}}%
\pgfpathlineto{\pgfqpoint{3.919691in}{3.324126in}}%
\pgfpathlineto{\pgfqpoint{3.921171in}{3.324126in}}%
\pgfpathlineto{\pgfqpoint{3.922651in}{1.516741in}}%
\pgfpathlineto{\pgfqpoint{3.924130in}{3.330038in}}%
\pgfpathlineto{\pgfqpoint{3.924623in}{3.330038in}}%
\pgfpathlineto{\pgfqpoint{3.925610in}{3.365940in}}%
\pgfpathlineto{\pgfqpoint{3.927090in}{3.286456in}}%
\pgfpathlineto{\pgfqpoint{3.928076in}{3.286456in}}%
\pgfpathlineto{\pgfqpoint{3.928569in}{2.744835in}}%
\pgfpathlineto{\pgfqpoint{3.929063in}{3.314333in}}%
\pgfpathlineto{\pgfqpoint{3.930049in}{3.314333in}}%
\pgfpathlineto{\pgfqpoint{3.930542in}{3.358218in}}%
\pgfpathlineto{\pgfqpoint{3.932515in}{3.178742in}}%
\pgfpathlineto{\pgfqpoint{3.934488in}{3.178742in}}%
\pgfpathlineto{\pgfqpoint{3.935475in}{2.963663in}}%
\pgfpathlineto{\pgfqpoint{3.936954in}{3.379762in}}%
\pgfpathlineto{\pgfqpoint{3.938434in}{3.351510in}}%
\pgfpathlineto{\pgfqpoint{3.940407in}{3.351510in}}%
\pgfpathlineto{\pgfqpoint{3.941887in}{2.997935in}}%
\pgfpathlineto{\pgfqpoint{3.943366in}{2.997935in}}%
\pgfpathlineto{\pgfqpoint{3.943859in}{3.071880in}}%
\pgfpathlineto{\pgfqpoint{3.945339in}{3.355791in}}%
\pgfpathlineto{\pgfqpoint{3.945832in}{3.355791in}}%
\pgfpathlineto{\pgfqpoint{3.946819in}{2.801183in}}%
\pgfpathlineto{\pgfqpoint{3.947312in}{2.819264in}}%
\pgfpathlineto{\pgfqpoint{3.948299in}{2.819264in}}%
\pgfpathlineto{\pgfqpoint{3.949778in}{3.299107in}}%
\pgfpathlineto{\pgfqpoint{3.950765in}{3.299107in}}%
\pgfpathlineto{\pgfqpoint{3.951751in}{3.342360in}}%
\pgfpathlineto{\pgfqpoint{3.952244in}{3.332750in}}%
\pgfpathlineto{\pgfqpoint{3.955697in}{3.332750in}}%
\pgfpathlineto{\pgfqpoint{3.957177in}{3.218100in}}%
\pgfpathlineto{\pgfqpoint{3.957670in}{3.378569in}}%
\pgfpathlineto{\pgfqpoint{3.959150in}{3.106828in}}%
\pgfpathlineto{\pgfqpoint{3.959643in}{3.106828in}}%
\pgfpathlineto{\pgfqpoint{3.960136in}{2.941967in}}%
\pgfpathlineto{\pgfqpoint{3.961616in}{3.340647in}}%
\pgfpathlineto{\pgfqpoint{3.963096in}{3.327002in}}%
\pgfpathlineto{\pgfqpoint{3.963589in}{3.327002in}}%
\pgfpathlineto{\pgfqpoint{3.964082in}{3.374832in}}%
\pgfpathlineto{\pgfqpoint{3.964575in}{3.330077in}}%
\pgfpathlineto{\pgfqpoint{3.965068in}{3.212223in}}%
\pgfpathlineto{\pgfqpoint{3.965562in}{3.270499in}}%
\pgfpathlineto{\pgfqpoint{3.969508in}{3.270499in}}%
\pgfpathlineto{\pgfqpoint{3.970494in}{3.377967in}}%
\pgfpathlineto{\pgfqpoint{3.970987in}{3.352127in}}%
\pgfpathlineto{\pgfqpoint{3.971480in}{3.352127in}}%
\pgfpathlineto{\pgfqpoint{3.972960in}{2.382944in}}%
\pgfpathlineto{\pgfqpoint{3.974440in}{2.678417in}}%
\pgfpathlineto{\pgfqpoint{3.975920in}{2.547070in}}%
\pgfpathlineto{\pgfqpoint{3.977399in}{3.332185in}}%
\pgfpathlineto{\pgfqpoint{3.978386in}{3.332185in}}%
\pgfpathlineto{\pgfqpoint{3.980852in}{2.146531in}}%
\pgfpathlineto{\pgfqpoint{3.982332in}{3.341062in}}%
\pgfpathlineto{\pgfqpoint{3.982825in}{3.341062in}}%
\pgfpathlineto{\pgfqpoint{3.983318in}{3.372320in}}%
\pgfpathlineto{\pgfqpoint{3.983811in}{3.253823in}}%
\pgfpathlineto{\pgfqpoint{3.984304in}{3.371482in}}%
\pgfpathlineto{\pgfqpoint{3.988744in}{3.371482in}}%
\pgfpathlineto{\pgfqpoint{3.990223in}{3.374457in}}%
\pgfpathlineto{\pgfqpoint{3.991210in}{3.374457in}}%
\pgfpathlineto{\pgfqpoint{3.993676in}{3.294634in}}%
\pgfpathlineto{\pgfqpoint{3.994169in}{3.294634in}}%
\pgfpathlineto{\pgfqpoint{3.995649in}{3.338015in}}%
\pgfpathlineto{\pgfqpoint{3.996142in}{3.338015in}}%
\pgfpathlineto{\pgfqpoint{3.998115in}{3.264817in}}%
\pgfpathlineto{\pgfqpoint{3.999595in}{3.264817in}}%
\pgfpathlineto{\pgfqpoint{4.000088in}{3.347949in}}%
\pgfpathlineto{\pgfqpoint{4.001074in}{2.883811in}}%
\pgfpathlineto{\pgfqpoint{4.002554in}{3.241176in}}%
\pgfpathlineto{\pgfqpoint{4.003047in}{3.378786in}}%
\pgfpathlineto{\pgfqpoint{4.003540in}{3.104773in}}%
\pgfpathlineto{\pgfqpoint{4.004034in}{3.201515in}}%
\pgfpathlineto{\pgfqpoint{4.004527in}{3.276911in}}%
\pgfpathlineto{\pgfqpoint{4.006007in}{3.133296in}}%
\pgfpathlineto{\pgfqpoint{4.006500in}{3.133296in}}%
\pgfpathlineto{\pgfqpoint{4.007980in}{3.371533in}}%
\pgfpathlineto{\pgfqpoint{4.008473in}{3.241839in}}%
\pgfpathlineto{\pgfqpoint{4.008966in}{3.332432in}}%
\pgfpathlineto{\pgfqpoint{4.011432in}{3.332432in}}%
\pgfpathlineto{\pgfqpoint{4.012912in}{3.370313in}}%
\pgfpathlineto{\pgfqpoint{4.014392in}{3.192613in}}%
\pgfpathlineto{\pgfqpoint{4.014885in}{3.373781in}}%
\pgfpathlineto{\pgfqpoint{4.015378in}{2.970730in}}%
\pgfpathlineto{\pgfqpoint{4.015871in}{3.379850in}}%
\pgfpathlineto{\pgfqpoint{4.020804in}{3.379191in}}%
\pgfpathlineto{\pgfqpoint{4.022283in}{2.551191in}}%
\pgfpathlineto{\pgfqpoint{4.023270in}{3.083890in}}%
\pgfpathlineto{\pgfqpoint{4.023763in}{2.111329in}}%
\pgfpathlineto{\pgfqpoint{4.024256in}{2.849701in}}%
\pgfpathlineto{\pgfqpoint{4.024749in}{2.849701in}}%
\pgfpathlineto{\pgfqpoint{4.025243in}{2.858127in}}%
\pgfpathlineto{\pgfqpoint{4.026722in}{3.193886in}}%
\pgfpathlineto{\pgfqpoint{4.027216in}{3.315155in}}%
\pgfpathlineto{\pgfqpoint{4.027709in}{3.192921in}}%
\pgfpathlineto{\pgfqpoint{4.028695in}{2.668068in}}%
\pgfpathlineto{\pgfqpoint{4.030668in}{3.352132in}}%
\pgfpathlineto{\pgfqpoint{4.031655in}{3.352132in}}%
\pgfpathlineto{\pgfqpoint{4.032148in}{3.369064in}}%
\pgfpathlineto{\pgfqpoint{4.033628in}{3.276472in}}%
\pgfpathlineto{\pgfqpoint{4.035107in}{3.276472in}}%
\pgfpathlineto{\pgfqpoint{4.036094in}{3.216173in}}%
\pgfpathlineto{\pgfqpoint{4.037573in}{2.918438in}}%
\pgfpathlineto{\pgfqpoint{4.038560in}{2.918438in}}%
\pgfpathlineto{\pgfqpoint{4.039053in}{3.352088in}}%
\pgfpathlineto{\pgfqpoint{4.039546in}{3.215963in}}%
\pgfpathlineto{\pgfqpoint{4.040533in}{3.215963in}}%
\pgfpathlineto{\pgfqpoint{4.042012in}{3.186531in}}%
\pgfpathlineto{\pgfqpoint{4.042506in}{3.186531in}}%
\pgfpathlineto{\pgfqpoint{4.043985in}{3.249473in}}%
\pgfpathlineto{\pgfqpoint{4.045465in}{3.227847in}}%
\pgfpathlineto{\pgfqpoint{4.047438in}{3.227847in}}%
\pgfpathlineto{\pgfqpoint{4.048918in}{3.260135in}}%
\pgfpathlineto{\pgfqpoint{4.050397in}{3.260135in}}%
\pgfpathlineto{\pgfqpoint{4.051877in}{3.127224in}}%
\pgfpathlineto{\pgfqpoint{4.052370in}{2.466824in}}%
\pgfpathlineto{\pgfqpoint{4.053357in}{2.574460in}}%
\pgfpathlineto{\pgfqpoint{4.053850in}{3.136129in}}%
\pgfpathlineto{\pgfqpoint{4.054343in}{2.708540in}}%
\pgfpathlineto{\pgfqpoint{4.055330in}{2.708540in}}%
\pgfpathlineto{\pgfqpoint{4.055823in}{2.809239in}}%
\pgfpathlineto{\pgfqpoint{4.056809in}{2.499299in}}%
\pgfpathlineto{\pgfqpoint{4.057796in}{3.034591in}}%
\pgfpathlineto{\pgfqpoint{4.058289in}{3.006963in}}%
\pgfpathlineto{\pgfqpoint{4.059276in}{3.006963in}}%
\pgfpathlineto{\pgfqpoint{4.059769in}{3.377665in}}%
\pgfpathlineto{\pgfqpoint{4.060262in}{3.234566in}}%
\pgfpathlineto{\pgfqpoint{4.062235in}{3.234566in}}%
\pgfpathlineto{\pgfqpoint{4.063221in}{3.368661in}}%
\pgfpathlineto{\pgfqpoint{4.064701in}{3.313854in}}%
\pgfpathlineto{\pgfqpoint{4.065194in}{3.313854in}}%
\pgfpathlineto{\pgfqpoint{4.066674in}{3.379322in}}%
\pgfpathlineto{\pgfqpoint{4.067167in}{3.379322in}}%
\pgfpathlineto{\pgfqpoint{4.067661in}{3.336788in}}%
\pgfpathlineto{\pgfqpoint{4.068154in}{3.171000in}}%
\pgfpathlineto{\pgfqpoint{4.068647in}{3.319865in}}%
\pgfpathlineto{\pgfqpoint{4.069633in}{3.319865in}}%
\pgfpathlineto{\pgfqpoint{4.071113in}{3.371167in}}%
\pgfpathlineto{\pgfqpoint{4.073579in}{3.371167in}}%
\pgfpathlineto{\pgfqpoint{4.074566in}{3.373866in}}%
\pgfpathlineto{\pgfqpoint{4.076045in}{3.207866in}}%
\pgfpathlineto{\pgfqpoint{4.077525in}{3.207866in}}%
\pgfpathlineto{\pgfqpoint{4.079005in}{3.371127in}}%
\pgfpathlineto{\pgfqpoint{4.079991in}{3.371127in}}%
\pgfpathlineto{\pgfqpoint{4.081471in}{3.354014in}}%
\pgfpathlineto{\pgfqpoint{4.082951in}{3.375012in}}%
\pgfpathlineto{\pgfqpoint{4.083937in}{3.375012in}}%
\pgfpathlineto{\pgfqpoint{4.085417in}{3.286397in}}%
\pgfpathlineto{\pgfqpoint{4.085910in}{2.791690in}}%
\pgfpathlineto{\pgfqpoint{4.086403in}{3.375990in}}%
\pgfpathlineto{\pgfqpoint{4.088376in}{3.376936in}}%
\pgfpathlineto{\pgfqpoint{4.088869in}{3.379211in}}%
\pgfpathlineto{\pgfqpoint{4.090842in}{1.910991in}}%
\pgfpathlineto{\pgfqpoint{4.092322in}{3.061728in}}%
\pgfpathlineto{\pgfqpoint{4.092815in}{3.061728in}}%
\pgfpathlineto{\pgfqpoint{4.093309in}{3.194512in}}%
\pgfpathlineto{\pgfqpoint{4.094788in}{1.632171in}}%
\pgfpathlineto{\pgfqpoint{4.096268in}{3.085235in}}%
\pgfpathlineto{\pgfqpoint{4.096761in}{3.085235in}}%
\pgfpathlineto{\pgfqpoint{4.098241in}{3.378708in}}%
\pgfpathlineto{\pgfqpoint{4.100214in}{3.378708in}}%
\pgfpathlineto{\pgfqpoint{4.101693in}{3.323683in}}%
\pgfpathlineto{\pgfqpoint{4.102680in}{3.323683in}}%
\pgfpathlineto{\pgfqpoint{4.103666in}{3.377491in}}%
\pgfpathlineto{\pgfqpoint{4.105146in}{3.133619in}}%
\pgfpathlineto{\pgfqpoint{4.105639in}{3.133619in}}%
\pgfpathlineto{\pgfqpoint{4.107119in}{3.079183in}}%
\pgfpathlineto{\pgfqpoint{4.109585in}{3.079183in}}%
\pgfpathlineto{\pgfqpoint{4.111065in}{2.772532in}}%
\pgfpathlineto{\pgfqpoint{4.111558in}{2.772532in}}%
\pgfpathlineto{\pgfqpoint{4.113038in}{3.259110in}}%
\pgfpathlineto{\pgfqpoint{4.114024in}{3.379855in}}%
\pgfpathlineto{\pgfqpoint{4.115504in}{3.203652in}}%
\pgfpathlineto{\pgfqpoint{4.116984in}{3.352520in}}%
\pgfpathlineto{\pgfqpoint{4.117970in}{3.352520in}}%
\pgfpathlineto{\pgfqpoint{4.118957in}{3.364984in}}%
\pgfpathlineto{\pgfqpoint{4.119450in}{3.341618in}}%
\pgfpathlineto{\pgfqpoint{4.121423in}{2.804447in}}%
\pgfpathlineto{\pgfqpoint{4.122902in}{2.804447in}}%
\pgfpathlineto{\pgfqpoint{4.123396in}{3.329660in}}%
\pgfpathlineto{\pgfqpoint{4.124382in}{3.320802in}}%
\pgfpathlineto{\pgfqpoint{4.125862in}{3.313330in}}%
\pgfpathlineto{\pgfqpoint{4.126355in}{3.367078in}}%
\pgfpathlineto{\pgfqpoint{4.127341in}{2.785092in}}%
\pgfpathlineto{\pgfqpoint{4.128821in}{3.350916in}}%
\pgfpathlineto{\pgfqpoint{4.129314in}{3.240961in}}%
\pgfpathlineto{\pgfqpoint{4.129808in}{3.240961in}}%
\pgfpathlineto{\pgfqpoint{4.130301in}{3.267887in}}%
\pgfpathlineto{\pgfqpoint{4.130794in}{3.372227in}}%
\pgfpathlineto{\pgfqpoint{4.131781in}{3.194379in}}%
\pgfpathlineto{\pgfqpoint{4.132274in}{3.239152in}}%
\pgfpathlineto{\pgfqpoint{4.133260in}{3.239152in}}%
\pgfpathlineto{\pgfqpoint{4.133753in}{2.412911in}}%
\pgfpathlineto{\pgfqpoint{4.134740in}{2.531011in}}%
\pgfpathlineto{\pgfqpoint{4.136220in}{3.304079in}}%
\pgfpathlineto{\pgfqpoint{4.136713in}{3.304079in}}%
\pgfpathlineto{\pgfqpoint{4.138193in}{3.092704in}}%
\pgfpathlineto{\pgfqpoint{4.138686in}{3.092704in}}%
\pgfpathlineto{\pgfqpoint{4.139179in}{3.113410in}}%
\pgfpathlineto{\pgfqpoint{4.139672in}{3.378720in}}%
\pgfpathlineto{\pgfqpoint{4.140659in}{3.347361in}}%
\pgfpathlineto{\pgfqpoint{4.141645in}{3.347361in}}%
\pgfpathlineto{\pgfqpoint{4.143125in}{3.243268in}}%
\pgfpathlineto{\pgfqpoint{4.145591in}{3.243268in}}%
\pgfpathlineto{\pgfqpoint{4.146084in}{2.993304in}}%
\pgfpathlineto{\pgfqpoint{4.147564in}{3.370745in}}%
\pgfpathlineto{\pgfqpoint{4.148057in}{3.370745in}}%
\pgfpathlineto{\pgfqpoint{4.148550in}{3.140161in}}%
\pgfpathlineto{\pgfqpoint{4.149044in}{3.378965in}}%
\pgfpathlineto{\pgfqpoint{4.150030in}{3.378965in}}%
\pgfpathlineto{\pgfqpoint{4.152003in}{3.065137in}}%
\pgfpathlineto{\pgfqpoint{4.153483in}{3.239189in}}%
\pgfpathlineto{\pgfqpoint{4.153976in}{3.064016in}}%
\pgfpathlineto{\pgfqpoint{4.154962in}{3.344747in}}%
\pgfpathlineto{\pgfqpoint{4.156442in}{2.656623in}}%
\pgfpathlineto{\pgfqpoint{4.157922in}{3.346057in}}%
\pgfpathlineto{\pgfqpoint{4.158415in}{3.346057in}}%
\pgfpathlineto{\pgfqpoint{4.159895in}{3.379970in}}%
\pgfpathlineto{\pgfqpoint{4.160388in}{3.379970in}}%
\pgfpathlineto{\pgfqpoint{4.161868in}{3.313671in}}%
\pgfpathlineto{\pgfqpoint{4.162854in}{3.313671in}}%
\pgfpathlineto{\pgfqpoint{4.163841in}{3.157885in}}%
\pgfpathlineto{\pgfqpoint{4.164827in}{3.277463in}}%
\pgfpathlineto{\pgfqpoint{4.165320in}{3.236666in}}%
\pgfpathlineto{\pgfqpoint{4.168773in}{3.236666in}}%
\pgfpathlineto{\pgfqpoint{4.169759in}{3.379286in}}%
\pgfpathlineto{\pgfqpoint{4.170253in}{3.367354in}}%
\pgfpathlineto{\pgfqpoint{4.170746in}{3.367354in}}%
\pgfpathlineto{\pgfqpoint{4.172226in}{3.151023in}}%
\pgfpathlineto{\pgfqpoint{4.176665in}{3.151023in}}%
\pgfpathlineto{\pgfqpoint{4.177651in}{2.945297in}}%
\pgfpathlineto{\pgfqpoint{4.178144in}{2.713060in}}%
\pgfpathlineto{\pgfqpoint{4.178638in}{3.379732in}}%
\pgfpathlineto{\pgfqpoint{4.179624in}{3.375572in}}%
\pgfpathlineto{\pgfqpoint{4.180117in}{3.375572in}}%
\pgfpathlineto{\pgfqpoint{4.181597in}{3.321239in}}%
\pgfpathlineto{\pgfqpoint{4.182090in}{2.811009in}}%
\pgfpathlineto{\pgfqpoint{4.182583in}{3.376594in}}%
\pgfpathlineto{\pgfqpoint{4.183570in}{3.376594in}}%
\pgfpathlineto{\pgfqpoint{4.184063in}{3.301960in}}%
\pgfpathlineto{\pgfqpoint{4.185543in}{2.993726in}}%
\pgfpathlineto{\pgfqpoint{4.187022in}{3.379964in}}%
\pgfpathlineto{\pgfqpoint{4.189489in}{3.379964in}}%
\pgfpathlineto{\pgfqpoint{4.190968in}{3.341866in}}%
\pgfpathlineto{\pgfqpoint{4.193434in}{3.341866in}}%
\pgfpathlineto{\pgfqpoint{4.194914in}{3.311622in}}%
\pgfpathlineto{\pgfqpoint{4.195407in}{3.311622in}}%
\pgfpathlineto{\pgfqpoint{4.196394in}{2.247270in}}%
\pgfpathlineto{\pgfqpoint{4.197874in}{3.099865in}}%
\pgfpathlineto{\pgfqpoint{4.199353in}{3.099865in}}%
\pgfpathlineto{\pgfqpoint{4.200340in}{3.254357in}}%
\pgfpathlineto{\pgfqpoint{4.200833in}{2.306715in}}%
\pgfpathlineto{\pgfqpoint{4.201326in}{3.291526in}}%
\pgfpathlineto{\pgfqpoint{4.202806in}{3.121918in}}%
\pgfpathlineto{\pgfqpoint{4.203792in}{3.150548in}}%
\pgfpathlineto{\pgfqpoint{4.204779in}{3.110284in}}%
\pgfpathlineto{\pgfqpoint{4.205765in}{3.378201in}}%
\pgfpathlineto{\pgfqpoint{4.207245in}{2.951825in}}%
\pgfpathlineto{\pgfqpoint{4.208725in}{3.370503in}}%
\pgfpathlineto{\pgfqpoint{4.209711in}{3.370503in}}%
\pgfpathlineto{\pgfqpoint{4.211191in}{3.289111in}}%
\pgfpathlineto{\pgfqpoint{4.213164in}{3.289111in}}%
\pgfpathlineto{\pgfqpoint{4.214643in}{3.378899in}}%
\pgfpathlineto{\pgfqpoint{4.215137in}{3.378899in}}%
\pgfpathlineto{\pgfqpoint{4.216616in}{2.829883in}}%
\pgfpathlineto{\pgfqpoint{4.217110in}{3.302557in}}%
\pgfpathlineto{\pgfqpoint{4.218096in}{3.300573in}}%
\pgfpathlineto{\pgfqpoint{4.218589in}{2.502389in}}%
\pgfpathlineto{\pgfqpoint{4.219082in}{3.270315in}}%
\pgfpathlineto{\pgfqpoint{4.220069in}{3.270315in}}%
\pgfpathlineto{\pgfqpoint{4.221055in}{3.221919in}}%
\pgfpathlineto{\pgfqpoint{4.222535in}{3.343706in}}%
\pgfpathlineto{\pgfqpoint{4.224015in}{2.623571in}}%
\pgfpathlineto{\pgfqpoint{4.225494in}{2.623571in}}%
\pgfpathlineto{\pgfqpoint{4.226974in}{3.379586in}}%
\pgfpathlineto{\pgfqpoint{4.228947in}{3.379586in}}%
\pgfpathlineto{\pgfqpoint{4.229934in}{3.357831in}}%
\pgfpathlineto{\pgfqpoint{4.230427in}{3.119976in}}%
\pgfpathlineto{\pgfqpoint{4.230920in}{3.326075in}}%
\pgfpathlineto{\pgfqpoint{4.232400in}{3.326075in}}%
\pgfpathlineto{\pgfqpoint{4.233879in}{3.375386in}}%
\pgfpathlineto{\pgfqpoint{4.234373in}{3.375386in}}%
\pgfpathlineto{\pgfqpoint{4.234866in}{3.273600in}}%
\pgfpathlineto{\pgfqpoint{4.235359in}{2.556564in}}%
\pgfpathlineto{\pgfqpoint{4.235852in}{3.122033in}}%
\pgfpathlineto{\pgfqpoint{4.236346in}{3.122033in}}%
\pgfpathlineto{\pgfqpoint{4.236839in}{3.148547in}}%
\pgfpathlineto{\pgfqpoint{4.238318in}{3.335143in}}%
\pgfpathlineto{\pgfqpoint{4.239305in}{3.373528in}}%
\pgfpathlineto{\pgfqpoint{4.240291in}{2.641946in}}%
\pgfpathlineto{\pgfqpoint{4.241278in}{3.320030in}}%
\pgfpathlineto{\pgfqpoint{4.242264in}{3.369372in}}%
\pgfpathlineto{\pgfqpoint{4.242758in}{2.647237in}}%
\pgfpathlineto{\pgfqpoint{4.244237in}{3.323345in}}%
\pgfpathlineto{\pgfqpoint{4.244730in}{3.323345in}}%
\pgfpathlineto{\pgfqpoint{4.245224in}{3.328653in}}%
\pgfpathlineto{\pgfqpoint{4.246703in}{3.362309in}}%
\pgfpathlineto{\pgfqpoint{4.248183in}{3.378761in}}%
\pgfpathlineto{\pgfqpoint{4.249663in}{3.378761in}}%
\pgfpathlineto{\pgfqpoint{4.250156in}{3.276675in}}%
\pgfpathlineto{\pgfqpoint{4.250649in}{2.940433in}}%
\pgfpathlineto{\pgfqpoint{4.251143in}{3.210084in}}%
\pgfpathlineto{\pgfqpoint{4.252622in}{3.362541in}}%
\pgfpathlineto{\pgfqpoint{4.253115in}{3.298701in}}%
\pgfpathlineto{\pgfqpoint{4.253609in}{3.329335in}}%
\pgfpathlineto{\pgfqpoint{4.255088in}{3.329335in}}%
\pgfpathlineto{\pgfqpoint{4.256075in}{3.375542in}}%
\pgfpathlineto{\pgfqpoint{4.257555in}{3.012422in}}%
\pgfpathlineto{\pgfqpoint{4.259527in}{3.334713in}}%
\pgfpathlineto{\pgfqpoint{4.261994in}{3.334713in}}%
\pgfpathlineto{\pgfqpoint{4.262487in}{3.225388in}}%
\pgfpathlineto{\pgfqpoint{4.262980in}{3.277759in}}%
\pgfpathlineto{\pgfqpoint{4.267419in}{3.277759in}}%
\pgfpathlineto{\pgfqpoint{4.267912in}{3.048465in}}%
\pgfpathlineto{\pgfqpoint{4.268406in}{3.309184in}}%
\pgfpathlineto{\pgfqpoint{4.268899in}{3.309184in}}%
\pgfpathlineto{\pgfqpoint{4.270379in}{3.310644in}}%
\pgfpathlineto{\pgfqpoint{4.272351in}{3.310644in}}%
\pgfpathlineto{\pgfqpoint{4.272845in}{3.366586in}}%
\pgfpathlineto{\pgfqpoint{4.273831in}{3.364008in}}%
\pgfpathlineto{\pgfqpoint{4.275804in}{3.364008in}}%
\pgfpathlineto{\pgfqpoint{4.276297in}{3.300025in}}%
\pgfpathlineto{\pgfqpoint{4.277777in}{3.083248in}}%
\pgfpathlineto{\pgfqpoint{4.279257in}{3.082442in}}%
\pgfpathlineto{\pgfqpoint{4.280243in}{3.082442in}}%
\pgfpathlineto{\pgfqpoint{4.280736in}{2.403024in}}%
\pgfpathlineto{\pgfqpoint{4.282216in}{3.379598in}}%
\pgfpathlineto{\pgfqpoint{4.283203in}{3.379598in}}%
\pgfpathlineto{\pgfqpoint{4.283696in}{3.052481in}}%
\pgfpathlineto{\pgfqpoint{4.284189in}{3.238175in}}%
\pgfpathlineto{\pgfqpoint{4.285175in}{3.378338in}}%
\pgfpathlineto{\pgfqpoint{4.285669in}{3.375145in}}%
\pgfpathlineto{\pgfqpoint{4.286655in}{3.375145in}}%
\pgfpathlineto{\pgfqpoint{4.287148in}{3.275085in}}%
\pgfpathlineto{\pgfqpoint{4.287642in}{3.378876in}}%
\pgfpathlineto{\pgfqpoint{4.288628in}{3.378876in}}%
\pgfpathlineto{\pgfqpoint{4.290108in}{3.372753in}}%
\pgfpathlineto{\pgfqpoint{4.291094in}{3.264096in}}%
\pgfpathlineto{\pgfqpoint{4.292574in}{3.361773in}}%
\pgfpathlineto{\pgfqpoint{4.295533in}{2.659695in}}%
\pgfpathlineto{\pgfqpoint{4.297013in}{3.379761in}}%
\pgfpathlineto{\pgfqpoint{4.299479in}{3.379761in}}%
\pgfpathlineto{\pgfqpoint{4.300959in}{3.332166in}}%
\pgfpathlineto{\pgfqpoint{4.301945in}{3.332166in}}%
\pgfpathlineto{\pgfqpoint{4.302439in}{3.325996in}}%
\pgfpathlineto{\pgfqpoint{4.302932in}{3.230359in}}%
\pgfpathlineto{\pgfqpoint{4.304411in}{3.364352in}}%
\pgfpathlineto{\pgfqpoint{4.306384in}{3.364352in}}%
\pgfpathlineto{\pgfqpoint{4.306878in}{3.292786in}}%
\pgfpathlineto{\pgfqpoint{4.307371in}{2.934466in}}%
\pgfpathlineto{\pgfqpoint{4.307864in}{3.090110in}}%
\pgfpathlineto{\pgfqpoint{4.309344in}{3.366827in}}%
\pgfpathlineto{\pgfqpoint{4.309837in}{3.366827in}}%
\pgfpathlineto{\pgfqpoint{4.311317in}{3.039288in}}%
\pgfpathlineto{\pgfqpoint{4.312303in}{3.039288in}}%
\pgfpathlineto{\pgfqpoint{4.312796in}{2.873091in}}%
\pgfpathlineto{\pgfqpoint{4.313290in}{3.365930in}}%
\pgfpathlineto{\pgfqpoint{4.314276in}{2.443026in}}%
\pgfpathlineto{\pgfqpoint{4.316249in}{3.299804in}}%
\pgfpathlineto{\pgfqpoint{4.316742in}{3.299804in}}%
\pgfpathlineto{\pgfqpoint{4.317235in}{2.221274in}}%
\pgfpathlineto{\pgfqpoint{4.317729in}{3.353178in}}%
\pgfpathlineto{\pgfqpoint{4.318222in}{3.070092in}}%
\pgfpathlineto{\pgfqpoint{4.318715in}{3.358697in}}%
\pgfpathlineto{\pgfqpoint{4.321181in}{3.358697in}}%
\pgfpathlineto{\pgfqpoint{4.322661in}{3.223771in}}%
\pgfpathlineto{\pgfqpoint{4.324141in}{3.223771in}}%
\pgfpathlineto{\pgfqpoint{4.324634in}{2.627924in}}%
\pgfpathlineto{\pgfqpoint{4.325127in}{2.938839in}}%
\pgfpathlineto{\pgfqpoint{4.326114in}{2.938839in}}%
\pgfpathlineto{\pgfqpoint{4.327593in}{3.369093in}}%
\pgfpathlineto{\pgfqpoint{4.328580in}{3.369093in}}%
\pgfpathlineto{\pgfqpoint{4.330059in}{3.153678in}}%
\pgfpathlineto{\pgfqpoint{4.330553in}{3.153678in}}%
\pgfpathlineto{\pgfqpoint{4.331539in}{3.100930in}}%
\pgfpathlineto{\pgfqpoint{4.332032in}{2.757083in}}%
\pgfpathlineto{\pgfqpoint{4.333512in}{3.291756in}}%
\pgfpathlineto{\pgfqpoint{4.334005in}{3.291756in}}%
\pgfpathlineto{\pgfqpoint{4.334499in}{3.346955in}}%
\pgfpathlineto{\pgfqpoint{4.335485in}{3.082052in}}%
\pgfpathlineto{\pgfqpoint{4.337458in}{3.376360in}}%
\pgfpathlineto{\pgfqpoint{4.338444in}{3.324157in}}%
\pgfpathlineto{\pgfqpoint{4.339924in}{3.029470in}}%
\pgfpathlineto{\pgfqpoint{4.340417in}{2.925136in}}%
\pgfpathlineto{\pgfqpoint{4.341897in}{3.303738in}}%
\pgfpathlineto{\pgfqpoint{4.343870in}{3.303738in}}%
\pgfpathlineto{\pgfqpoint{4.344856in}{2.941971in}}%
\pgfpathlineto{\pgfqpoint{4.346336in}{3.379756in}}%
\pgfpathlineto{\pgfqpoint{4.346829in}{3.379756in}}%
\pgfpathlineto{\pgfqpoint{4.347816in}{3.377539in}}%
\pgfpathlineto{\pgfqpoint{4.349295in}{2.394227in}}%
\pgfpathlineto{\pgfqpoint{4.350775in}{2.394227in}}%
\pgfpathlineto{\pgfqpoint{4.353735in}{3.379417in}}%
\pgfpathlineto{\pgfqpoint{4.355708in}{3.379417in}}%
\pgfpathlineto{\pgfqpoint{4.356694in}{3.341388in}}%
\pgfpathlineto{\pgfqpoint{4.357187in}{3.363175in}}%
\pgfpathlineto{\pgfqpoint{4.357680in}{2.761011in}}%
\pgfpathlineto{\pgfqpoint{4.358174in}{3.378464in}}%
\pgfpathlineto{\pgfqpoint{4.365572in}{3.378464in}}%
\pgfpathlineto{\pgfqpoint{4.367052in}{2.792720in}}%
\pgfpathlineto{\pgfqpoint{4.368532in}{3.379932in}}%
\pgfpathlineto{\pgfqpoint{4.369518in}{3.379932in}}%
\pgfpathlineto{\pgfqpoint{4.370998in}{3.378016in}}%
\pgfpathlineto{\pgfqpoint{4.371491in}{3.378441in}}%
\pgfpathlineto{\pgfqpoint{4.372477in}{3.163688in}}%
\pgfpathlineto{\pgfqpoint{4.373957in}{3.371191in}}%
\pgfpathlineto{\pgfqpoint{4.374944in}{3.371191in}}%
\pgfpathlineto{\pgfqpoint{4.376423in}{3.331138in}}%
\pgfpathlineto{\pgfqpoint{4.378396in}{3.331138in}}%
\pgfpathlineto{\pgfqpoint{4.379383in}{3.374331in}}%
\pgfpathlineto{\pgfqpoint{4.379876in}{3.366086in}}%
\pgfpathlineto{\pgfqpoint{4.381849in}{2.463942in}}%
\pgfpathlineto{\pgfqpoint{4.383328in}{3.379093in}}%
\pgfpathlineto{\pgfqpoint{4.384315in}{3.379093in}}%
\pgfpathlineto{\pgfqpoint{4.385795in}{2.573844in}}%
\pgfpathlineto{\pgfqpoint{4.386288in}{2.951349in}}%
\pgfpathlineto{\pgfqpoint{4.386781in}{2.951349in}}%
\pgfpathlineto{\pgfqpoint{4.388261in}{2.619961in}}%
\pgfpathlineto{\pgfqpoint{4.388754in}{3.310782in}}%
\pgfpathlineto{\pgfqpoint{4.389740in}{3.275762in}}%
\pgfpathlineto{\pgfqpoint{4.391713in}{3.275762in}}%
\pgfpathlineto{\pgfqpoint{4.393193in}{3.370984in}}%
\pgfpathlineto{\pgfqpoint{4.393686in}{3.370984in}}%
\pgfpathlineto{\pgfqpoint{4.394673in}{3.375467in}}%
\pgfpathlineto{\pgfqpoint{4.395659in}{3.167376in}}%
\pgfpathlineto{\pgfqpoint{4.396152in}{3.183031in}}%
\pgfpathlineto{\pgfqpoint{4.397632in}{3.368420in}}%
\pgfpathlineto{\pgfqpoint{4.398125in}{3.368420in}}%
\pgfpathlineto{\pgfqpoint{4.398619in}{3.379493in}}%
\pgfpathlineto{\pgfqpoint{4.400592in}{3.090408in}}%
\pgfpathlineto{\pgfqpoint{4.401085in}{3.372881in}}%
\pgfpathlineto{\pgfqpoint{4.402071in}{3.367094in}}%
\pgfpathlineto{\pgfqpoint{4.404044in}{3.367094in}}%
\pgfpathlineto{\pgfqpoint{4.405524in}{2.916462in}}%
\pgfpathlineto{\pgfqpoint{4.407004in}{3.377132in}}%
\pgfpathlineto{\pgfqpoint{4.407497in}{3.124800in}}%
\pgfpathlineto{\pgfqpoint{4.407990in}{3.378908in}}%
\pgfpathlineto{\pgfqpoint{4.410949in}{3.378908in}}%
\pgfpathlineto{\pgfqpoint{4.411443in}{3.301026in}}%
\pgfpathlineto{\pgfqpoint{4.411936in}{3.379689in}}%
\pgfpathlineto{\pgfqpoint{4.414402in}{3.131842in}}%
\pgfpathlineto{\pgfqpoint{4.414895in}{3.376868in}}%
\pgfpathlineto{\pgfqpoint{4.416375in}{2.703714in}}%
\pgfpathlineto{\pgfqpoint{4.416868in}{2.703714in}}%
\pgfpathlineto{\pgfqpoint{4.418348in}{3.326547in}}%
\pgfpathlineto{\pgfqpoint{4.419334in}{3.326547in}}%
\pgfpathlineto{\pgfqpoint{4.420321in}{3.228317in}}%
\pgfpathlineto{\pgfqpoint{4.421307in}{3.329892in}}%
\pgfpathlineto{\pgfqpoint{4.422787in}{3.188074in}}%
\pgfpathlineto{\pgfqpoint{4.423773in}{3.188074in}}%
\pgfpathlineto{\pgfqpoint{4.424760in}{3.342476in}}%
\pgfpathlineto{\pgfqpoint{4.425746in}{2.884771in}}%
\pgfpathlineto{\pgfqpoint{4.427226in}{3.237293in}}%
\pgfpathlineto{\pgfqpoint{4.429692in}{3.237293in}}%
\pgfpathlineto{\pgfqpoint{4.431172in}{3.378078in}}%
\pgfpathlineto{\pgfqpoint{4.438570in}{3.378078in}}%
\pgfpathlineto{\pgfqpoint{4.439064in}{3.348105in}}%
\pgfpathlineto{\pgfqpoint{4.439557in}{3.379351in}}%
\pgfpathlineto{\pgfqpoint{4.441036in}{3.379351in}}%
\pgfpathlineto{\pgfqpoint{4.442023in}{3.352261in}}%
\pgfpathlineto{\pgfqpoint{4.443009in}{3.379547in}}%
\pgfpathlineto{\pgfqpoint{4.443503in}{3.337687in}}%
\pgfpathlineto{\pgfqpoint{4.443996in}{3.339115in}}%
\pgfpathlineto{\pgfqpoint{4.445476in}{3.378748in}}%
\pgfpathlineto{\pgfqpoint{4.445969in}{3.342414in}}%
\pgfpathlineto{\pgfqpoint{4.446462in}{3.356524in}}%
\pgfpathlineto{\pgfqpoint{4.446955in}{3.356524in}}%
\pgfpathlineto{\pgfqpoint{4.447448in}{3.317785in}}%
\pgfpathlineto{\pgfqpoint{4.447942in}{3.343263in}}%
\pgfpathlineto{\pgfqpoint{4.448435in}{3.339695in}}%
\pgfpathlineto{\pgfqpoint{4.449915in}{3.379187in}}%
\pgfpathlineto{\pgfqpoint{4.450901in}{3.348389in}}%
\pgfpathlineto{\pgfqpoint{4.452381in}{3.153926in}}%
\pgfpathlineto{\pgfqpoint{4.453367in}{3.153926in}}%
\pgfpathlineto{\pgfqpoint{4.454847in}{3.378647in}}%
\pgfpathlineto{\pgfqpoint{4.456327in}{3.378647in}}%
\pgfpathlineto{\pgfqpoint{4.456820in}{2.983137in}}%
\pgfpathlineto{\pgfqpoint{4.457313in}{3.250449in}}%
\pgfpathlineto{\pgfqpoint{4.460766in}{3.250449in}}%
\pgfpathlineto{\pgfqpoint{4.462245in}{3.361087in}}%
\pgfpathlineto{\pgfqpoint{4.463725in}{3.365195in}}%
\pgfpathlineto{\pgfqpoint{4.466685in}{3.365195in}}%
\pgfpathlineto{\pgfqpoint{4.469151in}{3.217424in}}%
\pgfpathlineto{\pgfqpoint{4.471124in}{3.217424in}}%
\pgfpathlineto{\pgfqpoint{4.472603in}{3.338855in}}%
\pgfpathlineto{\pgfqpoint{4.473097in}{3.159702in}}%
\pgfpathlineto{\pgfqpoint{4.473590in}{3.172852in}}%
\pgfpathlineto{\pgfqpoint{4.475069in}{3.332426in}}%
\pgfpathlineto{\pgfqpoint{4.476056in}{3.332426in}}%
\pgfpathlineto{\pgfqpoint{4.477536in}{3.316584in}}%
\pgfpathlineto{\pgfqpoint{4.479015in}{3.316584in}}%
\pgfpathlineto{\pgfqpoint{4.480495in}{2.981577in}}%
\pgfpathlineto{\pgfqpoint{4.480988in}{2.981577in}}%
\pgfpathlineto{\pgfqpoint{4.482468in}{3.378835in}}%
\pgfpathlineto{\pgfqpoint{4.483948in}{3.283188in}}%
\pgfpathlineto{\pgfqpoint{4.484441in}{3.283188in}}%
\pgfpathlineto{\pgfqpoint{4.485427in}{3.276864in}}%
\pgfpathlineto{\pgfqpoint{4.485921in}{3.310112in}}%
\pgfpathlineto{\pgfqpoint{4.486414in}{3.283546in}}%
\pgfpathlineto{\pgfqpoint{4.487400in}{3.283546in}}%
\pgfpathlineto{\pgfqpoint{4.488387in}{2.770309in}}%
\pgfpathlineto{\pgfqpoint{4.488880in}{3.353728in}}%
\pgfpathlineto{\pgfqpoint{4.489373in}{3.338377in}}%
\pgfpathlineto{\pgfqpoint{4.490853in}{2.539465in}}%
\pgfpathlineto{\pgfqpoint{4.491839in}{3.276856in}}%
\pgfpathlineto{\pgfqpoint{4.492333in}{3.233379in}}%
\pgfpathlineto{\pgfqpoint{4.493319in}{3.254354in}}%
\pgfpathlineto{\pgfqpoint{4.493812in}{3.216735in}}%
\pgfpathlineto{\pgfqpoint{4.495292in}{3.378774in}}%
\pgfpathlineto{\pgfqpoint{4.496772in}{2.616876in}}%
\pgfpathlineto{\pgfqpoint{4.497265in}{2.702193in}}%
\pgfpathlineto{\pgfqpoint{4.497758in}{3.251787in}}%
\pgfpathlineto{\pgfqpoint{4.498745in}{3.138673in}}%
\pgfpathlineto{\pgfqpoint{4.500224in}{3.369631in}}%
\pgfpathlineto{\pgfqpoint{4.501704in}{3.369631in}}%
\pgfpathlineto{\pgfqpoint{4.502690in}{3.136203in}}%
\pgfpathlineto{\pgfqpoint{4.503184in}{3.188292in}}%
\pgfpathlineto{\pgfqpoint{4.503677in}{3.188292in}}%
\pgfpathlineto{\pgfqpoint{4.504170in}{3.377555in}}%
\pgfpathlineto{\pgfqpoint{4.505157in}{3.360203in}}%
\pgfpathlineto{\pgfqpoint{4.506143in}{3.360203in}}%
\pgfpathlineto{\pgfqpoint{4.507623in}{3.020005in}}%
\pgfpathlineto{\pgfqpoint{4.508116in}{3.020005in}}%
\pgfpathlineto{\pgfqpoint{4.508609in}{3.379172in}}%
\pgfpathlineto{\pgfqpoint{4.509102in}{3.212842in}}%
\pgfpathlineto{\pgfqpoint{4.510089in}{3.212842in}}%
\pgfpathlineto{\pgfqpoint{4.511075in}{3.371927in}}%
\pgfpathlineto{\pgfqpoint{4.512555in}{3.142647in}}%
\pgfpathlineto{\pgfqpoint{4.514528in}{3.142647in}}%
\pgfpathlineto{\pgfqpoint{4.515021in}{2.926459in}}%
\pgfpathlineto{\pgfqpoint{4.515514in}{3.146143in}}%
\pgfpathlineto{\pgfqpoint{4.516008in}{3.146143in}}%
\pgfpathlineto{\pgfqpoint{4.516501in}{2.889903in}}%
\pgfpathlineto{\pgfqpoint{4.516994in}{3.364572in}}%
\pgfpathlineto{\pgfqpoint{4.517981in}{3.310708in}}%
\pgfpathlineto{\pgfqpoint{4.521433in}{3.310708in}}%
\pgfpathlineto{\pgfqpoint{4.521926in}{2.950113in}}%
\pgfpathlineto{\pgfqpoint{4.522420in}{3.324442in}}%
\pgfpathlineto{\pgfqpoint{4.523406in}{3.324442in}}%
\pgfpathlineto{\pgfqpoint{4.524886in}{3.163124in}}%
\pgfpathlineto{\pgfqpoint{4.525379in}{3.163124in}}%
\pgfpathlineto{\pgfqpoint{4.527845in}{2.977980in}}%
\pgfpathlineto{\pgfqpoint{4.529325in}{3.358200in}}%
\pgfpathlineto{\pgfqpoint{4.530805in}{3.327148in}}%
\pgfpathlineto{\pgfqpoint{4.531298in}{3.327148in}}%
\pgfpathlineto{\pgfqpoint{4.531791in}{2.474259in}}%
\pgfpathlineto{\pgfqpoint{4.532284in}{3.187832in}}%
\pgfpathlineto{\pgfqpoint{4.534257in}{3.249019in}}%
\pgfpathlineto{\pgfqpoint{4.536230in}{3.249019in}}%
\pgfpathlineto{\pgfqpoint{4.537710in}{2.771535in}}%
\pgfpathlineto{\pgfqpoint{4.541656in}{2.771535in}}%
\pgfpathlineto{\pgfqpoint{4.542149in}{3.286815in}}%
\pgfpathlineto{\pgfqpoint{4.543135in}{3.209114in}}%
\pgfpathlineto{\pgfqpoint{4.544122in}{3.209114in}}%
\pgfpathlineto{\pgfqpoint{4.545601in}{3.278883in}}%
\pgfpathlineto{\pgfqpoint{4.546095in}{3.278883in}}%
\pgfpathlineto{\pgfqpoint{4.547081in}{3.180113in}}%
\pgfpathlineto{\pgfqpoint{4.548068in}{3.357256in}}%
\pgfpathlineto{\pgfqpoint{4.548561in}{2.905236in}}%
\pgfpathlineto{\pgfqpoint{4.549054in}{3.073758in}}%
\pgfpathlineto{\pgfqpoint{4.549547in}{3.073758in}}%
\pgfpathlineto{\pgfqpoint{4.550041in}{3.125335in}}%
\pgfpathlineto{\pgfqpoint{4.551027in}{3.379507in}}%
\pgfpathlineto{\pgfqpoint{4.551520in}{3.353322in}}%
\pgfpathlineto{\pgfqpoint{4.554480in}{3.353322in}}%
\pgfpathlineto{\pgfqpoint{4.554973in}{3.350383in}}%
\pgfpathlineto{\pgfqpoint{4.555959in}{2.988173in}}%
\pgfpathlineto{\pgfqpoint{4.557932in}{3.317041in}}%
\pgfpathlineto{\pgfqpoint{4.561385in}{3.018101in}}%
\pgfpathlineto{\pgfqpoint{4.561878in}{3.082554in}}%
\pgfpathlineto{\pgfqpoint{4.563358in}{3.374158in}}%
\pgfpathlineto{\pgfqpoint{4.563851in}{2.975162in}}%
\pgfpathlineto{\pgfqpoint{4.564344in}{3.151366in}}%
\pgfpathlineto{\pgfqpoint{4.565824in}{3.151366in}}%
\pgfpathlineto{\pgfqpoint{4.566317in}{3.197839in}}%
\pgfpathlineto{\pgfqpoint{4.567797in}{3.379808in}}%
\pgfpathlineto{\pgfqpoint{4.568290in}{3.207514in}}%
\pgfpathlineto{\pgfqpoint{4.568783in}{3.365985in}}%
\pgfpathlineto{\pgfqpoint{4.569277in}{3.365985in}}%
\pgfpathlineto{\pgfqpoint{4.570263in}{3.377417in}}%
\pgfpathlineto{\pgfqpoint{4.571250in}{3.015547in}}%
\pgfpathlineto{\pgfqpoint{4.571743in}{3.289375in}}%
\pgfpathlineto{\pgfqpoint{4.572236in}{3.117853in}}%
\pgfpathlineto{\pgfqpoint{4.575689in}{3.117853in}}%
\pgfpathlineto{\pgfqpoint{4.576675in}{2.787426in}}%
\pgfpathlineto{\pgfqpoint{4.578155in}{3.278409in}}%
\pgfpathlineto{\pgfqpoint{4.580128in}{2.556983in}}%
\pgfpathlineto{\pgfqpoint{4.581607in}{2.556983in}}%
\pgfpathlineto{\pgfqpoint{4.583087in}{3.361083in}}%
\pgfpathlineto{\pgfqpoint{4.583580in}{3.005655in}}%
\pgfpathlineto{\pgfqpoint{4.584074in}{3.376787in}}%
\pgfpathlineto{\pgfqpoint{4.585553in}{3.375488in}}%
\pgfpathlineto{\pgfqpoint{4.589992in}{3.375488in}}%
\pgfpathlineto{\pgfqpoint{4.590486in}{3.183562in}}%
\pgfpathlineto{\pgfqpoint{4.590979in}{3.320212in}}%
\pgfpathlineto{\pgfqpoint{4.591472in}{3.320212in}}%
\pgfpathlineto{\pgfqpoint{4.592952in}{3.338235in}}%
\pgfpathlineto{\pgfqpoint{4.594431in}{3.338235in}}%
\pgfpathlineto{\pgfqpoint{4.594925in}{3.174328in}}%
\pgfpathlineto{\pgfqpoint{4.595418in}{3.375643in}}%
\pgfpathlineto{\pgfqpoint{4.596404in}{3.375643in}}%
\pgfpathlineto{\pgfqpoint{4.598377in}{2.375231in}}%
\pgfpathlineto{\pgfqpoint{4.598870in}{2.375231in}}%
\pgfpathlineto{\pgfqpoint{4.600350in}{3.344446in}}%
\pgfpathlineto{\pgfqpoint{4.601337in}{3.319353in}}%
\pgfpathlineto{\pgfqpoint{4.601830in}{2.766871in}}%
\pgfpathlineto{\pgfqpoint{4.602323in}{3.297863in}}%
\pgfpathlineto{\pgfqpoint{4.603310in}{3.297863in}}%
\pgfpathlineto{\pgfqpoint{4.604789in}{3.016907in}}%
\pgfpathlineto{\pgfqpoint{4.605282in}{3.016907in}}%
\pgfpathlineto{\pgfqpoint{4.606762in}{3.259112in}}%
\pgfpathlineto{\pgfqpoint{4.608242in}{3.305628in}}%
\pgfpathlineto{\pgfqpoint{4.608735in}{3.305628in}}%
\pgfpathlineto{\pgfqpoint{4.610215in}{3.326559in}}%
\pgfpathlineto{\pgfqpoint{4.615147in}{3.326559in}}%
\pgfpathlineto{\pgfqpoint{4.616134in}{2.357542in}}%
\pgfpathlineto{\pgfqpoint{4.617613in}{3.354003in}}%
\pgfpathlineto{\pgfqpoint{4.618106in}{3.354003in}}%
\pgfpathlineto{\pgfqpoint{4.619586in}{2.789543in}}%
\pgfpathlineto{\pgfqpoint{4.621066in}{3.121263in}}%
\pgfpathlineto{\pgfqpoint{4.623039in}{3.121263in}}%
\pgfpathlineto{\pgfqpoint{4.624025in}{3.377262in}}%
\pgfpathlineto{\pgfqpoint{4.625012in}{3.223779in}}%
\pgfpathlineto{\pgfqpoint{4.625505in}{3.249720in}}%
\pgfpathlineto{\pgfqpoint{4.627478in}{3.249720in}}%
\pgfpathlineto{\pgfqpoint{4.628958in}{3.234707in}}%
\pgfpathlineto{\pgfqpoint{4.629451in}{3.343756in}}%
\pgfpathlineto{\pgfqpoint{4.629944in}{3.046593in}}%
\pgfpathlineto{\pgfqpoint{4.630437in}{3.111942in}}%
\pgfpathlineto{\pgfqpoint{4.631917in}{3.365063in}}%
\pgfpathlineto{\pgfqpoint{4.635370in}{3.365063in}}%
\pgfpathlineto{\pgfqpoint{4.636849in}{3.318361in}}%
\pgfpathlineto{\pgfqpoint{4.637342in}{3.318361in}}%
\pgfpathlineto{\pgfqpoint{4.638822in}{3.351014in}}%
\pgfpathlineto{\pgfqpoint{4.640302in}{3.351014in}}%
\pgfpathlineto{\pgfqpoint{4.640795in}{2.808605in}}%
\pgfpathlineto{\pgfqpoint{4.641288in}{3.338265in}}%
\pgfpathlineto{\pgfqpoint{4.642275in}{3.338265in}}%
\pgfpathlineto{\pgfqpoint{4.642768in}{3.071112in}}%
\pgfpathlineto{\pgfqpoint{4.643261in}{3.358160in}}%
\pgfpathlineto{\pgfqpoint{4.643754in}{3.358160in}}%
\pgfpathlineto{\pgfqpoint{4.644248in}{1.807324in}}%
\pgfpathlineto{\pgfqpoint{4.644741in}{2.280177in}}%
\pgfpathlineto{\pgfqpoint{4.646221in}{2.280177in}}%
\pgfpathlineto{\pgfqpoint{4.649180in}{2.947020in}}%
\pgfpathlineto{\pgfqpoint{4.651153in}{2.947020in}}%
\pgfpathlineto{\pgfqpoint{4.652633in}{3.368955in}}%
\pgfpathlineto{\pgfqpoint{4.656085in}{3.368955in}}%
\pgfpathlineto{\pgfqpoint{4.656578in}{3.356607in}}%
\pgfpathlineto{\pgfqpoint{4.658058in}{3.200795in}}%
\pgfpathlineto{\pgfqpoint{4.658551in}{3.200795in}}%
\pgfpathlineto{\pgfqpoint{4.659538in}{3.350786in}}%
\pgfpathlineto{\pgfqpoint{4.661018in}{2.989528in}}%
\pgfpathlineto{\pgfqpoint{4.662497in}{3.379721in}}%
\pgfpathlineto{\pgfqpoint{4.663977in}{3.068809in}}%
\pgfpathlineto{\pgfqpoint{4.664963in}{1.741353in}}%
\pgfpathlineto{\pgfqpoint{4.667430in}{2.943646in}}%
\pgfpathlineto{\pgfqpoint{4.669403in}{2.943646in}}%
\pgfpathlineto{\pgfqpoint{4.670882in}{3.378001in}}%
\pgfpathlineto{\pgfqpoint{4.673348in}{3.378925in}}%
\pgfpathlineto{\pgfqpoint{4.673842in}{3.379989in}}%
\pgfpathlineto{\pgfqpoint{4.674828in}{3.200199in}}%
\pgfpathlineto{\pgfqpoint{4.675321in}{3.374212in}}%
\pgfpathlineto{\pgfqpoint{4.676308in}{3.361695in}}%
\pgfpathlineto{\pgfqpoint{4.679267in}{3.361695in}}%
\pgfpathlineto{\pgfqpoint{4.680254in}{2.484158in}}%
\pgfpathlineto{\pgfqpoint{4.681240in}{3.150787in}}%
\pgfpathlineto{\pgfqpoint{4.681733in}{2.347689in}}%
\pgfpathlineto{\pgfqpoint{4.683213in}{3.354938in}}%
\pgfpathlineto{\pgfqpoint{4.683706in}{3.354938in}}%
\pgfpathlineto{\pgfqpoint{4.685186in}{3.226328in}}%
\pgfpathlineto{\pgfqpoint{4.686666in}{3.226328in}}%
\pgfpathlineto{\pgfqpoint{4.687159in}{2.621515in}}%
\pgfpathlineto{\pgfqpoint{4.687652in}{2.803687in}}%
\pgfpathlineto{\pgfqpoint{4.689625in}{3.378901in}}%
\pgfpathlineto{\pgfqpoint{4.690118in}{3.064103in}}%
\pgfpathlineto{\pgfqpoint{4.690611in}{3.379700in}}%
\pgfpathlineto{\pgfqpoint{4.692584in}{3.379700in}}%
\pgfpathlineto{\pgfqpoint{4.694064in}{3.068929in}}%
\pgfpathlineto{\pgfqpoint{4.694557in}{3.252127in}}%
\pgfpathlineto{\pgfqpoint{4.695051in}{2.018865in}}%
\pgfpathlineto{\pgfqpoint{4.695544in}{3.339878in}}%
\pgfpathlineto{\pgfqpoint{4.698503in}{2.904288in}}%
\pgfpathlineto{\pgfqpoint{4.699490in}{3.333667in}}%
\pgfpathlineto{\pgfqpoint{4.699983in}{3.090968in}}%
\pgfpathlineto{\pgfqpoint{4.700476in}{3.377652in}}%
\pgfpathlineto{\pgfqpoint{4.702942in}{3.377652in}}%
\pgfpathlineto{\pgfqpoint{4.704915in}{2.921602in}}%
\pgfpathlineto{\pgfqpoint{4.705408in}{2.921602in}}%
\pgfpathlineto{\pgfqpoint{4.705902in}{3.349379in}}%
\pgfpathlineto{\pgfqpoint{4.706395in}{3.148765in}}%
\pgfpathlineto{\pgfqpoint{4.706888in}{3.148765in}}%
\pgfpathlineto{\pgfqpoint{4.708368in}{3.370301in}}%
\pgfpathlineto{\pgfqpoint{4.709847in}{3.379733in}}%
\pgfpathlineto{\pgfqpoint{4.710341in}{3.379733in}}%
\pgfpathlineto{\pgfqpoint{4.710834in}{3.306731in}}%
\pgfpathlineto{\pgfqpoint{4.711327in}{3.378539in}}%
\pgfpathlineto{\pgfqpoint{4.715273in}{3.378539in}}%
\pgfpathlineto{\pgfqpoint{4.716753in}{3.323258in}}%
\pgfpathlineto{\pgfqpoint{4.717246in}{3.323258in}}%
\pgfpathlineto{\pgfqpoint{4.718232in}{2.665173in}}%
\pgfpathlineto{\pgfqpoint{4.718726in}{3.032433in}}%
\pgfpathlineto{\pgfqpoint{4.719219in}{3.032433in}}%
\pgfpathlineto{\pgfqpoint{4.720699in}{3.379796in}}%
\pgfpathlineto{\pgfqpoint{4.721192in}{3.379796in}}%
\pgfpathlineto{\pgfqpoint{4.722671in}{3.334630in}}%
\pgfpathlineto{\pgfqpoint{4.723165in}{3.039423in}}%
\pgfpathlineto{\pgfqpoint{4.723658in}{3.315400in}}%
\pgfpathlineto{\pgfqpoint{4.725138in}{3.337694in}}%
\pgfpathlineto{\pgfqpoint{4.726124in}{3.337694in}}%
\pgfpathlineto{\pgfqpoint{4.727604in}{3.190557in}}%
\pgfpathlineto{\pgfqpoint{4.728097in}{3.190557in}}%
\pgfpathlineto{\pgfqpoint{4.728590in}{3.373089in}}%
\pgfpathlineto{\pgfqpoint{4.729577in}{3.344194in}}%
\pgfpathlineto{\pgfqpoint{4.730070in}{3.344194in}}%
\pgfpathlineto{\pgfqpoint{4.732043in}{2.231963in}}%
\pgfpathlineto{\pgfqpoint{4.732536in}{3.371249in}}%
\pgfpathlineto{\pgfqpoint{4.733523in}{3.171606in}}%
\pgfpathlineto{\pgfqpoint{4.734016in}{3.171606in}}%
\pgfpathlineto{\pgfqpoint{4.735002in}{3.159981in}}%
\pgfpathlineto{\pgfqpoint{4.736482in}{3.365468in}}%
\pgfpathlineto{\pgfqpoint{4.737962in}{3.365468in}}%
\pgfpathlineto{\pgfqpoint{4.738455in}{3.379993in}}%
\pgfpathlineto{\pgfqpoint{4.739935in}{3.291086in}}%
\pgfpathlineto{\pgfqpoint{4.740921in}{2.922014in}}%
\pgfpathlineto{\pgfqpoint{4.741414in}{3.029904in}}%
\pgfpathlineto{\pgfqpoint{4.741907in}{2.989030in}}%
\pgfpathlineto{\pgfqpoint{4.743387in}{3.337583in}}%
\pgfpathlineto{\pgfqpoint{4.743880in}{3.337583in}}%
\pgfpathlineto{\pgfqpoint{4.744374in}{3.314224in}}%
\pgfpathlineto{\pgfqpoint{4.744867in}{2.286819in}}%
\pgfpathlineto{\pgfqpoint{4.745360in}{3.128178in}}%
\pgfpathlineto{\pgfqpoint{4.746347in}{3.128178in}}%
\pgfpathlineto{\pgfqpoint{4.747826in}{2.244327in}}%
\pgfpathlineto{\pgfqpoint{4.749306in}{2.933807in}}%
\pgfpathlineto{\pgfqpoint{4.751279in}{2.933807in}}%
\pgfpathlineto{\pgfqpoint{4.751772in}{3.378902in}}%
\pgfpathlineto{\pgfqpoint{4.752265in}{3.217552in}}%
\pgfpathlineto{\pgfqpoint{4.752759in}{3.157852in}}%
\pgfpathlineto{\pgfqpoint{4.753252in}{3.379734in}}%
\pgfpathlineto{\pgfqpoint{4.753745in}{3.178542in}}%
\pgfpathlineto{\pgfqpoint{4.754238in}{3.178542in}}%
\pgfpathlineto{\pgfqpoint{4.755718in}{3.331249in}}%
\pgfpathlineto{\pgfqpoint{4.756211in}{3.301891in}}%
\pgfpathlineto{\pgfqpoint{4.757691in}{3.370954in}}%
\pgfpathlineto{\pgfqpoint{4.759664in}{3.370954in}}%
\pgfpathlineto{\pgfqpoint{4.761144in}{3.172313in}}%
\pgfpathlineto{\pgfqpoint{4.762623in}{3.369883in}}%
\pgfpathlineto{\pgfqpoint{4.764103in}{3.267918in}}%
\pgfpathlineto{\pgfqpoint{4.764596in}{3.267918in}}%
\pgfpathlineto{\pgfqpoint{4.766076in}{3.315516in}}%
\pgfpathlineto{\pgfqpoint{4.767062in}{3.315516in}}%
\pgfpathlineto{\pgfqpoint{4.767556in}{3.253192in}}%
\pgfpathlineto{\pgfqpoint{4.768049in}{3.287473in}}%
\pgfpathlineto{\pgfqpoint{4.768542in}{3.376054in}}%
\pgfpathlineto{\pgfqpoint{4.769035in}{3.287932in}}%
\pgfpathlineto{\pgfqpoint{4.771501in}{3.287932in}}%
\pgfpathlineto{\pgfqpoint{4.772981in}{3.058334in}}%
\pgfpathlineto{\pgfqpoint{4.774461in}{3.058334in}}%
\pgfpathlineto{\pgfqpoint{4.774954in}{3.037933in}}%
\pgfpathlineto{\pgfqpoint{4.776927in}{3.368038in}}%
\pgfpathlineto{\pgfqpoint{4.777913in}{3.379770in}}%
\pgfpathlineto{\pgfqpoint{4.778900in}{3.249309in}}%
\pgfpathlineto{\pgfqpoint{4.779886in}{2.768527in}}%
\pgfpathlineto{\pgfqpoint{4.781859in}{3.310464in}}%
\pgfpathlineto{\pgfqpoint{4.782352in}{3.310464in}}%
\pgfpathlineto{\pgfqpoint{4.783339in}{3.341572in}}%
\pgfpathlineto{\pgfqpoint{4.784325in}{3.089286in}}%
\pgfpathlineto{\pgfqpoint{4.784819in}{3.213845in}}%
\pgfpathlineto{\pgfqpoint{4.786298in}{2.747139in}}%
\pgfpathlineto{\pgfqpoint{4.788271in}{2.747139in}}%
\pgfpathlineto{\pgfqpoint{4.789751in}{3.379230in}}%
\pgfpathlineto{\pgfqpoint{4.790737in}{3.379230in}}%
\pgfpathlineto{\pgfqpoint{4.792710in}{2.866279in}}%
\pgfpathlineto{\pgfqpoint{4.795176in}{2.866279in}}%
\pgfpathlineto{\pgfqpoint{4.797149in}{2.696723in}}%
\pgfpathlineto{\pgfqpoint{4.798629in}{2.690733in}}%
\pgfpathlineto{\pgfqpoint{4.799616in}{2.690733in}}%
\pgfpathlineto{\pgfqpoint{4.801095in}{3.374684in}}%
\pgfpathlineto{\pgfqpoint{4.801588in}{3.181813in}}%
\pgfpathlineto{\pgfqpoint{4.802082in}{3.378241in}}%
\pgfpathlineto{\pgfqpoint{4.802575in}{3.226441in}}%
\pgfpathlineto{\pgfqpoint{4.804055in}{3.226441in}}%
\pgfpathlineto{\pgfqpoint{4.806028in}{2.691561in}}%
\pgfpathlineto{\pgfqpoint{4.806521in}{2.691561in}}%
\pgfpathlineto{\pgfqpoint{4.808000in}{3.086272in}}%
\pgfpathlineto{\pgfqpoint{4.809480in}{2.987781in}}%
\pgfpathlineto{\pgfqpoint{4.811453in}{2.987781in}}%
\pgfpathlineto{\pgfqpoint{4.812933in}{3.374088in}}%
\pgfpathlineto{\pgfqpoint{4.814906in}{3.374088in}}%
\pgfpathlineto{\pgfqpoint{4.816385in}{3.363745in}}%
\pgfpathlineto{\pgfqpoint{4.817865in}{3.375438in}}%
\pgfpathlineto{\pgfqpoint{4.819345in}{3.375438in}}%
\pgfpathlineto{\pgfqpoint{4.819838in}{2.720785in}}%
\pgfpathlineto{\pgfqpoint{4.820331in}{3.064568in}}%
\pgfpathlineto{\pgfqpoint{4.821318in}{3.332500in}}%
\pgfpathlineto{\pgfqpoint{4.821811in}{2.951614in}}%
\pgfpathlineto{\pgfqpoint{4.822304in}{3.336128in}}%
\pgfpathlineto{\pgfqpoint{4.823784in}{3.245584in}}%
\pgfpathlineto{\pgfqpoint{4.824277in}{3.245584in}}%
\pgfpathlineto{\pgfqpoint{4.825757in}{3.310324in}}%
\pgfpathlineto{\pgfqpoint{4.827730in}{3.310324in}}%
\pgfpathlineto{\pgfqpoint{4.828223in}{3.216305in}}%
\pgfpathlineto{\pgfqpoint{4.828716in}{3.306570in}}%
\pgfpathlineto{\pgfqpoint{4.830196in}{2.791176in}}%
\pgfpathlineto{\pgfqpoint{4.830689in}{2.919382in}}%
\pgfpathlineto{\pgfqpoint{4.831676in}{3.336882in}}%
\pgfpathlineto{\pgfqpoint{4.832662in}{3.059088in}}%
\pgfpathlineto{\pgfqpoint{4.833648in}{3.147780in}}%
\pgfpathlineto{\pgfqpoint{4.835128in}{2.865548in}}%
\pgfpathlineto{\pgfqpoint{4.836608in}{2.774643in}}%
\pgfpathlineto{\pgfqpoint{4.837594in}{3.172823in}}%
\pgfpathlineto{\pgfqpoint{4.838088in}{3.116137in}}%
\pgfpathlineto{\pgfqpoint{4.840554in}{3.116137in}}%
\pgfpathlineto{\pgfqpoint{4.841540in}{2.902064in}}%
\pgfpathlineto{\pgfqpoint{4.843020in}{3.379999in}}%
\pgfpathlineto{\pgfqpoint{4.843513in}{3.379999in}}%
\pgfpathlineto{\pgfqpoint{4.844006in}{3.376047in}}%
\pgfpathlineto{\pgfqpoint{4.844500in}{3.379592in}}%
\pgfpathlineto{\pgfqpoint{4.845486in}{3.379592in}}%
\pgfpathlineto{\pgfqpoint{4.846966in}{3.242517in}}%
\pgfpathlineto{\pgfqpoint{4.847459in}{3.242517in}}%
\pgfpathlineto{\pgfqpoint{4.848939in}{3.117366in}}%
\pgfpathlineto{\pgfqpoint{4.850418in}{3.117366in}}%
\pgfpathlineto{\pgfqpoint{4.850912in}{3.364097in}}%
\pgfpathlineto{\pgfqpoint{4.851405in}{3.126401in}}%
\pgfpathlineto{\pgfqpoint{4.851898in}{3.126401in}}%
\pgfpathlineto{\pgfqpoint{4.852391in}{3.379894in}}%
\pgfpathlineto{\pgfqpoint{4.853378in}{3.349198in}}%
\pgfpathlineto{\pgfqpoint{4.853871in}{3.349198in}}%
\pgfpathlineto{\pgfqpoint{4.854364in}{3.328491in}}%
\pgfpathlineto{\pgfqpoint{4.855844in}{2.761801in}}%
\pgfpathlineto{\pgfqpoint{4.857324in}{3.363127in}}%
\pgfpathlineto{\pgfqpoint{4.858803in}{3.106197in}}%
\pgfpathlineto{\pgfqpoint{4.859790in}{3.106197in}}%
\pgfpathlineto{\pgfqpoint{4.860283in}{3.349393in}}%
\pgfpathlineto{\pgfqpoint{4.861269in}{3.308477in}}%
\pgfpathlineto{\pgfqpoint{4.862749in}{3.254475in}}%
\pgfpathlineto{\pgfqpoint{4.863736in}{3.254475in}}%
\pgfpathlineto{\pgfqpoint{4.864722in}{3.299412in}}%
\pgfpathlineto{\pgfqpoint{4.866202in}{3.238051in}}%
\pgfpathlineto{\pgfqpoint{4.867188in}{3.378341in}}%
\pgfpathlineto{\pgfqpoint{4.867681in}{3.346005in}}%
\pgfpathlineto{\pgfqpoint{4.868668in}{3.326438in}}%
\pgfpathlineto{\pgfqpoint{4.870148in}{3.174270in}}%
\pgfpathlineto{\pgfqpoint{4.871627in}{3.318444in}}%
\pgfpathlineto{\pgfqpoint{4.873107in}{3.318444in}}%
\pgfpathlineto{\pgfqpoint{4.876066in}{3.379806in}}%
\pgfpathlineto{\pgfqpoint{4.877053in}{3.379806in}}%
\pgfpathlineto{\pgfqpoint{4.878039in}{3.343260in}}%
\pgfpathlineto{\pgfqpoint{4.880012in}{2.820951in}}%
\pgfpathlineto{\pgfqpoint{4.880505in}{3.098773in}}%
\pgfpathlineto{\pgfqpoint{4.880999in}{2.843392in}}%
\pgfpathlineto{\pgfqpoint{4.881492in}{2.843392in}}%
\pgfpathlineto{\pgfqpoint{4.882972in}{3.026479in}}%
\pgfpathlineto{\pgfqpoint{4.886424in}{3.377968in}}%
\pgfpathlineto{\pgfqpoint{4.888890in}{3.377968in}}%
\pgfpathlineto{\pgfqpoint{4.889384in}{3.359378in}}%
\pgfpathlineto{\pgfqpoint{4.890370in}{3.274112in}}%
\pgfpathlineto{\pgfqpoint{4.891850in}{3.308452in}}%
\pgfpathlineto{\pgfqpoint{4.893329in}{3.319206in}}%
\pgfpathlineto{\pgfqpoint{4.894316in}{3.319206in}}%
\pgfpathlineto{\pgfqpoint{4.894809in}{3.284429in}}%
\pgfpathlineto{\pgfqpoint{4.896289in}{3.372321in}}%
\pgfpathlineto{\pgfqpoint{4.896782in}{3.372321in}}%
\pgfpathlineto{\pgfqpoint{4.899248in}{3.312990in}}%
\pgfpathlineto{\pgfqpoint{4.900728in}{3.339720in}}%
\pgfpathlineto{\pgfqpoint{4.901221in}{3.339720in}}%
\pgfpathlineto{\pgfqpoint{4.901714in}{3.301959in}}%
\pgfpathlineto{\pgfqpoint{4.902208in}{3.161256in}}%
\pgfpathlineto{\pgfqpoint{4.903194in}{3.345836in}}%
\pgfpathlineto{\pgfqpoint{4.904181in}{2.608610in}}%
\pgfpathlineto{\pgfqpoint{4.905660in}{2.958446in}}%
\pgfpathlineto{\pgfqpoint{4.906647in}{2.950688in}}%
\pgfpathlineto{\pgfqpoint{4.908126in}{3.282206in}}%
\pgfpathlineto{\pgfqpoint{4.908620in}{3.129302in}}%
\pgfpathlineto{\pgfqpoint{4.909113in}{3.354677in}}%
\pgfpathlineto{\pgfqpoint{4.909606in}{3.275799in}}%
\pgfpathlineto{\pgfqpoint{4.910593in}{3.275799in}}%
\pgfpathlineto{\pgfqpoint{4.912565in}{2.372499in}}%
\pgfpathlineto{\pgfqpoint{4.914045in}{2.447433in}}%
\pgfpathlineto{\pgfqpoint{4.915525in}{3.302139in}}%
\pgfpathlineto{\pgfqpoint{4.916511in}{3.302139in}}%
\pgfpathlineto{\pgfqpoint{4.917498in}{3.318436in}}%
\pgfpathlineto{\pgfqpoint{4.917991in}{3.054751in}}%
\pgfpathlineto{\pgfqpoint{4.918484in}{3.367899in}}%
\pgfpathlineto{\pgfqpoint{4.920457in}{3.367899in}}%
\pgfpathlineto{\pgfqpoint{4.921937in}{3.379796in}}%
\pgfpathlineto{\pgfqpoint{4.922923in}{2.846397in}}%
\pgfpathlineto{\pgfqpoint{4.924403in}{3.370672in}}%
\pgfpathlineto{\pgfqpoint{4.925883in}{3.370672in}}%
\pgfpathlineto{\pgfqpoint{4.927362in}{3.360203in}}%
\pgfpathlineto{\pgfqpoint{4.928842in}{3.360203in}}%
\pgfpathlineto{\pgfqpoint{4.930322in}{3.198713in}}%
\pgfpathlineto{\pgfqpoint{4.932295in}{3.198713in}}%
\pgfpathlineto{\pgfqpoint{4.932788in}{3.303704in}}%
\pgfpathlineto{\pgfqpoint{4.933774in}{3.283084in}}%
\pgfpathlineto{\pgfqpoint{4.934761in}{3.353739in}}%
\pgfpathlineto{\pgfqpoint{4.935747in}{3.075774in}}%
\pgfpathlineto{\pgfqpoint{4.937227in}{3.185184in}}%
\pgfpathlineto{\pgfqpoint{4.938707in}{2.991217in}}%
\pgfpathlineto{\pgfqpoint{4.939200in}{3.161754in}}%
\pgfpathlineto{\pgfqpoint{4.939693in}{3.079848in}}%
\pgfpathlineto{\pgfqpoint{4.940680in}{3.079848in}}%
\pgfpathlineto{\pgfqpoint{4.942159in}{3.377047in}}%
\pgfpathlineto{\pgfqpoint{4.943146in}{3.377047in}}%
\pgfpathlineto{\pgfqpoint{4.944132in}{3.182369in}}%
\pgfpathlineto{\pgfqpoint{4.944625in}{3.374820in}}%
\pgfpathlineto{\pgfqpoint{4.946105in}{3.069787in}}%
\pgfpathlineto{\pgfqpoint{4.947585in}{3.294256in}}%
\pgfpathlineto{\pgfqpoint{4.948571in}{3.294256in}}%
\pgfpathlineto{\pgfqpoint{4.950051in}{3.346899in}}%
\pgfpathlineto{\pgfqpoint{4.951531in}{3.346899in}}%
\pgfpathlineto{\pgfqpoint{4.952517in}{3.379704in}}%
\pgfpathlineto{\pgfqpoint{4.953504in}{2.972267in}}%
\pgfpathlineto{\pgfqpoint{4.953997in}{2.594084in}}%
\pgfpathlineto{\pgfqpoint{4.955477in}{3.294586in}}%
\pgfpathlineto{\pgfqpoint{4.955970in}{3.294586in}}%
\pgfpathlineto{\pgfqpoint{4.957449in}{3.346172in}}%
\pgfpathlineto{\pgfqpoint{4.957943in}{3.081615in}}%
\pgfpathlineto{\pgfqpoint{4.958436in}{3.116037in}}%
\pgfpathlineto{\pgfqpoint{4.960409in}{3.330995in}}%
\pgfpathlineto{\pgfqpoint{4.961395in}{3.300289in}}%
\pgfpathlineto{\pgfqpoint{4.962382in}{3.052469in}}%
\pgfpathlineto{\pgfqpoint{4.962875in}{3.363377in}}%
\pgfpathlineto{\pgfqpoint{4.963861in}{3.319818in}}%
\pgfpathlineto{\pgfqpoint{4.965341in}{3.319818in}}%
\pgfpathlineto{\pgfqpoint{4.966821in}{3.273267in}}%
\pgfpathlineto{\pgfqpoint{4.967314in}{2.558006in}}%
\pgfpathlineto{\pgfqpoint{4.967807in}{3.224728in}}%
\pgfpathlineto{\pgfqpoint{4.969287in}{3.262762in}}%
\pgfpathlineto{\pgfqpoint{4.970274in}{3.162188in}}%
\pgfpathlineto{\pgfqpoint{4.970767in}{3.173747in}}%
\pgfpathlineto{\pgfqpoint{4.971753in}{3.173747in}}%
\pgfpathlineto{\pgfqpoint{4.972246in}{2.528021in}}%
\pgfpathlineto{\pgfqpoint{4.972740in}{3.333796in}}%
\pgfpathlineto{\pgfqpoint{4.973233in}{3.333796in}}%
\pgfpathlineto{\pgfqpoint{4.973726in}{3.299267in}}%
\pgfpathlineto{\pgfqpoint{4.974219in}{3.361480in}}%
\pgfpathlineto{\pgfqpoint{4.975699in}{3.234650in}}%
\pgfpathlineto{\pgfqpoint{4.976192in}{3.234650in}}%
\pgfpathlineto{\pgfqpoint{4.977672in}{2.950868in}}%
\pgfpathlineto{\pgfqpoint{4.978165in}{3.343268in}}%
\pgfpathlineto{\pgfqpoint{4.979152in}{3.324492in}}%
\pgfpathlineto{\pgfqpoint{4.981125in}{3.324492in}}%
\pgfpathlineto{\pgfqpoint{4.982111in}{3.231491in}}%
\pgfpathlineto{\pgfqpoint{4.982604in}{3.249660in}}%
\pgfpathlineto{\pgfqpoint{4.985070in}{3.249660in}}%
\pgfpathlineto{\pgfqpoint{4.986057in}{3.204431in}}%
\pgfpathlineto{\pgfqpoint{4.987043in}{3.357713in}}%
\pgfpathlineto{\pgfqpoint{4.990003in}{2.419220in}}%
\pgfpathlineto{\pgfqpoint{4.990496in}{3.377420in}}%
\pgfpathlineto{\pgfqpoint{4.991482in}{3.213695in}}%
\pgfpathlineto{\pgfqpoint{4.991976in}{3.213695in}}%
\pgfpathlineto{\pgfqpoint{4.993455in}{3.344715in}}%
\pgfpathlineto{\pgfqpoint{4.995428in}{3.344715in}}%
\pgfpathlineto{\pgfqpoint{4.996908in}{3.367539in}}%
\pgfpathlineto{\pgfqpoint{4.998881in}{3.367539in}}%
\pgfpathlineto{\pgfqpoint{5.000361in}{2.870461in}}%
\pgfpathlineto{\pgfqpoint{5.001840in}{3.340523in}}%
\pgfpathlineto{\pgfqpoint{5.003320in}{3.340523in}}%
\pgfpathlineto{\pgfqpoint{5.004800in}{3.055595in}}%
\pgfpathlineto{\pgfqpoint{5.005293in}{3.055595in}}%
\pgfpathlineto{\pgfqpoint{5.006773in}{3.364617in}}%
\pgfpathlineto{\pgfqpoint{5.007266in}{3.350213in}}%
\pgfpathlineto{\pgfqpoint{5.008252in}{3.369368in}}%
\pgfpathlineto{\pgfqpoint{5.010225in}{3.309271in}}%
\pgfpathlineto{\pgfqpoint{5.011212in}{3.002233in}}%
\pgfpathlineto{\pgfqpoint{5.012691in}{3.352181in}}%
\pgfpathlineto{\pgfqpoint{5.013678in}{2.482165in}}%
\pgfpathlineto{\pgfqpoint{5.014664in}{3.337762in}}%
\pgfpathlineto{\pgfqpoint{5.015158in}{3.335685in}}%
\pgfpathlineto{\pgfqpoint{5.015651in}{3.335685in}}%
\pgfpathlineto{\pgfqpoint{5.016144in}{3.124650in}}%
\pgfpathlineto{\pgfqpoint{5.016637in}{3.267790in}}%
\pgfpathlineto{\pgfqpoint{5.018117in}{3.359568in}}%
\pgfpathlineto{\pgfqpoint{5.021076in}{3.359568in}}%
\pgfpathlineto{\pgfqpoint{5.021570in}{3.002022in}}%
\pgfpathlineto{\pgfqpoint{5.022063in}{3.275706in}}%
\pgfpathlineto{\pgfqpoint{5.023049in}{3.379871in}}%
\pgfpathlineto{\pgfqpoint{5.023542in}{3.088970in}}%
\pgfpathlineto{\pgfqpoint{5.024036in}{3.163716in}}%
\pgfpathlineto{\pgfqpoint{5.024529in}{3.163716in}}%
\pgfpathlineto{\pgfqpoint{5.026009in}{3.322339in}}%
\pgfpathlineto{\pgfqpoint{5.027982in}{3.322339in}}%
\pgfpathlineto{\pgfqpoint{5.028475in}{3.372219in}}%
\pgfpathlineto{\pgfqpoint{5.028968in}{3.348731in}}%
\pgfpathlineto{\pgfqpoint{5.031927in}{3.348731in}}%
\pgfpathlineto{\pgfqpoint{5.033407in}{3.253852in}}%
\pgfpathlineto{\pgfqpoint{5.034394in}{3.253852in}}%
\pgfpathlineto{\pgfqpoint{5.034887in}{3.219853in}}%
\pgfpathlineto{\pgfqpoint{5.035873in}{2.800302in}}%
\pgfpathlineto{\pgfqpoint{5.036860in}{3.373041in}}%
\pgfpathlineto{\pgfqpoint{5.037353in}{2.931745in}}%
\pgfpathlineto{\pgfqpoint{5.037846in}{3.299182in}}%
\pgfpathlineto{\pgfqpoint{5.038339in}{3.299182in}}%
\pgfpathlineto{\pgfqpoint{5.040312in}{3.096927in}}%
\pgfpathlineto{\pgfqpoint{5.041792in}{3.378191in}}%
\pgfpathlineto{\pgfqpoint{5.043765in}{3.378191in}}%
\pgfpathlineto{\pgfqpoint{5.044751in}{3.379633in}}%
\pgfpathlineto{\pgfqpoint{5.045245in}{3.379046in}}%
\pgfpathlineto{\pgfqpoint{5.046724in}{2.795865in}}%
\pgfpathlineto{\pgfqpoint{5.047218in}{3.289341in}}%
\pgfpathlineto{\pgfqpoint{5.048204in}{3.283394in}}%
\pgfpathlineto{\pgfqpoint{5.050177in}{3.283394in}}%
\pgfpathlineto{\pgfqpoint{5.052150in}{2.788227in}}%
\pgfpathlineto{\pgfqpoint{5.053136in}{3.237051in}}%
\pgfpathlineto{\pgfqpoint{5.055109in}{2.884513in}}%
\pgfpathlineto{\pgfqpoint{5.056589in}{2.884513in}}%
\pgfpathlineto{\pgfqpoint{5.057575in}{2.949496in}}%
\pgfpathlineto{\pgfqpoint{5.058562in}{3.360282in}}%
\pgfpathlineto{\pgfqpoint{5.060042in}{3.212733in}}%
\pgfpathlineto{\pgfqpoint{5.061028in}{3.160476in}}%
\pgfpathlineto{\pgfqpoint{5.062508in}{2.892594in}}%
\pgfpathlineto{\pgfqpoint{5.064481in}{3.379465in}}%
\pgfpathlineto{\pgfqpoint{5.065467in}{3.326554in}}%
\pgfpathlineto{\pgfqpoint{5.065960in}{3.265903in}}%
\pgfpathlineto{\pgfqpoint{5.066454in}{3.379844in}}%
\pgfpathlineto{\pgfqpoint{5.066947in}{3.376661in}}%
\pgfpathlineto{\pgfqpoint{5.068427in}{3.021019in}}%
\pgfpathlineto{\pgfqpoint{5.069906in}{2.780101in}}%
\pgfpathlineto{\pgfqpoint{5.071879in}{3.231014in}}%
\pgfpathlineto{\pgfqpoint{5.073852in}{3.231014in}}%
\pgfpathlineto{\pgfqpoint{5.075332in}{2.947872in}}%
\pgfpathlineto{\pgfqpoint{5.075825in}{3.342136in}}%
\pgfpathlineto{\pgfqpoint{5.076318in}{3.312285in}}%
\pgfpathlineto{\pgfqpoint{5.077305in}{2.405705in}}%
\pgfpathlineto{\pgfqpoint{5.077798in}{2.410396in}}%
\pgfpathlineto{\pgfqpoint{5.078291in}{2.410396in}}%
\pgfpathlineto{\pgfqpoint{5.079771in}{3.328468in}}%
\pgfpathlineto{\pgfqpoint{5.080757in}{3.328468in}}%
\pgfpathlineto{\pgfqpoint{5.082730in}{3.371008in}}%
\pgfpathlineto{\pgfqpoint{5.083717in}{3.102142in}}%
\pgfpathlineto{\pgfqpoint{5.084703in}{3.324257in}}%
\pgfpathlineto{\pgfqpoint{5.085196in}{3.191036in}}%
\pgfpathlineto{\pgfqpoint{5.086676in}{3.380000in}}%
\pgfpathlineto{\pgfqpoint{5.087169in}{3.380000in}}%
\pgfpathlineto{\pgfqpoint{5.088649in}{3.053272in}}%
\pgfpathlineto{\pgfqpoint{5.091115in}{3.053272in}}%
\pgfpathlineto{\pgfqpoint{5.092595in}{3.373064in}}%
\pgfpathlineto{\pgfqpoint{5.095061in}{3.318318in}}%
\pgfpathlineto{\pgfqpoint{5.096541in}{3.318318in}}%
\pgfpathlineto{\pgfqpoint{5.098020in}{3.367874in}}%
\pgfpathlineto{\pgfqpoint{5.099007in}{3.367874in}}%
\pgfpathlineto{\pgfqpoint{5.099993in}{3.008991in}}%
\pgfpathlineto{\pgfqpoint{5.100487in}{3.290916in}}%
\pgfpathlineto{\pgfqpoint{5.100980in}{3.285565in}}%
\pgfpathlineto{\pgfqpoint{5.102459in}{2.015453in}}%
\pgfpathlineto{\pgfqpoint{5.102953in}{3.373070in}}%
\pgfpathlineto{\pgfqpoint{5.103939in}{3.034115in}}%
\pgfpathlineto{\pgfqpoint{5.104432in}{3.093387in}}%
\pgfpathlineto{\pgfqpoint{5.105912in}{3.364988in}}%
\pgfpathlineto{\pgfqpoint{5.106899in}{2.915331in}}%
\pgfpathlineto{\pgfqpoint{5.107392in}{3.306947in}}%
\pgfpathlineto{\pgfqpoint{5.108378in}{3.236197in}}%
\pgfpathlineto{\pgfqpoint{5.108871in}{3.236197in}}%
\pgfpathlineto{\pgfqpoint{5.110351in}{3.057792in}}%
\pgfpathlineto{\pgfqpoint{5.110844in}{3.057792in}}%
\pgfpathlineto{\pgfqpoint{5.112324in}{1.930816in}}%
\pgfpathlineto{\pgfqpoint{5.114297in}{3.329049in}}%
\pgfpathlineto{\pgfqpoint{5.114790in}{3.329049in}}%
\pgfpathlineto{\pgfqpoint{5.115777in}{2.992984in}}%
\pgfpathlineto{\pgfqpoint{5.117256in}{3.309929in}}%
\pgfpathlineto{\pgfqpoint{5.119229in}{3.309929in}}%
\pgfpathlineto{\pgfqpoint{5.120709in}{3.351151in}}%
\pgfpathlineto{\pgfqpoint{5.123175in}{3.351151in}}%
\pgfpathlineto{\pgfqpoint{5.124162in}{3.202901in}}%
\pgfpathlineto{\pgfqpoint{5.125641in}{3.379380in}}%
\pgfpathlineto{\pgfqpoint{5.129587in}{3.379995in}}%
\pgfpathlineto{\pgfqpoint{5.130574in}{3.304115in}}%
\pgfpathlineto{\pgfqpoint{5.131067in}{3.312339in}}%
\pgfpathlineto{\pgfqpoint{5.132053in}{3.312339in}}%
\pgfpathlineto{\pgfqpoint{5.133533in}{3.341480in}}%
\pgfpathlineto{\pgfqpoint{5.134026in}{3.341480in}}%
\pgfpathlineto{\pgfqpoint{5.135013in}{1.640105in}}%
\pgfpathlineto{\pgfqpoint{5.135999in}{3.281762in}}%
\pgfpathlineto{\pgfqpoint{5.136492in}{3.259647in}}%
\pgfpathlineto{\pgfqpoint{5.137479in}{3.259647in}}%
\pgfpathlineto{\pgfqpoint{5.138959in}{3.311612in}}%
\pgfpathlineto{\pgfqpoint{5.139452in}{3.311612in}}%
\pgfpathlineto{\pgfqpoint{5.140931in}{3.345223in}}%
\pgfpathlineto{\pgfqpoint{5.141918in}{3.361399in}}%
\pgfpathlineto{\pgfqpoint{5.143398in}{3.341257in}}%
\pgfpathlineto{\pgfqpoint{5.149316in}{3.341257in}}%
\pgfpathlineto{\pgfqpoint{5.150796in}{3.312321in}}%
\pgfpathlineto{\pgfqpoint{5.151289in}{3.312321in}}%
\pgfpathlineto{\pgfqpoint{5.151783in}{3.117792in}}%
\pgfpathlineto{\pgfqpoint{5.152276in}{3.320890in}}%
\pgfpathlineto{\pgfqpoint{5.152769in}{3.320890in}}%
\pgfpathlineto{\pgfqpoint{5.153755in}{3.365664in}}%
\pgfpathlineto{\pgfqpoint{5.155235in}{3.345762in}}%
\pgfpathlineto{\pgfqpoint{5.156222in}{3.345762in}}%
\pgfpathlineto{\pgfqpoint{5.156715in}{2.992441in}}%
\pgfpathlineto{\pgfqpoint{5.157208in}{3.222408in}}%
\pgfpathlineto{\pgfqpoint{5.158688in}{3.328355in}}%
\pgfpathlineto{\pgfqpoint{5.159181in}{3.328355in}}%
\pgfpathlineto{\pgfqpoint{5.160661in}{3.254680in}}%
\pgfpathlineto{\pgfqpoint{5.161154in}{3.254680in}}%
\pgfpathlineto{\pgfqpoint{5.161647in}{3.328732in}}%
\pgfpathlineto{\pgfqpoint{5.163127in}{3.109022in}}%
\pgfpathlineto{\pgfqpoint{5.163620in}{3.109022in}}%
\pgfpathlineto{\pgfqpoint{5.165100in}{3.340100in}}%
\pgfpathlineto{\pgfqpoint{5.166579in}{3.342690in}}%
\pgfpathlineto{\pgfqpoint{5.167566in}{3.342690in}}%
\pgfpathlineto{\pgfqpoint{5.168552in}{2.912967in}}%
\pgfpathlineto{\pgfqpoint{5.170032in}{3.376448in}}%
\pgfpathlineto{\pgfqpoint{5.171019in}{3.134296in}}%
\pgfpathlineto{\pgfqpoint{5.172498in}{3.379999in}}%
\pgfpathlineto{\pgfqpoint{5.172992in}{3.379999in}}%
\pgfpathlineto{\pgfqpoint{5.176444in}{2.954452in}}%
\pgfpathlineto{\pgfqpoint{5.176937in}{3.283622in}}%
\pgfpathlineto{\pgfqpoint{5.177924in}{3.262340in}}%
\pgfpathlineto{\pgfqpoint{5.178910in}{3.262340in}}%
\pgfpathlineto{\pgfqpoint{5.180390in}{3.376474in}}%
\pgfpathlineto{\pgfqpoint{5.181376in}{2.870751in}}%
\pgfpathlineto{\pgfqpoint{5.182856in}{3.260895in}}%
\pgfpathlineto{\pgfqpoint{5.183349in}{3.205169in}}%
\pgfpathlineto{\pgfqpoint{5.183843in}{2.354280in}}%
\pgfpathlineto{\pgfqpoint{5.184336in}{2.765791in}}%
\pgfpathlineto{\pgfqpoint{5.184829in}{2.765791in}}%
\pgfpathlineto{\pgfqpoint{5.185816in}{2.860722in}}%
\pgfpathlineto{\pgfqpoint{5.187295in}{3.347654in}}%
\pgfpathlineto{\pgfqpoint{5.188775in}{3.368270in}}%
\pgfpathlineto{\pgfqpoint{5.190255in}{3.368270in}}%
\pgfpathlineto{\pgfqpoint{5.191734in}{3.372454in}}%
\pgfpathlineto{\pgfqpoint{5.194694in}{3.372454in}}%
\pgfpathlineto{\pgfqpoint{5.195187in}{2.899569in}}%
\pgfpathlineto{\pgfqpoint{5.196173in}{2.986286in}}%
\pgfpathlineto{\pgfqpoint{5.197653in}{2.986286in}}%
\pgfpathlineto{\pgfqpoint{5.199133in}{2.613551in}}%
\pgfpathlineto{\pgfqpoint{5.200119in}{2.569518in}}%
\pgfpathlineto{\pgfqpoint{5.200612in}{3.309502in}}%
\pgfpathlineto{\pgfqpoint{5.201599in}{3.300956in}}%
\pgfpathlineto{\pgfqpoint{5.202092in}{3.378533in}}%
\pgfpathlineto{\pgfqpoint{5.202585in}{3.172064in}}%
\pgfpathlineto{\pgfqpoint{5.203079in}{3.360633in}}%
\pgfpathlineto{\pgfqpoint{5.205052in}{3.360633in}}%
\pgfpathlineto{\pgfqpoint{5.206038in}{3.002429in}}%
\pgfpathlineto{\pgfqpoint{5.207518in}{3.302671in}}%
\pgfpathlineto{\pgfqpoint{5.208504in}{3.064835in}}%
\pgfpathlineto{\pgfqpoint{5.208997in}{3.378803in}}%
\pgfpathlineto{\pgfqpoint{5.209984in}{3.378309in}}%
\pgfpathlineto{\pgfqpoint{5.213930in}{3.378309in}}%
\pgfpathlineto{\pgfqpoint{5.215409in}{3.358640in}}%
\pgfpathlineto{\pgfqpoint{5.216396in}{3.362097in}}%
\pgfpathlineto{\pgfqpoint{5.217382in}{2.882926in}}%
\pgfpathlineto{\pgfqpoint{5.218862in}{3.198157in}}%
\pgfpathlineto{\pgfqpoint{5.220835in}{3.198157in}}%
\pgfpathlineto{\pgfqpoint{5.222315in}{3.276795in}}%
\pgfpathlineto{\pgfqpoint{5.223794in}{3.235584in}}%
\pgfpathlineto{\pgfqpoint{5.224781in}{3.235584in}}%
\pgfpathlineto{\pgfqpoint{5.225767in}{3.203198in}}%
\pgfpathlineto{\pgfqpoint{5.227247in}{3.004251in}}%
\pgfpathlineto{\pgfqpoint{5.227740in}{3.004251in}}%
\pgfpathlineto{\pgfqpoint{5.230206in}{3.376855in}}%
\pgfpathlineto{\pgfqpoint{5.230700in}{3.376855in}}%
\pgfpathlineto{\pgfqpoint{5.231193in}{3.078488in}}%
\pgfpathlineto{\pgfqpoint{5.231686in}{3.254403in}}%
\pgfpathlineto{\pgfqpoint{5.232179in}{3.200367in}}%
\pgfpathlineto{\pgfqpoint{5.233659in}{3.361273in}}%
\pgfpathlineto{\pgfqpoint{5.235139in}{3.361273in}}%
\pgfpathlineto{\pgfqpoint{5.237112in}{3.035756in}}%
\pgfpathlineto{\pgfqpoint{5.238591in}{3.035756in}}%
\pgfpathlineto{\pgfqpoint{5.241057in}{3.348617in}}%
\pgfpathlineto{\pgfqpoint{5.241551in}{3.348617in}}%
\pgfpathlineto{\pgfqpoint{5.243030in}{3.316787in}}%
\pgfpathlineto{\pgfqpoint{5.244510in}{3.317290in}}%
\pgfpathlineto{\pgfqpoint{5.245003in}{3.194963in}}%
\pgfpathlineto{\pgfqpoint{5.246483in}{3.379829in}}%
\pgfpathlineto{\pgfqpoint{5.247963in}{3.379829in}}%
\pgfpathlineto{\pgfqpoint{5.249442in}{3.133506in}}%
\pgfpathlineto{\pgfqpoint{5.249936in}{3.133506in}}%
\pgfpathlineto{\pgfqpoint{5.251415in}{3.322387in}}%
\pgfpathlineto{\pgfqpoint{5.251908in}{3.291293in}}%
\pgfpathlineto{\pgfqpoint{5.253388in}{3.124134in}}%
\pgfpathlineto{\pgfqpoint{5.254868in}{3.379690in}}%
\pgfpathlineto{\pgfqpoint{5.258814in}{3.379690in}}%
\pgfpathlineto{\pgfqpoint{5.259800in}{2.252917in}}%
\pgfpathlineto{\pgfqpoint{5.260293in}{2.360616in}}%
\pgfpathlineto{\pgfqpoint{5.261773in}{2.970715in}}%
\pgfpathlineto{\pgfqpoint{5.263253in}{3.211806in}}%
\pgfpathlineto{\pgfqpoint{5.264732in}{3.253882in}}%
\pgfpathlineto{\pgfqpoint{5.265226in}{3.253882in}}%
\pgfpathlineto{\pgfqpoint{5.266705in}{3.372934in}}%
\pgfpathlineto{\pgfqpoint{5.270158in}{3.372934in}}%
\pgfpathlineto{\pgfqpoint{5.270651in}{3.364054in}}%
\pgfpathlineto{\pgfqpoint{5.272131in}{3.327407in}}%
\pgfpathlineto{\pgfqpoint{5.274597in}{3.327407in}}%
\pgfpathlineto{\pgfqpoint{5.275584in}{3.377994in}}%
\pgfpathlineto{\pgfqpoint{5.276077in}{3.374828in}}%
\pgfpathlineto{\pgfqpoint{5.279529in}{3.374828in}}%
\pgfpathlineto{\pgfqpoint{5.281009in}{3.379994in}}%
\pgfpathlineto{\pgfqpoint{5.281502in}{3.379994in}}%
\pgfpathlineto{\pgfqpoint{5.282489in}{3.045529in}}%
\pgfpathlineto{\pgfqpoint{5.283969in}{3.370037in}}%
\pgfpathlineto{\pgfqpoint{5.284955in}{3.370037in}}%
\pgfpathlineto{\pgfqpoint{5.286435in}{3.035293in}}%
\pgfpathlineto{\pgfqpoint{5.287914in}{3.035293in}}%
\pgfpathlineto{\pgfqpoint{5.288901in}{3.018243in}}%
\pgfpathlineto{\pgfqpoint{5.290381in}{3.314379in}}%
\pgfpathlineto{\pgfqpoint{5.290874in}{3.314379in}}%
\pgfpathlineto{\pgfqpoint{5.292353in}{2.292271in}}%
\pgfpathlineto{\pgfqpoint{5.293833in}{3.326926in}}%
\pgfpathlineto{\pgfqpoint{5.294820in}{3.326926in}}%
\pgfpathlineto{\pgfqpoint{5.296299in}{3.333076in}}%
\pgfpathlineto{\pgfqpoint{5.306164in}{3.333076in}}%
\pgfpathlineto{\pgfqpoint{5.307644in}{3.337261in}}%
\pgfpathlineto{\pgfqpoint{5.308137in}{3.337261in}}%
\pgfpathlineto{\pgfqpoint{5.309617in}{3.377226in}}%
\pgfpathlineto{\pgfqpoint{5.310110in}{2.746712in}}%
\pgfpathlineto{\pgfqpoint{5.310603in}{3.361373in}}%
\pgfpathlineto{\pgfqpoint{5.313069in}{2.874907in}}%
\pgfpathlineto{\pgfqpoint{5.313562in}{3.361186in}}%
\pgfpathlineto{\pgfqpoint{5.314056in}{2.539644in}}%
\pgfpathlineto{\pgfqpoint{5.314549in}{3.209253in}}%
\pgfpathlineto{\pgfqpoint{5.315042in}{2.908573in}}%
\pgfpathlineto{\pgfqpoint{5.316029in}{3.372003in}}%
\pgfpathlineto{\pgfqpoint{5.317015in}{3.192128in}}%
\pgfpathlineto{\pgfqpoint{5.317508in}{3.379501in}}%
\pgfpathlineto{\pgfqpoint{5.318001in}{2.237130in}}%
\pgfpathlineto{\pgfqpoint{5.318495in}{3.368819in}}%
\pgfpathlineto{\pgfqpoint{5.318988in}{3.343522in}}%
\pgfpathlineto{\pgfqpoint{5.320468in}{3.029306in}}%
\pgfpathlineto{\pgfqpoint{5.321947in}{3.126969in}}%
\pgfpathlineto{\pgfqpoint{5.322441in}{3.126969in}}%
\pgfpathlineto{\pgfqpoint{5.322934in}{3.378264in}}%
\pgfpathlineto{\pgfqpoint{5.323427in}{3.074551in}}%
\pgfpathlineto{\pgfqpoint{5.323920in}{3.086078in}}%
\pgfpathlineto{\pgfqpoint{5.325400in}{3.194062in}}%
\pgfpathlineto{\pgfqpoint{5.326880in}{3.194062in}}%
\pgfpathlineto{\pgfqpoint{5.328359in}{3.317568in}}%
\pgfpathlineto{\pgfqpoint{5.329839in}{3.317568in}}%
\pgfpathlineto{\pgfqpoint{5.330825in}{3.037765in}}%
\pgfpathlineto{\pgfqpoint{5.331319in}{3.063135in}}%
\pgfpathlineto{\pgfqpoint{5.332305in}{3.063135in}}%
\pgfpathlineto{\pgfqpoint{5.334771in}{3.376681in}}%
\pgfpathlineto{\pgfqpoint{5.335265in}{3.376681in}}%
\pgfpathlineto{\pgfqpoint{5.335758in}{3.348527in}}%
\pgfpathlineto{\pgfqpoint{5.336744in}{3.216000in}}%
\pgfpathlineto{\pgfqpoint{5.337237in}{3.248732in}}%
\pgfpathlineto{\pgfqpoint{5.338717in}{3.354703in}}%
\pgfpathlineto{\pgfqpoint{5.339704in}{3.354703in}}%
\pgfpathlineto{\pgfqpoint{5.340690in}{3.351031in}}%
\pgfpathlineto{\pgfqpoint{5.341183in}{3.377367in}}%
\pgfpathlineto{\pgfqpoint{5.341677in}{2.592267in}}%
\pgfpathlineto{\pgfqpoint{5.342170in}{3.379731in}}%
\pgfpathlineto{\pgfqpoint{5.343156in}{3.379731in}}%
\pgfpathlineto{\pgfqpoint{5.343649in}{3.086138in}}%
\pgfpathlineto{\pgfqpoint{5.344143in}{3.143706in}}%
\pgfpathlineto{\pgfqpoint{5.344636in}{3.358996in}}%
\pgfpathlineto{\pgfqpoint{5.345129in}{3.296936in}}%
\pgfpathlineto{\pgfqpoint{5.346116in}{3.296936in}}%
\pgfpathlineto{\pgfqpoint{5.346609in}{3.220486in}}%
\pgfpathlineto{\pgfqpoint{5.347102in}{3.313961in}}%
\pgfpathlineto{\pgfqpoint{5.347595in}{3.313961in}}%
\pgfpathlineto{\pgfqpoint{5.348582in}{3.245183in}}%
\pgfpathlineto{\pgfqpoint{5.350061in}{3.347594in}}%
\pgfpathlineto{\pgfqpoint{5.351541in}{3.341162in}}%
\pgfpathlineto{\pgfqpoint{5.352034in}{3.341162in}}%
\pgfpathlineto{\pgfqpoint{5.353514in}{3.359578in}}%
\pgfpathlineto{\pgfqpoint{5.354501in}{3.329186in}}%
\pgfpathlineto{\pgfqpoint{5.355980in}{3.359387in}}%
\pgfpathlineto{\pgfqpoint{5.357953in}{3.359387in}}%
\pgfpathlineto{\pgfqpoint{5.359433in}{3.376089in}}%
\pgfpathlineto{\pgfqpoint{5.359926in}{3.376089in}}%
\pgfpathlineto{\pgfqpoint{5.360419in}{3.285757in}}%
\pgfpathlineto{\pgfqpoint{5.361406in}{3.288429in}}%
\pgfpathlineto{\pgfqpoint{5.362392in}{3.288429in}}%
\pgfpathlineto{\pgfqpoint{5.364365in}{2.976293in}}%
\pgfpathlineto{\pgfqpoint{5.365845in}{3.379971in}}%
\pgfpathlineto{\pgfqpoint{5.367325in}{3.362233in}}%
\pgfpathlineto{\pgfqpoint{5.368311in}{2.240452in}}%
\pgfpathlineto{\pgfqpoint{5.369791in}{3.377479in}}%
\pgfpathlineto{\pgfqpoint{5.370284in}{3.377479in}}%
\pgfpathlineto{\pgfqpoint{5.371764in}{3.334915in}}%
\pgfpathlineto{\pgfqpoint{5.373243in}{3.377763in}}%
\pgfpathlineto{\pgfqpoint{5.375710in}{3.377763in}}%
\pgfpathlineto{\pgfqpoint{5.377189in}{3.379833in}}%
\pgfpathlineto{\pgfqpoint{5.377682in}{3.379833in}}%
\pgfpathlineto{\pgfqpoint{5.379162in}{3.350858in}}%
\pgfpathlineto{\pgfqpoint{5.380149in}{3.350858in}}%
\pgfpathlineto{\pgfqpoint{5.381135in}{3.375386in}}%
\pgfpathlineto{\pgfqpoint{5.382615in}{2.874461in}}%
\pgfpathlineto{\pgfqpoint{5.384588in}{2.874461in}}%
\pgfpathlineto{\pgfqpoint{5.385081in}{3.228321in}}%
\pgfpathlineto{\pgfqpoint{5.386067in}{3.159786in}}%
\pgfpathlineto{\pgfqpoint{5.387547in}{3.331171in}}%
\pgfpathlineto{\pgfqpoint{5.388040in}{2.711687in}}%
\pgfpathlineto{\pgfqpoint{5.388534in}{3.346642in}}%
\pgfpathlineto{\pgfqpoint{5.390013in}{3.376507in}}%
\pgfpathlineto{\pgfqpoint{5.392479in}{3.376507in}}%
\pgfpathlineto{\pgfqpoint{5.392973in}{3.360052in}}%
\pgfpathlineto{\pgfqpoint{5.393466in}{3.367245in}}%
\pgfpathlineto{\pgfqpoint{5.394946in}{3.121657in}}%
\pgfpathlineto{\pgfqpoint{5.396425in}{3.368154in}}%
\pgfpathlineto{\pgfqpoint{5.397905in}{3.356724in}}%
\pgfpathlineto{\pgfqpoint{5.398398in}{2.916420in}}%
\pgfpathlineto{\pgfqpoint{5.398891in}{3.371414in}}%
\pgfpathlineto{\pgfqpoint{5.406783in}{3.371414in}}%
\pgfpathlineto{\pgfqpoint{5.407770in}{3.321444in}}%
\pgfpathlineto{\pgfqpoint{5.408756in}{3.374266in}}%
\pgfpathlineto{\pgfqpoint{5.410236in}{3.288310in}}%
\pgfpathlineto{\pgfqpoint{5.411715in}{3.288310in}}%
\pgfpathlineto{\pgfqpoint{5.413195in}{3.340066in}}%
\pgfpathlineto{\pgfqpoint{5.414182in}{3.340066in}}%
\pgfpathlineto{\pgfqpoint{5.415661in}{3.369563in}}%
\pgfpathlineto{\pgfqpoint{5.418621in}{3.369563in}}%
\pgfpathlineto{\pgfqpoint{5.419114in}{3.220896in}}%
\pgfpathlineto{\pgfqpoint{5.419607in}{3.376101in}}%
\pgfpathlineto{\pgfqpoint{5.421087in}{3.376101in}}%
\pgfpathlineto{\pgfqpoint{5.421580in}{3.379520in}}%
\pgfpathlineto{\pgfqpoint{5.423060in}{3.356635in}}%
\pgfpathlineto{\pgfqpoint{5.423553in}{3.356635in}}%
\pgfpathlineto{\pgfqpoint{5.424046in}{3.372215in}}%
\pgfpathlineto{\pgfqpoint{5.426019in}{3.238924in}}%
\pgfpathlineto{\pgfqpoint{5.426512in}{3.279655in}}%
\pgfpathlineto{\pgfqpoint{5.427499in}{3.274127in}}%
\pgfpathlineto{\pgfqpoint{5.428978in}{3.274127in}}%
\pgfpathlineto{\pgfqpoint{5.430458in}{3.379776in}}%
\pgfpathlineto{\pgfqpoint{5.431938in}{2.956124in}}%
\pgfpathlineto{\pgfqpoint{5.432431in}{3.378790in}}%
\pgfpathlineto{\pgfqpoint{5.432924in}{2.980866in}}%
\pgfpathlineto{\pgfqpoint{5.433418in}{2.980866in}}%
\pgfpathlineto{\pgfqpoint{5.434897in}{3.374748in}}%
\pgfpathlineto{\pgfqpoint{5.436870in}{3.374748in}}%
\pgfpathlineto{\pgfqpoint{5.438350in}{3.216812in}}%
\pgfpathlineto{\pgfqpoint{5.438843in}{3.346347in}}%
\pgfpathlineto{\pgfqpoint{5.439336in}{3.230018in}}%
\pgfpathlineto{\pgfqpoint{5.440323in}{3.230018in}}%
\pgfpathlineto{\pgfqpoint{5.440816in}{2.960616in}}%
\pgfpathlineto{\pgfqpoint{5.441309in}{3.372779in}}%
\pgfpathlineto{\pgfqpoint{5.441802in}{3.267140in}}%
\pgfpathlineto{\pgfqpoint{5.442296in}{3.267140in}}%
\pgfpathlineto{\pgfqpoint{5.442789in}{3.011775in}}%
\pgfpathlineto{\pgfqpoint{5.444269in}{3.352052in}}%
\pgfpathlineto{\pgfqpoint{5.444762in}{1.402418in}}%
\pgfpathlineto{\pgfqpoint{5.445255in}{3.313347in}}%
\pgfpathlineto{\pgfqpoint{5.445748in}{3.313347in}}%
\pgfpathlineto{\pgfqpoint{5.447228in}{3.296488in}}%
\pgfpathlineto{\pgfqpoint{5.447721in}{3.296488in}}%
\pgfpathlineto{\pgfqpoint{5.449201in}{3.375861in}}%
\pgfpathlineto{\pgfqpoint{5.451174in}{3.375861in}}%
\pgfpathlineto{\pgfqpoint{5.451667in}{3.217655in}}%
\pgfpathlineto{\pgfqpoint{5.452654in}{3.225573in}}%
\pgfpathlineto{\pgfqpoint{5.454133in}{3.225573in}}%
\pgfpathlineto{\pgfqpoint{5.455120in}{3.378099in}}%
\pgfpathlineto{\pgfqpoint{5.455613in}{3.377666in}}%
\pgfpathlineto{\pgfqpoint{5.456106in}{3.367005in}}%
\pgfpathlineto{\pgfqpoint{5.458572in}{3.155204in}}%
\pgfpathlineto{\pgfqpoint{5.459559in}{3.155204in}}%
\pgfpathlineto{\pgfqpoint{5.461038in}{3.367359in}}%
\pgfpathlineto{\pgfqpoint{5.462025in}{2.835873in}}%
\pgfpathlineto{\pgfqpoint{5.462518in}{3.065580in}}%
\pgfpathlineto{\pgfqpoint{5.463998in}{3.234416in}}%
\pgfpathlineto{\pgfqpoint{5.464491in}{3.052293in}}%
\pgfpathlineto{\pgfqpoint{5.465971in}{3.369948in}}%
\pgfpathlineto{\pgfqpoint{5.466464in}{2.521640in}}%
\pgfpathlineto{\pgfqpoint{5.466957in}{3.087271in}}%
\pgfpathlineto{\pgfqpoint{5.468437in}{3.338350in}}%
\pgfpathlineto{\pgfqpoint{5.468930in}{3.378347in}}%
\pgfpathlineto{\pgfqpoint{5.470410in}{3.067605in}}%
\pgfpathlineto{\pgfqpoint{5.471396in}{3.364941in}}%
\pgfpathlineto{\pgfqpoint{5.471890in}{3.358114in}}%
\pgfpathlineto{\pgfqpoint{5.473369in}{3.221290in}}%
\pgfpathlineto{\pgfqpoint{5.475342in}{3.221290in}}%
\pgfpathlineto{\pgfqpoint{5.476822in}{3.378134in}}%
\pgfpathlineto{\pgfqpoint{5.477315in}{3.378134in}}%
\pgfpathlineto{\pgfqpoint{5.477808in}{3.000366in}}%
\pgfpathlineto{\pgfqpoint{5.478302in}{3.379351in}}%
\pgfpathlineto{\pgfqpoint{5.478795in}{3.379351in}}%
\pgfpathlineto{\pgfqpoint{5.480275in}{3.374386in}}%
\pgfpathlineto{\pgfqpoint{5.481754in}{3.193549in}}%
\pgfpathlineto{\pgfqpoint{5.482741in}{3.193549in}}%
\pgfpathlineto{\pgfqpoint{5.483234in}{3.379862in}}%
\pgfpathlineto{\pgfqpoint{5.484220in}{3.375772in}}%
\pgfpathlineto{\pgfqpoint{5.484714in}{3.375772in}}%
\pgfpathlineto{\pgfqpoint{5.486193in}{3.210728in}}%
\pgfpathlineto{\pgfqpoint{5.487180in}{3.293597in}}%
\pgfpathlineto{\pgfqpoint{5.487673in}{3.142725in}}%
\pgfpathlineto{\pgfqpoint{5.488166in}{3.303234in}}%
\pgfpathlineto{\pgfqpoint{5.488659in}{3.303234in}}%
\pgfpathlineto{\pgfqpoint{5.489646in}{3.346195in}}%
\pgfpathlineto{\pgfqpoint{5.491126in}{3.274643in}}%
\pgfpathlineto{\pgfqpoint{5.492112in}{3.274643in}}%
\pgfpathlineto{\pgfqpoint{5.492605in}{3.354877in}}%
\pgfpathlineto{\pgfqpoint{5.493099in}{3.288130in}}%
\pgfpathlineto{\pgfqpoint{5.494578in}{3.333271in}}%
\pgfpathlineto{\pgfqpoint{5.495071in}{2.859942in}}%
\pgfpathlineto{\pgfqpoint{5.495565in}{3.031790in}}%
\pgfpathlineto{\pgfqpoint{5.496551in}{3.031790in}}%
\pgfpathlineto{\pgfqpoint{5.498031in}{3.204825in}}%
\pgfpathlineto{\pgfqpoint{5.498524in}{3.204825in}}%
\pgfpathlineto{\pgfqpoint{5.500004in}{3.369976in}}%
\pgfpathlineto{\pgfqpoint{5.500497in}{3.369976in}}%
\pgfpathlineto{\pgfqpoint{5.500990in}{3.312316in}}%
\pgfpathlineto{\pgfqpoint{5.501483in}{3.366768in}}%
\pgfpathlineto{\pgfqpoint{5.502470in}{2.978976in}}%
\pgfpathlineto{\pgfqpoint{5.503456in}{3.379986in}}%
\pgfpathlineto{\pgfqpoint{5.503950in}{3.379446in}}%
\pgfpathlineto{\pgfqpoint{5.504443in}{3.379446in}}%
\pgfpathlineto{\pgfqpoint{5.505923in}{3.058885in}}%
\pgfpathlineto{\pgfqpoint{5.509375in}{3.058885in}}%
\pgfpathlineto{\pgfqpoint{5.510855in}{3.287299in}}%
\pgfpathlineto{\pgfqpoint{5.513814in}{3.287299in}}%
\pgfpathlineto{\pgfqpoint{5.514307in}{3.378606in}}%
\pgfpathlineto{\pgfqpoint{5.514801in}{3.358648in}}%
\pgfpathlineto{\pgfqpoint{5.515294in}{2.975332in}}%
\pgfpathlineto{\pgfqpoint{5.515787in}{3.278893in}}%
\pgfpathlineto{\pgfqpoint{5.516280in}{3.278893in}}%
\pgfpathlineto{\pgfqpoint{5.516774in}{3.167270in}}%
\pgfpathlineto{\pgfqpoint{5.517267in}{3.270586in}}%
\pgfpathlineto{\pgfqpoint{5.518747in}{3.270586in}}%
\pgfpathlineto{\pgfqpoint{5.520226in}{2.328695in}}%
\pgfpathlineto{\pgfqpoint{5.520719in}{2.328695in}}%
\pgfpathlineto{\pgfqpoint{5.522199in}{2.817389in}}%
\pgfpathlineto{\pgfqpoint{5.523186in}{2.817389in}}%
\pgfpathlineto{\pgfqpoint{5.523679in}{2.372993in}}%
\pgfpathlineto{\pgfqpoint{5.525159in}{3.287578in}}%
\pgfpathlineto{\pgfqpoint{5.525652in}{3.287578in}}%
\pgfpathlineto{\pgfqpoint{5.527131in}{2.854244in}}%
\pgfpathlineto{\pgfqpoint{5.527625in}{2.854244in}}%
\pgfpathlineto{\pgfqpoint{5.529104in}{3.305806in}}%
\pgfpathlineto{\pgfqpoint{5.529598in}{3.197707in}}%
\pgfpathlineto{\pgfqpoint{5.530091in}{3.296760in}}%
\pgfpathlineto{\pgfqpoint{5.531077in}{3.296760in}}%
\pgfpathlineto{\pgfqpoint{5.532557in}{3.329111in}}%
\pgfpathlineto{\pgfqpoint{5.536503in}{3.329111in}}%
\pgfpathlineto{\pgfqpoint{5.537983in}{2.684524in}}%
\pgfpathlineto{\pgfqpoint{5.538969in}{3.305876in}}%
\pgfpathlineto{\pgfqpoint{5.539462in}{3.247906in}}%
\pgfpathlineto{\pgfqpoint{5.539955in}{3.247906in}}%
\pgfpathlineto{\pgfqpoint{5.541435in}{2.754022in}}%
\pgfpathlineto{\pgfqpoint{5.542422in}{3.237557in}}%
\pgfpathlineto{\pgfqpoint{5.543901in}{3.098457in}}%
\pgfpathlineto{\pgfqpoint{5.546367in}{3.098457in}}%
\pgfpathlineto{\pgfqpoint{5.547847in}{3.374839in}}%
\pgfpathlineto{\pgfqpoint{5.548340in}{3.374839in}}%
\pgfpathlineto{\pgfqpoint{5.548834in}{3.175852in}}%
\pgfpathlineto{\pgfqpoint{5.549820in}{3.203444in}}%
\pgfpathlineto{\pgfqpoint{5.550313in}{3.203444in}}%
\pgfpathlineto{\pgfqpoint{5.551793in}{3.299211in}}%
\pgfpathlineto{\pgfqpoint{5.554259in}{3.299211in}}%
\pgfpathlineto{\pgfqpoint{5.555739in}{3.379970in}}%
\pgfpathlineto{\pgfqpoint{5.557219in}{3.375819in}}%
\pgfpathlineto{\pgfqpoint{5.558205in}{3.375819in}}%
\pgfpathlineto{\pgfqpoint{5.558698in}{2.753061in}}%
\pgfpathlineto{\pgfqpoint{5.559191in}{3.044467in}}%
\pgfpathlineto{\pgfqpoint{5.560671in}{2.872311in}}%
\pgfpathlineto{\pgfqpoint{5.562644in}{3.337986in}}%
\pgfpathlineto{\pgfqpoint{5.563631in}{3.337986in}}%
\pgfpathlineto{\pgfqpoint{5.564124in}{3.344322in}}%
\pgfpathlineto{\pgfqpoint{5.565110in}{3.068409in}}%
\pgfpathlineto{\pgfqpoint{5.566590in}{3.379327in}}%
\pgfpathlineto{\pgfqpoint{5.568070in}{3.379327in}}%
\pgfpathlineto{\pgfqpoint{5.569056in}{3.372400in}}%
\pgfpathlineto{\pgfqpoint{5.570043in}{2.713811in}}%
\pgfpathlineto{\pgfqpoint{5.570536in}{2.805718in}}%
\pgfpathlineto{\pgfqpoint{5.572015in}{3.367730in}}%
\pgfpathlineto{\pgfqpoint{5.573495in}{3.356807in}}%
\pgfpathlineto{\pgfqpoint{5.573988in}{3.356807in}}%
\pgfpathlineto{\pgfqpoint{5.575468in}{3.219367in}}%
\pgfpathlineto{\pgfqpoint{5.575961in}{3.219367in}}%
\pgfpathlineto{\pgfqpoint{5.576455in}{3.364112in}}%
\pgfpathlineto{\pgfqpoint{5.576948in}{2.442240in}}%
\pgfpathlineto{\pgfqpoint{5.577441in}{3.280561in}}%
\pgfpathlineto{\pgfqpoint{5.578921in}{3.379541in}}%
\pgfpathlineto{\pgfqpoint{5.582867in}{3.379541in}}%
\pgfpathlineto{\pgfqpoint{5.584346in}{2.896576in}}%
\pgfpathlineto{\pgfqpoint{5.584840in}{2.896576in}}%
\pgfpathlineto{\pgfqpoint{5.585333in}{3.375723in}}%
\pgfpathlineto{\pgfqpoint{5.586319in}{3.363969in}}%
\pgfpathlineto{\pgfqpoint{5.587799in}{3.339540in}}%
\pgfpathlineto{\pgfqpoint{5.590758in}{3.339540in}}%
\pgfpathlineto{\pgfqpoint{5.591745in}{3.159999in}}%
\pgfpathlineto{\pgfqpoint{5.593224in}{3.228956in}}%
\pgfpathlineto{\pgfqpoint{5.595197in}{3.228956in}}%
\pgfpathlineto{\pgfqpoint{5.596677in}{3.053215in}}%
\pgfpathlineto{\pgfqpoint{5.597170in}{3.053215in}}%
\pgfpathlineto{\pgfqpoint{5.598650in}{3.279276in}}%
\pgfpathlineto{\pgfqpoint{5.600130in}{3.378095in}}%
\pgfpathlineto{\pgfqpoint{5.607528in}{3.378095in}}%
\pgfpathlineto{\pgfqpoint{5.609008in}{3.374200in}}%
\pgfpathlineto{\pgfqpoint{5.609994in}{2.145563in}}%
\pgfpathlineto{\pgfqpoint{5.610981in}{3.376819in}}%
\pgfpathlineto{\pgfqpoint{5.611474in}{3.375932in}}%
\pgfpathlineto{\pgfqpoint{5.611967in}{3.375932in}}%
\pgfpathlineto{\pgfqpoint{5.613447in}{2.509644in}}%
\pgfpathlineto{\pgfqpoint{5.613940in}{2.509644in}}%
\pgfpathlineto{\pgfqpoint{5.614433in}{2.453593in}}%
\pgfpathlineto{\pgfqpoint{5.615913in}{2.653154in}}%
\pgfpathlineto{\pgfqpoint{5.616406in}{2.653154in}}%
\pgfpathlineto{\pgfqpoint{5.616900in}{3.379301in}}%
\pgfpathlineto{\pgfqpoint{5.617886in}{3.342128in}}%
\pgfpathlineto{\pgfqpoint{5.619366in}{3.179805in}}%
\pgfpathlineto{\pgfqpoint{5.619859in}{3.179805in}}%
\pgfpathlineto{\pgfqpoint{5.621339in}{3.378822in}}%
\pgfpathlineto{\pgfqpoint{5.621832in}{3.379725in}}%
\pgfpathlineto{\pgfqpoint{5.623312in}{3.366479in}}%
\pgfpathlineto{\pgfqpoint{5.625284in}{3.366479in}}%
\pgfpathlineto{\pgfqpoint{5.626764in}{3.378869in}}%
\pgfpathlineto{\pgfqpoint{5.627257in}{3.378869in}}%
\pgfpathlineto{\pgfqpoint{5.627751in}{1.586152in}}%
\pgfpathlineto{\pgfqpoint{5.628244in}{2.984177in}}%
\pgfpathlineto{\pgfqpoint{5.628737in}{2.693383in}}%
\pgfpathlineto{\pgfqpoint{5.630217in}{3.133541in}}%
\pgfpathlineto{\pgfqpoint{5.631203in}{3.133541in}}%
\pgfpathlineto{\pgfqpoint{5.632190in}{2.871590in}}%
\pgfpathlineto{\pgfqpoint{5.633669in}{3.125239in}}%
\pgfpathlineto{\pgfqpoint{5.635149in}{3.031980in}}%
\pgfpathlineto{\pgfqpoint{5.635642in}{3.031980in}}%
\pgfpathlineto{\pgfqpoint{5.636136in}{3.053347in}}%
\pgfpathlineto{\pgfqpoint{5.637615in}{3.132627in}}%
\pgfpathlineto{\pgfqpoint{5.641068in}{3.132627in}}%
\pgfpathlineto{\pgfqpoint{5.642548in}{3.379046in}}%
\pgfpathlineto{\pgfqpoint{5.643041in}{3.379046in}}%
\pgfpathlineto{\pgfqpoint{5.644027in}{3.193220in}}%
\pgfpathlineto{\pgfqpoint{5.645507in}{3.244887in}}%
\pgfpathlineto{\pgfqpoint{5.646987in}{3.244887in}}%
\pgfpathlineto{\pgfqpoint{5.648466in}{3.337993in}}%
\pgfpathlineto{\pgfqpoint{5.649453in}{3.288250in}}%
\pgfpathlineto{\pgfqpoint{5.650932in}{3.308129in}}%
\pgfpathlineto{\pgfqpoint{5.652412in}{3.308129in}}%
\pgfpathlineto{\pgfqpoint{5.653399in}{2.644566in}}%
\pgfpathlineto{\pgfqpoint{5.654878in}{3.359274in}}%
\pgfpathlineto{\pgfqpoint{5.656851in}{3.088985in}}%
\pgfpathlineto{\pgfqpoint{5.657838in}{3.088985in}}%
\pgfpathlineto{\pgfqpoint{5.658331in}{3.379870in}}%
\pgfpathlineto{\pgfqpoint{5.659317in}{3.360594in}}%
\pgfpathlineto{\pgfqpoint{5.659811in}{3.360594in}}%
\pgfpathlineto{\pgfqpoint{5.661290in}{3.293877in}}%
\pgfpathlineto{\pgfqpoint{5.662770in}{2.914601in}}%
\pgfpathlineto{\pgfqpoint{5.664250in}{3.294758in}}%
\pgfpathlineto{\pgfqpoint{5.665236in}{3.338241in}}%
\pgfpathlineto{\pgfqpoint{5.665729in}{2.826045in}}%
\pgfpathlineto{\pgfqpoint{5.666223in}{2.972177in}}%
\pgfpathlineto{\pgfqpoint{5.668196in}{2.972177in}}%
\pgfpathlineto{\pgfqpoint{5.669182in}{3.007105in}}%
\pgfpathlineto{\pgfqpoint{5.670168in}{3.379922in}}%
\pgfpathlineto{\pgfqpoint{5.670662in}{3.358733in}}%
\pgfpathlineto{\pgfqpoint{5.671155in}{2.600986in}}%
\pgfpathlineto{\pgfqpoint{5.671648in}{2.810577in}}%
\pgfpathlineto{\pgfqpoint{5.672141in}{2.810577in}}%
\pgfpathlineto{\pgfqpoint{5.673621in}{3.379543in}}%
\pgfpathlineto{\pgfqpoint{5.674114in}{3.379543in}}%
\pgfpathlineto{\pgfqpoint{5.675594in}{3.197451in}}%
\pgfpathlineto{\pgfqpoint{5.676580in}{3.197451in}}%
\pgfpathlineto{\pgfqpoint{5.677074in}{3.058622in}}%
\pgfpathlineto{\pgfqpoint{5.678553in}{2.460747in}}%
\pgfpathlineto{\pgfqpoint{5.680033in}{2.460747in}}%
\pgfpathlineto{\pgfqpoint{5.681513in}{3.328025in}}%
\pgfpathlineto{\pgfqpoint{5.682993in}{3.017874in}}%
\pgfpathlineto{\pgfqpoint{5.684472in}{3.017874in}}%
\pgfpathlineto{\pgfqpoint{5.685952in}{3.146716in}}%
\pgfpathlineto{\pgfqpoint{5.687432in}{3.002823in}}%
\pgfpathlineto{\pgfqpoint{5.688418in}{3.002823in}}%
\pgfpathlineto{\pgfqpoint{5.689898in}{3.377316in}}%
\pgfpathlineto{\pgfqpoint{5.692364in}{3.377316in}}%
\pgfpathlineto{\pgfqpoint{5.693844in}{2.638976in}}%
\pgfpathlineto{\pgfqpoint{5.694337in}{2.638976in}}%
\pgfpathlineto{\pgfqpoint{5.695817in}{3.226010in}}%
\pgfpathlineto{\pgfqpoint{5.697296in}{3.226010in}}%
\pgfpathlineto{\pgfqpoint{5.698776in}{2.020553in}}%
\pgfpathlineto{\pgfqpoint{5.699762in}{3.334233in}}%
\pgfpathlineto{\pgfqpoint{5.701242in}{2.666887in}}%
\pgfpathlineto{\pgfqpoint{5.702722in}{1.821649in}}%
\pgfpathlineto{\pgfqpoint{5.703215in}{3.225616in}}%
\pgfpathlineto{\pgfqpoint{5.704201in}{3.206646in}}%
\pgfpathlineto{\pgfqpoint{5.705188in}{3.206646in}}%
\pgfpathlineto{\pgfqpoint{5.706668in}{3.076546in}}%
\pgfpathlineto{\pgfqpoint{5.710120in}{3.076546in}}%
\pgfpathlineto{\pgfqpoint{5.711600in}{3.379992in}}%
\pgfpathlineto{\pgfqpoint{5.712586in}{3.368736in}}%
\pgfpathlineto{\pgfqpoint{5.713080in}{2.655676in}}%
\pgfpathlineto{\pgfqpoint{5.713573in}{2.788909in}}%
\pgfpathlineto{\pgfqpoint{5.715053in}{3.357719in}}%
\pgfpathlineto{\pgfqpoint{5.715546in}{3.357719in}}%
\pgfpathlineto{\pgfqpoint{5.717025in}{3.372670in}}%
\pgfpathlineto{\pgfqpoint{5.718012in}{3.372670in}}%
\pgfpathlineto{\pgfqpoint{5.718505in}{3.310367in}}%
\pgfpathlineto{\pgfqpoint{5.719985in}{2.948858in}}%
\pgfpathlineto{\pgfqpoint{5.720478in}{2.948858in}}%
\pgfpathlineto{\pgfqpoint{5.721958in}{2.894529in}}%
\pgfpathlineto{\pgfqpoint{5.722451in}{2.894529in}}%
\pgfpathlineto{\pgfqpoint{5.723931in}{2.328799in}}%
\pgfpathlineto{\pgfqpoint{5.724917in}{2.328799in}}%
\pgfpathlineto{\pgfqpoint{5.725904in}{3.285904in}}%
\pgfpathlineto{\pgfqpoint{5.726397in}{3.138256in}}%
\pgfpathlineto{\pgfqpoint{5.727877in}{3.009448in}}%
\pgfpathlineto{\pgfqpoint{5.729356in}{3.009448in}}%
\pgfpathlineto{\pgfqpoint{5.730343in}{3.369742in}}%
\pgfpathlineto{\pgfqpoint{5.730836in}{3.043560in}}%
\pgfpathlineto{\pgfqpoint{5.731822in}{3.073990in}}%
\pgfpathlineto{\pgfqpoint{5.732809in}{3.073990in}}%
\pgfpathlineto{\pgfqpoint{5.735275in}{2.803001in}}%
\pgfpathlineto{\pgfqpoint{5.735768in}{2.803001in}}%
\pgfpathlineto{\pgfqpoint{5.736261in}{2.750316in}}%
\pgfpathlineto{\pgfqpoint{5.737741in}{3.327390in}}%
\pgfpathlineto{\pgfqpoint{5.738234in}{3.176165in}}%
\pgfpathlineto{\pgfqpoint{5.738728in}{3.221094in}}%
\pgfpathlineto{\pgfqpoint{5.739221in}{3.221094in}}%
\pgfpathlineto{\pgfqpoint{5.740207in}{3.378277in}}%
\pgfpathlineto{\pgfqpoint{5.740701in}{3.013866in}}%
\pgfpathlineto{\pgfqpoint{5.741194in}{3.378995in}}%
\pgfpathlineto{\pgfqpoint{5.742180in}{3.379228in}}%
\pgfpathlineto{\pgfqpoint{5.743167in}{2.816759in}}%
\pgfpathlineto{\pgfqpoint{5.744153in}{3.171140in}}%
\pgfpathlineto{\pgfqpoint{5.744646in}{3.151410in}}%
\pgfpathlineto{\pgfqpoint{5.746126in}{3.372306in}}%
\pgfpathlineto{\pgfqpoint{5.747113in}{3.372306in}}%
\pgfpathlineto{\pgfqpoint{5.747606in}{3.289535in}}%
\pgfpathlineto{\pgfqpoint{5.748592in}{3.293672in}}%
\pgfpathlineto{\pgfqpoint{5.749579in}{3.293672in}}%
\pgfpathlineto{\pgfqpoint{5.751058in}{3.379955in}}%
\pgfpathlineto{\pgfqpoint{5.754018in}{3.379955in}}%
\pgfpathlineto{\pgfqpoint{5.755497in}{2.443386in}}%
\pgfpathlineto{\pgfqpoint{5.755991in}{3.282850in}}%
\pgfpathlineto{\pgfqpoint{5.756484in}{2.991185in}}%
\pgfpathlineto{\pgfqpoint{5.756977in}{2.991185in}}%
\pgfpathlineto{\pgfqpoint{5.757964in}{3.369017in}}%
\pgfpathlineto{\pgfqpoint{5.758950in}{2.919216in}}%
\pgfpathlineto{\pgfqpoint{5.759443in}{3.153702in}}%
\pgfpathlineto{\pgfqpoint{5.760923in}{3.375880in}}%
\pgfpathlineto{\pgfqpoint{5.761909in}{3.375880in}}%
\pgfpathlineto{\pgfqpoint{5.762896in}{3.114626in}}%
\pgfpathlineto{\pgfqpoint{5.763882in}{3.336439in}}%
\pgfpathlineto{\pgfqpoint{5.764869in}{2.565750in}}%
\pgfpathlineto{\pgfqpoint{5.766349in}{3.339621in}}%
\pgfpathlineto{\pgfqpoint{5.766842in}{3.269823in}}%
\pgfpathlineto{\pgfqpoint{5.767335in}{3.289101in}}%
\pgfpathlineto{\pgfqpoint{5.771281in}{3.289101in}}%
\pgfpathlineto{\pgfqpoint{5.772761in}{3.369741in}}%
\pgfpathlineto{\pgfqpoint{5.773747in}{3.369741in}}%
\pgfpathlineto{\pgfqpoint{5.774240in}{3.361054in}}%
\pgfpathlineto{\pgfqpoint{5.775720in}{3.231030in}}%
\pgfpathlineto{\pgfqpoint{5.777200in}{3.149696in}}%
\pgfpathlineto{\pgfqpoint{5.778186in}{3.345834in}}%
\pgfpathlineto{\pgfqpoint{5.779666in}{3.205223in}}%
\pgfpathlineto{\pgfqpoint{5.781146in}{3.366243in}}%
\pgfpathlineto{\pgfqpoint{5.786078in}{3.366243in}}%
\pgfpathlineto{\pgfqpoint{5.787558in}{3.376420in}}%
\pgfpathlineto{\pgfqpoint{5.788051in}{3.376420in}}%
\pgfpathlineto{\pgfqpoint{5.788544in}{3.378947in}}%
\pgfpathlineto{\pgfqpoint{5.789037in}{2.354795in}}%
\pgfpathlineto{\pgfqpoint{5.789530in}{3.370714in}}%
\pgfpathlineto{\pgfqpoint{5.790024in}{3.370714in}}%
\pgfpathlineto{\pgfqpoint{5.790517in}{3.328873in}}%
\pgfpathlineto{\pgfqpoint{5.791503in}{2.654316in}}%
\pgfpathlineto{\pgfqpoint{5.793476in}{3.380000in}}%
\pgfpathlineto{\pgfqpoint{5.794956in}{3.376722in}}%
\pgfpathlineto{\pgfqpoint{5.796436in}{3.373619in}}%
\pgfpathlineto{\pgfqpoint{5.796929in}{3.217484in}}%
\pgfpathlineto{\pgfqpoint{5.797422in}{3.357256in}}%
\pgfpathlineto{\pgfqpoint{5.799395in}{3.357256in}}%
\pgfpathlineto{\pgfqpoint{5.800875in}{3.374475in}}%
\pgfpathlineto{\pgfqpoint{5.802354in}{3.356003in}}%
\pgfpathlineto{\pgfqpoint{5.802848in}{3.356003in}}%
\pgfpathlineto{\pgfqpoint{5.804821in}{3.086116in}}%
\pgfpathlineto{\pgfqpoint{5.805314in}{3.086116in}}%
\pgfpathlineto{\pgfqpoint{5.806794in}{3.173198in}}%
\pgfpathlineto{\pgfqpoint{5.807780in}{3.173198in}}%
\pgfpathlineto{\pgfqpoint{5.808766in}{3.296282in}}%
\pgfpathlineto{\pgfqpoint{5.810739in}{3.065352in}}%
\pgfpathlineto{\pgfqpoint{5.811726in}{3.127298in}}%
\pgfpathlineto{\pgfqpoint{5.812219in}{2.324268in}}%
\pgfpathlineto{\pgfqpoint{5.812712in}{2.883765in}}%
\pgfpathlineto{\pgfqpoint{5.814192in}{3.379712in}}%
\pgfpathlineto{\pgfqpoint{5.814685in}{3.379712in}}%
\pgfpathlineto{\pgfqpoint{5.815178in}{3.332752in}}%
\pgfpathlineto{\pgfqpoint{5.816658in}{3.090597in}}%
\pgfpathlineto{\pgfqpoint{5.819124in}{3.090597in}}%
\pgfpathlineto{\pgfqpoint{5.820111in}{2.691727in}}%
\pgfpathlineto{\pgfqpoint{5.820604in}{3.357539in}}%
\pgfpathlineto{\pgfqpoint{5.821097in}{3.122356in}}%
\pgfpathlineto{\pgfqpoint{5.822084in}{2.600464in}}%
\pgfpathlineto{\pgfqpoint{5.823563in}{3.214859in}}%
\pgfpathlineto{\pgfqpoint{5.824057in}{3.214859in}}%
\pgfpathlineto{\pgfqpoint{5.826030in}{3.379348in}}%
\pgfpathlineto{\pgfqpoint{5.826523in}{3.379348in}}%
\pgfpathlineto{\pgfqpoint{5.827016in}{2.367417in}}%
\pgfpathlineto{\pgfqpoint{5.827509in}{3.273340in}}%
\pgfpathlineto{\pgfqpoint{5.828002in}{3.379980in}}%
\pgfpathlineto{\pgfqpoint{5.828496in}{3.379112in}}%
\pgfpathlineto{\pgfqpoint{5.829975in}{3.014151in}}%
\pgfpathlineto{\pgfqpoint{5.830469in}{3.014151in}}%
\pgfpathlineto{\pgfqpoint{5.831948in}{3.300535in}}%
\pgfpathlineto{\pgfqpoint{5.832442in}{3.300535in}}%
\pgfpathlineto{\pgfqpoint{5.833921in}{3.364490in}}%
\pgfpathlineto{\pgfqpoint{5.837867in}{3.364490in}}%
\pgfpathlineto{\pgfqpoint{5.838360in}{2.758256in}}%
\pgfpathlineto{\pgfqpoint{5.838854in}{3.092260in}}%
\pgfpathlineto{\pgfqpoint{5.840333in}{3.149351in}}%
\pgfpathlineto{\pgfqpoint{5.840826in}{3.149351in}}%
\pgfpathlineto{\pgfqpoint{5.841320in}{3.345440in}}%
\pgfpathlineto{\pgfqpoint{5.842306in}{3.330423in}}%
\pgfpathlineto{\pgfqpoint{5.842799in}{3.330423in}}%
\pgfpathlineto{\pgfqpoint{5.843786in}{3.349148in}}%
\pgfpathlineto{\pgfqpoint{5.845266in}{3.300193in}}%
\pgfpathlineto{\pgfqpoint{5.845759in}{2.930036in}}%
\pgfpathlineto{\pgfqpoint{5.846252in}{3.378161in}}%
\pgfpathlineto{\pgfqpoint{5.847732in}{3.378161in}}%
\pgfpathlineto{\pgfqpoint{5.848718in}{2.895507in}}%
\pgfpathlineto{\pgfqpoint{5.850198in}{3.314522in}}%
\pgfpathlineto{\pgfqpoint{5.851678in}{3.373343in}}%
\pgfpathlineto{\pgfqpoint{5.853157in}{3.373343in}}%
\pgfpathlineto{\pgfqpoint{5.854637in}{3.350456in}}%
\pgfpathlineto{\pgfqpoint{5.855623in}{3.350456in}}%
\pgfpathlineto{\pgfqpoint{5.856117in}{3.215232in}}%
\pgfpathlineto{\pgfqpoint{5.856610in}{3.349277in}}%
\pgfpathlineto{\pgfqpoint{5.858090in}{3.298740in}}%
\pgfpathlineto{\pgfqpoint{5.858583in}{3.298740in}}%
\pgfpathlineto{\pgfqpoint{5.859076in}{3.077975in}}%
\pgfpathlineto{\pgfqpoint{5.859569in}{3.081722in}}%
\pgfpathlineto{\pgfqpoint{5.861049in}{3.281482in}}%
\pgfpathlineto{\pgfqpoint{5.862529in}{3.281482in}}%
\pgfpathlineto{\pgfqpoint{5.864008in}{3.139314in}}%
\pgfpathlineto{\pgfqpoint{5.864502in}{3.362626in}}%
\pgfpathlineto{\pgfqpoint{5.864995in}{3.183858in}}%
\pgfpathlineto{\pgfqpoint{5.865981in}{3.017713in}}%
\pgfpathlineto{\pgfqpoint{5.867954in}{3.369243in}}%
\pgfpathlineto{\pgfqpoint{5.870420in}{3.369243in}}%
\pgfpathlineto{\pgfqpoint{5.870914in}{2.944343in}}%
\pgfpathlineto{\pgfqpoint{5.871407in}{3.378628in}}%
\pgfpathlineto{\pgfqpoint{5.871900in}{3.378628in}}%
\pgfpathlineto{\pgfqpoint{5.872393in}{3.379918in}}%
\pgfpathlineto{\pgfqpoint{5.873380in}{3.343482in}}%
\pgfpathlineto{\pgfqpoint{5.874859in}{3.363583in}}%
\pgfpathlineto{\pgfqpoint{5.875353in}{3.293240in}}%
\pgfpathlineto{\pgfqpoint{5.875846in}{2.998016in}}%
\pgfpathlineto{\pgfqpoint{5.876339in}{3.046949in}}%
\pgfpathlineto{\pgfqpoint{5.877819in}{3.373368in}}%
\pgfpathlineto{\pgfqpoint{5.878805in}{3.373368in}}%
\pgfpathlineto{\pgfqpoint{5.880285in}{3.112501in}}%
\pgfpathlineto{\pgfqpoint{5.881271in}{3.112501in}}%
\pgfpathlineto{\pgfqpoint{5.881765in}{3.151995in}}%
\pgfpathlineto{\pgfqpoint{5.882751in}{3.336749in}}%
\pgfpathlineto{\pgfqpoint{5.884231in}{3.243316in}}%
\pgfpathlineto{\pgfqpoint{5.885217in}{3.374129in}}%
\pgfpathlineto{\pgfqpoint{5.885711in}{3.320150in}}%
\pgfpathlineto{\pgfqpoint{5.886204in}{3.336956in}}%
\pgfpathlineto{\pgfqpoint{5.887683in}{3.372067in}}%
\pgfpathlineto{\pgfqpoint{5.891629in}{3.372067in}}%
\pgfpathlineto{\pgfqpoint{5.893109in}{3.377170in}}%
\pgfpathlineto{\pgfqpoint{5.894589in}{3.299223in}}%
\pgfpathlineto{\pgfqpoint{5.896068in}{3.374715in}}%
\pgfpathlineto{\pgfqpoint{5.898535in}{3.374715in}}%
\pgfpathlineto{\pgfqpoint{5.899028in}{3.094267in}}%
\pgfpathlineto{\pgfqpoint{5.899521in}{3.334478in}}%
\pgfpathlineto{\pgfqpoint{5.901001in}{3.266370in}}%
\pgfpathlineto{\pgfqpoint{5.901494in}{3.266370in}}%
\pgfpathlineto{\pgfqpoint{5.902974in}{3.162317in}}%
\pgfpathlineto{\pgfqpoint{5.904453in}{3.318555in}}%
\pgfpathlineto{\pgfqpoint{5.905440in}{3.318555in}}%
\pgfpathlineto{\pgfqpoint{5.906919in}{2.419393in}}%
\pgfpathlineto{\pgfqpoint{5.908399in}{3.359403in}}%
\pgfpathlineto{\pgfqpoint{5.909879in}{3.359403in}}%
\pgfpathlineto{\pgfqpoint{5.911359in}{3.070616in}}%
\pgfpathlineto{\pgfqpoint{5.912345in}{3.222757in}}%
\pgfpathlineto{\pgfqpoint{5.913331in}{2.884166in}}%
\pgfpathlineto{\pgfqpoint{5.913825in}{3.379307in}}%
\pgfpathlineto{\pgfqpoint{5.914318in}{3.214096in}}%
\pgfpathlineto{\pgfqpoint{5.915304in}{3.214096in}}%
\pgfpathlineto{\pgfqpoint{5.916784in}{3.174047in}}%
\pgfpathlineto{\pgfqpoint{5.917771in}{3.174047in}}%
\pgfpathlineto{\pgfqpoint{5.919250in}{3.191278in}}%
\pgfpathlineto{\pgfqpoint{5.920730in}{3.366302in}}%
\pgfpathlineto{\pgfqpoint{5.922210in}{3.085044in}}%
\pgfpathlineto{\pgfqpoint{5.923196in}{3.299238in}}%
\pgfpathlineto{\pgfqpoint{5.924183in}{3.145650in}}%
\pgfpathlineto{\pgfqpoint{5.925662in}{3.378143in}}%
\pgfpathlineto{\pgfqpoint{5.926649in}{3.378143in}}%
\pgfpathlineto{\pgfqpoint{5.927142in}{2.791202in}}%
\pgfpathlineto{\pgfqpoint{5.927635in}{3.374639in}}%
\pgfpathlineto{\pgfqpoint{5.929608in}{3.374639in}}%
\pgfpathlineto{\pgfqpoint{5.931088in}{2.816630in}}%
\pgfpathlineto{\pgfqpoint{5.932074in}{3.379232in}}%
\pgfpathlineto{\pgfqpoint{5.932567in}{3.337923in}}%
\pgfpathlineto{\pgfqpoint{5.934540in}{3.337923in}}%
\pgfpathlineto{\pgfqpoint{5.935527in}{3.202098in}}%
\pgfpathlineto{\pgfqpoint{5.936513in}{3.307484in}}%
\pgfpathlineto{\pgfqpoint{5.937007in}{3.057528in}}%
\pgfpathlineto{\pgfqpoint{5.937993in}{3.091196in}}%
\pgfpathlineto{\pgfqpoint{5.938486in}{3.091196in}}%
\pgfpathlineto{\pgfqpoint{5.939966in}{3.372109in}}%
\pgfpathlineto{\pgfqpoint{5.940459in}{3.372109in}}%
\pgfpathlineto{\pgfqpoint{5.941446in}{3.132921in}}%
\pgfpathlineto{\pgfqpoint{5.942925in}{3.375965in}}%
\pgfpathlineto{\pgfqpoint{5.943419in}{3.375965in}}%
\pgfpathlineto{\pgfqpoint{5.944898in}{3.216656in}}%
\pgfpathlineto{\pgfqpoint{5.945391in}{3.216656in}}%
\pgfpathlineto{\pgfqpoint{5.946871in}{3.268219in}}%
\pgfpathlineto{\pgfqpoint{5.947364in}{3.268219in}}%
\pgfpathlineto{\pgfqpoint{5.948844in}{2.977295in}}%
\pgfpathlineto{\pgfqpoint{5.949337in}{2.967002in}}%
\pgfpathlineto{\pgfqpoint{5.950817in}{3.373590in}}%
\pgfpathlineto{\pgfqpoint{5.953283in}{3.373590in}}%
\pgfpathlineto{\pgfqpoint{5.954270in}{3.379798in}}%
\pgfpathlineto{\pgfqpoint{5.957722in}{3.189970in}}%
\pgfpathlineto{\pgfqpoint{5.958215in}{3.354974in}}%
\pgfpathlineto{\pgfqpoint{5.959695in}{2.595921in}}%
\pgfpathlineto{\pgfqpoint{5.960188in}{2.595921in}}%
\pgfpathlineto{\pgfqpoint{5.960682in}{2.088378in}}%
\pgfpathlineto{\pgfqpoint{5.961175in}{2.364434in}}%
\pgfpathlineto{\pgfqpoint{5.962655in}{3.330515in}}%
\pgfpathlineto{\pgfqpoint{5.964134in}{3.330515in}}%
\pgfpathlineto{\pgfqpoint{5.964627in}{3.309044in}}%
\pgfpathlineto{\pgfqpoint{5.965121in}{3.319970in}}%
\pgfpathlineto{\pgfqpoint{5.965614in}{3.319970in}}%
\pgfpathlineto{\pgfqpoint{5.967094in}{2.480206in}}%
\pgfpathlineto{\pgfqpoint{5.968573in}{2.480206in}}%
\pgfpathlineto{\pgfqpoint{5.970053in}{3.312536in}}%
\pgfpathlineto{\pgfqpoint{5.970546in}{3.312536in}}%
\pgfpathlineto{\pgfqpoint{5.971039in}{2.821807in}}%
\pgfpathlineto{\pgfqpoint{5.971533in}{3.137755in}}%
\pgfpathlineto{\pgfqpoint{5.973012in}{3.305980in}}%
\pgfpathlineto{\pgfqpoint{5.975479in}{3.305980in}}%
\pgfpathlineto{\pgfqpoint{5.975972in}{3.339995in}}%
\pgfpathlineto{\pgfqpoint{5.977451in}{3.268026in}}%
\pgfpathlineto{\pgfqpoint{5.978438in}{3.268026in}}%
\pgfpathlineto{\pgfqpoint{5.979424in}{3.379989in}}%
\pgfpathlineto{\pgfqpoint{5.979918in}{3.379909in}}%
\pgfpathlineto{\pgfqpoint{5.981891in}{3.379909in}}%
\pgfpathlineto{\pgfqpoint{5.982384in}{2.957557in}}%
\pgfpathlineto{\pgfqpoint{5.983370in}{3.058258in}}%
\pgfpathlineto{\pgfqpoint{5.984850in}{3.217229in}}%
\pgfpathlineto{\pgfqpoint{5.986330in}{2.967505in}}%
\pgfpathlineto{\pgfqpoint{5.987809in}{3.347815in}}%
\pgfpathlineto{\pgfqpoint{5.988303in}{3.330736in}}%
\pgfpathlineto{\pgfqpoint{5.989782in}{3.368997in}}%
\pgfpathlineto{\pgfqpoint{5.990276in}{3.368997in}}%
\pgfpathlineto{\pgfqpoint{5.991755in}{3.146829in}}%
\pgfpathlineto{\pgfqpoint{5.992742in}{3.146829in}}%
\pgfpathlineto{\pgfqpoint{5.994221in}{3.231915in}}%
\pgfpathlineto{\pgfqpoint{5.994715in}{3.231915in}}%
\pgfpathlineto{\pgfqpoint{5.995208in}{3.369055in}}%
\pgfpathlineto{\pgfqpoint{5.995701in}{3.211991in}}%
\pgfpathlineto{\pgfqpoint{5.997181in}{3.082707in}}%
\pgfpathlineto{\pgfqpoint{5.998660in}{3.095656in}}%
\pgfpathlineto{\pgfqpoint{5.999154in}{3.095656in}}%
\pgfpathlineto{\pgfqpoint{6.000140in}{3.310585in}}%
\pgfpathlineto{\pgfqpoint{6.002113in}{3.149148in}}%
\pgfpathlineto{\pgfqpoint{6.003593in}{3.364430in}}%
\pgfpathlineto{\pgfqpoint{6.004086in}{3.364430in}}%
\pgfpathlineto{\pgfqpoint{6.005566in}{3.207128in}}%
\pgfpathlineto{\pgfqpoint{6.006059in}{3.335564in}}%
\pgfpathlineto{\pgfqpoint{6.006552in}{3.270021in}}%
\pgfpathlineto{\pgfqpoint{6.007539in}{2.981305in}}%
\pgfpathlineto{\pgfqpoint{6.009018in}{3.274376in}}%
\pgfpathlineto{\pgfqpoint{6.010991in}{3.274376in}}%
\pgfpathlineto{\pgfqpoint{6.012471in}{3.248487in}}%
\pgfpathlineto{\pgfqpoint{6.013951in}{2.973463in}}%
\pgfpathlineto{\pgfqpoint{6.014444in}{2.973463in}}%
\pgfpathlineto{\pgfqpoint{6.015430in}{3.318486in}}%
\pgfpathlineto{\pgfqpoint{6.018390in}{2.693460in}}%
\pgfpathlineto{\pgfqpoint{6.019376in}{3.379779in}}%
\pgfpathlineto{\pgfqpoint{6.019869in}{2.705401in}}%
\pgfpathlineto{\pgfqpoint{6.020363in}{3.228807in}}%
\pgfpathlineto{\pgfqpoint{6.022829in}{3.228807in}}%
\pgfpathlineto{\pgfqpoint{6.024308in}{3.372037in}}%
\pgfpathlineto{\pgfqpoint{6.029241in}{3.372037in}}%
\pgfpathlineto{\pgfqpoint{6.030227in}{3.379791in}}%
\pgfpathlineto{\pgfqpoint{6.030720in}{1.481358in}}%
\pgfpathlineto{\pgfqpoint{6.031214in}{2.973279in}}%
\pgfpathlineto{\pgfqpoint{6.031707in}{2.973279in}}%
\pgfpathlineto{\pgfqpoint{6.033187in}{3.358891in}}%
\pgfpathlineto{\pgfqpoint{6.034173in}{3.151692in}}%
\pgfpathlineto{\pgfqpoint{6.034666in}{3.376110in}}%
\pgfpathlineto{\pgfqpoint{6.035653in}{3.361663in}}%
\pgfpathlineto{\pgfqpoint{6.041078in}{3.361663in}}%
\pgfpathlineto{\pgfqpoint{6.044531in}{3.330261in}}%
\pgfpathlineto{\pgfqpoint{6.048477in}{3.330261in}}%
\pgfpathlineto{\pgfqpoint{6.050450in}{2.785518in}}%
\pgfpathlineto{\pgfqpoint{6.050943in}{2.709894in}}%
\pgfpathlineto{\pgfqpoint{6.052423in}{3.260596in}}%
\pgfpathlineto{\pgfqpoint{6.052916in}{3.375083in}}%
\pgfpathlineto{\pgfqpoint{6.053409in}{1.565280in}}%
\pgfpathlineto{\pgfqpoint{6.053409in}{1.565280in}}%
\pgfusepath{stroke}%
\end{pgfscope}%
\begin{pgfscope}%
\pgfsetrectcap%
\pgfsetmiterjoin%
\pgfsetlinewidth{0.000000pt}%
\definecolor{currentstroke}{rgb}{1.000000,1.000000,1.000000}%
\pgfsetstrokecolor{currentstroke}%
\pgfsetdash{}{0pt}%
\pgfpathmoveto{\pgfqpoint{0.875000in}{0.440000in}}%
\pgfpathlineto{\pgfqpoint{0.875000in}{3.520000in}}%
\pgfusepath{}%
\end{pgfscope}%
\begin{pgfscope}%
\pgfsetrectcap%
\pgfsetmiterjoin%
\pgfsetlinewidth{0.000000pt}%
\definecolor{currentstroke}{rgb}{1.000000,1.000000,1.000000}%
\pgfsetstrokecolor{currentstroke}%
\pgfsetdash{}{0pt}%
\pgfpathmoveto{\pgfqpoint{6.300000in}{0.440000in}}%
\pgfpathlineto{\pgfqpoint{6.300000in}{3.520000in}}%
\pgfusepath{}%
\end{pgfscope}%
\begin{pgfscope}%
\pgfsetrectcap%
\pgfsetmiterjoin%
\pgfsetlinewidth{0.000000pt}%
\definecolor{currentstroke}{rgb}{1.000000,1.000000,1.000000}%
\pgfsetstrokecolor{currentstroke}%
\pgfsetdash{}{0pt}%
\pgfpathmoveto{\pgfqpoint{0.875000in}{0.440000in}}%
\pgfpathlineto{\pgfqpoint{6.300000in}{0.440000in}}%
\pgfusepath{}%
\end{pgfscope}%
\begin{pgfscope}%
\pgfsetrectcap%
\pgfsetmiterjoin%
\pgfsetlinewidth{0.000000pt}%
\definecolor{currentstroke}{rgb}{1.000000,1.000000,1.000000}%
\pgfsetstrokecolor{currentstroke}%
\pgfsetdash{}{0pt}%
\pgfpathmoveto{\pgfqpoint{0.875000in}{3.520000in}}%
\pgfpathlineto{\pgfqpoint{6.300000in}{3.520000in}}%
\pgfusepath{}%
\end{pgfscope}%
\begin{pgfscope}%
\definecolor{textcolor}{rgb}{0.150000,0.150000,0.150000}%
\pgfsetstrokecolor{textcolor}%
\pgfsetfillcolor{textcolor}%
\pgftext[x=3.500000in,y=3.920000in,,top]{\color{textcolor}\rmfamily\fontsize{12.000000}{14.400000}\selectfont Gráfica de \(\displaystyle \log f(X_t)\) para propuesta \(\displaystyle U(0,1)\), \(\displaystyle n=40, r=\)15}%
\end{pgfscope}%
\end{pgfpicture}%
\makeatother%
\endgroup%

    \end{center}

    En este caso tenemos que la primera gráfica tiene a estabilizarse relativamente pronto, por lo
    que se pueden descartar los primeros valores como \textit{burn-in}. La segunda gráfica presenta
    un comportamiento más irregular que las anteriores, esto puede deberse a que efectivamente se muestrea desde la
    densidad objetivo desde el comienzo, o a que la propuesta dada no permite una convergencia rápida.
    Decidimos descartar los primeros 20 valores de la cadena con $n=5, r = 3$ y conservar todos en el
    caso de la cadena con $n=40, r = 15$.

    Una pregunta interesante es cuál de las dos funciones de transición propuestas nos da la mejor
    velocidad de convergencia. Un indicador de esta velocidad es que tan frecuentemente se rechaza la
    transición; si la transición se acepta con más frecuencia, entonces las propuestas son más parecidas
    a la distribución objetivo, y por lo tanto se muestrea más rápido.

    En las siguiente gráficas podemos comparar el comportamiento de las cadenas con transición beta y
    uniforme para $r_5$ y $r_{40}$. Notemos que a pesar de ser la misma cantidad de puntos en cada caso,
    la cadena con distribución beta y $n=5, r=3$ rechaza la transición con más frecuencia (tiene más intervalos 
    de constancia), puede ser causado porque
    el soporte de la densidad beta incluye puntos que no se encuentran en el intervalo $(0,1/2)$.
    Lo anterior podría significar que la cadena con función de transición beta realiza un muestreo más
    lento en este caso. En el caso con $n=40$ y $r=13$, la cadena con densidad de transición uniforme
    parece rechazar con más frecuencia.

    \begin{center}
        %% Creator: Matplotlib, PGF backend
%%
%% To include the figure in your LaTeX document, write
%%   \input{<filename>.pgf}
%%
%% Make sure the required packages are loaded in your preamble
%%   \usepackage{pgf}
%%
%% Also ensure that all the required font packages are loaded; for instance,
%% the lmodern package is sometimes necessary when using math font.
%%   \usepackage{lmodern}
%%
%% Figures using additional raster images can only be included by \input if
%% they are in the same directory as the main LaTeX file. For loading figures
%% from other directories you can use the `import` package
%%   \usepackage{import}
%%
%% and then include the figures with
%%   \import{<path to file>}{<filename>.pgf}
%%
%% Matplotlib used the following preamble
%%   
%%   \makeatletter\@ifpackageloaded{underscore}{}{\usepackage[strings]{underscore}}\makeatother
%%
\begingroup%
\makeatletter%
\begin{pgfpicture}%
\pgfpathrectangle{\pgfpointorigin}{\pgfqpoint{7.000000in}{4.000000in}}%
\pgfusepath{use as bounding box, clip}%
\begin{pgfscope}%
\pgfsetbuttcap%
\pgfsetmiterjoin%
\definecolor{currentfill}{rgb}{1.000000,1.000000,1.000000}%
\pgfsetfillcolor{currentfill}%
\pgfsetlinewidth{0.000000pt}%
\definecolor{currentstroke}{rgb}{1.000000,1.000000,1.000000}%
\pgfsetstrokecolor{currentstroke}%
\pgfsetdash{}{0pt}%
\pgfpathmoveto{\pgfqpoint{0.000000in}{0.000000in}}%
\pgfpathlineto{\pgfqpoint{7.000000in}{0.000000in}}%
\pgfpathlineto{\pgfqpoint{7.000000in}{4.000000in}}%
\pgfpathlineto{\pgfqpoint{0.000000in}{4.000000in}}%
\pgfpathlineto{\pgfqpoint{0.000000in}{0.000000in}}%
\pgfpathclose%
\pgfusepath{fill}%
\end{pgfscope}%
\begin{pgfscope}%
\pgfsetbuttcap%
\pgfsetmiterjoin%
\definecolor{currentfill}{rgb}{0.917647,0.917647,0.949020}%
\pgfsetfillcolor{currentfill}%
\pgfsetlinewidth{0.000000pt}%
\definecolor{currentstroke}{rgb}{0.000000,0.000000,0.000000}%
\pgfsetstrokecolor{currentstroke}%
\pgfsetstrokeopacity{0.000000}%
\pgfsetdash{}{0pt}%
\pgfpathmoveto{\pgfqpoint{0.875000in}{0.440000in}}%
\pgfpathlineto{\pgfqpoint{6.300000in}{0.440000in}}%
\pgfpathlineto{\pgfqpoint{6.300000in}{3.520000in}}%
\pgfpathlineto{\pgfqpoint{0.875000in}{3.520000in}}%
\pgfpathlineto{\pgfqpoint{0.875000in}{0.440000in}}%
\pgfpathclose%
\pgfusepath{fill}%
\end{pgfscope}%
\begin{pgfscope}%
\pgfpathrectangle{\pgfqpoint{0.875000in}{0.440000in}}{\pgfqpoint{5.425000in}{3.080000in}}%
\pgfusepath{clip}%
\pgfsetroundcap%
\pgfsetroundjoin%
\pgfsetlinewidth{1.003750pt}%
\definecolor{currentstroke}{rgb}{1.000000,1.000000,1.000000}%
\pgfsetstrokecolor{currentstroke}%
\pgfsetdash{}{0pt}%
\pgfpathmoveto{\pgfqpoint{1.121591in}{0.440000in}}%
\pgfpathlineto{\pgfqpoint{1.121591in}{3.520000in}}%
\pgfusepath{stroke}%
\end{pgfscope}%
\begin{pgfscope}%
\definecolor{textcolor}{rgb}{0.150000,0.150000,0.150000}%
\pgfsetstrokecolor{textcolor}%
\pgfsetfillcolor{textcolor}%
\pgftext[x=1.121591in,y=0.342778in,,top]{\color{textcolor}\rmfamily\fontsize{10.000000}{12.000000}\selectfont \(\displaystyle {0}\)}%
\end{pgfscope}%
\begin{pgfscope}%
\pgfpathrectangle{\pgfqpoint{0.875000in}{0.440000in}}{\pgfqpoint{5.425000in}{3.080000in}}%
\pgfusepath{clip}%
\pgfsetroundcap%
\pgfsetroundjoin%
\pgfsetlinewidth{1.003750pt}%
\definecolor{currentstroke}{rgb}{1.000000,1.000000,1.000000}%
\pgfsetstrokecolor{currentstroke}%
\pgfsetdash{}{0pt}%
\pgfpathmoveto{\pgfqpoint{2.107955in}{0.440000in}}%
\pgfpathlineto{\pgfqpoint{2.107955in}{3.520000in}}%
\pgfusepath{stroke}%
\end{pgfscope}%
\begin{pgfscope}%
\definecolor{textcolor}{rgb}{0.150000,0.150000,0.150000}%
\pgfsetstrokecolor{textcolor}%
\pgfsetfillcolor{textcolor}%
\pgftext[x=2.107955in,y=0.342778in,,top]{\color{textcolor}\rmfamily\fontsize{10.000000}{12.000000}\selectfont \(\displaystyle {200}\)}%
\end{pgfscope}%
\begin{pgfscope}%
\pgfpathrectangle{\pgfqpoint{0.875000in}{0.440000in}}{\pgfqpoint{5.425000in}{3.080000in}}%
\pgfusepath{clip}%
\pgfsetroundcap%
\pgfsetroundjoin%
\pgfsetlinewidth{1.003750pt}%
\definecolor{currentstroke}{rgb}{1.000000,1.000000,1.000000}%
\pgfsetstrokecolor{currentstroke}%
\pgfsetdash{}{0pt}%
\pgfpathmoveto{\pgfqpoint{3.094318in}{0.440000in}}%
\pgfpathlineto{\pgfqpoint{3.094318in}{3.520000in}}%
\pgfusepath{stroke}%
\end{pgfscope}%
\begin{pgfscope}%
\definecolor{textcolor}{rgb}{0.150000,0.150000,0.150000}%
\pgfsetstrokecolor{textcolor}%
\pgfsetfillcolor{textcolor}%
\pgftext[x=3.094318in,y=0.342778in,,top]{\color{textcolor}\rmfamily\fontsize{10.000000}{12.000000}\selectfont \(\displaystyle {400}\)}%
\end{pgfscope}%
\begin{pgfscope}%
\pgfpathrectangle{\pgfqpoint{0.875000in}{0.440000in}}{\pgfqpoint{5.425000in}{3.080000in}}%
\pgfusepath{clip}%
\pgfsetroundcap%
\pgfsetroundjoin%
\pgfsetlinewidth{1.003750pt}%
\definecolor{currentstroke}{rgb}{1.000000,1.000000,1.000000}%
\pgfsetstrokecolor{currentstroke}%
\pgfsetdash{}{0pt}%
\pgfpathmoveto{\pgfqpoint{4.080682in}{0.440000in}}%
\pgfpathlineto{\pgfqpoint{4.080682in}{3.520000in}}%
\pgfusepath{stroke}%
\end{pgfscope}%
\begin{pgfscope}%
\definecolor{textcolor}{rgb}{0.150000,0.150000,0.150000}%
\pgfsetstrokecolor{textcolor}%
\pgfsetfillcolor{textcolor}%
\pgftext[x=4.080682in,y=0.342778in,,top]{\color{textcolor}\rmfamily\fontsize{10.000000}{12.000000}\selectfont \(\displaystyle {600}\)}%
\end{pgfscope}%
\begin{pgfscope}%
\pgfpathrectangle{\pgfqpoint{0.875000in}{0.440000in}}{\pgfqpoint{5.425000in}{3.080000in}}%
\pgfusepath{clip}%
\pgfsetroundcap%
\pgfsetroundjoin%
\pgfsetlinewidth{1.003750pt}%
\definecolor{currentstroke}{rgb}{1.000000,1.000000,1.000000}%
\pgfsetstrokecolor{currentstroke}%
\pgfsetdash{}{0pt}%
\pgfpathmoveto{\pgfqpoint{5.067045in}{0.440000in}}%
\pgfpathlineto{\pgfqpoint{5.067045in}{3.520000in}}%
\pgfusepath{stroke}%
\end{pgfscope}%
\begin{pgfscope}%
\definecolor{textcolor}{rgb}{0.150000,0.150000,0.150000}%
\pgfsetstrokecolor{textcolor}%
\pgfsetfillcolor{textcolor}%
\pgftext[x=5.067045in,y=0.342778in,,top]{\color{textcolor}\rmfamily\fontsize{10.000000}{12.000000}\selectfont \(\displaystyle {800}\)}%
\end{pgfscope}%
\begin{pgfscope}%
\pgfpathrectangle{\pgfqpoint{0.875000in}{0.440000in}}{\pgfqpoint{5.425000in}{3.080000in}}%
\pgfusepath{clip}%
\pgfsetroundcap%
\pgfsetroundjoin%
\pgfsetlinewidth{1.003750pt}%
\definecolor{currentstroke}{rgb}{1.000000,1.000000,1.000000}%
\pgfsetstrokecolor{currentstroke}%
\pgfsetdash{}{0pt}%
\pgfpathmoveto{\pgfqpoint{6.053409in}{0.440000in}}%
\pgfpathlineto{\pgfqpoint{6.053409in}{3.520000in}}%
\pgfusepath{stroke}%
\end{pgfscope}%
\begin{pgfscope}%
\definecolor{textcolor}{rgb}{0.150000,0.150000,0.150000}%
\pgfsetstrokecolor{textcolor}%
\pgfsetfillcolor{textcolor}%
\pgftext[x=6.053409in,y=0.342778in,,top]{\color{textcolor}\rmfamily\fontsize{10.000000}{12.000000}\selectfont \(\displaystyle {1000}\)}%
\end{pgfscope}%
\begin{pgfscope}%
\definecolor{textcolor}{rgb}{0.150000,0.150000,0.150000}%
\pgfsetstrokecolor{textcolor}%
\pgfsetfillcolor{textcolor}%
\pgftext[x=3.587500in,y=0.163766in,,top]{\color{textcolor}\rmfamily\fontsize{11.000000}{13.200000}\selectfont Paso de la cadena (\(\displaystyle t\))}%
\end{pgfscope}%
\begin{pgfscope}%
\pgfpathrectangle{\pgfqpoint{0.875000in}{0.440000in}}{\pgfqpoint{5.425000in}{3.080000in}}%
\pgfusepath{clip}%
\pgfsetroundcap%
\pgfsetroundjoin%
\pgfsetlinewidth{1.003750pt}%
\definecolor{currentstroke}{rgb}{1.000000,1.000000,1.000000}%
\pgfsetstrokecolor{currentstroke}%
\pgfsetdash{}{0pt}%
\pgfpathmoveto{\pgfqpoint{0.875000in}{0.544899in}}%
\pgfpathlineto{\pgfqpoint{6.300000in}{0.544899in}}%
\pgfusepath{stroke}%
\end{pgfscope}%
\begin{pgfscope}%
\definecolor{textcolor}{rgb}{0.150000,0.150000,0.150000}%
\pgfsetstrokecolor{textcolor}%
\pgfsetfillcolor{textcolor}%
\pgftext[x=0.530863in, y=0.496673in, left, base]{\color{textcolor}\rmfamily\fontsize{10.000000}{12.000000}\selectfont \(\displaystyle {0.10}\)}%
\end{pgfscope}%
\begin{pgfscope}%
\pgfpathrectangle{\pgfqpoint{0.875000in}{0.440000in}}{\pgfqpoint{5.425000in}{3.080000in}}%
\pgfusepath{clip}%
\pgfsetroundcap%
\pgfsetroundjoin%
\pgfsetlinewidth{1.003750pt}%
\definecolor{currentstroke}{rgb}{1.000000,1.000000,1.000000}%
\pgfsetstrokecolor{currentstroke}%
\pgfsetdash{}{0pt}%
\pgfpathmoveto{\pgfqpoint{0.875000in}{0.930638in}}%
\pgfpathlineto{\pgfqpoint{6.300000in}{0.930638in}}%
\pgfusepath{stroke}%
\end{pgfscope}%
\begin{pgfscope}%
\definecolor{textcolor}{rgb}{0.150000,0.150000,0.150000}%
\pgfsetstrokecolor{textcolor}%
\pgfsetfillcolor{textcolor}%
\pgftext[x=0.530863in, y=0.882412in, left, base]{\color{textcolor}\rmfamily\fontsize{10.000000}{12.000000}\selectfont \(\displaystyle {0.15}\)}%
\end{pgfscope}%
\begin{pgfscope}%
\pgfpathrectangle{\pgfqpoint{0.875000in}{0.440000in}}{\pgfqpoint{5.425000in}{3.080000in}}%
\pgfusepath{clip}%
\pgfsetroundcap%
\pgfsetroundjoin%
\pgfsetlinewidth{1.003750pt}%
\definecolor{currentstroke}{rgb}{1.000000,1.000000,1.000000}%
\pgfsetstrokecolor{currentstroke}%
\pgfsetdash{}{0pt}%
\pgfpathmoveto{\pgfqpoint{0.875000in}{1.316377in}}%
\pgfpathlineto{\pgfqpoint{6.300000in}{1.316377in}}%
\pgfusepath{stroke}%
\end{pgfscope}%
\begin{pgfscope}%
\definecolor{textcolor}{rgb}{0.150000,0.150000,0.150000}%
\pgfsetstrokecolor{textcolor}%
\pgfsetfillcolor{textcolor}%
\pgftext[x=0.530863in, y=1.268151in, left, base]{\color{textcolor}\rmfamily\fontsize{10.000000}{12.000000}\selectfont \(\displaystyle {0.20}\)}%
\end{pgfscope}%
\begin{pgfscope}%
\pgfpathrectangle{\pgfqpoint{0.875000in}{0.440000in}}{\pgfqpoint{5.425000in}{3.080000in}}%
\pgfusepath{clip}%
\pgfsetroundcap%
\pgfsetroundjoin%
\pgfsetlinewidth{1.003750pt}%
\definecolor{currentstroke}{rgb}{1.000000,1.000000,1.000000}%
\pgfsetstrokecolor{currentstroke}%
\pgfsetdash{}{0pt}%
\pgfpathmoveto{\pgfqpoint{0.875000in}{1.702116in}}%
\pgfpathlineto{\pgfqpoint{6.300000in}{1.702116in}}%
\pgfusepath{stroke}%
\end{pgfscope}%
\begin{pgfscope}%
\definecolor{textcolor}{rgb}{0.150000,0.150000,0.150000}%
\pgfsetstrokecolor{textcolor}%
\pgfsetfillcolor{textcolor}%
\pgftext[x=0.530863in, y=1.653891in, left, base]{\color{textcolor}\rmfamily\fontsize{10.000000}{12.000000}\selectfont \(\displaystyle {0.25}\)}%
\end{pgfscope}%
\begin{pgfscope}%
\pgfpathrectangle{\pgfqpoint{0.875000in}{0.440000in}}{\pgfqpoint{5.425000in}{3.080000in}}%
\pgfusepath{clip}%
\pgfsetroundcap%
\pgfsetroundjoin%
\pgfsetlinewidth{1.003750pt}%
\definecolor{currentstroke}{rgb}{1.000000,1.000000,1.000000}%
\pgfsetstrokecolor{currentstroke}%
\pgfsetdash{}{0pt}%
\pgfpathmoveto{\pgfqpoint{0.875000in}{2.087855in}}%
\pgfpathlineto{\pgfqpoint{6.300000in}{2.087855in}}%
\pgfusepath{stroke}%
\end{pgfscope}%
\begin{pgfscope}%
\definecolor{textcolor}{rgb}{0.150000,0.150000,0.150000}%
\pgfsetstrokecolor{textcolor}%
\pgfsetfillcolor{textcolor}%
\pgftext[x=0.530863in, y=2.039630in, left, base]{\color{textcolor}\rmfamily\fontsize{10.000000}{12.000000}\selectfont \(\displaystyle {0.30}\)}%
\end{pgfscope}%
\begin{pgfscope}%
\pgfpathrectangle{\pgfqpoint{0.875000in}{0.440000in}}{\pgfqpoint{5.425000in}{3.080000in}}%
\pgfusepath{clip}%
\pgfsetroundcap%
\pgfsetroundjoin%
\pgfsetlinewidth{1.003750pt}%
\definecolor{currentstroke}{rgb}{1.000000,1.000000,1.000000}%
\pgfsetstrokecolor{currentstroke}%
\pgfsetdash{}{0pt}%
\pgfpathmoveto{\pgfqpoint{0.875000in}{2.473594in}}%
\pgfpathlineto{\pgfqpoint{6.300000in}{2.473594in}}%
\pgfusepath{stroke}%
\end{pgfscope}%
\begin{pgfscope}%
\definecolor{textcolor}{rgb}{0.150000,0.150000,0.150000}%
\pgfsetstrokecolor{textcolor}%
\pgfsetfillcolor{textcolor}%
\pgftext[x=0.530863in, y=2.425369in, left, base]{\color{textcolor}\rmfamily\fontsize{10.000000}{12.000000}\selectfont \(\displaystyle {0.35}\)}%
\end{pgfscope}%
\begin{pgfscope}%
\pgfpathrectangle{\pgfqpoint{0.875000in}{0.440000in}}{\pgfqpoint{5.425000in}{3.080000in}}%
\pgfusepath{clip}%
\pgfsetroundcap%
\pgfsetroundjoin%
\pgfsetlinewidth{1.003750pt}%
\definecolor{currentstroke}{rgb}{1.000000,1.000000,1.000000}%
\pgfsetstrokecolor{currentstroke}%
\pgfsetdash{}{0pt}%
\pgfpathmoveto{\pgfqpoint{0.875000in}{2.859333in}}%
\pgfpathlineto{\pgfqpoint{6.300000in}{2.859333in}}%
\pgfusepath{stroke}%
\end{pgfscope}%
\begin{pgfscope}%
\definecolor{textcolor}{rgb}{0.150000,0.150000,0.150000}%
\pgfsetstrokecolor{textcolor}%
\pgfsetfillcolor{textcolor}%
\pgftext[x=0.530863in, y=2.811108in, left, base]{\color{textcolor}\rmfamily\fontsize{10.000000}{12.000000}\selectfont \(\displaystyle {0.40}\)}%
\end{pgfscope}%
\begin{pgfscope}%
\pgfpathrectangle{\pgfqpoint{0.875000in}{0.440000in}}{\pgfqpoint{5.425000in}{3.080000in}}%
\pgfusepath{clip}%
\pgfsetroundcap%
\pgfsetroundjoin%
\pgfsetlinewidth{1.003750pt}%
\definecolor{currentstroke}{rgb}{1.000000,1.000000,1.000000}%
\pgfsetstrokecolor{currentstroke}%
\pgfsetdash{}{0pt}%
\pgfpathmoveto{\pgfqpoint{0.875000in}{3.245072in}}%
\pgfpathlineto{\pgfqpoint{6.300000in}{3.245072in}}%
\pgfusepath{stroke}%
\end{pgfscope}%
\begin{pgfscope}%
\definecolor{textcolor}{rgb}{0.150000,0.150000,0.150000}%
\pgfsetstrokecolor{textcolor}%
\pgfsetfillcolor{textcolor}%
\pgftext[x=0.530863in, y=3.196847in, left, base]{\color{textcolor}\rmfamily\fontsize{10.000000}{12.000000}\selectfont \(\displaystyle {0.45}\)}%
\end{pgfscope}%
\begin{pgfscope}%
\definecolor{textcolor}{rgb}{0.150000,0.150000,0.150000}%
\pgfsetstrokecolor{textcolor}%
\pgfsetfillcolor{textcolor}%
\pgftext[x=0.475308in,y=1.980000in,,bottom,rotate=90.000000]{\color{textcolor}\rmfamily\fontsize{11.000000}{13.200000}\selectfont Valor de \(\displaystyle X_t\)}%
\end{pgfscope}%
\begin{pgfscope}%
\pgfpathrectangle{\pgfqpoint{0.875000in}{0.440000in}}{\pgfqpoint{5.425000in}{3.080000in}}%
\pgfusepath{clip}%
\pgfsetroundcap%
\pgfsetroundjoin%
\pgfsetlinewidth{1.756562pt}%
\definecolor{currentstroke}{rgb}{0.298039,0.447059,0.690196}%
\pgfsetstrokecolor{currentstroke}%
\pgfsetdash{}{0pt}%
\pgfpathmoveto{\pgfqpoint{1.121591in}{2.132487in}}%
\pgfpathlineto{\pgfqpoint{1.126523in}{2.538437in}}%
\pgfpathlineto{\pgfqpoint{1.205432in}{2.538437in}}%
\pgfpathlineto{\pgfqpoint{1.210364in}{1.138860in}}%
\pgfpathlineto{\pgfqpoint{1.254750in}{1.138860in}}%
\pgfpathlineto{\pgfqpoint{1.259682in}{1.198577in}}%
\pgfpathlineto{\pgfqpoint{1.264614in}{1.198577in}}%
\pgfpathlineto{\pgfqpoint{1.269545in}{2.131838in}}%
\pgfpathlineto{\pgfqpoint{1.279409in}{2.131838in}}%
\pgfpathlineto{\pgfqpoint{1.284341in}{1.766266in}}%
\pgfpathlineto{\pgfqpoint{1.299136in}{1.766266in}}%
\pgfpathlineto{\pgfqpoint{1.304068in}{2.254650in}}%
\pgfpathlineto{\pgfqpoint{1.309000in}{2.254650in}}%
\pgfpathlineto{\pgfqpoint{1.313932in}{3.101189in}}%
\pgfpathlineto{\pgfqpoint{1.323795in}{3.101189in}}%
\pgfpathlineto{\pgfqpoint{1.328727in}{3.167287in}}%
\pgfpathlineto{\pgfqpoint{1.373114in}{3.167287in}}%
\pgfpathlineto{\pgfqpoint{1.378045in}{3.099080in}}%
\pgfpathlineto{\pgfqpoint{1.407636in}{3.099080in}}%
\pgfpathlineto{\pgfqpoint{1.412568in}{2.570708in}}%
\pgfpathlineto{\pgfqpoint{1.427364in}{2.570708in}}%
\pgfpathlineto{\pgfqpoint{1.432295in}{2.643079in}}%
\pgfpathlineto{\pgfqpoint{1.437227in}{2.643079in}}%
\pgfpathlineto{\pgfqpoint{1.442159in}{3.196052in}}%
\pgfpathlineto{\pgfqpoint{1.447091in}{3.196052in}}%
\pgfpathlineto{\pgfqpoint{1.452023in}{2.408414in}}%
\pgfpathlineto{\pgfqpoint{1.456955in}{2.572959in}}%
\pgfpathlineto{\pgfqpoint{1.461886in}{2.572959in}}%
\pgfpathlineto{\pgfqpoint{1.466818in}{1.199845in}}%
\pgfpathlineto{\pgfqpoint{1.501341in}{1.199845in}}%
\pgfpathlineto{\pgfqpoint{1.506273in}{3.010603in}}%
\pgfpathlineto{\pgfqpoint{1.511205in}{3.010603in}}%
\pgfpathlineto{\pgfqpoint{1.516136in}{2.235486in}}%
\pgfpathlineto{\pgfqpoint{1.535864in}{2.235486in}}%
\pgfpathlineto{\pgfqpoint{1.540795in}{0.967103in}}%
\pgfpathlineto{\pgfqpoint{1.555591in}{0.967103in}}%
\pgfpathlineto{\pgfqpoint{1.560523in}{3.380000in}}%
\pgfpathlineto{\pgfqpoint{1.565455in}{1.660574in}}%
\pgfpathlineto{\pgfqpoint{1.570386in}{1.543241in}}%
\pgfpathlineto{\pgfqpoint{1.575318in}{2.201234in}}%
\pgfpathlineto{\pgfqpoint{1.590114in}{2.201234in}}%
\pgfpathlineto{\pgfqpoint{1.595045in}{2.025821in}}%
\pgfpathlineto{\pgfqpoint{1.609841in}{2.025821in}}%
\pgfpathlineto{\pgfqpoint{1.614773in}{2.339120in}}%
\pgfpathlineto{\pgfqpoint{1.639432in}{2.339120in}}%
\pgfpathlineto{\pgfqpoint{1.644364in}{2.015445in}}%
\pgfpathlineto{\pgfqpoint{1.649295in}{2.446631in}}%
\pgfpathlineto{\pgfqpoint{1.669023in}{2.446631in}}%
\pgfpathlineto{\pgfqpoint{1.673955in}{2.300872in}}%
\pgfpathlineto{\pgfqpoint{1.678886in}{2.300872in}}%
\pgfpathlineto{\pgfqpoint{1.683818in}{1.283723in}}%
\pgfpathlineto{\pgfqpoint{1.688750in}{1.283723in}}%
\pgfpathlineto{\pgfqpoint{1.693682in}{2.808361in}}%
\pgfpathlineto{\pgfqpoint{1.713409in}{2.808361in}}%
\pgfpathlineto{\pgfqpoint{1.718341in}{1.780635in}}%
\pgfpathlineto{\pgfqpoint{1.767659in}{1.780635in}}%
\pgfpathlineto{\pgfqpoint{1.772591in}{3.004963in}}%
\pgfpathlineto{\pgfqpoint{1.777523in}{3.004963in}}%
\pgfpathlineto{\pgfqpoint{1.782455in}{1.749364in}}%
\pgfpathlineto{\pgfqpoint{1.861364in}{1.749364in}}%
\pgfpathlineto{\pgfqpoint{1.866295in}{3.036993in}}%
\pgfpathlineto{\pgfqpoint{1.876159in}{3.036993in}}%
\pgfpathlineto{\pgfqpoint{1.881091in}{3.046761in}}%
\pgfpathlineto{\pgfqpoint{1.886023in}{3.046761in}}%
\pgfpathlineto{\pgfqpoint{1.890955in}{1.985396in}}%
\pgfpathlineto{\pgfqpoint{1.910682in}{1.985396in}}%
\pgfpathlineto{\pgfqpoint{1.915614in}{2.571819in}}%
\pgfpathlineto{\pgfqpoint{1.940273in}{2.571819in}}%
\pgfpathlineto{\pgfqpoint{1.945205in}{1.684090in}}%
\pgfpathlineto{\pgfqpoint{1.974795in}{1.684090in}}%
\pgfpathlineto{\pgfqpoint{1.979727in}{2.230296in}}%
\pgfpathlineto{\pgfqpoint{1.994523in}{2.230296in}}%
\pgfpathlineto{\pgfqpoint{1.999455in}{3.193438in}}%
\pgfpathlineto{\pgfqpoint{2.004386in}{0.927486in}}%
\pgfpathlineto{\pgfqpoint{2.009318in}{0.781833in}}%
\pgfpathlineto{\pgfqpoint{2.043841in}{0.781833in}}%
\pgfpathlineto{\pgfqpoint{2.048773in}{1.891295in}}%
\pgfpathlineto{\pgfqpoint{2.058636in}{1.891295in}}%
\pgfpathlineto{\pgfqpoint{2.063568in}{3.149432in}}%
\pgfpathlineto{\pgfqpoint{2.068500in}{2.898274in}}%
\pgfpathlineto{\pgfqpoint{2.083295in}{2.898274in}}%
\pgfpathlineto{\pgfqpoint{2.088227in}{2.180214in}}%
\pgfpathlineto{\pgfqpoint{2.093159in}{2.490537in}}%
\pgfpathlineto{\pgfqpoint{2.098091in}{2.695157in}}%
\pgfpathlineto{\pgfqpoint{2.107955in}{2.695157in}}%
\pgfpathlineto{\pgfqpoint{2.112886in}{2.389679in}}%
\pgfpathlineto{\pgfqpoint{2.117818in}{1.799058in}}%
\pgfpathlineto{\pgfqpoint{2.152341in}{1.799058in}}%
\pgfpathlineto{\pgfqpoint{2.157273in}{3.357831in}}%
\pgfpathlineto{\pgfqpoint{2.167136in}{3.357831in}}%
\pgfpathlineto{\pgfqpoint{2.172068in}{3.199723in}}%
\pgfpathlineto{\pgfqpoint{2.191795in}{3.199723in}}%
\pgfpathlineto{\pgfqpoint{2.196727in}{2.188951in}}%
\pgfpathlineto{\pgfqpoint{2.201659in}{2.188951in}}%
\pgfpathlineto{\pgfqpoint{2.206591in}{2.407594in}}%
\pgfpathlineto{\pgfqpoint{2.221386in}{2.407594in}}%
\pgfpathlineto{\pgfqpoint{2.226318in}{2.644530in}}%
\pgfpathlineto{\pgfqpoint{2.236182in}{2.644530in}}%
\pgfpathlineto{\pgfqpoint{2.241114in}{2.610509in}}%
\pgfpathlineto{\pgfqpoint{2.246045in}{3.205814in}}%
\pgfpathlineto{\pgfqpoint{2.255909in}{3.205814in}}%
\pgfpathlineto{\pgfqpoint{2.260841in}{3.160482in}}%
\pgfpathlineto{\pgfqpoint{2.265773in}{2.429968in}}%
\pgfpathlineto{\pgfqpoint{2.275636in}{2.429968in}}%
\pgfpathlineto{\pgfqpoint{2.280568in}{2.817600in}}%
\pgfpathlineto{\pgfqpoint{2.300295in}{2.817600in}}%
\pgfpathlineto{\pgfqpoint{2.305227in}{2.932689in}}%
\pgfpathlineto{\pgfqpoint{2.320023in}{2.932689in}}%
\pgfpathlineto{\pgfqpoint{2.324955in}{1.186662in}}%
\pgfpathlineto{\pgfqpoint{2.329886in}{2.519058in}}%
\pgfpathlineto{\pgfqpoint{2.349614in}{2.519058in}}%
\pgfpathlineto{\pgfqpoint{2.354545in}{2.235817in}}%
\pgfpathlineto{\pgfqpoint{2.359477in}{2.235817in}}%
\pgfpathlineto{\pgfqpoint{2.364409in}{2.791503in}}%
\pgfpathlineto{\pgfqpoint{2.389068in}{2.791503in}}%
\pgfpathlineto{\pgfqpoint{2.394000in}{2.322289in}}%
\pgfpathlineto{\pgfqpoint{2.398932in}{2.322289in}}%
\pgfpathlineto{\pgfqpoint{2.403864in}{2.467388in}}%
\pgfpathlineto{\pgfqpoint{2.418659in}{2.467388in}}%
\pgfpathlineto{\pgfqpoint{2.423591in}{2.207347in}}%
\pgfpathlineto{\pgfqpoint{2.443318in}{2.207347in}}%
\pgfpathlineto{\pgfqpoint{2.448250in}{1.855877in}}%
\pgfpathlineto{\pgfqpoint{2.517295in}{1.855877in}}%
\pgfpathlineto{\pgfqpoint{2.522227in}{2.812424in}}%
\pgfpathlineto{\pgfqpoint{2.537023in}{2.812424in}}%
\pgfpathlineto{\pgfqpoint{2.541955in}{2.993763in}}%
\pgfpathlineto{\pgfqpoint{2.576477in}{2.993763in}}%
\pgfpathlineto{\pgfqpoint{2.581409in}{2.566237in}}%
\pgfpathlineto{\pgfqpoint{2.611000in}{2.566237in}}%
\pgfpathlineto{\pgfqpoint{2.615932in}{2.220233in}}%
\pgfpathlineto{\pgfqpoint{2.620864in}{2.220233in}}%
\pgfpathlineto{\pgfqpoint{2.625795in}{1.480330in}}%
\pgfpathlineto{\pgfqpoint{2.680045in}{1.480330in}}%
\pgfpathlineto{\pgfqpoint{2.684977in}{2.587308in}}%
\pgfpathlineto{\pgfqpoint{2.729364in}{2.587308in}}%
\pgfpathlineto{\pgfqpoint{2.734295in}{1.774563in}}%
\pgfpathlineto{\pgfqpoint{2.788545in}{1.774563in}}%
\pgfpathlineto{\pgfqpoint{2.793477in}{3.353165in}}%
\pgfpathlineto{\pgfqpoint{2.803341in}{3.353165in}}%
\pgfpathlineto{\pgfqpoint{2.808273in}{2.131785in}}%
\pgfpathlineto{\pgfqpoint{2.897045in}{2.131785in}}%
\pgfpathlineto{\pgfqpoint{2.901977in}{1.855026in}}%
\pgfpathlineto{\pgfqpoint{2.906909in}{1.950501in}}%
\pgfpathlineto{\pgfqpoint{2.911841in}{1.950501in}}%
\pgfpathlineto{\pgfqpoint{2.916773in}{1.932533in}}%
\pgfpathlineto{\pgfqpoint{2.921705in}{0.870796in}}%
\pgfpathlineto{\pgfqpoint{2.926636in}{0.870796in}}%
\pgfpathlineto{\pgfqpoint{2.931568in}{1.546707in}}%
\pgfpathlineto{\pgfqpoint{2.946364in}{1.546707in}}%
\pgfpathlineto{\pgfqpoint{2.951295in}{2.944157in}}%
\pgfpathlineto{\pgfqpoint{2.956227in}{2.233027in}}%
\pgfpathlineto{\pgfqpoint{2.961159in}{2.233027in}}%
\pgfpathlineto{\pgfqpoint{2.966091in}{1.709353in}}%
\pgfpathlineto{\pgfqpoint{3.000614in}{1.709353in}}%
\pgfpathlineto{\pgfqpoint{3.005545in}{2.583173in}}%
\pgfpathlineto{\pgfqpoint{3.020341in}{2.583173in}}%
\pgfpathlineto{\pgfqpoint{3.025273in}{0.980231in}}%
\pgfpathlineto{\pgfqpoint{3.040068in}{0.980231in}}%
\pgfpathlineto{\pgfqpoint{3.045000in}{1.576988in}}%
\pgfpathlineto{\pgfqpoint{3.059795in}{1.576988in}}%
\pgfpathlineto{\pgfqpoint{3.064727in}{3.339922in}}%
\pgfpathlineto{\pgfqpoint{3.099250in}{3.339922in}}%
\pgfpathlineto{\pgfqpoint{3.104182in}{3.065863in}}%
\pgfpathlineto{\pgfqpoint{3.138705in}{3.065863in}}%
\pgfpathlineto{\pgfqpoint{3.143636in}{2.011396in}}%
\pgfpathlineto{\pgfqpoint{3.192955in}{2.011396in}}%
\pgfpathlineto{\pgfqpoint{3.197886in}{2.539172in}}%
\pgfpathlineto{\pgfqpoint{3.252136in}{2.539172in}}%
\pgfpathlineto{\pgfqpoint{3.257068in}{1.815203in}}%
\pgfpathlineto{\pgfqpoint{3.262000in}{1.815203in}}%
\pgfpathlineto{\pgfqpoint{3.266932in}{1.619035in}}%
\pgfpathlineto{\pgfqpoint{3.276795in}{1.619035in}}%
\pgfpathlineto{\pgfqpoint{3.281727in}{2.923836in}}%
\pgfpathlineto{\pgfqpoint{3.360636in}{2.923836in}}%
\pgfpathlineto{\pgfqpoint{3.365568in}{1.745875in}}%
\pgfpathlineto{\pgfqpoint{3.390227in}{1.745875in}}%
\pgfpathlineto{\pgfqpoint{3.395159in}{3.158498in}}%
\pgfpathlineto{\pgfqpoint{3.400091in}{3.158498in}}%
\pgfpathlineto{\pgfqpoint{3.405023in}{2.010337in}}%
\pgfpathlineto{\pgfqpoint{3.419818in}{2.010337in}}%
\pgfpathlineto{\pgfqpoint{3.424750in}{2.414008in}}%
\pgfpathlineto{\pgfqpoint{3.429682in}{2.414008in}}%
\pgfpathlineto{\pgfqpoint{3.434614in}{2.304845in}}%
\pgfpathlineto{\pgfqpoint{3.439545in}{1.870529in}}%
\pgfpathlineto{\pgfqpoint{3.464205in}{1.870529in}}%
\pgfpathlineto{\pgfqpoint{3.469136in}{2.215517in}}%
\pgfpathlineto{\pgfqpoint{3.474068in}{2.215517in}}%
\pgfpathlineto{\pgfqpoint{3.479000in}{3.138709in}}%
\pgfpathlineto{\pgfqpoint{3.488864in}{3.138709in}}%
\pgfpathlineto{\pgfqpoint{3.493795in}{3.193572in}}%
\pgfpathlineto{\pgfqpoint{3.518455in}{3.193572in}}%
\pgfpathlineto{\pgfqpoint{3.523386in}{2.387234in}}%
\pgfpathlineto{\pgfqpoint{3.528318in}{2.387234in}}%
\pgfpathlineto{\pgfqpoint{3.533250in}{1.980143in}}%
\pgfpathlineto{\pgfqpoint{3.538182in}{3.213493in}}%
\pgfpathlineto{\pgfqpoint{3.552977in}{3.213493in}}%
\pgfpathlineto{\pgfqpoint{3.557909in}{1.167538in}}%
\pgfpathlineto{\pgfqpoint{3.587500in}{1.167538in}}%
\pgfpathlineto{\pgfqpoint{3.597364in}{2.960774in}}%
\pgfpathlineto{\pgfqpoint{3.602295in}{2.960774in}}%
\pgfpathlineto{\pgfqpoint{3.607227in}{2.636810in}}%
\pgfpathlineto{\pgfqpoint{3.686136in}{2.636810in}}%
\pgfpathlineto{\pgfqpoint{3.691068in}{2.969026in}}%
\pgfpathlineto{\pgfqpoint{3.696000in}{2.977553in}}%
\pgfpathlineto{\pgfqpoint{3.700932in}{3.001056in}}%
\pgfpathlineto{\pgfqpoint{3.710795in}{3.001056in}}%
\pgfpathlineto{\pgfqpoint{3.715727in}{3.092786in}}%
\pgfpathlineto{\pgfqpoint{3.730523in}{3.092786in}}%
\pgfpathlineto{\pgfqpoint{3.735455in}{3.280549in}}%
\pgfpathlineto{\pgfqpoint{3.740386in}{3.280549in}}%
\pgfpathlineto{\pgfqpoint{3.745318in}{2.334181in}}%
\pgfpathlineto{\pgfqpoint{3.765045in}{2.334181in}}%
\pgfpathlineto{\pgfqpoint{3.769977in}{3.010511in}}%
\pgfpathlineto{\pgfqpoint{3.774909in}{2.685590in}}%
\pgfpathlineto{\pgfqpoint{3.794636in}{2.685590in}}%
\pgfpathlineto{\pgfqpoint{3.799568in}{2.825710in}}%
\pgfpathlineto{\pgfqpoint{3.814364in}{2.825710in}}%
\pgfpathlineto{\pgfqpoint{3.819295in}{1.870286in}}%
\pgfpathlineto{\pgfqpoint{3.858750in}{1.870286in}}%
\pgfpathlineto{\pgfqpoint{3.863682in}{2.452483in}}%
\pgfpathlineto{\pgfqpoint{3.873545in}{2.452483in}}%
\pgfpathlineto{\pgfqpoint{3.878477in}{1.989792in}}%
\pgfpathlineto{\pgfqpoint{3.962318in}{1.989792in}}%
\pgfpathlineto{\pgfqpoint{3.967250in}{2.520283in}}%
\pgfpathlineto{\pgfqpoint{3.972182in}{2.520283in}}%
\pgfpathlineto{\pgfqpoint{3.977114in}{3.016228in}}%
\pgfpathlineto{\pgfqpoint{3.982045in}{3.016228in}}%
\pgfpathlineto{\pgfqpoint{3.986977in}{2.928655in}}%
\pgfpathlineto{\pgfqpoint{4.016568in}{2.928655in}}%
\pgfpathlineto{\pgfqpoint{4.021500in}{2.084997in}}%
\pgfpathlineto{\pgfqpoint{4.065886in}{2.084997in}}%
\pgfpathlineto{\pgfqpoint{4.070818in}{2.636627in}}%
\pgfpathlineto{\pgfqpoint{4.075750in}{2.636627in}}%
\pgfpathlineto{\pgfqpoint{4.080682in}{2.825607in}}%
\pgfpathlineto{\pgfqpoint{4.085614in}{2.759961in}}%
\pgfpathlineto{\pgfqpoint{4.090545in}{2.759961in}}%
\pgfpathlineto{\pgfqpoint{4.095477in}{2.328420in}}%
\pgfpathlineto{\pgfqpoint{4.100409in}{3.190872in}}%
\pgfpathlineto{\pgfqpoint{4.115205in}{3.190872in}}%
\pgfpathlineto{\pgfqpoint{4.120136in}{2.435053in}}%
\pgfpathlineto{\pgfqpoint{4.149727in}{2.435053in}}%
\pgfpathlineto{\pgfqpoint{4.154659in}{2.601225in}}%
\pgfpathlineto{\pgfqpoint{4.169455in}{2.601225in}}%
\pgfpathlineto{\pgfqpoint{4.174386in}{2.289501in}}%
\pgfpathlineto{\pgfqpoint{4.189182in}{2.289501in}}%
\pgfpathlineto{\pgfqpoint{4.194114in}{3.321841in}}%
\pgfpathlineto{\pgfqpoint{4.199045in}{3.324255in}}%
\pgfpathlineto{\pgfqpoint{4.208909in}{3.324255in}}%
\pgfpathlineto{\pgfqpoint{4.213841in}{1.249198in}}%
\pgfpathlineto{\pgfqpoint{4.218773in}{2.696320in}}%
\pgfpathlineto{\pgfqpoint{4.233568in}{2.696320in}}%
\pgfpathlineto{\pgfqpoint{4.238500in}{3.013951in}}%
\pgfpathlineto{\pgfqpoint{4.243432in}{3.013951in}}%
\pgfpathlineto{\pgfqpoint{4.248364in}{3.123159in}}%
\pgfpathlineto{\pgfqpoint{4.258227in}{3.123159in}}%
\pgfpathlineto{\pgfqpoint{4.263159in}{2.767384in}}%
\pgfpathlineto{\pgfqpoint{4.273023in}{2.767384in}}%
\pgfpathlineto{\pgfqpoint{4.277955in}{2.133629in}}%
\pgfpathlineto{\pgfqpoint{4.282886in}{1.176595in}}%
\pgfpathlineto{\pgfqpoint{4.381523in}{1.176595in}}%
\pgfpathlineto{\pgfqpoint{4.386455in}{1.942496in}}%
\pgfpathlineto{\pgfqpoint{4.406182in}{1.942496in}}%
\pgfpathlineto{\pgfqpoint{4.411114in}{1.707264in}}%
\pgfpathlineto{\pgfqpoint{4.416045in}{2.727913in}}%
\pgfpathlineto{\pgfqpoint{4.420977in}{0.580000in}}%
\pgfpathlineto{\pgfqpoint{4.475227in}{0.580000in}}%
\pgfpathlineto{\pgfqpoint{4.480159in}{2.478386in}}%
\pgfpathlineto{\pgfqpoint{4.490023in}{2.478386in}}%
\pgfpathlineto{\pgfqpoint{4.494955in}{1.343796in}}%
\pgfpathlineto{\pgfqpoint{4.598523in}{1.343796in}}%
\pgfpathlineto{\pgfqpoint{4.603455in}{1.915429in}}%
\pgfpathlineto{\pgfqpoint{4.633045in}{1.915429in}}%
\pgfpathlineto{\pgfqpoint{4.637977in}{1.723900in}}%
\pgfpathlineto{\pgfqpoint{4.687295in}{1.723900in}}%
\pgfpathlineto{\pgfqpoint{4.692227in}{3.000611in}}%
\pgfpathlineto{\pgfqpoint{4.697159in}{2.272664in}}%
\pgfpathlineto{\pgfqpoint{4.702091in}{2.272664in}}%
\pgfpathlineto{\pgfqpoint{4.707023in}{2.890827in}}%
\pgfpathlineto{\pgfqpoint{4.711955in}{2.890827in}}%
\pgfpathlineto{\pgfqpoint{4.716886in}{1.433661in}}%
\pgfpathlineto{\pgfqpoint{4.781000in}{1.433661in}}%
\pgfpathlineto{\pgfqpoint{4.785932in}{0.814971in}}%
\pgfpathlineto{\pgfqpoint{4.879636in}{0.814971in}}%
\pgfpathlineto{\pgfqpoint{4.884568in}{1.763685in}}%
\pgfpathlineto{\pgfqpoint{4.914159in}{1.763685in}}%
\pgfpathlineto{\pgfqpoint{4.919091in}{2.841769in}}%
\pgfpathlineto{\pgfqpoint{4.924023in}{1.635112in}}%
\pgfpathlineto{\pgfqpoint{4.933886in}{1.635112in}}%
\pgfpathlineto{\pgfqpoint{4.938818in}{2.272699in}}%
\pgfpathlineto{\pgfqpoint{4.948682in}{2.272699in}}%
\pgfpathlineto{\pgfqpoint{4.953614in}{2.712330in}}%
\pgfpathlineto{\pgfqpoint{4.958545in}{2.712330in}}%
\pgfpathlineto{\pgfqpoint{4.963477in}{2.065323in}}%
\pgfpathlineto{\pgfqpoint{4.968409in}{2.065323in}}%
\pgfpathlineto{\pgfqpoint{4.973341in}{1.774390in}}%
\pgfpathlineto{\pgfqpoint{5.062114in}{1.774390in}}%
\pgfpathlineto{\pgfqpoint{5.067045in}{2.693758in}}%
\pgfpathlineto{\pgfqpoint{5.076909in}{2.693758in}}%
\pgfpathlineto{\pgfqpoint{5.081841in}{1.980379in}}%
\pgfpathlineto{\pgfqpoint{5.086773in}{2.522655in}}%
\pgfpathlineto{\pgfqpoint{5.096636in}{2.522655in}}%
\pgfpathlineto{\pgfqpoint{5.101568in}{2.179810in}}%
\pgfpathlineto{\pgfqpoint{5.106500in}{2.179810in}}%
\pgfpathlineto{\pgfqpoint{5.111432in}{1.734976in}}%
\pgfpathlineto{\pgfqpoint{5.160750in}{1.734976in}}%
\pgfpathlineto{\pgfqpoint{5.165682in}{2.231638in}}%
\pgfpathlineto{\pgfqpoint{5.219932in}{2.231638in}}%
\pgfpathlineto{\pgfqpoint{5.224864in}{3.025223in}}%
\pgfpathlineto{\pgfqpoint{5.229795in}{2.622997in}}%
\pgfpathlineto{\pgfqpoint{5.234727in}{2.587574in}}%
\pgfpathlineto{\pgfqpoint{5.249523in}{2.587574in}}%
\pgfpathlineto{\pgfqpoint{5.254455in}{3.319889in}}%
\pgfpathlineto{\pgfqpoint{5.259386in}{2.200253in}}%
\pgfpathlineto{\pgfqpoint{5.264318in}{2.200253in}}%
\pgfpathlineto{\pgfqpoint{5.269250in}{1.381964in}}%
\pgfpathlineto{\pgfqpoint{5.274182in}{2.327242in}}%
\pgfpathlineto{\pgfqpoint{5.279114in}{1.633791in}}%
\pgfpathlineto{\pgfqpoint{5.284045in}{2.413644in}}%
\pgfpathlineto{\pgfqpoint{5.288977in}{2.413644in}}%
\pgfpathlineto{\pgfqpoint{5.293909in}{2.469993in}}%
\pgfpathlineto{\pgfqpoint{5.303773in}{2.469993in}}%
\pgfpathlineto{\pgfqpoint{5.308705in}{2.856572in}}%
\pgfpathlineto{\pgfqpoint{5.333364in}{2.856572in}}%
\pgfpathlineto{\pgfqpoint{5.338295in}{2.006935in}}%
\pgfpathlineto{\pgfqpoint{5.377750in}{2.006935in}}%
\pgfpathlineto{\pgfqpoint{5.382682in}{3.251529in}}%
\pgfpathlineto{\pgfqpoint{5.387614in}{3.251529in}}%
\pgfpathlineto{\pgfqpoint{5.392545in}{2.350569in}}%
\pgfpathlineto{\pgfqpoint{5.407341in}{2.350569in}}%
\pgfpathlineto{\pgfqpoint{5.412273in}{3.236104in}}%
\pgfpathlineto{\pgfqpoint{5.417205in}{2.469239in}}%
\pgfpathlineto{\pgfqpoint{5.446795in}{2.469239in}}%
\pgfpathlineto{\pgfqpoint{5.451727in}{3.015121in}}%
\pgfpathlineto{\pgfqpoint{5.456659in}{2.713362in}}%
\pgfpathlineto{\pgfqpoint{5.466523in}{2.713362in}}%
\pgfpathlineto{\pgfqpoint{5.471455in}{2.507204in}}%
\pgfpathlineto{\pgfqpoint{5.491182in}{2.507204in}}%
\pgfpathlineto{\pgfqpoint{5.496114in}{2.814618in}}%
\pgfpathlineto{\pgfqpoint{5.575023in}{2.814618in}}%
\pgfpathlineto{\pgfqpoint{5.579955in}{3.126319in}}%
\pgfpathlineto{\pgfqpoint{5.589818in}{3.126319in}}%
\pgfpathlineto{\pgfqpoint{5.594750in}{3.034694in}}%
\pgfpathlineto{\pgfqpoint{5.599682in}{2.100377in}}%
\pgfpathlineto{\pgfqpoint{5.634205in}{2.100377in}}%
\pgfpathlineto{\pgfqpoint{5.639136in}{1.929645in}}%
\pgfpathlineto{\pgfqpoint{5.713114in}{1.929645in}}%
\pgfpathlineto{\pgfqpoint{5.718045in}{2.770679in}}%
\pgfpathlineto{\pgfqpoint{5.722977in}{2.489452in}}%
\pgfpathlineto{\pgfqpoint{5.727909in}{2.489452in}}%
\pgfpathlineto{\pgfqpoint{5.732841in}{1.360397in}}%
\pgfpathlineto{\pgfqpoint{5.762432in}{1.360397in}}%
\pgfpathlineto{\pgfqpoint{5.767364in}{2.014359in}}%
\pgfpathlineto{\pgfqpoint{5.792023in}{2.014359in}}%
\pgfpathlineto{\pgfqpoint{5.796955in}{2.433557in}}%
\pgfpathlineto{\pgfqpoint{5.841341in}{2.433557in}}%
\pgfpathlineto{\pgfqpoint{5.846273in}{2.183292in}}%
\pgfpathlineto{\pgfqpoint{5.870932in}{2.183292in}}%
\pgfpathlineto{\pgfqpoint{5.875864in}{2.104334in}}%
\pgfpathlineto{\pgfqpoint{5.915318in}{2.104334in}}%
\pgfpathlineto{\pgfqpoint{5.920250in}{3.362050in}}%
\pgfpathlineto{\pgfqpoint{5.925182in}{2.834535in}}%
\pgfpathlineto{\pgfqpoint{5.984364in}{2.834535in}}%
\pgfpathlineto{\pgfqpoint{5.989295in}{2.707925in}}%
\pgfpathlineto{\pgfqpoint{6.023818in}{2.707925in}}%
\pgfpathlineto{\pgfqpoint{6.028750in}{1.427360in}}%
\pgfpathlineto{\pgfqpoint{6.033682in}{1.427360in}}%
\pgfpathlineto{\pgfqpoint{6.038614in}{2.682681in}}%
\pgfpathlineto{\pgfqpoint{6.053409in}{2.682681in}}%
\pgfpathlineto{\pgfqpoint{6.053409in}{2.682681in}}%
\pgfusepath{stroke}%
\end{pgfscope}%
\begin{pgfscope}%
\pgfsetrectcap%
\pgfsetmiterjoin%
\pgfsetlinewidth{0.000000pt}%
\definecolor{currentstroke}{rgb}{1.000000,1.000000,1.000000}%
\pgfsetstrokecolor{currentstroke}%
\pgfsetdash{}{0pt}%
\pgfpathmoveto{\pgfqpoint{0.875000in}{0.440000in}}%
\pgfpathlineto{\pgfqpoint{0.875000in}{3.520000in}}%
\pgfusepath{}%
\end{pgfscope}%
\begin{pgfscope}%
\pgfsetrectcap%
\pgfsetmiterjoin%
\pgfsetlinewidth{0.000000pt}%
\definecolor{currentstroke}{rgb}{1.000000,1.000000,1.000000}%
\pgfsetstrokecolor{currentstroke}%
\pgfsetdash{}{0pt}%
\pgfpathmoveto{\pgfqpoint{6.300000in}{0.440000in}}%
\pgfpathlineto{\pgfqpoint{6.300000in}{3.520000in}}%
\pgfusepath{}%
\end{pgfscope}%
\begin{pgfscope}%
\pgfsetrectcap%
\pgfsetmiterjoin%
\pgfsetlinewidth{0.000000pt}%
\definecolor{currentstroke}{rgb}{1.000000,1.000000,1.000000}%
\pgfsetstrokecolor{currentstroke}%
\pgfsetdash{}{0pt}%
\pgfpathmoveto{\pgfqpoint{0.875000in}{0.440000in}}%
\pgfpathlineto{\pgfqpoint{6.300000in}{0.440000in}}%
\pgfusepath{}%
\end{pgfscope}%
\begin{pgfscope}%
\pgfsetrectcap%
\pgfsetmiterjoin%
\pgfsetlinewidth{0.000000pt}%
\definecolor{currentstroke}{rgb}{1.000000,1.000000,1.000000}%
\pgfsetstrokecolor{currentstroke}%
\pgfsetdash{}{0pt}%
\pgfpathmoveto{\pgfqpoint{0.875000in}{3.520000in}}%
\pgfpathlineto{\pgfqpoint{6.300000in}{3.520000in}}%
\pgfusepath{}%
\end{pgfscope}%
\begin{pgfscope}%
\definecolor{textcolor}{rgb}{0.150000,0.150000,0.150000}%
\pgfsetstrokecolor{textcolor}%
\pgfsetfillcolor{textcolor}%
\pgftext[x=3.500000in,y=3.920000in,,top]{\color{textcolor}\rmfamily\fontsize{12.000000}{14.400000}\selectfont Muestreo con propuesta beta y \(\displaystyle n=5\)}%
\end{pgfscope}%
\end{pgfpicture}%
\makeatother%
\endgroup%

        %% Creator: Matplotlib, PGF backend
%%
%% To include the figure in your LaTeX document, write
%%   \input{<filename>.pgf}
%%
%% Make sure the required packages are loaded in your preamble
%%   \usepackage{pgf}
%%
%% Also ensure that all the required font packages are loaded; for instance,
%% the lmodern package is sometimes necessary when using math font.
%%   \usepackage{lmodern}
%%
%% Figures using additional raster images can only be included by \input if
%% they are in the same directory as the main LaTeX file. For loading figures
%% from other directories you can use the `import` package
%%   \usepackage{import}
%%
%% and then include the figures with
%%   \import{<path to file>}{<filename>.pgf}
%%
%% Matplotlib used the following preamble
%%   
%%   \makeatletter\@ifpackageloaded{underscore}{}{\usepackage[strings]{underscore}}\makeatother
%%
\begingroup%
\makeatletter%
\begin{pgfpicture}%
\pgfpathrectangle{\pgfpointorigin}{\pgfqpoint{7.000000in}{4.000000in}}%
\pgfusepath{use as bounding box, clip}%
\begin{pgfscope}%
\pgfsetbuttcap%
\pgfsetmiterjoin%
\definecolor{currentfill}{rgb}{1.000000,1.000000,1.000000}%
\pgfsetfillcolor{currentfill}%
\pgfsetlinewidth{0.000000pt}%
\definecolor{currentstroke}{rgb}{1.000000,1.000000,1.000000}%
\pgfsetstrokecolor{currentstroke}%
\pgfsetdash{}{0pt}%
\pgfpathmoveto{\pgfqpoint{0.000000in}{0.000000in}}%
\pgfpathlineto{\pgfqpoint{7.000000in}{0.000000in}}%
\pgfpathlineto{\pgfqpoint{7.000000in}{4.000000in}}%
\pgfpathlineto{\pgfqpoint{0.000000in}{4.000000in}}%
\pgfpathlineto{\pgfqpoint{0.000000in}{0.000000in}}%
\pgfpathclose%
\pgfusepath{fill}%
\end{pgfscope}%
\begin{pgfscope}%
\pgfsetbuttcap%
\pgfsetmiterjoin%
\definecolor{currentfill}{rgb}{0.917647,0.917647,0.949020}%
\pgfsetfillcolor{currentfill}%
\pgfsetlinewidth{0.000000pt}%
\definecolor{currentstroke}{rgb}{0.000000,0.000000,0.000000}%
\pgfsetstrokecolor{currentstroke}%
\pgfsetstrokeopacity{0.000000}%
\pgfsetdash{}{0pt}%
\pgfpathmoveto{\pgfqpoint{0.875000in}{0.440000in}}%
\pgfpathlineto{\pgfqpoint{6.300000in}{0.440000in}}%
\pgfpathlineto{\pgfqpoint{6.300000in}{3.520000in}}%
\pgfpathlineto{\pgfqpoint{0.875000in}{3.520000in}}%
\pgfpathlineto{\pgfqpoint{0.875000in}{0.440000in}}%
\pgfpathclose%
\pgfusepath{fill}%
\end{pgfscope}%
\begin{pgfscope}%
\pgfpathrectangle{\pgfqpoint{0.875000in}{0.440000in}}{\pgfqpoint{5.425000in}{3.080000in}}%
\pgfusepath{clip}%
\pgfsetroundcap%
\pgfsetroundjoin%
\pgfsetlinewidth{1.003750pt}%
\definecolor{currentstroke}{rgb}{1.000000,1.000000,1.000000}%
\pgfsetstrokecolor{currentstroke}%
\pgfsetdash{}{0pt}%
\pgfpathmoveto{\pgfqpoint{1.121591in}{0.440000in}}%
\pgfpathlineto{\pgfqpoint{1.121591in}{3.520000in}}%
\pgfusepath{stroke}%
\end{pgfscope}%
\begin{pgfscope}%
\definecolor{textcolor}{rgb}{0.150000,0.150000,0.150000}%
\pgfsetstrokecolor{textcolor}%
\pgfsetfillcolor{textcolor}%
\pgftext[x=1.121591in,y=0.342778in,,top]{\color{textcolor}\rmfamily\fontsize{10.000000}{12.000000}\selectfont \(\displaystyle {0}\)}%
\end{pgfscope}%
\begin{pgfscope}%
\pgfpathrectangle{\pgfqpoint{0.875000in}{0.440000in}}{\pgfqpoint{5.425000in}{3.080000in}}%
\pgfusepath{clip}%
\pgfsetroundcap%
\pgfsetroundjoin%
\pgfsetlinewidth{1.003750pt}%
\definecolor{currentstroke}{rgb}{1.000000,1.000000,1.000000}%
\pgfsetstrokecolor{currentstroke}%
\pgfsetdash{}{0pt}%
\pgfpathmoveto{\pgfqpoint{2.108942in}{0.440000in}}%
\pgfpathlineto{\pgfqpoint{2.108942in}{3.520000in}}%
\pgfusepath{stroke}%
\end{pgfscope}%
\begin{pgfscope}%
\definecolor{textcolor}{rgb}{0.150000,0.150000,0.150000}%
\pgfsetstrokecolor{textcolor}%
\pgfsetfillcolor{textcolor}%
\pgftext[x=2.108942in,y=0.342778in,,top]{\color{textcolor}\rmfamily\fontsize{10.000000}{12.000000}\selectfont \(\displaystyle {200}\)}%
\end{pgfscope}%
\begin{pgfscope}%
\pgfpathrectangle{\pgfqpoint{0.875000in}{0.440000in}}{\pgfqpoint{5.425000in}{3.080000in}}%
\pgfusepath{clip}%
\pgfsetroundcap%
\pgfsetroundjoin%
\pgfsetlinewidth{1.003750pt}%
\definecolor{currentstroke}{rgb}{1.000000,1.000000,1.000000}%
\pgfsetstrokecolor{currentstroke}%
\pgfsetdash{}{0pt}%
\pgfpathmoveto{\pgfqpoint{3.096293in}{0.440000in}}%
\pgfpathlineto{\pgfqpoint{3.096293in}{3.520000in}}%
\pgfusepath{stroke}%
\end{pgfscope}%
\begin{pgfscope}%
\definecolor{textcolor}{rgb}{0.150000,0.150000,0.150000}%
\pgfsetstrokecolor{textcolor}%
\pgfsetfillcolor{textcolor}%
\pgftext[x=3.096293in,y=0.342778in,,top]{\color{textcolor}\rmfamily\fontsize{10.000000}{12.000000}\selectfont \(\displaystyle {400}\)}%
\end{pgfscope}%
\begin{pgfscope}%
\pgfpathrectangle{\pgfqpoint{0.875000in}{0.440000in}}{\pgfqpoint{5.425000in}{3.080000in}}%
\pgfusepath{clip}%
\pgfsetroundcap%
\pgfsetroundjoin%
\pgfsetlinewidth{1.003750pt}%
\definecolor{currentstroke}{rgb}{1.000000,1.000000,1.000000}%
\pgfsetstrokecolor{currentstroke}%
\pgfsetdash{}{0pt}%
\pgfpathmoveto{\pgfqpoint{4.083644in}{0.440000in}}%
\pgfpathlineto{\pgfqpoint{4.083644in}{3.520000in}}%
\pgfusepath{stroke}%
\end{pgfscope}%
\begin{pgfscope}%
\definecolor{textcolor}{rgb}{0.150000,0.150000,0.150000}%
\pgfsetstrokecolor{textcolor}%
\pgfsetfillcolor{textcolor}%
\pgftext[x=4.083644in,y=0.342778in,,top]{\color{textcolor}\rmfamily\fontsize{10.000000}{12.000000}\selectfont \(\displaystyle {600}\)}%
\end{pgfscope}%
\begin{pgfscope}%
\pgfpathrectangle{\pgfqpoint{0.875000in}{0.440000in}}{\pgfqpoint{5.425000in}{3.080000in}}%
\pgfusepath{clip}%
\pgfsetroundcap%
\pgfsetroundjoin%
\pgfsetlinewidth{1.003750pt}%
\definecolor{currentstroke}{rgb}{1.000000,1.000000,1.000000}%
\pgfsetstrokecolor{currentstroke}%
\pgfsetdash{}{0pt}%
\pgfpathmoveto{\pgfqpoint{5.070995in}{0.440000in}}%
\pgfpathlineto{\pgfqpoint{5.070995in}{3.520000in}}%
\pgfusepath{stroke}%
\end{pgfscope}%
\begin{pgfscope}%
\definecolor{textcolor}{rgb}{0.150000,0.150000,0.150000}%
\pgfsetstrokecolor{textcolor}%
\pgfsetfillcolor{textcolor}%
\pgftext[x=5.070995in,y=0.342778in,,top]{\color{textcolor}\rmfamily\fontsize{10.000000}{12.000000}\selectfont \(\displaystyle {800}\)}%
\end{pgfscope}%
\begin{pgfscope}%
\pgfpathrectangle{\pgfqpoint{0.875000in}{0.440000in}}{\pgfqpoint{5.425000in}{3.080000in}}%
\pgfusepath{clip}%
\pgfsetroundcap%
\pgfsetroundjoin%
\pgfsetlinewidth{1.003750pt}%
\definecolor{currentstroke}{rgb}{1.000000,1.000000,1.000000}%
\pgfsetstrokecolor{currentstroke}%
\pgfsetdash{}{0pt}%
\pgfpathmoveto{\pgfqpoint{6.058346in}{0.440000in}}%
\pgfpathlineto{\pgfqpoint{6.058346in}{3.520000in}}%
\pgfusepath{stroke}%
\end{pgfscope}%
\begin{pgfscope}%
\definecolor{textcolor}{rgb}{0.150000,0.150000,0.150000}%
\pgfsetstrokecolor{textcolor}%
\pgfsetfillcolor{textcolor}%
\pgftext[x=6.058346in,y=0.342778in,,top]{\color{textcolor}\rmfamily\fontsize{10.000000}{12.000000}\selectfont \(\displaystyle {1000}\)}%
\end{pgfscope}%
\begin{pgfscope}%
\definecolor{textcolor}{rgb}{0.150000,0.150000,0.150000}%
\pgfsetstrokecolor{textcolor}%
\pgfsetfillcolor{textcolor}%
\pgftext[x=3.587500in,y=0.163766in,,top]{\color{textcolor}\rmfamily\fontsize{11.000000}{13.200000}\selectfont Paso de la cadena (\(\displaystyle t\))}%
\end{pgfscope}%
\begin{pgfscope}%
\pgfpathrectangle{\pgfqpoint{0.875000in}{0.440000in}}{\pgfqpoint{5.425000in}{3.080000in}}%
\pgfusepath{clip}%
\pgfsetroundcap%
\pgfsetroundjoin%
\pgfsetlinewidth{1.003750pt}%
\definecolor{currentstroke}{rgb}{1.000000,1.000000,1.000000}%
\pgfsetstrokecolor{currentstroke}%
\pgfsetdash{}{0pt}%
\pgfpathmoveto{\pgfqpoint{0.875000in}{0.843787in}}%
\pgfpathlineto{\pgfqpoint{6.300000in}{0.843787in}}%
\pgfusepath{stroke}%
\end{pgfscope}%
\begin{pgfscope}%
\definecolor{textcolor}{rgb}{0.150000,0.150000,0.150000}%
\pgfsetstrokecolor{textcolor}%
\pgfsetfillcolor{textcolor}%
\pgftext[x=0.600308in, y=0.795562in, left, base]{\color{textcolor}\rmfamily\fontsize{10.000000}{12.000000}\selectfont \(\displaystyle {0.1}\)}%
\end{pgfscope}%
\begin{pgfscope}%
\pgfpathrectangle{\pgfqpoint{0.875000in}{0.440000in}}{\pgfqpoint{5.425000in}{3.080000in}}%
\pgfusepath{clip}%
\pgfsetroundcap%
\pgfsetroundjoin%
\pgfsetlinewidth{1.003750pt}%
\definecolor{currentstroke}{rgb}{1.000000,1.000000,1.000000}%
\pgfsetstrokecolor{currentstroke}%
\pgfsetdash{}{0pt}%
\pgfpathmoveto{\pgfqpoint{0.875000in}{1.485634in}}%
\pgfpathlineto{\pgfqpoint{6.300000in}{1.485634in}}%
\pgfusepath{stroke}%
\end{pgfscope}%
\begin{pgfscope}%
\definecolor{textcolor}{rgb}{0.150000,0.150000,0.150000}%
\pgfsetstrokecolor{textcolor}%
\pgfsetfillcolor{textcolor}%
\pgftext[x=0.600308in, y=1.437408in, left, base]{\color{textcolor}\rmfamily\fontsize{10.000000}{12.000000}\selectfont \(\displaystyle {0.2}\)}%
\end{pgfscope}%
\begin{pgfscope}%
\pgfpathrectangle{\pgfqpoint{0.875000in}{0.440000in}}{\pgfqpoint{5.425000in}{3.080000in}}%
\pgfusepath{clip}%
\pgfsetroundcap%
\pgfsetroundjoin%
\pgfsetlinewidth{1.003750pt}%
\definecolor{currentstroke}{rgb}{1.000000,1.000000,1.000000}%
\pgfsetstrokecolor{currentstroke}%
\pgfsetdash{}{0pt}%
\pgfpathmoveto{\pgfqpoint{0.875000in}{2.127480in}}%
\pgfpathlineto{\pgfqpoint{6.300000in}{2.127480in}}%
\pgfusepath{stroke}%
\end{pgfscope}%
\begin{pgfscope}%
\definecolor{textcolor}{rgb}{0.150000,0.150000,0.150000}%
\pgfsetstrokecolor{textcolor}%
\pgfsetfillcolor{textcolor}%
\pgftext[x=0.600308in, y=2.079255in, left, base]{\color{textcolor}\rmfamily\fontsize{10.000000}{12.000000}\selectfont \(\displaystyle {0.3}\)}%
\end{pgfscope}%
\begin{pgfscope}%
\pgfpathrectangle{\pgfqpoint{0.875000in}{0.440000in}}{\pgfqpoint{5.425000in}{3.080000in}}%
\pgfusepath{clip}%
\pgfsetroundcap%
\pgfsetroundjoin%
\pgfsetlinewidth{1.003750pt}%
\definecolor{currentstroke}{rgb}{1.000000,1.000000,1.000000}%
\pgfsetstrokecolor{currentstroke}%
\pgfsetdash{}{0pt}%
\pgfpathmoveto{\pgfqpoint{0.875000in}{2.769326in}}%
\pgfpathlineto{\pgfqpoint{6.300000in}{2.769326in}}%
\pgfusepath{stroke}%
\end{pgfscope}%
\begin{pgfscope}%
\definecolor{textcolor}{rgb}{0.150000,0.150000,0.150000}%
\pgfsetstrokecolor{textcolor}%
\pgfsetfillcolor{textcolor}%
\pgftext[x=0.600308in, y=2.721101in, left, base]{\color{textcolor}\rmfamily\fontsize{10.000000}{12.000000}\selectfont \(\displaystyle {0.4}\)}%
\end{pgfscope}%
\begin{pgfscope}%
\pgfpathrectangle{\pgfqpoint{0.875000in}{0.440000in}}{\pgfqpoint{5.425000in}{3.080000in}}%
\pgfusepath{clip}%
\pgfsetroundcap%
\pgfsetroundjoin%
\pgfsetlinewidth{1.003750pt}%
\definecolor{currentstroke}{rgb}{1.000000,1.000000,1.000000}%
\pgfsetstrokecolor{currentstroke}%
\pgfsetdash{}{0pt}%
\pgfpathmoveto{\pgfqpoint{0.875000in}{3.411172in}}%
\pgfpathlineto{\pgfqpoint{6.300000in}{3.411172in}}%
\pgfusepath{stroke}%
\end{pgfscope}%
\begin{pgfscope}%
\definecolor{textcolor}{rgb}{0.150000,0.150000,0.150000}%
\pgfsetstrokecolor{textcolor}%
\pgfsetfillcolor{textcolor}%
\pgftext[x=0.600308in, y=3.362947in, left, base]{\color{textcolor}\rmfamily\fontsize{10.000000}{12.000000}\selectfont \(\displaystyle {0.5}\)}%
\end{pgfscope}%
\begin{pgfscope}%
\definecolor{textcolor}{rgb}{0.150000,0.150000,0.150000}%
\pgfsetstrokecolor{textcolor}%
\pgfsetfillcolor{textcolor}%
\pgftext[x=0.544752in,y=1.980000in,,bottom,rotate=90.000000]{\color{textcolor}\rmfamily\fontsize{11.000000}{13.200000}\selectfont Valor de \(\displaystyle X_t\)}%
\end{pgfscope}%
\begin{pgfscope}%
\pgfpathrectangle{\pgfqpoint{0.875000in}{0.440000in}}{\pgfqpoint{5.425000in}{3.080000in}}%
\pgfusepath{clip}%
\pgfsetroundcap%
\pgfsetroundjoin%
\pgfsetlinewidth{1.756562pt}%
\definecolor{currentstroke}{rgb}{0.298039,0.447059,0.690196}%
\pgfsetstrokecolor{currentstroke}%
\pgfsetdash{}{0pt}%
\pgfpathmoveto{\pgfqpoint{1.121591in}{0.944839in}}%
\pgfpathlineto{\pgfqpoint{1.126528in}{2.124810in}}%
\pgfpathlineto{\pgfqpoint{1.136401in}{2.124810in}}%
\pgfpathlineto{\pgfqpoint{1.141338in}{1.826940in}}%
\pgfpathlineto{\pgfqpoint{1.170958in}{1.826940in}}%
\pgfpathlineto{\pgfqpoint{1.175895in}{1.459800in}}%
\pgfpathlineto{\pgfqpoint{1.180832in}{1.459800in}}%
\pgfpathlineto{\pgfqpoint{1.185769in}{1.083435in}}%
\pgfpathlineto{\pgfqpoint{1.190705in}{1.582009in}}%
\pgfpathlineto{\pgfqpoint{1.195642in}{2.251158in}}%
\pgfpathlineto{\pgfqpoint{1.205516in}{2.251158in}}%
\pgfpathlineto{\pgfqpoint{1.210452in}{2.950482in}}%
\pgfpathlineto{\pgfqpoint{1.215389in}{1.548269in}}%
\pgfpathlineto{\pgfqpoint{1.220326in}{2.620519in}}%
\pgfpathlineto{\pgfqpoint{1.225263in}{2.472263in}}%
\pgfpathlineto{\pgfqpoint{1.230200in}{2.162936in}}%
\pgfpathlineto{\pgfqpoint{1.235136in}{3.257906in}}%
\pgfpathlineto{\pgfqpoint{1.254883in}{3.257906in}}%
\pgfpathlineto{\pgfqpoint{1.259820in}{0.982638in}}%
\pgfpathlineto{\pgfqpoint{1.264757in}{0.982638in}}%
\pgfpathlineto{\pgfqpoint{1.269694in}{3.380000in}}%
\pgfpathlineto{\pgfqpoint{1.274630in}{3.077277in}}%
\pgfpathlineto{\pgfqpoint{1.279567in}{2.145187in}}%
\pgfpathlineto{\pgfqpoint{1.284504in}{2.508429in}}%
\pgfpathlineto{\pgfqpoint{1.289441in}{2.508429in}}%
\pgfpathlineto{\pgfqpoint{1.294377in}{2.588136in}}%
\pgfpathlineto{\pgfqpoint{1.299314in}{1.811404in}}%
\pgfpathlineto{\pgfqpoint{1.304251in}{3.172038in}}%
\pgfpathlineto{\pgfqpoint{1.309188in}{2.320897in}}%
\pgfpathlineto{\pgfqpoint{1.323998in}{2.320897in}}%
\pgfpathlineto{\pgfqpoint{1.328935in}{2.106560in}}%
\pgfpathlineto{\pgfqpoint{1.333871in}{1.051021in}}%
\pgfpathlineto{\pgfqpoint{1.338808in}{1.051021in}}%
\pgfpathlineto{\pgfqpoint{1.343745in}{1.598278in}}%
\pgfpathlineto{\pgfqpoint{1.348682in}{2.378609in}}%
\pgfpathlineto{\pgfqpoint{1.353618in}{2.378609in}}%
\pgfpathlineto{\pgfqpoint{1.358555in}{1.772309in}}%
\pgfpathlineto{\pgfqpoint{1.363492in}{2.659888in}}%
\pgfpathlineto{\pgfqpoint{1.373365in}{2.659888in}}%
\pgfpathlineto{\pgfqpoint{1.378302in}{2.852121in}}%
\pgfpathlineto{\pgfqpoint{1.383239in}{2.852121in}}%
\pgfpathlineto{\pgfqpoint{1.388176in}{2.103526in}}%
\pgfpathlineto{\pgfqpoint{1.393112in}{2.558660in}}%
\pgfpathlineto{\pgfqpoint{1.398049in}{2.558660in}}%
\pgfpathlineto{\pgfqpoint{1.402986in}{2.414505in}}%
\pgfpathlineto{\pgfqpoint{1.407923in}{2.414505in}}%
\pgfpathlineto{\pgfqpoint{1.412859in}{1.122735in}}%
\pgfpathlineto{\pgfqpoint{1.417796in}{2.849695in}}%
\pgfpathlineto{\pgfqpoint{1.427670in}{2.849695in}}%
\pgfpathlineto{\pgfqpoint{1.432606in}{2.398271in}}%
\pgfpathlineto{\pgfqpoint{1.447417in}{2.398271in}}%
\pgfpathlineto{\pgfqpoint{1.452353in}{1.502528in}}%
\pgfpathlineto{\pgfqpoint{1.457290in}{1.502528in}}%
\pgfpathlineto{\pgfqpoint{1.462227in}{1.875515in}}%
\pgfpathlineto{\pgfqpoint{1.467164in}{1.483068in}}%
\pgfpathlineto{\pgfqpoint{1.472101in}{2.562394in}}%
\pgfpathlineto{\pgfqpoint{1.477037in}{1.349338in}}%
\pgfpathlineto{\pgfqpoint{1.481974in}{2.929063in}}%
\pgfpathlineto{\pgfqpoint{1.486911in}{1.579806in}}%
\pgfpathlineto{\pgfqpoint{1.491848in}{1.579806in}}%
\pgfpathlineto{\pgfqpoint{1.496784in}{2.328498in}}%
\pgfpathlineto{\pgfqpoint{1.501721in}{2.328498in}}%
\pgfpathlineto{\pgfqpoint{1.506658in}{3.248312in}}%
\pgfpathlineto{\pgfqpoint{1.511595in}{2.292082in}}%
\pgfpathlineto{\pgfqpoint{1.516531in}{2.292082in}}%
\pgfpathlineto{\pgfqpoint{1.521468in}{2.356052in}}%
\pgfpathlineto{\pgfqpoint{1.541215in}{2.356052in}}%
\pgfpathlineto{\pgfqpoint{1.546152in}{0.748671in}}%
\pgfpathlineto{\pgfqpoint{1.551089in}{3.290148in}}%
\pgfpathlineto{\pgfqpoint{1.556025in}{1.110666in}}%
\pgfpathlineto{\pgfqpoint{1.560962in}{1.825544in}}%
\pgfpathlineto{\pgfqpoint{1.565899in}{0.973773in}}%
\pgfpathlineto{\pgfqpoint{1.570836in}{0.973773in}}%
\pgfpathlineto{\pgfqpoint{1.575772in}{3.098478in}}%
\pgfpathlineto{\pgfqpoint{1.580709in}{3.098478in}}%
\pgfpathlineto{\pgfqpoint{1.585646in}{2.961594in}}%
\pgfpathlineto{\pgfqpoint{1.590583in}{2.961594in}}%
\pgfpathlineto{\pgfqpoint{1.595519in}{1.900579in}}%
\pgfpathlineto{\pgfqpoint{1.600456in}{2.960584in}}%
\pgfpathlineto{\pgfqpoint{1.610330in}{2.960584in}}%
\pgfpathlineto{\pgfqpoint{1.615266in}{1.476701in}}%
\pgfpathlineto{\pgfqpoint{1.620203in}{1.661379in}}%
\pgfpathlineto{\pgfqpoint{1.625140in}{2.913402in}}%
\pgfpathlineto{\pgfqpoint{1.630077in}{1.430600in}}%
\pgfpathlineto{\pgfqpoint{1.635013in}{3.030863in}}%
\pgfpathlineto{\pgfqpoint{1.639950in}{3.030863in}}%
\pgfpathlineto{\pgfqpoint{1.644887in}{2.802083in}}%
\pgfpathlineto{\pgfqpoint{1.664634in}{2.802083in}}%
\pgfpathlineto{\pgfqpoint{1.669571in}{2.069407in}}%
\pgfpathlineto{\pgfqpoint{1.674507in}{2.069407in}}%
\pgfpathlineto{\pgfqpoint{1.679444in}{3.367428in}}%
\pgfpathlineto{\pgfqpoint{1.684381in}{3.367428in}}%
\pgfpathlineto{\pgfqpoint{1.689318in}{1.838093in}}%
\pgfpathlineto{\pgfqpoint{1.694254in}{1.838093in}}%
\pgfpathlineto{\pgfqpoint{1.699191in}{1.120046in}}%
\pgfpathlineto{\pgfqpoint{1.704128in}{1.120046in}}%
\pgfpathlineto{\pgfqpoint{1.709065in}{3.109811in}}%
\pgfpathlineto{\pgfqpoint{1.714002in}{2.164469in}}%
\pgfpathlineto{\pgfqpoint{1.723875in}{2.164469in}}%
\pgfpathlineto{\pgfqpoint{1.728812in}{2.748188in}}%
\pgfpathlineto{\pgfqpoint{1.733749in}{2.411941in}}%
\pgfpathlineto{\pgfqpoint{1.738685in}{2.411941in}}%
\pgfpathlineto{\pgfqpoint{1.743622in}{1.564989in}}%
\pgfpathlineto{\pgfqpoint{1.748559in}{2.889110in}}%
\pgfpathlineto{\pgfqpoint{1.753496in}{2.619216in}}%
\pgfpathlineto{\pgfqpoint{1.778179in}{2.619216in}}%
\pgfpathlineto{\pgfqpoint{1.783116in}{2.102375in}}%
\pgfpathlineto{\pgfqpoint{1.788053in}{2.520055in}}%
\pgfpathlineto{\pgfqpoint{1.792990in}{2.126108in}}%
\pgfpathlineto{\pgfqpoint{1.797926in}{2.038530in}}%
\pgfpathlineto{\pgfqpoint{1.807800in}{2.038530in}}%
\pgfpathlineto{\pgfqpoint{1.812737in}{1.914402in}}%
\pgfpathlineto{\pgfqpoint{1.817673in}{2.883869in}}%
\pgfpathlineto{\pgfqpoint{1.822610in}{2.883869in}}%
\pgfpathlineto{\pgfqpoint{1.827547in}{1.812843in}}%
\pgfpathlineto{\pgfqpoint{1.832484in}{1.812843in}}%
\pgfpathlineto{\pgfqpoint{1.837420in}{2.573998in}}%
\pgfpathlineto{\pgfqpoint{1.847294in}{2.573998in}}%
\pgfpathlineto{\pgfqpoint{1.852231in}{2.365312in}}%
\pgfpathlineto{\pgfqpoint{1.857167in}{2.365312in}}%
\pgfpathlineto{\pgfqpoint{1.862104in}{1.601070in}}%
\pgfpathlineto{\pgfqpoint{1.867041in}{2.107255in}}%
\pgfpathlineto{\pgfqpoint{1.871978in}{3.049828in}}%
\pgfpathlineto{\pgfqpoint{1.876914in}{1.678653in}}%
\pgfpathlineto{\pgfqpoint{1.881851in}{1.678653in}}%
\pgfpathlineto{\pgfqpoint{1.886788in}{2.889129in}}%
\pgfpathlineto{\pgfqpoint{1.891725in}{1.081201in}}%
\pgfpathlineto{\pgfqpoint{1.896661in}{1.353640in}}%
\pgfpathlineto{\pgfqpoint{1.901598in}{1.353640in}}%
\pgfpathlineto{\pgfqpoint{1.906535in}{3.084173in}}%
\pgfpathlineto{\pgfqpoint{1.911472in}{2.656244in}}%
\pgfpathlineto{\pgfqpoint{1.916408in}{2.606067in}}%
\pgfpathlineto{\pgfqpoint{1.921345in}{1.368829in}}%
\pgfpathlineto{\pgfqpoint{1.926282in}{3.082123in}}%
\pgfpathlineto{\pgfqpoint{1.936155in}{3.082123in}}%
\pgfpathlineto{\pgfqpoint{1.941092in}{2.431009in}}%
\pgfpathlineto{\pgfqpoint{1.960839in}{2.431009in}}%
\pgfpathlineto{\pgfqpoint{1.965776in}{2.357808in}}%
\pgfpathlineto{\pgfqpoint{1.970713in}{2.357808in}}%
\pgfpathlineto{\pgfqpoint{1.975650in}{1.513986in}}%
\pgfpathlineto{\pgfqpoint{1.980586in}{2.711357in}}%
\pgfpathlineto{\pgfqpoint{1.985523in}{1.514255in}}%
\pgfpathlineto{\pgfqpoint{1.990460in}{1.514255in}}%
\pgfpathlineto{\pgfqpoint{1.995397in}{2.653570in}}%
\pgfpathlineto{\pgfqpoint{2.005270in}{2.653570in}}%
\pgfpathlineto{\pgfqpoint{2.010207in}{2.179575in}}%
\pgfpathlineto{\pgfqpoint{2.025017in}{2.179575in}}%
\pgfpathlineto{\pgfqpoint{2.029954in}{2.553138in}}%
\pgfpathlineto{\pgfqpoint{2.039827in}{2.553138in}}%
\pgfpathlineto{\pgfqpoint{2.044764in}{2.623038in}}%
\pgfpathlineto{\pgfqpoint{2.049701in}{2.787503in}}%
\pgfpathlineto{\pgfqpoint{2.069448in}{2.787503in}}%
\pgfpathlineto{\pgfqpoint{2.074385in}{2.183213in}}%
\pgfpathlineto{\pgfqpoint{2.079321in}{2.183213in}}%
\pgfpathlineto{\pgfqpoint{2.084258in}{1.966933in}}%
\pgfpathlineto{\pgfqpoint{2.089195in}{2.383082in}}%
\pgfpathlineto{\pgfqpoint{2.104005in}{2.383082in}}%
\pgfpathlineto{\pgfqpoint{2.108942in}{2.849335in}}%
\pgfpathlineto{\pgfqpoint{2.113879in}{3.021973in}}%
\pgfpathlineto{\pgfqpoint{2.118815in}{3.021973in}}%
\pgfpathlineto{\pgfqpoint{2.123752in}{2.422761in}}%
\pgfpathlineto{\pgfqpoint{2.143499in}{2.422761in}}%
\pgfpathlineto{\pgfqpoint{2.148436in}{2.994015in}}%
\pgfpathlineto{\pgfqpoint{2.153373in}{2.952071in}}%
\pgfpathlineto{\pgfqpoint{2.168183in}{2.952071in}}%
\pgfpathlineto{\pgfqpoint{2.173120in}{1.795276in}}%
\pgfpathlineto{\pgfqpoint{2.182993in}{1.795276in}}%
\pgfpathlineto{\pgfqpoint{2.187930in}{2.468429in}}%
\pgfpathlineto{\pgfqpoint{2.202740in}{2.468429in}}%
\pgfpathlineto{\pgfqpoint{2.207677in}{2.777667in}}%
\pgfpathlineto{\pgfqpoint{2.212614in}{2.694302in}}%
\pgfpathlineto{\pgfqpoint{2.232361in}{2.694302in}}%
\pgfpathlineto{\pgfqpoint{2.237298in}{2.529471in}}%
\pgfpathlineto{\pgfqpoint{2.242234in}{1.579888in}}%
\pgfpathlineto{\pgfqpoint{2.247171in}{1.579888in}}%
\pgfpathlineto{\pgfqpoint{2.252108in}{1.024519in}}%
\pgfpathlineto{\pgfqpoint{2.257045in}{2.600999in}}%
\pgfpathlineto{\pgfqpoint{2.266918in}{1.194336in}}%
\pgfpathlineto{\pgfqpoint{2.271855in}{2.598342in}}%
\pgfpathlineto{\pgfqpoint{2.276792in}{2.598342in}}%
\pgfpathlineto{\pgfqpoint{2.281728in}{1.137625in}}%
\pgfpathlineto{\pgfqpoint{2.286665in}{1.634445in}}%
\pgfpathlineto{\pgfqpoint{2.291602in}{1.634445in}}%
\pgfpathlineto{\pgfqpoint{2.296539in}{1.245651in}}%
\pgfpathlineto{\pgfqpoint{2.301475in}{2.304727in}}%
\pgfpathlineto{\pgfqpoint{2.306412in}{2.391708in}}%
\pgfpathlineto{\pgfqpoint{2.311349in}{2.391708in}}%
\pgfpathlineto{\pgfqpoint{2.316286in}{2.739200in}}%
\pgfpathlineto{\pgfqpoint{2.331096in}{2.739200in}}%
\pgfpathlineto{\pgfqpoint{2.336033in}{1.895378in}}%
\pgfpathlineto{\pgfqpoint{2.340969in}{2.185261in}}%
\pgfpathlineto{\pgfqpoint{2.345906in}{2.372634in}}%
\pgfpathlineto{\pgfqpoint{2.355780in}{2.372634in}}%
\pgfpathlineto{\pgfqpoint{2.360716in}{2.601753in}}%
\pgfpathlineto{\pgfqpoint{2.365653in}{2.367604in}}%
\pgfpathlineto{\pgfqpoint{2.370590in}{1.775949in}}%
\pgfpathlineto{\pgfqpoint{2.375527in}{1.775949in}}%
\pgfpathlineto{\pgfqpoint{2.380463in}{1.546380in}}%
\pgfpathlineto{\pgfqpoint{2.390337in}{1.546380in}}%
\pgfpathlineto{\pgfqpoint{2.395274in}{1.872151in}}%
\pgfpathlineto{\pgfqpoint{2.405147in}{1.872151in}}%
\pgfpathlineto{\pgfqpoint{2.410084in}{1.659912in}}%
\pgfpathlineto{\pgfqpoint{2.415021in}{2.711377in}}%
\pgfpathlineto{\pgfqpoint{2.419957in}{2.108115in}}%
\pgfpathlineto{\pgfqpoint{2.424894in}{2.108115in}}%
\pgfpathlineto{\pgfqpoint{2.429831in}{1.687295in}}%
\pgfpathlineto{\pgfqpoint{2.439704in}{1.687295in}}%
\pgfpathlineto{\pgfqpoint{2.444641in}{1.663514in}}%
\pgfpathlineto{\pgfqpoint{2.449578in}{2.709431in}}%
\pgfpathlineto{\pgfqpoint{2.454515in}{1.841210in}}%
\pgfpathlineto{\pgfqpoint{2.459451in}{2.678285in}}%
\pgfpathlineto{\pgfqpoint{2.464388in}{3.105198in}}%
\pgfpathlineto{\pgfqpoint{2.469325in}{1.224323in}}%
\pgfpathlineto{\pgfqpoint{2.474262in}{2.577344in}}%
\pgfpathlineto{\pgfqpoint{2.479199in}{2.929358in}}%
\pgfpathlineto{\pgfqpoint{2.484135in}{2.929358in}}%
\pgfpathlineto{\pgfqpoint{2.489072in}{1.610931in}}%
\pgfpathlineto{\pgfqpoint{2.494009in}{1.610931in}}%
\pgfpathlineto{\pgfqpoint{2.498946in}{2.455588in}}%
\pgfpathlineto{\pgfqpoint{2.508819in}{2.455588in}}%
\pgfpathlineto{\pgfqpoint{2.513756in}{2.664561in}}%
\pgfpathlineto{\pgfqpoint{2.518693in}{2.205191in}}%
\pgfpathlineto{\pgfqpoint{2.523629in}{2.930522in}}%
\pgfpathlineto{\pgfqpoint{2.528566in}{2.930522in}}%
\pgfpathlineto{\pgfqpoint{2.533503in}{2.755342in}}%
\pgfpathlineto{\pgfqpoint{2.538440in}{2.755342in}}%
\pgfpathlineto{\pgfqpoint{2.543376in}{1.090167in}}%
\pgfpathlineto{\pgfqpoint{2.548313in}{1.090167in}}%
\pgfpathlineto{\pgfqpoint{2.553250in}{1.283175in}}%
\pgfpathlineto{\pgfqpoint{2.558187in}{0.733235in}}%
\pgfpathlineto{\pgfqpoint{2.563123in}{1.329975in}}%
\pgfpathlineto{\pgfqpoint{2.568060in}{3.019941in}}%
\pgfpathlineto{\pgfqpoint{2.577934in}{3.019941in}}%
\pgfpathlineto{\pgfqpoint{2.582870in}{2.954521in}}%
\pgfpathlineto{\pgfqpoint{2.587807in}{2.612055in}}%
\pgfpathlineto{\pgfqpoint{2.592744in}{2.612055in}}%
\pgfpathlineto{\pgfqpoint{2.597681in}{2.053678in}}%
\pgfpathlineto{\pgfqpoint{2.602617in}{2.258846in}}%
\pgfpathlineto{\pgfqpoint{2.607554in}{2.650007in}}%
\pgfpathlineto{\pgfqpoint{2.612491in}{2.650007in}}%
\pgfpathlineto{\pgfqpoint{2.617428in}{1.471327in}}%
\pgfpathlineto{\pgfqpoint{2.627301in}{2.440520in}}%
\pgfpathlineto{\pgfqpoint{2.637175in}{2.440520in}}%
\pgfpathlineto{\pgfqpoint{2.642111in}{2.524064in}}%
\pgfpathlineto{\pgfqpoint{2.651985in}{2.524064in}}%
\pgfpathlineto{\pgfqpoint{2.656922in}{2.246769in}}%
\pgfpathlineto{\pgfqpoint{2.666795in}{2.246769in}}%
\pgfpathlineto{\pgfqpoint{2.671732in}{2.051166in}}%
\pgfpathlineto{\pgfqpoint{2.681605in}{2.051166in}}%
\pgfpathlineto{\pgfqpoint{2.686542in}{2.729587in}}%
\pgfpathlineto{\pgfqpoint{2.691479in}{2.905922in}}%
\pgfpathlineto{\pgfqpoint{2.696416in}{1.752398in}}%
\pgfpathlineto{\pgfqpoint{2.701352in}{1.662193in}}%
\pgfpathlineto{\pgfqpoint{2.706289in}{2.331543in}}%
\pgfpathlineto{\pgfqpoint{2.711226in}{1.232604in}}%
\pgfpathlineto{\pgfqpoint{2.716163in}{2.477343in}}%
\pgfpathlineto{\pgfqpoint{2.721100in}{1.699912in}}%
\pgfpathlineto{\pgfqpoint{2.730973in}{1.699912in}}%
\pgfpathlineto{\pgfqpoint{2.735910in}{1.348004in}}%
\pgfpathlineto{\pgfqpoint{2.740847in}{1.603774in}}%
\pgfpathlineto{\pgfqpoint{2.745783in}{1.682633in}}%
\pgfpathlineto{\pgfqpoint{2.750720in}{2.334715in}}%
\pgfpathlineto{\pgfqpoint{2.760594in}{2.334715in}}%
\pgfpathlineto{\pgfqpoint{2.765530in}{2.426875in}}%
\pgfpathlineto{\pgfqpoint{2.770467in}{1.197822in}}%
\pgfpathlineto{\pgfqpoint{2.775404in}{1.197822in}}%
\pgfpathlineto{\pgfqpoint{2.780341in}{1.530102in}}%
\pgfpathlineto{\pgfqpoint{2.785277in}{3.007162in}}%
\pgfpathlineto{\pgfqpoint{2.790214in}{1.385261in}}%
\pgfpathlineto{\pgfqpoint{2.795151in}{2.963940in}}%
\pgfpathlineto{\pgfqpoint{2.800088in}{2.963940in}}%
\pgfpathlineto{\pgfqpoint{2.809961in}{1.453770in}}%
\pgfpathlineto{\pgfqpoint{2.814898in}{1.453770in}}%
\pgfpathlineto{\pgfqpoint{2.819835in}{2.781448in}}%
\pgfpathlineto{\pgfqpoint{2.829708in}{2.781448in}}%
\pgfpathlineto{\pgfqpoint{2.834645in}{3.078529in}}%
\pgfpathlineto{\pgfqpoint{2.839582in}{1.226251in}}%
\pgfpathlineto{\pgfqpoint{2.844518in}{1.487970in}}%
\pgfpathlineto{\pgfqpoint{2.849455in}{0.907322in}}%
\pgfpathlineto{\pgfqpoint{2.854392in}{2.376146in}}%
\pgfpathlineto{\pgfqpoint{2.864265in}{2.376146in}}%
\pgfpathlineto{\pgfqpoint{2.869202in}{2.690520in}}%
\pgfpathlineto{\pgfqpoint{2.874139in}{2.630295in}}%
\pgfpathlineto{\pgfqpoint{2.898823in}{2.630295in}}%
\pgfpathlineto{\pgfqpoint{2.903759in}{2.594343in}}%
\pgfpathlineto{\pgfqpoint{2.908696in}{2.864472in}}%
\pgfpathlineto{\pgfqpoint{2.913633in}{2.864472in}}%
\pgfpathlineto{\pgfqpoint{2.918570in}{1.887928in}}%
\pgfpathlineto{\pgfqpoint{2.928443in}{1.887928in}}%
\pgfpathlineto{\pgfqpoint{2.933380in}{2.675532in}}%
\pgfpathlineto{\pgfqpoint{2.938317in}{2.908061in}}%
\pgfpathlineto{\pgfqpoint{2.943253in}{0.835617in}}%
\pgfpathlineto{\pgfqpoint{2.948190in}{0.835617in}}%
\pgfpathlineto{\pgfqpoint{2.953127in}{1.218633in}}%
\pgfpathlineto{\pgfqpoint{2.958064in}{2.841699in}}%
\pgfpathlineto{\pgfqpoint{2.963001in}{2.379223in}}%
\pgfpathlineto{\pgfqpoint{2.967937in}{2.379223in}}%
\pgfpathlineto{\pgfqpoint{2.972874in}{2.922715in}}%
\pgfpathlineto{\pgfqpoint{2.977811in}{2.740323in}}%
\pgfpathlineto{\pgfqpoint{2.987684in}{2.740323in}}%
\pgfpathlineto{\pgfqpoint{2.992621in}{2.369596in}}%
\pgfpathlineto{\pgfqpoint{3.002495in}{2.369596in}}%
\pgfpathlineto{\pgfqpoint{3.007431in}{2.762761in}}%
\pgfpathlineto{\pgfqpoint{3.012368in}{2.762761in}}%
\pgfpathlineto{\pgfqpoint{3.017305in}{1.882107in}}%
\pgfpathlineto{\pgfqpoint{3.022242in}{1.882107in}}%
\pgfpathlineto{\pgfqpoint{3.027178in}{1.422257in}}%
\pgfpathlineto{\pgfqpoint{3.032115in}{3.205478in}}%
\pgfpathlineto{\pgfqpoint{3.037052in}{1.851670in}}%
\pgfpathlineto{\pgfqpoint{3.041989in}{2.200931in}}%
\pgfpathlineto{\pgfqpoint{3.046925in}{2.100923in}}%
\pgfpathlineto{\pgfqpoint{3.051862in}{2.569131in}}%
\pgfpathlineto{\pgfqpoint{3.056799in}{2.582903in}}%
\pgfpathlineto{\pgfqpoint{3.061736in}{1.924926in}}%
\pgfpathlineto{\pgfqpoint{3.066672in}{1.924926in}}%
\pgfpathlineto{\pgfqpoint{3.071609in}{2.434712in}}%
\pgfpathlineto{\pgfqpoint{3.076546in}{0.646231in}}%
\pgfpathlineto{\pgfqpoint{3.081483in}{3.141585in}}%
\pgfpathlineto{\pgfqpoint{3.086419in}{1.499891in}}%
\pgfpathlineto{\pgfqpoint{3.091356in}{0.909149in}}%
\pgfpathlineto{\pgfqpoint{3.096293in}{2.937006in}}%
\pgfpathlineto{\pgfqpoint{3.101230in}{2.226593in}}%
\pgfpathlineto{\pgfqpoint{3.106166in}{2.226593in}}%
\pgfpathlineto{\pgfqpoint{3.111103in}{2.322069in}}%
\pgfpathlineto{\pgfqpoint{3.145660in}{2.322069in}}%
\pgfpathlineto{\pgfqpoint{3.150597in}{1.996903in}}%
\pgfpathlineto{\pgfqpoint{3.165407in}{1.996903in}}%
\pgfpathlineto{\pgfqpoint{3.170344in}{2.526971in}}%
\pgfpathlineto{\pgfqpoint{3.175281in}{1.936126in}}%
\pgfpathlineto{\pgfqpoint{3.180218in}{2.639439in}}%
\pgfpathlineto{\pgfqpoint{3.185154in}{2.269506in}}%
\pgfpathlineto{\pgfqpoint{3.190091in}{2.269506in}}%
\pgfpathlineto{\pgfqpoint{3.195028in}{1.164661in}}%
\pgfpathlineto{\pgfqpoint{3.199965in}{2.774609in}}%
\pgfpathlineto{\pgfqpoint{3.209838in}{2.774609in}}%
\pgfpathlineto{\pgfqpoint{3.214775in}{3.069262in}}%
\pgfpathlineto{\pgfqpoint{3.224649in}{1.105922in}}%
\pgfpathlineto{\pgfqpoint{3.229585in}{1.399262in}}%
\pgfpathlineto{\pgfqpoint{3.234522in}{1.999649in}}%
\pgfpathlineto{\pgfqpoint{3.239459in}{1.999649in}}%
\pgfpathlineto{\pgfqpoint{3.244396in}{2.794486in}}%
\pgfpathlineto{\pgfqpoint{3.249332in}{2.132013in}}%
\pgfpathlineto{\pgfqpoint{3.254269in}{2.774510in}}%
\pgfpathlineto{\pgfqpoint{3.269079in}{2.774510in}}%
\pgfpathlineto{\pgfqpoint{3.278953in}{2.250636in}}%
\pgfpathlineto{\pgfqpoint{3.283890in}{2.250636in}}%
\pgfpathlineto{\pgfqpoint{3.288826in}{1.162537in}}%
\pgfpathlineto{\pgfqpoint{3.293763in}{1.387469in}}%
\pgfpathlineto{\pgfqpoint{3.298700in}{2.946298in}}%
\pgfpathlineto{\pgfqpoint{3.303637in}{2.783210in}}%
\pgfpathlineto{\pgfqpoint{3.308573in}{2.783210in}}%
\pgfpathlineto{\pgfqpoint{3.313510in}{2.674873in}}%
\pgfpathlineto{\pgfqpoint{3.318447in}{2.865731in}}%
\pgfpathlineto{\pgfqpoint{3.328320in}{2.865731in}}%
\pgfpathlineto{\pgfqpoint{3.333257in}{1.683813in}}%
\pgfpathlineto{\pgfqpoint{3.338194in}{1.162013in}}%
\pgfpathlineto{\pgfqpoint{3.353004in}{1.162013in}}%
\pgfpathlineto{\pgfqpoint{3.357941in}{1.984471in}}%
\pgfpathlineto{\pgfqpoint{3.367814in}{1.984471in}}%
\pgfpathlineto{\pgfqpoint{3.372751in}{2.734656in}}%
\pgfpathlineto{\pgfqpoint{3.377688in}{2.734656in}}%
\pgfpathlineto{\pgfqpoint{3.382625in}{2.664513in}}%
\pgfpathlineto{\pgfqpoint{3.397435in}{2.664513in}}%
\pgfpathlineto{\pgfqpoint{3.402372in}{2.443077in}}%
\pgfpathlineto{\pgfqpoint{3.407308in}{1.771068in}}%
\pgfpathlineto{\pgfqpoint{3.417182in}{1.771068in}}%
\pgfpathlineto{\pgfqpoint{3.422119in}{2.243284in}}%
\pgfpathlineto{\pgfqpoint{3.436929in}{2.243284in}}%
\pgfpathlineto{\pgfqpoint{3.441866in}{1.740283in}}%
\pgfpathlineto{\pgfqpoint{3.446802in}{2.821512in}}%
\pgfpathlineto{\pgfqpoint{3.451739in}{2.537290in}}%
\pgfpathlineto{\pgfqpoint{3.456676in}{2.457398in}}%
\pgfpathlineto{\pgfqpoint{3.461613in}{2.171156in}}%
\pgfpathlineto{\pgfqpoint{3.476423in}{2.171156in}}%
\pgfpathlineto{\pgfqpoint{3.481360in}{2.487029in}}%
\pgfpathlineto{\pgfqpoint{3.486297in}{2.633343in}}%
\pgfpathlineto{\pgfqpoint{3.501107in}{2.633343in}}%
\pgfpathlineto{\pgfqpoint{3.506044in}{1.839629in}}%
\pgfpathlineto{\pgfqpoint{3.510980in}{2.486965in}}%
\pgfpathlineto{\pgfqpoint{3.515917in}{2.486965in}}%
\pgfpathlineto{\pgfqpoint{3.520854in}{2.443042in}}%
\pgfpathlineto{\pgfqpoint{3.525791in}{1.309688in}}%
\pgfpathlineto{\pgfqpoint{3.540601in}{1.309688in}}%
\pgfpathlineto{\pgfqpoint{3.545538in}{1.765569in}}%
\pgfpathlineto{\pgfqpoint{3.560348in}{1.765569in}}%
\pgfpathlineto{\pgfqpoint{3.565285in}{1.666730in}}%
\pgfpathlineto{\pgfqpoint{3.570221in}{1.991323in}}%
\pgfpathlineto{\pgfqpoint{3.575158in}{0.862911in}}%
\pgfpathlineto{\pgfqpoint{3.580095in}{2.388472in}}%
\pgfpathlineto{\pgfqpoint{3.585032in}{2.089266in}}%
\pgfpathlineto{\pgfqpoint{3.589968in}{2.903250in}}%
\pgfpathlineto{\pgfqpoint{3.594905in}{1.569650in}}%
\pgfpathlineto{\pgfqpoint{3.599842in}{1.759340in}}%
\pgfpathlineto{\pgfqpoint{3.604779in}{2.924996in}}%
\pgfpathlineto{\pgfqpoint{3.609715in}{2.693962in}}%
\pgfpathlineto{\pgfqpoint{3.614652in}{1.125503in}}%
\pgfpathlineto{\pgfqpoint{3.619589in}{1.121113in}}%
\pgfpathlineto{\pgfqpoint{3.624526in}{1.121113in}}%
\pgfpathlineto{\pgfqpoint{3.629462in}{1.496866in}}%
\pgfpathlineto{\pgfqpoint{3.634399in}{1.496866in}}%
\pgfpathlineto{\pgfqpoint{3.639336in}{1.458569in}}%
\pgfpathlineto{\pgfqpoint{3.644273in}{1.458569in}}%
\pgfpathlineto{\pgfqpoint{3.649209in}{0.743024in}}%
\pgfpathlineto{\pgfqpoint{3.654146in}{3.094573in}}%
\pgfpathlineto{\pgfqpoint{3.659083in}{3.195784in}}%
\pgfpathlineto{\pgfqpoint{3.664020in}{3.089382in}}%
\pgfpathlineto{\pgfqpoint{3.668956in}{1.989150in}}%
\pgfpathlineto{\pgfqpoint{3.688703in}{1.989150in}}%
\pgfpathlineto{\pgfqpoint{3.693640in}{3.161008in}}%
\pgfpathlineto{\pgfqpoint{3.698577in}{2.994831in}}%
\pgfpathlineto{\pgfqpoint{3.703514in}{1.253563in}}%
\pgfpathlineto{\pgfqpoint{3.708450in}{1.253563in}}%
\pgfpathlineto{\pgfqpoint{3.713387in}{0.911365in}}%
\pgfpathlineto{\pgfqpoint{3.718324in}{0.910195in}}%
\pgfpathlineto{\pgfqpoint{3.723261in}{1.514050in}}%
\pgfpathlineto{\pgfqpoint{3.728198in}{1.144392in}}%
\pgfpathlineto{\pgfqpoint{3.733134in}{1.144392in}}%
\pgfpathlineto{\pgfqpoint{3.738071in}{3.118985in}}%
\pgfpathlineto{\pgfqpoint{3.743008in}{3.084950in}}%
\pgfpathlineto{\pgfqpoint{3.747945in}{3.084950in}}%
\pgfpathlineto{\pgfqpoint{3.752881in}{2.819357in}}%
\pgfpathlineto{\pgfqpoint{3.762755in}{2.819357in}}%
\pgfpathlineto{\pgfqpoint{3.767692in}{2.350828in}}%
\pgfpathlineto{\pgfqpoint{3.772628in}{1.081021in}}%
\pgfpathlineto{\pgfqpoint{3.777565in}{0.876244in}}%
\pgfpathlineto{\pgfqpoint{3.782502in}{3.097638in}}%
\pgfpathlineto{\pgfqpoint{3.787439in}{1.495005in}}%
\pgfpathlineto{\pgfqpoint{3.792375in}{1.537377in}}%
\pgfpathlineto{\pgfqpoint{3.802249in}{2.791179in}}%
\pgfpathlineto{\pgfqpoint{3.807186in}{2.528252in}}%
\pgfpathlineto{\pgfqpoint{3.812122in}{2.498161in}}%
\pgfpathlineto{\pgfqpoint{3.856553in}{2.498161in}}%
\pgfpathlineto{\pgfqpoint{3.861490in}{1.662194in}}%
\pgfpathlineto{\pgfqpoint{3.866427in}{1.141784in}}%
\pgfpathlineto{\pgfqpoint{3.876300in}{1.141784in}}%
\pgfpathlineto{\pgfqpoint{3.881237in}{2.950274in}}%
\pgfpathlineto{\pgfqpoint{3.886174in}{2.922359in}}%
\pgfpathlineto{\pgfqpoint{3.891110in}{0.692784in}}%
\pgfpathlineto{\pgfqpoint{3.896047in}{3.073164in}}%
\pgfpathlineto{\pgfqpoint{3.900984in}{2.321883in}}%
\pgfpathlineto{\pgfqpoint{3.905921in}{2.402399in}}%
\pgfpathlineto{\pgfqpoint{3.915794in}{2.402399in}}%
\pgfpathlineto{\pgfqpoint{3.920731in}{1.429957in}}%
\pgfpathlineto{\pgfqpoint{3.925668in}{2.770852in}}%
\pgfpathlineto{\pgfqpoint{3.930604in}{2.770852in}}%
\pgfpathlineto{\pgfqpoint{3.935541in}{1.685927in}}%
\pgfpathlineto{\pgfqpoint{3.940478in}{2.192798in}}%
\pgfpathlineto{\pgfqpoint{3.945415in}{2.546436in}}%
\pgfpathlineto{\pgfqpoint{3.950351in}{3.069382in}}%
\pgfpathlineto{\pgfqpoint{3.955288in}{3.069382in}}%
\pgfpathlineto{\pgfqpoint{3.960225in}{2.770572in}}%
\pgfpathlineto{\pgfqpoint{3.965162in}{2.770572in}}%
\pgfpathlineto{\pgfqpoint{3.970099in}{2.125799in}}%
\pgfpathlineto{\pgfqpoint{3.975035in}{2.542669in}}%
\pgfpathlineto{\pgfqpoint{3.979972in}{2.542669in}}%
\pgfpathlineto{\pgfqpoint{3.984909in}{1.797725in}}%
\pgfpathlineto{\pgfqpoint{3.989846in}{3.027905in}}%
\pgfpathlineto{\pgfqpoint{3.994782in}{2.358422in}}%
\pgfpathlineto{\pgfqpoint{3.999719in}{2.358422in}}%
\pgfpathlineto{\pgfqpoint{4.004656in}{2.526276in}}%
\pgfpathlineto{\pgfqpoint{4.009593in}{2.737974in}}%
\pgfpathlineto{\pgfqpoint{4.024403in}{2.737974in}}%
\pgfpathlineto{\pgfqpoint{4.029340in}{3.083879in}}%
\pgfpathlineto{\pgfqpoint{4.034276in}{1.656074in}}%
\pgfpathlineto{\pgfqpoint{4.039213in}{1.656074in}}%
\pgfpathlineto{\pgfqpoint{4.044150in}{1.759472in}}%
\pgfpathlineto{\pgfqpoint{4.049087in}{2.205604in}}%
\pgfpathlineto{\pgfqpoint{4.054023in}{2.917505in}}%
\pgfpathlineto{\pgfqpoint{4.068834in}{2.917505in}}%
\pgfpathlineto{\pgfqpoint{4.073770in}{2.250131in}}%
\pgfpathlineto{\pgfqpoint{4.083644in}{2.250131in}}%
\pgfpathlineto{\pgfqpoint{4.088581in}{1.908714in}}%
\pgfpathlineto{\pgfqpoint{4.093517in}{3.088882in}}%
\pgfpathlineto{\pgfqpoint{4.103391in}{1.427589in}}%
\pgfpathlineto{\pgfqpoint{4.108328in}{1.948506in}}%
\pgfpathlineto{\pgfqpoint{4.113264in}{2.150680in}}%
\pgfpathlineto{\pgfqpoint{4.133011in}{2.150680in}}%
\pgfpathlineto{\pgfqpoint{4.137948in}{2.472871in}}%
\pgfpathlineto{\pgfqpoint{4.142885in}{1.928304in}}%
\pgfpathlineto{\pgfqpoint{4.147822in}{3.179806in}}%
\pgfpathlineto{\pgfqpoint{4.152758in}{3.179806in}}%
\pgfpathlineto{\pgfqpoint{4.157695in}{1.636535in}}%
\pgfpathlineto{\pgfqpoint{4.162632in}{1.636535in}}%
\pgfpathlineto{\pgfqpoint{4.167569in}{3.203661in}}%
\pgfpathlineto{\pgfqpoint{4.172505in}{3.203661in}}%
\pgfpathlineto{\pgfqpoint{4.177442in}{1.321082in}}%
\pgfpathlineto{\pgfqpoint{4.182379in}{2.773534in}}%
\pgfpathlineto{\pgfqpoint{4.187316in}{2.623843in}}%
\pgfpathlineto{\pgfqpoint{4.197189in}{2.623843in}}%
\pgfpathlineto{\pgfqpoint{4.202126in}{1.574542in}}%
\pgfpathlineto{\pgfqpoint{4.207063in}{1.310919in}}%
\pgfpathlineto{\pgfqpoint{4.211999in}{0.969714in}}%
\pgfpathlineto{\pgfqpoint{4.216936in}{3.113037in}}%
\pgfpathlineto{\pgfqpoint{4.221873in}{2.030165in}}%
\pgfpathlineto{\pgfqpoint{4.231747in}{1.636548in}}%
\pgfpathlineto{\pgfqpoint{4.236683in}{1.223065in}}%
\pgfpathlineto{\pgfqpoint{4.241620in}{1.223065in}}%
\pgfpathlineto{\pgfqpoint{4.251494in}{3.132433in}}%
\pgfpathlineto{\pgfqpoint{4.256430in}{3.132433in}}%
\pgfpathlineto{\pgfqpoint{4.261367in}{2.871203in}}%
\pgfpathlineto{\pgfqpoint{4.266304in}{2.871203in}}%
\pgfpathlineto{\pgfqpoint{4.271241in}{1.778265in}}%
\pgfpathlineto{\pgfqpoint{4.276177in}{1.455547in}}%
\pgfpathlineto{\pgfqpoint{4.281114in}{3.167946in}}%
\pgfpathlineto{\pgfqpoint{4.286051in}{3.167946in}}%
\pgfpathlineto{\pgfqpoint{4.290988in}{1.387262in}}%
\pgfpathlineto{\pgfqpoint{4.295924in}{1.140978in}}%
\pgfpathlineto{\pgfqpoint{4.300861in}{2.247660in}}%
\pgfpathlineto{\pgfqpoint{4.305798in}{0.912220in}}%
\pgfpathlineto{\pgfqpoint{4.310735in}{3.097467in}}%
\pgfpathlineto{\pgfqpoint{4.315671in}{1.772133in}}%
\pgfpathlineto{\pgfqpoint{4.320608in}{2.710301in}}%
\pgfpathlineto{\pgfqpoint{4.325545in}{2.131967in}}%
\pgfpathlineto{\pgfqpoint{4.330482in}{2.813627in}}%
\pgfpathlineto{\pgfqpoint{4.335418in}{2.395356in}}%
\pgfpathlineto{\pgfqpoint{4.340355in}{0.700212in}}%
\pgfpathlineto{\pgfqpoint{4.345292in}{0.703412in}}%
\pgfpathlineto{\pgfqpoint{4.350229in}{1.043274in}}%
\pgfpathlineto{\pgfqpoint{4.355165in}{2.314286in}}%
\pgfpathlineto{\pgfqpoint{4.374912in}{2.314286in}}%
\pgfpathlineto{\pgfqpoint{4.379849in}{1.997774in}}%
\pgfpathlineto{\pgfqpoint{4.389723in}{1.997774in}}%
\pgfpathlineto{\pgfqpoint{4.394659in}{0.783471in}}%
\pgfpathlineto{\pgfqpoint{4.399596in}{0.752906in}}%
\pgfpathlineto{\pgfqpoint{4.404533in}{2.770541in}}%
\pgfpathlineto{\pgfqpoint{4.409470in}{1.830471in}}%
\pgfpathlineto{\pgfqpoint{4.414406in}{3.029080in}}%
\pgfpathlineto{\pgfqpoint{4.419343in}{3.029080in}}%
\pgfpathlineto{\pgfqpoint{4.424280in}{3.208766in}}%
\pgfpathlineto{\pgfqpoint{4.429217in}{2.699786in}}%
\pgfpathlineto{\pgfqpoint{4.434153in}{2.699786in}}%
\pgfpathlineto{\pgfqpoint{4.439090in}{2.991022in}}%
\pgfpathlineto{\pgfqpoint{4.444027in}{2.313134in}}%
\pgfpathlineto{\pgfqpoint{4.453900in}{2.313134in}}%
\pgfpathlineto{\pgfqpoint{4.458837in}{3.145289in}}%
\pgfpathlineto{\pgfqpoint{4.463774in}{1.860252in}}%
\pgfpathlineto{\pgfqpoint{4.468711in}{1.893278in}}%
\pgfpathlineto{\pgfqpoint{4.473648in}{2.046028in}}%
\pgfpathlineto{\pgfqpoint{4.478584in}{2.547910in}}%
\pgfpathlineto{\pgfqpoint{4.483521in}{2.547910in}}%
\pgfpathlineto{\pgfqpoint{4.488458in}{2.624180in}}%
\pgfpathlineto{\pgfqpoint{4.493395in}{1.521459in}}%
\pgfpathlineto{\pgfqpoint{4.503268in}{1.521459in}}%
\pgfpathlineto{\pgfqpoint{4.508205in}{2.686128in}}%
\pgfpathlineto{\pgfqpoint{4.513142in}{2.517027in}}%
\pgfpathlineto{\pgfqpoint{4.518078in}{1.825318in}}%
\pgfpathlineto{\pgfqpoint{4.523015in}{2.577219in}}%
\pgfpathlineto{\pgfqpoint{4.532889in}{2.577219in}}%
\pgfpathlineto{\pgfqpoint{4.537825in}{2.754739in}}%
\pgfpathlineto{\pgfqpoint{4.542762in}{1.308892in}}%
\pgfpathlineto{\pgfqpoint{4.547699in}{1.308892in}}%
\pgfpathlineto{\pgfqpoint{4.557572in}{2.796782in}}%
\pgfpathlineto{\pgfqpoint{4.562509in}{2.796782in}}%
\pgfpathlineto{\pgfqpoint{4.567446in}{1.878213in}}%
\pgfpathlineto{\pgfqpoint{4.572383in}{2.946633in}}%
\pgfpathlineto{\pgfqpoint{4.577319in}{1.989364in}}%
\pgfpathlineto{\pgfqpoint{4.582256in}{2.961532in}}%
\pgfpathlineto{\pgfqpoint{4.587193in}{1.970155in}}%
\pgfpathlineto{\pgfqpoint{4.592130in}{2.718357in}}%
\pgfpathlineto{\pgfqpoint{4.606940in}{2.718357in}}%
\pgfpathlineto{\pgfqpoint{4.611877in}{2.691540in}}%
\pgfpathlineto{\pgfqpoint{4.626687in}{2.691540in}}%
\pgfpathlineto{\pgfqpoint{4.631624in}{2.606551in}}%
\pgfpathlineto{\pgfqpoint{4.636560in}{2.606551in}}%
\pgfpathlineto{\pgfqpoint{4.641497in}{1.556380in}}%
\pgfpathlineto{\pgfqpoint{4.646434in}{0.903305in}}%
\pgfpathlineto{\pgfqpoint{4.651371in}{2.458479in}}%
\pgfpathlineto{\pgfqpoint{4.656307in}{1.971775in}}%
\pgfpathlineto{\pgfqpoint{4.666181in}{1.971775in}}%
\pgfpathlineto{\pgfqpoint{4.671118in}{2.810448in}}%
\pgfpathlineto{\pgfqpoint{4.685928in}{2.810448in}}%
\pgfpathlineto{\pgfqpoint{4.690865in}{2.337771in}}%
\pgfpathlineto{\pgfqpoint{4.695801in}{2.648612in}}%
\pgfpathlineto{\pgfqpoint{4.700738in}{1.572499in}}%
\pgfpathlineto{\pgfqpoint{4.705675in}{2.866433in}}%
\pgfpathlineto{\pgfqpoint{4.710612in}{2.089405in}}%
\pgfpathlineto{\pgfqpoint{4.715549in}{2.003273in}}%
\pgfpathlineto{\pgfqpoint{4.720485in}{3.127230in}}%
\pgfpathlineto{\pgfqpoint{4.725422in}{3.104066in}}%
\pgfpathlineto{\pgfqpoint{4.730359in}{0.881056in}}%
\pgfpathlineto{\pgfqpoint{4.735296in}{0.746393in}}%
\pgfpathlineto{\pgfqpoint{4.740232in}{1.199819in}}%
\pgfpathlineto{\pgfqpoint{4.745169in}{1.985500in}}%
\pgfpathlineto{\pgfqpoint{4.750106in}{2.573662in}}%
\pgfpathlineto{\pgfqpoint{4.789600in}{2.573662in}}%
\pgfpathlineto{\pgfqpoint{4.794537in}{2.586983in}}%
\pgfpathlineto{\pgfqpoint{4.804410in}{2.586983in}}%
\pgfpathlineto{\pgfqpoint{4.809347in}{2.697640in}}%
\pgfpathlineto{\pgfqpoint{4.814284in}{1.540935in}}%
\pgfpathlineto{\pgfqpoint{4.819220in}{3.148364in}}%
\pgfpathlineto{\pgfqpoint{4.824157in}{1.533108in}}%
\pgfpathlineto{\pgfqpoint{4.829094in}{2.181661in}}%
\pgfpathlineto{\pgfqpoint{4.834031in}{2.493952in}}%
\pgfpathlineto{\pgfqpoint{4.838967in}{2.493952in}}%
\pgfpathlineto{\pgfqpoint{4.843904in}{2.032743in}}%
\pgfpathlineto{\pgfqpoint{4.858714in}{2.032743in}}%
\pgfpathlineto{\pgfqpoint{4.863651in}{2.002856in}}%
\pgfpathlineto{\pgfqpoint{4.868588in}{2.235564in}}%
\pgfpathlineto{\pgfqpoint{4.873525in}{2.235564in}}%
\pgfpathlineto{\pgfqpoint{4.878461in}{1.997687in}}%
\pgfpathlineto{\pgfqpoint{4.883398in}{3.347252in}}%
\pgfpathlineto{\pgfqpoint{4.893272in}{3.347252in}}%
\pgfpathlineto{\pgfqpoint{4.898208in}{0.954062in}}%
\pgfpathlineto{\pgfqpoint{4.903145in}{0.954062in}}%
\pgfpathlineto{\pgfqpoint{4.908082in}{1.812917in}}%
\pgfpathlineto{\pgfqpoint{4.913019in}{3.043303in}}%
\pgfpathlineto{\pgfqpoint{4.917955in}{3.043303in}}%
\pgfpathlineto{\pgfqpoint{4.922892in}{3.234426in}}%
\pgfpathlineto{\pgfqpoint{4.927829in}{2.123768in}}%
\pgfpathlineto{\pgfqpoint{4.932766in}{1.427969in}}%
\pgfpathlineto{\pgfqpoint{4.937702in}{2.053349in}}%
\pgfpathlineto{\pgfqpoint{4.952513in}{2.053349in}}%
\pgfpathlineto{\pgfqpoint{4.957449in}{1.492452in}}%
\pgfpathlineto{\pgfqpoint{4.962386in}{1.683094in}}%
\pgfpathlineto{\pgfqpoint{4.967323in}{1.683094in}}%
\pgfpathlineto{\pgfqpoint{4.972260in}{1.707227in}}%
\pgfpathlineto{\pgfqpoint{4.977197in}{2.401056in}}%
\pgfpathlineto{\pgfqpoint{5.001880in}{2.401056in}}%
\pgfpathlineto{\pgfqpoint{5.006817in}{2.843938in}}%
\pgfpathlineto{\pgfqpoint{5.011754in}{1.644138in}}%
\pgfpathlineto{\pgfqpoint{5.016691in}{1.599597in}}%
\pgfpathlineto{\pgfqpoint{5.021627in}{2.378984in}}%
\pgfpathlineto{\pgfqpoint{5.036438in}{2.378984in}}%
\pgfpathlineto{\pgfqpoint{5.041374in}{1.744931in}}%
\pgfpathlineto{\pgfqpoint{5.046311in}{1.744931in}}%
\pgfpathlineto{\pgfqpoint{5.051248in}{2.043769in}}%
\pgfpathlineto{\pgfqpoint{5.056185in}{2.043769in}}%
\pgfpathlineto{\pgfqpoint{5.061121in}{2.392262in}}%
\pgfpathlineto{\pgfqpoint{5.066058in}{2.392262in}}%
\pgfpathlineto{\pgfqpoint{5.070995in}{2.087877in}}%
\pgfpathlineto{\pgfqpoint{5.075932in}{2.561962in}}%
\pgfpathlineto{\pgfqpoint{5.080868in}{2.561962in}}%
\pgfpathlineto{\pgfqpoint{5.085805in}{1.735723in}}%
\pgfpathlineto{\pgfqpoint{5.090742in}{1.832602in}}%
\pgfpathlineto{\pgfqpoint{5.095679in}{3.149596in}}%
\pgfpathlineto{\pgfqpoint{5.100615in}{3.149596in}}%
\pgfpathlineto{\pgfqpoint{5.105552in}{1.423209in}}%
\pgfpathlineto{\pgfqpoint{5.110489in}{3.038813in}}%
\pgfpathlineto{\pgfqpoint{5.115426in}{2.717400in}}%
\pgfpathlineto{\pgfqpoint{5.125299in}{1.569518in}}%
\pgfpathlineto{\pgfqpoint{5.130236in}{1.569518in}}%
\pgfpathlineto{\pgfqpoint{5.135173in}{3.216519in}}%
\pgfpathlineto{\pgfqpoint{5.140109in}{3.081964in}}%
\pgfpathlineto{\pgfqpoint{5.145046in}{2.466560in}}%
\pgfpathlineto{\pgfqpoint{5.149983in}{3.044690in}}%
\pgfpathlineto{\pgfqpoint{5.154920in}{1.973781in}}%
\pgfpathlineto{\pgfqpoint{5.164793in}{1.973781in}}%
\pgfpathlineto{\pgfqpoint{5.169730in}{2.856808in}}%
\pgfpathlineto{\pgfqpoint{5.179603in}{2.856808in}}%
\pgfpathlineto{\pgfqpoint{5.184540in}{1.885420in}}%
\pgfpathlineto{\pgfqpoint{5.189477in}{1.294510in}}%
\pgfpathlineto{\pgfqpoint{5.194414in}{1.294510in}}%
\pgfpathlineto{\pgfqpoint{5.199350in}{1.177801in}}%
\pgfpathlineto{\pgfqpoint{5.204287in}{1.122998in}}%
\pgfpathlineto{\pgfqpoint{5.209224in}{1.904372in}}%
\pgfpathlineto{\pgfqpoint{5.214161in}{1.904372in}}%
\pgfpathlineto{\pgfqpoint{5.219098in}{1.955107in}}%
\pgfpathlineto{\pgfqpoint{5.224034in}{1.301373in}}%
\pgfpathlineto{\pgfqpoint{5.228971in}{1.301373in}}%
\pgfpathlineto{\pgfqpoint{5.233908in}{3.138472in}}%
\pgfpathlineto{\pgfqpoint{5.238845in}{3.138472in}}%
\pgfpathlineto{\pgfqpoint{5.243781in}{2.194024in}}%
\pgfpathlineto{\pgfqpoint{5.248718in}{2.405781in}}%
\pgfpathlineto{\pgfqpoint{5.258592in}{2.405781in}}%
\pgfpathlineto{\pgfqpoint{5.263528in}{2.559893in}}%
\pgfpathlineto{\pgfqpoint{5.268465in}{2.559893in}}%
\pgfpathlineto{\pgfqpoint{5.273402in}{2.925628in}}%
\pgfpathlineto{\pgfqpoint{5.283275in}{2.925628in}}%
\pgfpathlineto{\pgfqpoint{5.288212in}{1.968433in}}%
\pgfpathlineto{\pgfqpoint{5.293149in}{2.138571in}}%
\pgfpathlineto{\pgfqpoint{5.303022in}{2.138571in}}%
\pgfpathlineto{\pgfqpoint{5.307959in}{2.196448in}}%
\pgfpathlineto{\pgfqpoint{5.312896in}{2.196448in}}%
\pgfpathlineto{\pgfqpoint{5.317833in}{3.148866in}}%
\pgfpathlineto{\pgfqpoint{5.322769in}{2.708680in}}%
\pgfpathlineto{\pgfqpoint{5.327706in}{2.599933in}}%
\pgfpathlineto{\pgfqpoint{5.332643in}{1.782110in}}%
\pgfpathlineto{\pgfqpoint{5.337580in}{2.573206in}}%
\pgfpathlineto{\pgfqpoint{5.342516in}{2.573206in}}%
\pgfpathlineto{\pgfqpoint{5.347453in}{2.400645in}}%
\pgfpathlineto{\pgfqpoint{5.352390in}{2.331102in}}%
\pgfpathlineto{\pgfqpoint{5.357327in}{3.033955in}}%
\pgfpathlineto{\pgfqpoint{5.362263in}{3.033955in}}%
\pgfpathlineto{\pgfqpoint{5.367200in}{2.490645in}}%
\pgfpathlineto{\pgfqpoint{5.372137in}{3.343547in}}%
\pgfpathlineto{\pgfqpoint{5.377074in}{3.343547in}}%
\pgfpathlineto{\pgfqpoint{5.382010in}{1.596009in}}%
\pgfpathlineto{\pgfqpoint{5.386947in}{1.596009in}}%
\pgfpathlineto{\pgfqpoint{5.391884in}{1.219006in}}%
\pgfpathlineto{\pgfqpoint{5.396821in}{1.219006in}}%
\pgfpathlineto{\pgfqpoint{5.401757in}{1.991202in}}%
\pgfpathlineto{\pgfqpoint{5.406694in}{2.384118in}}%
\pgfpathlineto{\pgfqpoint{5.411631in}{2.195947in}}%
\pgfpathlineto{\pgfqpoint{5.416568in}{2.195947in}}%
\pgfpathlineto{\pgfqpoint{5.421504in}{2.276344in}}%
\pgfpathlineto{\pgfqpoint{5.436315in}{2.276344in}}%
\pgfpathlineto{\pgfqpoint{5.441251in}{0.580000in}}%
\pgfpathlineto{\pgfqpoint{5.446188in}{2.063661in}}%
\pgfpathlineto{\pgfqpoint{5.451125in}{2.612588in}}%
\pgfpathlineto{\pgfqpoint{5.456062in}{2.612588in}}%
\pgfpathlineto{\pgfqpoint{5.460998in}{2.146422in}}%
\pgfpathlineto{\pgfqpoint{5.470872in}{2.146422in}}%
\pgfpathlineto{\pgfqpoint{5.475809in}{2.332913in}}%
\pgfpathlineto{\pgfqpoint{5.485682in}{2.332913in}}%
\pgfpathlineto{\pgfqpoint{5.490619in}{2.065450in}}%
\pgfpathlineto{\pgfqpoint{5.495556in}{1.917392in}}%
\pgfpathlineto{\pgfqpoint{5.500493in}{1.917392in}}%
\pgfpathlineto{\pgfqpoint{5.505429in}{2.315956in}}%
\pgfpathlineto{\pgfqpoint{5.510366in}{3.227745in}}%
\pgfpathlineto{\pgfqpoint{5.515303in}{2.585283in}}%
\pgfpathlineto{\pgfqpoint{5.520240in}{2.544528in}}%
\pgfpathlineto{\pgfqpoint{5.525176in}{2.544528in}}%
\pgfpathlineto{\pgfqpoint{5.530113in}{2.736009in}}%
\pgfpathlineto{\pgfqpoint{5.535050in}{2.347958in}}%
\pgfpathlineto{\pgfqpoint{5.549860in}{2.347958in}}%
\pgfpathlineto{\pgfqpoint{5.554797in}{2.909516in}}%
\pgfpathlineto{\pgfqpoint{5.559734in}{2.909516in}}%
\pgfpathlineto{\pgfqpoint{5.564670in}{2.575756in}}%
\pgfpathlineto{\pgfqpoint{5.594291in}{2.575756in}}%
\pgfpathlineto{\pgfqpoint{5.599228in}{1.123853in}}%
\pgfpathlineto{\pgfqpoint{5.604164in}{2.950892in}}%
\pgfpathlineto{\pgfqpoint{5.623911in}{2.950892in}}%
\pgfpathlineto{\pgfqpoint{5.628848in}{1.820076in}}%
\pgfpathlineto{\pgfqpoint{5.638722in}{1.820076in}}%
\pgfpathlineto{\pgfqpoint{5.643658in}{1.961938in}}%
\pgfpathlineto{\pgfqpoint{5.653532in}{1.961938in}}%
\pgfpathlineto{\pgfqpoint{5.658469in}{2.598439in}}%
\pgfpathlineto{\pgfqpoint{5.663405in}{1.703761in}}%
\pgfpathlineto{\pgfqpoint{5.668342in}{1.703761in}}%
\pgfpathlineto{\pgfqpoint{5.673279in}{3.056649in}}%
\pgfpathlineto{\pgfqpoint{5.688089in}{3.056649in}}%
\pgfpathlineto{\pgfqpoint{5.693026in}{2.878390in}}%
\pgfpathlineto{\pgfqpoint{5.697963in}{2.015347in}}%
\pgfpathlineto{\pgfqpoint{5.702899in}{2.702340in}}%
\pgfpathlineto{\pgfqpoint{5.707836in}{2.522243in}}%
\pgfpathlineto{\pgfqpoint{5.712773in}{2.041470in}}%
\pgfpathlineto{\pgfqpoint{5.722647in}{2.041470in}}%
\pgfpathlineto{\pgfqpoint{5.727583in}{1.782290in}}%
\pgfpathlineto{\pgfqpoint{5.732520in}{1.019426in}}%
\pgfpathlineto{\pgfqpoint{5.737457in}{2.286926in}}%
\pgfpathlineto{\pgfqpoint{5.747330in}{2.286926in}}%
\pgfpathlineto{\pgfqpoint{5.752267in}{1.404362in}}%
\pgfpathlineto{\pgfqpoint{5.757204in}{1.386626in}}%
\pgfpathlineto{\pgfqpoint{5.762141in}{1.921970in}}%
\pgfpathlineto{\pgfqpoint{5.767077in}{2.021165in}}%
\pgfpathlineto{\pgfqpoint{5.781888in}{2.021165in}}%
\pgfpathlineto{\pgfqpoint{5.786824in}{2.722946in}}%
\pgfpathlineto{\pgfqpoint{5.791761in}{2.032399in}}%
\pgfpathlineto{\pgfqpoint{5.806571in}{2.032399in}}%
\pgfpathlineto{\pgfqpoint{5.811508in}{1.971291in}}%
\pgfpathlineto{\pgfqpoint{5.821382in}{1.971291in}}%
\pgfpathlineto{\pgfqpoint{5.826318in}{2.862129in}}%
\pgfpathlineto{\pgfqpoint{5.831255in}{2.418745in}}%
\pgfpathlineto{\pgfqpoint{5.851002in}{2.418745in}}%
\pgfpathlineto{\pgfqpoint{5.855939in}{1.729667in}}%
\pgfpathlineto{\pgfqpoint{5.860876in}{1.734966in}}%
\pgfpathlineto{\pgfqpoint{5.865812in}{1.560628in}}%
\pgfpathlineto{\pgfqpoint{5.870749in}{1.560628in}}%
\pgfpathlineto{\pgfqpoint{5.875686in}{2.302140in}}%
\pgfpathlineto{\pgfqpoint{5.880623in}{2.634786in}}%
\pgfpathlineto{\pgfqpoint{5.885559in}{1.277380in}}%
\pgfpathlineto{\pgfqpoint{5.890496in}{2.321011in}}%
\pgfpathlineto{\pgfqpoint{5.895433in}{2.321011in}}%
\pgfpathlineto{\pgfqpoint{5.900370in}{1.837028in}}%
\pgfpathlineto{\pgfqpoint{5.905306in}{1.837028in}}%
\pgfpathlineto{\pgfqpoint{5.910243in}{2.190769in}}%
\pgfpathlineto{\pgfqpoint{5.915180in}{2.645940in}}%
\pgfpathlineto{\pgfqpoint{5.920117in}{1.966751in}}%
\pgfpathlineto{\pgfqpoint{5.939864in}{1.966751in}}%
\pgfpathlineto{\pgfqpoint{5.944800in}{1.016678in}}%
\pgfpathlineto{\pgfqpoint{5.949737in}{1.016678in}}%
\pgfpathlineto{\pgfqpoint{5.954674in}{3.254067in}}%
\pgfpathlineto{\pgfqpoint{5.959611in}{3.254067in}}%
\pgfpathlineto{\pgfqpoint{5.964548in}{1.796756in}}%
\pgfpathlineto{\pgfqpoint{5.969484in}{1.914155in}}%
\pgfpathlineto{\pgfqpoint{5.974421in}{2.776261in}}%
\pgfpathlineto{\pgfqpoint{5.984295in}{1.760855in}}%
\pgfpathlineto{\pgfqpoint{5.989231in}{1.789344in}}%
\pgfpathlineto{\pgfqpoint{5.994168in}{1.789344in}}%
\pgfpathlineto{\pgfqpoint{5.999105in}{1.671228in}}%
\pgfpathlineto{\pgfqpoint{6.004042in}{2.254410in}}%
\pgfpathlineto{\pgfqpoint{6.023789in}{2.254410in}}%
\pgfpathlineto{\pgfqpoint{6.028725in}{2.567415in}}%
\pgfpathlineto{\pgfqpoint{6.033662in}{2.567415in}}%
\pgfpathlineto{\pgfqpoint{6.038599in}{1.610746in}}%
\pgfpathlineto{\pgfqpoint{6.043536in}{1.932373in}}%
\pgfpathlineto{\pgfqpoint{6.048472in}{1.932373in}}%
\pgfpathlineto{\pgfqpoint{6.053409in}{2.080347in}}%
\pgfpathlineto{\pgfqpoint{6.053409in}{2.080347in}}%
\pgfusepath{stroke}%
\end{pgfscope}%
\begin{pgfscope}%
\pgfsetrectcap%
\pgfsetmiterjoin%
\pgfsetlinewidth{0.000000pt}%
\definecolor{currentstroke}{rgb}{1.000000,1.000000,1.000000}%
\pgfsetstrokecolor{currentstroke}%
\pgfsetdash{}{0pt}%
\pgfpathmoveto{\pgfqpoint{0.875000in}{0.440000in}}%
\pgfpathlineto{\pgfqpoint{0.875000in}{3.520000in}}%
\pgfusepath{}%
\end{pgfscope}%
\begin{pgfscope}%
\pgfsetrectcap%
\pgfsetmiterjoin%
\pgfsetlinewidth{0.000000pt}%
\definecolor{currentstroke}{rgb}{1.000000,1.000000,1.000000}%
\pgfsetstrokecolor{currentstroke}%
\pgfsetdash{}{0pt}%
\pgfpathmoveto{\pgfqpoint{6.300000in}{0.440000in}}%
\pgfpathlineto{\pgfqpoint{6.300000in}{3.520000in}}%
\pgfusepath{}%
\end{pgfscope}%
\begin{pgfscope}%
\pgfsetrectcap%
\pgfsetmiterjoin%
\pgfsetlinewidth{0.000000pt}%
\definecolor{currentstroke}{rgb}{1.000000,1.000000,1.000000}%
\pgfsetstrokecolor{currentstroke}%
\pgfsetdash{}{0pt}%
\pgfpathmoveto{\pgfqpoint{0.875000in}{0.440000in}}%
\pgfpathlineto{\pgfqpoint{6.300000in}{0.440000in}}%
\pgfusepath{}%
\end{pgfscope}%
\begin{pgfscope}%
\pgfsetrectcap%
\pgfsetmiterjoin%
\pgfsetlinewidth{0.000000pt}%
\definecolor{currentstroke}{rgb}{1.000000,1.000000,1.000000}%
\pgfsetstrokecolor{currentstroke}%
\pgfsetdash{}{0pt}%
\pgfpathmoveto{\pgfqpoint{0.875000in}{3.520000in}}%
\pgfpathlineto{\pgfqpoint{6.300000in}{3.520000in}}%
\pgfusepath{}%
\end{pgfscope}%
\begin{pgfscope}%
\definecolor{textcolor}{rgb}{0.150000,0.150000,0.150000}%
\pgfsetstrokecolor{textcolor}%
\pgfsetfillcolor{textcolor}%
\pgftext[x=3.500000in,y=3.920000in,,top]{\color{textcolor}\rmfamily\fontsize{12.000000}{14.400000}\selectfont Muestreo con propuesta \(\displaystyle U(01)\) y \(\displaystyle n=5\)}%
\end{pgfscope}%
\end{pgfpicture}%
\makeatother%
\endgroup%


        %% Creator: Matplotlib, PGF backend
%%
%% To include the figure in your LaTeX document, write
%%   \input{<filename>.pgf}
%%
%% Make sure the required packages are loaded in your preamble
%%   \usepackage{pgf}
%%
%% Also ensure that all the required font packages are loaded; for instance,
%% the lmodern package is sometimes necessary when using math font.
%%   \usepackage{lmodern}
%%
%% Figures using additional raster images can only be included by \input if
%% they are in the same directory as the main LaTeX file. For loading figures
%% from other directories you can use the `import` package
%%   \usepackage{import}
%%
%% and then include the figures with
%%   \import{<path to file>}{<filename>.pgf}
%%
%% Matplotlib used the following preamble
%%   
%%   \makeatletter\@ifpackageloaded{underscore}{}{\usepackage[strings]{underscore}}\makeatother
%%
\begingroup%
\makeatletter%
\begin{pgfpicture}%
\pgfpathrectangle{\pgfpointorigin}{\pgfqpoint{7.000000in}{4.000000in}}%
\pgfusepath{use as bounding box, clip}%
\begin{pgfscope}%
\pgfsetbuttcap%
\pgfsetmiterjoin%
\definecolor{currentfill}{rgb}{1.000000,1.000000,1.000000}%
\pgfsetfillcolor{currentfill}%
\pgfsetlinewidth{0.000000pt}%
\definecolor{currentstroke}{rgb}{1.000000,1.000000,1.000000}%
\pgfsetstrokecolor{currentstroke}%
\pgfsetdash{}{0pt}%
\pgfpathmoveto{\pgfqpoint{0.000000in}{0.000000in}}%
\pgfpathlineto{\pgfqpoint{7.000000in}{0.000000in}}%
\pgfpathlineto{\pgfqpoint{7.000000in}{4.000000in}}%
\pgfpathlineto{\pgfqpoint{0.000000in}{4.000000in}}%
\pgfpathlineto{\pgfqpoint{0.000000in}{0.000000in}}%
\pgfpathclose%
\pgfusepath{fill}%
\end{pgfscope}%
\begin{pgfscope}%
\pgfsetbuttcap%
\pgfsetmiterjoin%
\definecolor{currentfill}{rgb}{0.917647,0.917647,0.949020}%
\pgfsetfillcolor{currentfill}%
\pgfsetlinewidth{0.000000pt}%
\definecolor{currentstroke}{rgb}{0.000000,0.000000,0.000000}%
\pgfsetstrokecolor{currentstroke}%
\pgfsetstrokeopacity{0.000000}%
\pgfsetdash{}{0pt}%
\pgfpathmoveto{\pgfqpoint{0.875000in}{0.440000in}}%
\pgfpathlineto{\pgfqpoint{6.300000in}{0.440000in}}%
\pgfpathlineto{\pgfqpoint{6.300000in}{3.520000in}}%
\pgfpathlineto{\pgfqpoint{0.875000in}{3.520000in}}%
\pgfpathlineto{\pgfqpoint{0.875000in}{0.440000in}}%
\pgfpathclose%
\pgfusepath{fill}%
\end{pgfscope}%
\begin{pgfscope}%
\pgfpathrectangle{\pgfqpoint{0.875000in}{0.440000in}}{\pgfqpoint{5.425000in}{3.080000in}}%
\pgfusepath{clip}%
\pgfsetroundcap%
\pgfsetroundjoin%
\pgfsetlinewidth{1.003750pt}%
\definecolor{currentstroke}{rgb}{1.000000,1.000000,1.000000}%
\pgfsetstrokecolor{currentstroke}%
\pgfsetdash{}{0pt}%
\pgfpathmoveto{\pgfqpoint{1.121591in}{0.440000in}}%
\pgfpathlineto{\pgfqpoint{1.121591in}{3.520000in}}%
\pgfusepath{stroke}%
\end{pgfscope}%
\begin{pgfscope}%
\definecolor{textcolor}{rgb}{0.150000,0.150000,0.150000}%
\pgfsetstrokecolor{textcolor}%
\pgfsetfillcolor{textcolor}%
\pgftext[x=1.121591in,y=0.342778in,,top]{\color{textcolor}\rmfamily\fontsize{10.000000}{12.000000}\selectfont \(\displaystyle {0}\)}%
\end{pgfscope}%
\begin{pgfscope}%
\pgfpathrectangle{\pgfqpoint{0.875000in}{0.440000in}}{\pgfqpoint{5.425000in}{3.080000in}}%
\pgfusepath{clip}%
\pgfsetroundcap%
\pgfsetroundjoin%
\pgfsetlinewidth{1.003750pt}%
\definecolor{currentstroke}{rgb}{1.000000,1.000000,1.000000}%
\pgfsetstrokecolor{currentstroke}%
\pgfsetdash{}{0pt}%
\pgfpathmoveto{\pgfqpoint{2.107955in}{0.440000in}}%
\pgfpathlineto{\pgfqpoint{2.107955in}{3.520000in}}%
\pgfusepath{stroke}%
\end{pgfscope}%
\begin{pgfscope}%
\definecolor{textcolor}{rgb}{0.150000,0.150000,0.150000}%
\pgfsetstrokecolor{textcolor}%
\pgfsetfillcolor{textcolor}%
\pgftext[x=2.107955in,y=0.342778in,,top]{\color{textcolor}\rmfamily\fontsize{10.000000}{12.000000}\selectfont \(\displaystyle {200}\)}%
\end{pgfscope}%
\begin{pgfscope}%
\pgfpathrectangle{\pgfqpoint{0.875000in}{0.440000in}}{\pgfqpoint{5.425000in}{3.080000in}}%
\pgfusepath{clip}%
\pgfsetroundcap%
\pgfsetroundjoin%
\pgfsetlinewidth{1.003750pt}%
\definecolor{currentstroke}{rgb}{1.000000,1.000000,1.000000}%
\pgfsetstrokecolor{currentstroke}%
\pgfsetdash{}{0pt}%
\pgfpathmoveto{\pgfqpoint{3.094318in}{0.440000in}}%
\pgfpathlineto{\pgfqpoint{3.094318in}{3.520000in}}%
\pgfusepath{stroke}%
\end{pgfscope}%
\begin{pgfscope}%
\definecolor{textcolor}{rgb}{0.150000,0.150000,0.150000}%
\pgfsetstrokecolor{textcolor}%
\pgfsetfillcolor{textcolor}%
\pgftext[x=3.094318in,y=0.342778in,,top]{\color{textcolor}\rmfamily\fontsize{10.000000}{12.000000}\selectfont \(\displaystyle {400}\)}%
\end{pgfscope}%
\begin{pgfscope}%
\pgfpathrectangle{\pgfqpoint{0.875000in}{0.440000in}}{\pgfqpoint{5.425000in}{3.080000in}}%
\pgfusepath{clip}%
\pgfsetroundcap%
\pgfsetroundjoin%
\pgfsetlinewidth{1.003750pt}%
\definecolor{currentstroke}{rgb}{1.000000,1.000000,1.000000}%
\pgfsetstrokecolor{currentstroke}%
\pgfsetdash{}{0pt}%
\pgfpathmoveto{\pgfqpoint{4.080682in}{0.440000in}}%
\pgfpathlineto{\pgfqpoint{4.080682in}{3.520000in}}%
\pgfusepath{stroke}%
\end{pgfscope}%
\begin{pgfscope}%
\definecolor{textcolor}{rgb}{0.150000,0.150000,0.150000}%
\pgfsetstrokecolor{textcolor}%
\pgfsetfillcolor{textcolor}%
\pgftext[x=4.080682in,y=0.342778in,,top]{\color{textcolor}\rmfamily\fontsize{10.000000}{12.000000}\selectfont \(\displaystyle {600}\)}%
\end{pgfscope}%
\begin{pgfscope}%
\pgfpathrectangle{\pgfqpoint{0.875000in}{0.440000in}}{\pgfqpoint{5.425000in}{3.080000in}}%
\pgfusepath{clip}%
\pgfsetroundcap%
\pgfsetroundjoin%
\pgfsetlinewidth{1.003750pt}%
\definecolor{currentstroke}{rgb}{1.000000,1.000000,1.000000}%
\pgfsetstrokecolor{currentstroke}%
\pgfsetdash{}{0pt}%
\pgfpathmoveto{\pgfqpoint{5.067045in}{0.440000in}}%
\pgfpathlineto{\pgfqpoint{5.067045in}{3.520000in}}%
\pgfusepath{stroke}%
\end{pgfscope}%
\begin{pgfscope}%
\definecolor{textcolor}{rgb}{0.150000,0.150000,0.150000}%
\pgfsetstrokecolor{textcolor}%
\pgfsetfillcolor{textcolor}%
\pgftext[x=5.067045in,y=0.342778in,,top]{\color{textcolor}\rmfamily\fontsize{10.000000}{12.000000}\selectfont \(\displaystyle {800}\)}%
\end{pgfscope}%
\begin{pgfscope}%
\pgfpathrectangle{\pgfqpoint{0.875000in}{0.440000in}}{\pgfqpoint{5.425000in}{3.080000in}}%
\pgfusepath{clip}%
\pgfsetroundcap%
\pgfsetroundjoin%
\pgfsetlinewidth{1.003750pt}%
\definecolor{currentstroke}{rgb}{1.000000,1.000000,1.000000}%
\pgfsetstrokecolor{currentstroke}%
\pgfsetdash{}{0pt}%
\pgfpathmoveto{\pgfqpoint{6.053409in}{0.440000in}}%
\pgfpathlineto{\pgfqpoint{6.053409in}{3.520000in}}%
\pgfusepath{stroke}%
\end{pgfscope}%
\begin{pgfscope}%
\definecolor{textcolor}{rgb}{0.150000,0.150000,0.150000}%
\pgfsetstrokecolor{textcolor}%
\pgfsetfillcolor{textcolor}%
\pgftext[x=6.053409in,y=0.342778in,,top]{\color{textcolor}\rmfamily\fontsize{10.000000}{12.000000}\selectfont \(\displaystyle {1000}\)}%
\end{pgfscope}%
\begin{pgfscope}%
\definecolor{textcolor}{rgb}{0.150000,0.150000,0.150000}%
\pgfsetstrokecolor{textcolor}%
\pgfsetfillcolor{textcolor}%
\pgftext[x=3.587500in,y=0.163766in,,top]{\color{textcolor}\rmfamily\fontsize{11.000000}{13.200000}\selectfont Paso de la cadena (\(\displaystyle t\))}%
\end{pgfscope}%
\begin{pgfscope}%
\pgfpathrectangle{\pgfqpoint{0.875000in}{0.440000in}}{\pgfqpoint{5.425000in}{3.080000in}}%
\pgfusepath{clip}%
\pgfsetroundcap%
\pgfsetroundjoin%
\pgfsetlinewidth{1.003750pt}%
\definecolor{currentstroke}{rgb}{1.000000,1.000000,1.000000}%
\pgfsetstrokecolor{currentstroke}%
\pgfsetdash{}{0pt}%
\pgfpathmoveto{\pgfqpoint{0.875000in}{0.440106in}}%
\pgfpathlineto{\pgfqpoint{6.300000in}{0.440106in}}%
\pgfusepath{stroke}%
\end{pgfscope}%
\begin{pgfscope}%
\definecolor{textcolor}{rgb}{0.150000,0.150000,0.150000}%
\pgfsetstrokecolor{textcolor}%
\pgfsetfillcolor{textcolor}%
\pgftext[x=0.600308in, y=0.391881in, left, base]{\color{textcolor}\rmfamily\fontsize{10.000000}{12.000000}\selectfont \(\displaystyle {0.0}\)}%
\end{pgfscope}%
\begin{pgfscope}%
\pgfpathrectangle{\pgfqpoint{0.875000in}{0.440000in}}{\pgfqpoint{5.425000in}{3.080000in}}%
\pgfusepath{clip}%
\pgfsetroundcap%
\pgfsetroundjoin%
\pgfsetlinewidth{1.003750pt}%
\definecolor{currentstroke}{rgb}{1.000000,1.000000,1.000000}%
\pgfsetstrokecolor{currentstroke}%
\pgfsetdash{}{0pt}%
\pgfpathmoveto{\pgfqpoint{0.875000in}{1.042889in}}%
\pgfpathlineto{\pgfqpoint{6.300000in}{1.042889in}}%
\pgfusepath{stroke}%
\end{pgfscope}%
\begin{pgfscope}%
\definecolor{textcolor}{rgb}{0.150000,0.150000,0.150000}%
\pgfsetstrokecolor{textcolor}%
\pgfsetfillcolor{textcolor}%
\pgftext[x=0.600308in, y=0.994664in, left, base]{\color{textcolor}\rmfamily\fontsize{10.000000}{12.000000}\selectfont \(\displaystyle {0.1}\)}%
\end{pgfscope}%
\begin{pgfscope}%
\pgfpathrectangle{\pgfqpoint{0.875000in}{0.440000in}}{\pgfqpoint{5.425000in}{3.080000in}}%
\pgfusepath{clip}%
\pgfsetroundcap%
\pgfsetroundjoin%
\pgfsetlinewidth{1.003750pt}%
\definecolor{currentstroke}{rgb}{1.000000,1.000000,1.000000}%
\pgfsetstrokecolor{currentstroke}%
\pgfsetdash{}{0pt}%
\pgfpathmoveto{\pgfqpoint{0.875000in}{1.645672in}}%
\pgfpathlineto{\pgfqpoint{6.300000in}{1.645672in}}%
\pgfusepath{stroke}%
\end{pgfscope}%
\begin{pgfscope}%
\definecolor{textcolor}{rgb}{0.150000,0.150000,0.150000}%
\pgfsetstrokecolor{textcolor}%
\pgfsetfillcolor{textcolor}%
\pgftext[x=0.600308in, y=1.597447in, left, base]{\color{textcolor}\rmfamily\fontsize{10.000000}{12.000000}\selectfont \(\displaystyle {0.2}\)}%
\end{pgfscope}%
\begin{pgfscope}%
\pgfpathrectangle{\pgfqpoint{0.875000in}{0.440000in}}{\pgfqpoint{5.425000in}{3.080000in}}%
\pgfusepath{clip}%
\pgfsetroundcap%
\pgfsetroundjoin%
\pgfsetlinewidth{1.003750pt}%
\definecolor{currentstroke}{rgb}{1.000000,1.000000,1.000000}%
\pgfsetstrokecolor{currentstroke}%
\pgfsetdash{}{0pt}%
\pgfpathmoveto{\pgfqpoint{0.875000in}{2.248455in}}%
\pgfpathlineto{\pgfqpoint{6.300000in}{2.248455in}}%
\pgfusepath{stroke}%
\end{pgfscope}%
\begin{pgfscope}%
\definecolor{textcolor}{rgb}{0.150000,0.150000,0.150000}%
\pgfsetstrokecolor{textcolor}%
\pgfsetfillcolor{textcolor}%
\pgftext[x=0.600308in, y=2.200229in, left, base]{\color{textcolor}\rmfamily\fontsize{10.000000}{12.000000}\selectfont \(\displaystyle {0.3}\)}%
\end{pgfscope}%
\begin{pgfscope}%
\pgfpathrectangle{\pgfqpoint{0.875000in}{0.440000in}}{\pgfqpoint{5.425000in}{3.080000in}}%
\pgfusepath{clip}%
\pgfsetroundcap%
\pgfsetroundjoin%
\pgfsetlinewidth{1.003750pt}%
\definecolor{currentstroke}{rgb}{1.000000,1.000000,1.000000}%
\pgfsetstrokecolor{currentstroke}%
\pgfsetdash{}{0pt}%
\pgfpathmoveto{\pgfqpoint{0.875000in}{2.851238in}}%
\pgfpathlineto{\pgfqpoint{6.300000in}{2.851238in}}%
\pgfusepath{stroke}%
\end{pgfscope}%
\begin{pgfscope}%
\definecolor{textcolor}{rgb}{0.150000,0.150000,0.150000}%
\pgfsetstrokecolor{textcolor}%
\pgfsetfillcolor{textcolor}%
\pgftext[x=0.600308in, y=2.803012in, left, base]{\color{textcolor}\rmfamily\fontsize{10.000000}{12.000000}\selectfont \(\displaystyle {0.4}\)}%
\end{pgfscope}%
\begin{pgfscope}%
\pgfpathrectangle{\pgfqpoint{0.875000in}{0.440000in}}{\pgfqpoint{5.425000in}{3.080000in}}%
\pgfusepath{clip}%
\pgfsetroundcap%
\pgfsetroundjoin%
\pgfsetlinewidth{1.003750pt}%
\definecolor{currentstroke}{rgb}{1.000000,1.000000,1.000000}%
\pgfsetstrokecolor{currentstroke}%
\pgfsetdash{}{0pt}%
\pgfpathmoveto{\pgfqpoint{0.875000in}{3.454020in}}%
\pgfpathlineto{\pgfqpoint{6.300000in}{3.454020in}}%
\pgfusepath{stroke}%
\end{pgfscope}%
\begin{pgfscope}%
\definecolor{textcolor}{rgb}{0.150000,0.150000,0.150000}%
\pgfsetstrokecolor{textcolor}%
\pgfsetfillcolor{textcolor}%
\pgftext[x=0.600308in, y=3.405795in, left, base]{\color{textcolor}\rmfamily\fontsize{10.000000}{12.000000}\selectfont \(\displaystyle {0.5}\)}%
\end{pgfscope}%
\begin{pgfscope}%
\definecolor{textcolor}{rgb}{0.150000,0.150000,0.150000}%
\pgfsetstrokecolor{textcolor}%
\pgfsetfillcolor{textcolor}%
\pgftext[x=0.544752in,y=1.980000in,,bottom,rotate=90.000000]{\color{textcolor}\rmfamily\fontsize{11.000000}{13.200000}\selectfont Valor de \(\displaystyle X_t\)}%
\end{pgfscope}%
\begin{pgfscope}%
\pgfpathrectangle{\pgfqpoint{0.875000in}{0.440000in}}{\pgfqpoint{5.425000in}{3.080000in}}%
\pgfusepath{clip}%
\pgfsetroundcap%
\pgfsetroundjoin%
\pgfsetlinewidth{1.756562pt}%
\definecolor{currentstroke}{rgb}{0.298039,0.447059,0.690196}%
\pgfsetstrokecolor{currentstroke}%
\pgfsetdash{}{0pt}%
\pgfpathmoveto{\pgfqpoint{1.121591in}{0.580000in}}%
\pgfpathlineto{\pgfqpoint{1.151182in}{0.580000in}}%
\pgfpathlineto{\pgfqpoint{1.156114in}{2.549526in}}%
\pgfpathlineto{\pgfqpoint{1.161045in}{2.783253in}}%
\pgfpathlineto{\pgfqpoint{1.170909in}{2.393588in}}%
\pgfpathlineto{\pgfqpoint{1.175841in}{1.712357in}}%
\pgfpathlineto{\pgfqpoint{1.180773in}{1.712357in}}%
\pgfpathlineto{\pgfqpoint{1.185705in}{2.497434in}}%
\pgfpathlineto{\pgfqpoint{1.190636in}{2.216286in}}%
\pgfpathlineto{\pgfqpoint{1.200500in}{2.216286in}}%
\pgfpathlineto{\pgfqpoint{1.205432in}{2.518446in}}%
\pgfpathlineto{\pgfqpoint{1.210364in}{2.184571in}}%
\pgfpathlineto{\pgfqpoint{1.215295in}{2.318253in}}%
\pgfpathlineto{\pgfqpoint{1.220227in}{2.318253in}}%
\pgfpathlineto{\pgfqpoint{1.225159in}{1.880322in}}%
\pgfpathlineto{\pgfqpoint{1.230091in}{2.416241in}}%
\pgfpathlineto{\pgfqpoint{1.235023in}{2.055943in}}%
\pgfpathlineto{\pgfqpoint{1.239955in}{2.055943in}}%
\pgfpathlineto{\pgfqpoint{1.244886in}{2.926076in}}%
\pgfpathlineto{\pgfqpoint{1.249818in}{2.493516in}}%
\pgfpathlineto{\pgfqpoint{1.254750in}{2.718879in}}%
\pgfpathlineto{\pgfqpoint{1.259682in}{2.718879in}}%
\pgfpathlineto{\pgfqpoint{1.264614in}{2.427086in}}%
\pgfpathlineto{\pgfqpoint{1.269545in}{2.427086in}}%
\pgfpathlineto{\pgfqpoint{1.274477in}{2.608312in}}%
\pgfpathlineto{\pgfqpoint{1.279409in}{2.853847in}}%
\pgfpathlineto{\pgfqpoint{1.284341in}{2.604921in}}%
\pgfpathlineto{\pgfqpoint{1.294205in}{2.604921in}}%
\pgfpathlineto{\pgfqpoint{1.299136in}{2.370283in}}%
\pgfpathlineto{\pgfqpoint{1.304068in}{2.409657in}}%
\pgfpathlineto{\pgfqpoint{1.309000in}{2.409657in}}%
\pgfpathlineto{\pgfqpoint{1.313932in}{2.742578in}}%
\pgfpathlineto{\pgfqpoint{1.318864in}{2.546926in}}%
\pgfpathlineto{\pgfqpoint{1.323795in}{2.546926in}}%
\pgfpathlineto{\pgfqpoint{1.328727in}{3.225070in}}%
\pgfpathlineto{\pgfqpoint{1.333659in}{2.626118in}}%
\pgfpathlineto{\pgfqpoint{1.338591in}{2.613220in}}%
\pgfpathlineto{\pgfqpoint{1.343523in}{2.684443in}}%
\pgfpathlineto{\pgfqpoint{1.348455in}{2.684443in}}%
\pgfpathlineto{\pgfqpoint{1.353386in}{2.275191in}}%
\pgfpathlineto{\pgfqpoint{1.358318in}{2.275191in}}%
\pgfpathlineto{\pgfqpoint{1.363250in}{2.776366in}}%
\pgfpathlineto{\pgfqpoint{1.368182in}{2.750384in}}%
\pgfpathlineto{\pgfqpoint{1.373114in}{2.348053in}}%
\pgfpathlineto{\pgfqpoint{1.378045in}{2.209581in}}%
\pgfpathlineto{\pgfqpoint{1.382977in}{2.120084in}}%
\pgfpathlineto{\pgfqpoint{1.387909in}{2.878993in}}%
\pgfpathlineto{\pgfqpoint{1.392841in}{2.868514in}}%
\pgfpathlineto{\pgfqpoint{1.397773in}{3.100859in}}%
\pgfpathlineto{\pgfqpoint{1.402705in}{2.294996in}}%
\pgfpathlineto{\pgfqpoint{1.407636in}{2.294996in}}%
\pgfpathlineto{\pgfqpoint{1.412568in}{1.940746in}}%
\pgfpathlineto{\pgfqpoint{1.427364in}{1.940746in}}%
\pgfpathlineto{\pgfqpoint{1.432295in}{2.518605in}}%
\pgfpathlineto{\pgfqpoint{1.437227in}{2.414564in}}%
\pgfpathlineto{\pgfqpoint{1.442159in}{2.643977in}}%
\pgfpathlineto{\pgfqpoint{1.447091in}{2.808257in}}%
\pgfpathlineto{\pgfqpoint{1.452023in}{2.767750in}}%
\pgfpathlineto{\pgfqpoint{1.456955in}{2.668204in}}%
\pgfpathlineto{\pgfqpoint{1.461886in}{2.668204in}}%
\pgfpathlineto{\pgfqpoint{1.466818in}{2.549249in}}%
\pgfpathlineto{\pgfqpoint{1.471750in}{1.874274in}}%
\pgfpathlineto{\pgfqpoint{1.501341in}{1.874274in}}%
\pgfpathlineto{\pgfqpoint{1.506273in}{2.538388in}}%
\pgfpathlineto{\pgfqpoint{1.511205in}{2.538388in}}%
\pgfpathlineto{\pgfqpoint{1.516136in}{2.343369in}}%
\pgfpathlineto{\pgfqpoint{1.521068in}{3.254157in}}%
\pgfpathlineto{\pgfqpoint{1.526000in}{2.236419in}}%
\pgfpathlineto{\pgfqpoint{1.530932in}{2.896442in}}%
\pgfpathlineto{\pgfqpoint{1.535864in}{2.645206in}}%
\pgfpathlineto{\pgfqpoint{1.540795in}{2.490851in}}%
\pgfpathlineto{\pgfqpoint{1.545727in}{2.952504in}}%
\pgfpathlineto{\pgfqpoint{1.550659in}{2.461836in}}%
\pgfpathlineto{\pgfqpoint{1.555591in}{2.630967in}}%
\pgfpathlineto{\pgfqpoint{1.560523in}{3.064241in}}%
\pgfpathlineto{\pgfqpoint{1.565455in}{3.189167in}}%
\pgfpathlineto{\pgfqpoint{1.570386in}{2.614127in}}%
\pgfpathlineto{\pgfqpoint{1.575318in}{2.176409in}}%
\pgfpathlineto{\pgfqpoint{1.595045in}{2.176409in}}%
\pgfpathlineto{\pgfqpoint{1.599977in}{2.291838in}}%
\pgfpathlineto{\pgfqpoint{1.604909in}{2.291838in}}%
\pgfpathlineto{\pgfqpoint{1.609841in}{2.400411in}}%
\pgfpathlineto{\pgfqpoint{1.614773in}{2.402819in}}%
\pgfpathlineto{\pgfqpoint{1.619705in}{2.472076in}}%
\pgfpathlineto{\pgfqpoint{1.624636in}{2.472076in}}%
\pgfpathlineto{\pgfqpoint{1.629568in}{2.394702in}}%
\pgfpathlineto{\pgfqpoint{1.634500in}{1.813085in}}%
\pgfpathlineto{\pgfqpoint{1.639432in}{1.813085in}}%
\pgfpathlineto{\pgfqpoint{1.644364in}{2.170880in}}%
\pgfpathlineto{\pgfqpoint{1.649295in}{2.170880in}}%
\pgfpathlineto{\pgfqpoint{1.654227in}{2.452928in}}%
\pgfpathlineto{\pgfqpoint{1.659159in}{3.140969in}}%
\pgfpathlineto{\pgfqpoint{1.664091in}{2.803017in}}%
\pgfpathlineto{\pgfqpoint{1.669023in}{2.934188in}}%
\pgfpathlineto{\pgfqpoint{1.673955in}{2.271408in}}%
\pgfpathlineto{\pgfqpoint{1.678886in}{2.719649in}}%
\pgfpathlineto{\pgfqpoint{1.683818in}{2.071857in}}%
\pgfpathlineto{\pgfqpoint{1.693682in}{2.071857in}}%
\pgfpathlineto{\pgfqpoint{1.698614in}{2.788525in}}%
\pgfpathlineto{\pgfqpoint{1.703545in}{2.789126in}}%
\pgfpathlineto{\pgfqpoint{1.708477in}{3.091079in}}%
\pgfpathlineto{\pgfqpoint{1.713409in}{2.167769in}}%
\pgfpathlineto{\pgfqpoint{1.718341in}{2.467291in}}%
\pgfpathlineto{\pgfqpoint{1.723273in}{2.467291in}}%
\pgfpathlineto{\pgfqpoint{1.728205in}{3.126824in}}%
\pgfpathlineto{\pgfqpoint{1.733136in}{3.126824in}}%
\pgfpathlineto{\pgfqpoint{1.738068in}{2.562077in}}%
\pgfpathlineto{\pgfqpoint{1.743000in}{2.904494in}}%
\pgfpathlineto{\pgfqpoint{1.747932in}{2.937236in}}%
\pgfpathlineto{\pgfqpoint{1.752864in}{3.117732in}}%
\pgfpathlineto{\pgfqpoint{1.757795in}{2.687488in}}%
\pgfpathlineto{\pgfqpoint{1.762727in}{2.564489in}}%
\pgfpathlineto{\pgfqpoint{1.767659in}{2.813022in}}%
\pgfpathlineto{\pgfqpoint{1.772591in}{2.580925in}}%
\pgfpathlineto{\pgfqpoint{1.777523in}{3.311526in}}%
\pgfpathlineto{\pgfqpoint{1.782455in}{3.311526in}}%
\pgfpathlineto{\pgfqpoint{1.787386in}{2.156107in}}%
\pgfpathlineto{\pgfqpoint{1.797250in}{2.156107in}}%
\pgfpathlineto{\pgfqpoint{1.802182in}{3.093592in}}%
\pgfpathlineto{\pgfqpoint{1.807114in}{2.855576in}}%
\pgfpathlineto{\pgfqpoint{1.812045in}{2.821666in}}%
\pgfpathlineto{\pgfqpoint{1.816977in}{2.729424in}}%
\pgfpathlineto{\pgfqpoint{1.826841in}{2.437507in}}%
\pgfpathlineto{\pgfqpoint{1.831773in}{2.233376in}}%
\pgfpathlineto{\pgfqpoint{1.836705in}{2.233376in}}%
\pgfpathlineto{\pgfqpoint{1.846568in}{3.200854in}}%
\pgfpathlineto{\pgfqpoint{1.851500in}{2.171975in}}%
\pgfpathlineto{\pgfqpoint{1.856432in}{3.158413in}}%
\pgfpathlineto{\pgfqpoint{1.861364in}{2.012423in}}%
\pgfpathlineto{\pgfqpoint{1.866295in}{2.012423in}}%
\pgfpathlineto{\pgfqpoint{1.871227in}{2.125123in}}%
\pgfpathlineto{\pgfqpoint{1.876159in}{2.471198in}}%
\pgfpathlineto{\pgfqpoint{1.881091in}{2.210948in}}%
\pgfpathlineto{\pgfqpoint{1.895886in}{2.210948in}}%
\pgfpathlineto{\pgfqpoint{1.900818in}{2.535696in}}%
\pgfpathlineto{\pgfqpoint{1.905750in}{2.737913in}}%
\pgfpathlineto{\pgfqpoint{1.910682in}{2.334773in}}%
\pgfpathlineto{\pgfqpoint{1.920545in}{3.027737in}}%
\pgfpathlineto{\pgfqpoint{1.925477in}{2.774286in}}%
\pgfpathlineto{\pgfqpoint{1.930409in}{2.846957in}}%
\pgfpathlineto{\pgfqpoint{1.935341in}{2.072958in}}%
\pgfpathlineto{\pgfqpoint{1.940273in}{1.886015in}}%
\pgfpathlineto{\pgfqpoint{1.945205in}{2.668358in}}%
\pgfpathlineto{\pgfqpoint{1.950136in}{2.858954in}}%
\pgfpathlineto{\pgfqpoint{1.955068in}{2.858954in}}%
\pgfpathlineto{\pgfqpoint{1.960000in}{2.832739in}}%
\pgfpathlineto{\pgfqpoint{1.964932in}{2.583548in}}%
\pgfpathlineto{\pgfqpoint{1.969864in}{2.583548in}}%
\pgfpathlineto{\pgfqpoint{1.974795in}{3.048077in}}%
\pgfpathlineto{\pgfqpoint{1.979727in}{3.048077in}}%
\pgfpathlineto{\pgfqpoint{1.984659in}{2.403185in}}%
\pgfpathlineto{\pgfqpoint{1.989591in}{2.620675in}}%
\pgfpathlineto{\pgfqpoint{1.994523in}{2.674257in}}%
\pgfpathlineto{\pgfqpoint{1.999455in}{2.063005in}}%
\pgfpathlineto{\pgfqpoint{2.004386in}{2.850349in}}%
\pgfpathlineto{\pgfqpoint{2.009318in}{2.924205in}}%
\pgfpathlineto{\pgfqpoint{2.014250in}{2.204123in}}%
\pgfpathlineto{\pgfqpoint{2.019182in}{2.868563in}}%
\pgfpathlineto{\pgfqpoint{2.024114in}{2.868563in}}%
\pgfpathlineto{\pgfqpoint{2.029045in}{2.452263in}}%
\pgfpathlineto{\pgfqpoint{2.033977in}{2.452263in}}%
\pgfpathlineto{\pgfqpoint{2.038909in}{2.957055in}}%
\pgfpathlineto{\pgfqpoint{2.043841in}{2.957055in}}%
\pgfpathlineto{\pgfqpoint{2.048773in}{2.600882in}}%
\pgfpathlineto{\pgfqpoint{2.053705in}{2.490003in}}%
\pgfpathlineto{\pgfqpoint{2.058636in}{2.490003in}}%
\pgfpathlineto{\pgfqpoint{2.063568in}{2.274894in}}%
\pgfpathlineto{\pgfqpoint{2.068500in}{2.274894in}}%
\pgfpathlineto{\pgfqpoint{2.073432in}{2.594808in}}%
\pgfpathlineto{\pgfqpoint{2.078364in}{2.722369in}}%
\pgfpathlineto{\pgfqpoint{2.083295in}{3.008015in}}%
\pgfpathlineto{\pgfqpoint{2.088227in}{2.632269in}}%
\pgfpathlineto{\pgfqpoint{2.093159in}{3.256917in}}%
\pgfpathlineto{\pgfqpoint{2.098091in}{2.299625in}}%
\pgfpathlineto{\pgfqpoint{2.103023in}{2.268886in}}%
\pgfpathlineto{\pgfqpoint{2.107955in}{2.268886in}}%
\pgfpathlineto{\pgfqpoint{2.117818in}{3.165328in}}%
\pgfpathlineto{\pgfqpoint{2.122750in}{2.955828in}}%
\pgfpathlineto{\pgfqpoint{2.127682in}{2.955828in}}%
\pgfpathlineto{\pgfqpoint{2.132614in}{2.929084in}}%
\pgfpathlineto{\pgfqpoint{2.137545in}{1.920964in}}%
\pgfpathlineto{\pgfqpoint{2.152341in}{1.920964in}}%
\pgfpathlineto{\pgfqpoint{2.157273in}{2.585770in}}%
\pgfpathlineto{\pgfqpoint{2.162205in}{2.585770in}}%
\pgfpathlineto{\pgfqpoint{2.167136in}{2.719387in}}%
\pgfpathlineto{\pgfqpoint{2.172068in}{2.523490in}}%
\pgfpathlineto{\pgfqpoint{2.177000in}{1.887322in}}%
\pgfpathlineto{\pgfqpoint{2.186864in}{1.887322in}}%
\pgfpathlineto{\pgfqpoint{2.191795in}{2.511625in}}%
\pgfpathlineto{\pgfqpoint{2.196727in}{2.126412in}}%
\pgfpathlineto{\pgfqpoint{2.201659in}{2.607454in}}%
\pgfpathlineto{\pgfqpoint{2.206591in}{2.290486in}}%
\pgfpathlineto{\pgfqpoint{2.211523in}{2.290486in}}%
\pgfpathlineto{\pgfqpoint{2.216455in}{2.323222in}}%
\pgfpathlineto{\pgfqpoint{2.221386in}{2.931861in}}%
\pgfpathlineto{\pgfqpoint{2.226318in}{3.211643in}}%
\pgfpathlineto{\pgfqpoint{2.236182in}{3.211643in}}%
\pgfpathlineto{\pgfqpoint{2.241114in}{2.817128in}}%
\pgfpathlineto{\pgfqpoint{2.246045in}{1.992020in}}%
\pgfpathlineto{\pgfqpoint{2.250977in}{2.379498in}}%
\pgfpathlineto{\pgfqpoint{2.255909in}{2.379498in}}%
\pgfpathlineto{\pgfqpoint{2.260841in}{2.270250in}}%
\pgfpathlineto{\pgfqpoint{2.265773in}{2.270250in}}%
\pgfpathlineto{\pgfqpoint{2.270705in}{2.470348in}}%
\pgfpathlineto{\pgfqpoint{2.275636in}{2.470348in}}%
\pgfpathlineto{\pgfqpoint{2.280568in}{2.453296in}}%
\pgfpathlineto{\pgfqpoint{2.285500in}{3.002712in}}%
\pgfpathlineto{\pgfqpoint{2.290432in}{1.868522in}}%
\pgfpathlineto{\pgfqpoint{2.295364in}{1.868522in}}%
\pgfpathlineto{\pgfqpoint{2.300295in}{2.669697in}}%
\pgfpathlineto{\pgfqpoint{2.305227in}{2.930743in}}%
\pgfpathlineto{\pgfqpoint{2.310159in}{2.798566in}}%
\pgfpathlineto{\pgfqpoint{2.315091in}{1.978190in}}%
\pgfpathlineto{\pgfqpoint{2.320023in}{1.978190in}}%
\pgfpathlineto{\pgfqpoint{2.324955in}{3.022967in}}%
\pgfpathlineto{\pgfqpoint{2.329886in}{2.411325in}}%
\pgfpathlineto{\pgfqpoint{2.334818in}{2.625312in}}%
\pgfpathlineto{\pgfqpoint{2.339750in}{2.332864in}}%
\pgfpathlineto{\pgfqpoint{2.344682in}{2.584059in}}%
\pgfpathlineto{\pgfqpoint{2.349614in}{2.208162in}}%
\pgfpathlineto{\pgfqpoint{2.354545in}{1.335888in}}%
\pgfpathlineto{\pgfqpoint{2.359477in}{2.906951in}}%
\pgfpathlineto{\pgfqpoint{2.374273in}{2.906951in}}%
\pgfpathlineto{\pgfqpoint{2.379205in}{2.607025in}}%
\pgfpathlineto{\pgfqpoint{2.384136in}{2.094679in}}%
\pgfpathlineto{\pgfqpoint{2.389068in}{2.570867in}}%
\pgfpathlineto{\pgfqpoint{2.394000in}{2.017392in}}%
\pgfpathlineto{\pgfqpoint{2.398932in}{2.017392in}}%
\pgfpathlineto{\pgfqpoint{2.403864in}{2.019091in}}%
\pgfpathlineto{\pgfqpoint{2.408795in}{3.004022in}}%
\pgfpathlineto{\pgfqpoint{2.413727in}{2.416596in}}%
\pgfpathlineto{\pgfqpoint{2.418659in}{3.050443in}}%
\pgfpathlineto{\pgfqpoint{2.423591in}{3.050443in}}%
\pgfpathlineto{\pgfqpoint{2.433455in}{2.106141in}}%
\pgfpathlineto{\pgfqpoint{2.438386in}{2.106141in}}%
\pgfpathlineto{\pgfqpoint{2.443318in}{2.487651in}}%
\pgfpathlineto{\pgfqpoint{2.453182in}{2.487651in}}%
\pgfpathlineto{\pgfqpoint{2.458114in}{2.706728in}}%
\pgfpathlineto{\pgfqpoint{2.463045in}{2.540310in}}%
\pgfpathlineto{\pgfqpoint{2.467977in}{2.954929in}}%
\pgfpathlineto{\pgfqpoint{2.472909in}{3.120140in}}%
\pgfpathlineto{\pgfqpoint{2.477841in}{2.637449in}}%
\pgfpathlineto{\pgfqpoint{2.482773in}{2.787540in}}%
\pgfpathlineto{\pgfqpoint{2.487705in}{3.244602in}}%
\pgfpathlineto{\pgfqpoint{2.492636in}{3.244602in}}%
\pgfpathlineto{\pgfqpoint{2.497568in}{2.952770in}}%
\pgfpathlineto{\pgfqpoint{2.502500in}{1.735403in}}%
\pgfpathlineto{\pgfqpoint{2.507432in}{2.420888in}}%
\pgfpathlineto{\pgfqpoint{2.512364in}{2.622743in}}%
\pgfpathlineto{\pgfqpoint{2.517295in}{2.537174in}}%
\pgfpathlineto{\pgfqpoint{2.522227in}{2.537174in}}%
\pgfpathlineto{\pgfqpoint{2.527159in}{2.358530in}}%
\pgfpathlineto{\pgfqpoint{2.532091in}{2.358530in}}%
\pgfpathlineto{\pgfqpoint{2.537023in}{2.749564in}}%
\pgfpathlineto{\pgfqpoint{2.541955in}{2.316312in}}%
\pgfpathlineto{\pgfqpoint{2.546886in}{2.316312in}}%
\pgfpathlineto{\pgfqpoint{2.551818in}{2.263962in}}%
\pgfpathlineto{\pgfqpoint{2.556750in}{2.263962in}}%
\pgfpathlineto{\pgfqpoint{2.561682in}{2.763735in}}%
\pgfpathlineto{\pgfqpoint{2.566614in}{2.842514in}}%
\pgfpathlineto{\pgfqpoint{2.571545in}{2.842514in}}%
\pgfpathlineto{\pgfqpoint{2.576477in}{1.967812in}}%
\pgfpathlineto{\pgfqpoint{2.581409in}{1.967812in}}%
\pgfpathlineto{\pgfqpoint{2.591273in}{2.960762in}}%
\pgfpathlineto{\pgfqpoint{2.596205in}{2.635910in}}%
\pgfpathlineto{\pgfqpoint{2.601136in}{2.635910in}}%
\pgfpathlineto{\pgfqpoint{2.606068in}{2.431901in}}%
\pgfpathlineto{\pgfqpoint{2.611000in}{2.917235in}}%
\pgfpathlineto{\pgfqpoint{2.615932in}{2.075284in}}%
\pgfpathlineto{\pgfqpoint{2.620864in}{2.129652in}}%
\pgfpathlineto{\pgfqpoint{2.625795in}{2.129652in}}%
\pgfpathlineto{\pgfqpoint{2.630727in}{2.642130in}}%
\pgfpathlineto{\pgfqpoint{2.635659in}{2.907842in}}%
\pgfpathlineto{\pgfqpoint{2.645523in}{2.907842in}}%
\pgfpathlineto{\pgfqpoint{2.650455in}{2.940381in}}%
\pgfpathlineto{\pgfqpoint{2.655386in}{2.383179in}}%
\pgfpathlineto{\pgfqpoint{2.660318in}{2.905768in}}%
\pgfpathlineto{\pgfqpoint{2.665250in}{2.905768in}}%
\pgfpathlineto{\pgfqpoint{2.670182in}{2.809262in}}%
\pgfpathlineto{\pgfqpoint{2.675114in}{2.333203in}}%
\pgfpathlineto{\pgfqpoint{2.680045in}{2.333203in}}%
\pgfpathlineto{\pgfqpoint{2.684977in}{2.280796in}}%
\pgfpathlineto{\pgfqpoint{2.689909in}{2.842440in}}%
\pgfpathlineto{\pgfqpoint{2.694841in}{3.089128in}}%
\pgfpathlineto{\pgfqpoint{2.699773in}{2.757226in}}%
\pgfpathlineto{\pgfqpoint{2.704705in}{2.246688in}}%
\pgfpathlineto{\pgfqpoint{2.709636in}{2.245182in}}%
\pgfpathlineto{\pgfqpoint{2.719500in}{2.245182in}}%
\pgfpathlineto{\pgfqpoint{2.724432in}{2.728770in}}%
\pgfpathlineto{\pgfqpoint{2.729364in}{2.678130in}}%
\pgfpathlineto{\pgfqpoint{2.734295in}{2.677057in}}%
\pgfpathlineto{\pgfqpoint{2.739227in}{2.356812in}}%
\pgfpathlineto{\pgfqpoint{2.744159in}{2.225268in}}%
\pgfpathlineto{\pgfqpoint{2.749091in}{2.790465in}}%
\pgfpathlineto{\pgfqpoint{2.754023in}{2.500089in}}%
\pgfpathlineto{\pgfqpoint{2.758955in}{2.509132in}}%
\pgfpathlineto{\pgfqpoint{2.763886in}{2.509132in}}%
\pgfpathlineto{\pgfqpoint{2.768818in}{2.691235in}}%
\pgfpathlineto{\pgfqpoint{2.773750in}{2.691235in}}%
\pgfpathlineto{\pgfqpoint{2.778682in}{2.612922in}}%
\pgfpathlineto{\pgfqpoint{2.783614in}{2.139210in}}%
\pgfpathlineto{\pgfqpoint{2.788545in}{2.184701in}}%
\pgfpathlineto{\pgfqpoint{2.793477in}{2.208989in}}%
\pgfpathlineto{\pgfqpoint{2.798409in}{2.379953in}}%
\pgfpathlineto{\pgfqpoint{2.803341in}{3.100857in}}%
\pgfpathlineto{\pgfqpoint{2.808273in}{1.493887in}}%
\pgfpathlineto{\pgfqpoint{2.818136in}{1.493887in}}%
\pgfpathlineto{\pgfqpoint{2.823068in}{2.892284in}}%
\pgfpathlineto{\pgfqpoint{2.828000in}{2.442316in}}%
\pgfpathlineto{\pgfqpoint{2.832932in}{2.801711in}}%
\pgfpathlineto{\pgfqpoint{2.837864in}{2.405224in}}%
\pgfpathlineto{\pgfqpoint{2.842795in}{3.207844in}}%
\pgfpathlineto{\pgfqpoint{2.847727in}{2.366171in}}%
\pgfpathlineto{\pgfqpoint{2.852659in}{2.912605in}}%
\pgfpathlineto{\pgfqpoint{2.857591in}{2.455494in}}%
\pgfpathlineto{\pgfqpoint{2.862523in}{2.639975in}}%
\pgfpathlineto{\pgfqpoint{2.867455in}{2.639975in}}%
\pgfpathlineto{\pgfqpoint{2.872386in}{2.275410in}}%
\pgfpathlineto{\pgfqpoint{2.877318in}{2.275410in}}%
\pgfpathlineto{\pgfqpoint{2.882250in}{2.504876in}}%
\pgfpathlineto{\pgfqpoint{2.887182in}{2.504876in}}%
\pgfpathlineto{\pgfqpoint{2.892114in}{2.452987in}}%
\pgfpathlineto{\pgfqpoint{2.897045in}{3.139536in}}%
\pgfpathlineto{\pgfqpoint{2.901977in}{2.956377in}}%
\pgfpathlineto{\pgfqpoint{2.906909in}{2.369759in}}%
\pgfpathlineto{\pgfqpoint{2.911841in}{3.068675in}}%
\pgfpathlineto{\pgfqpoint{2.916773in}{2.714207in}}%
\pgfpathlineto{\pgfqpoint{2.921705in}{2.714207in}}%
\pgfpathlineto{\pgfqpoint{2.926636in}{2.309152in}}%
\pgfpathlineto{\pgfqpoint{2.931568in}{2.255160in}}%
\pgfpathlineto{\pgfqpoint{2.936500in}{1.577705in}}%
\pgfpathlineto{\pgfqpoint{2.941432in}{1.577705in}}%
\pgfpathlineto{\pgfqpoint{2.946364in}{2.775704in}}%
\pgfpathlineto{\pgfqpoint{2.951295in}{2.694427in}}%
\pgfpathlineto{\pgfqpoint{2.956227in}{2.776764in}}%
\pgfpathlineto{\pgfqpoint{2.961159in}{2.889243in}}%
\pgfpathlineto{\pgfqpoint{2.966091in}{2.489676in}}%
\pgfpathlineto{\pgfqpoint{2.975955in}{2.489676in}}%
\pgfpathlineto{\pgfqpoint{2.980886in}{2.437132in}}%
\pgfpathlineto{\pgfqpoint{2.985818in}{2.499681in}}%
\pgfpathlineto{\pgfqpoint{2.990750in}{2.499681in}}%
\pgfpathlineto{\pgfqpoint{2.995682in}{2.662665in}}%
\pgfpathlineto{\pgfqpoint{3.000614in}{2.431369in}}%
\pgfpathlineto{\pgfqpoint{3.010477in}{3.207238in}}%
\pgfpathlineto{\pgfqpoint{3.015409in}{3.201073in}}%
\pgfpathlineto{\pgfqpoint{3.020341in}{2.791888in}}%
\pgfpathlineto{\pgfqpoint{3.025273in}{2.791888in}}%
\pgfpathlineto{\pgfqpoint{3.030205in}{1.995463in}}%
\pgfpathlineto{\pgfqpoint{3.035136in}{2.792848in}}%
\pgfpathlineto{\pgfqpoint{3.040068in}{2.618220in}}%
\pgfpathlineto{\pgfqpoint{3.045000in}{2.791328in}}%
\pgfpathlineto{\pgfqpoint{3.049932in}{2.674804in}}%
\pgfpathlineto{\pgfqpoint{3.054864in}{2.674804in}}%
\pgfpathlineto{\pgfqpoint{3.059795in}{3.133750in}}%
\pgfpathlineto{\pgfqpoint{3.064727in}{2.496542in}}%
\pgfpathlineto{\pgfqpoint{3.074591in}{2.849416in}}%
\pgfpathlineto{\pgfqpoint{3.079523in}{2.849416in}}%
\pgfpathlineto{\pgfqpoint{3.084455in}{2.803233in}}%
\pgfpathlineto{\pgfqpoint{3.089386in}{3.023265in}}%
\pgfpathlineto{\pgfqpoint{3.094318in}{2.602903in}}%
\pgfpathlineto{\pgfqpoint{3.104182in}{3.352547in}}%
\pgfpathlineto{\pgfqpoint{3.109114in}{2.021577in}}%
\pgfpathlineto{\pgfqpoint{3.114045in}{2.021577in}}%
\pgfpathlineto{\pgfqpoint{3.118977in}{2.528102in}}%
\pgfpathlineto{\pgfqpoint{3.128841in}{2.528102in}}%
\pgfpathlineto{\pgfqpoint{3.133773in}{2.228759in}}%
\pgfpathlineto{\pgfqpoint{3.148568in}{2.228759in}}%
\pgfpathlineto{\pgfqpoint{3.153500in}{2.339580in}}%
\pgfpathlineto{\pgfqpoint{3.158432in}{3.029107in}}%
\pgfpathlineto{\pgfqpoint{3.163364in}{2.488042in}}%
\pgfpathlineto{\pgfqpoint{3.168295in}{2.368946in}}%
\pgfpathlineto{\pgfqpoint{3.173227in}{2.383461in}}%
\pgfpathlineto{\pgfqpoint{3.178159in}{2.383461in}}%
\pgfpathlineto{\pgfqpoint{3.183091in}{2.541349in}}%
\pgfpathlineto{\pgfqpoint{3.188023in}{2.145222in}}%
\pgfpathlineto{\pgfqpoint{3.192955in}{2.989482in}}%
\pgfpathlineto{\pgfqpoint{3.197886in}{2.779332in}}%
\pgfpathlineto{\pgfqpoint{3.202818in}{3.057909in}}%
\pgfpathlineto{\pgfqpoint{3.212682in}{3.057909in}}%
\pgfpathlineto{\pgfqpoint{3.222545in}{2.849550in}}%
\pgfpathlineto{\pgfqpoint{3.227477in}{2.309398in}}%
\pgfpathlineto{\pgfqpoint{3.232409in}{2.309398in}}%
\pgfpathlineto{\pgfqpoint{3.237341in}{2.868681in}}%
\pgfpathlineto{\pgfqpoint{3.242273in}{2.833386in}}%
\pgfpathlineto{\pgfqpoint{3.247205in}{2.417925in}}%
\pgfpathlineto{\pgfqpoint{3.252136in}{2.417925in}}%
\pgfpathlineto{\pgfqpoint{3.257068in}{2.594329in}}%
\pgfpathlineto{\pgfqpoint{3.262000in}{2.673999in}}%
\pgfpathlineto{\pgfqpoint{3.266932in}{2.673999in}}%
\pgfpathlineto{\pgfqpoint{3.271864in}{2.728065in}}%
\pgfpathlineto{\pgfqpoint{3.276795in}{2.441994in}}%
\pgfpathlineto{\pgfqpoint{3.281727in}{2.470544in}}%
\pgfpathlineto{\pgfqpoint{3.291591in}{2.470544in}}%
\pgfpathlineto{\pgfqpoint{3.296523in}{2.517614in}}%
\pgfpathlineto{\pgfqpoint{3.301455in}{2.414018in}}%
\pgfpathlineto{\pgfqpoint{3.306386in}{3.281064in}}%
\pgfpathlineto{\pgfqpoint{3.311318in}{3.024412in}}%
\pgfpathlineto{\pgfqpoint{3.321182in}{3.024412in}}%
\pgfpathlineto{\pgfqpoint{3.326114in}{3.238532in}}%
\pgfpathlineto{\pgfqpoint{3.335977in}{2.305996in}}%
\pgfpathlineto{\pgfqpoint{3.340909in}{2.305996in}}%
\pgfpathlineto{\pgfqpoint{3.345841in}{2.063018in}}%
\pgfpathlineto{\pgfqpoint{3.350773in}{2.063018in}}%
\pgfpathlineto{\pgfqpoint{3.355705in}{1.890020in}}%
\pgfpathlineto{\pgfqpoint{3.360636in}{1.890020in}}%
\pgfpathlineto{\pgfqpoint{3.365568in}{1.747166in}}%
\pgfpathlineto{\pgfqpoint{3.370500in}{2.195191in}}%
\pgfpathlineto{\pgfqpoint{3.375432in}{3.380000in}}%
\pgfpathlineto{\pgfqpoint{3.385295in}{1.745742in}}%
\pgfpathlineto{\pgfqpoint{3.395159in}{1.745742in}}%
\pgfpathlineto{\pgfqpoint{3.400091in}{2.036700in}}%
\pgfpathlineto{\pgfqpoint{3.405023in}{2.036700in}}%
\pgfpathlineto{\pgfqpoint{3.414886in}{2.830646in}}%
\pgfpathlineto{\pgfqpoint{3.419818in}{2.824521in}}%
\pgfpathlineto{\pgfqpoint{3.424750in}{2.366730in}}%
\pgfpathlineto{\pgfqpoint{3.429682in}{2.366730in}}%
\pgfpathlineto{\pgfqpoint{3.434614in}{2.165681in}}%
\pgfpathlineto{\pgfqpoint{3.439545in}{2.446952in}}%
\pgfpathlineto{\pgfqpoint{3.454341in}{2.446952in}}%
\pgfpathlineto{\pgfqpoint{3.459273in}{2.695956in}}%
\pgfpathlineto{\pgfqpoint{3.464205in}{1.925324in}}%
\pgfpathlineto{\pgfqpoint{3.469136in}{1.925324in}}%
\pgfpathlineto{\pgfqpoint{3.474068in}{2.624860in}}%
\pgfpathlineto{\pgfqpoint{3.479000in}{2.544598in}}%
\pgfpathlineto{\pgfqpoint{3.483932in}{2.544598in}}%
\pgfpathlineto{\pgfqpoint{3.488864in}{2.351375in}}%
\pgfpathlineto{\pgfqpoint{3.498727in}{2.351375in}}%
\pgfpathlineto{\pgfqpoint{3.503659in}{2.991395in}}%
\pgfpathlineto{\pgfqpoint{3.508591in}{2.979488in}}%
\pgfpathlineto{\pgfqpoint{3.513523in}{3.060014in}}%
\pgfpathlineto{\pgfqpoint{3.518455in}{2.422839in}}%
\pgfpathlineto{\pgfqpoint{3.523386in}{2.841542in}}%
\pgfpathlineto{\pgfqpoint{3.528318in}{2.587346in}}%
\pgfpathlineto{\pgfqpoint{3.533250in}{2.587346in}}%
\pgfpathlineto{\pgfqpoint{3.538182in}{2.880439in}}%
\pgfpathlineto{\pgfqpoint{3.543114in}{2.713489in}}%
\pgfpathlineto{\pgfqpoint{3.548045in}{1.825518in}}%
\pgfpathlineto{\pgfqpoint{3.552977in}{1.825518in}}%
\pgfpathlineto{\pgfqpoint{3.557909in}{2.209814in}}%
\pgfpathlineto{\pgfqpoint{3.562841in}{1.916645in}}%
\pgfpathlineto{\pgfqpoint{3.567773in}{1.916645in}}%
\pgfpathlineto{\pgfqpoint{3.572705in}{1.773025in}}%
\pgfpathlineto{\pgfqpoint{3.577636in}{2.518653in}}%
\pgfpathlineto{\pgfqpoint{3.582568in}{2.340632in}}%
\pgfpathlineto{\pgfqpoint{3.592432in}{2.340632in}}%
\pgfpathlineto{\pgfqpoint{3.597364in}{2.425650in}}%
\pgfpathlineto{\pgfqpoint{3.602295in}{2.440500in}}%
\pgfpathlineto{\pgfqpoint{3.607227in}{2.440500in}}%
\pgfpathlineto{\pgfqpoint{3.612159in}{3.264039in}}%
\pgfpathlineto{\pgfqpoint{3.617091in}{2.839350in}}%
\pgfpathlineto{\pgfqpoint{3.622023in}{2.210998in}}%
\pgfpathlineto{\pgfqpoint{3.626955in}{2.655554in}}%
\pgfpathlineto{\pgfqpoint{3.631886in}{2.494560in}}%
\pgfpathlineto{\pgfqpoint{3.641750in}{3.059430in}}%
\pgfpathlineto{\pgfqpoint{3.646682in}{3.074001in}}%
\pgfpathlineto{\pgfqpoint{3.651614in}{3.074001in}}%
\pgfpathlineto{\pgfqpoint{3.656545in}{2.821007in}}%
\pgfpathlineto{\pgfqpoint{3.661477in}{2.244535in}}%
\pgfpathlineto{\pgfqpoint{3.666409in}{2.244535in}}%
\pgfpathlineto{\pgfqpoint{3.671341in}{2.972963in}}%
\pgfpathlineto{\pgfqpoint{3.676273in}{2.401492in}}%
\pgfpathlineto{\pgfqpoint{3.681205in}{2.401492in}}%
\pgfpathlineto{\pgfqpoint{3.686136in}{2.932929in}}%
\pgfpathlineto{\pgfqpoint{3.691068in}{2.198273in}}%
\pgfpathlineto{\pgfqpoint{3.696000in}{2.377383in}}%
\pgfpathlineto{\pgfqpoint{3.700932in}{2.377383in}}%
\pgfpathlineto{\pgfqpoint{3.705864in}{2.801506in}}%
\pgfpathlineto{\pgfqpoint{3.710795in}{2.801506in}}%
\pgfpathlineto{\pgfqpoint{3.715727in}{2.400710in}}%
\pgfpathlineto{\pgfqpoint{3.720659in}{2.911198in}}%
\pgfpathlineto{\pgfqpoint{3.725591in}{2.513382in}}%
\pgfpathlineto{\pgfqpoint{3.730523in}{2.673013in}}%
\pgfpathlineto{\pgfqpoint{3.735455in}{2.975457in}}%
\pgfpathlineto{\pgfqpoint{3.740386in}{2.975457in}}%
\pgfpathlineto{\pgfqpoint{3.745318in}{2.634539in}}%
\pgfpathlineto{\pgfqpoint{3.750250in}{3.056985in}}%
\pgfpathlineto{\pgfqpoint{3.755182in}{3.056985in}}%
\pgfpathlineto{\pgfqpoint{3.760114in}{3.286841in}}%
\pgfpathlineto{\pgfqpoint{3.765045in}{2.953498in}}%
\pgfpathlineto{\pgfqpoint{3.769977in}{3.313095in}}%
\pgfpathlineto{\pgfqpoint{3.779841in}{3.313095in}}%
\pgfpathlineto{\pgfqpoint{3.784773in}{2.417470in}}%
\pgfpathlineto{\pgfqpoint{3.789705in}{2.417249in}}%
\pgfpathlineto{\pgfqpoint{3.794636in}{2.300378in}}%
\pgfpathlineto{\pgfqpoint{3.799568in}{2.623040in}}%
\pgfpathlineto{\pgfqpoint{3.804500in}{2.623040in}}%
\pgfpathlineto{\pgfqpoint{3.809432in}{2.464889in}}%
\pgfpathlineto{\pgfqpoint{3.814364in}{2.464889in}}%
\pgfpathlineto{\pgfqpoint{3.819295in}{2.733417in}}%
\pgfpathlineto{\pgfqpoint{3.824227in}{2.646699in}}%
\pgfpathlineto{\pgfqpoint{3.829159in}{2.646699in}}%
\pgfpathlineto{\pgfqpoint{3.834091in}{2.756216in}}%
\pgfpathlineto{\pgfqpoint{3.839023in}{3.199507in}}%
\pgfpathlineto{\pgfqpoint{3.843955in}{2.966324in}}%
\pgfpathlineto{\pgfqpoint{3.848886in}{2.541029in}}%
\pgfpathlineto{\pgfqpoint{3.853818in}{2.684232in}}%
\pgfpathlineto{\pgfqpoint{3.858750in}{2.622718in}}%
\pgfpathlineto{\pgfqpoint{3.863682in}{2.414417in}}%
\pgfpathlineto{\pgfqpoint{3.868614in}{1.842616in}}%
\pgfpathlineto{\pgfqpoint{3.873545in}{2.859083in}}%
\pgfpathlineto{\pgfqpoint{3.878477in}{1.655699in}}%
\pgfpathlineto{\pgfqpoint{3.883409in}{1.655699in}}%
\pgfpathlineto{\pgfqpoint{3.888341in}{2.689407in}}%
\pgfpathlineto{\pgfqpoint{3.893273in}{2.741403in}}%
\pgfpathlineto{\pgfqpoint{3.898205in}{2.741403in}}%
\pgfpathlineto{\pgfqpoint{3.903136in}{2.974184in}}%
\pgfpathlineto{\pgfqpoint{3.908068in}{2.724928in}}%
\pgfpathlineto{\pgfqpoint{3.913000in}{2.394542in}}%
\pgfpathlineto{\pgfqpoint{3.917932in}{2.394542in}}%
\pgfpathlineto{\pgfqpoint{3.922864in}{2.506197in}}%
\pgfpathlineto{\pgfqpoint{3.927795in}{3.162355in}}%
\pgfpathlineto{\pgfqpoint{3.932727in}{2.862940in}}%
\pgfpathlineto{\pgfqpoint{3.937659in}{3.048090in}}%
\pgfpathlineto{\pgfqpoint{3.942591in}{2.238208in}}%
\pgfpathlineto{\pgfqpoint{3.967250in}{2.238208in}}%
\pgfpathlineto{\pgfqpoint{3.972182in}{1.919752in}}%
\pgfpathlineto{\pgfqpoint{3.982045in}{1.919752in}}%
\pgfpathlineto{\pgfqpoint{3.986977in}{2.683521in}}%
\pgfpathlineto{\pgfqpoint{3.991909in}{2.266156in}}%
\pgfpathlineto{\pgfqpoint{3.996841in}{2.736733in}}%
\pgfpathlineto{\pgfqpoint{4.001773in}{2.512410in}}%
\pgfpathlineto{\pgfqpoint{4.006705in}{2.491613in}}%
\pgfpathlineto{\pgfqpoint{4.011636in}{2.476411in}}%
\pgfpathlineto{\pgfqpoint{4.016568in}{2.572035in}}%
\pgfpathlineto{\pgfqpoint{4.021500in}{2.873580in}}%
\pgfpathlineto{\pgfqpoint{4.026432in}{2.552613in}}%
\pgfpathlineto{\pgfqpoint{4.031364in}{2.539505in}}%
\pgfpathlineto{\pgfqpoint{4.036295in}{2.099347in}}%
\pgfpathlineto{\pgfqpoint{4.041227in}{2.657137in}}%
\pgfpathlineto{\pgfqpoint{4.046159in}{2.657137in}}%
\pgfpathlineto{\pgfqpoint{4.051091in}{2.720794in}}%
\pgfpathlineto{\pgfqpoint{4.056023in}{2.720794in}}%
\pgfpathlineto{\pgfqpoint{4.060955in}{2.467486in}}%
\pgfpathlineto{\pgfqpoint{4.065886in}{2.467486in}}%
\pgfpathlineto{\pgfqpoint{4.070818in}{2.470917in}}%
\pgfpathlineto{\pgfqpoint{4.075750in}{2.265202in}}%
\pgfpathlineto{\pgfqpoint{4.080682in}{2.619034in}}%
\pgfpathlineto{\pgfqpoint{4.095477in}{2.619034in}}%
\pgfpathlineto{\pgfqpoint{4.100409in}{2.701483in}}%
\pgfpathlineto{\pgfqpoint{4.105341in}{2.701483in}}%
\pgfpathlineto{\pgfqpoint{4.110273in}{2.582254in}}%
\pgfpathlineto{\pgfqpoint{4.115205in}{2.902218in}}%
\pgfpathlineto{\pgfqpoint{4.120136in}{3.020455in}}%
\pgfpathlineto{\pgfqpoint{4.125068in}{2.663670in}}%
\pgfpathlineto{\pgfqpoint{4.130000in}{2.503713in}}%
\pgfpathlineto{\pgfqpoint{4.134932in}{2.886410in}}%
\pgfpathlineto{\pgfqpoint{4.139864in}{2.613093in}}%
\pgfpathlineto{\pgfqpoint{4.144795in}{3.028595in}}%
\pgfpathlineto{\pgfqpoint{4.149727in}{2.832898in}}%
\pgfpathlineto{\pgfqpoint{4.154659in}{3.010439in}}%
\pgfpathlineto{\pgfqpoint{4.159591in}{3.146987in}}%
\pgfpathlineto{\pgfqpoint{4.164523in}{3.146987in}}%
\pgfpathlineto{\pgfqpoint{4.169455in}{2.901798in}}%
\pgfpathlineto{\pgfqpoint{4.174386in}{2.901798in}}%
\pgfpathlineto{\pgfqpoint{4.179318in}{2.060351in}}%
\pgfpathlineto{\pgfqpoint{4.184250in}{2.060351in}}%
\pgfpathlineto{\pgfqpoint{4.189182in}{2.842449in}}%
\pgfpathlineto{\pgfqpoint{4.194114in}{2.439833in}}%
\pgfpathlineto{\pgfqpoint{4.199045in}{2.198488in}}%
\pgfpathlineto{\pgfqpoint{4.203977in}{2.717908in}}%
\pgfpathlineto{\pgfqpoint{4.208909in}{2.373415in}}%
\pgfpathlineto{\pgfqpoint{4.213841in}{1.763166in}}%
\pgfpathlineto{\pgfqpoint{4.218773in}{2.482199in}}%
\pgfpathlineto{\pgfqpoint{4.223705in}{2.881038in}}%
\pgfpathlineto{\pgfqpoint{4.228636in}{2.244263in}}%
\pgfpathlineto{\pgfqpoint{4.233568in}{2.244263in}}%
\pgfpathlineto{\pgfqpoint{4.238500in}{2.353025in}}%
\pgfpathlineto{\pgfqpoint{4.243432in}{2.751827in}}%
\pgfpathlineto{\pgfqpoint{4.248364in}{2.751827in}}%
\pgfpathlineto{\pgfqpoint{4.253295in}{2.726799in}}%
\pgfpathlineto{\pgfqpoint{4.258227in}{2.758752in}}%
\pgfpathlineto{\pgfqpoint{4.263159in}{2.712871in}}%
\pgfpathlineto{\pgfqpoint{4.268091in}{2.853827in}}%
\pgfpathlineto{\pgfqpoint{4.273023in}{2.853827in}}%
\pgfpathlineto{\pgfqpoint{4.277955in}{3.047485in}}%
\pgfpathlineto{\pgfqpoint{4.282886in}{2.787477in}}%
\pgfpathlineto{\pgfqpoint{4.287818in}{2.920165in}}%
\pgfpathlineto{\pgfqpoint{4.292750in}{1.964772in}}%
\pgfpathlineto{\pgfqpoint{4.302614in}{1.964772in}}%
\pgfpathlineto{\pgfqpoint{4.307545in}{2.766535in}}%
\pgfpathlineto{\pgfqpoint{4.312477in}{2.766535in}}%
\pgfpathlineto{\pgfqpoint{4.317409in}{2.602793in}}%
\pgfpathlineto{\pgfqpoint{4.322341in}{2.602793in}}%
\pgfpathlineto{\pgfqpoint{4.327273in}{2.559061in}}%
\pgfpathlineto{\pgfqpoint{4.332205in}{2.725089in}}%
\pgfpathlineto{\pgfqpoint{4.337136in}{2.725089in}}%
\pgfpathlineto{\pgfqpoint{4.342068in}{2.769678in}}%
\pgfpathlineto{\pgfqpoint{4.347000in}{2.769678in}}%
\pgfpathlineto{\pgfqpoint{4.351932in}{2.794132in}}%
\pgfpathlineto{\pgfqpoint{4.356864in}{2.648685in}}%
\pgfpathlineto{\pgfqpoint{4.361795in}{2.648685in}}%
\pgfpathlineto{\pgfqpoint{4.366727in}{2.629406in}}%
\pgfpathlineto{\pgfqpoint{4.371659in}{2.629406in}}%
\pgfpathlineto{\pgfqpoint{4.376591in}{3.375091in}}%
\pgfpathlineto{\pgfqpoint{4.381523in}{2.534942in}}%
\pgfpathlineto{\pgfqpoint{4.386455in}{2.580200in}}%
\pgfpathlineto{\pgfqpoint{4.391386in}{1.938559in}}%
\pgfpathlineto{\pgfqpoint{4.396318in}{1.938559in}}%
\pgfpathlineto{\pgfqpoint{4.401250in}{2.951119in}}%
\pgfpathlineto{\pgfqpoint{4.406182in}{2.532376in}}%
\pgfpathlineto{\pgfqpoint{4.411114in}{2.435998in}}%
\pgfpathlineto{\pgfqpoint{4.416045in}{2.435998in}}%
\pgfpathlineto{\pgfqpoint{4.420977in}{2.458287in}}%
\pgfpathlineto{\pgfqpoint{4.425909in}{3.097433in}}%
\pgfpathlineto{\pgfqpoint{4.430841in}{3.112447in}}%
\pgfpathlineto{\pgfqpoint{4.440705in}{3.112447in}}%
\pgfpathlineto{\pgfqpoint{4.445636in}{3.319371in}}%
\pgfpathlineto{\pgfqpoint{4.450568in}{2.732690in}}%
\pgfpathlineto{\pgfqpoint{4.455500in}{2.588564in}}%
\pgfpathlineto{\pgfqpoint{4.460432in}{2.003724in}}%
\pgfpathlineto{\pgfqpoint{4.465364in}{2.003724in}}%
\pgfpathlineto{\pgfqpoint{4.470295in}{2.735347in}}%
\pgfpathlineto{\pgfqpoint{4.475227in}{2.859358in}}%
\pgfpathlineto{\pgfqpoint{4.480159in}{2.808048in}}%
\pgfpathlineto{\pgfqpoint{4.485091in}{2.445173in}}%
\pgfpathlineto{\pgfqpoint{4.490023in}{2.445173in}}%
\pgfpathlineto{\pgfqpoint{4.494955in}{2.288925in}}%
\pgfpathlineto{\pgfqpoint{4.499886in}{2.621818in}}%
\pgfpathlineto{\pgfqpoint{4.504818in}{2.614387in}}%
\pgfpathlineto{\pgfqpoint{4.509750in}{3.315553in}}%
\pgfpathlineto{\pgfqpoint{4.514682in}{2.621556in}}%
\pgfpathlineto{\pgfqpoint{4.524545in}{2.621556in}}%
\pgfpathlineto{\pgfqpoint{4.529477in}{2.364898in}}%
\pgfpathlineto{\pgfqpoint{4.534409in}{2.364898in}}%
\pgfpathlineto{\pgfqpoint{4.539341in}{1.552060in}}%
\pgfpathlineto{\pgfqpoint{4.544273in}{2.555309in}}%
\pgfpathlineto{\pgfqpoint{4.554136in}{2.555309in}}%
\pgfpathlineto{\pgfqpoint{4.559068in}{2.299653in}}%
\pgfpathlineto{\pgfqpoint{4.564000in}{3.039654in}}%
\pgfpathlineto{\pgfqpoint{4.568932in}{2.727292in}}%
\pgfpathlineto{\pgfqpoint{4.573864in}{1.970809in}}%
\pgfpathlineto{\pgfqpoint{4.578795in}{2.363880in}}%
\pgfpathlineto{\pgfqpoint{4.583727in}{2.363880in}}%
\pgfpathlineto{\pgfqpoint{4.588659in}{2.441179in}}%
\pgfpathlineto{\pgfqpoint{4.593591in}{2.973798in}}%
\pgfpathlineto{\pgfqpoint{4.598523in}{2.966347in}}%
\pgfpathlineto{\pgfqpoint{4.603455in}{2.366725in}}%
\pgfpathlineto{\pgfqpoint{4.608386in}{2.440905in}}%
\pgfpathlineto{\pgfqpoint{4.613318in}{2.253691in}}%
\pgfpathlineto{\pgfqpoint{4.618250in}{2.253691in}}%
\pgfpathlineto{\pgfqpoint{4.623182in}{1.973664in}}%
\pgfpathlineto{\pgfqpoint{4.628114in}{1.973664in}}%
\pgfpathlineto{\pgfqpoint{4.633045in}{1.746262in}}%
\pgfpathlineto{\pgfqpoint{4.637977in}{1.746262in}}%
\pgfpathlineto{\pgfqpoint{4.642909in}{2.525167in}}%
\pgfpathlineto{\pgfqpoint{4.652773in}{2.525167in}}%
\pgfpathlineto{\pgfqpoint{4.657705in}{2.330065in}}%
\pgfpathlineto{\pgfqpoint{4.662636in}{2.330065in}}%
\pgfpathlineto{\pgfqpoint{4.667568in}{2.600449in}}%
\pgfpathlineto{\pgfqpoint{4.672500in}{2.690857in}}%
\pgfpathlineto{\pgfqpoint{4.682364in}{2.690857in}}%
\pgfpathlineto{\pgfqpoint{4.687295in}{2.670072in}}%
\pgfpathlineto{\pgfqpoint{4.692227in}{2.461772in}}%
\pgfpathlineto{\pgfqpoint{4.702091in}{2.461772in}}%
\pgfpathlineto{\pgfqpoint{4.707023in}{2.265143in}}%
\pgfpathlineto{\pgfqpoint{4.711955in}{2.265143in}}%
\pgfpathlineto{\pgfqpoint{4.716886in}{2.487565in}}%
\pgfpathlineto{\pgfqpoint{4.721818in}{2.487565in}}%
\pgfpathlineto{\pgfqpoint{4.726750in}{3.028205in}}%
\pgfpathlineto{\pgfqpoint{4.731682in}{3.028205in}}%
\pgfpathlineto{\pgfqpoint{4.736614in}{3.221489in}}%
\pgfpathlineto{\pgfqpoint{4.741545in}{2.681093in}}%
\pgfpathlineto{\pgfqpoint{4.746477in}{2.664709in}}%
\pgfpathlineto{\pgfqpoint{4.751409in}{2.440731in}}%
\pgfpathlineto{\pgfqpoint{4.756341in}{1.832946in}}%
\pgfpathlineto{\pgfqpoint{4.761273in}{1.979096in}}%
\pgfpathlineto{\pgfqpoint{4.766205in}{2.209572in}}%
\pgfpathlineto{\pgfqpoint{4.771136in}{2.209572in}}%
\pgfpathlineto{\pgfqpoint{4.776068in}{3.056372in}}%
\pgfpathlineto{\pgfqpoint{4.781000in}{1.736312in}}%
\pgfpathlineto{\pgfqpoint{4.830318in}{1.736312in}}%
\pgfpathlineto{\pgfqpoint{4.835250in}{2.488064in}}%
\pgfpathlineto{\pgfqpoint{4.840182in}{2.488064in}}%
\pgfpathlineto{\pgfqpoint{4.845114in}{2.390990in}}%
\pgfpathlineto{\pgfqpoint{4.850045in}{2.216957in}}%
\pgfpathlineto{\pgfqpoint{4.854977in}{3.221100in}}%
\pgfpathlineto{\pgfqpoint{4.859909in}{3.221100in}}%
\pgfpathlineto{\pgfqpoint{4.864841in}{2.317802in}}%
\pgfpathlineto{\pgfqpoint{4.869773in}{2.892497in}}%
\pgfpathlineto{\pgfqpoint{4.874705in}{3.250376in}}%
\pgfpathlineto{\pgfqpoint{4.879636in}{2.488409in}}%
\pgfpathlineto{\pgfqpoint{4.884568in}{2.948339in}}%
\pgfpathlineto{\pgfqpoint{4.889500in}{2.481443in}}%
\pgfpathlineto{\pgfqpoint{4.894432in}{3.081714in}}%
\pgfpathlineto{\pgfqpoint{4.899364in}{2.874309in}}%
\pgfpathlineto{\pgfqpoint{4.904295in}{2.283649in}}%
\pgfpathlineto{\pgfqpoint{4.914159in}{2.283649in}}%
\pgfpathlineto{\pgfqpoint{4.919091in}{2.833011in}}%
\pgfpathlineto{\pgfqpoint{4.924023in}{2.272976in}}%
\pgfpathlineto{\pgfqpoint{4.938818in}{2.272976in}}%
\pgfpathlineto{\pgfqpoint{4.943750in}{2.388386in}}%
\pgfpathlineto{\pgfqpoint{4.948682in}{2.210030in}}%
\pgfpathlineto{\pgfqpoint{4.968409in}{2.210030in}}%
\pgfpathlineto{\pgfqpoint{4.973341in}{2.484103in}}%
\pgfpathlineto{\pgfqpoint{4.978273in}{2.394413in}}%
\pgfpathlineto{\pgfqpoint{4.983205in}{2.648728in}}%
\pgfpathlineto{\pgfqpoint{4.988136in}{2.648728in}}%
\pgfpathlineto{\pgfqpoint{4.993068in}{2.710236in}}%
\pgfpathlineto{\pgfqpoint{4.998000in}{2.397943in}}%
\pgfpathlineto{\pgfqpoint{5.002932in}{2.216113in}}%
\pgfpathlineto{\pgfqpoint{5.007864in}{2.684059in}}%
\pgfpathlineto{\pgfqpoint{5.012795in}{2.854785in}}%
\pgfpathlineto{\pgfqpoint{5.017727in}{2.221613in}}%
\pgfpathlineto{\pgfqpoint{5.027591in}{2.221613in}}%
\pgfpathlineto{\pgfqpoint{5.032523in}{2.803442in}}%
\pgfpathlineto{\pgfqpoint{5.037455in}{3.145840in}}%
\pgfpathlineto{\pgfqpoint{5.042386in}{2.025398in}}%
\pgfpathlineto{\pgfqpoint{5.057182in}{2.025398in}}%
\pgfpathlineto{\pgfqpoint{5.062114in}{2.792404in}}%
\pgfpathlineto{\pgfqpoint{5.067045in}{3.028049in}}%
\pgfpathlineto{\pgfqpoint{5.071977in}{2.369604in}}%
\pgfpathlineto{\pgfqpoint{5.076909in}{2.025586in}}%
\pgfpathlineto{\pgfqpoint{5.081841in}{2.025586in}}%
\pgfpathlineto{\pgfqpoint{5.086773in}{2.875603in}}%
\pgfpathlineto{\pgfqpoint{5.091705in}{2.853618in}}%
\pgfpathlineto{\pgfqpoint{5.096636in}{2.988807in}}%
\pgfpathlineto{\pgfqpoint{5.101568in}{2.305151in}}%
\pgfpathlineto{\pgfqpoint{5.106500in}{2.546295in}}%
\pgfpathlineto{\pgfqpoint{5.111432in}{2.546295in}}%
\pgfpathlineto{\pgfqpoint{5.116364in}{2.773242in}}%
\pgfpathlineto{\pgfqpoint{5.121295in}{2.545236in}}%
\pgfpathlineto{\pgfqpoint{5.126227in}{2.235081in}}%
\pgfpathlineto{\pgfqpoint{5.131159in}{2.869013in}}%
\pgfpathlineto{\pgfqpoint{5.136091in}{2.989195in}}%
\pgfpathlineto{\pgfqpoint{5.141023in}{2.786749in}}%
\pgfpathlineto{\pgfqpoint{5.145955in}{2.115996in}}%
\pgfpathlineto{\pgfqpoint{5.150886in}{2.115996in}}%
\pgfpathlineto{\pgfqpoint{5.155818in}{2.429348in}}%
\pgfpathlineto{\pgfqpoint{5.165682in}{2.429348in}}%
\pgfpathlineto{\pgfqpoint{5.170614in}{2.227855in}}%
\pgfpathlineto{\pgfqpoint{5.180477in}{2.227855in}}%
\pgfpathlineto{\pgfqpoint{5.185409in}{2.538560in}}%
\pgfpathlineto{\pgfqpoint{5.190341in}{2.538560in}}%
\pgfpathlineto{\pgfqpoint{5.195273in}{2.585261in}}%
\pgfpathlineto{\pgfqpoint{5.200205in}{1.938082in}}%
\pgfpathlineto{\pgfqpoint{5.205136in}{1.938082in}}%
\pgfpathlineto{\pgfqpoint{5.210068in}{1.847748in}}%
\pgfpathlineto{\pgfqpoint{5.224864in}{1.847748in}}%
\pgfpathlineto{\pgfqpoint{5.229795in}{2.991220in}}%
\pgfpathlineto{\pgfqpoint{5.234727in}{3.160799in}}%
\pgfpathlineto{\pgfqpoint{5.239659in}{3.160799in}}%
\pgfpathlineto{\pgfqpoint{5.244591in}{2.786112in}}%
\pgfpathlineto{\pgfqpoint{5.249523in}{2.280429in}}%
\pgfpathlineto{\pgfqpoint{5.264318in}{2.280429in}}%
\pgfpathlineto{\pgfqpoint{5.269250in}{1.869247in}}%
\pgfpathlineto{\pgfqpoint{5.274182in}{2.749038in}}%
\pgfpathlineto{\pgfqpoint{5.279114in}{2.085801in}}%
\pgfpathlineto{\pgfqpoint{5.284045in}{2.498037in}}%
\pgfpathlineto{\pgfqpoint{5.288977in}{2.498037in}}%
\pgfpathlineto{\pgfqpoint{5.293909in}{2.444180in}}%
\pgfpathlineto{\pgfqpoint{5.298841in}{2.727246in}}%
\pgfpathlineto{\pgfqpoint{5.303773in}{2.727246in}}%
\pgfpathlineto{\pgfqpoint{5.308705in}{2.589258in}}%
\pgfpathlineto{\pgfqpoint{5.313636in}{2.196863in}}%
\pgfpathlineto{\pgfqpoint{5.318568in}{2.697837in}}%
\pgfpathlineto{\pgfqpoint{5.323500in}{2.110741in}}%
\pgfpathlineto{\pgfqpoint{5.328432in}{2.684200in}}%
\pgfpathlineto{\pgfqpoint{5.333364in}{2.684200in}}%
\pgfpathlineto{\pgfqpoint{5.338295in}{2.380464in}}%
\pgfpathlineto{\pgfqpoint{5.343227in}{2.975024in}}%
\pgfpathlineto{\pgfqpoint{5.348159in}{2.314939in}}%
\pgfpathlineto{\pgfqpoint{5.353091in}{2.314939in}}%
\pgfpathlineto{\pgfqpoint{5.358023in}{2.725542in}}%
\pgfpathlineto{\pgfqpoint{5.362955in}{2.718620in}}%
\pgfpathlineto{\pgfqpoint{5.367886in}{2.630829in}}%
\pgfpathlineto{\pgfqpoint{5.372818in}{2.323597in}}%
\pgfpathlineto{\pgfqpoint{5.377750in}{2.746499in}}%
\pgfpathlineto{\pgfqpoint{5.382682in}{2.385874in}}%
\pgfpathlineto{\pgfqpoint{5.387614in}{2.385874in}}%
\pgfpathlineto{\pgfqpoint{5.392545in}{2.297294in}}%
\pgfpathlineto{\pgfqpoint{5.407341in}{2.297294in}}%
\pgfpathlineto{\pgfqpoint{5.412273in}{2.260230in}}%
\pgfpathlineto{\pgfqpoint{5.417205in}{2.260230in}}%
\pgfpathlineto{\pgfqpoint{5.427068in}{2.968480in}}%
\pgfpathlineto{\pgfqpoint{5.432000in}{3.057022in}}%
\pgfpathlineto{\pgfqpoint{5.436932in}{2.937718in}}%
\pgfpathlineto{\pgfqpoint{5.441864in}{2.937718in}}%
\pgfpathlineto{\pgfqpoint{5.446795in}{2.540820in}}%
\pgfpathlineto{\pgfqpoint{5.451727in}{2.345909in}}%
\pgfpathlineto{\pgfqpoint{5.456659in}{2.897320in}}%
\pgfpathlineto{\pgfqpoint{5.461591in}{2.360222in}}%
\pgfpathlineto{\pgfqpoint{5.466523in}{2.360222in}}%
\pgfpathlineto{\pgfqpoint{5.471455in}{2.308529in}}%
\pgfpathlineto{\pgfqpoint{5.481318in}{3.074051in}}%
\pgfpathlineto{\pgfqpoint{5.486250in}{2.905885in}}%
\pgfpathlineto{\pgfqpoint{5.491182in}{2.398774in}}%
\pgfpathlineto{\pgfqpoint{5.496114in}{2.550309in}}%
\pgfpathlineto{\pgfqpoint{5.501045in}{2.194244in}}%
\pgfpathlineto{\pgfqpoint{5.505977in}{2.801416in}}%
\pgfpathlineto{\pgfqpoint{5.510909in}{2.801416in}}%
\pgfpathlineto{\pgfqpoint{5.515841in}{2.615351in}}%
\pgfpathlineto{\pgfqpoint{5.520773in}{2.540036in}}%
\pgfpathlineto{\pgfqpoint{5.525705in}{3.246250in}}%
\pgfpathlineto{\pgfqpoint{5.530636in}{3.020186in}}%
\pgfpathlineto{\pgfqpoint{5.540500in}{3.020186in}}%
\pgfpathlineto{\pgfqpoint{5.545432in}{2.968964in}}%
\pgfpathlineto{\pgfqpoint{5.550364in}{2.692364in}}%
\pgfpathlineto{\pgfqpoint{5.560227in}{2.692364in}}%
\pgfpathlineto{\pgfqpoint{5.565159in}{2.178081in}}%
\pgfpathlineto{\pgfqpoint{5.579955in}{2.178081in}}%
\pgfpathlineto{\pgfqpoint{5.584886in}{3.058152in}}%
\pgfpathlineto{\pgfqpoint{5.589818in}{2.791184in}}%
\pgfpathlineto{\pgfqpoint{5.594750in}{2.669984in}}%
\pgfpathlineto{\pgfqpoint{5.599682in}{2.961480in}}%
\pgfpathlineto{\pgfqpoint{5.604614in}{2.430366in}}%
\pgfpathlineto{\pgfqpoint{5.609545in}{2.687605in}}%
\pgfpathlineto{\pgfqpoint{5.614477in}{2.701574in}}%
\pgfpathlineto{\pgfqpoint{5.619409in}{1.833904in}}%
\pgfpathlineto{\pgfqpoint{5.624341in}{1.833904in}}%
\pgfpathlineto{\pgfqpoint{5.629273in}{2.071724in}}%
\pgfpathlineto{\pgfqpoint{5.634205in}{1.975158in}}%
\pgfpathlineto{\pgfqpoint{5.639136in}{2.021455in}}%
\pgfpathlineto{\pgfqpoint{5.644068in}{2.147223in}}%
\pgfpathlineto{\pgfqpoint{5.649000in}{2.709240in}}%
\pgfpathlineto{\pgfqpoint{5.653932in}{3.031669in}}%
\pgfpathlineto{\pgfqpoint{5.658864in}{2.481864in}}%
\pgfpathlineto{\pgfqpoint{5.663795in}{2.529425in}}%
\pgfpathlineto{\pgfqpoint{5.668727in}{2.529425in}}%
\pgfpathlineto{\pgfqpoint{5.673659in}{2.065386in}}%
\pgfpathlineto{\pgfqpoint{5.678591in}{2.464913in}}%
\pgfpathlineto{\pgfqpoint{5.683523in}{2.464913in}}%
\pgfpathlineto{\pgfqpoint{5.688455in}{2.558906in}}%
\pgfpathlineto{\pgfqpoint{5.693386in}{2.629627in}}%
\pgfpathlineto{\pgfqpoint{5.698318in}{3.041821in}}%
\pgfpathlineto{\pgfqpoint{5.703250in}{2.631732in}}%
\pgfpathlineto{\pgfqpoint{5.708182in}{2.794767in}}%
\pgfpathlineto{\pgfqpoint{5.713114in}{2.592244in}}%
\pgfpathlineto{\pgfqpoint{5.718045in}{2.302007in}}%
\pgfpathlineto{\pgfqpoint{5.732841in}{2.302007in}}%
\pgfpathlineto{\pgfqpoint{5.737773in}{2.768945in}}%
\pgfpathlineto{\pgfqpoint{5.742705in}{2.345452in}}%
\pgfpathlineto{\pgfqpoint{5.747636in}{2.182161in}}%
\pgfpathlineto{\pgfqpoint{5.752568in}{2.355667in}}%
\pgfpathlineto{\pgfqpoint{5.757500in}{2.907198in}}%
\pgfpathlineto{\pgfqpoint{5.762432in}{2.318805in}}%
\pgfpathlineto{\pgfqpoint{5.767364in}{2.230196in}}%
\pgfpathlineto{\pgfqpoint{5.772295in}{2.603382in}}%
\pgfpathlineto{\pgfqpoint{5.777227in}{2.391492in}}%
\pgfpathlineto{\pgfqpoint{5.782159in}{2.391492in}}%
\pgfpathlineto{\pgfqpoint{5.787091in}{2.309598in}}%
\pgfpathlineto{\pgfqpoint{5.792023in}{2.309598in}}%
\pgfpathlineto{\pgfqpoint{5.796955in}{2.351647in}}%
\pgfpathlineto{\pgfqpoint{5.801886in}{2.351647in}}%
\pgfpathlineto{\pgfqpoint{5.806818in}{2.290821in}}%
\pgfpathlineto{\pgfqpoint{5.811750in}{2.857061in}}%
\pgfpathlineto{\pgfqpoint{5.816682in}{2.857061in}}%
\pgfpathlineto{\pgfqpoint{5.821614in}{2.783749in}}%
\pgfpathlineto{\pgfqpoint{5.826545in}{3.131832in}}%
\pgfpathlineto{\pgfqpoint{5.831477in}{2.140222in}}%
\pgfpathlineto{\pgfqpoint{5.836409in}{2.278839in}}%
\pgfpathlineto{\pgfqpoint{5.841341in}{2.278839in}}%
\pgfpathlineto{\pgfqpoint{5.846273in}{2.755815in}}%
\pgfpathlineto{\pgfqpoint{5.851205in}{2.416576in}}%
\pgfpathlineto{\pgfqpoint{5.856136in}{2.637272in}}%
\pgfpathlineto{\pgfqpoint{5.861068in}{2.637272in}}%
\pgfpathlineto{\pgfqpoint{5.866000in}{2.820047in}}%
\pgfpathlineto{\pgfqpoint{5.870932in}{2.302118in}}%
\pgfpathlineto{\pgfqpoint{5.875864in}{2.302118in}}%
\pgfpathlineto{\pgfqpoint{5.880795in}{2.790838in}}%
\pgfpathlineto{\pgfqpoint{5.885727in}{2.677756in}}%
\pgfpathlineto{\pgfqpoint{5.890659in}{2.248620in}}%
\pgfpathlineto{\pgfqpoint{5.895591in}{3.062428in}}%
\pgfpathlineto{\pgfqpoint{5.900523in}{2.952488in}}%
\pgfpathlineto{\pgfqpoint{5.905455in}{2.952488in}}%
\pgfpathlineto{\pgfqpoint{5.910386in}{2.920554in}}%
\pgfpathlineto{\pgfqpoint{5.915318in}{2.898229in}}%
\pgfpathlineto{\pgfqpoint{5.920250in}{2.536280in}}%
\pgfpathlineto{\pgfqpoint{5.925182in}{2.536280in}}%
\pgfpathlineto{\pgfqpoint{5.930114in}{2.656185in}}%
\pgfpathlineto{\pgfqpoint{5.935045in}{2.706864in}}%
\pgfpathlineto{\pgfqpoint{5.939977in}{1.905240in}}%
\pgfpathlineto{\pgfqpoint{5.944909in}{2.068802in}}%
\pgfpathlineto{\pgfqpoint{5.959705in}{2.068802in}}%
\pgfpathlineto{\pgfqpoint{5.964636in}{2.737193in}}%
\pgfpathlineto{\pgfqpoint{5.969568in}{2.737193in}}%
\pgfpathlineto{\pgfqpoint{5.974500in}{2.479736in}}%
\pgfpathlineto{\pgfqpoint{5.979432in}{2.156073in}}%
\pgfpathlineto{\pgfqpoint{5.984364in}{2.233953in}}%
\pgfpathlineto{\pgfqpoint{5.989295in}{1.931610in}}%
\pgfpathlineto{\pgfqpoint{5.994227in}{2.249958in}}%
\pgfpathlineto{\pgfqpoint{5.999159in}{2.249958in}}%
\pgfpathlineto{\pgfqpoint{6.004091in}{2.277244in}}%
\pgfpathlineto{\pgfqpoint{6.009023in}{2.277244in}}%
\pgfpathlineto{\pgfqpoint{6.013955in}{2.494313in}}%
\pgfpathlineto{\pgfqpoint{6.018886in}{2.817165in}}%
\pgfpathlineto{\pgfqpoint{6.023818in}{2.498607in}}%
\pgfpathlineto{\pgfqpoint{6.028750in}{2.476537in}}%
\pgfpathlineto{\pgfqpoint{6.038614in}{2.476537in}}%
\pgfpathlineto{\pgfqpoint{6.043545in}{3.340531in}}%
\pgfpathlineto{\pgfqpoint{6.048477in}{2.679570in}}%
\pgfpathlineto{\pgfqpoint{6.053409in}{2.388558in}}%
\pgfpathlineto{\pgfqpoint{6.053409in}{2.388558in}}%
\pgfusepath{stroke}%
\end{pgfscope}%
\begin{pgfscope}%
\pgfsetrectcap%
\pgfsetmiterjoin%
\pgfsetlinewidth{0.000000pt}%
\definecolor{currentstroke}{rgb}{1.000000,1.000000,1.000000}%
\pgfsetstrokecolor{currentstroke}%
\pgfsetdash{}{0pt}%
\pgfpathmoveto{\pgfqpoint{0.875000in}{0.440000in}}%
\pgfpathlineto{\pgfqpoint{0.875000in}{3.520000in}}%
\pgfusepath{}%
\end{pgfscope}%
\begin{pgfscope}%
\pgfsetrectcap%
\pgfsetmiterjoin%
\pgfsetlinewidth{0.000000pt}%
\definecolor{currentstroke}{rgb}{1.000000,1.000000,1.000000}%
\pgfsetstrokecolor{currentstroke}%
\pgfsetdash{}{0pt}%
\pgfpathmoveto{\pgfqpoint{6.300000in}{0.440000in}}%
\pgfpathlineto{\pgfqpoint{6.300000in}{3.520000in}}%
\pgfusepath{}%
\end{pgfscope}%
\begin{pgfscope}%
\pgfsetrectcap%
\pgfsetmiterjoin%
\pgfsetlinewidth{0.000000pt}%
\definecolor{currentstroke}{rgb}{1.000000,1.000000,1.000000}%
\pgfsetstrokecolor{currentstroke}%
\pgfsetdash{}{0pt}%
\pgfpathmoveto{\pgfqpoint{0.875000in}{0.440000in}}%
\pgfpathlineto{\pgfqpoint{6.300000in}{0.440000in}}%
\pgfusepath{}%
\end{pgfscope}%
\begin{pgfscope}%
\pgfsetrectcap%
\pgfsetmiterjoin%
\pgfsetlinewidth{0.000000pt}%
\definecolor{currentstroke}{rgb}{1.000000,1.000000,1.000000}%
\pgfsetstrokecolor{currentstroke}%
\pgfsetdash{}{0pt}%
\pgfpathmoveto{\pgfqpoint{0.875000in}{3.520000in}}%
\pgfpathlineto{\pgfqpoint{6.300000in}{3.520000in}}%
\pgfusepath{}%
\end{pgfscope}%
\begin{pgfscope}%
\definecolor{textcolor}{rgb}{0.150000,0.150000,0.150000}%
\pgfsetstrokecolor{textcolor}%
\pgfsetfillcolor{textcolor}%
\pgftext[x=3.500000in,y=3.920000in,,top]{\color{textcolor}\rmfamily\fontsize{12.000000}{14.400000}\selectfont Muestreo con propuesta beta y \(\displaystyle n=40\)}%
\end{pgfscope}%
\end{pgfpicture}%
\makeatother%
\endgroup%

        %% Creator: Matplotlib, PGF backend
%%
%% To include the figure in your LaTeX document, write
%%   \input{<filename>.pgf}
%%
%% Make sure the required packages are loaded in your preamble
%%   \usepackage{pgf}
%%
%% Also ensure that all the required font packages are loaded; for instance,
%% the lmodern package is sometimes necessary when using math font.
%%   \usepackage{lmodern}
%%
%% Figures using additional raster images can only be included by \input if
%% they are in the same directory as the main LaTeX file. For loading figures
%% from other directories you can use the `import` package
%%   \usepackage{import}
%%
%% and then include the figures with
%%   \import{<path to file>}{<filename>.pgf}
%%
%% Matplotlib used the following preamble
%%   
%%   \makeatletter\@ifpackageloaded{underscore}{}{\usepackage[strings]{underscore}}\makeatother
%%
\begingroup%
\makeatletter%
\begin{pgfpicture}%
\pgfpathrectangle{\pgfpointorigin}{\pgfqpoint{7.000000in}{4.000000in}}%
\pgfusepath{use as bounding box, clip}%
\begin{pgfscope}%
\pgfsetbuttcap%
\pgfsetmiterjoin%
\definecolor{currentfill}{rgb}{1.000000,1.000000,1.000000}%
\pgfsetfillcolor{currentfill}%
\pgfsetlinewidth{0.000000pt}%
\definecolor{currentstroke}{rgb}{1.000000,1.000000,1.000000}%
\pgfsetstrokecolor{currentstroke}%
\pgfsetdash{}{0pt}%
\pgfpathmoveto{\pgfqpoint{0.000000in}{0.000000in}}%
\pgfpathlineto{\pgfqpoint{7.000000in}{0.000000in}}%
\pgfpathlineto{\pgfqpoint{7.000000in}{4.000000in}}%
\pgfpathlineto{\pgfqpoint{0.000000in}{4.000000in}}%
\pgfpathlineto{\pgfqpoint{0.000000in}{0.000000in}}%
\pgfpathclose%
\pgfusepath{fill}%
\end{pgfscope}%
\begin{pgfscope}%
\pgfsetbuttcap%
\pgfsetmiterjoin%
\definecolor{currentfill}{rgb}{0.917647,0.917647,0.949020}%
\pgfsetfillcolor{currentfill}%
\pgfsetlinewidth{0.000000pt}%
\definecolor{currentstroke}{rgb}{0.000000,0.000000,0.000000}%
\pgfsetstrokecolor{currentstroke}%
\pgfsetstrokeopacity{0.000000}%
\pgfsetdash{}{0pt}%
\pgfpathmoveto{\pgfqpoint{0.875000in}{0.440000in}}%
\pgfpathlineto{\pgfqpoint{6.300000in}{0.440000in}}%
\pgfpathlineto{\pgfqpoint{6.300000in}{3.520000in}}%
\pgfpathlineto{\pgfqpoint{0.875000in}{3.520000in}}%
\pgfpathlineto{\pgfqpoint{0.875000in}{0.440000in}}%
\pgfpathclose%
\pgfusepath{fill}%
\end{pgfscope}%
\begin{pgfscope}%
\pgfpathrectangle{\pgfqpoint{0.875000in}{0.440000in}}{\pgfqpoint{5.425000in}{3.080000in}}%
\pgfusepath{clip}%
\pgfsetroundcap%
\pgfsetroundjoin%
\pgfsetlinewidth{1.003750pt}%
\definecolor{currentstroke}{rgb}{1.000000,1.000000,1.000000}%
\pgfsetstrokecolor{currentstroke}%
\pgfsetdash{}{0pt}%
\pgfpathmoveto{\pgfqpoint{1.121591in}{0.440000in}}%
\pgfpathlineto{\pgfqpoint{1.121591in}{3.520000in}}%
\pgfusepath{stroke}%
\end{pgfscope}%
\begin{pgfscope}%
\definecolor{textcolor}{rgb}{0.150000,0.150000,0.150000}%
\pgfsetstrokecolor{textcolor}%
\pgfsetfillcolor{textcolor}%
\pgftext[x=1.121591in,y=0.342778in,,top]{\color{textcolor}\rmfamily\fontsize{10.000000}{12.000000}\selectfont \(\displaystyle {0}\)}%
\end{pgfscope}%
\begin{pgfscope}%
\pgfpathrectangle{\pgfqpoint{0.875000in}{0.440000in}}{\pgfqpoint{5.425000in}{3.080000in}}%
\pgfusepath{clip}%
\pgfsetroundcap%
\pgfsetroundjoin%
\pgfsetlinewidth{1.003750pt}%
\definecolor{currentstroke}{rgb}{1.000000,1.000000,1.000000}%
\pgfsetstrokecolor{currentstroke}%
\pgfsetdash{}{0pt}%
\pgfpathmoveto{\pgfqpoint{2.107955in}{0.440000in}}%
\pgfpathlineto{\pgfqpoint{2.107955in}{3.520000in}}%
\pgfusepath{stroke}%
\end{pgfscope}%
\begin{pgfscope}%
\definecolor{textcolor}{rgb}{0.150000,0.150000,0.150000}%
\pgfsetstrokecolor{textcolor}%
\pgfsetfillcolor{textcolor}%
\pgftext[x=2.107955in,y=0.342778in,,top]{\color{textcolor}\rmfamily\fontsize{10.000000}{12.000000}\selectfont \(\displaystyle {200}\)}%
\end{pgfscope}%
\begin{pgfscope}%
\pgfpathrectangle{\pgfqpoint{0.875000in}{0.440000in}}{\pgfqpoint{5.425000in}{3.080000in}}%
\pgfusepath{clip}%
\pgfsetroundcap%
\pgfsetroundjoin%
\pgfsetlinewidth{1.003750pt}%
\definecolor{currentstroke}{rgb}{1.000000,1.000000,1.000000}%
\pgfsetstrokecolor{currentstroke}%
\pgfsetdash{}{0pt}%
\pgfpathmoveto{\pgfqpoint{3.094318in}{0.440000in}}%
\pgfpathlineto{\pgfqpoint{3.094318in}{3.520000in}}%
\pgfusepath{stroke}%
\end{pgfscope}%
\begin{pgfscope}%
\definecolor{textcolor}{rgb}{0.150000,0.150000,0.150000}%
\pgfsetstrokecolor{textcolor}%
\pgfsetfillcolor{textcolor}%
\pgftext[x=3.094318in,y=0.342778in,,top]{\color{textcolor}\rmfamily\fontsize{10.000000}{12.000000}\selectfont \(\displaystyle {400}\)}%
\end{pgfscope}%
\begin{pgfscope}%
\pgfpathrectangle{\pgfqpoint{0.875000in}{0.440000in}}{\pgfqpoint{5.425000in}{3.080000in}}%
\pgfusepath{clip}%
\pgfsetroundcap%
\pgfsetroundjoin%
\pgfsetlinewidth{1.003750pt}%
\definecolor{currentstroke}{rgb}{1.000000,1.000000,1.000000}%
\pgfsetstrokecolor{currentstroke}%
\pgfsetdash{}{0pt}%
\pgfpathmoveto{\pgfqpoint{4.080682in}{0.440000in}}%
\pgfpathlineto{\pgfqpoint{4.080682in}{3.520000in}}%
\pgfusepath{stroke}%
\end{pgfscope}%
\begin{pgfscope}%
\definecolor{textcolor}{rgb}{0.150000,0.150000,0.150000}%
\pgfsetstrokecolor{textcolor}%
\pgfsetfillcolor{textcolor}%
\pgftext[x=4.080682in,y=0.342778in,,top]{\color{textcolor}\rmfamily\fontsize{10.000000}{12.000000}\selectfont \(\displaystyle {600}\)}%
\end{pgfscope}%
\begin{pgfscope}%
\pgfpathrectangle{\pgfqpoint{0.875000in}{0.440000in}}{\pgfqpoint{5.425000in}{3.080000in}}%
\pgfusepath{clip}%
\pgfsetroundcap%
\pgfsetroundjoin%
\pgfsetlinewidth{1.003750pt}%
\definecolor{currentstroke}{rgb}{1.000000,1.000000,1.000000}%
\pgfsetstrokecolor{currentstroke}%
\pgfsetdash{}{0pt}%
\pgfpathmoveto{\pgfqpoint{5.067045in}{0.440000in}}%
\pgfpathlineto{\pgfqpoint{5.067045in}{3.520000in}}%
\pgfusepath{stroke}%
\end{pgfscope}%
\begin{pgfscope}%
\definecolor{textcolor}{rgb}{0.150000,0.150000,0.150000}%
\pgfsetstrokecolor{textcolor}%
\pgfsetfillcolor{textcolor}%
\pgftext[x=5.067045in,y=0.342778in,,top]{\color{textcolor}\rmfamily\fontsize{10.000000}{12.000000}\selectfont \(\displaystyle {800}\)}%
\end{pgfscope}%
\begin{pgfscope}%
\pgfpathrectangle{\pgfqpoint{0.875000in}{0.440000in}}{\pgfqpoint{5.425000in}{3.080000in}}%
\pgfusepath{clip}%
\pgfsetroundcap%
\pgfsetroundjoin%
\pgfsetlinewidth{1.003750pt}%
\definecolor{currentstroke}{rgb}{1.000000,1.000000,1.000000}%
\pgfsetstrokecolor{currentstroke}%
\pgfsetdash{}{0pt}%
\pgfpathmoveto{\pgfqpoint{6.053409in}{0.440000in}}%
\pgfpathlineto{\pgfqpoint{6.053409in}{3.520000in}}%
\pgfusepath{stroke}%
\end{pgfscope}%
\begin{pgfscope}%
\definecolor{textcolor}{rgb}{0.150000,0.150000,0.150000}%
\pgfsetstrokecolor{textcolor}%
\pgfsetfillcolor{textcolor}%
\pgftext[x=6.053409in,y=0.342778in,,top]{\color{textcolor}\rmfamily\fontsize{10.000000}{12.000000}\selectfont \(\displaystyle {1000}\)}%
\end{pgfscope}%
\begin{pgfscope}%
\definecolor{textcolor}{rgb}{0.150000,0.150000,0.150000}%
\pgfsetstrokecolor{textcolor}%
\pgfsetfillcolor{textcolor}%
\pgftext[x=3.587500in,y=0.163766in,,top]{\color{textcolor}\rmfamily\fontsize{11.000000}{13.200000}\selectfont Paso de la cadena (\(\displaystyle t\))}%
\end{pgfscope}%
\begin{pgfscope}%
\pgfpathrectangle{\pgfqpoint{0.875000in}{0.440000in}}{\pgfqpoint{5.425000in}{3.080000in}}%
\pgfusepath{clip}%
\pgfsetroundcap%
\pgfsetroundjoin%
\pgfsetlinewidth{1.003750pt}%
\definecolor{currentstroke}{rgb}{1.000000,1.000000,1.000000}%
\pgfsetstrokecolor{currentstroke}%
\pgfsetdash{}{0pt}%
\pgfpathmoveto{\pgfqpoint{0.875000in}{0.504053in}}%
\pgfpathlineto{\pgfqpoint{6.300000in}{0.504053in}}%
\pgfusepath{stroke}%
\end{pgfscope}%
\begin{pgfscope}%
\definecolor{textcolor}{rgb}{0.150000,0.150000,0.150000}%
\pgfsetstrokecolor{textcolor}%
\pgfsetfillcolor{textcolor}%
\pgftext[x=0.530863in, y=0.455827in, left, base]{\color{textcolor}\rmfamily\fontsize{10.000000}{12.000000}\selectfont \(\displaystyle {0.10}\)}%
\end{pgfscope}%
\begin{pgfscope}%
\pgfpathrectangle{\pgfqpoint{0.875000in}{0.440000in}}{\pgfqpoint{5.425000in}{3.080000in}}%
\pgfusepath{clip}%
\pgfsetroundcap%
\pgfsetroundjoin%
\pgfsetlinewidth{1.003750pt}%
\definecolor{currentstroke}{rgb}{1.000000,1.000000,1.000000}%
\pgfsetstrokecolor{currentstroke}%
\pgfsetdash{}{0pt}%
\pgfpathmoveto{\pgfqpoint{0.875000in}{0.866785in}}%
\pgfpathlineto{\pgfqpoint{6.300000in}{0.866785in}}%
\pgfusepath{stroke}%
\end{pgfscope}%
\begin{pgfscope}%
\definecolor{textcolor}{rgb}{0.150000,0.150000,0.150000}%
\pgfsetstrokecolor{textcolor}%
\pgfsetfillcolor{textcolor}%
\pgftext[x=0.530863in, y=0.818559in, left, base]{\color{textcolor}\rmfamily\fontsize{10.000000}{12.000000}\selectfont \(\displaystyle {0.15}\)}%
\end{pgfscope}%
\begin{pgfscope}%
\pgfpathrectangle{\pgfqpoint{0.875000in}{0.440000in}}{\pgfqpoint{5.425000in}{3.080000in}}%
\pgfusepath{clip}%
\pgfsetroundcap%
\pgfsetroundjoin%
\pgfsetlinewidth{1.003750pt}%
\definecolor{currentstroke}{rgb}{1.000000,1.000000,1.000000}%
\pgfsetstrokecolor{currentstroke}%
\pgfsetdash{}{0pt}%
\pgfpathmoveto{\pgfqpoint{0.875000in}{1.229516in}}%
\pgfpathlineto{\pgfqpoint{6.300000in}{1.229516in}}%
\pgfusepath{stroke}%
\end{pgfscope}%
\begin{pgfscope}%
\definecolor{textcolor}{rgb}{0.150000,0.150000,0.150000}%
\pgfsetstrokecolor{textcolor}%
\pgfsetfillcolor{textcolor}%
\pgftext[x=0.530863in, y=1.181291in, left, base]{\color{textcolor}\rmfamily\fontsize{10.000000}{12.000000}\selectfont \(\displaystyle {0.20}\)}%
\end{pgfscope}%
\begin{pgfscope}%
\pgfpathrectangle{\pgfqpoint{0.875000in}{0.440000in}}{\pgfqpoint{5.425000in}{3.080000in}}%
\pgfusepath{clip}%
\pgfsetroundcap%
\pgfsetroundjoin%
\pgfsetlinewidth{1.003750pt}%
\definecolor{currentstroke}{rgb}{1.000000,1.000000,1.000000}%
\pgfsetstrokecolor{currentstroke}%
\pgfsetdash{}{0pt}%
\pgfpathmoveto{\pgfqpoint{0.875000in}{1.592248in}}%
\pgfpathlineto{\pgfqpoint{6.300000in}{1.592248in}}%
\pgfusepath{stroke}%
\end{pgfscope}%
\begin{pgfscope}%
\definecolor{textcolor}{rgb}{0.150000,0.150000,0.150000}%
\pgfsetstrokecolor{textcolor}%
\pgfsetfillcolor{textcolor}%
\pgftext[x=0.530863in, y=1.544023in, left, base]{\color{textcolor}\rmfamily\fontsize{10.000000}{12.000000}\selectfont \(\displaystyle {0.25}\)}%
\end{pgfscope}%
\begin{pgfscope}%
\pgfpathrectangle{\pgfqpoint{0.875000in}{0.440000in}}{\pgfqpoint{5.425000in}{3.080000in}}%
\pgfusepath{clip}%
\pgfsetroundcap%
\pgfsetroundjoin%
\pgfsetlinewidth{1.003750pt}%
\definecolor{currentstroke}{rgb}{1.000000,1.000000,1.000000}%
\pgfsetstrokecolor{currentstroke}%
\pgfsetdash{}{0pt}%
\pgfpathmoveto{\pgfqpoint{0.875000in}{1.954980in}}%
\pgfpathlineto{\pgfqpoint{6.300000in}{1.954980in}}%
\pgfusepath{stroke}%
\end{pgfscope}%
\begin{pgfscope}%
\definecolor{textcolor}{rgb}{0.150000,0.150000,0.150000}%
\pgfsetstrokecolor{textcolor}%
\pgfsetfillcolor{textcolor}%
\pgftext[x=0.530863in, y=1.906755in, left, base]{\color{textcolor}\rmfamily\fontsize{10.000000}{12.000000}\selectfont \(\displaystyle {0.30}\)}%
\end{pgfscope}%
\begin{pgfscope}%
\pgfpathrectangle{\pgfqpoint{0.875000in}{0.440000in}}{\pgfqpoint{5.425000in}{3.080000in}}%
\pgfusepath{clip}%
\pgfsetroundcap%
\pgfsetroundjoin%
\pgfsetlinewidth{1.003750pt}%
\definecolor{currentstroke}{rgb}{1.000000,1.000000,1.000000}%
\pgfsetstrokecolor{currentstroke}%
\pgfsetdash{}{0pt}%
\pgfpathmoveto{\pgfqpoint{0.875000in}{2.317712in}}%
\pgfpathlineto{\pgfqpoint{6.300000in}{2.317712in}}%
\pgfusepath{stroke}%
\end{pgfscope}%
\begin{pgfscope}%
\definecolor{textcolor}{rgb}{0.150000,0.150000,0.150000}%
\pgfsetstrokecolor{textcolor}%
\pgfsetfillcolor{textcolor}%
\pgftext[x=0.530863in, y=2.269487in, left, base]{\color{textcolor}\rmfamily\fontsize{10.000000}{12.000000}\selectfont \(\displaystyle {0.35}\)}%
\end{pgfscope}%
\begin{pgfscope}%
\pgfpathrectangle{\pgfqpoint{0.875000in}{0.440000in}}{\pgfqpoint{5.425000in}{3.080000in}}%
\pgfusepath{clip}%
\pgfsetroundcap%
\pgfsetroundjoin%
\pgfsetlinewidth{1.003750pt}%
\definecolor{currentstroke}{rgb}{1.000000,1.000000,1.000000}%
\pgfsetstrokecolor{currentstroke}%
\pgfsetdash{}{0pt}%
\pgfpathmoveto{\pgfqpoint{0.875000in}{2.680444in}}%
\pgfpathlineto{\pgfqpoint{6.300000in}{2.680444in}}%
\pgfusepath{stroke}%
\end{pgfscope}%
\begin{pgfscope}%
\definecolor{textcolor}{rgb}{0.150000,0.150000,0.150000}%
\pgfsetstrokecolor{textcolor}%
\pgfsetfillcolor{textcolor}%
\pgftext[x=0.530863in, y=2.632219in, left, base]{\color{textcolor}\rmfamily\fontsize{10.000000}{12.000000}\selectfont \(\displaystyle {0.40}\)}%
\end{pgfscope}%
\begin{pgfscope}%
\pgfpathrectangle{\pgfqpoint{0.875000in}{0.440000in}}{\pgfqpoint{5.425000in}{3.080000in}}%
\pgfusepath{clip}%
\pgfsetroundcap%
\pgfsetroundjoin%
\pgfsetlinewidth{1.003750pt}%
\definecolor{currentstroke}{rgb}{1.000000,1.000000,1.000000}%
\pgfsetstrokecolor{currentstroke}%
\pgfsetdash{}{0pt}%
\pgfpathmoveto{\pgfqpoint{0.875000in}{3.043176in}}%
\pgfpathlineto{\pgfqpoint{6.300000in}{3.043176in}}%
\pgfusepath{stroke}%
\end{pgfscope}%
\begin{pgfscope}%
\definecolor{textcolor}{rgb}{0.150000,0.150000,0.150000}%
\pgfsetstrokecolor{textcolor}%
\pgfsetfillcolor{textcolor}%
\pgftext[x=0.530863in, y=2.994951in, left, base]{\color{textcolor}\rmfamily\fontsize{10.000000}{12.000000}\selectfont \(\displaystyle {0.45}\)}%
\end{pgfscope}%
\begin{pgfscope}%
\pgfpathrectangle{\pgfqpoint{0.875000in}{0.440000in}}{\pgfqpoint{5.425000in}{3.080000in}}%
\pgfusepath{clip}%
\pgfsetroundcap%
\pgfsetroundjoin%
\pgfsetlinewidth{1.003750pt}%
\definecolor{currentstroke}{rgb}{1.000000,1.000000,1.000000}%
\pgfsetstrokecolor{currentstroke}%
\pgfsetdash{}{0pt}%
\pgfpathmoveto{\pgfqpoint{0.875000in}{3.405908in}}%
\pgfpathlineto{\pgfqpoint{6.300000in}{3.405908in}}%
\pgfusepath{stroke}%
\end{pgfscope}%
\begin{pgfscope}%
\definecolor{textcolor}{rgb}{0.150000,0.150000,0.150000}%
\pgfsetstrokecolor{textcolor}%
\pgfsetfillcolor{textcolor}%
\pgftext[x=0.530863in, y=3.357682in, left, base]{\color{textcolor}\rmfamily\fontsize{10.000000}{12.000000}\selectfont \(\displaystyle {0.50}\)}%
\end{pgfscope}%
\begin{pgfscope}%
\definecolor{textcolor}{rgb}{0.150000,0.150000,0.150000}%
\pgfsetstrokecolor{textcolor}%
\pgfsetfillcolor{textcolor}%
\pgftext[x=0.475308in,y=1.980000in,,bottom,rotate=90.000000]{\color{textcolor}\rmfamily\fontsize{11.000000}{13.200000}\selectfont Valor de \(\displaystyle X_t\)}%
\end{pgfscope}%
\begin{pgfscope}%
\pgfpathrectangle{\pgfqpoint{0.875000in}{0.440000in}}{\pgfqpoint{5.425000in}{3.080000in}}%
\pgfusepath{clip}%
\pgfsetroundcap%
\pgfsetroundjoin%
\pgfsetlinewidth{1.756562pt}%
\definecolor{currentstroke}{rgb}{0.298039,0.447059,0.690196}%
\pgfsetstrokecolor{currentstroke}%
\pgfsetdash{}{0pt}%
\pgfpathmoveto{\pgfqpoint{1.121591in}{2.881243in}}%
\pgfpathlineto{\pgfqpoint{1.161045in}{2.881243in}}%
\pgfpathlineto{\pgfqpoint{1.165977in}{2.095797in}}%
\pgfpathlineto{\pgfqpoint{1.170909in}{2.818416in}}%
\pgfpathlineto{\pgfqpoint{1.190636in}{2.818416in}}%
\pgfpathlineto{\pgfqpoint{1.195568in}{1.702213in}}%
\pgfpathlineto{\pgfqpoint{1.200500in}{1.702213in}}%
\pgfpathlineto{\pgfqpoint{1.205432in}{2.826725in}}%
\pgfpathlineto{\pgfqpoint{1.210364in}{2.826725in}}%
\pgfpathlineto{\pgfqpoint{1.215295in}{3.227344in}}%
\pgfpathlineto{\pgfqpoint{1.220227in}{1.978392in}}%
\pgfpathlineto{\pgfqpoint{1.225159in}{1.978392in}}%
\pgfpathlineto{\pgfqpoint{1.230091in}{1.643640in}}%
\pgfpathlineto{\pgfqpoint{1.235023in}{1.838844in}}%
\pgfpathlineto{\pgfqpoint{1.239955in}{2.768742in}}%
\pgfpathlineto{\pgfqpoint{1.244886in}{1.593998in}}%
\pgfpathlineto{\pgfqpoint{1.249818in}{2.923869in}}%
\pgfpathlineto{\pgfqpoint{1.254750in}{2.385612in}}%
\pgfpathlineto{\pgfqpoint{1.284341in}{2.385612in}}%
\pgfpathlineto{\pgfqpoint{1.294205in}{2.522852in}}%
\pgfpathlineto{\pgfqpoint{1.299136in}{2.522852in}}%
\pgfpathlineto{\pgfqpoint{1.304068in}{2.703289in}}%
\pgfpathlineto{\pgfqpoint{1.309000in}{2.287828in}}%
\pgfpathlineto{\pgfqpoint{1.313932in}{2.541449in}}%
\pgfpathlineto{\pgfqpoint{1.323795in}{2.541449in}}%
\pgfpathlineto{\pgfqpoint{1.328727in}{2.723422in}}%
\pgfpathlineto{\pgfqpoint{1.333659in}{2.563639in}}%
\pgfpathlineto{\pgfqpoint{1.338591in}{2.563639in}}%
\pgfpathlineto{\pgfqpoint{1.343523in}{2.674647in}}%
\pgfpathlineto{\pgfqpoint{1.348455in}{1.793049in}}%
\pgfpathlineto{\pgfqpoint{1.353386in}{2.455930in}}%
\pgfpathlineto{\pgfqpoint{1.358318in}{2.455930in}}%
\pgfpathlineto{\pgfqpoint{1.363250in}{2.832383in}}%
\pgfpathlineto{\pgfqpoint{1.368182in}{2.319015in}}%
\pgfpathlineto{\pgfqpoint{1.373114in}{1.225712in}}%
\pgfpathlineto{\pgfqpoint{1.378045in}{1.225712in}}%
\pgfpathlineto{\pgfqpoint{1.382977in}{3.245867in}}%
\pgfpathlineto{\pgfqpoint{1.387909in}{1.566094in}}%
\pgfpathlineto{\pgfqpoint{1.402705in}{1.566094in}}%
\pgfpathlineto{\pgfqpoint{1.407636in}{2.536551in}}%
\pgfpathlineto{\pgfqpoint{1.432295in}{2.536551in}}%
\pgfpathlineto{\pgfqpoint{1.437227in}{2.611807in}}%
\pgfpathlineto{\pgfqpoint{1.442159in}{2.324728in}}%
\pgfpathlineto{\pgfqpoint{1.447091in}{2.324728in}}%
\pgfpathlineto{\pgfqpoint{1.452023in}{2.370098in}}%
\pgfpathlineto{\pgfqpoint{1.481614in}{2.370098in}}%
\pgfpathlineto{\pgfqpoint{1.486545in}{2.642170in}}%
\pgfpathlineto{\pgfqpoint{1.491477in}{2.043575in}}%
\pgfpathlineto{\pgfqpoint{1.516136in}{2.043575in}}%
\pgfpathlineto{\pgfqpoint{1.521068in}{2.476527in}}%
\pgfpathlineto{\pgfqpoint{1.526000in}{2.476527in}}%
\pgfpathlineto{\pgfqpoint{1.530932in}{2.656847in}}%
\pgfpathlineto{\pgfqpoint{1.540795in}{2.656847in}}%
\pgfpathlineto{\pgfqpoint{1.545727in}{1.656404in}}%
\pgfpathlineto{\pgfqpoint{1.555591in}{1.656404in}}%
\pgfpathlineto{\pgfqpoint{1.560523in}{1.712536in}}%
\pgfpathlineto{\pgfqpoint{1.565455in}{2.688367in}}%
\pgfpathlineto{\pgfqpoint{1.599977in}{2.688367in}}%
\pgfpathlineto{\pgfqpoint{1.604909in}{2.602021in}}%
\pgfpathlineto{\pgfqpoint{1.609841in}{2.699811in}}%
\pgfpathlineto{\pgfqpoint{1.614773in}{2.833874in}}%
\pgfpathlineto{\pgfqpoint{1.619705in}{1.710161in}}%
\pgfpathlineto{\pgfqpoint{1.624636in}{2.685648in}}%
\pgfpathlineto{\pgfqpoint{1.634500in}{2.685648in}}%
\pgfpathlineto{\pgfqpoint{1.639432in}{2.511273in}}%
\pgfpathlineto{\pgfqpoint{1.649295in}{2.511273in}}%
\pgfpathlineto{\pgfqpoint{1.654227in}{1.778674in}}%
\pgfpathlineto{\pgfqpoint{1.669023in}{1.778674in}}%
\pgfpathlineto{\pgfqpoint{1.673955in}{1.187416in}}%
\pgfpathlineto{\pgfqpoint{1.678886in}{1.187416in}}%
\pgfpathlineto{\pgfqpoint{1.683818in}{1.938473in}}%
\pgfpathlineto{\pgfqpoint{1.688750in}{3.268655in}}%
\pgfpathlineto{\pgfqpoint{1.703545in}{3.268655in}}%
\pgfpathlineto{\pgfqpoint{1.708477in}{1.709052in}}%
\pgfpathlineto{\pgfqpoint{1.713409in}{1.612438in}}%
\pgfpathlineto{\pgfqpoint{1.718341in}{2.571532in}}%
\pgfpathlineto{\pgfqpoint{1.723273in}{2.915744in}}%
\pgfpathlineto{\pgfqpoint{1.728205in}{2.915744in}}%
\pgfpathlineto{\pgfqpoint{1.733136in}{2.584304in}}%
\pgfpathlineto{\pgfqpoint{1.743000in}{2.584304in}}%
\pgfpathlineto{\pgfqpoint{1.747932in}{2.328409in}}%
\pgfpathlineto{\pgfqpoint{1.757795in}{2.328409in}}%
\pgfpathlineto{\pgfqpoint{1.762727in}{2.283217in}}%
\pgfpathlineto{\pgfqpoint{1.767659in}{1.661311in}}%
\pgfpathlineto{\pgfqpoint{1.772591in}{1.661311in}}%
\pgfpathlineto{\pgfqpoint{1.777523in}{1.987853in}}%
\pgfpathlineto{\pgfqpoint{1.787386in}{1.987853in}}%
\pgfpathlineto{\pgfqpoint{1.792318in}{1.738301in}}%
\pgfpathlineto{\pgfqpoint{1.802182in}{1.738301in}}%
\pgfpathlineto{\pgfqpoint{1.807114in}{2.200420in}}%
\pgfpathlineto{\pgfqpoint{1.812045in}{2.200420in}}%
\pgfpathlineto{\pgfqpoint{1.816977in}{2.748297in}}%
\pgfpathlineto{\pgfqpoint{1.821909in}{1.444156in}}%
\pgfpathlineto{\pgfqpoint{1.826841in}{1.931186in}}%
\pgfpathlineto{\pgfqpoint{1.831773in}{3.069061in}}%
\pgfpathlineto{\pgfqpoint{1.836705in}{1.386673in}}%
\pgfpathlineto{\pgfqpoint{1.841636in}{1.662169in}}%
\pgfpathlineto{\pgfqpoint{1.851500in}{1.662169in}}%
\pgfpathlineto{\pgfqpoint{1.856432in}{2.723133in}}%
\pgfpathlineto{\pgfqpoint{1.871227in}{2.723133in}}%
\pgfpathlineto{\pgfqpoint{1.876159in}{1.512812in}}%
\pgfpathlineto{\pgfqpoint{1.881091in}{1.196079in}}%
\pgfpathlineto{\pgfqpoint{1.886023in}{2.857890in}}%
\pgfpathlineto{\pgfqpoint{1.890955in}{3.380000in}}%
\pgfpathlineto{\pgfqpoint{1.905750in}{3.380000in}}%
\pgfpathlineto{\pgfqpoint{1.915614in}{2.750703in}}%
\pgfpathlineto{\pgfqpoint{1.930409in}{2.750703in}}%
\pgfpathlineto{\pgfqpoint{1.935341in}{1.516731in}}%
\pgfpathlineto{\pgfqpoint{1.940273in}{1.516731in}}%
\pgfpathlineto{\pgfqpoint{1.945205in}{1.835831in}}%
\pgfpathlineto{\pgfqpoint{1.950136in}{1.955430in}}%
\pgfpathlineto{\pgfqpoint{1.955068in}{1.955430in}}%
\pgfpathlineto{\pgfqpoint{1.960000in}{2.818146in}}%
\pgfpathlineto{\pgfqpoint{1.964932in}{2.818146in}}%
\pgfpathlineto{\pgfqpoint{1.969864in}{2.341036in}}%
\pgfpathlineto{\pgfqpoint{1.984659in}{2.341036in}}%
\pgfpathlineto{\pgfqpoint{1.989591in}{2.032386in}}%
\pgfpathlineto{\pgfqpoint{1.994523in}{1.931007in}}%
\pgfpathlineto{\pgfqpoint{1.999455in}{1.677230in}}%
\pgfpathlineto{\pgfqpoint{2.014250in}{1.677230in}}%
\pgfpathlineto{\pgfqpoint{2.019182in}{2.557799in}}%
\pgfpathlineto{\pgfqpoint{2.038909in}{2.557799in}}%
\pgfpathlineto{\pgfqpoint{2.043841in}{1.995385in}}%
\pgfpathlineto{\pgfqpoint{2.068500in}{1.995385in}}%
\pgfpathlineto{\pgfqpoint{2.073432in}{1.749033in}}%
\pgfpathlineto{\pgfqpoint{2.078364in}{1.859318in}}%
\pgfpathlineto{\pgfqpoint{2.083295in}{1.859318in}}%
\pgfpathlineto{\pgfqpoint{2.088227in}{2.069926in}}%
\pgfpathlineto{\pgfqpoint{2.093159in}{2.069926in}}%
\pgfpathlineto{\pgfqpoint{2.098091in}{1.588514in}}%
\pgfpathlineto{\pgfqpoint{2.103023in}{2.830963in}}%
\pgfpathlineto{\pgfqpoint{2.117818in}{2.830963in}}%
\pgfpathlineto{\pgfqpoint{2.122750in}{1.605115in}}%
\pgfpathlineto{\pgfqpoint{2.142477in}{1.605115in}}%
\pgfpathlineto{\pgfqpoint{2.147409in}{2.755989in}}%
\pgfpathlineto{\pgfqpoint{2.152341in}{2.755989in}}%
\pgfpathlineto{\pgfqpoint{2.157273in}{3.036956in}}%
\pgfpathlineto{\pgfqpoint{2.177000in}{3.036956in}}%
\pgfpathlineto{\pgfqpoint{2.181932in}{1.106520in}}%
\pgfpathlineto{\pgfqpoint{2.186864in}{1.106520in}}%
\pgfpathlineto{\pgfqpoint{2.191795in}{3.136444in}}%
\pgfpathlineto{\pgfqpoint{2.196727in}{3.136444in}}%
\pgfpathlineto{\pgfqpoint{2.201659in}{3.224774in}}%
\pgfpathlineto{\pgfqpoint{2.206591in}{1.201122in}}%
\pgfpathlineto{\pgfqpoint{2.211523in}{2.295434in}}%
\pgfpathlineto{\pgfqpoint{2.226318in}{2.295434in}}%
\pgfpathlineto{\pgfqpoint{2.231250in}{2.350677in}}%
\pgfpathlineto{\pgfqpoint{2.236182in}{2.831392in}}%
\pgfpathlineto{\pgfqpoint{2.241114in}{1.726658in}}%
\pgfpathlineto{\pgfqpoint{2.246045in}{1.941507in}}%
\pgfpathlineto{\pgfqpoint{2.250977in}{1.941507in}}%
\pgfpathlineto{\pgfqpoint{2.255909in}{2.092944in}}%
\pgfpathlineto{\pgfqpoint{2.280568in}{2.092944in}}%
\pgfpathlineto{\pgfqpoint{2.285500in}{2.644828in}}%
\pgfpathlineto{\pgfqpoint{2.310159in}{2.644828in}}%
\pgfpathlineto{\pgfqpoint{2.315091in}{2.031072in}}%
\pgfpathlineto{\pgfqpoint{2.320023in}{2.104279in}}%
\pgfpathlineto{\pgfqpoint{2.324955in}{2.348335in}}%
\pgfpathlineto{\pgfqpoint{2.349614in}{2.348335in}}%
\pgfpathlineto{\pgfqpoint{2.354545in}{1.674391in}}%
\pgfpathlineto{\pgfqpoint{2.359477in}{2.164285in}}%
\pgfpathlineto{\pgfqpoint{2.364409in}{2.279867in}}%
\pgfpathlineto{\pgfqpoint{2.369341in}{2.601077in}}%
\pgfpathlineto{\pgfqpoint{2.374273in}{1.469282in}}%
\pgfpathlineto{\pgfqpoint{2.379205in}{2.133714in}}%
\pgfpathlineto{\pgfqpoint{2.394000in}{2.133714in}}%
\pgfpathlineto{\pgfqpoint{2.398932in}{2.564911in}}%
\pgfpathlineto{\pgfqpoint{2.408795in}{2.564911in}}%
\pgfpathlineto{\pgfqpoint{2.413727in}{1.860272in}}%
\pgfpathlineto{\pgfqpoint{2.443318in}{1.860272in}}%
\pgfpathlineto{\pgfqpoint{2.448250in}{2.485075in}}%
\pgfpathlineto{\pgfqpoint{2.453182in}{2.320105in}}%
\pgfpathlineto{\pgfqpoint{2.458114in}{1.955755in}}%
\pgfpathlineto{\pgfqpoint{2.463045in}{1.782086in}}%
\pgfpathlineto{\pgfqpoint{2.467977in}{1.782086in}}%
\pgfpathlineto{\pgfqpoint{2.472909in}{1.951805in}}%
\pgfpathlineto{\pgfqpoint{2.477841in}{2.456585in}}%
\pgfpathlineto{\pgfqpoint{2.487705in}{2.456585in}}%
\pgfpathlineto{\pgfqpoint{2.492636in}{2.182073in}}%
\pgfpathlineto{\pgfqpoint{2.512364in}{2.182073in}}%
\pgfpathlineto{\pgfqpoint{2.517295in}{1.840280in}}%
\pgfpathlineto{\pgfqpoint{2.522227in}{3.151626in}}%
\pgfpathlineto{\pgfqpoint{2.527159in}{2.567872in}}%
\pgfpathlineto{\pgfqpoint{2.532091in}{2.567872in}}%
\pgfpathlineto{\pgfqpoint{2.537023in}{2.244162in}}%
\pgfpathlineto{\pgfqpoint{2.571545in}{2.244162in}}%
\pgfpathlineto{\pgfqpoint{2.576477in}{2.685363in}}%
\pgfpathlineto{\pgfqpoint{2.591273in}{2.685363in}}%
\pgfpathlineto{\pgfqpoint{2.596205in}{1.615755in}}%
\pgfpathlineto{\pgfqpoint{2.606068in}{1.615755in}}%
\pgfpathlineto{\pgfqpoint{2.611000in}{2.738581in}}%
\pgfpathlineto{\pgfqpoint{2.615932in}{2.830966in}}%
\pgfpathlineto{\pgfqpoint{2.620864in}{1.734873in}}%
\pgfpathlineto{\pgfqpoint{2.630727in}{1.734873in}}%
\pgfpathlineto{\pgfqpoint{2.635659in}{2.813611in}}%
\pgfpathlineto{\pgfqpoint{2.640591in}{2.147097in}}%
\pgfpathlineto{\pgfqpoint{2.645523in}{2.061174in}}%
\pgfpathlineto{\pgfqpoint{2.675114in}{2.061174in}}%
\pgfpathlineto{\pgfqpoint{2.680045in}{1.741193in}}%
\pgfpathlineto{\pgfqpoint{2.694841in}{1.741193in}}%
\pgfpathlineto{\pgfqpoint{2.699773in}{2.250282in}}%
\pgfpathlineto{\pgfqpoint{2.709636in}{2.250282in}}%
\pgfpathlineto{\pgfqpoint{2.714568in}{2.590195in}}%
\pgfpathlineto{\pgfqpoint{2.729364in}{2.590195in}}%
\pgfpathlineto{\pgfqpoint{2.734295in}{2.088356in}}%
\pgfpathlineto{\pgfqpoint{2.744159in}{2.088356in}}%
\pgfpathlineto{\pgfqpoint{2.749091in}{1.384419in}}%
\pgfpathlineto{\pgfqpoint{2.754023in}{2.939449in}}%
\pgfpathlineto{\pgfqpoint{2.758955in}{2.925118in}}%
\pgfpathlineto{\pgfqpoint{2.768818in}{2.925118in}}%
\pgfpathlineto{\pgfqpoint{2.773750in}{2.469300in}}%
\pgfpathlineto{\pgfqpoint{2.783614in}{2.469300in}}%
\pgfpathlineto{\pgfqpoint{2.788545in}{2.465910in}}%
\pgfpathlineto{\pgfqpoint{2.808273in}{2.465910in}}%
\pgfpathlineto{\pgfqpoint{2.813205in}{2.685994in}}%
\pgfpathlineto{\pgfqpoint{2.818136in}{2.685994in}}%
\pgfpathlineto{\pgfqpoint{2.823068in}{2.930230in}}%
\pgfpathlineto{\pgfqpoint{2.842795in}{2.930230in}}%
\pgfpathlineto{\pgfqpoint{2.847727in}{2.349254in}}%
\pgfpathlineto{\pgfqpoint{2.852659in}{2.349254in}}%
\pgfpathlineto{\pgfqpoint{2.857591in}{2.204239in}}%
\pgfpathlineto{\pgfqpoint{2.882250in}{2.204239in}}%
\pgfpathlineto{\pgfqpoint{2.887182in}{1.426112in}}%
\pgfpathlineto{\pgfqpoint{2.897045in}{1.426112in}}%
\pgfpathlineto{\pgfqpoint{2.901977in}{2.915527in}}%
\pgfpathlineto{\pgfqpoint{2.906909in}{1.350168in}}%
\pgfpathlineto{\pgfqpoint{2.911841in}{1.504079in}}%
\pgfpathlineto{\pgfqpoint{2.916773in}{2.813315in}}%
\pgfpathlineto{\pgfqpoint{2.921705in}{2.677323in}}%
\pgfpathlineto{\pgfqpoint{2.926636in}{2.681599in}}%
\pgfpathlineto{\pgfqpoint{2.931568in}{1.430612in}}%
\pgfpathlineto{\pgfqpoint{2.936500in}{2.523040in}}%
\pgfpathlineto{\pgfqpoint{2.941432in}{2.892552in}}%
\pgfpathlineto{\pgfqpoint{2.946364in}{2.256571in}}%
\pgfpathlineto{\pgfqpoint{2.985818in}{2.256571in}}%
\pgfpathlineto{\pgfqpoint{2.990750in}{2.027353in}}%
\pgfpathlineto{\pgfqpoint{3.000614in}{2.027353in}}%
\pgfpathlineto{\pgfqpoint{3.005545in}{2.174870in}}%
\pgfpathlineto{\pgfqpoint{3.015409in}{2.174870in}}%
\pgfpathlineto{\pgfqpoint{3.020341in}{2.032739in}}%
\pgfpathlineto{\pgfqpoint{3.035136in}{2.032739in}}%
\pgfpathlineto{\pgfqpoint{3.040068in}{2.178656in}}%
\pgfpathlineto{\pgfqpoint{3.045000in}{2.114140in}}%
\pgfpathlineto{\pgfqpoint{3.049932in}{2.393036in}}%
\pgfpathlineto{\pgfqpoint{3.054864in}{2.393036in}}%
\pgfpathlineto{\pgfqpoint{3.059795in}{2.395030in}}%
\pgfpathlineto{\pgfqpoint{3.064727in}{2.395030in}}%
\pgfpathlineto{\pgfqpoint{3.069659in}{2.198283in}}%
\pgfpathlineto{\pgfqpoint{3.074591in}{2.245127in}}%
\pgfpathlineto{\pgfqpoint{3.089386in}{2.245127in}}%
\pgfpathlineto{\pgfqpoint{3.094318in}{1.262403in}}%
\pgfpathlineto{\pgfqpoint{3.099250in}{1.262403in}}%
\pgfpathlineto{\pgfqpoint{3.104182in}{2.574641in}}%
\pgfpathlineto{\pgfqpoint{3.109114in}{2.439193in}}%
\pgfpathlineto{\pgfqpoint{3.118977in}{2.439193in}}%
\pgfpathlineto{\pgfqpoint{3.123909in}{2.192791in}}%
\pgfpathlineto{\pgfqpoint{3.143636in}{2.192791in}}%
\pgfpathlineto{\pgfqpoint{3.148568in}{1.786855in}}%
\pgfpathlineto{\pgfqpoint{3.153500in}{2.026393in}}%
\pgfpathlineto{\pgfqpoint{3.158432in}{1.329671in}}%
\pgfpathlineto{\pgfqpoint{3.163364in}{1.767345in}}%
\pgfpathlineto{\pgfqpoint{3.168295in}{1.767345in}}%
\pgfpathlineto{\pgfqpoint{3.173227in}{2.952252in}}%
\pgfpathlineto{\pgfqpoint{3.183091in}{2.952252in}}%
\pgfpathlineto{\pgfqpoint{3.188023in}{1.729490in}}%
\pgfpathlineto{\pgfqpoint{3.192955in}{2.727326in}}%
\pgfpathlineto{\pgfqpoint{3.197886in}{2.727326in}}%
\pgfpathlineto{\pgfqpoint{3.202818in}{2.360740in}}%
\pgfpathlineto{\pgfqpoint{3.207750in}{1.602684in}}%
\pgfpathlineto{\pgfqpoint{3.212682in}{1.797705in}}%
\pgfpathlineto{\pgfqpoint{3.227477in}{1.797705in}}%
\pgfpathlineto{\pgfqpoint{3.232409in}{1.600482in}}%
\pgfpathlineto{\pgfqpoint{3.247205in}{1.600482in}}%
\pgfpathlineto{\pgfqpoint{3.252136in}{2.789990in}}%
\pgfpathlineto{\pgfqpoint{3.257068in}{2.789990in}}%
\pgfpathlineto{\pgfqpoint{3.262000in}{1.755614in}}%
\pgfpathlineto{\pgfqpoint{3.271864in}{1.755614in}}%
\pgfpathlineto{\pgfqpoint{3.276795in}{2.466804in}}%
\pgfpathlineto{\pgfqpoint{3.286659in}{2.466804in}}%
\pgfpathlineto{\pgfqpoint{3.291591in}{2.331972in}}%
\pgfpathlineto{\pgfqpoint{3.296523in}{2.578951in}}%
\pgfpathlineto{\pgfqpoint{3.301455in}{2.301295in}}%
\pgfpathlineto{\pgfqpoint{3.306386in}{2.301295in}}%
\pgfpathlineto{\pgfqpoint{3.311318in}{1.741569in}}%
\pgfpathlineto{\pgfqpoint{3.321182in}{1.741569in}}%
\pgfpathlineto{\pgfqpoint{3.326114in}{2.200364in}}%
\pgfpathlineto{\pgfqpoint{3.350773in}{2.200364in}}%
\pgfpathlineto{\pgfqpoint{3.355705in}{1.917401in}}%
\pgfpathlineto{\pgfqpoint{3.360636in}{2.437346in}}%
\pgfpathlineto{\pgfqpoint{3.365568in}{3.130521in}}%
\pgfpathlineto{\pgfqpoint{3.375432in}{3.130521in}}%
\pgfpathlineto{\pgfqpoint{3.380364in}{2.096078in}}%
\pgfpathlineto{\pgfqpoint{3.395159in}{2.096078in}}%
\pgfpathlineto{\pgfqpoint{3.400091in}{2.304990in}}%
\pgfpathlineto{\pgfqpoint{3.405023in}{2.654288in}}%
\pgfpathlineto{\pgfqpoint{3.419818in}{2.654288in}}%
\pgfpathlineto{\pgfqpoint{3.424750in}{3.007546in}}%
\pgfpathlineto{\pgfqpoint{3.429682in}{3.221090in}}%
\pgfpathlineto{\pgfqpoint{3.434614in}{2.511803in}}%
\pgfpathlineto{\pgfqpoint{3.464205in}{2.511803in}}%
\pgfpathlineto{\pgfqpoint{3.469136in}{2.305851in}}%
\pgfpathlineto{\pgfqpoint{3.474068in}{2.713490in}}%
\pgfpathlineto{\pgfqpoint{3.479000in}{2.944071in}}%
\pgfpathlineto{\pgfqpoint{3.483932in}{3.066949in}}%
\pgfpathlineto{\pgfqpoint{3.493795in}{3.066949in}}%
\pgfpathlineto{\pgfqpoint{3.498727in}{2.799912in}}%
\pgfpathlineto{\pgfqpoint{3.503659in}{2.799912in}}%
\pgfpathlineto{\pgfqpoint{3.508591in}{1.189721in}}%
\pgfpathlineto{\pgfqpoint{3.518455in}{1.189721in}}%
\pgfpathlineto{\pgfqpoint{3.523386in}{1.621804in}}%
\pgfpathlineto{\pgfqpoint{3.528318in}{1.621804in}}%
\pgfpathlineto{\pgfqpoint{3.533250in}{1.906742in}}%
\pgfpathlineto{\pgfqpoint{3.572705in}{1.906742in}}%
\pgfpathlineto{\pgfqpoint{3.577636in}{2.622731in}}%
\pgfpathlineto{\pgfqpoint{3.597364in}{2.622731in}}%
\pgfpathlineto{\pgfqpoint{3.602295in}{2.403830in}}%
\pgfpathlineto{\pgfqpoint{3.607227in}{2.403830in}}%
\pgfpathlineto{\pgfqpoint{3.612159in}{2.135324in}}%
\pgfpathlineto{\pgfqpoint{3.622023in}{2.135324in}}%
\pgfpathlineto{\pgfqpoint{3.626955in}{2.191303in}}%
\pgfpathlineto{\pgfqpoint{3.631886in}{2.191303in}}%
\pgfpathlineto{\pgfqpoint{3.636818in}{2.655518in}}%
\pgfpathlineto{\pgfqpoint{3.641750in}{2.655518in}}%
\pgfpathlineto{\pgfqpoint{3.646682in}{2.015721in}}%
\pgfpathlineto{\pgfqpoint{3.700932in}{2.015721in}}%
\pgfpathlineto{\pgfqpoint{3.705864in}{3.023523in}}%
\pgfpathlineto{\pgfqpoint{3.710795in}{3.178826in}}%
\pgfpathlineto{\pgfqpoint{3.715727in}{3.178826in}}%
\pgfpathlineto{\pgfqpoint{3.720659in}{1.698180in}}%
\pgfpathlineto{\pgfqpoint{3.730523in}{1.698180in}}%
\pgfpathlineto{\pgfqpoint{3.735455in}{2.465580in}}%
\pgfpathlineto{\pgfqpoint{3.740386in}{1.851169in}}%
\pgfpathlineto{\pgfqpoint{3.745318in}{1.840399in}}%
\pgfpathlineto{\pgfqpoint{3.750250in}{2.430337in}}%
\pgfpathlineto{\pgfqpoint{3.755182in}{2.430337in}}%
\pgfpathlineto{\pgfqpoint{3.760114in}{1.990765in}}%
\pgfpathlineto{\pgfqpoint{3.765045in}{1.990765in}}%
\pgfpathlineto{\pgfqpoint{3.769977in}{2.272175in}}%
\pgfpathlineto{\pgfqpoint{3.774909in}{2.272175in}}%
\pgfpathlineto{\pgfqpoint{3.779841in}{2.921327in}}%
\pgfpathlineto{\pgfqpoint{3.784773in}{2.410144in}}%
\pgfpathlineto{\pgfqpoint{3.819295in}{2.410144in}}%
\pgfpathlineto{\pgfqpoint{3.824227in}{1.993848in}}%
\pgfpathlineto{\pgfqpoint{3.843955in}{1.993848in}}%
\pgfpathlineto{\pgfqpoint{3.848886in}{0.901503in}}%
\pgfpathlineto{\pgfqpoint{3.853818in}{3.082695in}}%
\pgfpathlineto{\pgfqpoint{3.858750in}{3.082695in}}%
\pgfpathlineto{\pgfqpoint{3.863682in}{1.506297in}}%
\pgfpathlineto{\pgfqpoint{3.868614in}{2.413199in}}%
\pgfpathlineto{\pgfqpoint{3.873545in}{2.252376in}}%
\pgfpathlineto{\pgfqpoint{3.878477in}{1.945121in}}%
\pgfpathlineto{\pgfqpoint{3.883409in}{1.945121in}}%
\pgfpathlineto{\pgfqpoint{3.888341in}{3.014562in}}%
\pgfpathlineto{\pgfqpoint{3.893273in}{1.549005in}}%
\pgfpathlineto{\pgfqpoint{3.898205in}{1.361011in}}%
\pgfpathlineto{\pgfqpoint{3.903136in}{2.564302in}}%
\pgfpathlineto{\pgfqpoint{3.908068in}{2.186436in}}%
\pgfpathlineto{\pgfqpoint{3.917932in}{2.186436in}}%
\pgfpathlineto{\pgfqpoint{3.922864in}{2.374946in}}%
\pgfpathlineto{\pgfqpoint{3.932727in}{2.374946in}}%
\pgfpathlineto{\pgfqpoint{3.937659in}{2.016156in}}%
\pgfpathlineto{\pgfqpoint{3.947523in}{2.016156in}}%
\pgfpathlineto{\pgfqpoint{3.952455in}{2.054479in}}%
\pgfpathlineto{\pgfqpoint{3.957386in}{1.581556in}}%
\pgfpathlineto{\pgfqpoint{3.962318in}{1.944333in}}%
\pgfpathlineto{\pgfqpoint{3.972182in}{1.944333in}}%
\pgfpathlineto{\pgfqpoint{3.977114in}{2.657301in}}%
\pgfpathlineto{\pgfqpoint{3.982045in}{2.657301in}}%
\pgfpathlineto{\pgfqpoint{3.986977in}{2.867993in}}%
\pgfpathlineto{\pgfqpoint{3.991909in}{2.867993in}}%
\pgfpathlineto{\pgfqpoint{3.996841in}{1.879646in}}%
\pgfpathlineto{\pgfqpoint{4.001773in}{2.526520in}}%
\pgfpathlineto{\pgfqpoint{4.006705in}{1.520134in}}%
\pgfpathlineto{\pgfqpoint{4.021500in}{1.520134in}}%
\pgfpathlineto{\pgfqpoint{4.026432in}{2.386430in}}%
\pgfpathlineto{\pgfqpoint{4.041227in}{2.386430in}}%
\pgfpathlineto{\pgfqpoint{4.046159in}{2.044204in}}%
\pgfpathlineto{\pgfqpoint{4.051091in}{1.828873in}}%
\pgfpathlineto{\pgfqpoint{4.056023in}{1.828873in}}%
\pgfpathlineto{\pgfqpoint{4.060955in}{2.525254in}}%
\pgfpathlineto{\pgfqpoint{4.065886in}{2.525254in}}%
\pgfpathlineto{\pgfqpoint{4.070818in}{1.943446in}}%
\pgfpathlineto{\pgfqpoint{4.080682in}{1.943446in}}%
\pgfpathlineto{\pgfqpoint{4.085614in}{1.964261in}}%
\pgfpathlineto{\pgfqpoint{4.105341in}{1.964261in}}%
\pgfpathlineto{\pgfqpoint{4.110273in}{1.877459in}}%
\pgfpathlineto{\pgfqpoint{4.130000in}{1.877459in}}%
\pgfpathlineto{\pgfqpoint{4.134932in}{2.328311in}}%
\pgfpathlineto{\pgfqpoint{4.149727in}{2.328311in}}%
\pgfpathlineto{\pgfqpoint{4.154659in}{2.551812in}}%
\pgfpathlineto{\pgfqpoint{4.159591in}{2.551812in}}%
\pgfpathlineto{\pgfqpoint{4.169455in}{1.693969in}}%
\pgfpathlineto{\pgfqpoint{4.189182in}{1.693969in}}%
\pgfpathlineto{\pgfqpoint{4.194114in}{1.713432in}}%
\pgfpathlineto{\pgfqpoint{4.203977in}{1.713432in}}%
\pgfpathlineto{\pgfqpoint{4.208909in}{1.953690in}}%
\pgfpathlineto{\pgfqpoint{4.213841in}{2.515265in}}%
\pgfpathlineto{\pgfqpoint{4.218773in}{1.621301in}}%
\pgfpathlineto{\pgfqpoint{4.223705in}{1.621301in}}%
\pgfpathlineto{\pgfqpoint{4.228636in}{1.781912in}}%
\pgfpathlineto{\pgfqpoint{4.238500in}{1.781912in}}%
\pgfpathlineto{\pgfqpoint{4.243432in}{2.063885in}}%
\pgfpathlineto{\pgfqpoint{4.253295in}{2.063885in}}%
\pgfpathlineto{\pgfqpoint{4.258227in}{1.337238in}}%
\pgfpathlineto{\pgfqpoint{4.263159in}{2.569848in}}%
\pgfpathlineto{\pgfqpoint{4.268091in}{2.569848in}}%
\pgfpathlineto{\pgfqpoint{4.273023in}{2.472117in}}%
\pgfpathlineto{\pgfqpoint{4.277955in}{2.654890in}}%
\pgfpathlineto{\pgfqpoint{4.282886in}{2.654890in}}%
\pgfpathlineto{\pgfqpoint{4.287818in}{2.562157in}}%
\pgfpathlineto{\pgfqpoint{4.292750in}{2.869024in}}%
\pgfpathlineto{\pgfqpoint{4.297682in}{1.275408in}}%
\pgfpathlineto{\pgfqpoint{4.302614in}{1.159043in}}%
\pgfpathlineto{\pgfqpoint{4.307545in}{2.844097in}}%
\pgfpathlineto{\pgfqpoint{4.312477in}{2.844097in}}%
\pgfpathlineto{\pgfqpoint{4.317409in}{2.013180in}}%
\pgfpathlineto{\pgfqpoint{4.332205in}{2.013180in}}%
\pgfpathlineto{\pgfqpoint{4.337136in}{2.757027in}}%
\pgfpathlineto{\pgfqpoint{4.342068in}{2.221352in}}%
\pgfpathlineto{\pgfqpoint{4.347000in}{2.648424in}}%
\pgfpathlineto{\pgfqpoint{4.351932in}{2.574638in}}%
\pgfpathlineto{\pgfqpoint{4.386455in}{2.574638in}}%
\pgfpathlineto{\pgfqpoint{4.391386in}{2.129777in}}%
\pgfpathlineto{\pgfqpoint{4.396318in}{1.960122in}}%
\pgfpathlineto{\pgfqpoint{4.401250in}{1.960122in}}%
\pgfpathlineto{\pgfqpoint{4.406182in}{1.836712in}}%
\pgfpathlineto{\pgfqpoint{4.411114in}{1.836712in}}%
\pgfpathlineto{\pgfqpoint{4.416045in}{2.074119in}}%
\pgfpathlineto{\pgfqpoint{4.420977in}{2.074119in}}%
\pgfpathlineto{\pgfqpoint{4.425909in}{2.228317in}}%
\pgfpathlineto{\pgfqpoint{4.430841in}{2.439231in}}%
\pgfpathlineto{\pgfqpoint{4.435773in}{2.439231in}}%
\pgfpathlineto{\pgfqpoint{4.440705in}{1.522380in}}%
\pgfpathlineto{\pgfqpoint{4.450568in}{1.522380in}}%
\pgfpathlineto{\pgfqpoint{4.455500in}{2.343756in}}%
\pgfpathlineto{\pgfqpoint{4.465364in}{2.343756in}}%
\pgfpathlineto{\pgfqpoint{4.470295in}{2.393919in}}%
\pgfpathlineto{\pgfqpoint{4.490023in}{2.393919in}}%
\pgfpathlineto{\pgfqpoint{4.494955in}{2.022778in}}%
\pgfpathlineto{\pgfqpoint{4.499886in}{2.496411in}}%
\pgfpathlineto{\pgfqpoint{4.519614in}{2.496411in}}%
\pgfpathlineto{\pgfqpoint{4.524545in}{1.725717in}}%
\pgfpathlineto{\pgfqpoint{4.529477in}{2.886847in}}%
\pgfpathlineto{\pgfqpoint{4.534409in}{2.886847in}}%
\pgfpathlineto{\pgfqpoint{4.539341in}{2.283727in}}%
\pgfpathlineto{\pgfqpoint{4.583727in}{2.283727in}}%
\pgfpathlineto{\pgfqpoint{4.588659in}{2.048096in}}%
\pgfpathlineto{\pgfqpoint{4.598523in}{2.048096in}}%
\pgfpathlineto{\pgfqpoint{4.603455in}{2.475315in}}%
\pgfpathlineto{\pgfqpoint{4.608386in}{2.475315in}}%
\pgfpathlineto{\pgfqpoint{4.613318in}{2.245274in}}%
\pgfpathlineto{\pgfqpoint{4.618250in}{2.242203in}}%
\pgfpathlineto{\pgfqpoint{4.623182in}{2.242203in}}%
\pgfpathlineto{\pgfqpoint{4.628114in}{2.471338in}}%
\pgfpathlineto{\pgfqpoint{4.633045in}{2.167796in}}%
\pgfpathlineto{\pgfqpoint{4.637977in}{2.215617in}}%
\pgfpathlineto{\pgfqpoint{4.642909in}{2.525844in}}%
\pgfpathlineto{\pgfqpoint{4.692227in}{2.525844in}}%
\pgfpathlineto{\pgfqpoint{4.697159in}{1.927699in}}%
\pgfpathlineto{\pgfqpoint{4.702091in}{2.888158in}}%
\pgfpathlineto{\pgfqpoint{4.707023in}{2.888158in}}%
\pgfpathlineto{\pgfqpoint{4.711955in}{2.060853in}}%
\pgfpathlineto{\pgfqpoint{4.726750in}{2.060853in}}%
\pgfpathlineto{\pgfqpoint{4.731682in}{1.508424in}}%
\pgfpathlineto{\pgfqpoint{4.746477in}{1.508424in}}%
\pgfpathlineto{\pgfqpoint{4.751409in}{2.753935in}}%
\pgfpathlineto{\pgfqpoint{4.756341in}{1.657073in}}%
\pgfpathlineto{\pgfqpoint{4.761273in}{1.657073in}}%
\pgfpathlineto{\pgfqpoint{4.766205in}{3.246962in}}%
\pgfpathlineto{\pgfqpoint{4.771136in}{1.329671in}}%
\pgfpathlineto{\pgfqpoint{4.776068in}{1.475227in}}%
\pgfpathlineto{\pgfqpoint{4.781000in}{1.475227in}}%
\pgfpathlineto{\pgfqpoint{4.785932in}{1.513632in}}%
\pgfpathlineto{\pgfqpoint{4.790864in}{1.099598in}}%
\pgfpathlineto{\pgfqpoint{4.795795in}{0.580000in}}%
\pgfpathlineto{\pgfqpoint{4.800727in}{0.908157in}}%
\pgfpathlineto{\pgfqpoint{4.810591in}{0.908157in}}%
\pgfpathlineto{\pgfqpoint{4.815523in}{3.192392in}}%
\pgfpathlineto{\pgfqpoint{4.820455in}{2.286364in}}%
\pgfpathlineto{\pgfqpoint{4.845114in}{2.286364in}}%
\pgfpathlineto{\pgfqpoint{4.850045in}{1.931040in}}%
\pgfpathlineto{\pgfqpoint{4.859909in}{1.931040in}}%
\pgfpathlineto{\pgfqpoint{4.864841in}{2.330829in}}%
\pgfpathlineto{\pgfqpoint{4.879636in}{2.330829in}}%
\pgfpathlineto{\pgfqpoint{4.884568in}{2.140721in}}%
\pgfpathlineto{\pgfqpoint{4.909227in}{2.140721in}}%
\pgfpathlineto{\pgfqpoint{4.914159in}{3.131136in}}%
\pgfpathlineto{\pgfqpoint{4.919091in}{3.049225in}}%
\pgfpathlineto{\pgfqpoint{4.924023in}{2.501476in}}%
\pgfpathlineto{\pgfqpoint{4.928955in}{2.501476in}}%
\pgfpathlineto{\pgfqpoint{4.933886in}{2.958503in}}%
\pgfpathlineto{\pgfqpoint{4.938818in}{1.375597in}}%
\pgfpathlineto{\pgfqpoint{4.943750in}{1.375597in}}%
\pgfpathlineto{\pgfqpoint{4.948682in}{3.139941in}}%
\pgfpathlineto{\pgfqpoint{4.953614in}{2.657763in}}%
\pgfpathlineto{\pgfqpoint{4.978273in}{2.657763in}}%
\pgfpathlineto{\pgfqpoint{4.983205in}{1.711945in}}%
\pgfpathlineto{\pgfqpoint{5.002932in}{1.711945in}}%
\pgfpathlineto{\pgfqpoint{5.007864in}{2.000894in}}%
\pgfpathlineto{\pgfqpoint{5.012795in}{2.419291in}}%
\pgfpathlineto{\pgfqpoint{5.032523in}{2.419291in}}%
\pgfpathlineto{\pgfqpoint{5.037455in}{2.026439in}}%
\pgfpathlineto{\pgfqpoint{5.047318in}{2.026439in}}%
\pgfpathlineto{\pgfqpoint{5.052250in}{2.228819in}}%
\pgfpathlineto{\pgfqpoint{5.091705in}{2.228819in}}%
\pgfpathlineto{\pgfqpoint{5.096636in}{2.188355in}}%
\pgfpathlineto{\pgfqpoint{5.101568in}{2.857558in}}%
\pgfpathlineto{\pgfqpoint{5.106500in}{2.435844in}}%
\pgfpathlineto{\pgfqpoint{5.111432in}{1.554427in}}%
\pgfpathlineto{\pgfqpoint{5.121295in}{1.554427in}}%
\pgfpathlineto{\pgfqpoint{5.126227in}{2.114585in}}%
\pgfpathlineto{\pgfqpoint{5.136091in}{2.114585in}}%
\pgfpathlineto{\pgfqpoint{5.141023in}{3.254273in}}%
\pgfpathlineto{\pgfqpoint{5.145955in}{3.254273in}}%
\pgfpathlineto{\pgfqpoint{5.150886in}{1.884484in}}%
\pgfpathlineto{\pgfqpoint{5.155818in}{2.245710in}}%
\pgfpathlineto{\pgfqpoint{5.160750in}{2.245710in}}%
\pgfpathlineto{\pgfqpoint{5.165682in}{2.022048in}}%
\pgfpathlineto{\pgfqpoint{5.170614in}{2.236078in}}%
\pgfpathlineto{\pgfqpoint{5.175545in}{1.258800in}}%
\pgfpathlineto{\pgfqpoint{5.180477in}{2.366353in}}%
\pgfpathlineto{\pgfqpoint{5.185409in}{2.031090in}}%
\pgfpathlineto{\pgfqpoint{5.190341in}{2.031090in}}%
\pgfpathlineto{\pgfqpoint{5.195273in}{2.541114in}}%
\pgfpathlineto{\pgfqpoint{5.219932in}{2.541114in}}%
\pgfpathlineto{\pgfqpoint{5.224864in}{1.987368in}}%
\pgfpathlineto{\pgfqpoint{5.288977in}{1.987368in}}%
\pgfpathlineto{\pgfqpoint{5.298841in}{2.459355in}}%
\pgfpathlineto{\pgfqpoint{5.308705in}{2.459355in}}%
\pgfpathlineto{\pgfqpoint{5.313636in}{1.235626in}}%
\pgfpathlineto{\pgfqpoint{5.328432in}{1.235626in}}%
\pgfpathlineto{\pgfqpoint{5.333364in}{1.257209in}}%
\pgfpathlineto{\pgfqpoint{5.338295in}{0.948700in}}%
\pgfpathlineto{\pgfqpoint{5.343227in}{2.415122in}}%
\pgfpathlineto{\pgfqpoint{5.353091in}{2.415122in}}%
\pgfpathlineto{\pgfqpoint{5.358023in}{2.933182in}}%
\pgfpathlineto{\pgfqpoint{5.367886in}{2.933182in}}%
\pgfpathlineto{\pgfqpoint{5.372818in}{2.165746in}}%
\pgfpathlineto{\pgfqpoint{5.387614in}{2.165746in}}%
\pgfpathlineto{\pgfqpoint{5.392545in}{2.247086in}}%
\pgfpathlineto{\pgfqpoint{5.397477in}{2.247086in}}%
\pgfpathlineto{\pgfqpoint{5.402409in}{2.613539in}}%
\pgfpathlineto{\pgfqpoint{5.417205in}{2.613539in}}%
\pgfpathlineto{\pgfqpoint{5.422136in}{2.899074in}}%
\pgfpathlineto{\pgfqpoint{5.432000in}{2.899074in}}%
\pgfpathlineto{\pgfqpoint{5.436932in}{1.509760in}}%
\pgfpathlineto{\pgfqpoint{5.441864in}{1.509760in}}%
\pgfpathlineto{\pgfqpoint{5.446795in}{2.064577in}}%
\pgfpathlineto{\pgfqpoint{5.451727in}{2.405661in}}%
\pgfpathlineto{\pgfqpoint{5.456659in}{1.551271in}}%
\pgfpathlineto{\pgfqpoint{5.461591in}{2.741657in}}%
\pgfpathlineto{\pgfqpoint{5.466523in}{1.357437in}}%
\pgfpathlineto{\pgfqpoint{5.471455in}{2.668378in}}%
\pgfpathlineto{\pgfqpoint{5.476386in}{2.672233in}}%
\pgfpathlineto{\pgfqpoint{5.481318in}{2.672233in}}%
\pgfpathlineto{\pgfqpoint{5.486250in}{2.388241in}}%
\pgfpathlineto{\pgfqpoint{5.491182in}{2.388241in}}%
\pgfpathlineto{\pgfqpoint{5.496114in}{2.594888in}}%
\pgfpathlineto{\pgfqpoint{5.501045in}{2.594888in}}%
\pgfpathlineto{\pgfqpoint{5.505977in}{1.497970in}}%
\pgfpathlineto{\pgfqpoint{5.510909in}{1.593024in}}%
\pgfpathlineto{\pgfqpoint{5.515841in}{2.568780in}}%
\pgfpathlineto{\pgfqpoint{5.525705in}{2.568780in}}%
\pgfpathlineto{\pgfqpoint{5.530636in}{2.131714in}}%
\pgfpathlineto{\pgfqpoint{5.535568in}{2.131714in}}%
\pgfpathlineto{\pgfqpoint{5.545432in}{2.564432in}}%
\pgfpathlineto{\pgfqpoint{5.639136in}{2.564432in}}%
\pgfpathlineto{\pgfqpoint{5.644068in}{2.375083in}}%
\pgfpathlineto{\pgfqpoint{5.688455in}{2.375083in}}%
\pgfpathlineto{\pgfqpoint{5.693386in}{1.813384in}}%
\pgfpathlineto{\pgfqpoint{5.718045in}{1.813384in}}%
\pgfpathlineto{\pgfqpoint{5.722977in}{2.932128in}}%
\pgfpathlineto{\pgfqpoint{5.727909in}{2.932128in}}%
\pgfpathlineto{\pgfqpoint{5.732841in}{1.783985in}}%
\pgfpathlineto{\pgfqpoint{5.747636in}{1.783985in}}%
\pgfpathlineto{\pgfqpoint{5.757500in}{2.159537in}}%
\pgfpathlineto{\pgfqpoint{5.767364in}{2.159537in}}%
\pgfpathlineto{\pgfqpoint{5.772295in}{2.040759in}}%
\pgfpathlineto{\pgfqpoint{5.777227in}{2.040759in}}%
\pgfpathlineto{\pgfqpoint{5.782159in}{2.786297in}}%
\pgfpathlineto{\pgfqpoint{5.792023in}{2.786297in}}%
\pgfpathlineto{\pgfqpoint{5.796955in}{2.542192in}}%
\pgfpathlineto{\pgfqpoint{5.801886in}{2.542192in}}%
\pgfpathlineto{\pgfqpoint{5.806818in}{2.520562in}}%
\pgfpathlineto{\pgfqpoint{5.826545in}{2.520562in}}%
\pgfpathlineto{\pgfqpoint{5.831477in}{1.489236in}}%
\pgfpathlineto{\pgfqpoint{5.836409in}{2.219208in}}%
\pgfpathlineto{\pgfqpoint{5.866000in}{2.219208in}}%
\pgfpathlineto{\pgfqpoint{5.870932in}{1.832258in}}%
\pgfpathlineto{\pgfqpoint{5.875864in}{1.832258in}}%
\pgfpathlineto{\pgfqpoint{5.880795in}{2.154482in}}%
\pgfpathlineto{\pgfqpoint{5.890659in}{2.154482in}}%
\pgfpathlineto{\pgfqpoint{5.895591in}{2.561091in}}%
\pgfpathlineto{\pgfqpoint{5.915318in}{2.561091in}}%
\pgfpathlineto{\pgfqpoint{5.920250in}{2.338114in}}%
\pgfpathlineto{\pgfqpoint{5.925182in}{2.338114in}}%
\pgfpathlineto{\pgfqpoint{5.930114in}{1.899499in}}%
\pgfpathlineto{\pgfqpoint{5.935045in}{1.899499in}}%
\pgfpathlineto{\pgfqpoint{5.939977in}{2.535575in}}%
\pgfpathlineto{\pgfqpoint{5.944909in}{2.535575in}}%
\pgfpathlineto{\pgfqpoint{5.949841in}{2.103948in}}%
\pgfpathlineto{\pgfqpoint{5.954773in}{2.103948in}}%
\pgfpathlineto{\pgfqpoint{5.959705in}{1.979841in}}%
\pgfpathlineto{\pgfqpoint{5.984364in}{1.979841in}}%
\pgfpathlineto{\pgfqpoint{5.989295in}{1.841018in}}%
\pgfpathlineto{\pgfqpoint{5.994227in}{1.841018in}}%
\pgfpathlineto{\pgfqpoint{5.999159in}{2.769788in}}%
\pgfpathlineto{\pgfqpoint{6.004091in}{2.769788in}}%
\pgfpathlineto{\pgfqpoint{6.009023in}{1.751169in}}%
\pgfpathlineto{\pgfqpoint{6.013955in}{2.516566in}}%
\pgfpathlineto{\pgfqpoint{6.018886in}{2.516566in}}%
\pgfpathlineto{\pgfqpoint{6.023818in}{1.366371in}}%
\pgfpathlineto{\pgfqpoint{6.028750in}{2.878704in}}%
\pgfpathlineto{\pgfqpoint{6.033682in}{3.071631in}}%
\pgfpathlineto{\pgfqpoint{6.043545in}{3.071631in}}%
\pgfpathlineto{\pgfqpoint{6.048477in}{1.711935in}}%
\pgfpathlineto{\pgfqpoint{6.053409in}{1.962663in}}%
\pgfpathlineto{\pgfqpoint{6.053409in}{1.962663in}}%
\pgfusepath{stroke}%
\end{pgfscope}%
\begin{pgfscope}%
\pgfsetrectcap%
\pgfsetmiterjoin%
\pgfsetlinewidth{0.000000pt}%
\definecolor{currentstroke}{rgb}{1.000000,1.000000,1.000000}%
\pgfsetstrokecolor{currentstroke}%
\pgfsetdash{}{0pt}%
\pgfpathmoveto{\pgfqpoint{0.875000in}{0.440000in}}%
\pgfpathlineto{\pgfqpoint{0.875000in}{3.520000in}}%
\pgfusepath{}%
\end{pgfscope}%
\begin{pgfscope}%
\pgfsetrectcap%
\pgfsetmiterjoin%
\pgfsetlinewidth{0.000000pt}%
\definecolor{currentstroke}{rgb}{1.000000,1.000000,1.000000}%
\pgfsetstrokecolor{currentstroke}%
\pgfsetdash{}{0pt}%
\pgfpathmoveto{\pgfqpoint{6.300000in}{0.440000in}}%
\pgfpathlineto{\pgfqpoint{6.300000in}{3.520000in}}%
\pgfusepath{}%
\end{pgfscope}%
\begin{pgfscope}%
\pgfsetrectcap%
\pgfsetmiterjoin%
\pgfsetlinewidth{0.000000pt}%
\definecolor{currentstroke}{rgb}{1.000000,1.000000,1.000000}%
\pgfsetstrokecolor{currentstroke}%
\pgfsetdash{}{0pt}%
\pgfpathmoveto{\pgfqpoint{0.875000in}{0.440000in}}%
\pgfpathlineto{\pgfqpoint{6.300000in}{0.440000in}}%
\pgfusepath{}%
\end{pgfscope}%
\begin{pgfscope}%
\pgfsetrectcap%
\pgfsetmiterjoin%
\pgfsetlinewidth{0.000000pt}%
\definecolor{currentstroke}{rgb}{1.000000,1.000000,1.000000}%
\pgfsetstrokecolor{currentstroke}%
\pgfsetdash{}{0pt}%
\pgfpathmoveto{\pgfqpoint{0.875000in}{3.520000in}}%
\pgfpathlineto{\pgfqpoint{6.300000in}{3.520000in}}%
\pgfusepath{}%
\end{pgfscope}%
\begin{pgfscope}%
\definecolor{textcolor}{rgb}{0.150000,0.150000,0.150000}%
\pgfsetstrokecolor{textcolor}%
\pgfsetfillcolor{textcolor}%
\pgftext[x=3.500000in,y=3.920000in,,top]{\color{textcolor}\rmfamily\fontsize{12.000000}{14.400000}\selectfont Muestreo con propuesta \(\displaystyle U(0,1)\) y \(\displaystyle n=40\)}%
\end{pgfscope}%
\end{pgfpicture}%
\makeatother%
\endgroup%

    \end{center}

    Tras observar los ejemplos anteriores podemos concluir que, a pesar de que la implementación del
    algoritmo de Metropolis-Hastings es conceptualmente simple, obtener un algoritmo que pueda muestrear
    de una distribución objetivo de manera eficiente requiere varias decisiones de diseño no triviales
    y que afectan notablemente el desempeño práctico del algoritmo.
   
\end{enumerate}




 \end{document}